\chapter[Límites]{Límites}

\section{Aproximar}

\textit{Aproximar} es la palabra clave de la ciencia y de la tecnología modernas. En efecto,
consciente o no, el trabajo de todo científico ha sido el de elaborar modelos que se
\textit{aproximan} a una realidad compleja que el científico quiere comprender y explicar. Los
números son, quizás, la primera herramienta que creó el ser humano para la elaboración de dichos
modelos.

A pesar de que se han construido conceptos de un alto nivel de abstracción para la noción de
número, la práctica cotidiana se remite casi exclusivamente a la utilización de números decimales.
Más aún, se utilizan frecuentemente una o dos cifras decimales a lo más. Por ejemplo, en la
representación de cantidades de dinero, se utilizan hasta dos cifras decimales que indican los
centavos. O, si se requiere dividir un terreno de 15 hectáreas entre siete herederos y en partes
iguales, se dividirá el terreno en parcelas de aproximadamente $\frac{15}{7}$ de hectárea y, en la
práctica, cada parcela tendrá unas $2.14$ hectáreas.

En realidad, todo lo que se hace con cualquier número a la hora de realizar cálculos es
\textit{aproximarlo} mediante números decimales. Es así que, cuando en los modelos aparecen números
como $\pi$, $\sqrt{2}$ o $\frac{4}{3}$, en su lugar se utilizan números decimales.

Veamos esta situación más de cerca. Si convenimos en utilizar números decimales con solo dos cifras
después del punto, ?`qué número o números decimales deberíamos elegir para aproximar, por ejemplo,
el número $\frac{4}{3}$? Los siguientes son algunos de los números decimales con dos cifras después
del punto que pueden ser utilizados para aproximar a $\frac{4}{3}$:

\[
1.31; \ 1.32; \ 1.33; \ 1.34; \ 1.35.
\]

Dado que ninguno de ellos es exactamente el número $\frac{4}{3}$, el que elijamos como aproximación
debería ser el \textit{menos diferente} del número $\frac{4}{3}$. Ahora, la \textit{resta} entre
números es el modo como se establece la \textit{diferencia} entre dos números. A esta diferencia le
vamos a considerar como el \textit{error} que se comete al aproximar $\frac{4}{3}$ con un número
decimal con dos cifras después del punto.

Así, si se utilizara $1.31$ como aproximación, el error cometido se calcularía de la siguiente
manera:

\begin{align*}
\frac{4}{3} - 1.31 &= \frac{4}{3} - \frac{131}{100} \\
&= \frac{400}{300} - \frac{393}{300} = \frac{7}{300}.
\end{align*}

Es decir, el error que se comete al aproximar $\frac{4}{3}$ con $1.31$ es $\frac{7}{300}$.

En cambio, si se utilizara $1.32$ para aproximar $\frac{4}{3}$, el error cometido sería:

\begin{align*}
\frac{4}{3} - 1.32 &=\frac{4}{3} - \frac{132}{100} \\
&= \frac{400}{300} - \frac{396}{300} = \frac{4}{300}.
\end{align*}

Vemos que $1.32$ es una \emph{mejor aproximación} de $\frac{4}{3}$ que $1.31$ en el sentido de que
el error que se comete al aproximar $\frac{4}{3}$ con $1.32$ es de $\frac{4}{300}$, que es menor
que $\frac{7}{300}$, el error que se comete al aproximar $\frac{4}{3}$ con $1.31$.

Veamos qué sucede si aproximamos $\displaystyle{\frac{4}{3}}$ con $1.33$. En este caso, el error
cometido sería:
\begin{align*}
\frac{4}{3} - 1.33 &=\frac{4}{3} - \frac{133}{100} \\
&= \frac{400}{300} - \frac{399}{300} = \frac{1}{300}.
\end{align*}

Ésta tercera aproximación es mejor que las dos anteriores, porque el error cometido cuando se
aproxima $\frac{4}{3}$ con $1.33$, que es $\frac{1}{300}$, es menor que el error de aproximar
$\frac{4}{3}$ con $1.31$ o $1.32$.

El error cometido al aproximar $\frac{4}{3}$ con $1.34$ es:
\begin{align*}
\frac{4}{3} - 1.34 &=\frac{4}{3} - \frac{134}{100} \\
&= \frac{400}{300} - \frac{402}{300} = -\frac{2}{300}.
\end{align*}
El signo negativo indica que el número utilizado como aproximación es mayor que el número que se
quiere aproximar. Sin embargo, si no nos interesa saber si el número que aproxima a $\frac{4}{3}$
es mayor o menor que éste, podemos tomar siempre la \textit{diferencia positiva}, es decir, la
diferencia entre el número mayor y menor, la misma que es igual al valor absoluto de la resta de
ambos números (sin importar cuál es el mayor). En este caso, la diferencia positiva es:

\[
\left|\frac{4}{3} - 1.34\right| = \left|-\frac{2}{300}\right| = \frac{2}{300}.
\]

A este número le denominamos \emph{error absoluto} cometido al aproximar $\frac{4}{3}$ con el
número $1.34$.

Análogamente, calculemos el error absoluto que se comete al aproximar $\frac{4}{3}$ con $1.35$:
\begin{align*}
\left|\frac{4}{3} - 1.35\right| &= \left|\frac{4}{3} - \frac{135}{100}\right| \\
&= \left|\frac{400}{300} - \frac{405}{300}\right| = \frac{5}{300}.
\end{align*}

Finalmente, si se toma un número decimal con dos cifras después del punto, mayor que $1.35$ o menor
que $1.31$ para aproximar $\frac{4}{3}$, podemos ver que el error absoluto cometido en esta
aproximación será mayor que cualquiera de las ya obtenidas. Por lo tanto, vemos que $1.33$ es,
efectivamente, la mejor aproximación de $\frac{4}{3}$ con un número decimal con dos cifras
decimales después del punto, porque el error que se comete al hacerlo es menor que el cometido con
cualquier otro número con solo dos cifras después del punto.

Podemos, entonces, decir que la \textit{proximidad} entre dos números se mide a través del valor
absoluto de la diferencia entre ellos. De manera más precisa: si $x$ e $y$ son dos números reales,
entonces la cantidad

\[
|x - y|
\]

mide la proximidad entre $x$ e $y$; y, mientras más pequeña sea esta cantidad, consideraremos a los
números $x$ e $y$ más próximos. Al contrario, si esta diferencia es grande, diremos que los números
son menos próximos.

En lo que sigue, usaremos como sinónimo de \textit{aproximar} la palabra \textit{acercar} y de la
palabra \textit{próximo}, la palabra \emph{cerca}. Así, significará lo mismo la frase ``$x$ está
próximo a $y$'' que la frase ``$x$ está cerca de $y$''.

\section{La recta tangente a una curva}

\begin{wrapfigure}{r}{4.5cm}
\begin{center}
\begin{pspicture}(0,.5)(4,3)
\pscircle[linecolor=gray](2,1.5){1.5}%
\psline(2,1.5)(0.9393,2.5607)% radio
\psline(-.1213,1.5)(2,3.6213)% tangente
\rput[lt](2.05,1.5){\footnotesize{$O$}} % etiqueta del centro del círculo
\rput[l](2.05,3.6213){$t$} % etiqueta de la tangente
\rput[rb](0.9,2.6){\footnotesize{$P$}}
\end{pspicture}
\end{center}
\end{wrapfigure}

El dibujar primero y luego el encontrar la ecuación de la recta tangente a una curva en un punto
dado de ella no es, en absoluto, un problema trivial.

Euclides (325 a.C.-265 a.C.) definió la tangente a una circunferencia como la recta que tocándola
no corta a la circunferencia.
%\footnote{Definición 2 del tercer libro de los \emph{Elementos} de Euclides, traducción
%de María Luisa Puertas Castaños, edición de Gredos, Madrid, 1991.}.
Esto significa que la recta tangente tiene con la circunferencia un único punto en común: el punto
de tangencia. Con regla y compás, el geómetra griego construyó la línea recta tangente $t$ a una
circunferencia en el punto $P$ de la siguiente manera: trazó por el punto $P$ la recta
perpendicular al radio de la circunferencia que tiene por uno de sus extremos el punto $P$.
Demostró, además, que ninguna otra recta se interpondrá en el espacio entre la tangente y la
circunferencia.
%\footnote{Proposición 16
%del tercer libro de los \emph{Elementos}.}.

Esta última propiedad se convirtió, más adelante, en la definición de la recta tangente, pues la
definición de Euclides no describe el caso general. En efecto, en las siguientes figuras se muestra
dos ejemplos de lo afirmado:

\begin{figure}[h]
%
\begin{center}
\subfloat[]{
\begin{pspicture}(0,0)(4,5.2) \psset{plotpoints=200}
\psplot{.5}{3.5}{x 2 sub x 2 sub mul 2.5 add}% y = x^2
\rput[br](.5,4.75){\footnotesize{$y = x^2$}}%
\psplot{2.25}{3.6}{x 2 sub 2 mul 1.5 add}% la tangente y = 2x - 1
\rput[bl](3.7,4.7){\footnotesize{$y = 2x - 1$}} %
\psline(3,2)(3,4.5)% la recta x = 1
\rput[tl](3.1,2){\footnotesize{$x = 1$}} %
\psline{->}(0,2.5)(4,2.5)% eje x
\rput[l](4.1,2.5){\footnotesize{$x$}} %
\psline{->}(2,1.5)(2,4.5)% eje y
\rput[b](2,4.6){\footnotesize{$y$}} %
\rput[l](3.1,3.5){\footnotesize{$(1,1)$}}
\end{pspicture}}
%
\hspace{.175\textwidth}
%
\subfloat[]{
\begin{pspicture}(0,0)(4,5.2)
\psset{plotpoints=200}
\psplot{.8}{3.2}{x 2 sub 3 exp 2.5 add}%
\rput[l](3.3,4.228){\footnotesize{$y = x^3$}}
\psline{->}(0,2.5)(4,2.5)% eje x
\rput[l](4.1,2.5){\footnotesize{$x$}} %
\psline{->}(2,.5)(2,4.5)% eje y
\rput[b](2,4.6){\footnotesize{$y$}} %
\psplot{.8}{3.2}{x .75 mul .75 add}%
\rput[l](3.3,3.15){\footnotesize{$y = 0.75x - 0.25x$}}
\end{pspicture}}
\end{center}
\end{figure}
En el primero, la parábola con ecuación $y = x^2$ y la recta con ecuación $x = 1$ tienen un único
punto de contacto: el de coordenadas $(1,1)$; sin embargo, esta recta no es tangente a la curva en
este punto. Es decir, para ser tangente no es suficiente con tener un único punto en común con la
curva. Por otro lado, puede apreciarse, en la figura, que la propiedad de Euclides --de que ninguna
otra recta se interpondrá en el espacio entre la tangente y la curva-- sí es verdadera en este
caso.

En el segundo ejemplo, se puede observar que la recta tangente a la curva cuya ecuación es $y =
x^3$ en el punto $(0.5,1.25)$ tiene por ecuación $y = 0.75x - 0.25$ (esto se probará más adelante).
Sin embargo, esta recta tiene aún otro punto en común con la curva, sin que por ello deje de ser
tangente en el punto $(0.5,1.25)$. Puede apreciarse que la propiedad de Euclides es verdadera pero
solo en una región cercana al punto de tangencia.

En general, la propiedad: ``entre la curva y la recta tangente, alrededor del punto de tangencia,
no se interpone ninguna recta'' pasó a ser la definición de tangente, la misma que ya fue utilizada
por los geómetras griegos posteriores a Euclides.

Aunque el problema de formular una definición de tangente adecuada para cualquier caso fue resuelto
como se indicó, los matemáticos griegos y los de la edad media no encontraron un método general
para obtener esa recta tangente a cualquier curva y en cualquier punto de ella. Este problema fue
uno de los temas centrales de la matemática en la modernidad: uno de los primeros intentos por
resolverlo fue realizado por el francés Pierre Fermat (1601-1665) en 1636. Poco después, se obtuvo
una solución. Pero ésta trajo consigo un nuevo concepto en las matemáticas: el de \emph{derivada}.
Entre los protagonistas de estos descubrimientos estuvieron el matemático inglés Isaac Newton
(1643-1727) y el alemán Gottfried Wilhelm Leibniz (1646-1716). Con el concepto de derivada se
encontró un método general para obtener la recta tangente a una curva de una clase amplia de
curvas.

Ahora bien, el concepto de derivada descansa sobre otro: el de límite. En los tiempos de Newton y
Leibiniz, este concepto fue tratado de una manera informal, lo que provocó serias críticas y dudas
sobre la validez del método. Tuvieron que transcurrir aproximadamente 150 años para que la
comunidad matemática cuente con el fundamento de la derivada: recién en 1823, el matemático francés
Augustin Cauchy (1789-1857) propuso una definición rigurosa de límite, y ésa es la que usamos hasta
hoy en día. El tema de este capítulo es estudiar, precisamente, esa definición de límite como base
para el estudio tanto del concepto de derivada como del concepto de integral, otro de los grandes
descubrimientos de la modernidad.

Pero antes de ello, vamos a presentar una solución no rigurosa del problema de encontrar la
tangente a una curva cualquiera.

\subsection{Formulación del problema}
Dada una curva general $C$, como la de la figura (a), y un punto $P$ en ella, se busca la recta $t$
tangente a $C$ en el punto $P$:
\begin{figure}[h]
\begin{center}
%
\subfloat[]{%
\begin{pspicture}(-.5,0)(6,4.3)

%La curva y = .08e^(x - .75) + .5
\psplot{0}{4.6}{.08 2.71828 x .75 sub exp mul .5 add}%

%La etiqueta de la curva
\rput[l](4.7,4.259){\footnotesize$C$}

%La marca del punto P
\pscircle[fillstyle=solid,fillcolor=black](1.5,.669){.04}%

%La etiqueta del punto P
\rput[b](1.5,.719){\footnotesize{$P$}}%

%La recta tangente a la curva en P: y = .169x + .415
\psplot{-.5}{5}{.169 x mul .415 add}%

%La etiqueta de la tangente buscada
\rput[l](5.1,1.277){$t$}

\end{pspicture}}
%
\subfloat[]{%
\begin{pspicture}(-.5,0)(6,4.3)
%La curva y = .08e^(x - .75) + .5
\psplot{0}{4.6}{.08 2.71828 x .75 sub exp mul .5 add}%

%La marca del punto P
\pscircle[fillstyle=solid,fillcolor=black](1.5,.669){.04}%

%La etiqueta del punto P
\rput[b](1.5,.719){\footnotesize{$P$}}%

%La recta tangente a la curva en P: y = .169x + .415
\psplot{-.5}{5}{.169 x mul .415 add}%

%La etiqueta de la tangente buscada
\rput[l](5.1,1.277){$t$}

%El punto Q
\pscircle[fillstyle=solid,fillcolor=black](4.5,3.902){.04}%
\rput[r](4.4,3.902){\footnotesize{$Q$}}%

%La secante QP
\psplot[linecolor=gray]{1}{4.9}{1.077 x mul -.947 add}%

%El punto Q1
\pscircle[fillstyle=solid,fillcolor=black](4.2,3.020){.04}%
\rput[l](4.35,3.020){\footnotesize{$Q_1$}}%

%La secante PQ1
\psplot[linecolor=gray]{1}{4.9}{.871 x mul -.637 add}%

%El punto Q2
\pscircle[fillstyle=solid,fillcolor=black](3.8,2.189){.04}%
\rput[l](4.1,2.189){\footnotesize{$Q_2$}}%

%La secante PQ2
\psplot[linecolor=gray]{1}{4.9}{.661 x mul -.322 add}%

%El punto Q3
\pscircle[fillstyle=solid,fillcolor=black](3.3,1.535){.04}%
\rput[l](3.55,1.535){\footnotesize{$Q_3$}}%

%La secante PQ3
\psplot[linecolor=gray]{1}{4.9}{.475 x mul -.043 add}%

%El punto adicional sin nombre
\pscircle[fillstyle=solid](2.6,1.009){.04}%

%La secante PQ3
\psplot[linecolor=gray]{1}{4.9}{.309 x mul .207 add}%

\end{pspicture}}
\end{center}
\end{figure}
\\
Procedamos de la siguiente manera. Imaginemos que un móvil puntual se mueve a lo largo de la curva
$C$ hacia el punto $P$ desde un punto $Q$, distinto de $P$. Sean $Q_1$, $Q_2$ y $Q_3$ algunos de
los puntos de la curva por los que el móvil pasa. Las rectas que unen cada uno de esos puntos y el
punto $P$ son rectas secantes, como se puede observar en la figura (b). El dibujo sugiere que, a
medida que el móvil \emph{está más próximo} al punto $P$, la correspondiente recta secante
\emph{está más próxima} a la recta tangente $t$; lo que se espera es que la recta tangente buscada
sea la recta a la cual se aproximan las rectas secantes obtenidas cuando el móvil se acerque al
punto $P$. A esa recta la llamaremos ``recta límite'' de las secantes.

Pero, ?`qué significa ser la ``recta límite''? Para tratar de encontrar un significado, supongamos
que la curva $C$ está en un plano cartesiano y que su ecuación es
\[
y = g(x),
\]
donde $g$ es una función real. Supongamos también que las coordenadas del punto $P$ son $(a,g(a))$.
Entonces, encontrar la recta tangente a la curva $C$ en el punto $P$ significa conocer la ecuación
de dicha recta en el sistema de coordenadas dado.

Ahora, para obtener la ecuación de una recta es suficiente conocer su pendiente (la tangente del
ángulo que forma la recta con el eje horizontal) y un punto por el que la recta pase. Como la
tangente $t$ debe pasar por $P$, ya tenemos el punto. Busquemos, entonces, la pendiente de la recta
$t$. ?`Cómo? Utilizando las pendientes de las rectas secantes que unen los puntos de la curva por
los que se desplaza el móvil desde $Q$ hacia al punto $P$, pues, así como creemos que las rectas
secantes alcanzarán la recta $t$ como una posición límite, tal vez, las pendientes de las rectas
secantes alcancen un valor límite: la pendiente de la recta tangente. Con esto en mente, calculemos
la pendiente de cualquiera de esas rectas secantes.

Sea $R$ cualquier punto de la curva $C$ que indica la posición del móvil en su trayecto desde $Q$
hasta $P$. Aunque el punto $R$ no se mueve (ni ningún otro punto de la curva), diremos que ``el
punto $R$ se mueve hacia $P$'' para indicar que es el móvil el que se está moviendo. Esto nos
permitirá indicar la posición del móvil a través de las coordenadas de los puntos de la curva. En
este sentido $R$ no indica un único punto, sino todos los puntos por dónde está pasando el móvil en
su camino hacia al punto $P$.

Sea $x$ la abscisa de $R$; entonces, sus coordenadas son $(x,g(x))$:
\begin{center}
\begin{pspicture}(-.5,-.5)(5.5,4)
%La curva y = .08e^(x - .75) + .5
\psplot{0}{4.3}{.08 2.71828 x .75 sub exp mul .5 add}%

%La marca del punto P
\pscircle[fillstyle=solid,fillcolor=black](1.5,.669){.04}%

%La etiqueta del punto P
\rput[b](1.5,.8){\footnotesize{$P$}}%

%El punto R
\pscircle[fillstyle=solid,fillcolor=black](4,2.563){.04}%
\rput[l](4.1,2.563){\footnotesize{$R$}}%

%La secante PR
\psplot[linecolor=gray]{1}{4.6}{.758 x mul -.467 add}%

%Ejes coordenadas
\psaxes[ticks=none,labels=none]{->}(5,3.6)%

%x
\rput[l](5.1,0){$x$}

%y
\rput[b](0,3.7){$y$}

%líneas para $P$
\psline[linecolor=gray,linestyle=dashed](1.5,.669)(1.5,0)%
\rput[t](1.5,-.09){\small{$a$}}%
\psline[linecolor=gray,linestyle=dashed](1.5,.669)(0,.669)%
\rput[r](-.09,.669){\small{$g(a)$}}%

%líneas para $R$
\psline[linecolor=gray,linestyle=dashed](4,2.563)(4,0)%
\rput[t](4,-.09){\small{$x$}}%
\psline[linecolor=gray,linestyle=dashed](4,2.563)(0,2.563)%
\rput[r](-.09,2.563){\small{$g(x)$}}%

%línea horizontal:
\psline[linecolor=gray,linestyle=dashed](1.5,.669)(4,.669)%
\rput[l](4,.75){\footnotesize{$S$}}

\end{pspicture}
\end{center}
Sabemos que la pendiente de la recta que pasa por $P$ y $R$ es igual a la tangente del ángulo que
forma la recta con el eje horizontal; este ángulo mide lo mismo que el ángulo $\angle SPR$ del
triángulo rectángulo $\triangle SPR$. Por lo tanto, la tangente de este ángulo es igual al cociente
entre la longitud $RS$ (que, en el caso de la curva de la figura, es igual a la diferencia $g(x) -
g(a)$) y la longitud $PS$ (que, en este caso, es igual a la diferencia $x - a \neq 0$, pues el
punto $R$ no es igual al punto $P$); es decir, si representamos con $m_x$ la tangente del ángulo
$\angle SPR$, podemos afirmar que:
\begin{equation}
\label{eqPenSec}%
m_x = \frac{g(x)-g(a)}{x-a}.
\end{equation}
La pendiente de cualquier secante que une el punto $R$, cuyas coordenadas son $(x,g(x))$ y que se
está moviendo hacia $P$, y el punto $P$ se calculará mediante la
fórmula~(\ref{eqPenSec})\footnote{La fórmula~(\ref{eqPenSec}) es válida no solamente para curvas
crecientes como la de la figura. Su validez será demostrada cuando se presente una definición
general para la pendiente de la recta tangente a una curva.}.

Ahora, notemos que el móvil ubicado en el punto $R$ se mueve hacia $P$ cuando la abscisa $x$ de $R$
se ``acerca'' hacia la abscisa $a$ del punto $P$. Entonces, el problema de obtener la pendiente de
la recta tangente a la curva $C$ en el punto $P$, en el que se concibe a dicha recta tangente como
la ``recta límite'' de las secantes que pasan por $P$ y $R$, que se aproxima a $P$, se sustituye
por el problema de encontrar un número al que los cocientes
\[
\frac{g(x)-g(a)}{x-a}
\]
se aproximan cuando el número $x$ se aproxima al número $a$. A ese número le llamaremos,
provisionalmente, ``límite de los cocientes''.

?`Y cómo se puede hallar este ``límite''? Como un primer acercamiento a la solución de este nuevo
problema, consideremos un ejemplo. Pero, antes, ampliemos un poco más el significado de
``aproximar'' que se discutió en la primera sección.

\subsection{Aproximación numérica al concepto de límite}
Supongamos que la curva $C$ es una parábola cuya ecuación es $y = 3x^2$ y el punto $P$ tiene
coordenadas $(2,12)$. Entonces, $g(x) = 3x^2$. Queremos calcular el ``límite'' de los cocientes
\[
m_x = \frac{g(x) - g(2)}{x - 2} = \frac{3x^2 - 12}{x - 2}
\]
cuando el número $x$ se aproxima al número $2$, pero $x\neq 2$. Si encontramos ese número
``límite'', lo usaremos como la pendiente de la recta tangente a la curva en el punto $(2,12)$.
Obtendremos luego la ecuación de la recta tangente.

Para empezar, observemos que, como $x\neq 2$, entonces:
\[
m_x = \frac{3x^2 - 12}{x - 2} = \frac{3(x - 2)(x + 2)}{x - 2}.
\]
Por lo tanto:
\[
m_x = 3(x + 2),
\]
para $x\neq 2$.

Para ver qué sucede con $m_x$ cuando el número $x$ se aproxima al número 2, construyamos una tabla
con los valores que $m_x$ toma para valores de $x$ próximos a 2, unos mayores y otros menores que
2:

\begin{wrapfigure}[19]{l}{.35\textwidth}
\centering
\begin{tabular}{|r|r|}\hline
\multicolumn{1}{|c|}{$x$} & \multicolumn{1}{c|}{$m_x$}\\\hline%
3 & 15 \\
2.5 & 13.5 \\
2.1 & 12.3 \\
2.01 & 12.03 \\
2.001 & 12.003 \\
2.0001 & 12.0003 \\
2.00001 & 12.00003 \\
2 & \multicolumn{1}{c|}{No existe} \\
1.99999 & 11.99997 \\
1.9999 & 11.9997 \\
1.999 & 11.997 \\
1.99 & 11.97 \\
1.9 & 11.7 \\
1.5 & 10.5 \\
1 & 9 \\ \hline
\end{tabular}
\end{wrapfigure}
La aproximación de $x$ a $2$ por valores mayores que 2 significa que el punto $R$ se aproxima al
punto $P$ desde la derecha, mientras que la aproximación de $x$ a 2 por valores menores que 2
significa que el punto $R$ se aproxima al punto $P$ desde la izquierda. En los dos casos, se puede
observar que, mientras $x$ está más cerca de $2$, $m_x$ está más cerca de 12; es decir, mientras el
punto $R$ está más cerca del punto $P$, la pendiente de la recta secante que pasa por $P$ y por $R$
está más cerca del número 12. Esta primera evidencia nos sugiere y alienta a pensar que la
pendiente de la recta tangente es igual a 12. Sin embargo, ?`cómo podemos estar seguros? Lo
siguiente nos proporciona una evidencia adicional que nos hace pensar que estamos en lo correcto.

Hemos visto que para todo $x\neq 2$, se verifica que
\[
m_x = 3(x + 2).
\]
Por otro lado, si evaluamos la expresión de la derecha en $x = 2$, obtenemos que:
\[
3(x + 2) = 3(2 + 2) = 12;
\]
que es el valor al que parece aproximarse $m_x$ cuando $x$ se aproxima a $2$.

Todo parece indicar, entonces, que el número $12$ es el ``límite'' de $m_x$ cuando $x$ se aproxima
a 2. Sin embargo, ?`podemos asegurar tal cosa?

Para poder responder esta pregunta, antes que nada necesitamos una definición para el ``límite''.
Ésta llegó en el año 1823 de la mano del matemático francés Augustin Cauchy. En la siguiente
sección vamos a estudiarla y, con ella, podremos asegurar que el límite de $m_x$ cuando $x$ se
aproxima a $2$ es, efectivamente, el número $12$.

Aceptando como verdadero este resultado por el momento, obtengamos la ecuación de la recta tangente
a la curva de ecuación $y = 3x^2$ en el punto $(2,12)$.

Ya sabemos, entonces, que la pendiente de dicha recta es igual a $12$. Recordaremos que la ecuación
de una recta de pendiente $m$ que pasa por un punto de coordenadas $(a,b)$ es
\[
y - b = m(x - a).
\]
En este caso $m = 12$ y $(a,b) = (2,12)$. Entonces, la ecuación de la recta tangente a la curva de
ecuación $y = 3x^2$ que pasa por el punto $(2,12)$ es
\[
y - 12 = 12(x - 2),
\]
que puede ser escrita de la siguiente manera:
\[
y = 12x - 12.
\]
En resumen:
\marcojc{.9}{1.5}{black}{black}{white}{%
La pendiente de la recta tangente a la curva $y = 3x^2$ en el punto $(2,12)$ es igual al límite de
\[
m_x = \frac{3x^2 - 12}{x - 2} = 3(x + 2),
\]
cuando $x$, siendo distinto de $2$, se aproxima a 2. Este límite es igual al número $12$.

La ecuación de la recta tangente es:
\[
y = 12x - 12.
\]
\eijc{-1.25} }

Aparte de la definición de límite, persiste aún otro problema: ?`podemos asegurar que la recta
encontrada es la recta tangente? Es decir, ?`cómo podemos estar seguros de que en el espacio entre
la recta de ecuación $y = 12x - 12$ y la curva $y = 3x^2$ no se interpondrá ninguna otra recta,
alrededor del punto $(2,12)?$

Más adelante, en un capítulo posterior, provistos ya con el concepto de límite dado por Agustin
Cauchy, probaremos que el método seguido para la consecución de la recta tangente es correcto y
general.

\begin{multicols}{2}[\subsection{Ejercicios}]
\begingroup
\small
\begin{enumerate}[leftmargin=*]
\item Sean $C$ una curva cuya ecuación es $y = x^3$, $s$ un número real distinto de $1$ y $m_s$
    la pendiente de la recta secante a $C$ en los puntos de coordenadas $(1,1)$ y $(s,s^3)$.
    \begin{enumerate}[leftmargin=*]
    \item Calcule $m_s$.
    \item Elabore una tabla de dos columnas. En la primera, coloque valores de $s$ cercanos
        a $1$; en la segunda, los valores de $m_s$ correspondientes. Con la ayuda de esta
        tabla, determine un candidato para el valor de la pendiente de la recta tangente a
        la curva $C$ en el punto de coordenadas $(1,1)$.
    \item Use el valor de la pendiente hallado en el literal anterior para escribir la
        ecuación de la recta tangente.
    \item Dibuje la curva $C$, las secantes correspondientes para $s\in\{1.1, 1.5, 2\}$ y
        la recta tangente.
    \end{enumerate}
\item Para cada una de las funciones definidas a continuación, elabore una tabla para los
    valores $f(x_i)$ con $x_i = a \pm 10^{-i}$ con $i\in\{1,2,\ldots, 5\}$. ?`Tiene $f(x)$ un
    ``límite'' cuando $x$ se aproxima a $a$? En otras palabras, ?`existe un número al que $f(x)$
    parece acercarse cuando $x$ toma valores cercanos al número $a$?
    \begin{enumerate}[leftmargin=*]
    \item
    \[
    f(x) =
        \begin{cases}
        3 + 2x - x^2 & \text{si } x < 1, \\
        x^2 - 4x + 7 & \text{si } x > 1,
        \end{cases}
        \quad\text{$a = 1$.}
    \]

    \item
    \[
    f(x) =
        \begin{cases}
        x^2 - 4x + 5 & \text{si } x < 1, \\
        3 & \text{si } x = 1, \\
        x + 3 & \text{si } x > 1,
        \end{cases}
        \quad a = 1.
    \]

    \item
    \[
        f(x) = \frac{3x - 15}{\sqrt{x^2 - 10x + 25}},\quad a = 5.
    \]

    \item
    \[
        f(x) = \frac{\sin(3x)}{2x},\quad a = 0.
    \]

    \item
    \[
        f(x) = \frac{1}{(x - 2)^2},\quad a = 2.
    \]


    \item
    \[
        f(x) = \sin\left(\frac{\pi}{x}\right),\quad a = 0.
    \]

    \end{enumerate}

    ?`Guardan alguna relación el hecho de que la función esté o no definida en $a$ y que parezca
    tener ``límite'' cuando $x$ se aproxima al número $a$?

\item El método utilizado en los ejercicios anteriores para encontrar el ``límite'' de una
    función puede sugerir la no necesidad de elaborar un concepto adecuado para la definición
    del límite y el correspondiente desarrollo de técnicas de cálculo. Sin embargo, la función
    $h$ definida por
    \[
      h(x) = \frac{\sqrt[3]{x^3 + 8} - 2}{x^3}
    \]
    nos alerta sobre el método heurístico de la ``aproximación numérica''. Intente determinar
    un número al que parece acercarse $h(x)$ cuando $x$ toma valores cercanos al número $0$
    utilizando el procedimiento propuesto en el ejercicio anterior.
\end{enumerate}
\endgroup
\end{multicols}

\section{La definición de límite}
Vamos a tratar de precisar lo que queremos decir con ``el límite de los cocientes
\[
m_x = \frac{3x^2 - 12}{x-2}
\]
es $12$ cuando $x$ se aproxima a $2$''.

Para empezar, este cociente no está definido en $x = 2$, pero para todo $x\neq 2$:
\[
m_x = 3(x + 2).
\]
?`Existirá algún $x\neq 2$ para el cual $m_x = 12$? Si así fuera, entonces
\[
3(x + 2) = 12.
\]
De aquí, obtendríamos que
\[
x = 2.
\]
Pero esto es absurdo, pues $x\neq 2$. ?`Qué podemos concluir? Que para $x\neq 2$:
\[
m_x \neq 12.
\]

\label{eqLimBeginPLim}Sin embargo, vimos en la sección anterior, que para valores de $x$ cercanos a
$2$, $m_x$ toma valores cercanos a $12$, aunque nunca tomará el valor $12$. Surge, entonces, la
siguiente pregunta: ?`qué tan cerca de 12 puede llegar $m_x$? En otras palabras, ?`podemos encontrar
valores de $x$, cercanos a 2, para los cuales $m_x$ no difiera de $12$ en alguna cantidad dada; por
ejemplo, que no difiera en más de $10^{-2}$ (es decir, en más de $0.01$)? Y ?`qué tan cerca debe
estar $x$ del número $2$ para que ello ocurra? Para responder esta pregunta, formulemos el problema
con mayor precisión.

En primer lugar, ?`qué queremos decir con ``que $m_x$ no difiera de $12$ en más de $10^{-2}$''? Que
el error de aproximar $12$ con $m_x$, es decir, el valor absoluto de la diferencia entre $m_x$ y
$12$ sea menor que $10^{-2}$. En otras palabras, que se verifique la siguiente desigualdad:
\[
|m_x - 12| < 10^{-2}.
\]

En segundo lugar, ?`existen valores de $x$ para los que se cumple esta desigualdad? Y si existen,
?`qué tan cerca de $2$ deberán estar los $x$? Más aún,  Para responder estas preguntas, primero
notemos que podemos medir la cercanía de $x$ a $2$ mediante el valor absoluto de la diferencia
entre $x$ y $2$:
\[
|x - 2|.
\]
En efecto, mientras más pequeño sea este valor absoluto, $x$ estará más cercano a $2$; por el
contrario, mientras más grande sea, $x$ estará más lejos de $2$. Por ello a $|x - 2|$ nos
referiremos también como ``distancia de $x$ a $2$''.

Notemos también que como $x$ es diferente de $2$, entonces se debe cumplir la desigualdad:
\[
0 < |x - 2|.
\]
En todo lo que sigue, supondremos que $x\neq 2$.

Ahora bien, la pregunta:
\begin{quote}
{\bfseries ?`qué tan cerca debe $x$ estar del número $2$ para asegurar que
\[
\bm{|m_x - 12| < 10^{-2}\text{?}}
\]
}
\end{quote}
equivale a la siguiente:
\begin{quote}
{\bfseries ?`a qué distancia debe estar $x$ de $2$ para asegurar que
\[
\bm{|m_x - 12| < 10^{-2}\text{?}}
\]
}
\end{quote}
A su vez, esta segunda pregunta equivale a esta otra:
\begin{quote}
{\bfseries ?`a qué cantidad debe ser inferior $|x - 2|$ para asegurar que
\[
\bm{|m_x - 12| < 10^{-2}\text{?}}
\]
}
\end{quote}
Y esta tercera pregunta equivale a la siguiente:
\begin{quote}
{\bfseries ?`existe un número $\delta > 0$ tal que
\[
\bf{\text{si } \ 0 < |x - 2| < \delta, \ \text{ entonces } \ |m_x - 12| < 10^{-2}\text{?}}
\]
}
\end{quote}

Responder esta pregunta equivale a resolver el siguiente problema:
\marcojc{.9}{1.5}{black}{black}{white}{%
Si para todo valor de $x\neq 2$ se tiene que
\[
m_x = 3(x + 2),
\]
se busca un número real $\delta > 0$ tal que, si las desigualdades
\begin{equation}
\label{eqLim012}
0 < |x - 2| < \delta
\end{equation}
fueran verdaderas, la desigualdad
\begin{equation}
\label{eqLim013}
|m_x - 12| < 10^{-2}
\end{equation}
también sería verdadera.}

\subsection{Solución del problema} Este es un problema de \emph{búsqueda}: debemos encontrar el
número $\delta$. Para ello, el método que vamos a aplicar consiste en suponer temporalmente que
$\delta$ ya ha sido encontrado; es decir, suponer que si $x$ es un número tal que $x\neq 2$ y que
satisface la desigualdad
\begin{equation}
\label{eqLim002} |x - 2| < \delta,
\end{equation}
entonces, se debe cumplir la desigualdad
\begin{equation}
\label{eqLim003} |m_x - 12| = |3(x+2) - 12| < 10^{-2}.
\end{equation}
A partir de esta última desigualdad vamos a tratar de encontrar propiedades del número $\delta$,
aún desconocido, que nos permitan hallarlo.

?`Qué camino seguir? Para no hacerlo a ciegas, el trabajo que realicemos con la
desigualdad~(\ref{eqLim003}), o con una parte de ella, debe llevarnos de alguna manera a $\delta$;
es decir, debe llevarnos a la desigualdad~(\ref{eqLim002}) o a una similar. Con esto en mente,
empecemos el trabajo con el miembro izquierdo de la desigualdad~(\ref{eqLim003}), en el cual
podemos aplicar propiedades conocidas del valor absoluto de un número:
\begin{align*}
|3(x + 2) - 12| &= |3x + 6 - 12| \\
&= |3x - 6| \\
&= 3|x - 2|.
\end{align*}
Es decir,
\begin{equation}
\label{eqLim004}%
|3(x + 2) - 12| = 3|x-2|.
\end{equation}
Pero hemos supuesto que
\[
|x - 2| < \delta.
\]
Entonces:
\[
3|x - 2| < 3\delta.
\]
Por lo tanto, por la propiedad transitiva de la relación ``menor que'', vemos que esta última
desigualdad y la igualdad~(\ref{eqLim004}) implican una nueva desigualdad:
\begin{equation}
\label{eqLim008}%
|3(x + 2) - 12| < 3\delta.
\end{equation}
En resumen:
\marcojc{.9}{1.5}{black}{black}{white}{%
bajo el supuesto de que existe el número $\delta > 0$, si $x\neq 2$ satisficiera la desigualdad
\[
\tag{\ref{eqLim002}}%
|x - 2| < \delta,
\]
se debería satisfacer la desigualdad
\[
\tag{\ref{eqLim008}}%
|3(x + 2) - 12| < 3\delta.
\]
En otras palabras,
\[
\text{si } \ 0 < |x - 2| < \delta, \ \text{ entonces } \ |3(x + 2) - 12| < 3\delta.
\]
\eijc{-.9}}%
Recordemos que queremos que el número $\delta$ que encontremos nos garantice el cumplimiento de la
desigualdad
\[
\tag{\ref{eqLim003}}%
|3(x + 2) - 12| < 10^{-2}.
\]
Para lograrlo, comparemos entre sí las desigualdades~(\ref{eqLim008}) y (\ref{eqLim003}). ?`Qué
podemos observar? Que si el número $3\delta$ fuera menor o igual que $10^{-2}$, entonces
obtendríamos:
\[
|3(x + 2) - 12| < 3\delta \leq 10^{-2};
\]
es decir:
\[
\text{si } \ 3\delta \leq 10^{-2}, \ \text{ entonces } \ |3(x + 2) - 12| < 10^{-2},
\]
donde la desigualdad de la derecha es la que queremos obtener. Por lo tanto, como la desigualdad
\[
3\delta \leq 10^{-2}
\]
es equivalente a la desigualdad
\[
\delta \leq \frac{10^{-2}}{3},
\]
podemos asegurar que:
\marcojc{.9}{1.5}{black}{black}{white}{%
si se elige el número $\delta > 0$ tal que
\[
\delta \leq \frac{10^{-2}}{3}
\]
y si $x\neq 2$ satisface la desigualdad
\[
\tag{\ref{eqLim002}}
|x - 2| < \delta,
\]
la desigualdad requerida
\[
\tag{\ref{eqLim003}}
|3(x + 2) - 12| < 10^{-2}.
\]
es satisfecha. Es decir,
\[ \text{si } 0 < \delta \leq  \frac{10^{-2}}{3}\ \text{ y } \
0 < |x - 2| < \delta, \ \text{ entonces } \ |3(x + 2) - 12| < 10^{-2}.
\]
\eijc{-1.5}\label{eqLimEndPLim}} Y esto es precisamente lo que queríamos hacer.

Resumamos el procedimiento seguido para la búsqueda de $\delta$. Observemos que consiste de dos
etapas: {\label{eqLim017}\begin{enumerate}
\item La \textit{búsqueda} del número $\delta$. En esta etapa se supone encontrado el número
    $\delta$. Bajo esta suposición, se encuentra uno o más valores candidatos para el número
    $\delta$.
\item La \textit{constatación} de que el valor o valores encontrados para $\delta$ satisfacen,
    efectivamente, las condiciones del problema.
\end{enumerate}}

Para nuestro caso, estas etapas se ejemplifican así:
\begin{enumerate}
\item \textit{Búsqueda}: se supone que existe un número $\delta > 0$ tal que para $x\neq 2$:
\[
\text{si } \ |x - 2| < \delta, \ \text{ entonces } \ |3(x + 2) - 12| < 10^{-2}.
\]
Trabajando con la expresión $|3(x + 2) - 12|$ y bajo las suposiciones de que $|x - 2| < \delta$
y $x\neq 2$, se demuestra que
\[
|3(x + 2) - 12| < 3\delta.
\]
Esta última desigualdad sugiere que el número $\delta > 0$ debe satisfacer la desigualdad:
\[
3\delta \leq 10^{-2},
\]
lo que equivale a sugerir que $\delta$ debe cumplir esta otra desigualdad:
\begin{equation}
\label{eqLim009}%
\delta \leq \frac{10^{-2}}{3}.
\end{equation}

\item \textit{Constatación}: con la elección del número $\delta$ que satisface la desigualdad
    (\ref{eqLim009}) y bajo los supuestos de que $x\neq 2$ y $|x - 2| < \delta$, se verifica el
    cumplimiento de la desigualdad
\[
|3(x + 2) - 12| < 10^{-2}.
\]
\end{enumerate}

La solución que acabamos de encontrar al problema planteado nos permite responder la pregunta:
\begin{quote}
{\bfseries ?`qué tan cerca debe estar $\bm{x}$ del número $\bm{2}$ para asegurar que $\bm{m_x}$
difiera de $\bm{12}$ en menos de $\bm{10^{-2}}$?}
\end{quote}
La respuesta es:
\begin{quote}
{\bfseries si $\bm{x}$ difiere de $\bm{2}$ en menos de $\bm{\displaystyle{\frac{10^{-2}}{3}}}$,
$\bm{m_x}$ difiere de $\bm{12}$ en menos de $\bm{10^{-2}}$.}
\end{quote}

?`Podremos encontrar valores de $x$ cercanos a $2$ que garanticen que $m_x$ difiera de $12$ en una
cantidad aún más pequeña que $10^{-2}$? Por ejemplo, ?`qué difiera en menos de $10^{-6}$? La
respuesta es afirmativa, pues, si repasamos el modo cómo se resolvió este mismo problema para el
caso en que queríamos que $m_x$ difiriera de $12$ en menos de $10^{-2}$, descubriremos lo
siguiente:
\begin{quote}
{\bfseries si $\bm{x}$ difiere de $\bm{2}$ en menos de $\bm{\displaystyle{\frac{10^{-6}}{3}}}$,
$\bm{m_x}$ difiere de $\bm{12}$ en menos de $\bm{10^{-6}}$.}
\end{quote}

Si el lector tiene dudas de este resultado, deberá leer una vez más,
\vpagerefrange{eqLimBeginPLim}{eqLimEndPLim}, el procedimiento para resolver el problema cuando la
diferencia entre $m_x$ y $12$ difería en menos de $10^{-2}$. Cada vez que encuentre un $10^{-2}$,
deberá sustituirlo por un $10^{-6}$. Esto lo convencerá del todo.

Y ahora podemos responder a una pregunta más general:
\begin{quote}
{\bfseries ?`qué tan cerca debe estar $x$ del número $2$ para asegurar que $m_x$ difiera de $12$ en
menos de $\epsilon$?}
\end{quote}
donde $\epsilon$ representa cualquier número positivo. Y la respuesta la encontraremos de manera
idéntica a cómo hemos respondido las dos preguntas anteriores; esa respuesta será la siguiente:
\begin{quote}
{\bfseries si $\bm{x}$ difiere de $\bm{2}$ en menos de $\bm{\displaystyle{\frac{\epsilon}{3}}}$,
$\bm{m_x}$ difiere de $\bm{12}$ en menos de $\bm{\epsilon}$.}
\end{quote}

Como este número $\epsilon$ puede ser cualquier número positivo, puede ser elegido tan pequeño como
queramos. Y lo que ya sabemos es que, en esa situación, si $x$ es tal que
\[
|x - 2| < \frac{\epsilon}{3},
\]
garantizamos que
\[
|m_x - 12| < \epsilon.
\]
Es decir, aseguraremos para tales $x$ que $m_x$ estará tan cerca del número $12$ como queramos.

Es más, podremos afirmar que se garantiza que $m_x$ está tan cerca como se desee del número $12$,
si $x$ está lo suficientemente cerca de $2$. En efecto, en este caso, para que la distancia de
$m_x$ a $12$ sea menor que $\epsilon$, bastará que la distancia de $x$ a $2$ sea menor que
$\frac{\epsilon}{3}$.

Esto también puede ser expresado de la siguiente manera:

\begin{quote}
{\bfseries $\bm{12}$ puede ser aproximado por los cocientes $\bm{m_x}$ con la precisión que se
desee, con la condición de que $\bm{x}$, siendo distinto de $\bm{2}$, esté suficientemente cerca de
$\bm{2}$.}
\end{quote}
Y, cuando una situación así ocurre, siguiendo a Cauchy, diremos que
\begin{quote}
{\bfseries $\bm{12}$ es el límite de $\bm{m_x}$ cuando $\bm{x}$ se aproxima al número $\bm{2}$ y
escribiremos:
\[
\bm{12 =} \bm{\limjc{m_x}{x}{2}}.
\]
}
\end{quote}

Al inicio de la sección, nos habíamos propuesto precisar la frase ``el límite de los cocientes
$m_x$ es $12$ cuando $x$ se aproxima a $2$''. De lo mostrado anteriormente, vemos que esta frase
debe ser cambiada por la siguiente: ``12 es aproximado por los cocientes $m_x$ con la precisión que
se desee, con tal que $x$, siendo distinto de $2$, esté lo suficientemente cerca de $2$''. Y ahora
esta frase tiene pleno sentido.

El proceso seguido para afirmar que $12$ es el límite de $m_x$ puede ser resumido de la siguiente
manera:
\marcojc{.9}{1.5}{black}{black}{white}{%
dado cualquier número $\epsilon > 0$, encontramos un número $\delta > 0$, que en nuestro caso fue
$\frac{\epsilon}{3}$, tal que $12$ puede ser aproximado por $m_x$ con un error de aproximación
menor que $\epsilon$, siempre que $x$, siendo distinto de $2$, se aproxime a $2$ a una distancia
menor que $\delta$.} Y este texto puede ser expresado simbólicamente mediante desigualdades de la
siguiente manera:
\marcojc{.9}{1.5}{black}{black}{white}{%
dado cualquier número $\epsilon > 0$, encontramos un número $\delta > 0$ tal que
\[
|m_x - 12| < \epsilon,
\]
siempre que
\[
0 < |x - 2| < \delta.
\]
\eijc{-1.5}} Y toda esta afirmación se expresa de manera simple por:
\[
12 = \limjc{m_x}{x}{2}.
\]

A partir de este ejemplo vamos a formular una definición general de límite de una función real.

\subsection{La definición de límite}
Sean:
\begin{enumerate}
\item $a$ y $L$ dos números reales,
\item $I$ un intervalo abierto que contiene el número $a$, y
\item $f$ una función real definida en $I$, salvo, tal vez, en $a$; es decir, $I \subset \Dm(f)
    \cup \{a\}$.
\end{enumerate}
El número $a$ generaliza a $2$, $L$ a $12$, $I$ a $(-\infty,+\infty)$ y $f(x)$ a $m_x$.

Lo que vamos a definir es:
\begin{quote}
{\bfseries $\bm{L}$ es el límite de $\bm{f(x)}$ cuando $\bm{x}$ se aproxima a $\bm{a}$.}
\end{quote}
Igual que en el ejemplo, esta frase se deberá entender como:
\begin{quote}
{\bfseries $\bm{L}$ puede ser aproximado por los valores de $\bm{f(x)}$ con la precisión que se
desee, con la condición de que $\bm{x}$, siendo distinto de $\bm{a}$, sea lo suficientemente
cercano a $\bm{a}$.}
\end{quote}
O, de forma equivalente, se entenderá como:
\begin{quote}
{\bfseries $\bm{L}$ es el límite de $\bm{f(x)}$ cuando $\bm{x}$ se aproxima a $\bm{a}$, lo que se
representará por:
\[
\bm{L =} \bm{\limjc{f(x)}{x}{a},}
\]
si para cualquier número $\bm{\epsilon > 0}$, existe un número $\bm{\delta > 0}$ tal que $\bm{L}$
puede ser aproximado por $\bm{f(x)}$ con un error de aproximación menor que $\bm{\epsilon}$,
siempre que $\bm{x}$, siendo distinto de $\bm{a}$, se aproxime a $\bm{a}$ a una distancia menor que
$\bm{\delta}$. }
\end{quote}
Finalmente, todo lo anterior nos lleva a la siguiente definición:

\begin{defical}[Límite de una función]\label{def:Limite} Sean:
\begin{enumerate}
\item $a$ y $L$ dos números reales,
\item $I$ un intervalo abierto que contiene el número $a$, y
\item $f$ una función real definida en $I$, salvo, tal vez, en $a$; es decir, $I \subset \Dm(f)
    \cup \{a\}$.
\end{enumerate}
Entonces:
\[
L = \limjc{f(x)}{x}{a}
\]
si y solo si para todo $\epsilon > 0$, existe un $\delta > 0$ tal que
\[
|f(x) - L| < \epsilon,
\]
siempre que
\[
0 < |x - a| < \delta.
\]
\end{defical}

Veamos algunos ejemplos para familiarizarnos con esta definición.

\begin{exemplo}[Solución]{\label{ex:lim001}%
Probemos que
\[
9 = \limjc{x^2}{x}{-3}.
\]
Es decir, probemos que $9$ puede ser aproximado por valores de $x^2$ siempre y cuando elijamos
valores de $x$ lo suficientemente cercanos a $-3$.}%
Para ello, de la definición de límite, sabemos que, dado cualquier $\epsilon
> 0$, debemos hallar un $\delta
> 0$ tal que
\begin{equation}
\label{eqLim018}
|x^2 - 9| < \epsilon
\end{equation}
siempre que $x \neq -3$ y
\begin{equation}
\label{eqLim019}
|x + 3| < \delta.
\end{equation}

\paragraph{Búsqueda de $\delta$:}
Empecemos investigando el miembro izquierdo de la desigualdad~(\ref{eqLim018}), que mide el error
de aproximar $9$ con $x^2$. Podemos expresarlo así:
\[
|x^2 - 9| = |(x - 3)(x + 3)| = |x - 3||x + 3|.
\]
Entonces, debemos encontrar los $x \neq -3$, pero cercanos a $-3$, para los que se verifique la
desigualdad:
\begin{equation}
\label{eqLim020}
|x - 3||x + 3| < \epsilon,
\end{equation}
que es equivalente a la desigualdad~(\ref{eqLim018}). Es decir, debemos encontrar los $x\neq -3$,
pero cercanos a $-3$, que hacen que el producto
\begin{equation}
\label{eqLim021}
|x - 3||x + 3|
\end{equation}
esté tan cerca de $0$ como se quiera.

Ahora bien, como $x$ está cerca a $-3$, el factor $|x+3|$ estará cerca de $0$. Por lo tanto, para
que el producto~(\ref{eqLim021}) esté tan cerca de $0$ como se quiera, será suficiente con que el
otro factor, $|x - 3|$, no supere un cierto límite; es decir, esté acotado por arriba. Veamos si
ése es el caso. Para ello, investiguemos el factor $|x - 3|$.

Como $x$ debe estar cerca a $-3$, consideremos solamente valores de $x$ que estén en un intervalo
que contenga al número $-3$. Por ejemplo, tomemos valores de $x$ distintos de $-3$ y tales que
estén en el intervalo con centro en $-3$ y radio 1, como el que se muestra en la siguiente figura:
\begin{center}
\begin{pspicture}(-4,-.5)(1,.5)
\psaxes[yAxis=false,labelsep=-20pt]{->}(0,0)(-4,0)(1,0)%
\psframe[hatchcolor=gray,fillstyle=hlines,hatchangle=45,linestyle=none,hatchsep=2pt](-4,-.1)(-2,.1)%
\end{pspicture}
\end{center}
Esto significa que $x$ debe cumplir las siguientes desigualdades:
\[
-4 < x < -2,
\]
que son equivalentes a estas otras:
\begin{equation}
\label{eqLim023}
-1 < x + 3 < 1,
\end{equation}
las que, a su vez, son equivalentes a la siguiente:
\begin{equation}
\label{eqLim022}
|x + 3| < 1.
\end{equation}

Bajo la suposición del cumplimiento de estas desigualdades, veamos si el factor $|x-3|$ está
acotado por arriba. Para ello, construyamos el factor $|x - 3|$ a partir de las
desigualdades~(\ref{eqLim023}).

En primer lugar, para obtener la diferencia $x - 3$, podemos sumar el número $-6$ a los miembros de
estas desigualdades. Obtendremos lo siguiente:
\begin{equation}
\label{eqLim024}
-7 < x - 3 < -5.
\end{equation}
Pero esto significa que, para $x \neq -3$ tal que
\[
\tag{\ref{eqLim022}}
|x + 3| < 1,
\]
la diferencia
\[
x - 3
\]
es negativa, y, por lo tanto:
\[
|x - 3| = -(x - 3);
\]
es decir:
\[
x - 3 = -|x - 3|,
\]
con lo cual podemos reescribir las desigualdades~(\ref{eqLim024}) de la siguiente manera:
\[
-7 < -|x - 3| < -5,
\]
que son equivalentes a las siguientes:
\[
7 > |x - 3| > 5.
\]

Hemos probado que, si $x\neq -3$ es tal que $|x + 3| < 1$, se verifica que el factor $|x - 3|$ está
acotado superiormente por el número $7$:
\[
|x - 3| < 7.
\]

Con este resultado, volvamos al producto~(\ref{eqLim021}). Ya podemos afirmar que, si $x\neq -3$ y
$|x + 3| < 1$, se debe verificar lo siguiente:
\[
|x^2 - 9| = |x - 3||x+3| < 7|x + 3|,
\]
es decir:{\bfseries
\begin{equation}
\label{eqLim025}
\bm{|x^2 - 9| < 7|x + 3|,}
\end{equation}
siempre que
\[
\bm{x\neq -3 \quad\text{y}\quad |x + 3| < 1}.
\]}

Ahora, si elegimos los $x\neq -3$ tales que $|x + 3| < 1$ y
\begin{equation}
\label{eqLim026}
7|x + 3| < \epsilon,
\end{equation}
por la desigualdad~(\ref{eqLim025}), obtendríamos la desigualdad~(\ref{eqLim018}):
\[
|x^2 - 9| < \epsilon,
\]
es decir, haríamos que el error que se comete al aproximar $9$ por $x^2$ sea menor que $\epsilon$.

Pero elegir $x$ de modo que se cumpla la desigualdad~(\ref{eqLim026}) equivale a elegir a $x$ de
modo que se cumpla la desigualdad:
\begin{equation}
\label{eqLim027}
|x + 3| < \frac{\epsilon}{7}.
\end{equation}

Por lo tanto: {\bfseries la desigualdad
\[
\tag{\ref{eqLim018}}
\bm{|x^2 - 9| < \epsilon}
\]
será satisfecha si se eligen los $\bm{x\neq -3}$ tales que se verifiquen, simultáneamente, las
desigualdades:
\[
\bm{|x + 3| < 1 \quad\text{y}\quad |x + 3| < \frac{\epsilon}{7}}.
\]
}

Esto quiere decir que para el $\delta > 0$ buscado, la desigualdad $|x + 3| < \delta$ deberá
garantizar el cumplimiento de estas dos desigualdades. ?`Cómo elegir $\delta$? Pues, como el más
pequeño entre los números $1$ y $\frac{\epsilon}{7}$:
\[
\delta = \min\left\{1,\frac{\epsilon}{7}\right\}.
\]
Al hacerlo, garantizamos que
\[
\delta \leq 1\quad\text{y}\quad \delta \leq \frac{\epsilon}{7},
\]
de donde, si $x\neq -3$ tal que
\[
|x + 3| < \delta,
\]
entonces
\[
|x + 3| < \delta \leq 1 \quad\text{y}\quad |x + 3| < \delta \leq \frac{\epsilon}{7},
\]
con lo cual garantizamos que se verifique la desigualadad~(\ref{eqLim018}):
\[
\tag{\ref{eqLim018}}
|x^2 - 9| < \epsilon.
\]

En resumen:
\marcojc{.9}{1.5}{black}{black}{white}{%
dado $\epsilon > 0$, podemos garantizar que
\[
|x^2 - 9| < \epsilon,
\]
siempre que elijamos $x\neq -3$ tal que
\[
|x + 3| < \min\left\{1,\frac{\epsilon}{7}\right\}.
\]
En otras palabras, el número $9$ puede ser aproximado tanto como se quiera por $x^2$ siempre que
$x$ esté lo suficientemente cerca de $-3$. Esto demuestra que $ 9 = \displaystyle\lim_{x \to -3}{x^2}.$}\vspace*{-1.4\baselineskip}
\end{exemplo}

Antes de estudiar otro ejemplo, tratemos de obtener un procedimiento para demostrar que un número
$L$ es el límite $f(x)$ cuando $x$ se aproxima al número $a$, a partir del que acabamos de utilizar
para encontrar el $\delta$ dado el $\epsilon$.

Si leemos la demostración realizada una vez más, podemos resumir, en los siguientes pasos, el
procedimiento seguido:
\begin{enumerate}
\item El valor absoluto de la diferencia entre $x^2$ y el límite $9$ se expresó como el
    producto de dos factores en valor absoluto:
    \[
        |x^2 - 9| = |x-3||x+3|.
    \]
    Uno de los factores es el valor absoluto de la diferencia entre $x$ y el número $-3$, que
    es el número a dónde se aproxima $x$.

\item Como $|x + 3|$ debe ser menor que el número $\delta$, para que la cantidad
\[
|x^2 - 9|
\]
sea tan pequeña como se desee (es decir, menor que $\epsilon$), se busca una cota superior para
el segundo factor $|x - 3|$. En este ejemplo, se encontró que:
\[
|x - 3| < 7,
\]
bajo el supuesto de que $|x+3| < 1$. Esta última suposición es realizada porque los $x$ deben
estar cerca de $-3$, por lo que se decide trabajar con valores de $x$ que estén en un intervalo
con centro en el número $3$. En este caso, un intervalo de radio $1$.

\item El resultado anterior sirve de prueba de la afirmación:
\[
|x^2 - 9| < 7|x + 3|,
\]
siempre que $0 < |x + 3| < \delta$ y $|x + 3| < 1$.

\item Para obtener $|x^2 - 9| < \epsilon$ siempre que $0 < |x + 3| < \delta$, el resultado
    precedente nos dice cómo elegir el número $\delta$:
\[
\delta = \min\left\{1, \frac{\epsilon}{7}\right\}.
\]
La elección de este $\delta$ muestra que:
\begin{align*}
|x^2 - 9| & = |x - 3||x + 3| \\
& < 7|x + 3|,\quad \text{pues }\ |x + 3| < \delta \leq 1, \\
& < 7 \delta, \quad \text{pues }\ |x + 3| < \delta, \\
& \leq 7\frac{\epsilon}{7} = \epsilon, \quad\text{pues }\ \delta \leq \frac{\epsilon}{7}.
\end{align*}
Por lo tanto:
\[
|x^2 - 9| < \epsilon,
\]
siempre que
\[
0 < |x + 3| < \delta = \min\left\{1, \frac{\epsilon}{7}\right\}.
\]
\end{enumerate}

Generalicemos este procedimiento para probar que
\[
L = \limjc{f(x)}{x}{a}.
\]
\begin{enumerate}
\item Hay que tratar de expresar el valor absoluto de la diferencia entre $f(x)$ y el límite
    $L$ para $x\neq a$ de la siguiente manera:
    \[
        |f(x) - L| = |g(x)||x-a|.
    \]

\item Se busca una cota superior para el factor $|g(x)|$. Supongamos que esta cota superior sea
    el número $M > 0$. Entonces, debe cumplirse la siguiente desigualdad:
    \[
    |g(x)| < M.
    \]
    Probablemente, para encontrar este valor $M$ haya que suponer adicionalmente que
    \[
        0 < |x - a| < \delta_1,
    \]
    con cierto $\delta_1 > 0$. Esta suposición puede ser el resultado de trabajar con valores
    cercanos al número $a$, para lo cual se decide trabajar con valores de $x$ en un intervalo
    con centro en el número $a$.

\item El resultado anterior sirve de prueba de la afirmación:
\[
|f(x) - L| < M|x - a| < \epsilon,
\]
siempre que $0 < |x - a| < \frac{\epsilon}{M}$ y $|x - a| < \delta_1$.

\item El resultado precedente nos dice cómo elegir el número $\delta$:
\[
\delta = \min\left\{\delta_1, \frac{\epsilon}{M}\right\}.
\]
La elección de este $\delta$ muestra que:
\begin{align*}
|f(x) - L| & = |g(x)||x - a| \\
& < M|x - a|,\quad \text{pues }\ |g(x)| < M \ \text{debido a que} \ |x - a| < \delta \leq \delta_1, \\
& < M \delta, \quad \text{pues }\ |x - a| < \delta, \\
& \leq M\frac{\epsilon}{M} = \epsilon, \quad\text{pues }\ \delta \leq \frac{\epsilon}{M}.
\end{align*}
Por lo tanto:
\[
|f(x) - L| < \epsilon,
\]
siempre que
\[
0 < |x - a| < \delta = \min\left\{\delta_1, \frac{\epsilon}{M}\right\}.
\]
\end{enumerate}

Apliquemos este procedimiento en el siguiente ejemplo.

\begin{exemplo}[Solución]{Demostremos que
\[
2 = \limjc{\frac{5x - 3}{x + 3}.}{x}{3}
\]
Es decir, probemos que el número $2$ puede ser aproximado por valores de
\[
\frac{5x - 3}{x + 3}
\]
siempre que $x\neq 3$ esté lo suficientemente cerca de $3$.} Dado $\epsilon > 0$, debemos encontrar
un número $\delta > 0$ tal que
\begin{equation}
\label{eqLim028}
\left|\frac{5x-3}{x+3} - 2\right| < \epsilon
\end{equation}
siempre que $x\neq 3$ y
\begin{equation}
\label{eqLim029}
|x - 3| < \delta.
\end{equation}

Vamos a aplicar el procedimiento descrito. Lo primero que tenemos que hacer es expresar el valor
absoluto de la diferencia entre $\frac{5x-3}{x+3}$ y $2$ para $x\neq 2$, de la siguiente manera:
\begin{equation}
\label{eqLim030}
\left|\frac{5x-3}{x+3} - 2\right| = |g(x)||x - 3|.
\end{equation}

Para ello, trabajemos con el lado izquierdo de la desigualdad~(\ref{eqLim028}). Éste puede ser
simplificado de la siguiente manera:
\begin{align*}
\left|\frac{5x-3}{x+3} - 2\right| &= \left|\frac{5x-3 - 2(x+3)}{x+3}\right| \\
&= \left|\frac{3x-9}{x+3}\right| = 3\left|\frac{x-3}{x+3}\right|\\
&= 3\frac{|x-3|}{|x+3|}.
\end{align*}
Es decir:
\begin{equation}
\label{eqLim036}
\left|\frac{5x-3}{x+3} - 2\right| = \frac{3}{|x+3|}|x-3|.
\end{equation}

Entonces, si definimos:
\[
g(x) = \frac{3}{x+3},
\]
ya tenemos la forma~(\ref{eqLim030}).

Lo segundo que hay que hacer es encontrar una cota superior para $g(x)$. Es decir, debemos hallar
un número positivo $M$ y, posiblemente, un número positivo $\delta_1 > 0$ tales que:
\begin{equation}
\label{eqLim033}
\frac{3}{|x + 3|} < M,
\end{equation}
siempre que
\begin{equation*}
0 < |x - 3| < \delta_1.
\end{equation*}

Para ello, como los $x$ deben tomar valores cercanos al número $3$, consideremos valores para $x$
que estén en el intervalo con centro en $3$ y radio $1$:
\begin{center}
\begin{pspicture}(0,-.5)(5,.5)
\psaxes[yAxis=false,labelsep=-20pt]{->}(0,0)(0,0)(5,0)%
\psframe[hatchcolor=gray,fillstyle=hlines,hatchangle=45,linestyle=none,hatchsep=2pt](2,-.1)(4,.1)%
\end{pspicture}
\end{center}
Esto significa que $x$ debe satisfacer las siguientes igualdades:
\begin{equation}
\label{eqLim031}
2 < x < 4,
\end{equation}
que son equivalentes a estas otras:
\[
-1 < x - 3 < 1,
\]
que, a su vez, son equivalentes a la siguiente desigualdad:
\begin{equation}
\label{eqLim032}
|x - 3| < 1.
\end{equation}

Lo que vamos a hacer a continuación es encontrar $M$ reconstruyendo $g(x)$ a partir de las
desigualdades~(\ref{eqLim031}). Para ello, sumemos el número $3$ a los miembros de estas
desigualdades. Obtendremos lo siguiente:
\begin{equation}
\label{eqLim034}
5 < x + 3 < 7.
\end{equation}
Esto significa que $x + 3 > 0$. Por lo tanto:
\[
x + 3 = |x + 3|,
\]
con lo que las desigualdades~(\ref{eqLim034}) se pueden reescribir así:
\begin{equation}
\label{eqLim035}
5 < |x + 3| < 7.
\end{equation}
Además, como $|x + 3| > 0$, existe el cociente
\[
\frac{1}{|x + 3|},
\]
y las desigualdades~(\ref{eqLim035}) son equivalentes a las siguientes:
\[
\frac{1}{5} > \frac{1}{|x + 3|} > \frac{1}{7}.
\]
Ahora, si multiplicamos por $3$ cada miembro de estas desigualdades, obtenemos que:
\[
\frac{3}{5} > \frac{3}{|x + 3|} > \frac{3}{7}.
\]
Es decir:
\[
|g(x)| = \frac{3}{|x + 3|} < \frac{3}{5},
\]
siempre que
\[
|x - 3| < 1.
\]
Acabamos de encontrar el número $M$, que es igual a $\frac{3}{5}$; y el número $\delta_1$, que es
igual a $1$.

Con este resultado, volvamos al producto~(\ref{eqLim036})\vpageref{eqLim036}. Ya podemos afirmar
que, si $x$ es tal que $|x - 3| < 1$, entonces debe satisfacer lo siguiente:
\[
\left|\frac{5x-3}{x+3} - 2\right| = \frac{3}{|x+3|}|x-3| < \frac{3}{5}|x - 3|,
\]
es decir: {\bfseries
\begin{equation}
\label{eqLim037}
\bm{\left|\frac{5x-3}{x+3} - 2\right| < \frac{3}{5}|x - 3| < \epsilon},
\end{equation}
siempre que
\[
\bm{|x - 3| < 1} \text{ \ \textbf{y} \ } \bm{0 < |x - 3| < \frac{5}{3}\epsilon}.
\]
}

La desigualdad~(\ref{eqLim037}) nos dice como elegir el $\delta$ buscado:
\[
\delta = \min\left\{1,\frac{5}{3}\epsilon\right\}.
\]

Esta elección de $\delta$ nos asegura que el valor absoluto de la diferencia entre
\[
\frac{5x - 3}{x + 3}
\]
y el número $2$ es menor que $\epsilon$. En efecto:
\begin{align*}
\left|\frac{5x - 3}{x + 3} - 2\right| &= \frac{3}{|x + 3|}|x - 3| \\
& < \frac{3}{5}|x - 3|, \quad\text{pues, al tener}\ |x - 3| < \delta \leq 1, \ \text{se tiene que }\
   \frac{3}{|x + 3|} < \frac{3}{5}, \\
& < \frac{3}{5}\delta, \quad\text{pues }\ |x - 3| < \delta, \\
& \leq \frac{3}{5}\times\frac{5}{3}\epsilon = \epsilon, \quad\text{pues }\ \delta \leq \frac{5}{3}\epsilon.
\end{align*}
Por lo tanto: {\bfseries la desigualdad
\[
\bm{\left|\frac{5x - 3}{x + 3} - 2\right| < \epsilon}
\]
se verifica si $\bm{x}$ satisface las desigualdades
\[
\bm{0 < |x - 3| < \delta = \min\left\{1,\frac{5}{3}\epsilon\right\}}.
\]}

Hemos demostrado, entonces, que el número $2$ es el límite de
\[
\frac{5x - 3}{x + 3}
\]
cuando $x$ se aproxima al número $3$.
\end{exemplo}

El procedimiento encontrado funcionó. Vamos a seguir aplicándolo en algunos ejemplos adicionales,
los mismos que nos servirán más adelante para obtener un método para calcular límites sin la
necesidad de recurrir todas las veces a la definición.

\begin{exemplo}[Solución]{Sea $a > 0$. Probemos que
\[
\limjc{\sqrt{x}}{x}{a} = \sqrt{a}.
\]
Es decir, demostremos que el número $\sqrt{a}$ puede ser aproximado por valores de $\sqrt{x}$
siempre que $x$ esté lo suficientemente cerca de $a$.} Sea $\epsilon > 0$. Buscamos $\delta > 0$
tal que
\[
|\sqrt{x} - \sqrt{a}| < \epsilon,
\]
siempre que $|x - a| < \delta$ y $x > 0$.

Para empezar, encontremos $g(x)$ tal que, para $x\neq a$ y $x > 0$,
\begin{equation}
\label{eqLim039}
|\sqrt{x} - \sqrt{a}| = |g(x)||x - a|.
\end{equation}

Para obtener el factor $|x - a|$ a partir de la diferencia
\[
|\sqrt{x} - \sqrt{a}|,
\]
utilicemos el factor conjugado de esta diferencia; es decir, el número:
\[
\sqrt{x} + \sqrt{a}.
\]
Como éste es estrictamente mayor que $0$, podemos proceder de la siguiente manera:
\begin{align*}
|\sqrt{x} - \sqrt{a}| &=
|\sqrt{x} - \sqrt{a}|\times\frac{|\sqrt{x} + \sqrt{a}|}{|\sqrt{x} + \sqrt{a}|} \\
&= \frac{|(\sqrt{x})^2 - (\sqrt{a})^2|}{|\sqrt{x} + \sqrt{a}|} \\
&= \frac{|x - a|}{\sqrt{x} + \sqrt{a}}.
\end{align*}
Por lo tanto:
\begin{equation}
\label{eqLim038}
|\sqrt{x} - \sqrt{a}| = \frac{1}{\sqrt{x} + \sqrt{a}}|x - a|.
\end{equation}
Si definimos
\[
g(x) = \frac{1}{\sqrt{x} + \sqrt{a}},
\]
ya tenemos la igualdad~(\ref{eqLim039}).

Ahora debemos acotar superiormente $g(x)$. Esto no es muy difícil, ya que, como $\sqrt{a} > 0$ y
$\sqrt{x} > 0$, entonces
\[
\sqrt{x} + \sqrt{a} > \sqrt{a},
\]
de donde se obtiene que, para todo $x > 0$, se verifica la desigualdad
\[
\frac{1}{\sqrt{x} + \sqrt{a}} < \frac{1}{\sqrt{a}}.
\]
Por lo tanto:
\[
|g(x)| < \frac{1}{\sqrt{a}}
\]
para todo $x > 0$.

Ya tenemos $M$:
\[
M = \frac{1}{\sqrt{a}}.
\]
Entonces:
\[
|\sqrt{x} - \sqrt{a}| < \frac{1}{\sqrt{a}}|x - a|,
\]
para todo $x > 0$. Por ello, si para un $\delta > 0$, se tuviera que
\[
|x - a| < \delta,
\]
entonces se cumpliría que:
\[
|\sqrt{x} - \sqrt{a}| < \frac{1}{\sqrt{a}}|x - a| < \frac{\delta}{\sqrt{a}}.
\]

De esta desigualdad se ve que podemos elegir $\delta$ de la siguiente manera:
\[
\frac{\delta}{\sqrt{a}} = \epsilon.
\]
Es decir:
\[
\delta = \epsilon\sqrt{a}.
\]

En efecto:
\begin{align*}
|\sqrt{x} - \sqrt{a}| &= \frac{1}{\sqrt{x} + \sqrt{a}}|x - a| \\
& < \frac{1}{\sqrt{a}}|x - a| \\
& < \frac{1}{\sqrt{a}}\delta \\
& = \frac{\epsilon\sqrt{a}}{\sqrt{a}} = \epsilon.
\end{align*}
Por lo tanto: {\bfseries la desigualdad
\[
\bm{|\sqrt{x} - \sqrt{a}| < \epsilon}
\]
se verifica si $\bm{x}$ satisface las desigualdades
\[
0 < \bm{|x - a| < \delta = \epsilon\sqrt{a}}.
\]
}
Hemos probado, entonces, que el número $\sqrt{a}$ es el límite de $\sqrt{x}$ cuando $x$ se
aproxima al número $a$.
\end{exemplo}

Observemos que, en este ejemplo, no hemos seguido, exactamente, el procedimiento desarrollado en
los anteriores. Esto se debe a que, para este caso, se cumple que
\begin{equation}
\label{eqLim043}
|f(x) - L| \leq M|x - a|
\end{equation}
para todos los elementos del dominio de la función $f$ (con $M = \frac{1}{\sqrt{a}}$). Eso
significa que la función $g$ del método es menor o igual que la constante $M$ en todo el dominio de
$f$. Esto significa, entonces, que no hay necesidad de buscar un $\delta_1$ para acotar $g$. A su
vez, esto permite que el número $\delta$ sea definido de la siguiente manera:
\[
\delta = \frac{\epsilon}{M}.
\]

Como se puede ver, esta versión del método para hallar $\delta$ dado el $\epsilon$ funcionará
siempre que se verifique la desigualdad~(\ref{eqLim043}).

\subsection{Dos observaciones a la definición de límite}
\subsubsection{Primera observación: delta depende de epsilon}
Tanto en la definición de límite como en los ejemplos desarrollados anteriormente, se puede
observar que el número $\delta$ depende del número $\epsilon$. Es decir, si el valor de $\epsilon$
cambia, también lo hace $\delta$. Por ejemplo, para probar que
\[
9 = \limjc{x^2}{x}{-3},
\]
encontramos que
\[
\delta = \min\left\{1,\frac{\epsilon}{7}\right\}.
\]
Así, si tomamos $\epsilon = 14$, entonces:
\[
\delta = \min\left\{1,\frac{14}{7}\right\} = \min\{1,2\} = 1.
\]

En cambio, si $\epsilon = \frac{7}{2}$, entonces:
\[
\delta = \min\left\{1,\frac{7}{2\times 7}\right\} = \min\{1,\frac{1}{2}\} = \frac{1}{2}.
\]

El significado del primer caso es que para que $9$ pueda ser aproximado por $x^2$ de modo que el
error de aproximación sea menor que $14$, es suficiente con elegir que $x$ esté a una distancia de
$-3$ menor que $1$. En el segundo caso, para lograr que $9$ sea aproximado por $x^2$ con un error
más pequeño que $\frac{7}{2}$ es suficiente con que $x$ esté a una distancia de $-3$ menor que
$\frac{1}{2}$.

Para recordar esta dependencia del número $\delta$ del número $\epsilon$, se suele escribir, en la
definición de límite, $\delta(\epsilon)$ en lugar de solo escribir $\delta$. En este libro, en
aquellas situaciones en las que tener en cuenta esta dependencia sea crítico, escribiremos delta
seguido de epsilon entre paréntesis.

Veamos un ejemplo más donde obtenemos $\delta$ dado un $\epsilon$, en el que se aprecia, una vez
más, cómo $\delta$ depende de $\epsilon$.

\begin{exemplo}[Solución]{%
Sea $f$ una función de $\mathbb{R}$ en $\mathbb{R}$ definida por:
\[
	f(x)=
\begin{cases}
x^2 + 1 & \text{si $x<1$,} \\
-2x^2 + 8x - 4 & \text{si $x>1$}.
\end{cases}
\]
Demuestre que $\displaystyle\lim_{x\to 1}f(x) = 2$.}%
El siguiente es un dibujo del gráfico de $f$ en el intervalo $[-1.5,3]$:
\begin{center}
\psset{xAxisLabel={},yAxisLabel={},plotpoints=1000}%
\def\f{x dup mul 1 add}
\def\g{2 neg x dup mul mul 8 x mul add 4 sub}

\begin{psgraph}[arrows=->](0,0)(-1.75,-0.5)(3.5,4.5){0.5\textwidth}{5cm}
  \uput[-90](3.5,0){$x$}%
  \uput[0](0,4.5){$y$}%

  \psline*[linecolor=lightgray]
    (0.9,0)(! 0.9 /x 0.9 def \f)(! 0 /x 0.9 def \f)(! 0 /x 1.075 def \g)%
    (! 1.075 /x 1.075 def \g)(1.075,0)%

  \psplot{-1.5}{1}{\f}%
  \psplot[arrows=o-]{1}{3}{\g}%

\end{psgraph}
\end{center}
Como se puede observar, $f(x)$ está tan cerca del número $2$ como se quiera si $x$ está lo
suficientemente cerca de $1$. A través de la definición de límite, vamos a demostrar que esta
conjetura es verdadera.

Sea $\epsilon > 0$. Debemos hallar $\delta > 0$ tal que
\begin{equation}
\label{eqLim048}
|f(x)-2|< \epsilon
\end{equation}
siempre que $x \neq 1$ y $|x-1| < \delta$.

Utilicemos el método descrito en esta sección para encontrar $\delta$ dado $\epsilon$. Lo primero
que tenemos que hacer es encontrar una función $g$ para expresar el valor absoluto de la diferencia
entre $f(x)$ y $2$, para $x\neq 1$, de la siguiente manera:
\begin{equation}
\label{eqLim049}
|f(x) - 2| = |g(x)||x - 1|.
\end{equation}
Para ello, debemos trabajar con el lado izquierdo de la desigualdad~(\ref{eqLim048}). Pero, como
$f$ está definida por dos fórmulas ---una para valores menores que $1$ y otra para valores mayores
que $1$---, vamos a dividir el análisis en dos casos: cuando $x < 1$ y cuando $x > 1$.

Antes de estudiar cada caso, como el límite que nos interesa es cuando $x$ tiende a $1$, nos
interesan únicamente valores cercanos a $1$. Por ello, en todo lo que sigue, suponemos que $x$ toma
valores en el intervalo de centro $1$ y radio $1$; es decir, suponemos que $x$ satisface la
desigualdad $|x-1| < 1$, que es equivalente a $0 < x < 2$.

Por lo tanto, los dos casos a analizar son: $0 < x < 1$ y $1 < x < 2$.

\paragraph{Caso 1: $0 < x < 1$.}
Analicemos el lado izquierdo de la desigualdad~(\ref{eqLim048}). Recordemos que para $x < 1$ se
tiene que $f(x) = x^2 + 1$. Por lo tanto:
\begin{align*}
	|f(x)-2| &= |(x^2+1)-2| = |(x^2 + 1) - 2| \\
  &= |x^2 - 1| = |x+1||x-1|.
\end{align*}
Como $x > 0$, entonces $x + 1 > 0$, de donde
\[
|f(x) - 2| = (x + 1)|x - 1|.
\]

Entonces, si definimos $g(x) = x + 1$, ya tenemos la forma~(\ref{eqLim049}).

Lo que ahora debemos hacer es encontrar una cota superior para $g(x)$. En este caso, esto es
sencillo, pues, como $x < 1$, entonces $g(x) = x + 1 < 2$. Por lo tanto, tenemos que
\[
	|f(x)-2| = g(x)|x - 1| < 2|x-1|.
\]

En resumen, si $0 < x < 1$, entonces
\begin{equation}
\label{eqLim050}
|f(x) - 2| < 2|x - 1|.
\end{equation}

Ahora bien, si el miembro de la derecha de la expresión anterior fuera menor que $\epsilon$, el de
la izquierda también lo sería; es decir, se verifica la siguiente implicación lógica:
\[
	2|x-1|<\epsilon \quad \Rightarrow \quad |f(x)-2|< 2|x - 1| < \epsilon,
\]
la misma que puede ser expresada de la siguiente manera:
\[
	|x-1| < \frac{\epsilon}{2} \quad \Rightarrow \quad |f(x)-2| < \epsilon.
\]
Vemos, entonces, que, si $0 < x < 1$, un buen candidato para $\delta$ es $\frac{\epsilon}{2}$.

Sin embargo, recordemos que supusimos que también se verifica la condición $|x - 1| < 1$. Por lo
tanto, el número $1$ también es un candidato para $\delta$. ?`Cuál de los dos debemos elegir? El más
pequeño.

Así, para este primer caso, la elección para $\delta$ es la siguiente:
\begin{equation}
\label{eqLim051}
\delta = \min\left\{1,\frac{\epsilon}{2}\right\}.
\end{equation}

\paragraph{Caso 2: $1 < x < 2$.}
Puesto que, para estos valores de $x$, se tiene que $f(x) = -2x^2 + 8x - 4$, el lado izquierdo de
la desigualdad~(\ref{eqLim048}) puede ser expresado de la siguiente manera:
\begin{align*}
|f(x) - 2| &= |-2x^2 + 8x - 4 - 2| \\
  &= |2x^2 - 8x + 6| \\
  &= 2|x^2 - 4x + 3| = 2|x - 3||x - 1|.
\end{align*}

Si definimos $g(x) = 2|x - 3|$, tenemos que:
\[
|f(x) - 2| < g(x)|x - 1|.
\]

Ahora encontremos una cota superior para $g(x)$. Para ello, recordemos que $1 < x < 2$. De estas
desigualdades, tenemos que:
\[
1 - 3 < x - 3 < 2 - 3.
\]
Es decir, se verifica que
\[
-2 < x - 3 < - 1 < 2.
\]
Por lo tanto:
\[
|x - 3| < 2 \yjc g(x) = 2|x - 3| < 4.
\]

Entonces, para $x$ tal que $1 < x < 2$, se verifica la desigualdad:
\[
|f(x) - 2| < 4|x - 1|.
\]

De esta desigualdad tenemos la siguiente implicación:
\[
	4|x-1|<\epsilon \quad \Rightarrow \quad |f(x)-2|< 4|x - 1| < \epsilon,
\]
la misma que puede ser expresada de la siguiente manera:
\[
	|x-1| < \frac{\epsilon}{4} \quad \Rightarrow \quad |f(x)-2| < \epsilon.
\]

Con un razonamiento similar al caso anterior, la elección de $\delta$ es la siguiente:
\begin{equation}
\label{eqLim052}
\delta = \min\left\{1,\frac{\epsilon}{4}\right\}.
\end{equation}

\paragraph{Conclusión.}
De los dos casos estudiados, podemos concluir que, si $0 < x < 2$, el $\delta$ buscado debe ser
elegido de la siguiente manera:
\[
\delta = \min\left\{1,\frac{\epsilon}{2}, \frac{\epsilon}{4}\right\}.
\]
Pero, como
\[
\frac{\epsilon}{4} < \frac{\epsilon}{2},
\]
si
\[
\delta = \min\left\{1,\frac{\epsilon}{4}\right\},
\]
se verifica la siguiente implicación:
\[
	0 < |x-1| < \delta \quad \Rightarrow \quad |f(x)-2| < \epsilon.
\]
Esto prueba que
\[
2 = \limjc{f(x)}{x}{1}.
\]
\end{exemplo}

\subsubsection{Segunda observación: no importa qué valor tome la función en el punto donde se calcula el límite} En
efecto, la definición de
\[
L = \limjc{f(x)}{x}{a}
\]
no exige que el número $a$ esté en el dominio de la función $f$. De hecho, en los ejemplos
anteriores, se puede ver que los $x$ que se utilizan para aproximar $L$ con $f(x)$ siempre son
distintos de $a$, pues estos $x$ satisfacen la desigualdad:
\[
0 < |x - a|.
\]

Los siguientes ejemplos nos muestran porqué no es necesario tomar en cuenta al número $a$ en la
definición de límite.

\paragraph{1.}Sea $\funcjc{f}{\mathbb{R}}{\mathbb{R}}$ la función definida por
\[
f(x) = 2x + 1.
\]
Entonces:
\[
3 = \limjc{f(x)}{x}{1}.
\]
En efecto: sea $\epsilon > 0$. Debemos encontrar un número $\delta > 0$ tal que se verifique la
igualdad
\begin{equation}
\label{eqLim042}
|f(x) - 3| < \epsilon,
\end{equation}
siempre que
\[
0 < |x - 1| < \delta.
\]

Para hallar el número $\delta$, ya sabemos qué hacer. En primer lugar, tenemos que:
\begin{align*}
|f(x) - 3| &= |(2x + 1) - 3| \\
&= |2x - 2| \\
&= 2|x - 1|.
\end{align*}
Es decir, se verifica que:
\begin{equation*}
|f(x) - 3| < 2|x-1|.
\end{equation*}
Por lo tanto, si existiera el número $\delta$, debería satisfacerse la desigualdad:
\begin{equation*}
|f(x) - 3| < 2|x-1| < 2\delta.
\end{equation*}
De esta última desigualdad, se ve que, para que se verifique la desigualdad~(\ref{eqLim042}), basta
elegir el número $\delta$ tal que
\[
2\delta = \epsilon,
\]
es decir, tal que
\[
\delta = \frac{\epsilon}{2}.
\]

En efecto: si $x$ es tal que
\[
0 < |x - 1| < \delta = \frac{\epsilon}{2},
\]
entonces:
\begin{align*}
|f(x) - 3| &= 2|x - 1| \\
&< 2\delta \\
&= 2\frac{\epsilon}{2} = \epsilon.
\end{align*}

\paragraph{2.} Sea $\funcjc{g}{\mathbb{R}}{\mathbb{R}}$ la función definida por
\[
g(x) =
\begin{cases}
2x + 1 & \text{si } x\neq 1, \\
1 & \text{si } x = 1.
\end{cases}
\]
Entonces:
\[
g(x) = f(x)
\]
para todo $x\neq 1$. Es decir, $g$ y $f$ son casi la misma función, excepto por el valor que cada
una de ellas toma en el número $x = 1$, pues
\[
g(1) = 1\yjc f(1) = 3.
\]
Sin embargo:
\[
3 = \limjc{g(x)}{x}{1}.
\]

En efecto: sea $\epsilon > 0$. Debemos hallar un número $\delta > 0$ tal que
\begin{equation}
\label{eqLim044}
|g(x) - 3| < \epsilon,
\end{equation}
siempre que
\[
0 < |x - 1| < \delta.
\]

Ahora bien, encontrado el número $\delta$, lo que tenemos que probar es que si
\[
0 < |x - 1| < \delta,
\]
se verifica la desigualdad~(\ref{eqLim044}). Es decir, debemos probar que para $x \neq 1$, pues $|x
- 1| > 0$, tal que $|x - 1| < \delta$, se verifica la desigualdad~(\ref{eqLim044}). Pero,
recordemos que
\[
g(x) = f(x)
\]
para todo $x\neq 1$. Entonces, para estos $x$, la desigualdad~(\ref{eqLim044}) se transforma en la
desigualdad:
\[
\tag{\ref{eqLim042}}
|f(x) - 3| < \epsilon.
\]
Y ya probamos, en el ejemplo anterior, que esta desigualdad se cumple siempre que
\[
0 < |x - 1| < \delta = \frac{\epsilon}{2}.
\]
De manera que para todo $x$ que cumpla con estas dos desigualdades se cumple la
desigualdad~(\ref{eqLim044}). Y esto significa que el número $3$ es el límite de $g(x)$ cuando $x$
se aproxima al número $1$.

\paragraph{3.} Sea $\funcjc{h}{\mathbb{R} - \{1\}}{\mathbb{R}}$ la función definida por
\[
h(x) = \frac{2x^2 - x - 1}{x - 1}.
\]
En este caso, $h$ no está definida en $1$ por lo que es distinta tanto de $f$ como de $g$. Sin
embargo, para todo $x\neq 1$, se verifican las igualdades:
\[
h(x) = \frac{2x^2 - x - 1}{x-1} = \frac{(2x+1)(x-1)}{x-1} = 2x + 1 = g(x) = f(x).
\]
De manera análoga al caso de $g$, podemos demostrar que
\[
3 = \limjc{h(x)}{x}{1}.
\]

En estos tres ejemplos podemos ver que, en lo que se refiere al límite de una función en un punto
$a$, \emph{no importa el valor que pueda tomar la función en el punto $a$}; incluso, la función
puede no estar definida en este punto. \emph{Lo que importa realmente es el comportamiento de la
función alrededor del punto $a$}. Las gráficas de las funciones $f$, $g$ y $h$, que están a
continuación, ilustran esta última afirmación:
\begin{center}
\begin{pspicture}(-1,-1)(7,3)
\psset{xunit=.8,yunit=.5}%
\psaxes[ticks=none,labels=none]{->}(0,0)(-1,-1)(3,6)%
\uput[-90](3,0){$x$}%
\uput[180](0,6){$f(x)$}%
\psplot{-1}{2}{2 x mul 1 add}%
\psline[linecolor=gray,linestyle=dashed](1,0)(1,3)(0,3)%
\rput(1,-.5){$1$}%
\rput(-.25,3){$3$}%

\rput[l](4,4){$\funcionjc{f}{\mathbb{R}}{\mathbb{R}}{x}{f(x) = 2x + 1}$}%
\rput[l](4,1.5){$\displaystyle{\limjc{f(x)}{x}{1} = 3 = f(1)}$}
\end{pspicture}
\end{center}
%
\begin{center}
\begin{pspicture}(-1,-1)(7,3)
\psset{xunit=.8,yunit=.5}%
\psaxes[ticks=none,labels=none]{->}(0,0)(-1,-1)(3,6)%
\uput[-90](3,0){$x$}%
\uput[180](0,6){$g(x)$}%
\psplot{-1}{.95}{2 x mul 1 add}%
\psplot{1.05}{2}{2 x mul 1 add}%
\pscircle(1,3){.05}%
\pscircle[fillstyle=solid,fillcolor=black](1,1){.05}%
\psline[linecolor=gray,linestyle=dashed](1,0)(1,1)(0,1)%
\rput(1,-.5){$1$}%
\rput(-.25,3){$3$}%
\rput(-.25,1){$1$}%
\rput[l](4,4){$\funcionjc{g}{\mathbb{R}}{\mathbb{R}}{x}{g(x) = %
\begin{cases}
2x + 1 & \text{si } x \neq 1 \\
1 & \text{si } x = 1
\end{cases}}$}%
\rput[l](4,1){$\displaystyle{\limjc{g(x)}{x}{1} = 3 \neq g(1)}$}
\end{pspicture}
\end{center}
%
\begin{center}
\begin{pspicture}(-1,-1)(7,3)
\psset{xunit=.8,yunit=.5}%
\psaxes[ticks=none,labels=none]{->}(0,0)(-1,-1)(3,6)%
\uput[-90](3,0){$x$}%
\uput[180](0,6){$h(x)$}%
\psplot{-1}{.95}{2 x mul 1 add}%
\psplot{1.05}{2}{2 x mul 1 add}%
\pscircle(1,3){.05}%
\rput(1,-.5){$1$}%
\rput(-.25,3){$3$}%
\rput[l](4,4){$\funcionjc{h}{\mathbb{R}-\{1\}}{\mathbb{R}}{x}{h(x) = 2x + 1}$}%
\rput[l](4,2){$\displaystyle{\limjc{h(x)}{x}{1} = 3}$}%
\rput[l](4,.75){$h(1)$ no existe}%
\end{pspicture}
\end{center}

Podemos resumir la situación que se ilustra en estos ejemplos, diciendo que si dos funciones
solo difieren en un punto de su dominio, o bien las dos tienen límite en dicho punto, y es el mismo
límite, o bien ninguna tiene límite. De manera más precisa se expresa este resultado en el teorema del límite de funciones localmente iguales, que presentaremos a continuación.

\subsection{Límite de funciones localmente iguales}

\begin{defical}[Funciones localmente iguales]
Sean: 
\begin{itemize}
      \item[] $a$ un número real;
      \item[] $I$ un intervalo abierto que contiene al número $a$; y,
      \item[] $f$ y $g$ dos funciones reales definidas en $I$, salvo talvez en $a$ (es decir, $I\subset \Dm(f)\cup \{a\}$; e $I\subset \Dm(g)\cup \{a\}$).
\end{itemize}
Diremos que ``$f=g$ localmente cerca de $a$'' o simplemente que ``$f=g$ cerca de $a$'', si existe $r>0$ tal que para todo $x\in ]a-r, a+r[ \setminus \{a\}$, $f(x)=g(x)$.
\end{defical}

En la definición, no importa cuán pequeño sea el valor de $r$. Esto justifica la expresión ``cerca de a''. En vez de ``$f(x)=g(x)$'' puede tomarse cualquier otra propiedad. Por ejemplo, diremos que ``$f > g$ cerca de $a$'' si en vez de ``$f(x)=g(x)$'' se exige que ``$f(x) > g(x)$''.

\begin{exemplo}[ ]{%
\[
	F(x)=
\begin{cases}
\displaystyle\frac{2x^2-x-1}{x-1} & \text{si $x<2$,} \\
x^2+1 & \text{si $x\geq 2$}.
\end{cases}
\]

\[
	G(x)=
\begin{cases}
1-x^2 & \text{si $x<0$,} \\
2x+1 & \text{si $x\geq 0$}.
\end{cases}
\]
}%
Claramente $F=G$ cerca de $1$. En efecto, si tomamos $r\in ]0,1[$, se puede ver que para todo $x\in ]1-r, 1+r[\setminus \{1\}$, $F(x)=G(x)$.

Podemos ahora enunciar el siguiente teorema.
\end{exemplo}


% \begin{teocal}[Límite de funciones localmente iguales]\label{eq:limitegeneral}%
% Sean $I$ y $J$ dos intervalos abiertos y $\funcjc{f}{I}{\mathbb{R}}$ y $\funcjc{g}{J}{\mathbb{R}}$
% tales que $a \in I \cap J$ y:
% \begin{enumerate}
% \item $f(x) = g(x)$ para todo $x \in I \cap J$ y $x \neq a$; y
% \item existe L tal que:
%    \[
%       L = \limjc{g(x)}{x}{a}.
%    \]
% \end{enumerate}
% Entonces $f$ también tiene límite en a y:
% \[
%    L = \limjc{f(x)}{x}{a}.
% \]
% \end{teocal}%Fin del teo
% 
% Este es el caso de las funciones $f$, $g$ y $h$. Se diferencian únicamente en $a = 1$. Como el
% límite de $f$ existe y es igual a $3$, entonces los límites de $g$ y $h$ existen también y son
% iguales a $3$.

\begin{teocal}[Límite de funciones localmente iguales]\label{eq:limitegeneral}%
Sean:
\begin{enumerate}
\item[] $a$ un número real;
\item[] $I$ un intervalo abierto que contine al número $a$;
\item[] $f$ y $g$ dos funciones definidas en $I$, salvo tal vez en $a$.
\end{enumerate}
Si $f=g$ cerca de $a$, entonces:
\begin{enumerate}
      \item Existe $\displaystyle\limjc{f(x)}{x}{a}$ si y solo si existe $\displaystyle\limjc{g(x)}{x}{a}$.
      \item Si los límites existen, son iguales.
\end{enumerate}
\end{teocal}%Fin del teo

En el ejemplo de las funciones $f$, $g$ y $h$ definidas anteriormente, vemos que $f=g=h$ cerca de $1$, y como existe $\displaystyle\limjc{f(x)}{x}{1}=3$, entonces también existen $\displaystyle\limjc{g(x)}{x}{1}$ y $\displaystyle\limjc{h(x)}{x}{1}$ y son iguales a $3$.

Este teorema es muy útil para el cálculo de límites y se lo utiliza de la siguiente manera.

Supongamos que queremos calcular $\displaystyle\limjc{f(x)}{x}{a}$. Se busca una función $g$ cuyo límite en $a$ se conozca y tal que $f=g$ cerca de $a$.

El teorema nos permite afirmar entonces que
\[
      \limjc{f(x)}{x}{a} = \limjc{g(x)}{x}{a}.
\]

% Este teorema se utiliza de la siguiente manera. Supongamos que queremos calcular
% \[
%    \limjc{f(x)}{x}{a}.
% \]
% Se busca una función $g$ cuyo límite en $a$ se conozca y tal que
% \[
% g(x) = f(x)
% \]
% para todo $x$ en la intersección de los dominios de $f$ y $g$ y que sea diferente de $a$. Entonces,
% lo que podemos afirmar es que
% \[
%    \limjc{f(x)}{x}{a} = \limjc{g(x)}{x}{a}.
% \]

\begin{exemplo}[Solución]{%
Supongamos conocido que
\[
   \limjc{(x + 2)}{x}{1} = 3.
\]
Calcular
\[
   \limjc{\frac{x^2 + x - 2}{x - 1}}{x}{1} .
\]
}%
Sea $\funcjc{f}{\mathbb{R} - {1}}{\mathbb{R}}$ tal que
\[
f(x) = \frac{x^2 + x - 2}{x - 1}.
\]
Debemos hallar una función $g$ que sea igual a $f$, excepto en $1$, y cuyo límite conozcamos.

Esto se puede hacer, pues
\[
   f(x) = \frac{x^2 + x - 2}{x - 1} = \frac{(x - 1)(x + 2)}{x - 1} = x + 2,
\]
para todo $x \neq 1$. Por lo tanto, si se define
\[
   g(x) = x + 2,
\]
sabemos que:
\begin{enumerate}
\item $f(x) = g(x)$ para todo $x \neq 1$; por lo que $f=g$ cerca de $1$; y,
\item $\displaystyle\limjc{g(x)}{x}{1} = 3.$
\end{enumerate}
Por lo tanto, gracias al teorema (\ref{eq:limitegeneral}), podemos afirmar que:
$\displaystyle
   \limjc{\frac{x^2 + x - 2}{x - 1}}{x}{1} = \limjc{g(x)}{x}{1} = 3$.
\end{exemplo}

Para terminar esta sección, presentamos el siguiente teorema, que es una consecuencia inmediata de
la definición de límite, que es muy útil en diferentes aplicaciones del concepto de límite.
%-----> 2008 09 12
%\textcolor{red}{[?`qué? También ameritaría uno o dos ejemplos que muestren cómo se utiliza este
%teorema. Además, un par de ejercicios sobre este teorema para el final de la sección.]}

\begin{teocal}[Caracterizaciones del límite]\label{teol:LEquiv0} Sea $\funcjc{f}{\Dm(f)}{\mathbb{R}}.$ Entonces:
\[
   L = \limjc{f(x)}{x}{a} \Leftrightarrow 0 = \limjc{(f(x) - L)}{x}{a} \Leftrightarrow
   0 = \limjc{|f(x) - L|}{x}{a}.
\]
\end{teocal}%

La demostración es un buen ejercicio para trabajar la definición de límite por lo que se sugiere al
lector la haga por sí mismo.

Un uso típico de este teorema es el siguiente. Queremos demostrar que
\[
3 = \limjc{\frac{x + 3}{x + 1}}{x}{0}.
\]
En lugar de ello, probaremos que
\[
0 = \limjc{\frac{x + 3}{x + 1} - 3}{x}{0}.
\]
Pero, como
\[
\frac{x + 3}{x + 1} - 3 = -\frac{2x}{x + 1},
\]
lo que hay que probar es
\[
0 = \limjc{-2\frac{x}{x + 1}}{x}{0}.
\]
Y, como
\[
\left\lvert-2\frac{x}{x + 1}\right\rvert = 2\left\lvert\frac{x}{x + 1}\right\rvert,
\]
una alternativa es probar que
\[
0 = 2\limjc{\frac{x}{x + 1}}{x}{0}.
\]

En sentido estricto, no hay mayor diferencia en la manera cómo se demuestra cualquiera de estas
igualdades, salvo, en ciertas ocasiones, en las que algunas operaciones algebraicas suelen
simplificarse. El lector debería realizar cada una de estas prueba para que compare y determine
cuáles podrían ser esas simplificaciones.

\subsection{Ejercicios}
\begingroup
\small
\begin{multicols}{2}
\begin{enumerate}[leftmargin=*]
\item Demuestre, usando la definición de límite, que:
\begin{enumerate}[leftmargin=*]
\item $\displaystyle
	\lim_{x\to 8}(\sqrt[3]{x^2} + 2\sqrt[3]{x} + 4) = 12 $

\item $\displaystyle
	\lim_{x\to 3}(8x-15)=9
$
\item $\displaystyle
	\lim_{x\to -2}(5x+14)=4
$

\item $\displaystyle
	\lim_{x\to 9}\dfrac{1}{\sqrt{x} + 3} = \frac{1}{6}
$

\item $\displaystyle
	\lim_{x\to -2}\dfrac{2x^2-8}{x+2}=-8
$
\item $\displaystyle
	\lim_{x\to 1}\dfrac{2x + 3}{x - 2} = -5 $
\item $\displaystyle
	\lim_{x\to 1}(x^2+x+1)=3
$
\item $\displaystyle \lim_{x\to \frac{1}{3}}(-9x^2 + 3x + 1) = 1$

\item $\displaystyle
	\lim_{x\to -1}\dfrac{x + 4}{x - 1} = -\frac{3}{2}$
\end{enumerate}
\item Use el teorema \ref{eq:limitegeneral} para hallar los siguientes límites:
\begin{enumerate}[leftmargin=*]
\item $\displaystyle
	\lim_{x\to 1}\dfrac{2x^2+x-3}{x^2-3x+2}$

\item $\displaystyle
	\lim_{x\to 9}\dfrac{\sqrt{x}-3}{x-9}$

\item $\displaystyle
	\lim_{x\to -1}\dfrac{x^2+5x+4}{x^2-1}$

\item $\displaystyle
	\lim_{x\to 8}\dfrac{x-8}{\sqrt[3]{x}-2}$

\end{enumerate}
\end{enumerate}
\end{multicols}
\endgroup

\section{Continuidad de una función} Las tres funciones utilizadas en la sección precedente tienen
límite en el número $1$. La primera y la segunda también están definidas en $1$; es decir, $1$ está
en el dominio de $f$ y $g$. Sin embargo, en el caso de la primera, como $f(1) = 3$, el valor de $f$
en $1$ es igual al límite; en el caso de la segunda, como $g(1) = 1 \neq 3$, esto no ocurre. Para
la tercera función, el número $1$ no está en el dominio de $h$.

Los dibujos de estas tres funciones muestran que en el gráfico de las funciones $g$ y $h$ hay un
``salto'' al cruzar la recta vertical de ecuación $x = 1$ (en la jerga matemática, se suele decir
que ``hay un salto al pasar por $1$''). Si dibujáramos los gráficos de estas funciones, trazándolo
de izquierda a derecha, en el caso de la segunda y de la tercera, deberíamos ``interrumpir'' o
``discontinuar'' el trazo. Para $f$ eso no ocurrirá. Por esa razón, la función $f$ va a ser una
función ``continua'', mientras que las otras dos no.

Lo que diferencia a $f$ de $g$ y $h$ es el hecho de que, a más de existir el límite en $1$, la
función está definida allí y su valor es igual al límite. Ésta es la definición de continuidad:

\begin{defical}[Función continua]
Una función $\funcjc{f}{\Dm(f)}{\mathbb{R}}$, donde $I\subseteq \Dm(f)$ es un intervalo abierto, es
\emph{continua} en $a$ si y solo si:
\begin{enumerate}
\item $a\in I$;
\item existe $\displaystyle\limjc{f(x)}{x}{a}$; y
\item $\displaystyle f(a) = \limjc{f(x)}{x}{a}$.
\end{enumerate}
Una función es continua en el intervalo abierto $I$ si es continua en todos y cada uno de los
elementos de $I$.
\end{defical}

La continuidad de una función es un tema central en el estudio del Cálculo. A lo largo de este
libro, conoceremos diversas propiedades de las funciones continuas. Por ahora, veamos que la
definición de límite ofrece una definición equivalente de continuidad:

\begin{teocal}[Función continua]%
Sea $\funcjc{f}{\Dm(f)}{\mathbb{R}}$, donde $I\subseteq\Dm(f)$ es un intervalo abierto. Sea $a\in
I$. Entonces, $f$ es continua en $a$ si y solo si para todo $\epsilon > 0$ existe un número $\delta
> 0$ tal que
\[
|f(x) - f(a)| < \epsilon,
\]
siempre que $|x - a| < \delta$ y $x \in I$.
\end{teocal}

Observemos que no hace falta excluir el caso $x = a$, como se hace en la definición de límite,
porque, al ser $f$ continua en $a$, si $x = a$, se verifica que:
\[
|f(x) - f(a)| = |f(a) - f(a)| = 0 < \epsilon.
\]

Veamos un par de ejemplos sencillos de funciones continuas.

\begin{exemplo}[Solución]{\label{ex:lim002}
La función constante es continua en su dominio.}%
Sean $c\in\mathbb{R}$ y $\funcjc{f}{\mathbb{R}}{\mathbb{R}}$ tal que
\[
f(x) = c
\]
para todo $x\in \mathbb{R}$. Probemos que la función $f$ es continua en todo $a\in\mathbb{R}$.

Para ello, sea $a\in\mathbb{R}$. Puesto que $a\in\Dm(f)$, solo nos falta verificar que:
\begin{enumerate}
\item existe $\displaystyle\limjc{f(x)}{x}{a}$; y que
\item $\displaystyle f(a) = c = \limjc{f(x)}{x}{a}.$
\end{enumerate}
Probaremos la segunda condición únicamente, ya que con ello es suficiente para probar la primera.

Sea $\epsilon > 0$. Debemos encontrar un número $\delta > 0$ tal que
\begin{equation}
\label{eq:lim001}
|f(x) - c| < \epsilon
\end{equation}
siempre que
\[
|x - a| < \delta.
\]

Ahora bien, puesto que
\[
|f(x) - c| = |c - c| = 0 < \epsilon
\]
para todo $x\in\mathbb{R}$, la desigualdad~(\ref{eq:lim001}) será verdadera cualquiera que sea el
$\delta > 0$ elegido. Es decir, el número $c$ es el límite de $f(x)$ cuando $x$ se aproxima al
número $a$.

Por lo tanto, este límite existe, lo que prueba que la función constante es continua en $a$ y, como
$a$ es cualquier elemento del dominio de $f$, la función es continua en su dominio.
\end{exemplo}

Este ejemplo nos muestra, además, la veracidad de la siguiente igualdad:
\begin{equation}
\label{eq:LimConstante}
c = \limjc{c}{x}{a}.
\end{equation}
Esta igualdad suele enunciarse de la siguiente manera:
\begin{quote}
\textbf{el límite de una constante es igual a la constante.}
\end{quote}

\begin{exemplo}[Solución]{\label{ex:lim003}%
La función identidad es continua en todo su dominio}%
Sea $\funcjc{f}{\mathbb{R}}{\mathbb{R}}$ tal que
\[
f(x) = x
\]
para todo $x\in \mathbb{R}$. Probemos que la función $f$ es continua en todo $a\in\mathbb{R}$.

Para ello, sea $a\in\mathbb{R}$. Puesto que $a\in\Dm(f)$, solo nos falta verificar que:
\begin{enumerate}
\item existe $\displaystyle\limjc{f(x)}{x}{a}$; y que
\item $\displaystyle f(a) = a = \limjc{f(x)}{x}{a}.$
\end{enumerate}
Probaremos la segunda condición únicamente, ya que con ello es suficiente para probar la primera.

Sea $\epsilon > 0$. Debemos encontrar un número $\delta > 0$ tal que
\begin{equation}
\label{eq:lim002}
|f(x) - a| < \epsilon
\end{equation}
siempre que
\[
|x - a| < \delta.
\]
Como
\[
|f(x) - a| = |x - a|,
\]
es obvio que el número $\delta$ buscado es igual a $\epsilon$. Si lo elegimos así, habremos probado
que $a$ es el límite de $f(x)$ cuando $x$ se aproxima al número $a$. Por lo tanto, éste límite
existe, lo que prueba que la función identidad es continua en $a$, de donde, es continua en su
dominio.
\end{exemplo}

Este ejemplo nos demuestra que la siguiente igualdad es verdadera:
\begin{equation}
\label{eq:LimIdentidad}
a = \limjc{x}{x}{a}.
\end{equation}
Esta igualdad se enuncia de la siguiente manera:
\begin{quote}
\textbf{el límite de $x$ es igual al número $a$ cuando $x$ se aproxima al número $a$.}
\end{quote}

En una sección posterior, desarrollaremos algunos procedimientos para el cálculo de límites lo que,
a su vez, nos permitirá estudiar la continuidad de una función.

\section{Interpretación geométrica de la definición de límite}
Como hemos podido ver, encontrar el número $\delta$ dado el número $\epsilon$ no siempre es una
tarea fácil. Los ejemplos que hemos trabajado son sencillos relativamente. En general, demostrar
que un cierto número es el límite de una función suele ser una tarea de considerable trabajo. Por
ello, el poder ``visualizar'' la definición es de mucha ayuda para poder manipular luego las
desigualdades con los $\epsilon$ y $\delta$. Esta visualización puede ser realizada de la siguiente
manera.

En primer lugar, recordemos que la desigualdad
\begin{equation}
\label{eqLim045}
|x - x_0| < r,
\end{equation}
donde $x$, $x_0$ y $r$ son números reales y, además, $r > 0$, es equivalente a la desigualdad
\[
x_0 - r < x < x_0 + r.
\]
Por lo tanto, el conjunto de todos los $x$ que satisfacen la desigualdad~(\ref{eqLim045}) puede ser
representado geométricamente por el intervalo $]x_0 - r, x_0 + r[$; es decir, por el intervalo
abierto con centro en $x_0$ y radio $r$:
\begin{center}
\begin{pspicture}(-2,-.5)(2,.5)
\psline{<->}(-2,0)(2,0)%
\psline(-1.5,.15)(-1.5,-.15)%
\psline(0,.15)(0,-.15)%
\psline(1.5,.15)(1.5,-.15)%
\rput(-1.5,-.35){$x_0 - r$}%
\rput(0,-.35){$x_0$}%
\rput(1.5,-.35){$x_0 + r$}%
\psframe[hatchcolor=gray,fillstyle=hlines,hatchangle=45,linestyle=none,hatchsep=2pt]%
      (-1.5,-.1)(1.5,.1)%
\end{pspicture}
\end{center}
La longitud de este intervalo es igual a $2r$.

Supongamos que el número $L$ es el límite de $f(x)$ cuando $x$ se aproxima al número $a$. Esto
significa que, dado cualquier número $\epsilon > 0$, existe un número $\delta > 0$ tal que se
verifica la desigualdad
\begin{equation}
\label{eqLim046}
|f(x) - L| < \epsilon,
\end{equation}
siempre que se satisfagan las desigualdades
\begin{equation}
\label{eqLim047}
0 < |x - a| < \delta.
\end{equation}

Con la interpretación geométrica realizada previamente, esta definición de límite puede expresarse
en términos geométricos de la siguiente manera:
\begin{quote}
{\bfseries dado cualquier número $\bm{\epsilon > 0}$, existe un número $\bm{\delta > 0}$ tal que
$\bm{f(x)}$ se encuentre en el intervalo de centro $\bm{L}$ y radio $\bm{\epsilon}$, siempre que
$\bm{x}$ se encuentre en el intervalo de centro $\bm{a}$ y radio $\bm{\delta}$ y $x\neq a$.}
\end{quote}

Esta formulación puede ser visualizada de la siguiente manera. En un sistema de coordenadas, en el
que se va a representar gráficamente la función $f$, dibujemos los dos intervalos que aparecen en
la definición de límite: $]a - \delta, a + \delta[$ y $]L - \epsilon, L + \epsilon[$. Obtendremos
lo siguiente:
\begin{center}
\psset{unit=0.9}
\begin{pspicture}(-.5,-.5)(5,4.5)
\psaxes[ticks=none,labels=none]{->}(0,0)(-.5,-.5)(5,4)%
\uput[-90](5,0){$x$}%
\uput[180](0,4){$f(x)$}%
\psset{xunit=.35mm,yunit=.35mm,plotpoints=200}%

\psframe[hatchcolor=gray,fillstyle=hlines,hatchangle=45,linestyle=none,hatchsep=2pt]%
      (67.5,-1.5)(97.5,1.5)%
\psline(67.5,-1)(67.5,1)% a - \delta
\rput[Br](67.5,-10){$a - \delta$}%
\psline(82.5,-1)(82.5,1)% a
\rput[B](82.5,-10){$a$}%
\psline(97.5,-1)(97.5,1)% a + \delta
\rput[Bl](97.5,-10){$a + \delta$}%

\psframe[hatchcolor=gray,fillstyle=hlines,hatchangle=45,linestyle=none,hatchsep=2pt]%
      (-1.5,52.36)(1.5,86.67)%
\psline(-1,52.36)(1,52.36)% L - \epsilon
\rput[r](-3,52.36){$L - \epsilon$}%
\psline(-1,71.79)(1,71.79)% L
\rput[r](-3,71.79){$L$}%
\psline(-1,86.67)(1,86.67)% L + \epsilon
\rput[r](-3,86.67){$L + \epsilon$}%

\end{pspicture}
\end{center}
A continuación, dibujemos dos bandas: una horizontal, limitada por las rectas horizontales cuyas
ecuaciones son $y = L - \epsilon$ y $y = L + \epsilon$, y una vertical, limitada por las rectas
verticales cuyas ecuaciones son $x = a - \delta$ y $x = a + \delta$:
\begin{center}
\psset{unit=0.9}
\begin{pspicture}(-.5,-.5)(5,4.5)
\psaxes[ticks=none,,labels=none]{->}(0,0)(-.5,-.5)(5,4)%
\uput[-90](5,0){$x$}%
\uput[180](0,4){$f(x)$}%
\psset{xunit=.35mm,yunit=.35mm,plotpoints=200}%

\psline(67.5,-1)(67.5,1)% a - \delta
\rput[Br](67.5,-10){$a - \delta$}%
\psline(82.5,-1)(82.5,1)% a
\rput[B](82.5,-10){$a$}%
\psline(97.5,-1)(97.5,1)% a + \delta
\rput[Bl](97.5,-10){$a + \delta$}%

\psline(-1,52.36)(1,52.36)% L - \epsilon
\rput[r](-3,52.36){$L - \epsilon$}%
\psline(-1,71.79)(1,71.79)% L
\rput[r](-3,71.79){$L$}%
\psline(-1,86.67)(1,86.67)% L + \epsilon
\rput[r](-3,86.67){$L + \epsilon$}%

\psset{linestyle=dashed}
\psframe[linestyle=none,fillstyle=solid,fillcolor=lightgray](1,52.36)(67.5,86.67)%
\psframe[linestyle=none,fillstyle=solid,fillcolor=lightgray](97.5,52.36)(120,86.67)%
\psline(0,52.36)(120,52.36)%
\psline(0,86.67)(120,86.67)%

\psframe[linestyle=none,fillstyle=solid,fillcolor=lightgray](67.5,1)(97.5,52.36)%
\psframe[linestyle=none,fillstyle=solid,fillcolor=lightgray](67.5,86.67)(97.5,110)%
\psline(67.5,0)(67.5,100)%
\psline(97.5,0)(97.5,100)%

\end{pspicture}
\end{center}

El rectángulo obtenido por la intersección de las dos bandas está representado por el siguiente
conjunto:
\[
C = \{(x,y) \in \mathbb{R}^2 : x \in\ ]a-\delta, a + \delta[, \ \ y\in\ ]L-\epsilon,L+\epsilon[\} = \ ]a-\delta, a+\delta [\times ]L-\epsilon, L+\epsilon [.
\]

La definición de límite afirma que $f(x)$ estará en el intervalo $]L-\epsilon, L+\epsilon[$ siempre
que $x$, siendo distinto de $a$, esté en el intervalo $]a-\delta, a+ \delta[$. Esto significa,
entonces, que la pareja $(x,f(x))$ está en el conjunto $C$. Es decir, todos los puntos de
coordenadas
\[
(x,f(x))
\]
tales que $x\neq a$ pero $x\in\ ]a-\delta, a +\delta[$ están en el interior del rectángulo
producido por la intersección de las dos bandas. Pero todos estos puntos no son más que la gráfica
de la función $f$ en el conjunto $]a-\delta, a+\delta[ - \{a\}$. En otras palabras:
\begin{quote}
{\bfseries dado cualquier número $\bm{\epsilon > 0}$, existe un número $\bm{\delta > 0}$ tal que la
gráfica de $f$ en el conjunto $]a-\delta, a+\delta[ - \{a\}$ está en el interior del conjunto $C$.}
\end{quote}

El siguiente dibujo muestra lo que sucede cuando $L$ es el límite de $f(x)$ cuando $x$ se aproxime
al número $a$:
\begin{center}
\psset{unit=0.9}
\begin{pspicture}(-.5,-.5)(5,4.5)
\psaxes[ticks=none,labels=none]{->}(0,0)(-.5,-.5)(5,4)%
\uput[-90](5,0){$x$}%
\uput[180](0,4){$f(x)$}%
\psset{xunit=.35mm,yunit=.35mm,plotpoints=200}%

\psline(67.5,-1)(67.5,1)% a - \delta
\rput[Br](67.5,-10){$a - \delta$}%
\psline(82.5,-1)(82.5,1)% a
\rput[B](82.5,-10){$a$}%
\psline(97.5,-1)(97.5,1)% a + \delta
\rput[Bl](97.5,-10){$a + \delta$}%

\psline(-1,52.36)(1,52.36)% L - \epsilon
\rput[r](-3,52.36){$L - \epsilon$}%
\psline(-1,71.79)(1,71.79)% L
\rput[r](-3,71.79){$L$}%
\psline(-1,86.67)(1,86.67)% L + \epsilon
\rput[r](-3,86.67){$L + \epsilon$}%

\psset{linestyle=dashed}
\psframe[linestyle=none,fillstyle=solid,fillcolor=lightgray](1,52.36)(67.5,86.67)%
\psframe[linestyle=none,fillstyle=solid,fillcolor=lightgray](97.5,52.36)(120,86.67)%
\psline(0,52.36)(120,52.36)%
\psline(0,86.67)(120,86.67)%

\psframe[linestyle=none,fillstyle=solid,fillcolor=lightgray](67.5,1)(97.5,52.36)%
\psframe[linestyle=none,fillstyle=solid,fillcolor=lightgray](67.5,86.67)(97.5,110)%
\psline(67.5,0)(67.5,100)%
\psline(97.5,0)(97.5,100)%

\psset{linestyle=solid,xunit=.35mm,yunit=.35mm,plotpoints=200}%
\psplot{0}{125}{x dup sin mul 10 sub}%

\pscircle[fillstyle=solid,fillcolor=white](82.5,71.79){.05}
\end{pspicture}
\end{center}

Al conjunto $C$ se le denomina \emph{caja} para el gráfico de $f$ en el punto de coordenadas
$(a,L)$. De manera más general, una \emph{caja para el gráfico de una función en el punto de
coordenadas} $(x,y)$ es el interior de una región rectangular cuyos lados son paralelos a los ejes
coordenados, que contiene al punto de coordenadas $(x,y)$, de modo que ningún punto del gráfico de
la función $f$ que esté en la banda vertical, salvo, tal vez, el punto de coordenadas $(x,f(x))$,
no puede estar ni sobre la región rectangular ni bajo de ella. Si el punto de coordenadas $(x,y)$
es también el centro de la región rectangular (es decir, es la intersección de las diagonales del
rectángulo), entonces la región rectangular es denominada \emph{caja centrada en el punto de
coordenadas} $(x,y)$.

Un ejemplo de una región rectangular que no es una caja para el gráfico de $f$ es el siguiente:
\begin{center}
\psset{unit=0.9}
\begin{pspicture}(-.5,-.5)(5,4.5)
\psaxes[ticks=none,labels=none]{->}(0,0)(-.5,-.5)(5,4)%
\uput[-90](5,0){$x$}%
\uput[180](0,4){$f(x)$}%
\psset{xunit=.35mm,yunit=.35mm,plotpoints=200}%

\psset{linestyle=dashed}
\psframe[linestyle=none,fillstyle=solid,fillcolor=lightgray](1,60)(67.5,80)%
\psframe[linestyle=none,fillstyle=solid,fillcolor=lightgray](97.5,60)(120,80)%
\psline(0,60)(120,60)%
\psline(0,80)(120,80)%

\psframe[linestyle=none,fillstyle=solid,fillcolor=lightgray](67.5,1)(97.5,60)%
\psframe[linestyle=none,fillstyle=solid,fillcolor=lightgray](67.5,80)(97.5,110)%
\psline(67.5,0)(67.5,100)%
\psline(97.5,0)(97.5,100)%

\psset{linestyle=solid,xunit=.35mm,yunit=.35mm,plotpoints=200}%
\psplot{0}{125}{x dup sin mul 10 sub}%

\pscircle[fillstyle=solid,fillcolor=white](82.5,71.79){.05}

\end{pspicture}
\end{center}

Con la definición de caja, podemos decir que el conjunto $C$ es, efectivamente, una caja para el
gráfico de $f$ centrada en el punto de coordenadas $(a,L)$. En este caso, la caja $C$ tiene una
altura igual a $2\epsilon$ y una base igual a $2\delta$.

En términos de cajas, la definición de límite garantiza que:
\begin{quote}
{\bfseries si $L$ es el límite de $f(x)$ cuando $x$ se aproxima a $a$, entonces, el gráfico de $f$
tiene cajas de todas las alturas positivas posibles centradas en el punto $(a,L)$.}
\end{quote}

El siguiente dibujo, muestra dos cajas para el gráfico de $f$, ambas centradas en $(a,L)$:
\begin{center}
\psset{unit=0.9}
\begin{pspicture}(-.5,-.5)(5,4.5)
\psaxes[ticks=none,labels=none]{->}(0,0)(-.5,-.5)(5,4)%
\uput[-90](5,0){$x$}%
\uput[180](0,4){$f(x)$}%
\psset{xunit=.35mm,yunit=.35mm,plotpoints=200}%

\psline(77.5,-1)(77.5,1)% a - \delta
\rput[Br](77.5,-10){$a - \delta$}%
\psline(82.5,-1)(82.5,1)% a
\rput[B](82.5,-10){$a$}%
\psline(87.5,-1)(87.5,1)% a + \delta
\rput[Bl](87.5,-10){$a + \delta$}%

\psline(-1,65.66)(1,65.66)% L - \epsilon
\rput[r](-3,65.66){$L - \epsilon$}%
\psline(-1,71.79)(1,71.79)% L

\psline(-1,77.42)(1,77.42)% L + \epsilon
\rput[r](-3,77.42){$L + \epsilon$}%

\psset{linestyle=dashed}
\psframe[linestyle=none,fillstyle=solid,fillcolor=lightgray](1,65.66)(77.5,77.42)%
\psframe[linestyle=none,fillstyle=solid,fillcolor=lightgray](87.5,65.66)(120,77.42)%
\psline(0,65.66)(120,65.66)%
\psline(0,77.42)(120,77.42)%

\psframe[linestyle=none,fillstyle=solid,fillcolor=lightgray](77.5,1)(87.5,65.66)%
\psframe[linestyle=none,fillstyle=solid,fillcolor=lightgray](77.5,77.42)(87.5,110)%
\psline(77.5,0)(77.5,100)%
\psline(87.5,0)(87.5,100)%

\rput[l](3,71.79){$L$}%

\psset{linestyle=solid,xunit=.35mm,yunit=.35mm,plotpoints=200}%
\psplot{0}{125}{x dup sin mul 10 sub}%

\pscircle[fillstyle=solid,fillcolor=white](82.5,71.79){.05}
\end{pspicture}
\hspace{1cm}
\begin{pspicture}(-.5,-.5)(5,4.5)
\psaxes[ticks=none,labels=none]{->}(0,0)(-.5,-.5)(5,4)%
\uput[-90](5,0){$x$}%
\uput[180](0,4){$f(x)$}%
\psset{xunit=.35mm,yunit=.35mm,plotpoints=200}%

\psline(72.5,-1)(72.5,1)% a - \delta
\rput[Br](72.5,-10){$a - \delta$}%
\psline(82.5,-1)(82.5,1)% a
\rput[B](82.5,-10){$a$}%
\psline(92.5,-1)(92.5,1)% a + \delta
\rput[Bl](92.5,-10){$a + \delta$}%

\psline(-1,59.15)(1,59.15)% L - \epsilon
\rput[r](-3,59.15){$L - \epsilon$}%
\psline(-1,71.79)(1,71.79)% L
\rput[r](-3,71.79){$L$}%
\psline(-1,82.41)(1,82.41)% L + \epsilon
\rput[r](-3,82.41){$L + \epsilon$}%

\psset{linestyle=dashed}
\psframe[linestyle=none,fillstyle=solid,fillcolor=lightgray](1,59.15)(72.5,82.41)%
\psframe[linestyle=none,fillstyle=solid,fillcolor=lightgray](92.5,59.15)(120,82.41)%
\psline(0,59.15)(120,59.15)%
\psline(0,82.41)(120,82.41)%

\psframe[linestyle=none,fillstyle=solid,fillcolor=lightgray](72.5,1)(92.5,59.15)%
\psframe[linestyle=none,fillstyle=solid,fillcolor=lightgray](72.5,82.41)(92.5,110)%
\psline(72.5,0)(72.5,100)%
\psline(92.5,0)(92.5,100)%

\psset{linestyle=solid,xunit=.35mm,yunit=.35mm,plotpoints=200}%
\psplot{0}{125}{x dup sin mul 10 sub}%

\pscircle[fillstyle=solid,fillcolor=white](82.5,71.79){.05}
\end{pspicture}
\end{center}

Estos dos últimos dibujos sugieren que el gráfico de $f$ tiene cajas centradas en $(a,L)$ de todas
las alturas posibles.

A continuación, podemos visualizar por qué las funciones $f$, $g$ y $h$ de la sección anterior, las
que utilizamos para mostrar por qué en la definición de límite no importa el valor que pueda tener
(o no tener), tienen el mismo límite:
\begin{center}
\psset{unit=0.9}
\begin{pspicture}(-1,-1)(3.5,3.5)
\psset{xunit=.8,yunit=.5}%
\psaxes[ticks=none,labels=none]{->}(0,0)(-1,-1)(4,6)%
\uput[-90](4,0){$x$}%
\uput[180](0,6){$f(x)$}%
\psplot{-1}{2}{2 x mul 1 add}%

\psframe[linestyle=none,fillstyle=solid,fillcolor=lightgray](0,2.5)(.75,3.5)%
\psframe[linestyle=none,fillstyle=solid,fillcolor=lightgray](1.25,2.5)(2,3.5)%
\psset{linestyle=dashed}
\psline(0,2.5)(2,2.5)%
\psline(0,3.5)(2,3.5)%

\psframe[linestyle=none,fillstyle=solid,fillcolor=lightgray](.75,0)(1.25,2.5)%
\psframe[linestyle=none,fillstyle=solid,fillcolor=lightgray](.75,3.5)(1.25,5)%
\psset{linestyle=dashed}
\psline(.75,0)(.75,5)%
\psline(1.25,0)(1.25,5)%

\psline[linecolor=gray,linestyle=dashed](1,0)(1,3)(0,3)%
\rput(1,-.5){$1$}%
\rput(-.25,3){$3$}%

\end{pspicture}
%
\begin{pspicture}(-1,-1)(3.5,3)
\psset{xunit=.8,yunit=.5}%
\psaxes[ticks=none,labels=none]{->}(0,0)(-1,-1)(4,6)%
\uput[-90](4,0){$x$}%
\uput[180](0,6){$g(x)$}%
\psplot{-1}{.95}{2 x mul 1 add}%
\psplot{1.05}{2}{2 x mul 1 add}%

\psframe[linestyle=none,fillstyle=solid,fillcolor=lightgray](0,2.5)(.75,3.5)%
\psframe[linestyle=none,fillstyle=solid,fillcolor=lightgray](1.25,2.5)(2,3.5)%
\psset{linestyle=dashed}
\psline(0,2.5)(2,2.5)%
\psline(0,3.5)(2,3.5)%

\psframe[linestyle=none,fillstyle=solid,fillcolor=lightgray](.75,0)(1.25,2.5)%
\psframe[linestyle=none,fillstyle=solid,fillcolor=lightgray](.75,3.5)(1.25,5)%
\psset{linestyle=dashed}
\psline(.75,0)(.75,5)%
\psline(1.25,0)(1.25,5)%

\pscircle(1,3){.05}%
\pscircle[fillstyle=solid,fillcolor=black](1,1){.05}%
\psline[linecolor=gray,linestyle=dashed](1,0)(1,1)(0,1)%
\rput(1,-.5){$1$}%
\rput(-.25,3){$3$}%
\rput(-.25,1){$1$}%

\end{pspicture}
%
\begin{pspicture}(-1,-1)(3.5,3)
\psset{xunit=.8,yunit=.5}%
\psaxes[ticks=none,labels=none]{->}(0,0)(-1,-1)(4,6)%
\uput[-90](4,0){$x$}%
\uput[180](0,6){$h(x)$}%
\psplot{-1}{.95}{2 x mul 1 add}%
\psplot{1.05}{2}{2 x mul 1 add}%

\psframe[linestyle=none,fillstyle=solid,fillcolor=lightgray](0,2.5)(.75,3.5)%
\psframe[linestyle=none,fillstyle=solid,fillcolor=lightgray](1.25,2.5)(2,3.5)%
\psset{linestyle=dashed}
\psline(0,2.5)(2,2.5)%
\psline(0,3.5)(2,3.5)%

\psframe[linestyle=none,fillstyle=solid,fillcolor=lightgray](.75,0)(1.25,2.5)%
\psframe[linestyle=none,fillstyle=solid,fillcolor=lightgray](.75,3.5)(1.25,5)%
\psset{linestyle=dashed}
\psline(.75,0)(.75,5)%
\psline(1.25,0)(1.25,5)%

\pscircle(1,3){.05}%
\rput(1,-.5){$1$}%
\rput(-.25,3){$3$}%
\end{pspicture}
\end{center}

Veamos un ejemplo.

\begin{exemplo}[Solución]{%
Sea $\funcjc{f}{\mathbb{R}}{\mathbb{R}}$ la función definida por $f(x)=x^2$ para todo
$x\in\mathbb{R}$.

\vspace*{0.5\baselineskip}
\begin{enumerate}
\item Encuentre un $\delta>0$ tal que si $0 < |x-3| < \delta$ entonces $|f(x) - 9| < \epsilon =
    \frac{1}{2}$.
\item Construya una caja, tal como está definida en el texto, que ilustre el resultado
    encontrado en el primer punto.
\end{enumerate}
}%
\begin{enumerate}[leftmargin=*]
\item En el ejemplo de la página~\pageref{ex:lim001}, probamos que
      \[
        9 = \limjc{x^2}{x}{3}.
      \]
      Para probar esta igualdad, dado $\epsilon > 0$, encontramos que el número
      \[
        \delta = \min\left\{1,\frac{\epsilon}{7}\right\}
      \]
      es tal que, si $|x - 3| < \delta$, entonces $|x^2 - 9| < \epsilon$.

      Por lo tanto, como $\epsilon = \frac{1}{2}$, el número buscado es:
      \[
        \delta = \min\left\{1,\frac{\frac{1}{2}}{7}\right\} = \frac{1}{14}.
      \]

      En resumen, si $|x - 3| < \frac{1}{14}$, entonces $|x^2 - 9| < \frac{1}{2}$.

\item Como $9 = \limjc{x^2}{x}{3}$, dado $\epsilon = \frac{1}{2}$, el conjunto
      \[
        C = \big\{(x,y) \in \mathbb{R}^2 : x \in \big]3 - \frac{1}{14}, 3 + \frac{1}{14}\big[, \
        y \in \big]9 - \frac{1}{2}, 9 + \frac{1}{2}\big[\big\}
      \]
      es una caja para el gráfico de $f$ en el punto de coordenadas $(3,9)$.

      El siguiente, es un dibujo de esta caja:
      \begin{center}
      \psset{unit=0.8}
      \psset{xAxisLabel={},yAxisLabel={},labelFontSize=\scriptstyle}
      \begin{psgraph}[arrows=->,Dy=3](0,0)(-0.25,-1)(3.75,11){0.8\textwidth}{6cm}
          \uput[0](0,11){$y$}%
          \uput[-90](4,0){$x$}%

          \psset{PointSymbol=none,PointName=none}
          \pstGeonode[]%
            (! 3 1 14 div sub 0){A}(! 3 1 14 div add 0){B}%
            (! 3 1 14 div sub 10.5){A'}%
            (! 3 1 14 div add 10.5){B'}%
            (! 0 9 1 2 div sub){C}(! 0 9 1 2 div add){D}%
            (! 3.5 9 1 2 div sub){C'}(! 3.5 9 1 2 div add){D'}

          \pstInterLL[]%
            {A}{A'}{C}{C'}{U}%
          \pstInterLL[]%
            {B}{B'}{C}{C'}{V}%
          \pstInterLL[]%
            {A}{A'}{D}{D'}{W}%
          \pstInterLL[]%
            {B}{B'}{D}{D'}{X}%

          {\psset{linestyle=none,fillstyle=solid,fillcolor=lightgray}%
          \psframe[]%
            (A)(V)%
          \psframe[]%
            (C)(W)%
          \psframe[]%
            (W)(B')%
          \psframe[]%
            (V)(D')%
          }

          {\psset{linestyle=dashed}
          \pstLineAB[]%
            {A}{A'}%
          \pstLineAB[]%
            {B}{B'}%
          \pstLineAB[]%
            {C}{C'}%
          \pstLineAB[]%
            {D}{D'}%
          }

          \scriptsize%
          \rput[tr](A){$3 - \frac{1}{14}$}%
          \rput[tl](B){$3 + \frac{1}{14}$}%
          \rput[tr](C){$9 - \frac{1}{2}$}%
          \rput[br](D){$9 + \frac{1}{2}$}%

      \end{psgraph}
      \end{center}

      Ahora bien, por la definición de caja, como
      \[
      9 = \limjc{x^2}{x}{3},
      \]
      el gráfico de la función $f$ en el intervalo $]3 - \frac{1}{14}, 3 + \frac{1}{14}[$ debe
      estar contenido plenamente en la caja. Constatemos esto.

      En primer lugar, veamos que $f(x)$ es mayor que $9 - \frac{1}{2}$ cuando $x = 3 -
      \frac{1}{14}$. Por un lado tenemos que:
      \begin{align*}
        f\left(3 - \frac{1}{14}\right) &= f\left(\frac{41}{14}\right) \\
          &= \frac{1\,681}{196} = 8 + \frac{113}{196},
      \end{align*}
      Por otro lado:
      \begin{align*}
      9 - \frac{1}{2} &= 8 + \frac{1}{2} = 8 + \frac{98}{196}.
      \end{align*}
      Por lo tanto:
      \[
          f\left(3 - \frac{1}{14}\right) = 8 + \frac{113}{196} > 8 + \frac{98}{196} = 9 - \frac{1}{2}.
      \]

      En segundo lugar, veamos que $f(x)$ es menor que $9 + \frac{1}{2}$ cuando $x = 3 +
      \frac{1}{14}$. Por un lado tenemos que:
      \begin{align*}
        f\left(3 + \frac{1}{14}\right) &= f\left(\frac{43}{14}\right) \\
          &= \frac{1\,849}{196} = 9 + \frac{85}{196},
      \end{align*}
      Por otro lado:
      \begin{align*}
      9 + \frac{1}{2} &= 9 + \frac{1}{2} = 9 + \frac{98}{196}.
      \end{align*}
      Por lo tanto:
      \[
          f\left(3 - \frac{1}{14}\right) = 9 + \frac{85}{196} < 9 + \frac{98}{196} = 9 + \frac{1}{2}.
      \]

      Finalmente, como la función $f$ es creciente en $[0,+\infty[$, entonces
      \[
        9 - \frac{1}{2} < f(x) < 9 + \frac{1}{2}
      \]
      siempre que
      \[
        3 - \frac{1}{14} < x < 3 + \frac{1}{14}.
      \]
      Por esta razón, al dibujar el gráfico de $f$ en este intervalo, éste deberá estar
      contenido plenamente en la caja, como lo asegura la definición de límite y de caja. El
      siguiente es un dibujo de la situación:
      \begin{center}
      \psset{unit=0.8}
      \psset{xAxisLabel={},yAxisLabel={},labelFontSize=\scriptstyle}
      \begin{psgraph}[arrows=->,Dy=3](0,0)(-0.25,-1)(3.75,11){0.8\textwidth}{6cm}
          \uput[0](0,11){$y$}%
          \uput[-90](4,0){$x$}%

          \psset{PointSymbol=none,PointName=none}
          \pstGeonode[]%
            (! 3 1 14 div sub 0){A}(! 3 1 14 div add 0){B}%
            (! 3 1 14 div sub 10.5){A'}%
            (! 3 1 14 div add 10.5){B'}%
            (! 0 9 1 2 div sub){C}(! 0 9 1 2 div add){D}%
            (! 3.5 9 1 2 div sub){C'}(! 3.5 9 1 2 div add){D'}

          \pstInterLL[]%
            {A}{A'}{C}{C'}{U}%
          \pstInterLL[]%
            {B}{B'}{C}{C'}{V}%
          \pstInterLL[]%
            {A}{A'}{D}{D'}{W}%
          \pstInterLL[]%
            {B}{B'}{D}{D'}{X}%

          {\psset{linestyle=none,fillstyle=solid,fillcolor=lightgray}%
          \psframe[]%
            (A)(V)%
          \psframe[]%
            (C)(W)%
          \psframe[]%
            (W)(B')%
          \psframe[]%
            (V)(D')%
          }

          {\psset{linestyle=dashed}
          \pstLineAB[]%
            {A}{A'}%
          \pstLineAB[]%
            {B}{B'}%
          \pstLineAB[]%
            {C}{C'}%
          \pstLineAB[]%
            {D}{D'}%
          }

          \psplot[plotpoints=200]%
            {0}{3.25}{x dup mul}%
          \uput[0](! 3.25 /x 3.25 def x dup mul){$y = x^2$}%

          \scriptsize%
          \rput[tr](A){$3 - \frac{1}{14}$}%
          \rput[tl](B){$3 + \frac{1}{14}$}%
          \rput[tr](C){$9 - \frac{1}{2}$}%
          \rput[br](D){$9 + \frac{1}{2}$}%

      \end{psgraph}
      \end{center}

\end{enumerate}

\end{exemplo}

Para terminar esta sección, veamos cómo la interpretación geométrica nos puede ser de ayuda para
demostrar que un cierto número $L$ no es límite de $f(x)$ cuando $x$ se aproxima al número $a$.

De la definición de límite, decir que $L$ no es el límite equivale a decir que:
\begin{quote}
{\bfseries existe un número $\bm{\epsilon > 0}$ tal que, para todo número $\bm{\delta > 0}$, existe
un número $\bm{x}$ tal que
\[
\bm{0 < |x - a| < \delta} \yjc \quad \bm{|f(x) - L| \geq \epsilon}.
\]
}
\end{quote}

Ilustremos estas ideas con un ejemplo.

\begin{exemplo}[Solución]{%
Demostrar que
\[
\limjc{f(x)}{x}{1} \neq \frac{3}{2}
\]
si
\[
f(x) =
\begin{cases}
x & \text{si } x \in\ [0,1],\\
x + 1 & \text{si } x \in\ ]1,2].
\end{cases}
\]
}%
Para demostrar que el número $\frac{3}{2}$ no es el límite de $f(x)$ cuando $x$ se aproxima a $1$
si $f$ está definida de la siguiente manera:
\[
f(x) =
\begin{cases}
x & \text{si } x \in\ [0,1],\\
x + 1 & \text{si } x \in\ ]1,2],
\end{cases}
\]
debemos hallar un número $\epsilon > 0$ tal que, para todo $\delta > 0$, encontremos un número
$x\in \Dm(f)$ tal que
\[
0 < |x - 1| < \delta \yjc \left|f(x) - \frac{3}{2}\right| \geq \epsilon.
\]

Para ello, primero dibujemos la función $f$:
\begin{center}
\begin{pspicture}(-.5,-.5)(4,3)
\psset{xunit=1.5,yunit=.8}%
\psaxes[ticks=none,labels=none]{->}(0,0)(-.5,-.5)(2.5,3.5)%
\uput[-90](2.5,0){$x$}%
\uput[180](0,3.5){$f(x)$}%

\psplot{0}{1}{x}%
\psplot[arrows=o-]{1.03}{2}{x 1 add}%
%\pscircle(1,2){.03}%

\psline(0,1)(.1,1)%
\rput[r](-.1,1){\footnotesize{$1$}}%

\psline(0,1.5)(.1,1.5)%
\rput[r](-.1,1.5){\small{$\frac{3}{2}$}}%

\psline(0,2)(.1,2)%
\rput[r](-.1,2){\footnotesize{$2$}}%

\psline(1,0)(1,.12)%
\rput[t](1,-.12){\footnotesize{$1$}}%

\end{pspicture}
\end{center}

Este dibujo nos permite ver que cualquier caja del gráfico de $f$ ubicada entre las rectas
horizontales de ecuaciones $y=1$ y $y = 2$ no contendrá ningún elemento del gráfico de $f$.
Recordemos que el alto de la caja es $2\epsilon$: una distancia de un $\epsilon$ sobre
$\frac{3}{2}$ y una distancia de un $\epsilon$ bajo $\frac{3}{2}$. Por lo tanto, si elegimos
$\epsilon$ igual a $\frac{1}{4}$, la banda horizontal alrededor de $\frac{3}{2}$ se vería así:
\begin{center}
\begin{pspicture}(-.5,-.5)(4,3)
\psset{xunit=1.5,yunit=.8}%
\psaxes[ticks=none,labels=none]{->}(0,0)(-.5,-.5)(2.5,3.5)%
\uput[-90](2.5,0){$x$}%
\uput[180](0,3.5){$f(x)$}%


\psplot{0}{1}{x}%
\psplot{1.03}{2}{x 1 add}%
\pscircle(1,2){.03}%

\psframe[linestyle=none,fillstyle=solid,fillcolor=lightgray](0,1.25)(1.75,1.75)%
\psset{linestyle=dashed}
\psline(0,1.25)(1.75,1.25)%
\psline(0,1.75)(1.75,1.75)%


\psline(0,1)(.1,1)%
\rput[r](-.1,1){\footnotesize{$1$}}%

\psline(0,1.5)(.1,1.5)%
\rput[r](-.1,1.5){\small{$\frac{3}{2}$}}%

\psline(0,2)(.1,2)%
\rput[r](-.1,2){\footnotesize{$2$}}%

\psline(1,0)(1,.12)%
\rput[t](1,-.12){\footnotesize{$1$}}%

\end{pspicture}
\end{center}

A continuación, podemos dibujar tres bandas verticales alrededor del número $1$, lo que nos
sugerirá que ninguna caja del gráfico de $f$ con el alto $2\epsilon$ elegido contendrá elementos
del gráfico:

\begin{center}
\begin{pspicture}(-.5,-.5)(4,3)
\psset{xunit=1.5,yunit=.8}%
\psaxes[ticks=none,labels=none]{->}(0,0)(-.5,-.5)(2.5,3.5)%
\uput[-90](2.5,0){$x$}%
\uput[180](0,3.5){$f(x)$}%

\rput[Br](.6,-.4){\footnotesize{$1 - \delta$}}%
\rput[Bl](1.4,-.4){\footnotesize{$1 + \delta$}}%

\psframe[linestyle=none,fillstyle=solid,fillcolor=lightgray](0,1.25)(.6,1.75)%
\psframe[linestyle=none,fillstyle=solid,fillcolor=lightgray](1.4,1.25)(1.75,1.75)%
\psset{linestyle=dashed}
\psline(0,1.25)(1.75,1.25)%
\psline(0,1.75)(1.75,1.75)%

\psframe[linestyle=none,fillstyle=solid,fillcolor=lightgray](.6,0)(1.4,1.25)%
\psframe[linestyle=none,fillstyle=solid,fillcolor=lightgray](.6,1.75)(1.4,2.5)%
\psset{linestyle=dashed}
\psline(.6,0)(.6,2.5)%
\psline(1.4,0)(1.4,2.5)%

\psline(0,1)(.1,1)%
\rput[r](-.1,1){\footnotesize{$1$}}%

\psline(0,1.5)(.1,1.5)%
\rput[r](-.1,1.5){\small{$\frac{3}{2}$}}%

\psline(0,2)(.1,2)%
\rput[r](-.1,2){\footnotesize{$2$}}%

\psline(1,0)(1,.12)%
\rput[B](1,-.4){\footnotesize{$1$}}%

\psset{linestyle=solid}
\psplot{0}{1}{x}%
\psplot{1.03}{2}{x 1 add}%
\pscircle(1,2){.03}%

\end{pspicture}
%
\begin{pspicture}(-.5,-.5)(4,3)
\psset{xunit=1.5,yunit=.8}%
\psaxes[ticks=none,labels=none]{->}(0,0)(-.5,-.5)(2.5,3.5)%
\uput[-90](2.5,0){$x$}%
\uput[180](0,3.5){$f(x)$}%

\rput[Br](.75,-.4){\footnotesize{$1 - \delta$}}%
\rput[Bl](1.25,-.4){\footnotesize{$1 + \delta$}}%

\psframe[linestyle=none,fillstyle=solid,fillcolor=lightgray](0,1.25)(.75,1.75)%
\psframe[linestyle=none,fillstyle=solid,fillcolor=lightgray](1.25,1.25)(1.75,1.75)%
\psset{linestyle=dashed}
\psline(0,1.25)(1.75,1.25)%
\psline(0,1.75)(1.75,1.75)%

\psframe[linestyle=none,fillstyle=solid,fillcolor=lightgray](.75,0)(1.25,1.25)%
\psframe[linestyle=none,fillstyle=solid,fillcolor=lightgray](.75,1.75)(1.25,2.5)%
\psset{linestyle=dashed}
\psline(.75,0)(.75,2.5)%
\psline(1.25,0)(1.25,2.5)%

\psline(0,1)(.1,1)%
\rput[r](-.1,1){\footnotesize{$1$}}%

\psline(0,1.5)(.1,1.5)%
\rput[r](-.1,1.5){\small{$\frac{3}{2}$}}%

\psline(0,2)(.1,2)%
\rput[r](-.1,2){\footnotesize{$2$}}%

\psline(1,0)(1,.12)%
\rput[B](1,-.3){\footnotesize{$1$}}%

\psset{linestyle=solid}
\psplot{0}{1}{x}%
\psplot{1.03}{2}{x 1 add}%
\pscircle(1,2){.03}%

\end{pspicture}
%
\begin{pspicture}(-.5,-.5)(4,3)
\psset{xunit=1.5,yunit=.8}%
\psaxes[ticks=none,labels=none]{->}(0,0)(-.5,-.5)(2.5,3.5)%
\uput[-90](2.5,0){$x$}%
\uput[180](0,3.5){$f(x)$}%

\rput[Br](.85,-.4){\footnotesize{$1 - \delta$}}%
\rput[Bl](1.15,-.4){\footnotesize{$1 + \delta$}}%

\psframe[linestyle=none,fillstyle=solid,fillcolor=lightgray](0,1.25)(.85,1.75)%
\psframe[linestyle=none,fillstyle=solid,fillcolor=lightgray](1.15,1.25)(1.75,1.75)%
\psset{linestyle=dashed}
\psline(0,1.25)(1.75,1.25)%
\psline(0,1.75)(1.75,1.75)%

\psframe[linestyle=none,fillstyle=solid,fillcolor=lightgray](.85,0)(1.15,1.25)%
\psframe[linestyle=none,fillstyle=solid,fillcolor=lightgray](.85,1.75)(1.15,2.5)%
\psset{linestyle=dashed}
\psline(.85,0)(.85,2.5)%
\psline(1.15,0)(1.15,2.5)%

\psline(0,1)(.1,1)%
\rput[r](-.1,1){\footnotesize{$1$}}%

\psline(0,1.5)(.1,1.5)%
\rput[r](-.1,1.5){\small{$\frac{3}{2}$}}%

\psline(0,2)(.1,2)%
\rput[r](-.1,2){\footnotesize{$2$}}%

\psline(1,0)(1,.12)%
\rput[B](1,-.4){\footnotesize{$1$}}%

\psset{linestyle=solid}
\psplot{0}{1}{x}%
\psplot{1.03}{2}{x 1 add}%
\pscircle(1,2){.03}%

\end{pspicture}
\end{center}
Estos dibujos nos sugieren cuál debe ser el $x$ que buscamos para el cual se verifiquen las
condiciones siguientes:
\[
0 < |x - 1| < \delta\yjc \left|f(x) - \frac{3}{2}\right| \geq \epsilon.
\]
Cualquier $x\neq 1$ que esté en el dominio de $f$ y en el intervalo $]1-\delta,1+\delta[$ satisfará
estas dos condiciones.

Procedamos, entonces, a demostrar que el número $\frac{3}{2}$ no es el límite de $f(x)$ cuando $x$
se aproxima a $1$.

Sean $\epsilon = \frac{1}{4}$ y $\delta > 0$. Si $\delta > 1$, la banda vertical abarcaría más que
el dominio de la función $f$. En este caso, $x = 0$, satisfaría las condiciones. En efecto:
\begin{enumerate}
\item $x$ está en el dominio de $f$;
\item Como $x = 0$, entonces $x\neq 1$ y
\[
|x - 1| = |0 - 1| = 1 < \delta;
\]
\item Como $0 \in\ [0,1]$, entonces $f(0) = 0$ y, por ello:
\begin{align*}
\left|f(x) - \frac{3}{2}\right| &= \left|0 - \frac{3}{2}\right| \\
&= \frac{3}{2} > \frac{1}{4} = \epsilon.
\end{align*}
\end{enumerate}

Ahora, si $\delta \leq 1$, podemos elegir el número $x$ que está en la mitad entre $1$ y
$1-\delta$. Ese número es:
\[
x = \frac{(1-\delta) + 1}{2} = 1 - \frac{\delta}{2}.
\]
Entonces:
\begin{enumerate}
\item $x$ está en el dominio de $f$, pues, como
\begin{enumerate}
\item $1 - \delta \geq 0$, ya que $\delta \leq 1$ y
\item $\frac{\delta}{2} < \delta$, ya que $\delta > 0$,
\end{enumerate}
se tiene que
\[
0 \leq 1 - \delta < 1 - \frac{\delta}{2} < 1;
\]
es decir, $x\in [0,1[$.
\item Como $\delta > 0$, entonces $\frac{\delta}{2} > 0$, entonces
\[
x = 1 - \frac{\delta}{2} \neq 1.
\]
\item Como $x\in\ [0,1]$, entonces
\[
f(x) = x = 1 - \frac{\delta}{2}.
\]
Por lo tanto:
\begin{align*}
\left|f(x) - \frac{3}{2}\right| &= \left|\left(1 - \frac{\delta}{2}\right) - \frac{3}{2}\right|
\\
&= \left|-\frac{1}{2} - \frac{\delta}{2}\right| = \frac{1}{2} + \frac{\delta}{2} > \frac{1}{4} = \epsilon,
\end{align*}
pues
\[
\frac{1}{2} > \frac{1}{4}\yjc \frac{\delta}{2} > 0.
\]
\end{enumerate}

Por lo tanto, el número $\frac{3}{2}$ no puede ser el límite de $f(x)$ cuando $x$ se aproxima a
$1$.
\end{exemplo}
%-----> 2008 09 12
%{\color{red} Quizás aquí amerite también un ejercicio en el que se muestre que ningún número real
%$L$ puede ser límite de la función $f$ de este último ejemplo.}

En la sección próxima, vamos a mostrar un ejemplo de cómo se puede resolver un problema aplicando
el método de encontrar un número $\delta$ dado el número $\epsilon$.

\subsection{Ejercicios}
\begingroup
\small En los ejercicios propuestos a continuación, se busca que el lector adquiera un dominio
básico del aspecto operativo de la definición de límite, a pesar de que luego, en la práctica del
cálculo de límites, no se recurra a esta definición, sino a las propiedades que se obtienen y se
demuestran a partir de esta definición.

\begin{multicols}{2}
\begin{enumerate}[leftmargin=*]
\item Sean $\funcjc{f}{I}{\mathbb{R}}$, $a\in\mathbb{R}$, $L\in\mathbb{R}$ y $\epsilon > 0$:
\begin{enumerate}[leftmargin=*]
\item encuentre un $\delta > 0$ tal que $|f(x) - L| < \epsilon$ siempre que $0 < |x - a| <
    \delta$; y
\item construya una caja de $f(x)$ centrada en $(a,L)$ y de altura $2\epsilon$ que ilustre
    el resultado encontrado en el punto anterior
\end{enumerate}
para cada $f$, $a$, $L$ y $\epsilon$ dados a continuación:
\begin{enumerate}[leftmargin=*]
\item $f(x) = x - 2$; $a = 5$; $L = 3$; $\epsilon = 0.02$.
\item $f(x) = x - 2$; $a = -2$; $L = -4$; $\epsilon = 0.01$.
\item $f(x) = -2x + 1$; $a = -2$; $L = 5$; $\epsilon = 0.05$.
\item $f(x) = 4x - 3$; $a = 1$; $L = 1$; $\epsilon = 0.03$.
\item $f(x) = \frac{2x^2 - 7x + 6}{x - 2}$; $a = 2$; $L = 1$; $\epsilon = 0.04$.
\item $f(x) = \frac{8x^2 + 10x + 3}{2x + 1}$; $a = -\frac{1}{2}$; $L = 1$; $\epsilon = 0.003$.
\item $f(x) = x^2$; $a = 3$; $L = 9$; $\epsilon = 1$.
\item $f(x) = x^2$; $a = 3$; $L = 9$; $\epsilon = 0.1$.
\item $f(x) = x^2$; $a = -2$; $L = 4$; $\epsilon = 0.3$.
\item $f(x) = x^2 + 3x - 5$; $a = 1$; $L = -1$; $\epsilon = 0.4$.
\item $f(x) = x^2 + 3x - 5$; $a = -1$; $L = -7$; $\epsilon = 0.08$.
\end{enumerate}

\item Demuestre, utilizando la definición de límite:
   \begin{enumerate}[leftmargin=*]
   \item $\displaystyle\limjc{\frac{3x^2 + 5x - 2}{x + 2}}{x}{-2} = -7$.
   \item $\displaystyle\limjc{\frac{6x}{x + 2}}{x}{1} = 2$.
   \item $\displaystyle\limjc{\sqrt{x + 7}}{x}{2} = 3$.
   \item $\displaystyle\limjc{\frac{x^3 - 2x^2 - 3x}{x - 3}}{x}{3} = 12$.
   \item $\displaystyle\limjc{\frac{8x^2 + 10x + 3}{2x + 1}}{x}{-\frac{1}{2}} = 1$.
   \item $\displaystyle\limjc{(2x^2 - 3x + 4)}{x}{1} = 3$.
   \item $\displaystyle\limjc{\frac{x^2 - 7x + 5}{-2x + 3}}{x}{1} = -1$.
   \end{enumerate}
\end{enumerate}
\end{multicols}
\endgroup

\section{Energía solar para Intipamba}
Intipamba es una pequeña comunidad asentada en una meseta. Desde allí, se tiene la sensación de
poder acariciar las trenzadas montañas que, más abajo, están rodeadas por un límpido cielo azul la
mayor parte del año. El sol es un visitante asiduo de Intipamba. En algunas ocasiones se ha quedado
tanto tiempo que las tierras alrededor de Intipamba no han podido dar su fruto, y los lugareños han
tenido que ir muy lejos en busca de agua.

Los viejos ya no recuerdan cuando llegó la energía eléctrica a Intipamba y los jóvenes, si no la
tuvieran ya, la extrañarían, pues, desde que tienen memoria, la han tenido. Pero en los últimos
años la energía eléctrica que llega al pueblo ya no es suficiente. El gobierno les ha dicho que
llevar más energía es muy costoso. Sin embargo, una organización no gubernamental ha propuesto a la
comunidad solventar la falta de energía mediante la construcción de una pequeña planta que genere
energía eléctrica a partir de energía solar, para aprovechar, así, ``el sol'' de Intipamba.

Para abaratar los costos, la comunidad entera participará en la construcción de la planta. Entre
las tareas que realizarán los lugareños, está la construcción de los paneles solares. Las
especificaciones técnicas indican que los paneles deben tener una forma rectangular, que el largo
debe medir el doble que el ancho, el cuál no debe superar los tres metros, y que el área debe ser
de ocho metros cuadrados, tolerándose a lo más un error del uno por ciento.

Se ha decidido que cada panel tenga por dimensiones 2 y 4 metros, lo que satisface el requerimiento
de que la longitud del largo sea el doble que la del ancho. El problema en que se encuentran los
encargados de la construcción de los paneles consiste en determinar con qué grado de precisión debe
calibrarse la maquinaria que corta los paneles para que se garantice que el área obtenida no
difiera de los ocho metros cuadrados en más del uno por ciento.

\subsection{Planteamiento del problema}

En primer lugar, ?`qué significa que el área no difiera de los ocho metros cuadrados en más del uno
por ciento? Ninguna máquina cortará los lados de cada panel de dos y cuatro metros exactamente,
sino que, en algunos cortes, por ejemplo, la dimensión del ancho será un poco menos que dos metros;
en otros cortes, podría ser un poco más; una situación similar ocurrirá respecto de la dimensión
del largo. El resultado de esto es que el área de cada panel no será exactamente igual a ocho
metros cuadrados.

La diferencia entre el área real de un panel y los ocho metros cuadrados esperados se denomina
\emph{error}. Esta cantidad puede ser positiva o negativa, según el área real sea mayor o menor a
ocho metros cuadrados. Para cuantificar únicamente la diferencia entre el área real y la esperada,
se define el concepto de \emph{error absoluto} como el valor absoluto del error. El error absoluto
no da cuenta de si el área obtenida es mayor o menor que los ocho metros cuadrados esperados, solo
da cuenta de la diferencia positiva entre estos dos valores.

Si se representa con la letra $a$ el número de metros cuadrados que mide el área real de un panel,
el error absoluto se expresa de la siguiente manera:
\[
\text{error absoluto } = |a - 8|.
\]

La especificación de que ``el área no difiera de los ocho metros cuadrados en más del uno por
ciento'' quiere decir, entonces, que el error absoluto sea menor que el uno por ciento del área
esperada; es decir, que el error absoluto sea menor que el uno por ciento de ocho metros, valor
igual a 0.08 metros cuadrados. Por lo tanto, para que se cumpla con la especificación técnica, la
calibración de la máquina debe asegurar la verificación de la siguiente desigualdad:
\[
|a - 8| < 0.08.
\]

En resumen, el problema a resolver por el equipo de Intipamba es:%
\marcojc{.9}{1.5}{black}{black}{white}{%
calibrar la máquina de los paneles para garantizar que la desigualdad
\begin{equation}
\label{eqLim001}%
|a - 8| < 0.08.
\end{equation}
sea verdadera.}

\subsection{El modelo}

?`Qué se quiere decir con ``calibrar la máquina'' para garantizar que la
desigualdad~(\ref{eqLim001}) sea verdadera? Desde el punto de vista de la máquina, cuando ésta
``dice que va a cortar el ancho de dos metros'', cortará, en realidad, de un poco más de dos
metros, en algunas ocasiones; en otras, de un poco menos que dos metros. Lo mismo sucederá con el
largo. Esto significa que el área del panel obtenida en cada corte no será, con toda seguridad,
igual a ocho metros cuadrados. La calibración consiste, entonces, en decirle a la máquina en cuánto
se puede equivocar a lo más en el corte del largo y del ancho para que asegure a los lugareños de
Intipamba que el ``error absoluto'' en el área del panel no difiera de ocho metros cuadrados en más
del uno por ciento. En otras palabras, los constructores de los paneles quieren saber en cuánto la
máquina puede equivocarse en el corte del largo y del ancho para que el área del panel no difiera
en más del uno por ciento; es decir, quieren saber el valor máximo permitido para el error en las
longitudes de los cortes, de manera que el área no difiera en más del uno por ciento con el área
esperada.

El área real de cada panel no es, entonces, constante; dependerá de las dos dimensiones del panel
que la máquina cortadora produzca realmente. Bajo la suposición de que ésta máquina siempre logra
cortar el largo de una longitud el doble que la del ancho, el área real de cada panel se puede
expresar exclusivamente en función de una de las dimensiones. Por ejemplo, el ancho.

En efecto, los encargados de la construcción de los paneles utilizan la letra $x$ para representar
el número de metros que mide el ancho real de un panel; entonces, $2x$ representa el número de
metros que mide el largo del panel. De esta manera, si el área de cada panel es de $a$ metros
cuadrados, $a$ puede ser expresada en función de $x$ de la siguiente manera:
\[
a = 2x^2.
\]
Vamos a representar con la letra $A$ esta relación funcional entre $a$ y $x$. Así, la función $A$
definida por
\[
A(x) = 2x^2
\]
nos permite escribir las siguientes igualdades:
\[
a = A(x) = 2x^2.
\]

?`Cuál es el dominio de la función $A$? La respuesta se encuentra después de averiguar los valores
que puede tomar la variable $x$. Dentro de las especificaciones técnicas para la construcción de
los paneles, se indica que el ancho del panel no puede superar los tres metros; como, además, el
ancho no puede ser medido por el número cero ni por un número negativo, se puede elegir\footnote{No
hay una sola manera de realizar esta elección; es decir, hay varias alternativas para el dominio de
la función $A$. En efecto, aunque los constructores de los paneles no saben de cuánto exactamente
es el ancho de cada panel, están seguros que no podrá ser muy diferente de dos metros. Esto
significa que se podría elegir como dominio de la función $A$, por ejemplo, el intervalo $[1.5,
2.5]$; también se podría tomar el intervalo $[1,2.5]$.} como dominio de $A$ el intervalo $]0,3]$.
En resumen, $A$ es una función de $]0,3]$ en $\mathbb{R}^+$.

El equipo de Intipamba encargado de los paneles ya pueden representar simbólicamente el error
absoluto que comete la máquina en el corte del ancho de un panel: el valor absoluto de la
diferencia entre $x$, el valor de la longitud real del ancho, y 2, el valor esperado para el ancho:
\[
\text{error absoluto } = |x - 2|.
\]

El problema que tienen los constructores es, entonces, saber de qué valor no debe superar este
error absoluto para asegurar que el área del panel no difiera del uno por ciento del área esperada.
Es decir, los constructores necesitan encontrar un número $\delta > 0$ tal que
\[
\text{si } |x - 2| < \delta, \text{ entonces } |A(x) - 8| < 0.08;
\]
es decir, deben hallar un $\delta > 0$ tal que
\[
\text{si } |x - 2| < \delta, \text{ entonces } |2x^2 - 8| < 0.08.
\]

Ahora los lugareños de Intipamba acaban de formular un \emph{modelo matemático} del problema; es
decir, tienen una representación mediante símbolos que, por un lado, representan elementos del
problema de construcción de los paneles, pero que, por otro lado, representan conceptos matemáticos
que, en su interrelación, plantean un problema ``matemático'' que, al ser resuelto, permita ofrecer
una solución al problema de los paneles solares. El modelo se puede resumir de la siguiente manera:
\marcojc{.92}{1.5}{black}{black}{white}{%
\begin{center}
\textbf{Modelo para el problema de los paneles solares}
\end{center}
Sean
  \eil\eijc{-.5}\\
    {\setlength\tabcolsep{3pt}
    \begin{tabular}{r p{0.9\textwidth}}
    $x:$ & \textsl{el número de metros que mide el ancho real de un panel y que no puede ser
    mayor que tres metros.}\\
    $a:$ & \textsl{el número de metros cuadrados que mide el área real de un panel cuyo ancho
    y largo miden $x$ y $2x$ metros, respectivamente.}
  \end{tabular}}
  \eil\\
$A$ es un función de $]0,3]$ en $\mathbb{R}^+$ definida por
\[
A(x) = a = 2x^2.
\]
Se busca un número $\delta > 0$ tal que si la desigualdad
\[
\tag{\ref{eqLim002}} |x - 2| < \delta
\]
fuera verdadera, la desigualdad
\[
\tag{\ref{eqLim003}} |A(x) - 8| = |2x^2 - 8| < 0.08
\]
también sería verdadera.%
}

\subsection{El problema matemático}
El modelo plantea un problema exclusivamente matemático:
\marcojc{.9}{1.5}{black}{black}{white}{%
Dada la función $\funjc{A}{]0,3]}{\mathbb{R}}$ definida por
\[
A(x) = 2x^2,
\]
se busca un número real $\delta > 0$ tal que si la desigualdad
\[
\tag{\ref{eqLim002}}%
|x - 2| < \delta
\]
fuera verdadera, la desigualdad
\[
\tag{\ref{eqLim003}}%
|A(x) - 8| = |2x^2 - 8| < 0.08
\]
también sería verdadera.}%
En esta formulación, el significado de los símbolos que aparecen ($A$, $x$ y $\delta$) carece de
importancia. Solo importa que estos símbolos representan, respectivamente, una función, cualquier
número real en el intervalo $]0,3]$ y un número real positivo.

El reto en este momento es resolver este problema. Si lo podemos hacer, usaremos la solución
encontrada para responder la pregunta que se formularon los habitantes de Intipamba: ?`cuál es la
calibración de la máquina que corta los paneles solares para garantizar que el área de los mismos
no difiera de los ocho metros cuadrados en más del uno por ciento?

\subsection{Solución del problema matemático}
Podemos aplicar el método aprendido en la sección anterior para encontrar el número $\delta$. Sin
embargo, para resolver este problema, por tratarse de $2x^2$, podemos hacer un atajo en el
método.

Para empezar, tenemos que:
\begin{align*}
|2x^2 - 8| &= |2(x^2 - 4)| \\
&= 2|x^2 - 4| \\
&= 2|(x - 2)(x + 2)| \\
&= 2|x-2||x+2|.
\end{align*}
Es decir,
\begin{equation}
\label{eqLim014}%
|2x^2 - 8| = 2|x-2||x+2|.
\end{equation}

Ahora debemos acotar $g(x) = 2|x+2|$. Para ello, recordemos que $x$ representa un número que
pertenece al dominio de la función $A$, es decir, $x$ está en el intervalo $]0,3]$. Por lo tanto,
se verifican las desigualdades siguientes:
\[
0 < x \leq 3.
\]
Ahora, si sumamos 2 a cada lado de estas desigualdades, obtendremos lo siguiente:
\begin{equation}
\label{eqLim005}%
2 < x + 2 \leq 5.
\end{equation}
Como $x + 2$ es mayor que cero, entonces,
\[
x + 2 = |x + 2|,
\]
de donde la desigualdad de la derecha en~(\ref{eqLim005}) se escribe así:
\begin{equation}
\label{eqLim006}%
|x + 2| \leq 5.
\end{equation}
Por lo tanto:
\[
g(x) \leq 10.
\]

Entonces, de la igualdad~(\ref{eqLim014}), concluimos que:
\begin{equation}
\label{eqLim007}%
|2x^2 - 8| \leq 10|x - 2|.
\end{equation}
para todo $x\in\ (0,3]$.

Ahora, si para el número $\delta$ que buscamos se cumple (\ref{eqLim002}):
\[
|x - 2| < \delta,
\]
entonces:
\begin{equation*}
|2x^2 - 8| < 10\delta,
\end{equation*}
para $x\in\ (0,3]$.

Por lo tanto, para que se verifique la desigualdad~(\ref{eqLim003}), podemos elegir $\delta$ de
modo que:
\[
10\delta = 0.08.
\]
Y de aquí, obtendríamos que
\[
\delta = \frac{0.08}{10} = 0.008.
\]

Si este es el valor del número $\delta$ que estamos buscando, estaremos seguros de lo siguiente: si
$x\in ]0, 3]$ tal que se verifica la desigualdad~(\ref{eqLim002}):
\[
|x - 2| < \delta,
\]
se cumplirá también la desigualdad~(\ref{eqLim003}):
\[
|2x^2 - 8| < 10\delta = 10\times 0.008 = 0.08.
\]
Y esto es precisamente lo que buscábamos.

Antes de seguir, observemos que hay más de un valor para $\delta$ que satisface la condición del
problema. En efecto, si en lugar de elegir $\delta = 0.008$, lo elegimos de tal manera que
\[
10\delta < 0.008,
\]
es decir, tal que
\[
\delta < \frac{0.08}{10} = 0.008,
\]
también se cumple la condición del problema: si $x \in\ ]0, 3]$ tal que se verifica la
desigualdad~(\ref{eqLim002}):
\[
|x - 2| < \delta,
\]
se cumplirá también la desigualdad~(\ref{eqLim003}):
\[
|2x^2 - 8| < 10\delta < 10\times 0.008 = 0.08,
\]
de donde
\[
|2x^2 - 8| < 0.08.
\]
Es decir, todo los números positivos menores que $0.008$ aseguran el cumplimiento de la
desigualdad~(\ref{eqLim003}).

El problema matemático ha sido resuelto. La solución puede resumirse así:%
\marcojc{.9}{1.5}{black}{black}{white}{%
si se elige el número $\delta > 0$ tal que
\[
\delta \leq 0.008,
\]
entonces, si $x\in\ ]0,3]$ y $|x - 2| < \delta$, necesariamente se cumplirá la siguiente
desigualdad:
\[
|2x^2 - 8| < 0.08.
\]
\eijc{-1.5} }

Observemos también que hemos hecho algo más que resolver este problema particular. Si en lugar de
$0.08$, tuviéramos cualquier número positivo $\epsilon$, lo que hemos demostrado es lo siguiente:
\begin{quote}
{\bfseries dado $\bm{\epsilon > 0}$, el número
\[
\bm{\delta = \frac{\epsilon}{10}}
\]
es tal que, si $\bm{x\in\ ]0,3]}$ y $\bm{|x - 2| < \delta}$, se verifica que
\[
\bm{|2x^2 - 8| < \epsilon}.
\]
}

En otras palabras, el número $8$ es el límite de $2x^2$ cuando $x$ se aproxima a $2$.
\end{quote}

\subsection{Solución del problema}
A partir de la solución matemática, es decir, de la solución del problema matemático, vamos a
proponer una solución al problema de los constructores de los paneles solares. Para ello, debemos
\textit{interpretar} los elementos matemáticos del problema, para lo cual debemos recordar el
significado que estos elementos tienen.

La letra $x$ representa el valor de la longitud del ancho de cualquier panel medida en metros. La
desigualdad
\[
|x - 2| < \delta
\]
expresa la cota superior admisible para el error absoluto en la longitud del ancho del panel.
Finalmente, la desigualdad
\[
|2x^2 - 8| < 0.08
\]
indica la cota superior admisible para el error absoluto en el área del panel solar cuyo ancho mide
$x$ metros y cuyo largo es $2x$ metros.

De la solución matemática, podemos asegurar que, si se elige $\delta < 0.008$, es decir, si se
asegura que el error cometido en el corte del ancho es menor que 8 milímetros, el área obtenida
para ese panel no difiere de ocho metros cuadrados en más de 0.08 metros cuadrados. En resumen:
\marcojc{.9}{1.5}{black}{black}{white}{%
\textsl{si se calibra la máquina cortadora de paneles de modo que se asegure que el error absoluto
cometido en el corte del largo de cada panel no supere los 8 milímetros, entonces se asegurará que
el área obtenida para cada panel no diferirá de ocho metros cuadrados en más del uno por ciento}.}

\subsection{Epílogo}
Luego de resolver el problema, el equipo encargado de la fabricación de los paneles procedió a la
calibración de la máquina cortadora. En las especificaciones del equipo, se indicaba que podía ser
calibrado para que el error absoluto en el corte de una cierta longitud no supere los 5 milímetros.
Dado que 5 es menor que 8, los constructores de los paneles solares procedieron a realizar dicha
calibración; estaban seguros que con ella el área de cada panel satisfaría las especificaciones
técnicas para asegurar el correcto funcionamiento de la planta de energía solar.

La comunidad entera trabajó para completar la construcción de la planta. Luego de mucho esfuerzo,
Intipamba ya tiene energía eléctrica para satisfacer las necesidades de su gente. El esfuerzo
hubiera sido mayor, si la gente de Intipamba no hubiera resuelto el problema, con lo que su
adelanto se habría detenido, y si no se hubieran sentado antes a pensar en el problema y buscar una
solución que, con la ayuda de la poderosa matemática, encontraron.

\subsection{Ejercicios}
\begingroup
\small
\begin{multicols}{2}
\begin{enumerate}[leftmargin=*]
\item PRODAUTO es una empresa productora de automóviles. La gerencia ha establecido que el
    costo de producción mensual, expresado en miles de dólares, de n vehículos puede ser
    obtenido aproximadamente a través de la siguiente fórmula:
    \[
    C = 2 000 + 11n - 0.000\,12n^2.
    \]
    Actualmente, el nivel de producción mensual es de 5 000 vehículos. El equipo de la gerencia
    ha previsto que, dadas las condiciones variables del mercado, habrán variaciones del nivel
    de producción. Esto significa que los correspondientes costos de producción variarán.

    Luego de un análisis de los flujos de caja y teniendo en cuenta la predisposición de los
inversionistas, se ha establecido que las variaciones del monto que se puede destinar a
financiar los costos de producción deberán ser menores a 4 millones de dólares por mes. Como la
planificación del trabajo, compra de insumos, planes de ventas, entre otras actividades de
PRODAUTO dependen del nivel de producción, el equipo de la gerencia debe establecer un margen
para el nivel de producción actual, sabiendo que este nivel no puede variar en más de 500 autos
mensualmente. Determinar el margen para el nivel de producción actual.

\item Los constructores de una central hidroeléctrica de reserva han establecido que la
    relación existente entre la altura $h$ metros del espejo de agua del reservorio respecto al
    nivel mínimo de operación y la energía acumulada $E$ Mwh que se puede generar si se activan
    una o más turbinas, según las necesidades planteadas por la demanda, viene dada por la
    siguiente fórmula:
    \[
    E = 75h + 10h^2 - 0.2h^3.
    \]
    A pesar de que el nivel máximo es de 35 m, han recomendado mantener el nivel en valores
    cercanos a 25 metros.

    En la época de estiaje es recomendable tener disponible la energía acumulada de este
    embalse, pero, al ser necesaria la operación de la central hidroeléctrica, se ha decidido
    permitir que las variaciones de la reserva de energía diarias sean menores a 500 MWh. Una
    manera simple de controlar el cumplimiento de esta restricción es evitar variaciones muy
    grandes del nivel de agua, de modo que se garantice que se respetarán las restricciones a
    las variaciones de la reserva de energía. Determinar la variación de la altura del agua
    permitida para asegurar que la variación de la energía disponible esté siempre por debajo
    de los 500 MWh.
\end{enumerate}
\end{multicols}
\endgroup

\section{Propiedades de los límites}
La definición de límite nos permite verificar si un número es el límite o no de una función. Sin
embargo, no nos permite encontrar el límite de una función. Las propiedades de los límites que
vamos a estudiar en esta sección son, en realidad, un conjunto de reglas de cálculo de límites a
partir del conocimiento de los límites de algunas funciones.

Por ejemplo, demostramos que el límite de una constante es la misma constante y que $a$ es el
límite de $x$ cuando $x$ se aproxima al número $a$. Probaremos más adelante que, cuando $x$ se
aproxima al número $a$, si $L$ es el límite de $f(x)$ y $M$ es el límite de $g(x)$, entonces $LM$
es el límite de $f(x)g(x)$. De estos tres resultados, podremos concluir que---sin más trámite que
su aplicación inmediata---, para una constante cualquiera $k$, el número $ka^2$ es el límite de
$kx^2$ cuando $x$ se aproxima al número $a$.

Vamos a ver que un conjunto relativamente pequeño de reglas de cálculo nos permitirán obtener los
límites de una gran variedad de funciones.

\begin{teocal}[Unicidad del límite]\label{teol:Unicidad} Si existe $\limjc{f(x)}{x}{a}$, entonces hay un único número $L$ tal que
\[
L = \limjc{f(x)}{x}{a}.
\]
\end{teocal}

Los dos teoremas siguientes dan las herramientas necesarias para poder calcular el límite de
cualquier función racional
\[
f(x) = \frac{P(x)}{Q(x)},
\]
donde $P$ y $Q$ son polinomios, cuando $x$ se aproxima al número $a$ y este número no es un cero de
$Q$.

El primer teorema ya lo demostramos. La prueba del anterior y de la mayoría de los siguientes, se
presentarán en el siguiente capítulo.

\begin{teocal}[Límites básicos]\label{teol:ConstanteIdentidad} Sean $a\in\mathbb{R}$ y
$k\in\mathbb{R}$. Supongamos que $k$ no depende de $x$. Entonces:
\begin{enumerate}
\item \emph{Límite de una constante}: $\displaystyle{\limjc{k}{x}{a} = k}$.

   Es decir, la función constante es continua.

\item \emph{Límite de la función identidad}: $\displaystyle{\limjc{x}{x}{a} = a}$.

   Es decir, la función identidad es continua.
\end{enumerate}\end{teocal}

El siguiente teorema muestra que el límite de una función ``preserva'' las operaciones de suma,
producto e inverso multiplicativo de funciones, pues el límite de la suma, del producto y del
inverso multiplicativo siempre es igual a la suma, producto e inverso multiplicativo de los
límites, siempre y cuando estos límites existan y, en el caso del inverso multiplicativo, el límite
sea diferente de cero.

\begin{teocal}[Propiedades algebraicas]\label{teol:Algebra} Si existen $L = \displaystyle
\limjc{f(x)}{x}{a}$ y $M = \displaystyle \limjc{g(x)}{x}{a}$, entonces:
\begin{enumerate}
\item \emph{Límite de la suma}: existe el límite de $f(x) + g(x)$ y:
\[
    L + M = \limjc{(f(x)+ g(x))}{x}{a}.
\]
\item \emph{Límite del producto}: existe el límite de $f(x)g(x)$ y:
\[
LM = \limjc{(f(x)g(x))}{x}{a}.
\]
\item \emph{Límite del inverso multiplicativo}: si $M\neq 0$, existe el límite de
    $\displaystyle{\frac{1}{g(x)}}$ y:
\[
\frac{1}{M} = \limjc{\frac{1}{g(x)}}{x}{a}.
\]
\end{enumerate}
\end{teocal}

Por ejemplo, si $a\neq 0$, por la tercera parte de este teorema, podemos afirmar que
\[
\frac{1}{a} = \limjc{\frac{1}{x}}{x}{a},
\]
pues, de la segunda parte del teorema del límite de la función identidad
(teorema~\ref{teol:ConstanteIdentidad}), podemos asegurar que
\[
a = \limjc{x}{x}{a}.
\]

Más adelante, calcularemos una gran variedad de límites con el uso de estos teoremas. Ahora hagamos
la demostración del primer numeral.

\begin{proof}[Demostración]
Sea $\epsilon > 0$. Debemos probar que existe un número
    $\delta > 0$ tal que
\begin{equation}
\label{eq:pl003}
|(f(x) + g(x)) - (L + M)| < \epsilon,
\end{equation}
siempre que
\[
0 < |x - a| < \delta\yjc x\in\Dm(f+g) = \Dm(f)\cap\Dm(g).
\]

Para todo $x\in\Dm(f+g)$, tenemos que:
\begin{align*}
|(f(x) + g(x)) - (L + M)| &= |(f(x) - L) + (g(x) - M)| \\
&\leq |f(x) - L| + |g(x) - M|.
\end{align*}
Por lo tanto, para hacer tan pequeño como queramos a
\[
|(f(x) + g(x)) - (L + M)|
\]
es suficiente que hagamos tan pequeño como queramos a
\[
|f(x) - L| \yjc |g(x) - M|.
\]
para el $x$ adecuado.

Ahora bien, estas dos diferencias sí pueden hacerse tan pequeñas como se quiera, pues $L$ es el
límite de $f(x)$ y $M$ el de $g(x)$ cuando $x$ se aproxima al número $a$. De manera más
precisa, para obtener la desigualdad~(\ref{eq:pl003}), es suficiente encontrar el $x$ adecuado
para que
\[
|f(x) - L| \yjc |g(x) - M|
\]
sean, cada uno, menores que $\frac{\epsilon}{2}$. Procedamos.

Como $L$ es el límite de $f(x)$ y $\frac{\epsilon}{2} > 0$, existe $\delta_1 > 0$ tal que
\begin{equation}
\label{eq:pl004}
|f(x) - L| < \frac{\epsilon}{2},
\end{equation}
siempre que
\[
0 < |x - a| < \delta_1 \yjc x\in\Dm(f).
\]
Además, por ser $M$ el límite de $g(x)$ y $\frac{\epsilon}{2} > 0$, existe $\delta_2 > 0$ tal
que
\begin{equation}
\label{eq:pl005}
|g(x) - M| < \frac{\epsilon}{2},
\end{equation}
siempre que
\[
0 < |x - a| < \delta_2 \yjc x\in\Dm(g).
\]

?`Qué $x$ satisfará simultáneamente las desigualdades~(\ref{eq:pl004}) y (\ref{eq:pl005})? Aquel
que satisfaga simultáneamente las siguientes condiciones:
\[
0 < |x - a| < \delta_1, \quad 0 < |x - a| < \delta_2, \quad x\in\Dm(f)\yjc x\in \Dm(g).
\]
?`Cómo podemos elegir, entonces, el número $\delta$? Así:
\[
\delta = \min\{\delta_1,\delta_2\},
\]
?`Y cómo elegir $x$? Así:
\[
0 < |x - a| < \delta\yj x\in\Dm(f)\cap \Dm(g).
\]
Tenemos entonces que:
\begin{align*}
|(f(x) + g(x)) - (L + M)| &\leq |f(x) - L| + |g(x) - M| \\
< \frac{\epsilon}{2} + \frac{\epsilon}{2} = \epsilon.
\end{align*}
En otras palabras, $L + M$ es el límite de $f(x) + g(x)$ cuando $x$ se aproxima al número $a$.

\end{proof}

De este teorema, es consecuencia inmediata el siguiente corolario.

\begin{corocal}[Propiedades algebraicas]\label{cor:limPropiedadesAlgebraicas}%
Sean $\alpha\in\mathbb{R}$, $\beta\in\mathbb{R}$. Si
existen
\[
L = \limjc{f(x)}{x}{a} \yjc M = \limjc{g(x)}{x}{a},
\]
entonces:
\begin{enumerate}
\item \emph{Límite del producto de un escalar por una función}: existe el límite de $\alpha
    f(x)$ y:
    \[
        \alpha L = \limjc{(\alpha f(x))}{x}{a}.
    \]
\item \emph{Límite de la resta}: existe el límite de $f(x) - g(x)$ y:
    \[
        L - M = \limjc{(f(x) - g(x))}{x}{a}.
    \]
\item \emph{Límite de una combinación lineal}: existe el límite de $\alpha f(x) + \beta g(x)$
    y:
    \[
        \alpha L + \beta M = \limjc{(\alpha f(x) + \beta g(x))}{x}{a}.
    \]
\item \emph{Límite del cociente}: existe el límite de $\frac{f(x)}{g(x)}$, siempre que $M\neq
    0$. Entonces:
    \[
        \frac{L}{M} = \limjc{\frac{f(x)}{g(x)}}{x}{a}.
    \]

\end{enumerate}
\end{corocal}

De este corolario, y su correspondiente teorema, se deduce que las operaciones algebraicas entre
funciones continuas producen funciones continuas. De manera precisa:

\begin{teocal}[Propiedades algebraicas de la continuidad]%
Si $\funcjc{f}{\Dm(f)}{\mathbb{R}}$ y $\funcjc{g}{\Dm(g)}{\mathbb{R}}$ son continuas en $a$.
Entonces
\begin{quote}
la \emph{suma} $(f + g)$, la \emph{resta} $(f - g)$, el \emph{producto} $(fg)$ y el \emph{cociente}
$(f/g)$ (siempre que $g(a) \neq 0$)
\end{quote}
son funciones continuas en a.
\end{teocal}

\begin{exemplo}[Solución]{%
Calcular
\[
\limjc{\frac{x + 3}{x - 5}}{x}{2}.
\]
}%
Sea $\funcjc{F}{\mathbb{R}}{\mathbb{R}}$ definida por
\[
F(x) = \frac{x + 3}{x - 5}.
\]
Entonces, el límite pedido es:
\[
\limjc{F(x)}{x}{2}.
\]

La función $F$ puede ser expresada como el cociente de las funciones
$\funcjc{\varphi}{\mathbb{R}}{\mathbb{R}}$ y $\funcjc{\psi}{\mathbb{R}}{\mathbb{R}}$, definidas por
\[
\varphi(x) = x + 3 \yjc \psi(x) = x - 5.
\]
Es decir,
\[
F(x) = \frac{\varphi(x)}{\psi(x)}.
\]
Entonces, para calcular el límite de $F(x)$, podemos utilizar el teorema del ``límite del
cociente'', es decir, el corolario~\ref{cor:limPropiedadesAlgebraicas} del teorema
\ref{teol:Algebra}. Para poder hacerlo, tenemos que constatar que se verifiquen las condiciones de
este teorema. Estas son tres:
\begin{enumerate}
\item Existe el límite
$\displaystyle
\limjc{\varphi(x)}{x}{2}$.

\item Existe el límite
$\displaystyle
\limjc{\psi(x)}{x}{2}$.

\item Si el límite anterior existe, éste debe ser diferente de cero.
\end{enumerate}
Verifiquemos que se cumplen estas tres condiciones.
\begin{enumerate}
\item La función $\varphi$ es la suma de la función identidad y una función constante. Entonces
    su límite existe y es igual a:
   \[
    \limjc{\varphi(x)}{x}{2} = \limjc{x}{x}{2} + \limjc{3}{x}{2} = 2 + 3 = 5.
   \]
\item La función $\psi$ es la resta de la función identidad y una función constante. Entonces
    su límite existe y es igual a:
   \[
    \limjc{\psi(x)}{x}{2} = \limjc{x}{x}{2} - \limjc{5}{x}{2} = 2 - 5 = -3.
   \]
   Como se puede observar, el límite de la función $\psi$ es diferente de $0$.
\end{enumerate}

Podemos, entonces, aplicar el teorema del ``límite del cociente''. Éste afirma que el límite del
cociente de dos funciones cuyo límites existen y el del denominador es diferente de $0$, es igual
al cociente de ambos límites. Por lo tanto, podemos proceder de la siguiente manera:
\begin{align*}
\limjc{\frac{\varphi(x)}{\psi(x)}}{x}{2} &= \frac{\displaystyle\limjc{\varphi(x)}{x}{2}}{\displaystyle\limjc{\psi(x)}{x}{2}} \\
   &= \frac{5}{-3}.
\end{align*}
Es decir:
$\displaystyle
\limjc{\frac{x + 3}{x - 5}}{x}{2} = -\frac{5}{3}$.
\end{exemplo}

\begin{exemplo}[Solución]{%
Calcule $\displaystyle\limjc{\frac{s^2 - 21}{s + 2}}{s}{7} $ si existe}%
Para aplicar el teorema del límite del cociente necesitamos saber si existen los límites de las
funciones numerador y denominador, y si el límite del denominador es distinto de $0$. Empecemos con
el numerador:
\begin{align*}
\limjc{(s^2 - 21)}{s}{7} &= \limjc{s^2}{s}{2} + \limjc{(-21)}{s}{2} \\
&= \limjc{s}{s}{2} \times \limjc{s}{s}{2} - 21 \\
&= 7 × 7 - 21 = 28.
\end{align*}
Ahora el límite del denominador:
\begin{align*}
\limjc{(s + 2)}{s}{7} &= \limjc{s}{s}{7} + \limjc{2}{s}{7}\\
  &= 7 + 2 = 9 \neq 0.
\end{align*}
Podemos, entonces, aplicar el teorema del límite del cociente:
\begin{align*}
\limjc{\frac{s^2 - 21}{s + 2}}{s}{7} &= \frac{\displaystyle\limjc{(s^2 - 21)}{s}{7}}{\displaystyle\limjc{(s + 2)}{s}{7}} = \frac{28}{9}.
\end{align*}
\end{exemplo}

\begin{exemplo}[Solución]{%
Calcule $\displaystyle\limjc{(2t + 3)(3t^2 - 5)}{t}{1}$ si existe.}%
\begin{align*}
\limjc{(2t + 3)(3t^2 - 5)}{t}{1} &= \left(\limjc{(2t + 3)}{t}{1}\right)\left(\limjc{(3t^2 -
5)}{t}{1}\right) \\
  &= \left(\limjc{2t}{t}{1} + \limjc{3}{t}{1}\right)\left(\limjc{3t^2}{t}{1} + \limjc{(-5)}{t}{1}\right)\\
  &= \left(2\limjc{t}{t}{1} + \limjc{3}{t}{1}\right)\left(3\limjc{t^2}{t}{1} + \limjc{(-5)}{t}{1}\right)\\
  &= \left(2\limjc{t}{t}{1} +
  \limjc{3}{t}{1}\right)\left(3\left[\limjc{t}{t}{1}\right]\left[\limjc{t}{t}{1}\right] +
  \limjc{(-5)}{t}{1}\right) \\
  &= (2\cdot 1 + 3)(3[1\cdot 1] + (-5)) \\
  &= (2 + 3)(3 - 5) = -10.
\end{align*}
Entonces
\[
\limjc{(2t + 3)(3t^2 - 5)}{t}{1} = -10.
\]
\end{exemplo}

\begin{exemplo}[Solución]{%
Calcule $\displaystyle\limjc{y\left(\frac{4}{y} - 1\right)}{y}{0}$ si existe.}%
Si intentamos calcular el límite como producto de límites no podríamos hacerlo, ya que no tenemos
aún a disposición ninguna herramienta que nos diga como obtener el límite del segundo factor. En
situaciones como ésta, es necesario efectuar una manipulación algebraica previa al cálculo del
límite:
\begin{align*}
\limjc{y\left(\frac{4}{y} - 1\right)}{y}{0}
  &=\limjc{\left(\frac{4}{y}y - y\right)}{y}{0}\\
  &= \limjc{(4 - y)}{y}{0} = 4 - 0 = 4.
\end{align*}
De donde:
\[
\limjc{y\left(\frac{4}{y} - 1\right)}{y}{0} = 4.
\]
\end{exemplo}

\subsection{Ejercicios}
\begingroup
\small
\begin{multicols}{2}
\begin{enumerate}[leftmargin=*]
\item En los siguientes ejercicios, calcule el límite dado si él existe. De ser necesario,
    realice primero una manipulación algebraica.
\begin{enumerate}[leftmargin=*]
\item $\displaystyle\limjc{(2x^2 - 3x + 2)}{x}{-2}$.
\item $\displaystyle\limjc{\frac{-x^2 + x - 1}{-3x + 1}}{x}{2}$.
\item $\displaystyle\limjc{\frac{-x + 3}{2x - 7}}{x}{3}$.
\item $\displaystyle\limjc{\frac{t^3 - 1}{t - 1}}{t}{1}$.
\item $\displaystyle\limjc{\frac{3z + 9}{36 - 4z^2}}{z}{-3}$.
\item $\displaystyle\limjc{\left(x - \frac{1}{x - 1}\right)}{x}{0}$.
\item $\displaystyle\limjc{\frac{t^3+1}{t^2 - 1}}{t}{-1}$.
\item $\displaystyle\limjc{2x^3(-x^4 + 3x^3)^{-1}}{x}{0}$.
\item $\displaystyle\limjc{\left(\frac{1}{x} + \frac{2x^2 - 5x - 1}{x}\right)}{x}{0}$.
\item $\displaystyle\limjc{(ar^2 - br)^3}{r}{1}$ con $a$ y $b$ constantes. No desarrolle el
    cubo.
\item $\displaystyle\limjc{\frac{1}{h}\left((x + h)^3 - x^3\right)}{h}{0}$.
\end{enumerate}

\item Calcule los límites dados usando las propiedades de los límites. Indique qué propiedades
    usó.
\begin{enumerate}[leftmargin=*]
\item $\displaystyle
	\lim_{x\to 3}7.9.
$
\item $\displaystyle
	\lim_{x\to 5}(4x-8).
$
\item $\displaystyle
	\lim_{x\to 2}(x^5-7x+1).
$
\item $\displaystyle
	\lim_{x\to 0}\dfrac{3x-7}{7x^2+8}.
$
\end{enumerate}

\end{enumerate}
\end{multicols}
\endgroup

\subsection{Generalizaciones}
Los teoremas y el corolario sobre las propiedades algebraicas de los límites y de la continuidad
están formulados para las propiedades de dos funciones: la suma de dos funciones, el producto,
etcétera. En el siguiente teorema se generalizan los anteriores para cualquier número de funciones.

\begin{corocal}[Generalización de las propiedades algebraicas]\label{teol:AlgebraGeneralizada}%
Sean $n\in\mathbb{N}$, $f_1, f_2, \ldots, f_n$ tales que
\[
L_i = \limjc{f_i(x)}{x}{a} \yjc M_i = \limjc{g_i(x)}{x}{a}
\]
para todo $i$ tal que $1\leq i \leq n$. Entonces:
\begin{enumerate}
\item \label{teol:SumaGeneralizada} $\displaystyle{\limjc{\sum_{i=1}^nf_i(x)}{x}{a} =
    \sum_{i=1}^n \limjc{f_i(x)}{x}{a} = \sum_{i=1}^n L_i}.$
\item \label{teol:ProductoGeneralizado} $\displaystyle{\limjc{\prod_{i=1}^nf_i(x)}{x}{a} =
    \prod_{i=1}^n \limjc{f_i(x)}{x}{a} = \prod_{i=1}^n L_i}.$
\item Si $f_i$ es continua en $a$ para todo $i$ tal que $1\leq i \leq n$, entonces
    $\sum_{i=1}^nf_i$ y $\prod_{i=1}^nf_i$ son continuas en $a$.
\end{enumerate}
\end{corocal}

La demostración es sencilla si se aplica el método de inducción matemática sobre $n$.

Ahora es fácil probar que los polinomios y las funciones racionales son continuas:

\begin{teocal}[Continuidad de un polinomio y una función
racional]\label{teol:PolinRacionContinuas}%
Todo polinomio es una función continua en $\mathbb{R}$ y toda función racional es continua en
$\mathbb{R}$, excepto en aquellos números en los que el denominador es igual a cero. De manera más
precisa: si $P$ y $Q$ son dos polinomios, entonces:
\begin{enumerate}
\item \label{teol:PolinContinua} $\displaystyle{\limjc{P(x)}{x}{a} = P(a).}$
\item \label{teol:RacionContinua} $\displaystyle{\limjc{\frac{P(x)}{Q(x)}}{x}{a} =
    \frac{P(a)}{Q(a)}}$, siempre que $Q(a) \neq 0$.
\end{enumerate}
\end{teocal}

Del límite del producto generalizado, se obtiene que si $f(x)$ tiene $L$ como límite cuando $x$ se
aproxima al número $a$, entonces $L^n$ es el límite de $f^n(x)$. Esto también es verdadero para el
caso de la raíz $n$-ésima, aunque la demostración de esta propiedad, que no se deriva del teorema
anterior, no es elemental.

\begin{teocal}[Límite de la raíz $n$-ésima]\label{teol:RaizGeneralizada}%
Si $L = \displaystyle{\limjca{f(x)}}$, entonces para todo $n\in\mathbb{N}$ impar, se verifica que
\[
\limjca{\sqrt[n]{f(x)}} = \sqrt[n]{\limjca{f(x)}} = \sqrt[n]{L}.
\]
Si $n$ es par, es necesario que $L > 0$.
\end{teocal}

Para la demostración de este teorema es necesario antes demostrar que la composición de funciones
también preserva el límite y la continuidad, pues, probaremos que la función $h_n$, definida por
\[
h_n(x) = \sqrt[n]{x}
\]
es continua en $\mathbb{R}$ si $n$ es impar, mientras que es continua en $[0,+\infty[$ si $n$ es
par. Eso lo haremos en el siguiente capítulo. Ahora veamos algunos ejemplos de límites que
involucran raíces.

\begin{exemplo}[Solución]{%
Calcule $\displaystyle \lim_{x\to 4}\sqrt{2x^2 -7}$ si existe.
}%
El límite pedido existirá si existe el $\displaystyle \lim_{x\to 4}(2x^2 -7)$ y si éste es
positivo. Entonces, calculemos primero dicho límite:
\begin{align*}
\lim_{x\to 4}(2x^2 -7) &= 2\lim_{x\to 4}x^2 + \lim_{x\to 4}(-7) \\
&= 2(4)(4)-7 = 25>0.
\end{align*}
Entonces, el límite buscado existe y se lo calcula así:
\begin{align*}
\lim_{x\to 4}\sqrt{2x^2 -7} &= \sqrt{\lim_{x\to 4}(2x^2 -7)} \\
&= \sqrt{25}=5.
\end{align*}
Por lo tanto:
$\displaystyle
\lim_{x\to 4}\sqrt{2x^2 -7}=5$.
\end{exemplo}

\begin{exemplo}[Solución]{%
Calcule $\displaystyle \lim_{x\to 3}\sqrt[3]{x -7}$ si existe.
}%
A diferencia del ejercicio anterior, y por ser raíz impar, basta que exista el $\displaystyle
\lim_{x\to 3}(x -7)$. Como
\[
\lim_{x\to 3}(x-7)=-4,
\]
entonces
\begin{align*}
\lim_{x\to 3}\sqrt[3]{x -7} &= \sqrt[3]{\lim_{x\to 3}(x-7)} \\
&= \sqrt[3]{-4} = -\sqrt[3]{4}.
\end{align*}
Por lo tanto:
$\displaystyle
\lim_{x\to 3}\sqrt[3]{x -7}=-\sqrt[3]{4}$.

\end{exemplo}

\subsection{Ejercicios}
\begingroup
\small
Si los límites existen, calcularlos:
\begin{multicols}{2}
\begin{enumerate}[leftmargin=*]
\item $\displaystyle \lim_{x\to 10}\sqrt{\frac{10x}{2x+5}}$.
%
\item $\displaystyle \lim_{x\to 2}\frac{\sqrt[3]{x^2-10}}{\sqrt{x^3-3}}$.
%
\item $\displaystyle \lim_{t\to 2}(t+2)^\frac{3}{2}(2t+4)^\frac{1}{3}$.
%
\item $\displaystyle \lim_{h\to 0}\frac{\sqrt{x+h}-\sqrt{x}}{h}, x>0$.
%
\item $\displaystyle \lim_{t\to 1}\frac{\sqrt{t}-1}{t-1}$.
%
\item $\displaystyle
	\lim_{x\to 1}\sqrt{\dfrac{2x^2-x+3}{x^2+x+1}}.
$
%
\item $\displaystyle
	\lim_{x\to 2}\sqrt[3]{\dfrac{3x-5}{2x^2-x+1}}.
$
%
\item $\displaystyle \limjc{\frac{\sqrt[4]{x^3 - 3x + 2}}{\sqrt[3]{1 - 2x + x^4}}}{x}{1}$.
\end{enumerate}
\end{multicols}
\endgroup

\section{Continuidad de funciones localmente iguales}

Una consecuencia inmediata del teorema del límite de funciones localmente iguales es el siguiente:

\begin{teocal}[Continuidad de funciones localmente iguales]
Sean:
\begin{itemize}
\item[] $a$ un número real;
\item[] $I$ un intervalo abierto que contiene al número $a$;
\item[] $f$ y $g$ dos funciones definidas en $I$ (es decir, $I\subset \Dm(f)$; e $I\subset \Dm(g)$).
\end{itemize}
Si $f(a)=g(a)$ y si $f=g$ cerca de $a$, entonces $f$ es continua en $a$ si y solo si $g$ es continua en $a$.
\end{teocal}%

Este teorema tiene como corolario el muy útil teorema siguiente:

\begin{teocal}[Continuidad de funciones iguales en un intervalo abierto]
Sean $I$ un intervalo abierto; y, $f$ y $g$ dos funciones reales que son iguales en $I$ (es decir que para todo $x\in I$, $f(x)=g(x)$, que es lo mismo, que $f|_{I}=g|_{I}$).

Entonces $f$ es continua en $I$ si y solo si $g$ es continua en $I$.
\end{teocal}%

\begin{exemplo}[Solución]{%
Sea
\[
f(x) = \begin{cases}
g(x) = -2x^2+8x-4 & \text{, si $x\geq 1$,}\\
h(x) = x^2 + 1 & \text{, si $x<1$.}
\end{cases}
\]
Pruebe que $f$ es continua
}%
\begin{enumerate}
\item[a)] Vemos que $f|_{]-\infty, 1[}=g|_{]-\infty, 1[}$ y como $g$ es continua por ser un polinomio, aplicando el teorema precedente tenemos que $f$ es continua en $]-\infty, 1[$.
\item[b)] Análogamente, como $f|_{]1, +\infty[}=h|_{]1, +\infty[}$ y como $h$ es continua por ser un polinomio, podemos concluir que $f$ es continua en $]1, +\infty[$.
\item[c)] Queda por verificar la continuidad en $1$. Como $f(1)=g(1)=2$ y, teniendo en cuenta lo calculado en el Ejemplo 1.4, se obtiene que
\[
\limjc{f(x)}{x}{1}=2
\]
En consecuencia, $f$ es continua en $1$.
\end{enumerate}
\end{exemplo}


\section{El límite de una composición: cambio de variable}
Más adelante, en este capítulo, probaremos que la función $\sen$ es continua en $\mathbb{R}$. Por
lo tanto, se cumplirá que:
\[
\limjc{\sen x}{x}{0} = \sen 0 = 0.
\]
Por otra parte, sabemos que
\[
\limjc{(x^2 - 1)}{x}{1} = 0.
\]
?`Cómo podemos utilizar estos dos resultados para calcular
\[
\limjc{\sen{(x^2 - 1)}}{x}{1}
\]
si tener que recurrir a la definición de límite?

Como puede verse, este último es el límite de la composición de las dos funciones presentes en los
dos primeros límites. El siguiente teorema, nos dice cómo se puede calcular el límite de una
composición a partir de los límites de las funciones que conforman la composición.

\begin{teocal}[Límite de una composición]
Supongamos que existe el límite $\displaystyle\lim_{x\to a}g(x)= b$ y que $f$ sea continua en $b$. Entonces
existe $\displaystyle\lim_{x\to a}f(g(x))$ y
\[
\lim_{x\to a}f(g(x)) = f\left(\limjc{g(x)}{x}{a}\right) = f(b).
\]
\end{teocal}%

\begin{corocal}[]
La composición de funciones continuas es continua.
\end{corocal}


Ahora, las hipótesis de este teorema no siempre se verifican. Por ejemplo, si queremos calcular
el siguiente límite
\[
\limjc{\frac{\sen(2x)}{x}}{x}{0},
\]
deberemos recurrir al límite
\[
\limjc{\frac{\sen y}{y}}{y}{0} = 1.
\]
Como se puede ver, la función $f$, definida por
\[
f(y) = \frac{\sen y}{y}
\]
para todo $y\neq 0$, no es continua en $0$, pues no está definida en este número. Sin embargo, aún
se puede calcular el límite de una composición si se modifican la hipótesis del teorema anterior.

De manera más precisa, probaremos que la hipótesis de que $f$ sea continua en $b$ puede
\emph{debilitarse}; es decir, puede ser sustituida por una condición que no exige la continuidad de
$f$ en $b$, sino solamente la propiedad de que exista un intervalo alrededor del número $a$ en el
cual, a excepción tal vez de $a$, la función $g$ es diferente de $b$. En los ejemplos que veremos a
continuación, así como en los ejercicios propuestos, es necesario verificar que la función $g$
satisface esta condición en los casos de que la función $f$ no sea continua en $b$.

En la práctica, este teorema se utiliza bajo la siguiente formulación:

\begin{teocal}[Cambio de variable para límites]\label{teo:LimCambioVariable}%
Para calcular el límite de una composición como el siguiente
\[
\limjc{f(g(x))}{x}{a},
\]
se puede usar el \emph{cambio de variable}
$
y = g(x)
$
cuando existan $\displaystyle b = \limjc{g(x)}{x}{a}$ y $\displaystyle L = \limjc{f(y)}{y}{b}$, y siempre que se satisfaga una de las tres condiciones siguientes:
\begin{enumerate}
\item $f$ es continua en $b$;
\item $f$ no está definida en $b$; y,
\item $L \neq f(b)$ y existe $r > 0$ tal que para todo $x \in\ ]a-r,a+r[-\{a\}$, $g(x) \neq b$ (es decir, $g\new b$ cerca de $a$).
\end{enumerate}
Se tiene, entonces, que:
\[
\limjc{f(g(x))}{x}{a} = \limjc{f(y)}{y}{b} = L.
\]
\end{teocal}

El utilizar la función $g$ y la composición con $f$ para calcular el límite de $f(g(x))$ se
denomina \emph{método del cambio de variable}. Veamos cómo se utiliza este teorema en los dos
ejemplos con los que se abre esta subsección.

\begin{exemplo}[Solución]{%
Calcular
\[
\limjc{\sen(x^2 - 1)}{x}{1}.
\]
}%
Sean $f$ y $g$ definidas por $f(x) = \sen x$ y $g(x) = x^2 - 1$. Además, hagamos $a = 1$. Entonces:
\[
\limjc{\sen(x^2 - 1)}{x}{1} = \limjc{f(g(x))}{x}{1}.
\]
Veamos si las funciones $f$ y $g$ satisfacen las condiciones del
teorema~\ref{teo:LimCambioVariable}. En primer lugar, determinemos si el límite de $g(x)$ existe
cuando $x$ tiende a $1$:
\begin{align*}
\limjc{g(x)}{x}{1} &= \limjc{(x^2 - 1)}{x}{1} \\
  &= (\limjc{x}{x}{1})^2 - \limjc{1}{x}{1} \\
  &= 1^2 - 1 = 0.
\end{align*}
Entonces:
\[
0 = \limjc{g(x)}{x}{1}.
\]
Por lo tanto, $b = 0$.

Ahora debemos determinar si $f$ es continua en $0$, o si existen $\displaystyle\limjc{f(y)}{y}{0}$ y un
intervalo centrado en $1$ en el cual $g$ sea diferente de $0$ (salvo, tal vez, en $1$).

Como la función $f$ es la función $\sen$, entonces $f$ es continua en $0$ y, además:
\[
\limjc{f(y)}{y}{0} = \limjc{\sen y}{y}{0} = \sen 0 = 0.
\]

Podemos, entonces, aplicar el teorema~\ref{teo:LimCambioVariable}. Al hacerlo, obtendremos que:
\begin{align*}
\limjc{f(g(x))}{x}{1} &= \limjc{f(y)}{y}{0} \\
  &= \limjc{\sen y}{y}{0} = 0.
\end{align*}
Es decir:
\[
\limjc{\sen(x^2 - 1)}{x}{1} = 0.
\]
\end{exemplo}

Veamos qué sucede en el segundo ejemplo.

\begin{exemplo}[Solución]{%
Calcular
\[
\limjc{\frac{\sen(2x)}{x}}{x}{0}.
\]
}%
Si hacemos el cambio de variable $y = g(x) = 2x$, tenemos que
\[
\frac{\sen(2x)}{x} = \frac{\sen y}{\frac{y}{2}} = 2\frac{\sen y}{y}.
\]
Si definimos $f$ por:
\[
f(y)= 2\frac{\sen y}{y},
\]
lo que debemos calcular es:
\[
\limjc{\frac{\sen(2x)}{x}}{x}{0} = \limjc{f(g(x))}{x}{0}.
\]

Apliquemos el teorema de cambio de variable. En este caso $a = 0$. Busquemos $b$. Para ello,
debemos calcular $\displaystyle\limjc{g(x)}{x}{0}$:
\[
\limjc{g(x)}{x}{0} = \limjc{2x}{x}{0} = 2\limjc{x}{x}{0} = 2\times 0 = 0.
\]
Por lo tanto, $b = 0$.

Para poder aplicar el teorema, nos falta determinar si $f$ o bien es continua en $0$ o si existe su
límite en $b = 0$ y, en un intervalo centrado en $a = 0$, la función $g$ es diferente de $b = 0$
(salvo, tal vez, en $0$).

En primer lugar, sí existe el límite de $f$ en $0$, pues
\[
\limjc{f(y)}{y}{0} = \limjc{2\frac{\sen y}{y}}{y}{0} = 2\limjc{\frac{\sen y}{y}}{y}{0} = 2\times 1 = 2.
\]
Ahora bien, dado que $f$ no está definida en $0$, no puede ser continua en $0$. Es decir, la
primera condición no se verifica, pero si se cumple la segunda.

También $g(x) = 2x$ es diferente de $0$ en todos los puntos de un intervalo centrado en $0$, salvo
en $0$, por lo que, en este caso, también se cumple la tercera condición.

Entonces, el teorema del cambio de variable es aplicable, con lo que obtenemos que:
\begin{align*}
\limjc{\frac{\sen(2x)}{x}}{x}{0} &= \limjc{f(g(x))}{x}{0} \\
&= \limjc{f(y)}{y}{0} = 2.
\end{align*}
Es decir:
\[
\limjc{\frac{\sen(2x)}{x}}{x}{0} = 2.
\]

En este segundo ejemplo, la función $f$ del teorema~\ref{teo:LimCambioVariable} satisface las
condiciones segunda y tercera, pero no la primera. Sin embargo, el límite de este ejemplo puede ser resuelto
aplicando la primera condición si se define $f$ de tal manera que sí sea continua en $0$.

En efecto, si definimos
\[
f(y) =
\begin{cases}
2\frac{\sen y}{y} & \text{si} \ y \neq 0, \\
2 & \text{si} \ y = 0,
\end{cases}
\]
entonces la función $f$ sí es continua en $0$, pues está definida en dicho punto, y su valor allí
es $2$, y el límite de $f$ en $0$ es $2$. Hemos ``extendido o prolongado de una manera continua la función $f$ al punto $0$''.

Obviamente, al aplicar el teorema de cambio de variable a esta $f$, obtenemos el mismo resultado
que con la definición anterior.
\end{exemplo}

En la práctica, muchos cambios de variable se realizan sin explicitar la función $f$. Por ejemplo,
el caso del último ejemplo, se suele proceder así:
\begin{align*}
\limjc{\frac{\sen(2x)}{x}}{x}{0} &= \limjc{2\frac{\sen(2x)}{2x}}{x}{0} \\
&= 2\limjc{\frac{\sen(2x)}{2x}}{x}{0} \\
&= 2\limjc{\frac{\sen(y)}{y}}{y}{0} = 2\times 1 = 2,
\end{align*}
donde $y = 2x$ y $\displaystyle\limjc{y}{x}{0} = \limjc{2x}{x}{0} = 0$.

Si bien se puede proceder como en este ejemplo, el lector debe verificar que se satisfagan las
hipótesis del teorema del cambio de variable, pues, en el caso contrario, sus
conclusiones podrían no son correctas, como nos lo muestra el ejemplo siguiente.

\begin{exemplo}[Solución]{%
Sean
\begin{displaymath}
f(t)=
\begin{cases}
4 & \text{si $t\neq 1$}\\
3 & \text{si $t= 1$},
\end{cases} \yjc
g(x)=
\begin{cases}
1 & \text{si $x\neq 2$}\\
0 & \text{si $x= 2$}.
\end{cases}
\end{displaymath}
Demostrar que
\[
\limjc{f(g(x))}{x}{2} \neq \limjc{f(t)}{t}{1}.
\]
Es decir, el teorema~\ref{teo:LimCambioVariable} del cambio de variable no es aplicable para
calcular el límite de la compuesta de $f$ con $g$ en $2$.
}%
En primer lugar, tenemos que $a = 2$ y, como se verifica que
\[
\limjc{g(x)}{x}{2} = 1,
\]
entonces $b = 1$. Además, si hacemos el cambio $t = g(x)$, entonces
\[
\limjc{f(t)}{t}{1} = 4.
\]

En segundo lugar, tenemos que
\begin{align*}
f(g(x)) &=
\begin{cases}
4 & \text{si} \ g(x) \neq 1 \\
3 & \text{si} \ g(x) = 1
\end{cases}
\\
&=
\begin{cases}
4 & \text{si} \ x = 2 \\
3 & \text{si} \ x \neq 2.
\end{cases}
\end{align*}
Por lo tanto:
\[
\limjc{f(g(x))}{x}{2} = 3.
\]
Y, como $3 \neq 4$, entonces:
\[
\limjc{f(g(x))}{x}{2} \neq \limjc{f(t)}{t}{1}.
\]

Esto significa que el teorema de cambio de variable no es aplicable. ?`Por qué?

Porque aunque $f$ sí está definida en $b = 1$, no es continua en este punto $b = 1$ y $g(x)$ es igual a $b = 1$ en cualquier intervalo centrado en
$a = 2$, salvo en $2$.
\end{exemplo}

Aplicar el teorema del cambio de variable (o cualquier teorema) sin verificar que las hipótesis
requeridas se satisfacen puede conducirnos a un error. Y, aunque no lo hiciera, hacerlo es un
error, porque no sabemos si el resultado obtenido es correcto o no.

Veamos algunos ejemplos adicionales en los que sí se puede aplicar el teorema del cambio de
variable.

\begin{exemplo}[Solución]{%
Sean
\begin{displaymath}
f(x)=
\begin{cases}
3 & \text{si $x\neq 1$}\\
4 & \text{si $x= 1$},
\end{cases} \yjc
g(x)=
\begin{cases}
0 & \text{si $x\neq 2$}\\
1 & \text{si $x= 2$}.
\end{cases}
\end{displaymath}

Utilizar el teorema del cambio de variable, si es aplicable, para mostrar que
\[
\limjc{f(g(x))}{x}{2} = f(\limjc{g(x)}{x}{2})
\]
\eijc{-1}}

En primer lugar, calculemos el límite de la composición sin utilizar el teorema del cambio de
variable. Luego lo aplicamos y miramos que el resultado obtenido es el mismo.

Puesto que
\[
f(g(x))=
\begin{cases}
f(0)&\text{si $x\neq 2$}\\
f(1) & \text{si $x= 2$}
\end{cases}
\quad =\quad
\begin{cases}
3 &\text{si $x\neq 2$}\\
4 & \text{si $x= 2$},
\end{cases}
\]
entonces
\[
\lim_{x\to 2}f(g(x))=\lim_{x\to 2}3=3.
\]

Por otra parte:
\[
\lim_{x\to 2}g(x)=0.
\]
Por lo tanto:
\[
f\left(\lim_{x\to 2}g(x)\right) = f(0) = 3.
\]
Esto prueba que:
\[
\lim_{x\to 2}f(g(x))= f\left( \lim_{x\to 2}g(x)  \right).
\]

Podemos llegar a la misma conclusión al aplicar el teorema de cambio de variable, pues existe el
limite de $f(y)$ cuando $y$ se aproxima a $0$ y es igual a $3$; existe el límite de $g(x)$ cuando
$x$ se aproxima a $2$ y es igual a $0$. Como $f$ es continua en $0$, el teorema es aplicable con $a
= 2$ y $b = 0$.
\end{exemplo}

\begin{exemplo}[Solución]{%
Probar que, si $\displaystyle f(x)=\frac{2\tan (x^2-1)}{x^4-x^2}$, entonces
\[
\lim_{x\to 1}f(x)=2.
\]
}%
Sean
\[
k(x) = \frac{2}{x^2\cos (x^2-1)} \yjc h(x)= \frac{\sen(x^2-1)}{x^2-1}.
\]
Como $f(x) = k(x)h(x)$ tenemos que:
\[
\lim_{x\to 1}f(x)=\left( \lim_{x\to 1}k(x) \right)  \left(\lim_{x\to 1}h(x)  \right),
\]
si los límites del lado derecho existen.

Probemos que sí existen. Para el primero, constatamos que
\begin{align*}
\lim_{x\to 1} \cos (x^2-1) & = \cos \left(\lim_{x\to 1}(x^2-1) \right)  \\
& =\cos (0) \\
&=1,
\end{align*}
pues podemos aplicar el teorema del límite de una composición, dado que la función $\cos$ es continua en
$\mathbb{R}$, como lo probaremos más adelante, y:
\[
\lim_{x\to 1}(x^2-1)= 1^2-1 = 0.
\]

Por consiguiente, usando el teorema de las propiedades algebraicas de los límites se tiene:
\[
\lim_{x\to 1}k(x)= \frac{2}{\left( \lim_{x\to 1}x^2 \right)
\left(\lim_{x\to 1}\cos (x^2-1)  \right)} = \frac{2}{(1)(1)} =2.
\]

Para calcular $\displaystyle\lim_{x\to 1}h(x)$, notemos que $h(x)=j(x^2-1)$. En este caso, utilicemos el cambio
de variable $t=g(x)= x^2-1$, para la cual conocemos que
\[
  \lim_{x\to 1}g(x) =\lim_{x\to 1}(x^2-1) =0.
\]
Como la función $j$ no es continua en 0, para utilizar el teorema de cambio de variable, debemos
probar que $g$ es distinta de $0$ en un intervalo centrado en $1$, excepto quizás en $1$. El
intervalo puede ser $[0,2]$, pues, allí, tenemos que
\[
g(x) = x^2 - 1 \neq 0
\]
si $x\neq 1$ (pues $g$ se hace $0$ únicamente en $1$ y en $-1$). Por el teorema de cambio de
variable tenemos que:
\[
\lim_{x\to 1}h(x)= \lim_{x\to 1}j(x^2-1)= \lim_{t\to 0}j(t)=1.
\]
Por lo tanto:
\[
\lim_{x\to 1}f(x)=\lim_{x\to 1}k(x)\cdot \lim_{x\to 1}h(x) =2\cdot 1= 2.
\]
\end{exemplo}

\subsection{Ejercicios}
\begingroup
\small
\begin{multicols}{2}
\begin{enumerate}[leftmargin=*]
\item Use el teorema del cambio de variable para calcular los límites dados, justificando su
    uso en cada caso.
            \begin{enumerate}[leftmargin=*]
            \item $\displaystyle \lim_{x\to 2}\frac{\sen(x^2-4)}{x^4-16}$.
            \item $\displaystyle \lim_{x\to 1}\sqrt{x^2-3x+4}$.
            \item $\displaystyle \lim_{x\to -1}\sqrt[3]{x^2-2x+2}$.
            \item $\displaystyle \lim_{x\to 0}\sqrt{\frac{-x+8}{x+2}}$.
            \end{enumerate}

\item Pruebe que si existe $\displaystyle\lim_{y\to 0}f(y)\neq f(0)$, no se puede usar el teorema del cambio de variable para
    calcular el límite
\[
\lim_{x\to 0}f\left(x\sen \frac{1}{x}\right),
\]
%aunque se conozca que existe $\displaystyle \lim_{y\to 0}f(y)=L$.
\item Supongamos que existen
\[
\lim_{y\to b}f(y) = l \quad \text{y} \quad \lim_{x\to a}g(x) = b.
\]
Pruebe que se puede aplicar el teorema de cambio de variable para calcular
$\displaystyle\limjc{f(g(x))}{x}{a}$, si poniendo $h(x)=g(x)-b$, la función $h$ cambia de signo en $a$.

\item Si existe, halle el valor del límite dado. Haga uso de las siguientes igualdades, cuya
    validez será demostrada más adelante:
    \[
    \lim_{x\to 0}\frac{\sen x}{x}=1 \yjc
    \lim_{x\to 0}\frac{1-\cos x}{x}=0.
    \]
    De ser el caso, realice un cambio de variable adecuado.

\begin{enumerate}[leftmargin=*]
\item $\displaystyle \lim_{s\to 0}\frac{\sen 3s}{5s}$.
\item $\displaystyle \lim_{t\to 0}\frac{\tan 8t}{7t}$.
\item $\displaystyle \lim_{x\to 0}\frac{\sen x+\cos x}{3+\cos x}$.
\item $\displaystyle \lim_{x\to 1}\frac{\sen (x^2-1)}{x^2-3x+2}$.
\item $\displaystyle \lim_{x\to 2}\frac{2x^2+x-10}{\tan (2-x)}$.
\item $\displaystyle \lim_{x\to 0}\frac{x}{\sen\sqrt[3]{x}}$.
\item $\displaystyle \lim_{x\to 0}\frac{1-\cos^2x}{\sen x}$.
\item $\displaystyle \lim_{x\to 0}\frac{x^2+x}{\sen x+\tan x}$.
\item $\displaystyle \lim_{x\to 0}\frac{\arcsen x}{x}$.
\item $\displaystyle \lim_{x\to 0}\frac{\arctan x}{x}$.
\end{enumerate}
\end{enumerate}
\end{multicols}
\endgroup

\section{El teorema del ``sandwich''}
En las últimas unidades, hemos venido utilizando el siguiente límite:
\[
1 = \limjc{\frac{\sen x}{x}}{x}{0}.
\]
Con las propiedades sobre los límites que hemos desarrollado hasta ahora, no es posible que
probemos que esta igualdad es verdadera.

En efecto, no podemos aplicar el límite de un cociente, porque el límite del denominador es igual a
$0$. Por otro lado, no se vislumbra un cambio de variable que nos conduzca a algún límite ya
calculado.

En esta sección, vamos a enunciar un teorema, conocido como el teorema del ``sandwich'', que nos
proveerá de una herramienta muy útil para el cálculo de límites. En el siguiente capítulo,
demostraremos la validez de este resultado.

Antes de enunciar el teorema, veamos cuál es la idea subyacente a él. Para ello, analicemos el caso
del límite
\[
1 = \limjc{\frac{\sen x}{x}}{x}{0}.
\]

La interpretación geométrica de las funciones $\sen$, $\cos$ y $\tan$ nos permiten establecer las
siguientes desigualdades: para todo $x \in \ ]-\frac{\pi}{2},\frac{\pi}{2}[ - \{0\}$, se verifica

\begin{equation*}
	0<\cos x< \frac{\sen x}{x}<1.
\end{equation*}

Para demostrarlas, consideremos un círculo de centro $O$ y radio igual a $1$, como los que se
muestran en la siguiente figura:
\begin{center}
\begin{pspicture}(-0.5,-2.5)(4.5,2.5)
\scriptsize

\psset{PointSymbol=none}

\pstGeonode[PosAngle={-90,0},PointNameSep=0.8em]%
  (0,0){O}(4,0){A}%

\pstCurvAbsNode[]%
  {O}{A}{P}{\pstDistVal{2}}%
\pstCurvAbsNode[]%
  {O}{A}{P'}{\pstDistVal{-2}}%

\pstArcOAB[]%
  {O}{P'}{P}%

\pstLineAB[]%
  {O}{P}%
\pstLineAB[]%
  {O}{P'}%
\pstLineAB[]%
  {P}{P'}%
\pstLineAB[]%
  {O}{A}%

\pstInterLL[PosAngle=225]%
  {P}{P'}{O}{A}{C}%

\pstMarkAngle[MarkAngleRadius=0.8]%
  {A}{O}{P}{}%
\pstMarkAngle[MarkAngleRadius=0.8]%
  {P'}{O}{A}{}%
  
\rput(4.1,1){$x$}
\rput(4.1,-1){$x$}

\pstMiddleAB[PointName=none]%
  {P}{C}{M}%
\uput[180](M){$\sen x$}%

\pstMiddleAB[PointName=none]%
  {O}{C}{N}%
\uput[-90](N){$\cos x$}%

\pstRightAngle[RightAngleSize=0.21]%
  {O}{C}{P}
\end{pspicture}
\hspace{5em}
\begin{pspicture}(-0.5,-0.5)(4.5,4.5)
\scriptsize

\psset{PointSymbol=none}

\pstGeonode[PosAngle={-90,0},PointNameSep=0.8em,PointName={default,default,none}]%
  (0,0){O}(4,0){A}(4,1){B}%

\pstCurvAbsNode[]%
  {O}{A}{P}{\pstDistVal{2}}%
\pstCurvAbsNode[]%
  {O}{A}{R}{\pstDistVal{4}}%

\pstLineAB[]%
  {O}{A}%

\pstMarkAngle[MarkAngleRadius=0.8]%
  {A}{O}{P}{}%
\pstMarkAngle[MarkAngleRadius=0.8]%
  {P}{O}{R}{}%
\pstRightAngle[RightAngleSize=0.21]%
  {O}{A}{B}%
 
\rput(3.5,1){$x$}
\rput(2.8,2.5){$x$}

\pstLineAB[]%
  {O}{A}%
\pstLineAB[]%
  {O}{R}%

\pstInterLL[]%
  {A}{B}{O}{P}{Q}%

\pstLineAB[]%
  {A}{Q}%
\pstLineAB[]%
  {Q}{R}%
\pstLineAB[]%
  {O}{Q}%
\pstRightAngle[RightAngleSize=0.21]%
  {Q}{R}{O}%

\pstArcOAB[]%
  {O}{A}{R}%

\pstMiddleAB[PointName=none]%
  {A}{Q}{S}%

\uput[0](S){$\tan x$}
\end{pspicture}
\end{center}
Entonces, si $x$ es la medida en radianes de un ángulo agudo; es decir, si $x \in\ ]0,\pi/2[$, entonces $\sen x$, $\cos x$ y $\tan
x$ son las medidas de los segmentos $\sjc{CP}$, $\sjc{CO}$ y $\sjc{AQ}$, respectivamente.

Adicionalmente, tenemos que:

\begin{enumerate}
\item $OA = OP = OP' = OR = 1$.
\item La longitud de la cuerda $\sjc{PP'}$ es menor que la longitud del arco $\wideparen{PAP'\
    }$. Puesto que $CP = CP' = \sen x$, la desigualdad entre la cuerda y el arco se traduce en
    la siguiente desigualdad:
    \begin{equation}
    \label{eq:Lim003}
    2\sen x < \ell(\wideparen{PAP'\ }),
    \end{equation}
    donde $\ell(\wideparen{PAP'\ })$ indica la longitud del arco $\wideparen{PAP'\ }$.
\item La longitud de la línea quebrada $\widehat{AQR\ }$ es mayor que la longitud del arco
    $\wideparen{ APR\ }$. Como $AQ = RQ = \tan x$, la desigualdad se traduce en:
    \begin{equation}
    \label{eq:Lim004}
    \ell(\wideparen{APR\ }) < 2\tan x.
    \end{equation}
\item Puesto que la longitud del arco $\wideparen{AP\ }$ es la mitad de la longitud del arco
    $\wideparen{PAP'\ }$ y la longitud del arco $\wideparen{AP}$ es la mitad de la longitud del
    arco $\wideparen{APR\ }$, las desigualdades~(\ref{eq:Lim004}) y (\ref{eq:Lim003}) implican
    las siguientes desigualdades:
    \[
        \ell(\wideparen{AP\ }) < \frac{\sen x}{\cos x} \yjc \sen x < \ell(\wideparen{AP\ }).
    \]
    Por lo tanto, se deben verificar las siguientes desigualdades:
    \[
        \ell(\wideparen{AP})\cos x < \sen x < \ell(\wideparen{AP}).
    \]

    Y, si suponemos que $x\neq 0$, entonces $\ell(\wideparen{AP}) \neq 0$. Entonces, estas dos
últimas desigualdades implican las siguientes:
    \begin{equation}
    \label{eq:Lim005}
    \cos x < \frac{\sen x}{\ell(\wideparen{AP})} < 1.
    \end{equation}

\item La longitud del arco $\wideparen{AP\ }$ es igual a $x$, pues, por la definición de
    radián, se verifica que:
    \[
     \ell(\wideparen{AP\ }) = OA x = 1\times x = x.
    \]
    Por lo tanto, las desigualdades~(\ref{eq:Lim005}) se escriben de la siguiente manera:
    \begin{equation}
    \label{eq:Lim006}
    \cos x < \frac{\sen x}{x} < 1.
    \end{equation}
\end{enumerate}

En resumen, como $x$ es un ángulo agudo, entonces $x\in ]0,\frac{\pi}{2}[$. Y es para estos valores
de $x$ que las igualdades~(\ref{eq:Lim006}) se verifican.

El mismo procedimiento realizado para $x\in ]0,\frac{\pi}{2}[$ puede ser realizado para $x\in
]-\frac{\pi}{2}, 0[$. Lo que prueba que las desigualdades~(\ref{eq:Lim006}) son verdaderas para
todo $x$ distinto de $0$ tal que $-\frac{\pi}{2} < x < \frac{\pi}{2}$.

Ahora bien, cuando $x$ tiende a $0$, probaremos más adelante que $\cos x$ tiende a $\cos 0$; es
decir, tiende a $1$. Lo mismo sucede con la constante $1$. Es decir, las dos cotas de la fracción
\[
\frac{\sen x}{x},
\]
la mayor y la menor, tienden a $1$ cuando $x$ tiende a $0$. ?`Podría suceder, entonces, que la
fracción no tendiera a $0$? El teorema del ``sandwich'' nos asegura que eso no puede suceder. Es
decir, este teorema nos asegura que es verdad que
\[
1 = \limjc{\frac{\sen x}{x}}{x}{0}.
\]

Este es el enunciado del teorema:

\begin{teocal}[Teorema del sandwich o de los dos gendarmes, o principio de intercalación]
Sean $a\in\mathbb{R}$, $f:\Dm(f) \rightarrow \mathbb{R}$, $g:\Dm(g) \rightarrow \mathbb{R} $,
$h:\Dm(h) \rightarrow \mathbb{R}$ tres funciones tales que sus dominios contienen un intervalo
abierto $I$ centrado en $a$, excepto quizás el punto $a$.
Supongamos que:
$
	f(x)\leq g(x)\leq h(x) \ \text{ para todo }\ x\in I - \{a\}
$
y que
$\displaystyle
	L = \lim_{x \rightarrow a}f(x)= \lim_{x \rightarrow a}h(x)
$.
Entonces
$\displaystyle
	L = \lim_{x \rightarrow a}g(x).
$
\end{teocal}

Para el caso del límite
\[
\limjc{\frac{\sen x}{x}}{x}{0},
\]
tenemos que $f(x) = \cos x$, $g(x) = \frac{\sen x}{x}$ y $h(x) = 1$ para todo $x \in D =
]-\frac{\pi}{2},\frac{\pi}{2}[ - \{0\}$ y que
\[
L = 1 = \limjc{f(x)}{x}{0} = \limjc{h(x)}{x}{0}.
\]

El nombre de ``sandwich'' --o en español ``emparedado''-- se debe al hecho de que la función $g$
(el queso) se ``coloca'' entre $f$ y $h$ (los panes). Los franceses lo llaman teorema ``de los dos
gendarmes'' ($f$ y $h$) que ``llevan preso a $g$ entre ellos''.

\begin{exemplo}[Solución]{%
Calcular
\[
\lim_{x\to 0}x^2\sen^2 \frac{\pi}{x} \yjc  \lim_{x\to 0}x\cos \frac{1}{x}.
\]
}
\begin{enumerate}[leftmargin=*]
\item Si definimos $f$, $g$ y $h$ tales que
\[
g(x)=0, \quad f(x)= x^2\sen^2 \frac{\pi}{x} \yjc h(x)=x^2,
\]
tenemos que, para todo $x\neq 0$:
\[
g(x)\leq f(x)\leq h(x).
\]
Como
\[
\lim_{x\to 0}g(x)= \lim_{x\to 0}h(x)=0,
\]
por el teorema del sandwich tenemos que:
\[
\lim_{x\to 0}f(x)=0.
\]

\item Definamos:
\[
g(x)=0, \quad f(x)=\left| x\cos \frac{1}{x}  \right| \yjc h(x)=|x|.
\]
Tenemos que, para todo $x\neq 0$:
\[
g(x)\leq f(x)\leq h(x).
\]
Por consiguiente:
\[
\lim_{x\to 0}|f(x)|= \lim_{x\to 0}\left|x\cos \frac{1}{x}\right|=0.
\]

Puesto que (ver el teorema~\ref{teol:LEquiv0})
\[
\lim_{x\to a} \varphi (x)=0\quad \Leftrightarrow \quad \lim_{x\to a} |\varphi (x)|=0,
\]
tendremos entonces que
\[
\lim_{x\to 0}x\cos \frac{1}{x} = 0.
\]
\end{enumerate}
\end{exemplo}

Como otro ejemplo, demostremos que las funciones $\sen$ y $\cos$ son continuas. El teorema del
sandwich será de ayuda para ello.

\begin{exemplo}[Solución]{
Las funciones $\sen$ y $\cos$ son continuas en $\mathbb{R}$. Y, por lo tanto, las otras cuatro
funciones trigonométricas son también continuas en sus respectivos dominios. }

\begin{enumerate}[leftmargin=*]
\item En primer lugar, probemos que $\sen$ es continua en 0. Para ello, probemos que
$\displaystyle
\lim_{x\to 0}\sen x=0.
$

Para todo $x\in \mathbb{R}$, se tiene que
$
0 \leq |\sen x| \leq |x|.
$
La certeza de esta desigualdad puede ser obtenida del procedimiento seguido para probar que el
cociente $\frac{\sen x}{x}$ está acotado entre $\cos x$ y $1$, desarrollado en páginas
anteriores.
Por el teorema del sandwich y el teorema~(\ref{teol:LEquiv0}) tenemos que:
\[
\lim_{x\to 0}|\sen x| = \limjc{0}{x}{0} = \limjc{|x|}{x}{0} = 0.
\]
Por lo tanto, otra vez por el teorema~\ref{teol:LEquiv0}, podemos afirmar que
$\displaystyle
\lim_{x\to 0}\sen x=0.
$

\item Ahora probemos que $\cos$ es continua en 0. Para ello, debemos probar que
$\displaystyle
\lim_{x\to 0}\cos x=1.
$
Esto equivale a probar que
$\displaystyle
\lim_{x\to 0}(1-\cos x) = 0.
$
Como $1-\cos x= 2 \sen^2\frac{x}{2}$ para todo $x\in \mathbb{R}$ y como
\[
\lim_{x\to 0}\sen\frac{x}{2}= \lim_{t\to 0}\sen t =0
\]
(usando el cambio de variable $t=\frac{x}{2}$, lo que se puede ya que $x\neq 0$ implica  $t\neq
0$), tenemos que
\[
\lim_{x\to 0}(1-\cos x)= 2 \left(\lim_{x\to 0} \sen \frac{x}{2}  \right)^2 =2(0)=0
\]
(por las propiedades algebraicas de los límites).

\item Sea $a\neq 0$. Vamos a probar que
$\displaystyle
\lim_{x\to a}\sen x=\sen a.
$
Para ello, probemos que
$\displaystyle
\lim_{x\to a}(\sen x- \sen a)=0.
$
Hagamos el cambio de variable $x=a+t$. Esto es posible, ya que $\displaystyle\lim_{x\to a}t=0$ y que $x\neq
a$ implica $t\neq 0$. Entonces, como $t=x-a$, tenemos que
\begin{align*}
\lim_{x\to a}(\sen x- \sen a) & = \lim_{t\to 0}[\sen (t+a)- \sen a] \\
& = \lim_{t\to 0}[\sen t \cos a +\cos t \sen a- \sen a]  \\
& = \lim_{t\to 0}[\sen t \cos a +\sen a(\cos t -1) ]  \\
& = \cos a \lim_{t\to 0}\sen t  +\sen a \lim_{t\to 0}(\cos t -1) ] \\
&= (\cos a)(0)+ (\sen a)(0) = 0.
\end{align*}

\item Sea $a\neq 0$. Probemos que
$\displaystyle
\lim_{x\to a}\cos x=\cos a.
$
Para ello, probemos que
$\displaystyle
\lim_{x\to a}(\cos x- \cos a)=0.
$
Con el mismo cambio de variable que en el numeral anterior, tenemos que:
\begin{align*}
\lim_{x\to a}(\cos x- \cos a) & = \lim_{t\to 0}[\cos (t+a)- \cos a] \\
& = \lim_{t\to 0}[\cos t \cos a -\sen t \sen a- \cos a]  \\
& = \lim_{t\to 0}[(\cos t-1) \cos a -\sen a\sen t ]  \\
& = \cos a \lim_{t\to 0}(\cos t -1) -\sen a \lim_{t\to 0}\sen t \\
&= (\cos a)(0)-(\sen a)(0) = 0.
\end{align*}

\end{enumerate}
\end{exemplo}

\subsection{Ejercicios}
\begingroup
\small
\begin{multicols}{2}
\begin{enumerate}[leftmargin=*]
\item Use el teorema del sandwich para calcular:
            \begin{enumerate}
            \item $\displaystyle \lim_{x\to 1}(x-1)^2\cos\frac{\pi}{x-1}$.
             \item $\displaystyle \lim_{x\to 1}f(x)$, si se sabe que $1-|x-1|\leq f(x)\leq
                 x^2-2x+2$.
             \item $\displaystyle \lim_{x\to -1}(x+1)^2g(x)$, si se conoce que existe $M>0$
                 tal que para todo $x$, $|g(x)|<M$.
             \end{enumerate}
\item Diga si se puede aplicar el teorema del sandwich para calcular $\displaystyle\limjc{g(x)}{x}{1}$ si se
    conoce que para todo $x$
\[
2-|x-1|\leq g(x)\leq x^2-2x+4.
\]
\item Diga si se puede aplicar el teorema del sandwich para calcular $\displaystyle\limjc{g(x)}{x}{1}$ si se conoce que para todo $x$, se verifica la desigualdad siguiente:
\[
|g(x)+3|\leq (x-1)^4.
\]

\item Calcule $\displaystyle \limjc{\frac{\tan x}{x}}{x}{0}$.
\end{enumerate}
\end{multicols}
\endgroup

\section{Límites unilaterales}
Como una aplicación de la propiedad arquimediana de los números reales, sabemos que para todo
$x\in\mathbb{R}$, existe un único número entero $n$ tal que
\begin{equation}
\label{prp:ExistenciaSuelo}
n \leq x < n + 1.
\end{equation}
A este número $n$ se le denomina el \emph{suelo} de $x$. Por ejemplo, el suelo de $32.45$ es $32$,
pues
\[
32 \leq 32.45 < 33.
\]
En este caso, $n = 32$.

En el caso de que $x$ sea mayor que $0$, el suelo de $x$ será la parte entera de su representación
decimal. Por ello, al número $n$ también se le conoce como la \emph{parte entera de $x$}  y se le
suele representar\footnote{La notación $[x]$ también suele ser utilizada para representar la parte
entera del número $x$.} con $\lfloor x \rfloor$.

Dado un $x\in\mathbb{R}$, la unicidad del número $n$ que satisface las
desigualdades~(\ref{prp:ExistenciaSuelo}), nos permite definir la función \emph{suelo} de la
siguiente manera:
\[
\funcionjc{\lfloor \ \rfloor}{\mathbb{R}}{\mathbb{Z}}{x}{\lfloor x \rfloor = n,}
\]
donde $n$ satisface las desigualdades~(\ref{prp:ExistenciaSuelo}).

El lector puede constatar por sí mismo que dibujar el gráfico de la función suelo es muy sencillo.
Deberá obtener algo similar al siguiente dibujo:
\begin{center}
\psset{unit=0.75}
\begin{pspicture}(-3.5,-3.5)(3.75,3.75)
\psset{labelFontSize=\scriptstyle}%

\psaxes[arrows=->,linecolor=gray]%
  (0,0)(-3.5,-3.5)(3.5,3.5)%
\uput[-90](3.5,0){$x$}%
\uput[0](0,3.5){$y$}%

\multido{\ii=-3+1,\is=-2+1}{6}{%
\psline[linewidth=\pslinewidth]%
  {*-o}(\ii,\ii)(\is,\ii)%
}%
\end{pspicture}
\end{center}

A pesar de la sencillez de la función suelo, es muy útil a la hora de cuantificar ciertas
magnitudes en situaciones en las que los números reales no pueden captar la esencia del problema.
Por ejemplo, un problema que aparece frecuentemente en el campo de las ciencias de la computación
es el de contar el número de veces que se ejecutan las instrucciones de un algoritmo.

Para concretar, imaginemos el caso de tener que buscar un número de cédula en una lista de un
millón de números de cédulas (como puede suceder en un padrón electoral). Supongamos que, cada vez
que se registra un número de cédula en la lista, la lista es ordenada en forma ascendente. A pesar
de este orden, a la hora de requerir la información relacionada a uno de estos números de cédula,
es necesario buscar dicho número en la lista. El algoritmo de búsqueda denominada \emph{búsqueda
binaria} realiza, cuando el número buscado no está en la lista o es localizado en la última
comparación,
\[
W(n) = \lfloor \lg(n + 1) \rfloor + 1
\]
comparaciones del valor que busca con los valores que están en la lista, donde $n$ es el número de
elementos en la lista y $\lg$ es la función logaritmo en base $2$.

En el caso del ejemplo, $n = 10^6$. Entonces
\[
W(10^6) = \lfloor \lg(10^6 + 1) \rfloor + 1 = 20.
\]

Es decir, el algoritmo de búsqueda binaria solo deberá realizar $20$ comparaciones para indicar que
el número de cédula buscado no está en la lista.

Como puede observarse, la función $\lg$ retorna un número real. La función suelo ``transforma''
este número real en un entero, positivo en este ejemplo, y que refleja correctamente la naturaleza
del problema.

En los ejercicios se presentará una guía para calcular $W(10^6)$ sin recurrir a una calculadora
electrónica para calcular el valor de $\lg(10^6 + 1)$.

Del gráfico de la función suelo, se puede observar que esta función es continua en todo número real
que no es un entero. Utilizando la definición de límite, se puede probar que la función suelo no es
continua en cada entero al mostrar que no existe el límite allí.

?`Por qué no existe ese límite, por ejemplo en el número $1$? Porque $\lfloor x \rfloor$ tiene un
comportamiento diferente antes del número $1$, pero cerca de él, y otro luego del número $1$, pero
también cerca de él.

En casos como el de este ejemplo, resulta conveniente introducir la noción de límites unilaterales
en el sentido de que $x$ se aproxima al número $1$ bajo la condición de que $x > 1$ o bajo la
condición de que $x < 1$. Si una función tiene límite en el punto $1$, la aproximación bajo
cualquiera de las dos condiciones deberá producir el mismo límite; es decir, deberá suceder que la
función tenga los dos límites unilaterales y, además, sean iguales.

Este principio es adecuado para probar también que el límite de una función no existe en un punto:
se prueba que o no existe uno de los unilaterales, o se prueba que son diferentes. Por ejemplo, es
fácil probar que
\[
1 = \limjc{\lfloor x \rfloor}{x}{1} \quad \text{cuando} \ x > 1
\]
y que
\[
0 = \limjc{\lfloor x \rfloor}{x}{1} \quad \text{cuando} \ x < 1.
\]

Las definiciones de límites unilaterales son similares a la definición general de límite, lo mismo
que las propiedades de estos límites.

% -----> 2008 09 12
%\newpage

\begin{defical}[Límites unilaterales]
$L$ es el límite de $f(x)$ cuando $x$ se aproxima a $a$ \emph{por la derecha}, y se escribe:
\begin{equation*}
	L=\lim_{x \rightarrow a^+}f(x),
\end{equation*}
si y solo si para todo $\epsilon > 0$, existe $\delta > 0$ tal que
\[
|f(x) - L| < \epsilon,
\]
siempre que $0 < x - a < \delta$.

Análogamente: $L$ es el límite de $f(x)$ cuando $x$ se aproxima a $a$ \emph{por la izquierda}, y se
escribe:
\begin{equation*}
	L=\lim_{x \rightarrow a^-}f(x),
\end{equation*}
si y solo si para todo $\epsilon > 0$, existe $\delta > 0$ tal que
\[
|f(x) - L| < \epsilon,
\]
siempre que $0 < a - x < \delta$.
\end{defical}

Con estas definiciones, podemos afirmar que:
\[
1 = \limjc{\lfloor x \rfloor}{x}{1^+} \yjc 0 = \limjc{\lfloor x \rfloor}{x}{1^-}.
\]
Entonces:
\[
\limjc{\lfloor x \rfloor}{x}{1^+} \neq \limjc{\lfloor x \rfloor}{x}{1^-},
\]
Esto es suficiente para afirmar que no existe el límite de $\lfloor x \rfloor$ cuando $x$ se
aproxima a $1$. De hecho, se tiene el siguiente teorema:

\begin{teocal}$L$ es el límite de $f(x)$ cuando $x$ se aproxima al número $a$ si y solo si
existen los dos límites unilaterales y son iguales a $L$.
\end{teocal}

Las propiedades de los límites que enunciamos en teoremas anteriores se verifican también para los
límites unilaterales con las evidentes modificaciones en cada caso.

Con la noción de límites unilaterales se puede definir la \emph{continuidad por la derecha} y por
\emph{la izquierda} mediante la siguiente modificación de la definición de continuidad: en lugar de
que $f(a)$ sea el límite de $f(x)$ cuando $x$ se aproxima a $a$, hay que cambiar, en el caso de la
continuidad por la derecha, que $x$ se aproxima a $a$ por la derecha; lo mismo para el caso de la
continuidad por la izquierda. Es inmediato de esta definición que una función será continua en un
punto si y solo si es continua por la derecha y por la izquierda del punto. Este hecho lo
expresamos en el siguiente teorema.

\begin{teocal}Una función es continua en $a$ si y solo si es continua en $a$ por la derecha y es
continua en $a$ por la izquierda.
\end{teocal}

\begin{exemplo}[Solución]{%
Sea $\funcjc{f}{[0,+\infty[}{[0,+\infty[}$ definida por $f(x)= \sqrt{x}$. Entonces $f$ es continua
en $0$ por la derecha.} Puesto que
\[
\lim_{x\to a}\sqrt{x}= \sqrt{a},
\]
para $a > 0$, $f$ es continua en $\mathbb{R}^+$. Como $f(x)$ no está definida para $x<0$, no existe
\[
\lim_{x\to 0}\sqrt{x}.
\]
Sin embargo, tenemos que \emph{$f$ es continua en $0$ por la derecha}, es decir:
\[
\lim_{x\to 0^+}\sqrt{x}=0=\sqrt{0}.
\]

En efecto: sea $\epsilon>0$. Debemos hallar $\delta >0$ tal que
\[
|\sqrt{x}-0|<\epsilon,
\]
siempre que $0<x<\delta$.

Sea $x>0$. Como $|\sqrt{x}- \sqrt{0}|=\sqrt{x}$, tenemos que
\[
|\sqrt{x}-0|<\epsilon \quad \Leftrightarrow \quad \sqrt{x}<\epsilon \quad \Leftrightarrow \quad x<\epsilon^2.
\]
Tomemos, entonces, $\delta=\epsilon^2$. En ese caso, tenemos:
\[
0<x<\delta \quad \Rightarrow \quad x<\epsilon^2 \quad \Rightarrow \quad
\sqrt{x}<\epsilon \quad \Rightarrow \quad |\sqrt{x}- \sqrt{0}|<\epsilon.
\]
Hemos probado entonces que $\displaystyle 0=\sqrt{0}= \lim_{x\to 0^+}\sqrt{x}$; es decir, hemos probado que $f$
es continua en $0$ por la derecha.

Es claro que la función $f$ no puede ser continua en $0$ por la izquierda, pues $f$ no está
definida para ningún $x < 0$.
\end{exemplo}

\begin{exemplo}[Solución]{%
Sea
\begin{equation*}
f(x) =
\begin{cases}
2x^2-1& $si $x>1$$\\
2& $si $x=1$$\\
-x^2+3x& $si $x<1$$
\end{cases}
\end{equation*}
?`Es $f$ continua en $1$, en $1$ por la derecha, en $1$ por la izquierda?}%
Para saber si es continua en
$1$, veamos si es continua por la izquierda y por la derecha. Para ello, calculemos $\displaystyle\lim_{x\to
1^-}f(x)$ y $\displaystyle\lim_{x\to 1^+}f(x)$. En primer lugar:
\[
\lim_{x\to 1^-}f(x)  = \lim_{x\to 1^-}(-x^2+3x),
\]
pues $f(x)=-x^2+3x$ si $x < 1$. Entonces:
\begin{align*}
\lim_{x\to 1^-}f(x) &= \lim_{x\to 1}(-x^2+3x) \\
&= -(1)^2 + 3\times 1 = 2.
\end{align*}

En segundo lugar, como $f(x) = 2x^2-1$ si $x > 1$, entonces:
\begin{align*}
\lim_{x\to 1^+}f(x) &= \lim_{x\to 1}(2x^2-1) \\
&= 2(1)^2 - 1 = 1.
\end{align*}

Por lo tanto:
\[
\lim_{x\to 1^-}f(x)  = f(1) = 2\neq 1 =\lim_{x\to 1^+}f(x) .
\]

En conclusión, $f$ es continua en 1 por la izquierda, pero no lo es por la derecha, ni tampoco es
continua en $1$.
\end{exemplo}

\begin{exemplo}[Solución]{%
Sea $\funcjc{f}{\mathbb{R}}{\mathbb{R}}$ definida por $f(x) = \lfloor x \rfloor$. Entonces $f$ es
discontinua en $x$ si y solo si $x\in \mathbb{Z}$.}

Sean $x$ y $n$ tales que $f(x) = n$. Entonces:
\[
n \leq x < n +1.
\]
Por lo tanto, $f$ es constante en $]n,n+1[$, por lo que es continua en cualquier intervalo entre
dos enteros consecutivos.

Veamos ahora que no es continua en ningún entero. Para ello calculemos los límites laterales en
$n$. Para empezar:
\[
\lim_{x\to n^+}f(x) = \lim_{x\to n}n = n = f(n).
\]
Por otra parte, si $x < n$, $f(x) = n - 1$. Entonces:
\[
\lim_{x\to n^-}f(x) = \lim_{x\to n}(n-1) = n-1 \neq f(n).
\]
Por lo tanto, para $n\in \mathbb{Z}$:
\[
\lim_{x\to n^-}f(x) = n-1\neq n= f(n) = \lim_{x\to n^+}f(x).
\]
Entonces: $f$ es discontinua en $n$ para todo $n\in \mathbb{Z}$.
\end{exemplo}

\begin{exemplo}[Solución]{%
Sea $\funcjc{g}{\mathbb{R}}{\mathbb{R}}$ definida por $g(x)=x-\lfloor x \rfloor$. El número $g(x)$
es la ``parte fraccionaria de $x$''. Entonces $g$ es discontinua en $x$ si y solo si $x\in
\mathbb{Z}$.}

Sean $f$ definida por $f(x) = \lfloor x\rfloor$ y $h$ definida por $h(x) = x$ para todo
$x\in\mathbb{R}$. Entonces $g = h - f$, pues
\[
g(x) = h(x) - f(x) = x - \lfloor x \rfloor.
\]

Si $g$ fuera continua en $n\in\mathbb{Z}$, como $h$ es continua en $\mathbb{R}$, es continua en
$n$. Por lo tanto: $g + h$ sería continua en $n$. Pero $f = h - g$ sería continua en $n$, ya que
$h$ y $g$ lo son. Pero sabemos por el ejercicio anterior que $f$ no es continua en ningún entero.
Por lo tanto, $g$ no puede ser continua en $n$.

?`Por qué el número $g(x)$ es llamado la ``parte fraccionaria de $x$''?
\end{exemplo}

\subsection{Límites unilaterales de funciones localmente iguales}

El concepto de funciones localmente iguales puede aplicarse también unilateralmente.

\begin{defical}[Funciones localmente iguales por la derecha o por la izquierda]
Sean: $a$ un número real; y, $f$ y $g$ dos funciones reales.

Diremos que $f=g$ cerca de $a$ por la derecha (respectivamente por la izquierda), si existe un número $r>0$ tal que para todo $x\in ]a,a+r[$ (respectivamente $x\in ]a-r,a[$), se tiene que $f(x)=g(x)$.
\end{defical}

Con esta definición se puede fácilmente demostrar el siguiente resultado.

\begin{teocal}[Límite unilaterales de funciones localmente iguales]%
Sean $a$ un número real; y $f$ y $g$ dos funciones reales localmente iguales por la derecha (respectivamente por la izquierda). Entonces:
\begin{enumerate}
\item Existe $\displaystyle\limjc{f(x)}{x}{a{+}}$ (respectivamente $\displaystyle\limjc{f(x)}{x}{a{-}}$) si y solo si existe $\displaystyle\limjc{g(x)}{x}{a{-}}$ (respectivamente $\displaystyle\limjc{g(x)}{x}{a{-}}$).
\item Si los límites existen, son iguales.
\end{enumerate}
\end{teocal}


\subsection{Ejercicios}
\begingroup
\small
\begin{multicols}{2}
\begin{enumerate}[leftmargin=*]
\item Dibuje la gráfica de $f$ y determine, si existen, los límites para el valor de $a$ dado:
\begin{equation*}
	\lim_{x\to a{-}}f(x), \quad \lim_{x\to a{+}}f(x), \quad \lim_{x\to a}f(x).
\end{equation*}
\begin{enumerate}[leftmargin=*]
\item
\begin{equation*}
	f(x)=
\begin{cases}
2x-1& \text{si $x<2$}\\
3 & \text{si $x=2$}\\
x+1 & \text{si $x>2$}.
\end{cases}
\quad a=2
\end{equation*}
\item
\begin{equation*}
	f(x)=
\begin{cases}
x^2-1 & \text{si $x\leq 2$}\\
\frac{1}{3}(11-x)& \text{si $x>2$}.
\end{cases}
\quad  a=2
\end{equation*}
\item
\begin{equation*}
	f(x)=
\begin{cases}
|x+1| & \text{si $x<1$}\\
\sqrt{2-x}& \text{si $x\geq 1$}.
\end{cases}
\quad a=1
\end{equation*}
\item
\begin{equation*}
	f(x)=
\begin{cases}
\sqrt{9-x^2}& \text{si $|x|< 3$}\\
x+1 & \text{si $|x|\geq 3$}.
\end{cases}
\quad a=3
\end{equation*}
\item $f(x) =1+\lfloor x\rfloor$; $a\in \mathbb{Z}$.
\end{enumerate}

\item Mostrar que
  \[
      \lfloor x \rfloor = \max\{n\in\mathbb{Z} : n \leq x\}
  \]
  para todo $x\in\mathbb{R}$.

\item Estudie la continuidad de la función $f$ cuyo dominio es $\mathbb{R}$ y definida por
    $f(x) = \lfloor x \rfloor - \lfloor x + 1 \rfloor$.

\item Calcule los límites siguientes, si existen. En los casos en que no exista el límite,
    demuéstrelo:
    \begin{enumerate}
    \item $\displaystyle{\limjc{\frac{\sqrt{x - 1}}{x^2 - 3x + 2}}{x}{1^+}}$.
    \item $\displaystyle{\limjc{\frac{\sqrt{1 - x}}{x^2 - 3x + 2}}{x}{1^+}}$.
    \item $\displaystyle{\limjc{\frac{x+1}{x^2 - 3x + 2}}{x}{2^-}}$.
    \item $\displaystyle{\limjc{\frac{2x^2 - x + 6}{x^2 - 3x + 2}}{x}{2^-}}$.
    \end{enumerate}

\item Diga si la función $f$ es continua en $1$ por la derecha o por la izquierda:
\begin{enumerate}
\item $\displaystyle{f(x) = \begin{cases} x^2 - 3x + 1 & \text{si} \ x > 1 \\
-1 & \text{si} \ x = 1 \\
2x^2 + x + 1 & \text{si} \ x < 1.
\end{cases}}$

\item $\displaystyle{f(x) = \begin{cases} \sqrt{x - 1} & \text{si} \ x > 1 \\
1 & \text{si} \ x = 1 \\
x^2 - 3x + 3 & \text{si} \ x < 1.
\end{cases}}$

\item $\displaystyle{f(x) = \begin{cases} \frac{1}{x - 1} & \text{si} \ x > 1 \\
0 & \text{si} \ x = 1 \\
\frac{x^2 - 3x + 2}{\sqrt{1 - x}} & \text{si} \ x < 1.
\end{cases}}$

\item $\displaystyle{f(x) = \begin{cases} \sqrt{x^2 + x - 2} & \text{si} \ x > 1 \\
\sqrt{1 - x^2} & \text{si} \ x \leq 1.
\end{cases}}$

\end{enumerate}
\item Use las definiciones de límites laterales para probar que:

\begin{enumerate}
\item $\displaystyle \lim_{x\to 2^-}f(x)=-1$, $\displaystyle \lim_{x\to 2^+}f(x)=5$, si
\[
    f(x)=
\begin{cases}
x^2+1 & \text{si $x>2$}\\
-x^2+3 & \text{si $x<2$}.
\end{cases}
\]
\item $\displaystyle \lim_{x\to 1^-}\sqrt{-x^3+1}=0.$
\item $\displaystyle \lim_{x\to \frac{2}{3}^+}\sqrt[4]{3x-2}=0$.
\end{enumerate}

\item Pruebe que no existe el límite dado.

\begin{enumerate}
\item $\displaystyle \lim_{x\to -1}f(x)$, si
\[
f(x)=
\begin{cases}
x-3 & \text{si $x<-1$}\\
2x+1 & \text{si $x>-1$}.
\end{cases}
\]
\item $\displaystyle \lim_{x\to 2}\frac{2x+1}{x^4-4}$.
\item $\displaystyle \lim_{x\to 1}\frac{1}{x^2-5x+4}$.
\item $\displaystyle \lim_{x\to 0}f(x)$, si $\displaystyle f(x)=
\begin{cases}
x+1 & \text{si $x<0$}\\
2x-3& \text{si $x>0$}.
\end{cases}
$
\end{enumerate}

\item Con ayuda de las propiedades de la función $\lg$ se puede probar, sin recurrir al uso de
    una calculadora electrónica, el valor de $W(10^6)$, dado por:
    \[
      W(10^6) = \lfloor\lg(10^6  + 1)\rfloor + 1.
    \]
    En primer lugar, el lector debe constar mediante un cálculo directo que
    \[
        2^{19} < 10^6 + 1 < 2^{20}.
    \]
    A continuación, debe recordar que la función $\lg$ es estrictamente creciente y $\lg(x)$ es
    el número real al que hay que elevar $2$ para obtener $x$. Es decir:
    \[
      y = \lg(x) \Longleftrightarrow x = 2^y.
    \]
    Esta información es suficiente para obtener el valor exacto de $W(10^6)$.

\item Sea 
\[ f(x)=
\begin{cases}
x^2+1 & \text{si $x\leq 0$}\\
1-x^2 & \text{si $0<x\leq 1$; y,}\\
\alpha (x-1) & \text{si $x>1$.}
\end{cases}
\]
?`Existen valores de $\alpha$ para los cuales $f$ sea derivable ?
\end{enumerate}
\end{multicols}
\endgroup

\section{Límites infinitos y al infinito}
Cuando una función tiene límite en un punto, se dice que la función \emph{converge} al límite en
ese punto. Cuando la función no tiene límite, se dice que \emph{diverge}. Por ejemplo, la función
suelo diverge en todo número entero.

Hay varias maneras de divergir. Una como la de la función suelo. Otra, cuando la función crece
indefinidamente o decrece indefinidamente. En ese caso se dice que la divergencia es al infinito, o
que el límite es ``infinito''. Por ejemplo, probaremos que $f(x) = \frac{1}{x^2}$ crece
indefinidamente cuando $x$ tiende a $0$.

Por otra parte, puede suceder que un número esté tan cerca como se quiera de $f(x)$ cuando $x$
crece indefinidamente o cuando decrece indefinidamente; en ese caso, no hay divergencia, pues el
límite existe. En estas circunstancias, se dice que los ``límites son al infinito''. Por ejemplo,
mostraremos que la $f(x) = \frac{1}{x^2}$ tiende a $0$ cuando $x$ crece indefinidamente.

Hay numerosas situaciones en las que surgen estos dos tipos de límites. En las siguientes
subsecciones, vamos a ver un ejemplo de cada uno.

\subsection{Límites infinitos}
En $1905$, Albert Einstein corrigió un error que encontró en las leyes del movimiento enunciadas
por Newton doscientos años atrás.

En efecto, la segunda ley de Newton asume implícitamente que la masa de un cuerpo es constante. Sin
embargo, Einstein descubrió que la masa de un cuerpo varía con su velocidad\footnote{Un tratamiento
conceptual, profundo, pero al mismo tiempo sencillo, se encuentra en el libro de Richard Feynman,
"The Feynman Lectures on Physics", en el capítulo 15 del primer volumen.}.

De manera más precisa, Einstein estableció que la masa de un cuerpo es una función de su velocidad,
que puede ser calculada a través de la siguiente igualdad:
\[
m(v) = \frac{m_0}{\sqrt{1 - \frac{v^2}{c^2}}},
\]
donde $m_0$ es la masa que el cuerpo tiene cuando está en reposo, $v\kilometros/\segundos$ es su
velocidad y $c\kilometros/\segundos$ es la velocidad de la luz (aproximadamente $3 \times 10^5
\kilometros/\segundos$). Se supone, además, que la velocidad del cuerpo es menor que la velocidad
de la luz.

?`Qué sucedería si la velocidad del cuerpo se acercara a la velocidad de la luz?

Esta pregunta puede ser expresada en términos de límites de la siguiente manera: ?`a qué es igual el
límite de $m(v)$ cuando $v$ tiende a $c$?

Si damos a $v$ algunos valores cercanos a $c$, veremos que $m(v)$ crece. Probaremos en esta sección
que esto es, efectivamente, así.

La idea subyacente de que $f(x)$ crece indefinidamente cuando $x$ se acerca a un número $a$
consiste en que, dado cualquier número positivo $R$, por más grande que éste sea, siempre hay un
intervalo alrededor de $a$ en donde $f(x)$ es más grande que $R$. En otras palabras, no es posible
acotar superiormente el conjunto de valores de la función $f$. Se dice que $f(x)$ tiende a ``más
infinito'' para representar este ``crecimiento indefinido'' y se utiliza el símbolo $+\infty$ para
representar este comportamiento de la función $f$ alrededor del punto $a$. En la siguiente
definición, precisamos esta idea.

\lteocal[Límites infinitos]{defi}{%
$\displaystyle{\lim_{x \to a}f(x) =+\infty}$ si y solo si para todo $R > 0$, existe $\delta
    > 0$ tal que
    \[
    f(x) > R,
    \]
    siempre que $x\in\Dm(f)$ y $0 < |x - a| < \delta$.

Análogamente: $\displaystyle{\lim_{x \to a}f(x) =-\infty}$ si y solo si para todo $R < 0$, existe
$\delta
    > 0$ tal que
    \[
    f(x) < R,
    \]
    siempre que $x\in\Dm(f)$ y $0 < |x - a| < \delta$.}

En los ejemplos y en los ejercicios de esta sección, veremos definiciones análogas a los límites
laterales para el caso de límites infinitos.

Veamos algunos ejemplos.

\begin{exemplo}[Solución]{%
Probar que $\displaystyle\limjc{\frac{1}{x^2}}{x}{0} = +\infty$.} Sea $R > 0$. Debemos encontrar un
número $\delta > 0$ tal que
\begin{equation}
\label{eq:Lim007}
\frac{1}{x^2} > R,
\end{equation}
siempre que $0 < |x| < \delta$.

Para encontrar el número $\delta$, analicemos, en primer lugar, la desigualdad~(\ref{eq:Lim007})
con miras a establecer una relación de esta desigualdad con las desigualdades $0 < |x| < \delta$.

Para $x \neq 0$, las siguientes equivalencias son verdaderas:
\begin{align*}
\frac{1}{x^2} > R & \ \Leftrightarrow \ 0 < x^2 < \frac{1}{R} \\
  & \ \Leftrightarrow \ 0 < |x| < \frac{1}{\sqrt{R}}.
\end{align*}
Por lo tanto:
\begin{equation}
\label{eq:Lim008}
\frac{1}{x^2} > R \ \Leftrightarrow \ 0 < |x| < \frac{1}{\sqrt{R}}.
\end{equation}

Esta última equivalencia nos garantiza que el número $\delta$ buscado es:
\[
\delta = \frac{1}{\sqrt{R}},
\]
pues, si se elige $x$ de modo que $0 < |x| < \delta$, por las equivalencia~(\ref{eq:Lim008}), se
verifica la desigualdad~(\ref{eq:Lim007}). Por lo tanto, hemos probado que:
\[
\limjc{\frac{1}{x^2}}{x}{0} = +\infty.
\]
\end{exemplo}

En el siguiente ejemplo, se requiere un trabajo mayor para encontrar el número $\delta$ que exige
la definición de límite infinito. Para encontrarlo, las siguientes reflexiones sobre este concepto
son de mucha ayuda.

Supongamos que
\[
\limjc{f(x)}{x}{a} = +\infty.
\]
Entonces, dado un número real $R > 0$, por la definición de límite, podemos asegurar la existencia
de un número $\delta > 0$ tal que
\[
f(x) > R
\]
para todo $x \in A = ]a - \delta, a + \delta[ - \{a\}$.

Como $R > 0$, podemos asegurar que $f(x) > 0$ para todo $x \in A$.

En resumen, si $f(x)$ tiende a $+\infty$ cuando $x$ tiende al número $a$, entonces $f(x)$ es
positiva, excepto, quizás, en $a$, en todos los puntos de un intervalo abierto centrado en $a$.

De manera similar, podemos concluir que, si $f(x)$ tiende a $-\infty$ cuando $x$ tiende al
número $a$, entonces $f(x)$ es negativa, excepto, quizás, en $a$, en todos los puntos de un
intervalo abierto centrado en $a$.

Resumamos estos resultados en el siguiente teorema:

\lteocal{teo}{\label{teo:LimInfPosNeg}%
Si
\[
\limjc{f(x)}{x}{a} = +\infty,
\]
existe $\delta > 0$ tal que
\[
f(x) > 0
\]
para todo $x \in ]a - \delta, a + \delta[ - \{a\}$.

Si
\[
\limjc{f(x)}{x}{a} = -\infty,
\]
existe $\delta > 0$ tal que
\[
f(x) < 0
\]
para todo $x \in ]a - \delta, a + \delta[ - \{a\}$.
}%fin de \lteocal

Esta característica de una función cuando tiende a más o menos infinito, reduce el espacio de
búsqueda del número $\delta$.

En efecto, en el caso de que se quiera demostrar que $f(x)$ tiende a $+\infty$, el análisis para la
búsqueda del número $\delta$ solo debe reducirse al subconjunto del dominio de $f$ en la que ésta
es positiva. En el caso de que sea $-\infty$, el análisis se realiza en el conjunto donde $f$ sea
negativa. En otras palabras, conocer los valores de $x$ donde $f(x)$ es positiva o es negativa es
de mucha ayuda a la hora de buscar el número $\delta$. Veamos cómo se puede hacer esto en el
siguiente ejemplo.


\begin{exemplo}[Solución]{%
Pruebe que $\displaystyle \lim_{x\to 2}\frac{(5-x)(1 - x)}{(x-2)^2}=-\infty$.}%
\def\f(x){\dfrac{(5-x)(1 - x)}{(x-2)^2}}

Sean $f(x)=\dfrac{(5-x)(1 - x)}{(x-2)^2}$ y $R<0$. Debemos hallar $\delta>0$ tal que
\begin{equation}
\label{eq:Lim009}
	f(x)<R,
\end{equation}
siempre que $0<|x-2|<\delta$ y $x\neq 2$.

Para encontrar el valor de $\delta$, analicemos la desigualdad~\ref{eq:Lim009}:
\[
\f(x) < R.
\]
Observemos que, como el factor $(x - 2)$ está en el denominador, si se verificara la condición
\[
0 < |x - 2| < \delta,
\]
entonces se verificaría la condición
\[
0 < (x - 2)^2 < \delta^2,
\]
de donde también se verificaría que
\begin{equation}
\label{eq:Lim010}
\frac{1}{(x - 2)^2} > \frac{1}{\delta^2}.
\end{equation}

Ahora, busquemos un intervalo centrado en $2$ donde $f(x)$ sea un número negativo. Como $(x - 2)^2
> 0$, entonces, en dicho intervalo, deberá ocurrir que
\[
(5 - x)(1 - x) < 0.
\]
Y, si encontráramos una constante $M < 0$ tal que
\[
(5 - x)(1 - x) < M
\]
en dicho entorno, la desigualdad~(\ref{eq:Lim010}) implicaría que
\begin{equation}
\label{eq:Lim011}
f(x) = \f(x) < \frac{M}{\delta^2},
\end{equation}
siempre que $0 < |x - 2| < \delta$ y $f(x) < 0$.

Entonces, si comparamos la desigualdad~(\ref{eq:Lim011}) con la desigualdad~(\ref{eq:Lim009}),
vemos que el número $\delta$ que hay que elegir es aquel garantice que
\[
\frac{M}{\delta^2} = R \yjc f(x) < 0.
\]

En resumen, lo que nos queda por hacer es encontrar:
\begin{enumerate}
\item un intervalo centrado en $2$ en el que $f(x) < 0$; y
\item una constante $M < 0$ tal que $(5 - x)(1 - x) < M$ en dicho entorno.
\end{enumerate}

Empecemos por encontrar los valores de $x \neq 2$ para los cuales $f(x) < 0$. Para ello
consideremos las siguientes equivalencias:
\begin{align*}
\frac{(5 - x)(1 - x)}{(x - 2)^2} < 0 &\Longleftrightarrow
(5 - x)(1 - x) < 0, \quad\text{pues}\ (x - 2)^2 > 0 \ \text{para todo}\ x\in\mathbb{R} \\
&\Longleftrightarrow
(x - 5)(x - 1) < 0 \\
&\Longleftrightarrow x \in \ ]1,5[.
\end{align*}
La verdad de la última equivalencia se puede determinar si se considera que el gráfico del
polinomio $(x -5)(x - 1)$ es una parábola convexa que corta el eje $x$ en los abscisas $1$ y $5$,
por lo que la parte de la parábola que está bajo el eje $x$ está entre $1$ y $5$, como se muestra
en el siguiente dibujo:
\begin{center}
\psset{xAxisLabel={},yAxisLabel={},plotpoints=1000}%
\def\f{x dup neg 5 add exch neg 1 add mul}

\begin{psgraph}[arrows=->,ticks=x](0,0)(-0.5,-4.5)(6,5.5){0.4\textwidth}{5cm}
  \uput[-90](6,0){$x$}%
  \uput[0](0,5.5){$y$}%

  \psplot{0}{5.5}{\f}%

\end{psgraph}

{\small El gráfico de $y = (5 - x)(1 - x)$}
\end{center}

En resumen, la función $f$ es negativa en el intervalo $]1, 5[ - \{2\}$. Es decir, vemos que
\[
f(x) < 0,
\]
siempre que $1 < x < 5$ y $x \neq 2$.

Ahora, encontremos la constante $M$. Como el límite es en $2$, elijamos valores de $x$ cercanos a
$2$. Por ejemplo, tomemos $x\neq 2$ tales que $|x-2| < \frac{1}{2}$. Esto equivale a
\[
-\frac{1}{2} < x - 2 < \frac{1}{2},
\]
de donde
\begin{equation}
\label{eq:Lim012}
\frac{3}{2} < x < \frac{5}{2}.
\end{equation}
Como para estos valores de $x$, $f(x) < 0$, si $x\neq 2$, entonces, busquemos $M$ en el intervalo
$\left]\frac{3}{2},\frac{5}{2}\right[$.

Para ello, definamos $g(x) = (5- x)(1 - x)$. Sabemos que, en el intervalo
$\left]\frac{3}{2},\frac{5}{2}\right[$, el gráfico de $g$ es una parábola, como se mostró
anteriormente. El mínimo de $g$ en este intervalo está en el punto medio de las dos raíces; es
decir, en $x = 3$. Por lo tanto, en el intervalo $\left]\frac{3}{2},3\right[$, y por ende en
$\left]\frac{3}{2},3\right[$, la función $g$ es decreciente, por lo que
\[
g(x) < g\left(\frac{3}{2}\right) = \left(5 - \frac{3}{2}\right)\left(1 - \frac{3}{2}\right) =
-\frac{7}{4}
\]
para todo $x \in\ \left]\frac{3}{2},\frac{5}{2}\right[$. Esto significa, entonces, que $M =
-\frac{7}{4}$.

Resumamos: si $x$ es tal que $|x - 2| < \frac{1}{2}$, entonces $(5 - x)(1 - x) < M = -\frac{7}{4}$.

Por otro lado, $\delta$ debe cumplir con
\[
\frac{M}{\delta^2} = R,
\]
de donde
\[
\delta^2 = \frac{M}{R},
\]
de donde, como $M < 0$ y $R < 0$, tenemos que $\delta$ puede ser elegido así:
\[
\delta = \sqrt{\frac{M}{R}} = \frac{1}{2}\sqrt{\frac{-7}{R}}.
\]

Por lo tanto, si elegimos $\delta$ tal que:
\begin{equation*}
	\delta = \min\left\{\frac{1}{2},\frac{1}{2}\sqrt{\frac{-7}{R}}\right\},
\end{equation*}
se verifica que
\[
\f(x) < R
\]
siempre que $0 < |x - 2| < \delta$.

Hemos probado, entonces, que
\[
\limjc{\f(x)}{x}{2} = -\infty.
\]
\end{exemplo}

La definición de límite lateral infinito es similar a la de límite infinito con la inclusión de la
condición de que $x$ sea o menor que $a$ o mayor que $a$. Así, la expresión
\[
\limjc{f(x)}{x}{a^+} = +\infty
\]
quiere decir que, dado cualquier número real positivo $R$, siempre existe un número $\delta > 0$
tal que
\[
f(x) > R
\]
siempre que $0 < x - a < \delta$.

Definiciones similares se tienen para el caso de que $x$ tiende al número $a$ por la izquierda.

Con estas definiciones, podemos expresar el hecho de que la masa de un cuerpo, según la teoría
especial de la relatividad de Einstein, crezca indefinidamente cuando su velocidad se acerca a la
de la luz de la siguiente manera:
\begin{equation}
\label{eq:LimMasaEinstein}
\limjc{\frac{m_0}{\sqrt{1 - \frac{v^2}{c^2}}}}{v}{c^-} = +\infty.
\end{equation}

Para probar que esta igualdad es verdadera, podríamos recurrir a la definición directamente. Sin
embargo, igual que ocurre con los límites ``comunes y corrientes'', vamos a utilizar propiedades de
los límites infinitos que nos permitirán calcular muchos límites, entre ellos el de la masa de un
cuerpo, conociendo algunos límites infinitos solamente.

\begin{teocal}\label{teo:LimUnoSobreCero}%
\[
\limjc{\frac{1}{x}}{x}{0^+} = +\infty \yjc \limjc{\frac{1}{x}}{x}{0^-} = -\infty.
\]
\end{teocal}

La demostración de este teorema es sencilla y se la deja para que el lector la realice.

El siguiente teorema es una generalización del anterior, y es la fuente de cálculo de muchos
límites, entre los que se encuentra el límite~(\ref{eq:LimMasaEinstein}). Pero, para una
formulación más compacta, necesitamos ampliar la idea de límites laterales al valor del límite.

De manera más precisa, si ocurre que
\[
L = \limjc{f(x)}{x}{a},
\]
y adicionalmente $f>L$ localmente cerca de $a$; es decir, si existe un número $r > 0$
tal que $f(x) > L$ para todo $x \in ]a-r,a+r[-{a}$, entonces diremos que ``$f(x)$ tiende a $L$ por la
derecha cuando $x$ tiende al número $a$'', y lo expresaremos simbólicamente así:
\[
\limjc{f(x)}{x}{a} = L^+\quad\text{o así}\quad f(x) \rightarrow L^+.
\]

Como un ejemplo, se tiene que
\[
\limjc{x^2}{x}{0} = 0^+,
\]
pues $x^2 > 0 $ para todo $x\neq 0$.

De manera similar hablaremos de que una función $f(x)$ ``tiende a $L$ por la izquierda'', y
escribiremos
\[
\limjc{f(x)}{x}{a} = L^-\quad\text{o así}\quad f(x) \rightarrow L^-,
\]
cuando $\displaystyle L = \limjc{f(x)}{x}{a}$ y exista $r > 0$ tal que $f(x) < L$ siempre que $x
\in ]a-r,a+r[-\{a\}$.

\begin{teocal}\label{teo:LimGeneralInfUnoSobreCero}%
Sean $I$, un intervalo abierto, $a\in I$ y $f$ una función real tal que $I \subset
\Dm(f) \cup \{a\}$. Entonces:
\begin{enumerate}
\item $\displaystyle\limjc{f(x)}{x}{a} = 0^+$ \ si y solo si \
    $\displaystyle\limjc{\frac{1}{f(x)}}{x}{a} = +\infty$.
\item $\displaystyle\limjc{f(x)}{x}{a} = 0^-$ \ si y solo si \
    $\displaystyle\limjc{\frac{1}{f(x)}}{x}{a} = -\infty$.
\end{enumerate}
\end{teocal}

Este teorema también es válido si los límites son laterales. En los ejercicios de esta sección, el
lector tendrá la oportunidad de probar esta afirmación.

Antes de estudiar la demostración de este teorema, vamos a utilizarlo para calcular el
límite~(\ref{eq:LimMasaEinstein}).

\begin{exemplo}[Solución]{%
Cálculo del límite
\[
\limjc{\frac{m_0}{\sqrt{1 - \frac{v^2}{c^2}}}}{v}{c^-}.
\]}%
Para poder utilizar la primera parte del teorema~(\ref{teo:LimUnoSobreCero}), primeramente
expresemos $m(v)$ de la siguiente manera:
\[
m(v) = \frac{1}{\frac{1}{m_0}\sqrt{1 - \frac{v^2}{c^2}}}.
\]
Ahora, podemos definir la función $f$ así:
\[
f(v) = \frac{1}{m_0}\sqrt{1 - \frac{v^2}{c^2}}
\]
para todo $v \in [0,c[$.

Lo que ahora tenemos que probar es que $f(v)$ tiende a $0^+$ cuando $x$ tiende a $c$ por la
izquierda.

Por un lado, tenemos que
\[
f(v) > 0
\]
para todo $v \in [0,c[$.

Por otro lado, cuando $v$ tiende a $c$ por la izquierda, entonces tenemos que:
\[
\limjc{\frac{v}{c}}{v}{c^-} = 1^-,
\]
pues $v < c$. Por lo tanto, podemos concluir que:
\[
\limjc{1 - \frac{v^2}{c^2}}{v}{c^-} = 0^+.
\]
Así que, obtenemos que:
\[
\limjc{\frac{1}{m_0}\sqrt{1 - \frac{v^2}{c^2}}}{v}{c^-} = 0^+.
\]

Por lo tanto, por el teorema~\ref{teo:LimUnoSobreCero}, podemos concluir que
\[
\limjc{m(v)}{v}{c^-} =\limjc{\frac{1}{\frac{1}{m_0}\sqrt{1 - \frac{v^2}{c^2}}}}{v}{c^-} =
+\infty.
\]
Es decir, cuando la velocidad de un cuerpo es cercana a la velocidad de la luz, su masa crece
indefinidamente.
\end{exemplo}

Probemos a continuación el primer numeral del teorema~\ref{teo:LimGeneralInfUnoSobreCero}. El
segundo se deja a que el lector lo realice como un ejercicio.
\begin{proof}
Supongamos que
\begin{equation}
\label{eq:Lim013}
\limjc{f(x)}{x}{a} = 0^+.
\end{equation}
Vamos a demostrar que
\[
\limjc{\frac{1}{f(x)}}{x}{a} = +\infty.
\]

Para ello, sea $R > 0$. Debemos hallar $\delta > 0$ tal que
\begin{equation}
\label{eq:Lim014}
\frac{1}{f(x)} > R
\end{equation}
siempre que $0 < |x - a| < \delta$.

La igualdad~(\ref{eq:Lim013}) implica, por un lado, que existe $\delta_1 > 0$ tal que
\begin{equation}
\label{eq:Lim015}
f(x) > 0
\end{equation}
siempre que $0 < |x - a| < \delta_1$.

Por otro lado, la igualdad~(\ref{eq:Lim013}) también implica que existe $\delta_2 > 0$ tal que
\begin{equation}
\label{eq:Lim016}
|f(x) - 0| = |f(x)| < \frac{1}{R}
\end{equation}
siempre que $0 < |x - a| < \delta_2$, ya que $\frac{1}{R} > 0$.

Por lo tanto, si tomamos $\delta = \min\{\delta_1,\delta_2\}$, entonces tenemos que las
desigualdades~(\ref{eq:Lim015}) y (\ref{eq:Lim016}) se verifican simultáneamente si $0 < |x - a| <
\delta$. Es decir, se verifica que
\[
0 < f(x) = |f(x)| < \frac{1}{R}
\]
siempre que $0 < |x - a| < \delta$.

Pero estas dos desigualdades últimas implican que
\[
\frac{1}{f(x)} > R
\]
siempre que $0 < |x - a| < \delta$.

En resumen, hemos probado que
$\displaystyle
\limjc{\frac{1}{f(x)}}{x}{a} = +\infty.
$
La demostración de que esta última igualdad implica la igualdad~(\ref{eq:Lim013}) es similar, por
lo que se deja al lector que la haga como un ejercicio.
\end{proof}

\subsection{Propiedades de los límites infinitos}
La definición de límites infinitos nos permite si un límite es $+\infty$ o $-\infty$, pero no nos
permite saber de antemano si un límite es o no infinito. Los límites infinitos poseen propiedades
que nos permiten determinar si una función diverge a partir de saber que ciertas funciones
divergen. En el teorema~\ref{teo:LimGeneralInfUnoSobreCero} se presentan dos de esas propiedades.
En el siguiente teorema, se reúnen algunas de las propiedades más útiles, que también son
verdaderas si los límites son laterales únicamente. Pero antes de enunciarlo, es necesario hablar
del concepto de \emph{función acotada} y su relación con el concepto de límite.

\begin{defical}[Función acotada]%
Sean $\funcjc{f}{\Dm(f)}{\mathbb{R}}$ y $A\subset \Dm(f)$. La función $f$ está:
\begin{enumerate}[leftmargin=*]
\item \emph{acotada superiormente en} $A$ si existe una constante $M$ tal que $f(x) < M$ para
    todo $x\in A$;

\item \emph{está acotada inferiormente en} $A$ si $f(x) > M$ para todo $x\in A$.

\item \emph{está acotada en $A$} si está acotada superiormente e inferiormente en $A$. Esto es equivalente a
    que exista un intervalo $]K,M[$ tal que $f(x) \in\ ]K,M[$. También es equivalente a la
    afirmación de que exista un número $P > 0$ tal que $|f(x)| < P$ para todo $x\in A$.
\end{enumerate}
Si $A = \mathbb{R}$, se dice, por ejemplo, ``$f$ está acotada \textit{siempre}'', en lugar de ``acotada en $\mathbb{R}$''.
\end{defical}

Veamos algunos ejemplos. La función $\funcjc{f}{\mathbb{R}}{\mathbb{R}}$ definida por
\[
f(x) = x
\]
\begin{enumerate}
\item está acotada en cualquier intervalo finito $[a,b]$ o $]a,b[$;
\item está acotada inferiormente, pero no superiormente en cualquier intervalo $[a,+\infty[$ o
    $]a,+\infty[$;
\item está acotada superiormente, pero no inferiormente en cualquier intervalo $]\!-\infty, a]$
    o $]\!-\infty,a[$.
\end{enumerate}

Las funciones $\sen$ y $\cos$ están siempre acotadas (es decir, están acotadas en $\mathbb{R}$), pues
\[
|\sen x| \leq 1 \yjc |\cos x| \leq 1
\]
para todo $x\in\mathbb{R}$.

A continuación, estudiemos la conexión entre el concepto de acotación y el de límite. Supongamos
que existe $L\in\mathbb{R}$ tal que $L = \displaystyle\limjc{f(x)}{x}{a}$. Sabemos, entonces, que
$f(x)$ puede estar tan cerca de $L$ como se quiera siempre que $x$ esté lo suficientemente cerca de
$a$. Esto significa que existirá un intervalo $I$ centrado en $a$ tal que $f$ esté acotada en $I -
\{a\}$. Probemos esta afirmación.

Sea $\epsilon > 0$. Existe, entonces, $\delta > 0$ tal que $|f(x) - L| < \epsilon$ siempre que $0 <
|x - a| < \delta$. Por lo tanto
\[
L -\epsilon < f(x) < L + \epsilon
\]
para todo $x\neq a$ tal que $a - \delta < x < a + \delta$.

Si definimos $K = L - \epsilon$, $M = L + \epsilon$ e $I = ]a - \delta, a + \delta[$, entonces
$f(x) \in\ ]K,M[$ para todo $x\in I - \{a\}$.

Resumamos estos razonamientos en el siguiente teorema.

\begin{teocal}\label{teo:LimLimAcotada}%
Si $L = \displaystyle\limjc{f(x)}{x}{a}$, entonces existe un intervalo abierto $I$, centrado en $a$ tal
que $f$ está acotada en $I - \{a\}$. Es decir, $f$ está acotada localmente cerca de $a$.
\end{teocal}

En el caso de que una función tienda a $+\infty$ o a $-\infty$, la función no está acotada
superiormente, en el primer caso, e inferiormente en el segundo. La demostración, que es similar a
la del caso de los límites finitos, la dejamos como un ejercicio para el lector. Enunciemos este
resultado como el siguiente teorema.

\begin{teocal}
Si $\displaystyle\limjc{f(x)}{x}{a} = +\infty$, entonces $f$ no está acotada superiormente en
ningún intervalo centrado en $a$. En cambio, si $\displaystyle\limjc{f(x)}{x}{a} = -\infty$, la
función $f$ no está acotada inferiormente en ningún intervalo centrado en $a$.
\end{teocal}

Ahora ya podemos formular algunas de las propiedades de los límites infinitos.

\begin{teocal}[Propiedades de límites infinitos]\label{teo:LimPropLimInf}%
Sean $f$ y $g$ dos funciones reales. Entonces:
\begin{enumerate}[leftmargin=*]
\item Si $\displaystyle\limjc{f(x)}{x}{a} = +\infty$ y $\displaystyle\limjc{g(x)}{x}{a} = +\infty $, entonces:
    \[
      \limjc{[f(x) + g(x)]}{x}{a} = +\infty \yjc
      \limjc{[f(x)g(x)]}{x}{a} = +\infty.
    \]
\item Si $\displaystyle\limjc{f(x)}{x}{a} = +\infty$ y $\lambda < 0$, entonces:
    \[
      \limjc{[\lambda f(x)]}{x}{a} = -\infty.
    \]
\item Si $\displaystyle\limjc{f(x)}{x}{a} = +\infty$ y $g$ está acotada inferiormente localmente cerca de $a$, entonces:
    \[
      \limjc{[f(x) + g(x)]}{x}{a} = +\infty
    \]
\item Si $\displaystyle\limjc{f(x)}{x}{a} = +\infty$ y $g$ está acotada inferiormente por un número positivo localmente cerca de $a$, entonces:
    \[
      \limjc{[f(x)g(x)]}{x}{a} = +\infty.
    \]
\item Si $f$ está acotada inferiormente por un número positivo localmente cerca de $a$ y $\displaystyle\limjc{g(x)}{x}{a} = 0^+$, entonces:
    \[
      \limjc{\frac{f(x)}{g(x)}}{x}{a} = +\infty.
    \]
\item Si $f$ está acotada localmente cerca de $a$ y $\displaystyle\limjc{g(x)}{x}{a} = +\infty$,
    entonces:
    \[
      \limjc{\frac{f(x)}{g(x)}}{x}{a} = 0.
    \]
\end{enumerate}
\end{teocal}

Se pueden formular propiedades similares cuando las funciones tienden a $-\infty$. Esto se hará en
el siguiente capítulo. En ese capítulo también se estudiarán algunas de las demostraciones. Sin
embargo, el lector interesado en adquirir una comprensión más profunda del concepto de límite,
debería realizar por sí mismo estas demostraciones.

A continuación, veamos algunos ejemplos del uso de las propiedades enunciadas.

\begin{exemplo}[Solución]{%
Calcular $\displaystyle\limjc{\left(\frac{1}{x^2} + \sen x\right)}{x}{0}$.}%
Sean $f(x) = \displaystyle \frac{1}{x^2}$ y $g(x) = \sen x$. Entonces, por el
teorema~\ref{teo:LimGeneralInfUnoSobreCero}, tenemos que
\[
\limjc{f(x)}{x}{0} = +\infty
\]
ya que
\[
\limjc{x^2}{x}{0} = 0^+.
\]
Por otro lado, para todo $x$, se tiene que $\sen x \geq -1$; es decir, la función $g$ está acotada
inferiormente en cualquier intervalo que contenga a $0$. Por lo tanto, por la tercera propiedad del
teorema~\ref{teo:LimPropLimInf}, podemos afirmar que:
\[
\limjc{\left(\frac{1}{x^2} + \sen x\right)}{x}{0} = \limjc{f(x) + g(x)}{x}{0} = +\infty.
\]
\end{exemplo}

\begin{exemplo}[Solución]{%
Calcular $\displaystyle\limjc{\frac{x^2 + 1}{x^3 - 1}}{x}{1^+}$.}%
Sean $f(x) = x^2 + 1$ y $\displaystyle g(x) = \frac{1}{x^3 - 1}$.

Por un lado, para todo $x\in\mathbb{R}$, se verifica que:
\[
f(x) = x^2 + 1 \geq 1.
\]
Por lo tanto, $f$ está acotada inferiormente por un número positivo en cualquier intervalo cuyo
extremo inferior sea el número $1$.

Por otro lado, dado que
\[
\limjc{x^3 - 1}{x}{1^+} = 0^+,
\]
entonces
\[
\limjc{g(x)}{x}{1^+} = \limjc{\frac{1}{x^3 - 1}}{x}{1^+} = +\infty.
\]

De modo que, si aplicamos la cuarta propiedad enunciada en el teorema~\ref{teo:LimPropLimInf},
podemos concluir que
\[
\limjc{\frac{x^2 + 1}{x^3 - 1}}{x}{1^+} = \limjc{f(x)g(x)}{x}{1^+} = +\infty.
\]

Podemos llegar a la misma conclusión de otra manera. En efecto, dado que $x\neq 0$, podemos
escribir lo siguiente:
\[
\frac{x^2 + 1}{x^3 - 1} = \frac{\displaystyle\frac{x^2}{x^2} + \frac{1}{x^2}}%
{\displaystyle\frac{x^3}{x^2} - \frac{1}{x^2}} =
\frac{\displaystyle 1 + \frac{1}{x^2}}{\displaystyle x - \frac{1}{x^2}},
\]
podemos utilizar la cuarta propiedad del teorema~\ref{teo:LimPropLimInf} para calcular el límite.

Para ello, definamos $f(x) = 1 + \frac{1}{x^2}$ y $g(x) = x - \frac{1}{x^2}$. Entonces tenemos que
$f$ está acotada inferiormente por el número $1$ en un intervalo abierto cuyo extremo inferior es
el número $1$, ya que $\displaystyle f(x) = 1 + \frac{1}{x^2} > 1$ pues $\displaystyle\frac{1}{x^2}
> 0$. Adicionalmente, tenemos que
\[
\limjc{g(x)}{x}{1^+} = 1 - 1 = 0
\]
y, como
\[
x - \frac{1}{x^2} > 0,
\]
pues
\begin{align*}
x > 1 &\Longrightarrow x^2 > 1 \\
&\Longrightarrow \frac{1}{x^2} < 1 \\
&\Longrightarrow -\frac{1}{x^2} > - 1 \\
&\Longrightarrow x - \frac{1}{x^2} > 0,
\end{align*}
tenemos que
\[
\limjc{g(x)}{x}{1^+} = 0^+.
\]

Entonces, si aplicamos la cuarta propiedad obtenemos que:
\[
\limjc{\frac{x^2 + 1}{x^3 - 1}}{x}{1^+} = \limjc{\frac{f(x)}{g(x)}}{x}{1^+} = +\infty.
\]
\end{exemplo}

Cuando $\displaystyle\limjc{f(x)}{x}{a} = \limjc{g(x)}{x}{a} = + \infty$, el
teorema~\ref{teo:LimPropLimInf} no dice nada sobre $\displaystyle\limjc{[f(x) - g(x)]}{x}{a}$ ni
sobre $\displaystyle\limjc{\frac{f(x)}{g(x)}}{x}{a}$. Los siguientes ejemplos nos van a decir por
qué.

\begin{exemplo}[Solución]{%
Calcular los límites
\[
\limjc{f(x)}{x}{a^+}, \quad \limjc{g(x)}{x}{a^+} \yjc \limjc{[f(x) - g(x)]}{x}{a^+}
\]
si:
\begin{enumerate}
\item $f(x) = \displaystyle\sqrt{\frac{x + 1}{x}}$ y $g(x) = \displaystyle\sqrt{\frac{1}{x}}$;
    $a = 0$.
\item $f(x) = \displaystyle\frac{1}{(x - 1)^3}$ y $g(x) = \displaystyle\frac{1}{(x - 1)^2}$; $a
    = 1$.
\end{enumerate}}
\begin{enumerate}[leftmargin=*]
\item Tenemos que $\displaystyle f(x) = \sqrt{\frac{x + 1}{x}} = \sqrt{1 + \frac{1}{x}}$, que
    $\displaystyle\limjc{\frac{1}{x}}{x}{0^+} = + \infty$ y que $\displaystyle\limjc{1 +
    \frac{1}{x}}{x}{0^+} = + \infty$ (por el numeral tres del teorema~\ref{teo:LimPropLimInf}).
    Es decir:
    \begin{equation}
    \label{eq:Lim017}
    \limjc{\frac{x + 1}{x}}{x}{o^+} = +\infty.
    \end{equation}

    Por otro lado, es fácil demostrar que si $\displaystyle\limjc{\phi(x)}{x}{a} = +\infty$,
    entonces\footnote{En los ejercicios de esta sección, se propone al lector la demostración
    de esta propiedad.} $\displaystyle\limjc{\sqrt{\phi(x)}}{x}{a} = +\infty$. Si aplicamos
    esta propiedad en~(\ref{eq:Lim017}), entonces tenemos que
    \[
      \limjc{f(x)}{x}{0^+} = \limjc{\sqrt{\frac{x+1}{x}}}{x}{0^+} = +\infty.
    \]

    Un procedimiento similar nos permite afirmar que
    \[
      \limjc{g(x)}{x}{0^+} = \limjc{\sqrt{\frac{1}{x}}}{x}{0^+} = +\infty.
    \]

    Ahora calculemos el límite de la diferencia $[f(x) - g(x)]$.

    Para empezar:
    \[
    f(x) - g(x) = \sqrt{1 + \frac{1}{x}} - \sqrt{\frac{1}{x}}
    = \frac{1}{\sqrt{1 + \frac{1}{x}} + \sqrt{\frac{1}{x}}}.
    \]
    Puesto que
    \[
      \limjc{\sqrt{1 + \frac{1}{x}}}{x}{0^+} = +\infty \yjc
      \limjc{\sqrt{\frac{1}{x}}}{x}{0^+} = +\infty,
    \]
    por el primer numeral del teorema~\ref{teo:LimPropLimInf}, tenemos que
    \[
      \limjc{\left(\sqrt{1 + \frac{1}{x}} + \sqrt{\frac{1}{x}}\right)}{x}{0^+} = +\infty,
    \]
    de donde, por el teorema~\ref{teo:LimGeneralInfUnoSobreCero} (página
    \pageref{teo:LimGeneralInfUnoSobreCero}), podemos concluir que:
    \[
      \limjc{[f(x) - g(x)]}{x}{0^+} =
      \limjc{\left(\frac{1}{\sqrt{1 + \frac{1}{x}} + \sqrt{\frac{1}{x}}}\right)}{x}{0^+} = 0.
    \]

\item Dado que, para $x > 1$, tenemos que $(x - 1)^3 > 0$, entonces $\displaystyle\limjc{(x -
    1)^3}{x}{1^+} = 0^+$. Por lo tanto, por el teorema~\ref{teo:LimGeneralInfUnoSobreCero},
    podemos concluir que:
    \[
      \limjc{f(x)}{x}{1^+} = \limjc{\frac{1}{(x - 1)^3}}{x}{1^+} = +\infty.
    \]

    Un razonamiento similar nos conduce a concluir que:
    \[
      \limjc{g(x)}{x}{1^+} = \limjc{\frac{1}{(x - 1)^2}}{x}{1^+} = +\infty.
    \]

    Por último, puesto que
    \[
      f(x) - g(x) = \frac{1}{(x - 1)^3} - \frac{1}{(x - 1)^2} = (2 - x)\frac{1}{(x - 1)^3},
    \]
    tenemos que, por la cuarta propiedad del teorema~\ref{teo:LimPropLimInf}, tenemos que
    \[
      \limjc{[f(x) - g(x)]}{x}{1^+} = \limjc{(2 - x)\frac{1}{(x - 1)^3}}{x}{1^+} = +\infty,
    \]
    ya que, para $x \in\ ]1, \frac{3}{2}[$, tenemos que
    \[
        2 - x > \frac{1}{2},
    \]
    es decir, $(2 - x)$ está acotado inferiormente por un número positivo.
\end{enumerate}
\end{exemplo}

En el primer límite de este ejemplo, obtenemos que el límite de la diferencia de dos funciones que
tienden a $+\infty$ es $0$; en el segundo, en cambio, el límite es $+\infty$. Y hay casos en que
ese límite puede ser un número real diferente de $0$, $-\infty$, etcétera. En otras palabras, no
podemos concluir nada sobre el límite de la diferencia; cada caso deberá ser analizado con sus
particularidades.

Ocurre algo similar para el caso de la división de dos funciones que tienden a $+\infty$. El
siguiente ejemplo ilustra lo dicho.

\begin{exemplo}[Solución]{%
Calcular el límite $\displaystyle\limjc{\frac{f(x)}{g(x)}}{x}{0^+}$ si:
\begin{multicols}{3}
\begin{enumerate}
\item $f(x) = g(x) = \displaystyle\frac{1}{x}$.
\item $f(x) = \displaystyle\frac{1}{x}$, $g(x) = \displaystyle\frac{1}{x^2}$.
\item $f(x) = \displaystyle\frac{1}{x^2}$, $g(x) = \displaystyle\frac{1}{x}$.
\end{enumerate}
\end{multicols}
}%
En los tres ejemplos, tenemos que:
\[
\limjc{f(x)}{x}{0^+} = \limjc{g(x)}{x}{0^+} = +\infty.
\]
Veamos lo que sucede con el límite del cociente entre $f(x)$ y $g(x)$ en cada caso.
\begin{enumerate}[leftmargin=*]
\item Tenemos que $\displaystyle\frac{f(x)}{g(x)} = 1$. Por lo tanto,
    $\displaystyle\limjc{\frac{f(x)}{g(x)}}{x}{0^+} = 1.$

\item Tenemos que $\displaystyle\frac{f(x)}{g(x)} = \dfrac{\dfrac{1}{x}}{\dfrac{1}{x^2}} = x.$
    Por lo tanto: $\displaystyle\limjc{\frac{f(x)}{g(x)}}{x}{0^+} = 0.$

\item Tenemos que $\displaystyle\frac{f(x)}{g(x)} = \dfrac{\dfrac{1}{x^2}}{\dfrac{1}{x}} =
    \frac{1}{x}$. Por lo tanto, $\displaystyle\limjc{\frac{f(x)}{g(x)}}{x}{0^+} = +\infty$.
\end{enumerate}
\end{exemplo}

Los límites de diferencias y cocientes de funciones que tienden a $+\infty$ se suelen representar
por $+\infty - \infty$ y $\frac{+\infty}{+\infty}$. Puesto que según las funciones, estos límites
pueden tomar cualquier número real, o ser infinitos, se los denomina \emph{formas indeterminadas},
y cada caso deberá ser resuelto de modo particular.

Estas formas indeterminadas ya aparecieron anteriormente; es el caso del cociente de dos funciones
que convergen cada una a $0$. Por ejemplo, tenemos los límites:
\[
\limjc{\frac{\sen x}{x}}{x}{0} = 1 \yjc \limjc{\frac{1 - \cos x}{x^2}}{x}{0} = \frac{1}{2}.
\]
Uno de los resultados estudiados en este capítulo que ayuda a calcular los límites del tipo
$\frac{0}{0}$ es el teorema~\ref{eq:limitegeneral} enunciado \vpageref{eq:limitegeneral}.

\subsection{Límites al infinito}
En las ciencias de la computación, para un mismo problema, se han desarrollado diversos algoritmos
para resolverlo. Por ejemplo, para ordenar un conjunto de datos (los números de cédula de los
ciudadanos empadronados para una elección), existen el ordenamiento por inserción, Quicksort,
Mergesort, entre otros.

La complejidad computacional es una rama de las ciencias de la computación dedicada a determinar
aproximadamente el número de instrucciones que un algoritmo realiza para una determinada tarea, y a
desarrollar nuevos algoritmos que hagan las mismas tareas en un número menor de instrucciones.

Por ejemplo, para multiplicar dos matrices cuadradas de orden $n$, existen al menos dos algoritmos:
el que utiliza la definición de multiplicación y que realiza
\[
  f(n) = 2n^3 - n^2
\]
operaciones (sumas, restas y multiplicaciones), y el algoritmo de Winograd (1970), que utiliza
\[
  g(n) = 2n^3 + 3n^2 - 2n
\]
operaciones.

La complejidad computacional busca determinar cuál de los dos algoritmos es más eficiente, para lo
cual se determina la relación entre $f(n)$ y $g(n)$ cuando $n$ es un valor grande. De manera más
precisa, el problema se plantea en términos de límites: ?`a dónde tiende el cociente $\displaystyle
\frac{f(n)}{g(n)}
$
cuando $n$ crece indefinidamente?

En otras palabras, la pregunta que se formula es: ?`existe un número real tan cerca como se quiera
del cociente $\displaystyle\frac{f(n)}{g(n)}$ para todo $n$ a partir de un cierto $n_0$?

El concepto de límite al infinito permite responder esta pregunta.

\begin{defical}[Límites al infinito]
$\displaystyle L=\lim_{x \to +\infty}f(x)$ si y solo si para todo $\epsilon > 0$, existe $R > 0$
tal que
\[
|f(x) - L| < \epsilon,
\]
siempre que $x\in\Dm(f)$ y $x > R$.

Análogamente: $\displaystyle L=\lim_{x \to -\infty}f(x)$ si y solo si para todo $\epsilon > 0$,
existe $M < 0$ tal que
\[
|f(x) - L| < \epsilon,
\]
siempre que $x\in\Dm(f)$ y $x < M$.
\end{defical}

De la definición se desprende que
\[
\limjc{x}{x}{+\infty} = +\infty \yjc \limjc{x}{x}{-\infty} = -\infty.
\]
Y de estos resultados, podemos demostrar que:
\[
\limjc{x^n}{x}{+\infty} = +\infty
\]
y
\[
\limjc{x^n}{x}{-\infty} =
\begin{cases}
+\infty & \text{si $n$ es par,} \\
-\infty & \text{si $n$ es impar}.
\end{cases}
\]

En el caso de que $x$ tienda a $+\infty$, estamos exigiendo que exista un intervalo del tipo
$]R,+\infty[$ con $R > 0$ que esté contenido en el dominio de la función $f$; esto se traduce en
que podemos excluir el caso $x \leq 0$ cuando analicemos un límite donde $x$ tiende a $+\infty$. De
manera similar, cuando $x$ tiende a $-\infty$, el dominio de la función debe contener un intervalo
del tipo $]-\infty,M[$ con $M < 0$. Por lo tanto, podemos excluir el caso $x\geq 0$ cuando se
analice un límite donde $x$ tiende a $-\infty$.

\begin{exemplo}[Solución]{%
Mostrar que $\displaystyle\limjc{\frac{1}{x}}{x}{+\infty} = 0$ y
$\displaystyle\limjc{\frac{1}{x}}{x}{-\infty} = 0$.
}%
\begin{enumerate}[leftmargin=*]
\item Sea $\epsilon > 0$. Debemos encontrar $R > 0$ tal que
      \begin{equation}
      \label{eq:Lim018}
        \left\lvert\frac{1}{x} - 0\right\rvert < \epsilon
      \end{equation}
      para todo $x > R$.

      Para ello, procedamos de manera similar al caso de los límites finitos. Analicemos, pues,
      la desigualdad~(\ref{eq:Lim018}).

      Como $x$ tiende a $+\infty$, supongamos que $x > 0$. Entonces, se tienen las siguientes
      equivalencias:
      \begin{align*}
      \left|\frac{1}{x} - 0\right| < \epsilon &\Longleftrightarrow
      \frac{1}{x} < \epsilon \\
      &\Longleftrightarrow x > \frac{1}{\epsilon}.
      \end{align*}
      Por lo tanto, si definimos $R = \frac{1}{\epsilon}$, obtendremos la
      desigualdad~(\ref{eq:Lim018}) siempre que $x > R$.

\item Sea $\epsilon > 0$. Debemos hallar $M < 0$ tal que
      \begin{equation}
      \label{eq:Lim019}
        \left\lvert\frac{1}{x} - 0\right\rvert < \epsilon
      \end{equation}
      para todo $x < M$.

      Como $x$ tiende a $-\infty$, supongamos que $x < 0$. Dado que
      \[
        \left\lvert\frac{1}{x} - 0\right\rvert < \epsilon \Longleftrightarrow
        -\frac{1}{x} < \epsilon \Longleftrightarrow x > -\frac{1}{\epsilon},
      \]
      si definimos $M = -\frac{1}{\epsilon}$, la desigualdad~(\ref{eq:Lim019}) es verdadera
      para todo $x > M$.
\end{enumerate}
\end{exemplo}

En la sección anterior se definió $\displaystyle\limjc{f(x)}{x}{a} = L^+$ y
$\displaystyle\limjc{f(x)}{x}{a} = L^-$. Podemos extender estas definiciones cuando, en lugar de
$a$ tenemos $+\infty$ o $-\infty$. La única diferencia está que, $f(x)$ debe ser mayor que $L$ o
menor que $L$ en un intervalo del tipo $]R,+\infty[$ con $R > 0$ cuando $x$ tiende a $+\infty$, y
en un intervalo del tipo $]-\infty,M[$ con $M < 0$ cuando $x$ tiende a $-\infty$.

Con estas extensiones, podemos reformular los dos límites del ejemplo de la siguiente manera:
\[
\limjc{\frac{1}{x}}{x}{+\infty} = 0^+ \yjc \limjc{\frac{1}{x}}{x}{-\infty} = 0^-.
\]

Examinemos un ejemplo más.

\begin{exemplo}[Solución]{%
Demostrar que:
\[
1 = \lim_{x \to +\infty}\frac{x+1}{x-1}.
\]
}%

Sea $\epsilon > 0$. Buscamos un número $R > 0$ tal que, si definimos
\[
f(x) = \frac{x+1}{x-1},
\]
se verifique que
\begin{equation}
\label{eq:Lim020}
\left|f(x)-1\right| < \epsilon
\end{equation}
siempre que $x > R$.

Podemos suponer que $x > 0$; más aún, podemos suponer que $x > 1$, así $f(x) > 1$, con lo cual
restringiremos el análisis a un intervalo donde $f$ es positiva.

Por otro lado, dado que:
\begin{equation*}
	f(x) - 1 = \frac{x+1}{x-1}-1 = \frac{x+1-x+1}{x-1} = \frac{2}{x - 1} > 1,
\end{equation*}
pues $x > 1$, tenemos que
\[
|f(x) - 1| = \frac{2}{x - 1}.
\]

Por lo tanto, para que se verifique la desigualdad~(\ref{eq:Lim020}), es suficiente que se cumpla
lo siguiente:
\[
\frac{2}{x - 1} < \epsilon.
\]
Pero:
\begin{align*}
\frac{2}{x - 1} < \epsilon &\Longleftrightarrow \frac{x - 1}{2} > \frac{1}{\epsilon} \\
&\Longleftrightarrow x - 1 > \frac{2}{\epsilon} \\
&\Longleftrightarrow x > 1 + \frac{2}{\epsilon}.
\end{align*}
Es decir, para que se verifique la desigualdad~(\ref{eq:Lim020}), es suficiente que $x > 1$ y que
\[
x > 1 + \frac{2}{\epsilon}.
\]

Dado que $\epsilon > 0$, entonces
\[
1 + \frac{2}{\epsilon} > 1,
\]
por lo que podemos elegir $R = 1 + \frac{2}{\epsilon}$ para estar seguros de que la
desigualdad~(\ref{eq:Lim020}) sea verdadera para todo $x > R$.

Podemos también afirmar que
\[
\limjc{\frac{x + 1}{x - 1}}{x}{+\infty} = 1^+,
\]
ya que $f(x) > 1$ para $x > 1$.
\end{exemplo}

Igual a lo que sucede con los límites finitos y los infinitos, existen propiedades de los límites
al infinito que nos permiten calcular los límites de muchas funciones a partir de ciertos límites
que ya nos son conocidos. Estas propiedades son similares a las de los límites finitos: el límite
de una suma, resta, multiplicación, división, raíz, composición, es la suma, resta, etcétera, de
los límites cuando estos existen con las correspondientes restricciones para el caso de la división
y las raíces pares. En otras palabras, en todos los teoremas que se enuncian las propiedades de los
límites finitos se pueden sustituir el número $a$ (a donde tiende $x$) por $+\infty$ o $-\infty$.
En aquellas propiedades en las que se exige un comportamiento de $f$ en un intervalo centrado en $a$,
en el caso de los límites al infinito, ese comportamiento deberá suceder en un intervalo del tipo
$]R,+\infty[$ con $R > 0$ o $]-\infty,M[$ con $M < 0$.

Las demostraciones son dejadas para que el lector las realice, pues no ofrecen ninguna dificultad y
son, más bien, un buen entrenamiento para su comprensión del concepto de límite. A continuación, se
ofrecen algunos ejemplos del uso de las propiedades mencionadas.

\begin{exemplo}[Solución]{%
Calcular $\displaystyle\limjc{\frac{3x + 1}{2x - 5}}{x}{-\infty}$.
}%
Puesto que $x\neq 0$, tenemos que
\[
\frac{3x + 1}{2x - 5} = \dfrac{3 + \dfrac{1}{x}}{2 - \dfrac{5}{x}}
\]
y $\displaystyle\limjc{\frac{1}{x}}{x}{-\infty} = 0$, entonces:
\begin{align*}
\limjc{\frac{3x + 1}{2x - 5}}{x}{-\infty} &=
\limjc{\dfrac{3 + \dfrac{1}{x}}{2 - \dfrac{5}{x}}}{x}{-\infty} \\
&= \frac{3 + 0}{2 - 0} = \frac{3}{2}.
\end{align*}
\end{exemplo}

\begin{exemplo}[Solución]{%
Estudiemos la relación entre las complejidades computacionales de los dos algoritmos para
multiplicar matrices dadas por las funciones
\[
f(x) = 2x^3 - x^2 \yjc g(x) = 2x^3 + 3x^2 - 2x.
\]
}%
Para ello, vamos a calcular el límite
\[
\limjc{\frac{f(x)}{g(x)}}{x}{+\infty} = \limjc{\frac{2x^3 - x^2}{2x^3 + 3x^2 - 2x}}{x}{+\infty}.
\]

Puesto que:
\[
\frac{2x^3 - x^2}{2x^3 + 3x^2 - 2x} = \dfrac{2 - \dfrac{1}{x}}{2 + 3\dfrac{1}{x} - 2\dfrac{1}{x^2}}
\]
y que
\[
\limjc{\frac{1}{x}}{x}{+\infty} = \limjc{\frac{1}{x^2}}{x}{+\infty} = 0,
\]
podemos concluir que:
\[
\limjc{\frac{2x^3 - x^2}{2x^3 + 3x^2 - 2x}}{x}{+\infty} = \frac{2 - 0}{2 + 3\times 0 - 2\times 0}
= 1.
\]
Es decir: $\displaystyle\limjc{\frac{f(x)}{g(x)}}{x}{+\infty} = 1$.

?`Cómo se interpreta este resultado?
Sea $\epsilon > 0$. Entonces, existe $R > 0$ tal que
\[
\left\lvert\frac{f(x)}{g(x)} - 1\right\rvert < \epsilon
\]
para todo $x > R$. Es decir,
$
\displaystyle
1 - \epsilon < \frac{f(x)}{g(x)} < 1 + \epsilon
$
siempre que $x > R$. Más aún, dado que $g(x) > 0$ para todo $x > 0$, se verifica para todo $x > R$:
\[
(1 - \epsilon)g(x) < f(x) < (1 + \epsilon)g(x).
\]


Estas dos desigualdades nos dicen algo importante: a partir de un cierto orden para las matrices
($x$ representa lo que $n$ en el ejemplo de inicio de esta sección), $f(x)$ está acotada por arriba
y por abajo por $g(x)$ multiplicada por sendas constantes que, si elegimos $\epsilon$ muy pequeño
(algo que podemos hacer a nuestro gusto), dichas constantes son cercanas a $1$. En otras palabras,
los valores de $f(x)$ y de $g(x)$ son similares, por lo que, cuando $x$ es grande, el número de
operaciones que realizan ambos algoritmos no difieren de manera significativa. En este sentido,
ambos algoritmos son o igual de buenos o igual de malos.
\end{exemplo}

Para terminar esta sección, podemos reunir los límites infinitos y al infinito. De manera más
precisa, podemos definir los siguientes límites:
\begin{align*}
\lim_{x \to +\infty}f(x) & = +\infty & \lim_{x \to +\infty}f(x) & = -\infty \\
\lim_{x \to -\infty}f(x) & = +\infty & \lim_{x \to -\infty}f(x) & = -\infty
\end{align*}

Por ejemplo: $\displaystyle \lim_{x \to -\infty}f(x)$ se tiene cuando para todo $R > 0$, existe $M
> 0$ tal que
$\displaystyle
f(x) > R
$
para todo $x\in\Dm(f)$ y $x < -M$.

Como un buen ejercicio, el lector debería formular las restantes definiciones.


La propiedades de estos límites se pueden expresar a través de las propiedades de los límites
infinitos y los límites al infinito. Por ejemplo, si
\[
\limjc{f(x)}{x}{+\infty} = -\infty,
\]
entonces
\[
\limjc{\frac{1}{f(x)}}{x}{+\infty} = 0^-.
\]
O, si
\[
\limjc{f(x)}{x}{-\infty} = +\infty \yjc \limjc{g(x)}{x}{-\infty} = +\infty,
\]
entonces:
\[
\limjc{[f(x)+g(x)]}{x}{-\infty} = +\infty \yjc \limjc{f(x)g(x)}{x}{-\infty} = +\infty.
\]

Por supuesto, también son indeterminados límites del tipo
\[
\limjc{[f(x) - g(x)]}{x}{+\infty} \yjc \limjc{\frac{f(x)}{g(x)}}{x}{+\infty}
\]
si tanto $f(x)$ como $g(x)$ tienden, por ejemplo, a $-\infty$.

En los ejercicios de esta sección, se pide demostrar algunas de las propiedades de estos límites.

\newpage
\subsection{Ejercicios}
\begingroup
\small
\begin{multicols}{2}
\begin{enumerate}[leftmargin=*]
\item Mediante la definición correspondiente, demuestre que:
            \begin{enumerate}[leftmargin=*]
            \item $\displaystyle \lim_{x\to -\infty}\frac{2x-1}{x+1}=2$.
            \item $\displaystyle \lim_{x\to +\infty}\frac{x+2}{x-1}=1$.
            \item $\displaystyle \lim_{x\to 1}\frac{(2-x)(2x-1)}{|x-1|}=+\infty$.
            \item $\displaystyle \lim_{x\to 0}\frac{x-2}{x^2+|x|}=-\infty$.
            \item $\displaystyle \limjc{\frac{(2x - 1)(7 - 2x)}{|x^2 - 3x + 2|}}{x}{2} =
                +\infty$.
            \item $\displaystyle \limjc{\frac{\sen x}{x}}{x}{-\infty} = 0$.
            \item $\displaystyle \limjc{\frac{x^2 - 2x}{|x^2 - 1|}}{x}{1} = -\infty$.
            \item $\displaystyle \limjc{\frac{2x^2 - 5x + 1}{x^2 + 1}}{x}{+\infty} = 2$.
            \item $\displaystyle \limjc{(1 - x^2)}{x}{+\infty} = -\infty$.
            \item $\displaystyle \limjc{\frac{-5 + 4x}{x^2 - 5x + 4}}{x}{1^+} =
                +\infty$.
             \end{enumerate}

\item Calcule los límites dados
            \begin{enumerate}
            \item $\displaystyle \lim_{x\to 1^+}\frac{1}{x^2-3x+2}$.
            \item $\displaystyle \lim_{x\to 1^-}\frac{x}{x^2-3x+2}$.
            \item $\displaystyle \lim_{x\to -\infty}\frac{x-2}{x^2+3}$.
            \item $\displaystyle \lim_{x\to +\infty}\frac{x-2}{x^2+3}$.
            \item $\displaystyle \lim_{x\to -\infty}\frac{x+1}{|x+1|}$.
            \item $\displaystyle \lim_{x\to +\infty}\frac{x+1}{|x+1|}$.
            \item $\displaystyle \limjc{\frac{(2x - 1)(4 - 3x)}{|x - 1|}}{x}{1}$.
            \item $\displaystyle \limjc{\frac{2x - 5}{|x - 2||x - 1|}}{x}{2}$.
            \end{enumerate}

\item Demostrar las siguientes propiedades de límites infinitos y al infinito:
      \begin{enumerate}[leftmargin=*]
      \item Si $\displaystyle \limjc{f(x)}{x}{+\infty} = L > 0$, entonces
      \[
          \limjc{\sqrt{f(x)}}{x}{+\infty} = \sqrt{L}.
      \]

      \item Si $\displaystyle \limjc{f(x)}{x}{-\infty} = -\infty$, entonces
      \[
          \limjc{\frac{1}{f(x)}}{x}{-\infty} = 0^-.
      \]

      \item Si $f > M > 0$ y $\displaystyle \limjc{g(x)}{x}{+\infty} = 0^+$, entonces
      \[
          \limjc{\frac{f(x)}{g(x)}}{x}{+\infty} = +\infty.
      \]

      \item Si $f$ está acotada y $\displaystyle \limjc{g(x)}{x}{+\infty} = +\infty$,
          entonces
      \[
            \limjc{\frac{f(x)}{g(x)}}{x}{+\infty} = 0.
      \]
      \end{enumerate}

  \item Una recta no vertical de ecuación $y = ax + b$ es una \emph{asíntota}\footnote{Cuando
      se estudie el concepto de derivada y se aplique a la obtención del gráfico de una curva,
      se hará un tratamiento más profundo del concepto de asíntota que el dado en este
      ejercicio} de la curva de ecuación $y = f(x)$ si una de las dos igualdades siguientes es
      verdadera (o ambas):
      \[
          \limjc{[f(x) - (ax + b)]}{x}{+\infty} = 0
      \]
      o
      \[
      \limjc{[f(x) - (ax + b)]}{x}{-\infty} = 0.
      \]

      Una recta vertical de ecuación $x = a$ es una \emph{asíntota vertical} de la curva de
      ecuación $y = f(x)$ si una de las dos igualdades siguientes es verdadera (o ambas):
      \[
          \limjc{|f(x)|}{x}{a^+} = +\infty \quad\text{o}\quad
          \limjc{|f(x)|}{x}{a^-} = +\infty,
      \]
      y si $f$ es monótona cerca de $a$ por la derecha o por la izquierda, según el caso.
      Determine si la curva de ecuación $y = f(x)$ tiene asíntotas verticales u horizontales.
            \begin{enumerate}
            \item $\displaystyle f(x)=\frac{1}{x^2+x+1}$.
            \item $\displaystyle f(x)=\frac{2x^2+1}{x^2-3x+2}$.
            \item $\displaystyle f(x)=\frac{x^2}{x+2}$.
            \item $\displaystyle f(x)=\frac{3x^2-1}{2x^2+1}$.
            \item $\displaystyle f(x)=\frac{2}{1- |x+1|}$.
            \item $\displaystyle f(x)=\sqrt{\frac{x+1}{x-1}}$.
            \end{enumerate}

\item En cada uno de los siguientes dibujos se muestra el gráfico de la función $f$. Con la
    información contenida en él, encuentre, si existen, los siguientes límites de $f(x)$ cuando
    $x$ tiende a $+\infty$, $-\infty$, $a^+$ y $a^-$.

    \begin{enumerate}[leftmargin=*]
    \item
  \begin{center}
  \psset{xAxisLabel={},yAxisLabel={},plotpoints=1000}%
  \begin{psgraph}[arrows=->,ticks=none,labels=none](0,0)(-4,-1)(4,8){0.4\textwidth}{5cm}
  \uput[-90](4,0){$x$}%
  \uput[0](0,8){$y$}%

  \psplot{-4}{0.85}{1 x 1 sub abs div 0.25 add}%

  \psline[linestyle=dashed,linecolor=gray]%
    (1,0)(1,7)%
  \uput[-90](1,0){\footnotesize$a$}%

  \psplot[arrows=o-]{1}{2.5}{x 3 mul 1 sub}%

  \psline[linewidth=0.1pt]%
    (-0.1,1)(0.1,1)%
  \uput[0](0,1){\footnotesize$a$}
  \psline[linewidth=0.1pt]%
    (-0.1,2)(0.1,2)%
  \uput[180](0,2){\footnotesize$2a$}
\end{psgraph}
\end{center}

\item
\begin{center}
\psset{xAxisLabel={},yAxisLabel={},plotpoints=1000}%
\begin{psgraph}[arrows=->,ticks=none,labels=none](0,0)(-2.5,-1)(5.5,8){0.4\textwidth}{5cm}
  \uput[-90](5.5,0){$x$}%
  \uput[0](0,8){$y$}%

  \psplot{1.15}{5.25}{1 x 1 sub abs div 0.25 add}%

  \psline[linestyle=dashed,linecolor=gray]%
    (1,0)(1,7)%
  \uput[-90](1,0){\footnotesize$a$}%

  \psplot[arrows=-o]{-2.5}{1}{x neg 0.75 mul 1.75 add}%

  \psline[linewidth=0.1pt]%
    (-0.1,1)(0.1,1)%
  \uput[0](0,1){\footnotesize$a$}
  \psline[linewidth=0.1pt]%
    (-0.1,2)(0.1,2)%
  \uput[0](0,2){\footnotesize$2a$}
\end{psgraph}
\end{center}

\item

\begin{center}
\psset{xAxisLabel={},yAxisLabel={},plotpoints=1000}%
\begin{psgraph}[arrows=->,ticks=none,labels=none](0,0)(-3,-4)(5,3){0.4\textwidth}{5cm}
  \uput[-90](5,0){$x$}%
  \uput[0](0,3){$y$}%

  \psplot{-3}{0.8}{x dup 1 sub div}%

  \psline[linestyle=dashed,linecolor=gray]%
    (1,0)(1,-4)%
  \uput[-90](1,0){\footnotesize$a$}%

  \psplot[arrows=o-]{1}{4}%
      {x 1 sub dup mul 2 div neg 2.71828182 exch exp x 1 sub mul 1.5 div 4 neg mul 2 add}%

  \psline[linestyle=dashed,linecolor=gray]%
    (-3,1)(0,1)%
  \psline[linewidth=0.1pt]%
    (-0.1,1)(0.1,1)%
  \uput[0](0,1){\footnotesize$a$}

  \psline[linestyle=dashed,linecolor=gray]%
    (0,2)(4,2)%
  \psline[linewidth=0.1pt]%
    (-0.1,2)(0.1,2)%
  \uput[180](0,2){\footnotesize$2a$}
\end{psgraph}
\end{center}

\item
\begin{center}
\psset{xAxisLabel={},yAxisLabel={},plotpoints=1000}%
\begin{psgraph}[arrows=->,ticks=none,labels=none](0,0)(-4,-2)(4.5,3){0.4\textwidth}{5cm}
  \uput[-90](4.5,0){$x$}%
  \uput[0](0,3){$y$}%

  \psplot[algebraic]{1}{4}{sin(5*(x-1)) + 1}%

  \psline[linestyle=dashed,linecolor=gray]%
    (1,0)(1,2)%
  \uput[-90](1,0){\footnotesize$a$}%

  \psplot[arrows=-o]{-4}{1}%
      {x 0.5 sub dup mul 2 div neg 2.71828182 exch exp x 0.5 sub mul 1.5 div 4 mul 0.176662537 sub}%

  \psline[linestyle=dashed,linecolor=gray]%
    (0,2)(4,2)%
  \psline[linewidth=0.1pt]%
    (-0.1,1)(0.1,1)%
  \uput[180](0,1){\footnotesize$a$}

  \psline[linestyle=dashed,linecolor=gray]%
    (0,1)(4,1)%
  \psline[linewidth=0.1pt]%
    (-0.1,2)(0.1,2)%
  \uput[180](0,2){\footnotesize$2a$}
\end{psgraph}
\end{center}

    \end{enumerate}

\item Dada $\funcjc{f}{\mathbb{R}}{\mathbb{R}}$, si existen los límites $\displaystyle \limjc{\frac{f(x)}{x}}{x}{+\infty} = m$ y $\displaystyle \limjc{f(x) - mx}{x}{+\infty} = b$, pruebe que la recta de ecuación $y = mx + b$ es asíntota horizontal si $m = 0$, u oblicua si $m \neq 0$ de $f$ del gráfico de $f$ por la derecha. El resultado es análogo por la izquierda, tomando en cada caso el límite cuando $x$ tiende a $-\infty$. Aplique este resultado para hallar las asíntotas horizontales u oblicuas de $f$ si:
    \begin{enumerate}[leftmargin=*]
    \item $\displaystyle f(x) = \frac{x}{x^2 - 4x + 3}$.
    \item $\displaystyle f(x) = \frac{x}{\sqrt{x^2 + 3}}$.
    \item $\displaystyle f(x) = e^{-x^2} + 2$.
    \item $\displaystyle f(x) = \frac{3x^2 - 2x + \cos(x + 1)}{x - 1}$.
    \item $\displaystyle f(x) = \frac{2x^3 - 7e^{-x}}{x + 2}$.
    \item $\displaystyle f(x) = \frac{4x^3 - 7x + 2}{5x^3 + 2x + \cos x}$.
    \end{enumerate}
\end{enumerate}
\end{multicols}
\endgroup
