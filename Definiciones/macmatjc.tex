%%%%
%%Redefine el texto para \chapter
\def\chaptername{Capítulo}
\def\tablename{Tabla}
\def\contentsname{Índice}
%%%%

%%%%
%%%% 19/04/07
%%%%

%%%%\tccbjc: tikz caja centrada borde jc
%%%% #1: color de fondo, #2: color del borde, #3: color del texto
%%%% #4: separaci?n interna, #5: ancho de la caja, #6: el texto
\newcommand{\tccbjc}[6]{%
\begin{center}
\begin{tikzpicture}[rounded corners]
\draw (0,0) node[text justified, text width=#5, fill = #1, inner sep = #4] {\color{#3} #6};
\draw[color = #2, rounded corners] (current bounding box.south west) rectangle (current bounding
box.north east);
\end{tikzpicture}
\end{center}
} %%

%%%%
%\marcojc: #1 - ancho del texto en fracci?n de \textwidth
%\marcojc: #2 - ancho de separaci?n entre el marco y el texto en factor de ex
%\marcojc: #3 - color del marco
%\marcojc: #4 - color del texto
%\marcojc: #5 - color de fondo
%\marcojc: #6 - el texto
\newcommand{\marcojc}[6]{%
\begin{center}
\begin{tikzpicture}
\draw (0,0) node[text justified, text width=#1\textwidth, fill = #5, inner sep = #2ex]%
{\color{#4} #6};%
\draw[color = #3, rounded corners] (current bounding box.south west) rectangle (current bounding
box.north east);
\end{tikzpicture}
\end{center}%
}
%%%%

%%%%
%Definicion del comando "\marco{Texto a enmarcar}{Ancho del texto}":
%
\newcommand{\marco}[2]{%
\setlength{\fboxsep}{8pt}\framebox[#2][c]{%
\begin{minipage}[c]{#2-.5cm}
#1
\end{minipage}}
}
%%%%%

%%%%%
%\Comentario
\newcommand{\Comentario}[1]{%
\textcolor[cmyk]{1,0.25,1,0.25}{\marco{\textsc{#1}}{0.75\textwidth}}}
%%%%%

%%%%%
%
\definecolor{sFase}{cmyk}{1,1,0.25,0}
\definecolor{miCeleste}{cmyk}{0.2,0,0,0}
\definecolor{verdeA}{cmyk}{0.05,0,0.2,0}
%
%%%%%

%%%%%
%\SubFase: #1: color, #2: texto
\newcommand{\SubFase}[2]{\subsection*{\textcolor{#1}{#2}}}
%
%%%%%

%
\definecolor{verdeC}{cmyk}{1,0.25,1,0.25}
%

%
\definecolor{azulM}{cmyk}{1,1,0,0.25}
%

%
\definecolor{amarilloB}{cmyk}{0,0,0.1,0}
%

%
\definecolor{rojo}{cmyk}{0,1,1,0}
%

%
\definecolor{ligero}{cmyk}{0,0,0,0.075}
%

%
\definecolor{fondoEnunciadoEjemplo}{cmyk}{0,0,0,0.10}
%

%
\definecolor{EstLibF}{cmyk}{0,0,0,0.05}
%

%
\definecolor{fondoGlosario}{cmyk}{0,.18,.6,0}
%

%
\definecolor{textoRelevante}{cmyk}{1,.66,0,.02}
%

%
\definecolor{textoProvisional}{cmyk}{1,0.25,1,0.25}
%

%
\definecolor{textoLista}{cmyk}{0,.5,1,.3}
%

%
\definecolor{fondoLista}{cmyk}{0,.12,.3,0}
%

%
\definecolor{textoEnunciado}{cmyk}{0,.5,1,.3}
%

%
\definecolor{fondoEnunciado}{cmyk}{0,.12,.3,0}
%

%
\definecolor{textoFormula}{cmyk}{1,.9,0,0}
%

%
\definecolor{fondoFormula}{cmyk}{.1,0,.04,0}
%

%
\definecolor{textoResumen}{cmyk}{1,.9,0,0}
%

%
\definecolor{fondoResumen}{cmyk}{.5,0,.08,0}
%\definecolor{fondoResumen}{cmyk}{0,0,0,0}
%

%
\definecolor{textoTabla}{cmyk}{1,.9,0,0}
%

%
\definecolor{fondoTabla}{cmyk}{0,.45,.7,0}
%

%
\definecolor{fondoModelo}{cmyk}{0,.12,.3,0}
%

%
\definecolor{textoModelo}{cmyk}{0,.5,1,.3}
%

%
\definecolor{textoProcedimiento}{cmyk}{.6,0,.65,.3}
%

%
\definecolor{fondoProcedimiento}{cmyk}{.12,0,.5,0}
%

%
\definecolor{textoSolucion}{cmyk}{.45,0,.65,.3}
%

%
\definecolor{fondoSolucion}{cmyk}{.03,0,.55,0}
%

%
\definecolor{textoRespuesta}{cmyk}{.45,0,.65,.3}
%

%
\definecolor{fondoRespuesta}{cmyk}{.03,0,.55,0}
%

%
\definecolor{textoConcepto}{cmyk}{0,.35,1,.3}
%

%
\definecolor{fondoConcepto}{cmyk}{0,.07,.45,0}
%

%
\definecolor{textoPrerrequisito}{cmyk}{0,0,0,1}
%

%
\definecolor{fondoPrerrequisito}{cmyk}{.1,.1,0,0}
%

%
\definecolor{sombraPrerrequisito}{cmyk}{0.6,.6,0,0}
%

%
\definecolor{primerIntervalo}{cmyk}{0,.45,.7,0}
%

%
\definecolor{segundoIntervalo}{cmyk}{0,1,0,.6}
%

%
\definecolor{tercerIntervalo}{cmyk}{.14,.05,0,0}
%

%
\definecolor{dimensiones}{cmyk}{0,.45,.7,.05}%en lugar del rojo
%

%
\definecolor{marcas}{cmyk}{0,1,0,.6}%en lugar del azul.
%

%
\definecolor{textos}{cmyk}{0,1,0,.6}%en lugar del gris
%

%
\definecolor{pregunta}{cmyk}{0,.45,.7,.05}
%

%
\definecolor{extra}{cmyk}{0,.7,1,0}
%

%
\definecolor{conceptual}{cmyk}{0,1,.62,0}
%

%
\definecolor{conceptualBajo}{cmyk}{0,.1,..05,0}
%

%
\definecolor{procedimental}{cmyk}{.61,.89,0,0}
%

%
\definecolor{procedimentalBajo}{cmyk}{.1,.05,0,0}
%

%
\definecolor{actitudinal}{cmyk}{.34,0,1,.1}
%

%
\definecolor{actitudinalBajo}{cmyk}{.04,0,.15,0}
%

%
\definecolor{fondoMetaTexto}{cmyk}{.03,.03,0,0}
%

%
\definecolor{sombraMetaTexto}{cmyk}{.1,.1,0,.1}
%

%
\definecolor{textoMetaTexto}{cmyk}{1,1,0,0}
%

%
\definecolor{sol}{cmyk}{0,.3,1,0}
%

%
\definecolor{tierra}{cmyk}{0,.54,1,.35}
%

\def\formula#1{%
\colorbox{fondoFormula}{\textcolor{textoFormula}{#1}}}
%

%%
\newcommand{\historia}[2]{%
\begin{center}
\begin{minipage}{\textwidth - #1}
\parindent 6.35mm
#2
\end{minipage}
\end{center}}
%%

%%
%%%%% #1: Color de fondo, #2: Color del texto, #3: ancho de la caja, #4: el texto.
\newcommand{\ccjc}[4]{%
\begin{center}
\colorbox{#1}{
{\color{#2}
\begin{minipage}[c]{#3}
#4
\end{minipage}
}}
\end{center}
}
%%

%%
%%%%% #1: Color de fondo, #2: Color del texto, #3: ancho de la caja, #4: el texto.
\newcommand{\cjc}[4]{%
\colorbox{#1}{
{\color{#2}
\begin{minipage}[c]{#3}
#4
\end{minipage}
}}
}
%%

%%%%% #1: Color del marco, #2: Color del texto, #3: ancho de la caja, #4: el texto.
\newcommand{\fccjc}[4]{%
\begin{center}
\fcolorbox{#1}{white}{
{\color{#2}
\begin{minipage}[c]{#3}
#4
\end{minipage}
}}
\end{center}
}

%
%
%%%%% #1:ancho, #2: el texto.
\newcommand{\respuesta}[2]{%
%\begin{center}
%\colorbox{fondoRespuesta}{
%{\color{textoRespuesta}
%\begin{minipage}[c]{#1}
%\textsl{#2}
%\end{minipage}
%}}
%\end{center}
\ccjc{fondoRespuesta}{textoRespuesta}{#1}{#2}
}
%%%%

%%%%% #1:ancho, #2: el texto.
\newcommand{\solucion}[2]{%
\begin{center}
\colorbox{fondoSolucion}{
{\color{textoSolucion}
\begin{minipage}[c]{#1}
#2
\end{minipage}
}}
\end{center}
}
%%%%

%%%% #1: ancho, #2: el texto.
\newcommand{\resumen}[2]{%
\fccjc{textoResumen}{textoResumen}{#1}{#2}}
%%%%

%%%% #1: ancho, #2: titulo, #3: el texto.
\newcommand{\lista}[3]{%
\ccjc{fondoLista}{textoLista}{#1}{%
\begin{center}
  \textbf{#2}
  \addcontentsline{toc}{subsection}{\numberline{}\dsenbjc #2}
\end{center}
#3
}
}
%%%%

%%%% #1: ancho, #2: el titulo, #3: el texto.
\newcommand{\enunciado}[3]{%
\ccjc{fondoEnunciado}{textoEnunciado}{#1}{%
\begin{center}
  \textbf{#2}
  \addcontentsline{toc}{subsection}{\numberline{}\dsenbjc #2}
\end{center}
{\parindent 1em
\textit{#3}
}
}
}
%%%%

%%%% #1: ancho, #2: titulo, #3: el texto.
\newcommand{\modelo}[3]{%
\ccjc{fondoModelo}{textoModelo}{#1}{%
  \begin{center}
    \textbf{#2}
    \addcontentsline{toc}{subsection}{\numberline{}\dsenbjc El modelo}
  \end{center}
#3
}
}
%%%%

%%%% #1: ancho exterior, #2: ancho interior, #3: el texto.
\newcommand{\glosario}[3]{%
\begin{wrapfigure}{r}{#1}
\colorbox{fondoGlosario}{%
\hspace{2ex}\begin{minipage}[c]{#2}
\parindent 0pt
\parskip 1ex
\footnotesize{%
\vspace{1.5ex}
          #3
\vspace{1.5ex}
}
\end{minipage}
\hspace{1ex}}
\end{wrapfigure}
}
%%%%

%%%% #1: ancho, #2: el texto.
\newcommand{\concepto}[2]{%
\ccjc{fondoConcepto}{textoConcepto}{#1}{%
#2}
}
%%%%

%%%% #1: ancho, #2: el texto.
\newcommand{\procedimiento}[2]{%
\ccjc{fondoProcedimiento}{textoProcedimiento}{#1}{%
#2}
}
%%%%

%%%% #1: ancho, #2: el texto.
\newcommand{\prerrequisito}[2]{%
\begin{center}
\psshadowbox[fillstyle=solid,fillcolor=fondoPrerrequisito,shadowcolor=sombraPrerrequisito,linecolor=fondoPrerrequisito]
{%\color{red}
\begin{minipage}[c]{#1}
\color{textoPrerrequisito}
#2
\end{minipage}
}
\end{center}
}
%%%%

%%%% #1: ancho, #6: el texto, #3: color de fondo, #4: color de la sombra, #5: color del borde, #2: color texto.
\newcommand{\metaTexto}[6]{%
\begin{center}
\psshadowbox[fillstyle=solid,fillcolor=#3,shadowcolor=#4,linecolor=#5]
{\begin{minipage}[c]{#1}
\color{#2}
#6
\end{minipage}
}
\end{center}
}
%%%%


%%%% #1: ancho, #2: el texto.
\newcommand{\formulas}[2]{%
\ccjc{fondoFormula}{textoFormula}{#1}{#2}
}
%%%%

%%%%Definicion de fuente para justificacion
\newcommand{\jdj}[1]{%
\footnotesize{\textit{#1}}}
%

%%%Definici\'on de caja para resumir resultados y justificativos
%%% #1: longitud que indica espacio extra vertical del tabulador en pt,
%%% #2: espacio en em entre las columnas,
%%% #3: fracci\'on de \textwidth para las justificaciones,
%%% #4: el texto
\newcommand{\cjjc}[4]{
\begin{center}
\setlength{\extrarowheight}{#1pt}
\footnotesize{
\begin{tabular}{cl@{\hspace{#2em}} p{#3\textwidth}}
#4
\end{tabular}
}
\end{center}
}
%%%
%%%

%%% \eccjc{Color de fondo}{Color del texto}{Formula (sin $$)}
\newcommand{\eccjc}[3]{%
\colorbox{#1}{\textcolor{#2}{$\displaystyle{#3}$}}
}
%

%%% \feccjc{Color del borde}{Color del texto}{Formula (sin $$)}
\newcommand{\feccjc}[3]{%
\fcolorbox{#1}{white}{\textcolor{#2}{$\displaystyle{#3}$}}
}
%

%%%% \presentacion{#1}: el texto.
\newcommand{\presentacion}[1]{
\begin{center}
\parbox{12.85cm}{
{\parindent 1.5em
\color{white}
#1
}
}
\end{center}
}
%%%%

%%% \story{#1}: el texto.
\newcommand{\story}[1]{%
\ccjc{ligero}{black}{0.95\textwidth}{%
\parindent 1.25em
%\begin{helvetica}
%\footnotesize{
\small{#1}
%}
%\end{helvetica}
}
}

%%Regla de tres
%%#1: Nombre de la magnitud Izquierda
%%#2: Unidad de medida de la magnitud Izquierda
%%#3: Primer valor de la magnitud Izquierda
%%#4: Nombre de la magnitud Derecha
%%#5: Unidad de medida de la magnitud Derecha
%%#6: Primer valor de la magnitud Derecha
%%#7: Segundo valor de la magnitud Izquierda
\newcommand{\regladetres}[7]{%
\[
\begin{tabular}{ccc}
\text{#1} & & \text{#4} \\
#3\text{ #2} & $\longrightarrow$ & #6\text{ #5} \\
#7\text{ #2} & $\longrightarrow$ & \text{? #5}
\end{tabular}
\]
}
%

%%Longitudes
\newlength{\ancho}

%%Fuentes nuevas
\newenvironment{bookman}
{\fontencoding{OT1}\fontfamily{ppl}\selectfont}{}

\newenvironment{helvetica}
{\fontencoding{OT1}\fontfamily{phv}\selectfont}{}


\newcommand{\eil}{\vspace{\baselineskip}}
%%%%
%%\eijc: #1 indica la proporci?n de salto de l?nea que se a?ade con \vspace. Puede ser negativo.
\newcommand{\eijc}[1]{\vspace{#1\baselineskip}}
%%

%494
%x- 3/7x = 1/3(494-x)

\newcommand{\linea}{\rule{\textwidth-55.22438pt}{0.075ex}}
\newcommand{\linearespuesta}[1]{\rule{\textwidth-#1pt}{0.075ex}}

\newcommand{\yj}{\quad\text{y}\quad}
\newcommand{\cj}{\quad\text{,}\quad}

\newcommand{\D}{\operatorname{D}}
\newcommand{\Img}{\operatorname{Im}}

%\newcommand{\wideparen}[1]{%
%
%}

%\newcommand{\wideparen}[1]{\accentset{\frown}{#1}}
%\DeclareMathAccent{\wparen}{\mathord}{largesymbols}{"F3}
%\newcommand{\wideparen}[1]{\accentset{\wparen}{$#1$}}

\DeclareMathOperator{\inte}{int}

\DeclareMathOperator{\mcm}{mcm}
%\DeclareMathOperator{\sen}{sen}
\DeclareMathOperator{\sinus}{sinus}%
\DeclareMathOperator{\sech}{sech}%
\DeclareMathOperator{\csch}{csch}%


\DeclareMathOperator{\ar}{\acute{a}rea}%
\DeclareMathOperator{\lon}{longitud}%
\DeclareMathOperator{\vol}{volumen}%
\DeclareMathOperator{\metros}{m}%
\DeclareMathOperator{\decimetros}{dm}%
\DeclareMathOperator{\MWh}{MW-h}%
\DeclareMathOperator{\litros}{l}%
\DeclareMathOperator{\kilogramos}{Kg}%
\DeclareMathOperator{\minutos}{min}%
\DeclareMathOperator{\segundos}{s}%
\DeclareMathOperator{\hora}{h}%
\DeclareMathOperator{\centimetros}{cm}%
\DeclareMathOperator{\kilometros}{km}%
\DeclareMathOperator{\miligramos}{mg}%

\DeclareMathOperator{\dm}{dm}%
\DeclareMathOperator{\Dm}{Dm}%
\DeclareMathOperator{\UL}{u_L}%
\DeclareMathOperator{\UA}{u_A}%
\DeclareMathOperator{\UT}{u_T}%
\DeclareMathOperator{\UV}{u_V}%
\DeclareMathOperator{\UF}{u_F}%
\DeclareMathOperator{\UR}{u_R}%
\DeclareMathOperator{\UM}{u_M}%
\DeclareMathOperator{\UD}{u_D}%
\DeclareMathOperator{\UPE}{u_{PE}}%
\DeclareMathOperator{\UP}{u_P}%
\DeclareMathOperator{\Um}{u_m}%
\DeclareMathOperator{\UTR}{u_{TR}}%
\DeclareMathOperator{\Ut}{u_t}%
\DeclareMathOperator{\UI}{u_I}%
\DeclareMathOperator{\UTI}{u_{TI}}%
\DeclareMathOperator{\U}{u}%

\DeclareMathOperator{\uL}{u_{L}}%
\DeclareMathOperator{\uA}{u_{L}^2}%
\DeclareMathOperator{\uV}{u_{L}^3}%
\DeclareMathOperator{\cm}{cm}%

\DeclareMathOperator{\arcsec}{arcsec}%
\DeclareMathOperator{\arccot}{arccot}%
\DeclareMathOperator{\arccsc}{arccsc}%



%\makeatletter
%\def\cleardoublepage{\clearpage\ifodd\c@page\else{
% \hbox{}
% \vspace*{\fill}
% \begin{center}
% Esta p\'agina contiene intencionalmente esta \'unica sentencia.
% \end{center}
% \vspace{\fill}
% \thispagestyle{empty}
% \newpage}\fi}
%\makeatother
%
%
%%%%%Macro para dibujar la recta numerica y un intervalo semiinfinito
%%%#1 posicion del extremo del intervalo
%%%#2 el extremo del intervalo
%%%#3 el color del intervalo
%%%#4 longitud en cm del intervalo
%%%#5 posicion del texto
%%%#6 el texto
%%%#7 el valor del extremo
%%%#8 p para positivo, n para negativo

\newcommand{\semiintervalo}[8]{%
\begin{center}
\begin{texdraw}
\linewd 0.005
\drawdim cm
\arrowheadtype t:V
\move(5 1)
\avec(10 1)
\move(5 1)
\avec(0 1)
%%Etiqueta de R en el extremo derecho
\htext(9.5 1.5){$\mathbb{R}$}
%%linea vertical para indicar la posicion del numero 10 y el numero 10
\move(#1 1.125)\lvec(#1 0.875)
\textref h:C v:C \htext(#1 0.5){#7}
%%Segmento de color que indica los mayores que #7
\ifthenelse{\equal{#8}{p}}{\textref h:L v:C }{\textref h:R v:C }
\htext(#2 1){\textcolor{#3}{\rule{#4cm}{2mm}}}
%%Nombre del conjunto
\textref h:C v:C \htext(#5 1.5){\textcolor{#3}{$#6$}}
\end{texdraw}
\end{center}
}

%%%%#1 el n\'umero
%%%%#2 texto para el lado menor
%%%%#3 color para el texto del lado menor
%%%%#4 texto para el lado mayor
%%%%#5 color para el texto del lado mayor
\newcommand{\rectanumerica}[5]{%
\begin{center}
\begin{texdraw}
\linewd 0.005
\drawdim cm
\arrowheadtype t:V
\move(5 1)
\avec(10 1)
\move(5 1)
\avec(0 1)
%%Etiqueta de R en el extremo derecho
\htext(9.5 1.5){$\mathbb{R}$}
%%linea vertical para indicar la posicion del numero #1 y el numero #1
\move(5.5 1.125)\lvec(5.5 0.875)
\textref h:C v:C \htext(5.5 0.5){#1}
%%Texto que indica el semiintervalo negativo
\textref h:C v:C \htext(2.75 1.5){\textcolor{#3}{#2}}
%%Texto que indica el semiintervalo positivo
\textref h:C v:C \htext(7.25 1.5){\textcolor{#5}{#4}}
\end{texdraw}
\end{center}
}

%%%%#1 Primer n\'umero
%%%%#2 Posici\'on del primer n\'umero
%%%%#3 Segundo n\'umero
%%%%#4 Posici\'on del segundo n\'umero
\newcommand{\semiintervalos}[4]{%
\begin{center}
\begin{texdraw}
\linewd 0.005
\drawdim cm
\arrowheadtype t:V
\move(5 1)
\avec(10 1)
\move(5 1)
\avec(0 1)
%%Etiqueta de R en el extremo derecho
\htext(9.5 1.5){$\mathbb{R}$}
%%linea vertical para indicar la posicion #2 del numero #1 y el numero #1
\move(#2 1.125)\lvec(#2 0.875)
\textref h:C v:C \htext(#2 0.5){#1}
%%linea vertical para indicar la posicion #4 del numero #3 y el numero #3
\move(#4 1.125)\lvec(#4 0.875)
\textref h:C v:C \htext(#4 0.5){#3}
\end{texdraw}
\end{center}
}

\newpsobject{showgrid}{psgrid}{subgriddiv=1,griddots=10,gridlabels=6pt}
\newcommand{\sjc}[1]{
\overline{#1}
}

\def\dtjc#1#2#3{
%%La medida del 'angulo de 90 grados
\psline[linecolor=textoModelo](.5,.8)(.8,.8)(.8,.5)
%%%El tri'angulo
\psline[linecolor=textoModelo](.5,.5)(6.5,.5)(.5,3.95)(.5,.5)
\rput(.3,.2){$#3$}
\rput(.3,3.95){$#2$}
\rput(6.7,.2){$#1$}
}

\def\triangulorectxyz{
\begin{center}
\begin{pspicture}(0,0)(7,4)
%%La medida del 'angulo de 90 grados
\psline(.5,.8)(.8,.8)(.8,.5)
%%La medida del 'angulo de 30 grados
\psarc{-}(6.5,.5){1}{151}{180}
\rput(5.2,.75){$30^\circ$}
%%La medida del 'angulo recto en $U$
\psline(3,.8)(3.3,.8)(3.3,.5)
%%%El tri'angulo
\psline(.5,.5)(6.5,.5)(.5,3.95)(.5,.5)
%%%La perpendicular desde U
\psline(3,.5)(3,2.5)
%%%Los puntos
\rput(.3,.2){$Z$}
\rput(.3,3.95){$Y$}
\rput(6.7,.2){$X$}
\rput(3,.2){$U$}
\rput(3,2.8){$V$}
\pscircle[fillstyle=solid,fillcolor=black](3,.5){.05}
\pscircle[fillstyle=solid,fillcolor=black](3,2.5){.05}
\end{pspicture}
\end{center}}
%\newenviroment{Dibujo}{\begin{center}\begin{pspicture}}{\end{pspicture}\end{center}}

%Definici\'on de \sobrej
\def\sobrej#1#2{%
\overset{\ #1}{#2}}

%%Definici\'on de \linejc
\def\linejc#1{%
\overset{\ \longleftrightarrow}{#1}}
\def\arcojc#1{%
\overset{\ \frown}{#1}}

%%%
%Definicion de comando \remarcar{texto}
\def\remarcar#1{%
\textcolor{textoRelevante}{#1}%
}

%%%
%secciones: bf y color azul
\newcommand{\secciones}[2]{%
\textbf{\textcolor{#1}{#2}}}

%%%Cortes de palabras en s\'ilabas
\hyphenation{pro-ble-ma pro-ble-mas abs-trac-tos mo-de-los es-tu-dian-te ope-ra-cio-nes%
Re-fle-xio-nes pro-ce-di-men-tal es-tu-dian-tes si-mi-lar re-le-van-te u-ni-dad u-ni-da-des%
me-nor me-no-res cual-quier e-la-bo-rar na-tu-ral si-guien-te si-guien-tes%
tan-gen-te tan-gen-tes co-tan-gen-te co-tan-gen-tes de-fi-ni-cio-nes mo-de-lo si-guien-tes%
re-pre-sen-ta re-pre-sen-tan sa-tis-fa-ce}

\newcommand{\oenbjc}{\!\!\!\!\!\!\!\!}
\newcommand{\dsenbjc}{\!\!\!\!\!\!\!\!\!\!\!\!\!\!\!\!\!\!\!\!}

%%%Comandos en base a tikz
%%
%% Lo mismo que \ccjc, pero utilizando el tikz
%%%%% #1: Color de fondo, #2: Color del texto, #3: ancho de la caja, #4: el texto.
\newcommand{\tccjc}[4]{%
\begin{center}
\begin{tikzpicture}[rounded corners]
\draw (0,0) node[text justified, text width=#3, fill = #1, inner sep = 1.5ex]
{\color{#2}
#4};
\end{tikzpicture}
\end{center}
}
%%

%%% \teccjc{Radio de esquinas}{Color de fondo}{Color del texto}{Formula (sin $$)}
\newcommand{\teccjc}[4]{%
%\begin{tikzpicture}[rounded corners]
%\draw (0,0) node[fill = #2, inner sep = 1ex]
%{\textcolor{#3}{$\displaystyle{#4}$}};
%\end{tikzpicture}
\psframebox[framesep=6pt,linestyle=none,linearc=#1,cornersize=absolute,fillstyle=solid,fillcolor=#2]%
{\textcolor{#3}{#4}}
}
%

%%%%% #1:ancho, #2: el texto.
\newcommand{\trespuesta}[2]{%
\tccjc{fondoRespuesta}{textoRespuesta}{#1}{{\textsl{#2}}}
}
%%

%%%%% #1:ancho, #2: el texto.
\newcommand{\tsolucion}[2]{%
\tccjc{fondoSolucion}{textoSolucion}{#1}{#2}
}
%%

%%%% \fjc{Radio de las esquinas}{Texto de formula sin $$}
\newcommand{\fjc}[2]{%
\teccjc{#1}{fondoFormula}{textoFormula}{#2}
}

\newenvironment{enumeratejc}%
                 {\renewcommand\pltopsep{.5\baselineskip}%
                 \renewcommand\plitemsep{.3\baselineskip}%
                 \begin{compactenum}[1.]}{\end{compactenum}}

\newenvironment{listajc}[1]%
               {\renewcommand\pltopsep{.5\baselineskip}%
               \renewcommand\plitemsep{.3\baselineskip}%
               \begin{compactenum}[#1]}{\end{compactenum}}

\newcommand{\tq}{\text{\ tal que \ }}

%%%%
%% \funcjc{nombre}{salida}{llegada}
\newcommand{\funcjc}[3]{%
#1\colon #2\longrightarrow #3 }
%

%%%% Ley de asignaci?n
%% \lajc{variable independiente}{variable dependendiente}
\newcommand{\lajc}[2]{%
#1 \longmapsto #2 }

%%%% Funci?n conjunta
%% \funcionjc{Nombre}{salida}{llegada}{independiente}{dependiente}
\newcommand{\funcionjc}[5]{%
{\setlength{\arraycolsep}{2pt}
\begin{array}{lccl}
#1\colon & #2 & \longrightarrow & #3\\
& #4 & \longmapsto & #5
\end{array}}
}

\newcommand{\yjc}{\quad\text{y}\quad}

\newcommand{\relphantom}[1]{\mathrel{\phantom{#1}}}

\newcommand{\Rbb}{\mathbb{R}}
\newcommand{\Zbb}{\mathbb{Z}}
\newcommand{\Nbb}{\mathbb{N}}
\newcommand{\Qbb}{\mathbb{Q}}

\newcommand{\saltoproof}{%
$\phantom{\rightarrow}$
}%
