% Esquema para los like-theorem

\newtheoremstyle{cal}% name
                {}% space-above
                {}% space-below
                {\sffamily}% body-style
                {}% indent
                {\sffamily\bfseries}% head-style
                {}% head-after-punct
                {\newline}%  head-after-space
                {}% head-full-spec
% Para teorema
\theoremstyle{cal}
\newtheorem{teo}{Teorema}[chapter]
\newtheorem{coro}[teo]{Corolario}
\newtheorem{defi}{Definición}[chapter]
\newtheorem{axioma}{Axioma}[chapter]
\newtheorem{lema}{Lema}[chapter]
\newtheorem{prob}{Problema}[chapter]
\newtheorem{algo}{Algoritmo}[chapter]

\newcommand{\lteocal}[3][]{%
\begin{center}
\begin{minipage}{.925\textwidth}
\rule[-.5\baselineskip]{\textwidth}{.1ex}
\begin{#2}[#1]
#3
\end{#2}
\rule[.5\baselineskip]{\textwidth}{.1ex}
\end{minipage}
\end{center}}

\newenvironment{teocal}[1][]%
{\penalty-60\parindent 0pt\rule[-.5\baselineskip]{\textwidth}{.1ex}
\begin{teo}[#1]}%
{\end{teo}\vspace*{-.25\baselineskip}\rule[0.5\baselineskip]{\textwidth}{.1ex}}

\newenvironment{defical}[1][]%
{\penalty-60\parindent 0pt\rule[-.5\baselineskip]{\textwidth}{.1ex}
\begin{defi}[#1]}%
{\end{defi}\vspace*{-.25\baselineskip}\rule[0.5\baselineskip]{\textwidth}{.1ex}}

\newenvironment{corocal}[1][]%
{\penalty-60\parindent 0pt\rule[-.5\baselineskip]{\textwidth}{.1ex}
\begin{coro}[#1]}%
{\end{coro}\vspace*{-.25\baselineskip}\rule[0.5\baselineskip]{\textwidth}{.1ex}}

\newenvironment{lemacal}[1][]%
{\penalty-60\parindent 0pt\rule[-.5\baselineskip]{\textwidth}{.1ex}
\begin{lema}[#1]}%
{\end{lema}\vspace*{-.25\baselineskip}\rule[0.5\baselineskip]{\textwidth}{.1ex}}

\newenvironment{probcal}[1][]%
{%
\begin{prob}[#1]}%
{\end{prob}}

\newenvironment{algocal}[1][]%
{%
\begin{algo}[#1]}%
{\end{algo}}

%%%
%%% #1: nombre de la funci?n
%%% #2: dominio
%%% #3: conjunto de llegada
\newcommand{\funjc}[3]{%
#1\colon #2 \longrightarrow #3}

%%%
\newlength{\lfbf}
\newcommand{\slfbf}[1]{%
\setlength{\lfbf}{#1\textwidth}}
\newlength{\lsimp}
\newcommand{\slsimp}[1]{%
\setlength{\lsimp}{#1\textwidth}}
\newcommand{\cen}{@{\hspace{.5em}}}
\newcommand{\sgr}{\hspace{6mm}}


%%%ep?grafe
\newcommand{\epigrafejc}[6]{%
\begin{tikzpicture}
\draw (0,0) node[text justified, text width=#1\textwidth, fill = #5, inner sep = #2ex]%
{\color{#4} #6};%
\draw[color = #3, rounded corners] (current bounding box.south west) rectangle (current bounding
box.north east);
\end{tikzpicture}
}

%%%
%% babel spanish de jc
\renewcommand{\figurename}{Figura}

%%%

%%%
%% \limjc{f(x)}{x}{a}
\newcommand{\limjc}[3]{%
\lim_{#2\rightarrow #3}#1}
%%

%\newcommand{\chapformat}[1]{%
%\parbox[b]{.5\textwidth}{\filleft\bfseries #1}%
%\quad\rule[-12pt]{2pt}{85pt}\quad
%{\fontsize{60}{60}\selectfont\thechapter}}
%\titleformat{\chapter}% command
%            [block]% shape
%            {\filleft\fontsize{29.86}{29.86}\selectfont\sffamily}% format
%            {}% label
%            {0pt}% sep
%            {\chapformat}% before code
%
%\titleformat{\section}% command
%            [hang]% shape
%            {\filright\bfseries}% format
%            {\Large\sffamily\thesection}% label
%            {1ex}% sep
%            {\Large\sffamily}% before code
%            [\vspace{-.85\baselineskip}\rule{\titlewidth}{.8pt}]%aftercode
%
%\titleformat{\subsection}% command
%            [hang]% shape
%            {\filright\bfseries}% format
%            {\large\sffamily\thesubsection}% label
%            {1ex}% sep
%            {\large\sffamily}% before code
%            [\vspace{-.8\baselineskip}\rule{\titlewidth}{.6pt}]%aftercode
%
%%\titlespacing{\subsection}{-2.5ex}{\baselineskip}{.2\baselineskip}
%
%\renewcommand{\chaptermark}[1]{%
%\markboth{\sffamily#1}{}}
%\renewcommand{\sectionmark}[1]{%
%\markright{\sffamily \thesection\ #1}}
%\fancyhf{}
%\fancyhead[LE,RO]{\sffamily\bfseries\thepage}
%\fancyhead[LO]{\bfseries\rightmark}
%\fancyhead[RE]{\bfseries\leftmark}
%\renewcommand{\headrulewidth}{.2pt}
%\renewcommand{\footrulewidth}{0pt}
%\addtolength{\headheight}{.1pt}
%\fancypagestyle{plain}{%
%\fancyhead{}
%\renewcommand{\headrulewidth}{0pt}}
%
%\definecolor{textoEjemplo}{cmyk}{0,0,0,.60}

\renewcommand\decimalcomma{\spanishdecimal{.}}
\decimalcomma

\newcommand{\ejemplo}[2]{%
\tccbjc{fondoEnunciadoEjemplo}{black}{black}{1.5ex}{.925\textwidth}{%
#1 } \emph{Solución}.
\begingroup
\small
#2
\endgroup
}

\newcommand{\razoncambio}[3]{%
\ifthenelse{\equal{#1}{}}{\frac{#2 - #3}{#2 - #3}}{\frac{#1(#2) - #1(#3)}{#2 - #3}} }

%%%% Ambiente exemplo (la x no es x sino chi) 08-05-15 JcTo
%%% Contador
\newcounter{exemplo}%[chapter]
\setcounter{exemplo}{1}

%%% Ambiente exemplo - #1 es el argumento opcional: la etiqueta que se requiere
%%% que aparezca como inicio del desarrollo de ejemplo: *Soluci?n*, *Discusi?n*,
%%% etc?tera. El valor por omisi?n (o predeterminado) es ""; es decir, nada.
%%% El segundo argumento es el enunciado del ejemplo.
%%% Requiere el paquete *ifthen*: \usepackage{ifthen}
\newenvironment{exemplo}[2][]%
  {%
  \subsubsection*{Ejemplo \thechapter.\theexemplo}
  \ifthenelse{\equal{#1}{}}%
             {\small}
             {\tccbjc{fondoEnunciadoEjemplo}{black}{black}{1.5ex}{.9\textwidth}{%
                      \small #2}
                \small\textit{#1}. }%
              }% begin-part
  {\addtocounter{exemplo}{1}%\hfill$\blacklozenge$
                            \normalsize\selectfont

  \vspace{0.5\baselineskip}}% end-part

\newcommand{\limjca}[1]{\lim_{x\to a}#1}
