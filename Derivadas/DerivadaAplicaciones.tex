\chapter{La derivada: aplicaciones}

\section{Romeo y Julieta: la modelización matemática}
\begingroup
\itshape%
Romeo estaba feliz. Acababa de recibir noticias de su amada Julieta, quien logró enviarle
clandestinamente un corto recado, escrito apuradamente en un trozo de papel:

---Te esperaré, luego de la cena, en el muelle de mi residencia. Estaré oculta tras el farol
que queda junto a los sauces.

El dichoso Romeo, quien descansaba ese momento en sus alojamientos, decidió ir lo antes posible en
su bote a la cita. Recordó que, a un costado del lago, crecía un hermoso jardín lleno de bellas
rosas en esta época del año, que él sabía que eran de sumo agrado para Julieta. ¡Cómo no llevarle
un ramo de esas rosas! Estaba, sin embargo, apurado, porque estaba a punto de atardecer. Debía
escoger el camino más corto posible para llegar cuanto antes donde su amada llevándole el más bello
ramo de rosas.
\endgroup

\vspace{\baselineskip}%
Desde el inicio del pensamiento inteligente, éste consistió, fundamentalmente, en la búsqueda de
modelos abstractos, teóricos, de la realidad. Pero, simultáneamente, desde el uso de piedrecillas
(cálculos) y luego del más sofisticado ábaco, para hacer cuentas, seguido del de la ingeniosa regla
de cálculo, y de las maravillosas máquinas electrónicas de que disponemos hoy en día, objetos
físicos y máquinas inventadas y construidas por el hombre se constituyen en modelos físicos de
objetos abstractos, como son los números y las operaciones que se realizan con ellos, las figuras y
objetos geométricos y su representación gráfica.

Las Ciencias Naturales y, cada vez más, las Ciencias Sociales utilizan modelos matemáticos como
instrumento fundamental para describir y entender lo más relevante de los objetos o fenómenos
estudiados. La tecnología de la producción de bienes y servicios utiliza cada vez más problemas
matemáticos como modelos de problemas reales que se presentan en el quehacer creativo. El
advenimiento de las computadoras, cada día más poderosas, ha favorecido este fenómeno, porque
problemas matemáticos cuya solución ``teórica'' es demasiado compleja, y a veces prácticamente
imposible, es ahora abordable, con excelentes resultados, mediante métodos numéricos que proveen de
soluciones aproximadas de calidad.

Es, pues, de vital importancia fortalecer la habilidad del matemático para tender puentes entre la
realidad y la matemática, esto es la de utilizar adecuadamente los modelos matemáticos. La
modelización matemática consiste, generalmente, en un proceso que puede ser descompuesto en las
siguientes etapas, que las ilustraremos con el ejemplo de Romeo y Julieta.

\begin{enumerate}[leftmargin=*]
\item \emph{Identificación clara del problema, objeto o fenómeno que se quiere modelizar}. Como
    resultado de esta fase, se obtiene el ``enunciado'' del problema o la descripción precisa
    del objeto o fenómeno estudiado.

\item \emph{Elaboración del modelo}. Consiste en la representación con símbolos y entes
    matemáticos (números constantes o variables, figuras geométricas, funciones, ecuaciones
    algebraicas o diferenciales, matrices y otros), los aspectos más relevantes del problema,
    objeto o fenómeno estudiado. Un aspecto fundamental en esta etapa es la del uso de un
    sistema coherente de unidades de medida, que permita escoger un modelo matemático que
    abstraiga este tipo de información, es decir que no contenga ya unidades de medida sino
    solo números, funciones numéricas, matrices numéricas, etcétera. Como resultado de esta
    fase, se obtiene el modelo matemático a utilizarse.

\item \emph{Solución del problema matemático}. Estudio del modelo matemático y solución del
    problema matemático, de ser el caso. Al final de esta etapa, se obtiene una solución
    matemática del problema.

\item \emph{Interpretación del modelo matemático}. Se interpretan los resultados del problema
    matemático para dar respuestas al problema original o se da la adecuada interpretación al
    estudio del modelo matemático escogido.

\end{enumerate}

En este capítulo, vamos a conocer los métodos matemáticos que permiten resolver problemas como el
de Romeo, entre otros. A manera de ejemplo, ilustremos las dos primeras etapas del proceso de
modelización con el problema de Romeo y Julieta. Luego de desarrollar las técnicas apropiadas para
resolver estos problemas mediante el cálculo diferencial, completaremos las otras dos etapas.

\subsection{Identificación del problema}
Sabemos que las residencias de Romeo y Julieta, representadas con los puntos $R$ y $J$
respectivamente, están separadas por una laguna rectangular de $3$ kilómetros de ancho como se
ilustra en el siguiente dibujo:
\begin{center}
\begin{pspicture}(6,4)
   \psframe(0.5,0.5)(6,3)%
   \uput[180](0.5,3){$A$}%
   \uput[180](0.5,0.5){$B$}%

   \pstGeonode[PosAngle={90,180,-90},PointSymbol=none,PointNameSep={1em,0.5em,1em}]%
      (2.5,3){R}(0.5,2){P}(1.5,0.5){J}%

   \psline{|<->|}(0.5,3.1)(2.5,3.1)%
   \uput[90](1.5,3.25){$2$ km}%

   \psline{|<->|}(0.5,0.4)(1.5,0.4)%
   \uput[-90](1,0.4){$1$ km}%

   \psline{|<->|}(6.1,0.5)(6.1,3)%
   \uput[0](6.1,1.5){$3$ km}%

   \psline(R)(P)(J)%

   \rput(4,2.5){Laguna}%
   \rput[t](0,1){\pstVerb{
/vshowdict 4 dict def /vshow { vshowdict begin /thestring exch def /lineskip exch def thestring {
/charcode exch def /thechar ( ) dup 0 charcode put def 0 lineskip neg rmoveto gsave thechar
stringwidth pop 2 div neg 0 rmoveto thechar show grestore } forall end } def
   64 (lasoR) vshow }}

\end{pspicture}
\end{center}

El jardín se halla entre el extremo $A$ del lago, situado a $2$ kilómetros de la residencia de
Romeo, y el extremo $B$ del lago, situado a $1$ kilómetro de la de Julieta. Lo que debemos hallar
es el punto $P$ situado en algún lugar de la orilla $AB$ del lago, para que el recorrido total, $RP
+ PJ$, sea lo más corto posible.

\subsection{Elaboración del modelo matemático}
Este es un problema de extremos, cuya solución se obtiene a través del concepto de derivada. Más
adelante estudiaremos el siguiente modelo matemático que es adecuado para el problema de Romeo y
Julieta.

\begin{quote}
Dado un intervalo $I$ y una función $f\colon I \longrightarrow \mathbb{R}$ continua en $I$, se
busca $x_m\in I$ tal que
\[
f(x_m) = \min_{x\in I} f(x).
\]
\end{quote}

Determinemos la que sería la variable $x$, la función objetivo $f$ y el intervalo $I$ en el caso
del problema de Romeo y Julieta.

Para ello, recordemos que lo que debemos minimizar es la longitud del recorrido de Romeo en su
barca. Nombremos con $d$ el número de kilómetros de dicha distancia. Entonces, tenemos que
\[
d = RP + PJ,
\]
donde la longitud del recorrido desde la casa de Romeo, situada en $R$ hasta el punto $P$ ---allí
el joven recogerá las rosas para su amada---, es igual a $RP$ kilómetros, y la distancia recorrida
desde $P$ hasta el punto $J$ ---lugar en el que Julieta se encuentra--- es igual a $PJ$ kilómetros.

Sea $x = AP$, donde $AP$ kilómetros es la distancia recorrida en línea recta desde $A$ hasta $P$.

Como datos, tenemos que $AR = 2$, $AB = 3$, $BJ = 1$ y que los ángulos $\angle RAP$ y $\angle ABJ$
son rectos. En el triángulo $\triangle RAP$, $RP$ es la hipotenusa, por lo que
\[
RP = \sqrt{x^2 + 4}.
\]
En el triángulo rectángulo $\triangle PBJ$, se tiene que $PB = 3 - x$. Entonces, por el teorema de
Pitágoras, otra vez, se tiene que:
\[
PJ = \sqrt{(3 - x)^2 + 1}.
\]
Finalmente, tenemos que:
\[
d = \sqrt{x^2 + 4} + \sqrt{(3 - x)^2 + 1}.
\]

Puesto que se busca que la distancia $d$ recorrida por Romeo sea la mínima, la función cuyo mínimo
hay que hallar es $f$ definida por:
\[
f(x) = \sqrt{x^2 + 4} + \sqrt{(3 - x)^2 + 1}.
\]

De la definición de $x$ ---la distancia entre $A$ y $P$---, se tiene que $x\in (0,3) = I$.
Entonces, $f\colon I \longrightarrow \mathbb{R}$ está definida para todo $x\in I$.

Lo que resta del capítulo, lo vamos a dedicar a aprender cómo encontrar dicho $x$ en $I$ que
optimice la función $f$.

\section{Extremos globales o absolutos}

\begin{defical}[Conjunto acotado]
Un conjunto no vacío $A\subset \Rbb$ es \textbf{acotado por abajo (arriba)} si existe $c_1\in\Rbb$
($c_2\in\Rbb$) tal que $x \geq c_1$ ($x \leq c_2$) para todo $x\in A$. A $c_1$ se le llama
\textbf{cota inferior de $A$} y a $c_2$, \textbf{cota superior de $A$}.
\end{defical}

Sea $A\subset\Rbb$ y no vacío. Si $A$ es acotado por abajo, como consecuencia del axioma de
completitud del conjunto de los números reales $\Rbb$, existe la más grande de las cotas inferiores
de $A$, a la que se le llama \textbf{ínfimo de $A$} y se la representa por $\inf A$.

Análogamente, si $A$ es acotado por arriba, existe la más pequeña de las cotas superiores de $A$,
que se llama \textbf{supremo de $A$} y se la representa por $\sup A$.

Evidentemente, en estos casos, se verifican las siguientes desigualdades:
\[
x \geq \inf A \yjc x \leq \sup A
\]
para todo $x\in A$.

Puede ocurrir que $\inf A \in A$. En este caso, al ínfimo se le llama \textbf{mínimo de $A$} y se
lo representa por $\min A$.

De manera similar, puede suceder que $\sup A\in A$, en cuyo caso, se lo denomina \textbf{máximo de
$A$} y se lo representa por $\max A$.

Si $A$ no es acotado por abajo, se dice que el ínfimo es $-\infty$ y se escribe $\inf A = -\infty$.
Si $A$ no es acotado por arriba, se dice que el supremo es $\infty$ y se escribe $\sup A = \infty$.

Usando la notación de los ``infinitos'', cuando $A$ es acotado por abajo, se suele escribir $\inf A
> -\infty$, y cuando es acotado por arriba, $\sup A < \infty$.

Resumamos estos conceptos en la siguiente definición:

\begin{defical}[Ínfimo, supremo, mínimo y máximo]
Sea $A \subset \Rbb$. Entonces, se define
\[
\begin{array}{l}
\inf A =
\begin{cases}
\text{más grande de las cotas inferiores de $A$} & \text{si $A$ es acotado por abajo,} \\
-\infty & \text{si $A$ no es acotado por abajo};
\end{cases}
\\[16pt]
\sup A =
\begin{cases}
\text{más pequeña de las cotas superiores de $A$} & \text{si $A$ es acotado por arriba,} \\
\infty & \text{si $A$ no es acotado por arriba};
\end{cases}
\\[16pt]
\min A = \inf A \ \text{si $\inf A \in A$ y $\inf A > -\infty$}; \\[4pt]
\max A = \sup A \ \text{si $\sup A \in A$ y $\sup A < \infty$}.
\end{array}
\]
\end{defical}

En el siguiente dibujo se ilustran estas definiciones:
\begin{center}
\begin{pspicture}(0,0)(8,7)
\footnotesize%
\psset{}%

\pstGeonode[PointName={c_1,c_2}]%
  (2,0.5){A}(! \psGetNodeCenter{A} A.x 4 add A.y){A'}
\pstGeonode[PointSymbol={default,o},PointName=none]%
  (2,1.5){B}(! \psGetNodeCenter{B} B.x 4 add B.y){B'}
\uput[0](B){$\min A$}%
\uput[0](B'){$\not\exists\min A$}%
\uput[180](B){$\inf A$}%
\uput[180](B'){$\inf A$}%

\pstGeonode[PointName=none]%
  (2,2){C}
\pstGeonode[PointName=none,PointSymbol={o,o}]%
  (2,2.5){D}(! \psGetNodeCenter{D} D.x 4 add D.y 0.5 add){D'}

\pstGeonode[PointName=none,PointSymbol={default,o}]%
  (2,3.5){E}(! \psGetNodeCenter{E} E.x 4 add E.y){E'}

\pstGeonode[PointName=none]%
  (2,4){F}(! \psGetNodeCenter{F} F.x 4 add F.y 1 add){F'}
\uput[0](F'){$\max A$}%
\uput[180](F'){$\sup A$}%

\pstGeonode[PointName=none,PointSymbol=o]%
  (2,5){H}
\uput[0](H){$\not\exists\max A$}%
\uput[180](H){$\sup A$}%

\pstGeonode[PointName={c_2,c_2}]%
  (2,6){G}(! \psGetNodeCenter{G} G.x 4 add G.y){G'}

\uput[0](! \psGetNodeCenter{D} D.x D.y 0.5 sub){$A$}%
\uput[0](D'){$A$}%
\psaxes[arrows=->,ticks=none,xAxis=false,labels=none]%
  (2,0)(2.5,0)(2.5,6.75)%
\psaxes[arrows=->,ticks=none,xAxis=false,labels=none]%
  (6,0)(6.5,0)(6.5,6.75)%
\end{pspicture}
\end{center}

En el dibujo, representamos, sobre sendas rectas reales verticales, dos casos cuando $A\subset\Rbb$
es acotado por arriba y por abajo. En casos así, se dice, simplemente, que $A$ es acotado. Se
tiene, entonces, la siguiente definición.

\begin{defical}[Conjunto acotado]
Un conjunto $A\subset\Rbb$ es acotado si y solo si $A$ es acotado por arriba y por abajo.
\end{defical}

En otras palabras, un conjunto $A$ es acotado si y solo existen dos constantes reales $c_1$ y $c_2$
tales que
\[
c_1 \leq x \leq c_2
\]
para todo $x \in A$. Si tomamos $R = |\max\{c_1,c_2\}|$, entonces, $A$ es acotado si y solo si
existe $R > 0$ tal que
\[
|x| < R
\]
para todo $x \in A$.

La siguiente es una caracterización del supremo y del ínfimo, muy útil a la hora de trabajar con
estos dos números.

\begin{teocal}\label{teo:daCaracterizacionSupInf}
Sea $A\subset\Rbb$. Entonces:
\begin{enumerate}
\item Si $\inf A > -\infty$, entonces para todo $\epsilon > 0$, existe $a \in A$ tal que $\inf
    A \leq a < \inf A + \epsilon$.
\item Si $\sup A < \infty$, entonces para todo $\epsilon > 0$, existe $a \in A$ tal que $\sup A
    - \epsilon < a \leq \sup A$.
\end{enumerate}
\end{teocal}

Es decir, el $\inf A$ ($\sup A$) puede ser aproximado por un elemento de $A$ con la precisión que
queramos.

Gracias a estos conceptos, podemos introducir otros análogos para las funciones reales, que son
aquellas cuyo conjunto de llegada es $\Rbb$.

\begin{defical}[Función acotada]
Sean $\Omega\neq\emptyset$ y $\funcjc{f}{\Omega}{\Rbb}$. Entonces, las función $f$ es:
\begin{enumerate}
\item \textbf{acotada por abajo (arriba)} si y solo si el conjunto $\Img(f)$ es acotado por
    abajo (arriba).
\item \textbf{acotada} si y solo si el conjunto $\Img(f)$ es acotado.
\end{enumerate}
\end{defical}

A partir de las definiciones de conjuntos acotados, podemos caracterizar de la siguiente manera a
las funciones acotadas.

\begin{teocal}\label{teo:daCaracterizacionFuncAcotadas}
Sean $\Omega\neq\emptyset$ y $\funcjc{f}{\Omega}{\Rbb}$. Entonces, las función $f$ es:
\begin{enumerate}
\item \textbf{acotada por abajo} si y solo si el conjunto existe $c_1\in\Rbb$ tal que
    \[
      f(x) \geq c_1,
    \]
\item \textbf{acotada por arriba} si y solo si el conjunto existe $c_2\in\Rbb$ tal que
    \[
      f(x) \leq c_2,
    \]
\item \textbf{acotada} si y solo si el conjunto existen $c_1\in\Rbb$ y $c_2\in\Rbb$ tales que
    \[
      c_1 \leq f(x) \leq c_2,
    \]
\item \textbf{acotada} si y solo si el conjunto existe $R > 0$ tal que
    \[
      |f(x)| < R,
    \]
\end{enumerate}
para todo $x\in\Omega$.
\end{teocal}

Tenemos las siguientes definiciones.

\begin{defical}
Sean $\Omega\neq\emptyset$ y $\funcjc{f}{\Omega}{\Rbb}$. Definimos:
\begin{enumerate}
\item $\displaystyle\inf_{x\in\Omega} f(x) = \inf\Img(f)$.
\item $\displaystyle\sup_{x\in\Omega} f(x) = \sup\Img(f)$.
\item $\displaystyle\min_{x\in\Omega} f(x) = \min\Img(f)$.
\item $\displaystyle\max_{x\in\Omega} f(x) = \max\Img(f)$.
\end{enumerate}
\end{defical}

En el caso de que exista $\min\Img(f)$, existe $x_m\in\Omega$ tal que
\[
f(x_m) = \min_{x\in\Omega} f(x).
\]
Se dice, entonces, que \textbf{la función $f$ alcanza su mínimo en $x_m$}.

De manera similar, si existiera $\max\Img(f)$, existiría $x_M\in\Omega$ tal que
\[
f(x_M) = \max_{x\in\Omega} f(x).
\]
Se dice, entonces, que \textbf{la función $f$ alcanza su máximo en $x_M$}.

Los números $x_m$ y $x_M$ podrían no ser únicos; es decir, una función podría alcanzar o su máximo
o su mínimo en varios elementos de su dominio. Por ejemplo, la función $f$ definida por $f(x) =
|\sen(x)|$ para todo $x \in [-\frac{\pi}{2},\frac{\pi}{2}]$, alcanza su valor máximo $1$ tanto en
$-\frac{\pi}{2}$ como en $\frac{\pi}{2}$.

A los números $\displaystyle\min_{x\in\Omega} f(x)$, que se lee ``mínimo de $f$ en $\Omega$'', y
$\displaystyle\max_{x\in\Omega} f(x)$, que se lee ``máximo de $f$ en $\Omega$'', se les llama
\textbf{extremos globales o absolutos de $f$}.

Ilustremos estos conceptos con un ejemplo sencillo.

\begin{exemplo}[]{}
Sean $\Omega\subset\Rbb$ y $\funcjc{f}{\Omega}{\Rbb}$ definida por
\[
f(x) = 1 + (x - 2)^2.
\]
\begin{enumerate}[leftmargin=*]
\item Supongamos que $\Omega = \Rbb$. Entonces, el gráfico de $f$ es la parábola cuyo vértice
    es el punto de coordenadas $(2,1)$, como se puede ver en el siguiente dibujo:
    \begin{center}
    \begin{pspicture}(-1,-0.5)(5,6.5)
      \psaxes[arrows=->,ticks=none,labels=none]%
        (0,0)(-1,-0.5)(4.75,6)%
      \uput[-90](4.75,0){\footnotesize$x$}%
      \uput[0](0,5.75){\footnotesize$y$}

      \psplot[]%
        {-0.225}{4.225}{x 2 sub dup mul 1 add}%

      \uput[180](0,1){$1$}%
      \uput[-90](2,0){$2$}%
      \psline[linestyle=dashed,linecolor=gray]%
        (0,1)(2,1)(2,0)
    \end{pspicture}
    \end{center}
    Se puede ver, entonces, que la función $f$ es:
    \begin{enumerate}[leftmargin=*]
    \item no acotada por arriba, por lo que $\displaystyle\sup_{x\in\Rbb} f(x) = +\infty$;
    \item acotada por abajo y:
        \[
          \inf_{x\in\Rbb} f(x) = \min_{x\in\Rbb} f(x) = f(2) = 1.
        \]
    \end{enumerate}

\item Supongamos que $\Omega = [1,4[$. Entonces la función $f$ es:
    \begin{enumerate}[leftmargin=*]
    \item acotada por arriba, $\displaystyle\sup_{x\in\Omega} f(x) = 5$ y no existe
        $\displaystyle\max_{x\in\Omega} f(x)$;
    \item acotada por abajo, $\displaystyle\inf_{x\in\Omega} f(x) = 1 = \min_{x\in\Omega}
        f(x) = f(2)$;
    \item acotada.
    \end{enumerate}

\item Supongamos que $\Omega = \ ]2,4]$. Entonces la función $f$ es:
    \begin{enumerate}[leftmargin=*]
    \item acotada por arriba, $\displaystyle\sup_{x\in\Omega} f(x) = 5 = \max_{x\in\Omega}
        f(x) = f(4)$;
    \item acotada por abajo, $\displaystyle\inf_{x\in\Omega} f(x) = 1$, no existe
        $\min_{x\in\Omega} f(x)$;
    \item acotada.
    \end{enumerate}

\item Supongamos que $\Omega = [1,4]$. Entonces la función $f$ es:
    \begin{enumerate}[leftmargin=*]
    \item acotada;
    \item $\displaystyle\sup_{x\in\Omega} f(x) = 5 = \max_{x\in\Omega} f(x) = f(4)$;
    \item $\displaystyle\inf_{x\in\Omega} f(x) = 1 = \min_{x\in\Omega} f(x) = f(2)$.
    \end{enumerate}
\end{enumerate}
\end{exemplo}

El último numeral del ejemplo es un caso particular del resultado notable sobre la existencia de
los extremos globales:

\begin{teocal}[Existencia de los extremos globales para funciones
continuas]\label{teo:daExistenciaExtremosGlobales} Sean $a$ y $b$ en $\Rbb$ tales que $a < b$ y
$\funcjc{f}{[a,b]}{\Rbb}$, una función continua en $[a,b]$. Entonces existen $x_m$ y $x_M$ en
$[a,b]$ tales que
\[
f(x_m) = \min_{x\in [a,b]} f(x) \yjc f(x_M) = \max_{x\in [a,b]} f(x).
\]

Además, se tiene que $\Img(f) \subset [f(x_m),f(x_M)]$; es decir:
\[
f(x_m) \leq f(x) \leq f(x_M)
\]
para todo $x\in [a,b]$.
\end{teocal}

Como se verá más adelante, en realidad, se tiene que $\Img(f) = [f(x_m),f(x_M)]$.

El último ejemplo muestra lo indispensable que resulta exigir que $\Omega$ sea un intervalo cerrado
y acotado. Por ejemplo, si $\Omega$ es uno de los siguientes intervalos: $]-\infty,b]$, $\Rbb$,
$[a,+\infty[$, $]a,b]$, $]a,b[$, etcétera, el resultado podría no darse.

Es también de suma importancia que la función $f$ sea continua. Esto se ilustra en el siguiente
ejemplo.

\begin{exemplo}[]{}
Sean $\Omega = [-1,3]$ y $\funcjc{f}{\Omega}{\Rbb}$, definida por
\[
f(x) =
\begin{cases}
(x + 1)^2 & \text{si } -1 \leq x < 1, \\
1 & \text{si } x = 1, \\
-2 + x & \text{si } 1 < x \leq 3.
\end{cases}
\]
El siguiente es el gráfico de $f$:
\begin{center}
\begin{pspicture}(-2,-2)(4.5,5)
\psaxes[arrows=->,Dy=4]%
  (0,0)(-2,-2)(4,5)%
\uput[-90](4,0){$x$}%
\uput[0](0,5){$y$}%

\psplot[arrows=*-o]%
  {-1}{1}{x 1 add dup mul}%
\psplot[arrows=o-*]%
  {1}{3}{x 2 sub}%
\psdot[](1,1)

\end{pspicture}
\end{center}
Como se puede ver, $f$ no es continua en $x = 1$, por lo que $f$ no es continua en $\Omega$.
También se puede ver que
\[
\sup_{x\in\Omega} f(x) = 4 \yjc \not\exists \max_{x\in\Omega} f(x)
\]
y que
\[
\inf_{x\in\Omega} f(x) = -1 \yjc \not\exists \min_{x\in\Omega} f(x).
\]
\end{exemplo}

Si $I$ es un intervalo y $\funcjc{f}{I}{\Rbb}$ es una función continua, el cálculo de $\Img(f)$
puede no ser tan simple como el dado en el teorema~\ref{teo:daExistenciaExtremosGlobales} para el
caso $I = [a,b]$. Es muy útil, en estos casos, expresar $I$ (cuando sea posible) como la unión
finita de subintervalos de $I$, digamos $\displaystyle I = \bigcup_{k=1}^N I_k$, de modo que $f$
sea monótona en cada uno de los subintervalos $I_k$, porque se puede aplicar el resultado que sigue
y el hecho de que $\displaystyle \Img(f) = \bigcup_{k=1}^N f(I_k)$, donde $f(A) = \{f(x) : x\in
A\}$ y $A$ es un subconjunto del dominio de $f$.

\begin{lemacal}\label{teo:daImagenMonotonas}
Sean $I = ]a,b[$ y $\funcjc{f}{I}{\Rbb}$, continua y monótona en $I$. Entonces $f(I)$ es el
intervalo abierto de extremos $\displaystyle\limjc{f(x)}{x}{a^+}$ y $\displaystyle\limjc{f(x)}{x}{b^-}$.
\end{lemacal}

De manera más precisa, si $A = \displaystyle\limjc{f(x)}{x}{a^+}$ y $B = \displaystyle\limjc{f(x)}{x}{b^-}$, entonces
\[
\Img(f) = \ ]A,B[
\]
si $f$ es creciente; en cambio:
\[
\Img(f) = \ ]B,A[
\]
si $f$ es decreciente.

Si alguno de los límites laterales es infinito, digamos que $A = -\infty$; entonces tenemos que
\[
\Img(f) = \ ]-\infty, B[.
\]

El lema sigue siendo verdadero si en lugar de intervalos finitos se tienen intervalos infinitos.
Por otro lado, si $I$ es cerrado en un de los dos extremos, o ambos, al ser $f$ una función
continua en $I$, el limite se reemplaza con el correspondiente valor de la función. Por ejemplo, si
$I = [a,b[$ y $f$ es decreciente, entonces
\[
\Img(f) = \ ]B, f(a)].
\]

\begin{exemplo}[]{}
Si $\funcjc{f}{I}{\Rbb}$ está definida por
\[
f(x) = x^2 - 4x + 3,
\]
donde $I = \ ]-1,4]$, vamos a encontrar su imagen.

Sabemos que el gráfico de $f$ es una parábola de vértice el punto de coordenadas $(2,-1)$ tal que
la parábola es decreciente en $]-\infty, 2]$ y creciente en $[2,+\infty[$. En particular, $f$ es
decreciente en $I_1 = \ ]-1,2]$ y creciente en $I_2 = [2,4]$. Además sabemos que $I = I_1 \cup
I_2$.

Sabemos que
\[
f(I_1) = [f(2),\limjc{f(x)}{x}{-1^+}[ = [f(2),f(-1)[,
\]
ya que $f$ es continua en $I$. Por lo tanto:
\[
f(I_1) = [-1,8[.
\]

Por otro lado, tenemos que:
\[
f(I_2) = [f(2),f(4)] = [-1,3].
\]

Entonces:
\[
\Img(f) = f(I_1) \cup f(I_2) = [-1,8[ \cup [-1,3] = [-1,8[.
\]
Además:
\[
\min_{x\in I} f(x) = -1 \yjc \sup_{x\in I} f(x) = 8
\]
y no existe el máximo de $f$ en $I$.
\end{exemplo}

\section{Extremos locales o relativos}

\begin{wrapfigure}[14]{r}{0pt}
\def\f{x RadtoDeg dup dup sin exch 2 mul sin add exch 3 mul sin add}%
\psset{plotpoints=1000}
\begin{pspicture}(-3.5,-5)(4,3)
   %\psgrid[subgriddiv=0,gridlabels=7pt,griddots=10]
   \psaxes[ticks=none,labels=none]{->}%
      (0,0)(-3.5,-2.5)(3.75,3)%
   \uput[-90](3.75,0){$x$}%
   \uput[0](0,3){$y$}%
%
   \psplot[linecolor=gray]%
      {-\psPi}{\psPi}{\f}%
\end{pspicture}
\end{wrapfigure}

Dada una función real $f$ definida en un intervalo $I$, el gráfico de $f$ puede tener ``montes y
valles'', como se ilustra en gráfico de la derecha. Es interesante poder hallar las cimas ---punto
más alto de los montes, cerros y collados--- y simas ---cavidad grande y muy profunda en la
tierra--- que corresponderán a ``máximos locales'' y a ``mínimos locales'', respectivamente. ?`Qué
queremos decir con ``locales''? Que, como se aprecia en la figura, son los puntos más altos o los
más bajos pero solamente en una ``zona próxima a ellos'' y no con respecto a todo el intervalo $I$.
En otras palabras, ``localmente'' son los máximos o los mínimos. A continuación, hagamos precisas
estas nociones y cómo los extremos locales nos permiten hallar los extremos globales para el caso
de las funciones continuas.

\begin{defical}[Extremos locales]
Sean $I$ un intervalo y $\funcjc{f}{I}{\Rbb}$. Sean $\underline{x}$ y $\overline{x}$ dos elementos
de $I$. Entonces:
\begin{enumerate}
\item la función $f$ \textbf{alcanza un mínimo local o relativo en $\underline{x}$} si y solo
    si existe $r > 0$ tal que
    \[
    f(\underline{x}) \leq f(x)
    \]
    para todo $x \in I \cap \ ]\underline{x} - r, \underline{x} + r[$.

\item la función $f$ \textbf{alcanza un máximo local o relativo en $\overline{x}$} si y solo si
    existe $r > 0$ tal que
    \[
    f(\overline{x}) \geq f(x)
    \]
    para todo $x \in I \cap \ ]\underline{x} - r, \underline{x} + r[$.
\end{enumerate}
\end{defical}

En otras palabras, un mínimo local en $\underline{x}$ es el valor más pequeño que toma una función
en un cierto intervalo centrado en $\underline{x}$. De manera similar, un máximo local en
$\overline{x}$ es el valor más grande que toma una función en un cierto intervalo centrado en
$\overline{x}$.

\begin{defical}[Interior de un intervalo]
Sea $I$ un intervalo abierto. El interior de $I$, representado con $I^\circ$, es el intervalo
abierto más grande que está contenido en $I$.
\end{defical}

\begin{exemplo}[]{}
El interior de $I = [a,b]$ es $I^\circ = ]a,b[$. El interior de $I = [a,b[$ es $I^\circ = ]a,b[$.
Para $I = ]-\infty,b[$, el interior $I^\circ$ es $]-\infty, b[$.
\end{exemplo}

\begin{wrapfigure}{r}{0pt}
\def\f{x RadtoDeg dup dup sin exch 2 mul sin add exch 3 mul sin add}%
\psset{plotpoints=1000}
\begin{pspicture}(-3.5,-3)(4,3)
   %\psgrid[subgriddiv=0,gridlabels=7pt,griddots=10]
   \psaxes[ticks=none,labels=none]{->}%
      (0,0)(-3.5,-3)(3.75,3)%
   \uput[-90](3.75,0){$x$}%
   \uput[0](0,3){$y$}%

   \psplot[linecolor=gray]%
      {-\psPi}{\psPi}{\f}%

   \psset{linewidth=0.7pt}
   \psplotTangent{0.667291072}{0.5}{\f}%
   \psplotTangent{1.81519841}{0.5}{\f}%
   \psplotTangent{2.640083649}{0.5}{\f}%
   \psplotTangent{-0.66729107}{0.5}{\f}%
   \psplotTangent{-1.81519841}{0.5}{\f}%
   \psplotTangent{-2.640083649}{0.5}{\f}%
\end{pspicture}
\end{wrapfigure}

Lo que ahora nos interesa es determinar un método para la obtención de los extremos locales de una
función; es decir, para la localización de los máximos y mínimos locales de una función. En el
dibujo de la derecha, podemos observar que las rectas tangentes en los extremos de la función lucen
como rectas horizontales, es decir, con pendiente igual a $0$. Si esto es así, la derivada de la
función debería ser igual a $0$ en estos puntos. Y esto es, efectivamente, así. Es decir, la recta
pendiente en un extremo es horizontal. Por ello, se introduce el concepto de punto crítico. Luego
de la siguiente definición se enuncia el teorema ilustrado por el dibujo.

\begin{defical}[Punto crítico]
Dada una función real $f$, continua en un intervalo $I$ de extremos $a$ y $b$ ($a < b$), a un número $c$ del interior de $I$ se lo llama \textbf{punto
crítico de $f$} si y solo si $f'(c) = 0$ o si no existe $f'(c)$, es decir, si $f$ no es derivable
en $c$.
\end{defical}

\begin{exemplo}[]{}
Sea $\funcjc{f}{[-2,3]}{\Rbb}$ definida por
\[
f(x) =
\begin{cases}
1 - x^2 & \text{si } -2 \leq x \leq 1, \\
2x^3 - 3x^2 - 12x + 13 & \text{si } 1 < x \leq 3.
\end{cases}
\]
Encontremos los puntos críticos de $f$.

Definamos $p_1$ y $p_2$ de la siguiente manera:
\[
p_1(x) = 1 - x^2 \yjc p_2(x) = 2x^3 - 3x^2 - 12x + 13.
\]
Entonces, $f$ es continua en $[-2,3]$ pues:
\begin{enumerate}[leftmargin=*,listparindent=\parindent]
\item En el intervalo $[-2,1]$, la función $f = p_1$, por lo que $f$ es continua en el abierto
    $]-2,1[$, continua en $-2$ por la derecha y continua en $1$ por la izquierda.

\item En el intervalo $]1,3]$, la función $f = p_2$, por lo que $f$ es continua en el abierto
    $]1,3[$ y continua en $3$ por la izquierda.

\item En el punto $x = 1$, tenemos que:
    \[
      f(1) = p_1(1) = 0 = p_2(1) = \limjc{p_2(x)}{x}{1^+},
    \]
    por lo que $f$ es continua en $1$.
\end{enumerate}

Por otro lado, tenemos que
\[
f'(x) =
\begin{cases}
-2x & \text{si } -2 \leq x < 1, \\
6x^2 - 6x - 12 & \text{si } 1 < x \leq 3,
\end{cases}
\]
y para $x = 1$, no existe $f'(x)$, pues
\[
 f_-'(1) = -2 \neq -12 = f_+'(1).
\]
Entonces
\[
f'(x) = 0 \ \Longleftrightarrow \
\begin{cases}
-2x = 0 & \text{si } -2 \leq x < 1, \\
6x^2 - 6x - 12 = 0 & \text{si } 1 < x \leq 3,
\end{cases}
\]
de donde
\[
f'(x) = 0 \ \Longleftrightarrow \
x \in\{0,2\}.
\]

Por lo tanto, de acuerdo a la definición de punto crítico, a más de $0$ y de $2$, el número $1$
también es un punto crítico de $f$.
\end{exemplo}

El siguiente teorema es el mecanismo que tenemos para la determinación de los extremos relativos.

\begin{teocal}[Extremos locales]\label{teo:daExtremosLocalesPuntosCriticos}
Sean $I$ un intervalo, $\funcjc{f}{I}{\Rbb}$ y $c\in I^\circ$. Si $f$ alcanza un extremo local en
$c$, entonces $c$ es un valor crítico de $f$.
\end{teocal}

Usaremos este teorema para la obtención de los extremos absolutos de una función real continua
definida en un intervalo cerrado y acotado.

En efecto, supongamos que $\funcjc{f}{[a,b]}{\Rbb}$ es continua. Entonces, existen $x_m$ y $x_M$ en
$[a,b]$ en los cuales $f$ alcanza el mínimo y el máximo globales, respectivamente. Sea $K$ el
conjunto de todos los puntos críticos de $f$; es decir:
\[
K = \{c \in \ ]a,b[ : c \ \text{ es un punto crítico de } \ f\}.
\]
Se tiene entonces que $x_m\in \{a,b\}\cup K$ y que $x_M\in \{a,b\}\cup K$, pues todo extremo global
también es un extremo local.

Ahora bien, si $K$ es finito, por ejemplo, $K = \{c_1, c_2,\ldots, c_N\}$, y es conocido, para
encontrar los extremos globales de $f$ es suficiente que hagamos una tabla de valores con los
elementos del conjunto
\[
\{f(a), f(c_1), f(c_2), \ldots, f(c_N), f(b)\}.
\]
Es obvio que el mayor de estos números corresponderá al máximo global de $f$ y el menor, al mínimo
global. En el caso de que $K = \emptyset$, los únicos extremos son $f(a)$ y $f(b)$. Veamos un
ejemplo.

\begin{exemplo}[]{}
Sea $I = [0,\frac{5}{2}]$ y $\funcjc{f}{I}{\Rbb}$ definida por
\[
f(x) = 2x^3 - 9x^2 + 12x.
\]
Determinemos el conjunto $K$.

Como $f'(x) = 6x^2 - 18x + 12 = 6(x - 2)(x-1)$, tenemos que los $c\in I$ que satisfacen la igualdad
$f'(c) = 0$ son $c = 1$ y $c = 2$. Además, como $f$ es derivable en el interior de $I$, el conjunto
$K$ es, entonces $K = \{1,2\}$.

Ahora construyamos la tabla de valores de $f$ en $K \cup \{f(0),f(\frac{5}{2})\}$:
\[
\setlength\extrarowheight{4pt}
\begin{array}{c|cccc}
x & 0 & 1 & 2 & \frac{5}{2} \\[4pt] \hline
f(x) & 0 & 5 & 4 & 5
\end{array}
\]
Por lo tanto, la función $f$ alcanza el mínimo en $0$ y el máximo en $\frac{5}{2}$. Además:
\[
\min_{x\in I} f(x) = 0 = f(0) \yjc \max_{x\in I} f(x) = 5 = f\left(\frac{5}{2}\right)=f(1).
\]
\end{exemplo}

\section{Monotonía}

Dada una función $\funcjc{f}{I}{\Rbb}$, donde $I$ es un intervalo, se dice que $f$ es creciente si
al aumentar el valor de la variable independiente $x$, lo hace también el valor $f(x)$. Si $f(x)$
decrece, entonces se dice que $f$ es decreciente. Precisemos estos conceptos.

\begin{defical}[Funciones creciente y decreciente]
Sean $I$ un intervalo y $\funcjc{f}{I}{\Rbb}$. Entonces la función $f$ es
\begin{enumerate}
\item \textbf{creciente en $I$} si y solo si
    $
      f(x_1) < f(x_2)
    $; para todo $x_1 \in I$ y todo $x_2 \in I$ tales que $x_1 < x_2$;
      
\item \textbf{no decreciente en $I$} si y solo si
    $
      f(x_1) \leq f(x_2)
    $; para todo $x_1 \in I$ y todo $x_2 \in I$ tales que $x_1 < x_2$;

\item \textbf{decreciente en $I$} si y solo si
    $
      f(x_1) > f(x_2)
    $; para todo $x_1 \in I$ y todo $x_2 \in I$ tales que $x_1 < x_2$;

\item \textbf{no creciente en $I$} si y solo si
    $
      f(x_1) \geq f(x_2)
    $; para todo $x_1 \in I$ y todo $x_2 \in I$ tales que $x_1 < x_2$; y,

\item \textbf{monótona} si y solo si es creciente o decreciente.
\end{enumerate}
\end{defical}

Es claro de la definición que una función puede ser creciente en un subintervalo pero creciente en
otro.

\begin{exemplo}[]{}
Sea $\funcjc{f}{]\frac{3}{2},+\infty [}{\Rbb}$ definida por $f(x) = x^2 - 3x + 2$. Determinemos si es
creciente o decreciente. Para ello, sean $x_1$ y $x_2$ tales que sean mayores que $\frac{3}{2}$ y
$x_1 < x_2$. Ahora estudiemos el signo de $f(x_2) - f(x_1)$.

Para ello, observemos que:
\begin{align*}
f(x_2) - f(x_1) &= (x_2^2 - 3x_2 + 2) - (x_1^2 - 3x_1 + 2) \\
  &= (x_2^2 - x_1^2) - 3(x_2 - x_1) \\
  &= (x_2 - x_1)(x_2 + x_1 - 3) \\
  &= (x_2 - x_1)[(x_2 - \frac{3}{2}) + (x_1 - \frac{3}{2})] > 0,
\end{align*}
pues $x_1 < x_2$, $x_1 > \frac{3}{2}$ y $x_2 > \frac{3}{2}$. Por lo tanto, $f(x_1) < f(x_2)$; es
decir, $f$ es creciente.
\end{exemplo}

Probar que una función es creciente o decreciente no siempre es una tarea sencilla como la que
muestra el último ejemplo. El lector podrá convencerse de esta afirmación si estudia la monotonía
de la función $f$ definida por $f(x) = 2x^3 - 9x^2 + 12x$ con el método utilizado en el ejemplo.
Constataría lo engorrosos que pueden llegar a ser los cálculos.

\begin{wrapfigure}[14]{r}{0pt}
\def\f{x RadtoDeg dup dup sin exch 2 mul sin add exch 3 mul sin add}%
\psset{plotpoints=1000}
\begin{pspicture}(-3.5,-3)(4,3)
   %\psgrid[subgriddiv=0,gridlabels=7pt,griddots=10]
   \psaxes[ticks=none,labels=none]{->}%
      (0,0)(-3.5,-3)(3.75,3)%
   \uput[-90](3.75,0){$x$}%
   \uput[0](0,3){$y$}%
%
   \psplot[linecolor=gray]%
      {-\psPi}{\psPi}{\f}%
%
   \psset{linewidth=0.6pt}
   \psplotTangent{-2.8}{0.5}{\f}%
   \psplotTangent{-2}{0.5}{\f}%
   \psplotTangent{-0.7}{0.5}{\f}%
   \psplotTangent{0.5}{0.5}{\f}%
   \psplotTangent{1.6}{0.5}{\f}%
   \psplotTangent{2.5}{0.5}{\f}%
\end{pspicture}
\end{wrapfigure}

Viene entonces en nuestra ayuda un resultado que se ilustra, previamente, en el dibujo de la
derecha. Podemos observar que la recta tangente en cualquier punto de la curva en los intervalos en
los que es creciente tiene una pendiente positiva. En cambio, en los intervalos en que la curva es
decreciente, la pendiente es negativa.

Esta observación, y el hecho de que la derivada representa la pendiente de la tangente, inducen a
pensar que si la derivada de una función fuera positiva en $(a,b)$, esa función debería ser
creciente en ese intervalo y, en el caso contrario, si la derivada fuera negativa, la función
debería ser decreciente.

\begin{teocal}\label{teo:daDerivadaPositiva}%
Sean $I$ un intervalo de extremos $a < b$ $f\colon I \rightarrow  \mathbb{R}$ continua en $I$. Si,
además, $f$ es derivable en $]a,b[$, entonces:
\begin{enumerate}
\item si $f'(x) > 0$ para todo $x\in\ ]a,b[$, entonces $f$ es creciente en $I$; y
\item si $f'(x) < 0$ para todo $x\in\ ]a,b[$, entonces $f$ es decreciente en $I$.
\end{enumerate}
\end{teocal}

Veamos un ejemplo.

\begin{exemplo}[]{}
Sea $\funcjc{f}{\Rbb}{\Rbb}$ definida por $f(x) = 2x^3 - 9x^2 + 12x$. Su derivada es $f'(x) = 6x^2
- 18x + 12$. Como $f'(x) = 6(x - 1)(x - 2)$, los signos de $f$ son los siguientes:
\[
\begin{array}{c|c|c|c|c|c}
x & & 1 & & 2 & \\ \hline
(x - 1) & - & 0 & + & + & + \\
(x - 2) & - & - & - & 0 & + \\
f'(x) & + & 0 & - & 0 & +
\end{array}
\]

Vemos, entonces, que $f'(x) > 0$ si $x \in \ ]-\infty, 1[ \cup \ ]2, +\infty[$ y que $f'(x) < 0$ si
$x \in \ ]1,2[$. Por lo tanto, la función $f$ es creciente en $]-\infty, 1[ \cup \ ]2, +\infty[$ y
es decreciente en $]1,2[$.
\end{exemplo}

\begin{corocal}[Extremos locales]
Sean $I$ un intervalo abierto, $c \in I$ y $\funcjc{f}{I}{\Rbb}$ una función continua en $I$ y
derivable en $I - \{c\}$. Si $c$ es un punto crítico de $f$ y $f'(x)$ cambia de signo en $c$,
entonces $f$ alcanza un extremo local en $c$. De manera más precisa: si $x\in I$ y
\begin{enumerate}
\item si $f'(x) < 0$ para todo $x < c$; y
\item si $f'(x) > 0$ para todo $x > c$,
\end{enumerate}
entonces $f$ alcanza un mínimo local en $c$; y
\begin{enumerate}
\item si $f'(x) > 0$ para todo $x < c$; y
\item si $f'(x) < 0$ para todo $x > c$,
\end{enumerate}
entonces $f$ alcanza un máximo local en $c$.
\end{corocal}

\begin{exemplo}[]{}
En el ejemplo precedente, si $f$ está definida por $f(x) = 2x^3 - 9x^2 + 12x$, la derivada $f'(x)$
de esta función cambia de signo en $1$ y en $2$. Antes de $1$ es creciente, luego de $1$ es
decreciente; por lo tanto, $f$ alcanza un máximo local en $1$. Antes de $2$ es decreciente y luego
es creciente. Entonces, $f$ alcanza un mínimo local en $2$.
\end{exemplo}

\begin{multicols}{2}[\section{Ejercicios}]
\begingroup
\small
\begin{enumerate}[leftmargin=*]
\item Para la función $f$, halle los puntos críticos y los intervalos de monotonía. Distinga
    los puntos críticos que son extremos locales:
    \begin{enumerate}[leftmargin=*]
    \item $\displaystyle f(x) = 3x^2 - 5x + 4$.
    \item $\displaystyle f(x) = 6x^4 - 8x^3 + 2$.
    \item $\displaystyle f(x) = \frac{x + 1}{x^2 + 1}$.
    \item $\displaystyle f(x) = x^{\frac{4}{2}} - x^{\frac{1}{3}}$.
    \item $\displaystyle f(x) = \frac{x - 1}{(x + 7)^{\frac{1}{3}}}$.
    \item $\displaystyle f(x) = x^3 + 2x^2 - 7x + 4$.
    \item $\displaystyle f(x) = \frac{x - 2}{\sqrt{1 - x}}$.
    \item $\displaystyle f(x) = (5x + 2)^{\frac{1}{3}}$.
    \item $\displaystyle f(x) = x^2(x + 7)^{\frac{1}{3}}$.
    \item $\displaystyle f(x) = 3\sen x + 4\cos x$.
    \end{enumerate}

\item Halle los extremos absolutos de $f$ en el intervalo $I$:
    \begin{enumerate}[leftmargin=*]
    \item $\displaystyle f(x) = x^3 - 2x^2 + x; \ I = [-1,4]$.
    \item $\displaystyle f(x) = 3x^2 - 5x + 2; \ I = [-1,3]$.
    \item $\displaystyle f(x) = (x - 1)^{\frac{2}{3}}; \ I = [0,10]$.
    \item $\displaystyle f(x) = (x - 1)^{\frac{2}{3}}(x^2 - 2x); \ I = [0,2]$.
    \item $\displaystyle f(x) = x^3 + x^2 - 5x + 3; \ I = [-2,4]$.
    \item $\displaystyle f(x) = (x + 1)^4(x - 2)^2; \ I = [0,5]$.
    \item $\displaystyle f(x) = \frac{\sqrt{x - 1}}{x^2 + 2}; \ I =
        \left[\frac{5}{4},5\right]$.
    \end{enumerate}

\item Halle los valores críticos y los extremos absolutos y relativos de la función $f$ en el
    intervalo $I$:
    \begin{enumerate}[leftmargin=*]
    \item $\displaystyle f(x) = x^4 - 4 - |x + 2|; \ I = [-3,3]$.
    \item $\displaystyle f(x) =
          \begin{cases}
            2x + 1 & \text{si } -2 \leq x < 1, \\
            x^2 + 2 & \text{si } 1 \leq x \leq 3;
          \end{cases}\\[6pt]
          I = [-2,3]$.
    \item $\displaystyle f(x) = 2x^3 - 3x^2 - 12x + 13; \ I = [-2,3]$.
    \item $\displaystyle f(x) =
          \begin{cases}
            x^2 + x - 2 & \text{si } -2 \leq x < 1, \\
            x^3 - 3x^2 + 2 & \text{si } 1 \leq x \leq 3;
          \end{cases}\\[6pt]
          I = [-2,3]$.
    \item $\displaystyle f(x) = x + \frac{1}{x - 1} - 1; \ I = [-10,10]$.
    \item $\displaystyle f(x) = \lfloor x \rfloor; \ I = [-10,2]$.
    \item $\displaystyle f(x) = \frac{ax + b}{cx + d}; \ I = [-L,L]$, con $L > 0$, y si $ad
        \neq bc$ o si $ad = bc$.
    \end{enumerate}
\end{enumerate}
\endgroup
\end{multicols}

\section{Teoremas del valor intermedio}

\begin{wrapfigure}[11]{r}{0pt}
\def\f{2*cos(x-4.5)-2*sin(2*x - 9)+2*cos(2*x - 9)-2*sin(x - 4.5)+ 2.5 }%
\def\g{x RadtoDeg dup dup dup 4.5 RadtoDeg sub cos 2 mul exch
   4.5 RadtoDeg sub 2 mul sin 2 mul sub exch
   4.5 RadtoDeg sub 2 mul cos 2 mul add exch
   4.5 RadtoDeg sub sin 2 mul sub 2.5 add }

\begin{pspicture}(-0.5,-0)(5,4.25)
   %\psgrid[subgriddiv=0,gridlabels=7pt,griddots=10]
   \psaxes[ticks=none,labels=none]{->}%
      (0,0)(-0.5,-0.5)(4.75,4)%
   \uput[-90](4.75,0){$x$}%
   \uput[0](0,4){$y$}%

   \begingroup
      \psset{linecolor=gray}
      \psline(2.712185027,0)(! 2.712185027 /x 2.712185027 def \g)%
      \psline(0.5,0)(! 0.5 /x 0.5 def \g)%
      \psline[linestyle=dashed,linecolor=gray]%
         (!0.5 /x 0.5 def \g)(! 2.712185027 /x 2.712185027 def \g)%

      \uput[-90](0.5,0){$a$}%
      \uput[180](! 0.5 /x 0.5 def \g 2 div){$f(a)$}%
      \uput[-90](2.712185027,0){$b$}%
      \uput[0](! 2.712185027 /x 2.712185027 def \g 2 div){$f(b)$}%
   \endgroup

   \psplot[linewidth=1.5\pslinewidth]%
      {0.5}{2.712185027}{\g}%

   \psplotTangent[plotpoints=1000]%
      {1.090732918}{1}{\g}

   \pnode(! 1.090732918 /x 1.090732918 def \g){M}

   \psline[linestyle=dashed,linecolor=gray,linewidth=0.75\pslinewidth]%
         (!1.090732918 0)(! 1.090732918 /x 1.090732918 def \g)%
   \uput[-90](1.090732918,0){$c$}%

   \pnode(2,3.2){A}%
   \psline[linecolor=gray]{->}(A)(M)%
   \uput[0](A){$f'(c) = 0$}

\end{pspicture}
\end{wrapfigure}
En el dibujo de la derecha, se muestra el gráfico de una función $f\colon [a,b] \rightarrow
\mathbb{R}$ que es continua en $[a,b]$, derivable en $(a,b)$ y tal que $f(a)=f(b)$. Parece que, en
el punto $c\in (a,b)$, punto en el que la función $f$ alcanza un máximo local, la tangente al
gráfico de $f$ es horizontal; es decir, $f'(c) = 0$, de donde $c$ es un punto crítico. Aunque no
está dibujado, el punto dónde se alcanza el mínimo local, la recta tangente es también paralela al
eje horizontal; esto significa también que ese punto es un punto crítico.

Y esta situación es siempre verdadera bajo las hipótesis adecuadas, como se expresa en el siguiente
teorema.

\begin{teocal}[Teorema de Rolle]
Si $f $ es continua en el intervalo $[a,b]$ y derivable en $(a,b)$, entonces existe un número $c\in
(a,b)$ tal que:
\[
f(a) = f(b).
\]
Entonces, existe un número $c\in (a,b)$ tal que $f'(c) = 0$.
\end{teocal}

\begin{exemplo}[]{}
Vamos a utilizar el teorema de Rolle para averiguar si la función $f$ definida por
\[
f(x) = x^3 - 3x^2 + 2
\]
tiene puntos críticos en el intervalo $]-\sqrt{3} + 1,\ \sqrt{3} + 1[$.

Como $f$ es derivable en $\Rbb$ por ser un polinomio, es continua en $\Rbb$ y, por consiguiente,
también es continua en $]-\sqrt{3} + 1,\ \sqrt{3} + 1[$. Por lo tanto, los puntos críticos son
todos los $x\in\ ]-\sqrt{3} + 1,\ \sqrt{3} + 1[$ para los cuales $f'(x) = 0$.

Como $f(-\sqrt{3} + 1) = f(\sqrt{3} - 1) = 0$, el teorema de Rolle nos garantiza que existe $c$ en
$]-\sqrt{3} + 1,\ \sqrt{3} + 1[$ tal que $f'(c) = 0$. Para hallar tales números $c$, debemos
resolver la ecuación $f'(c) = 0$.

Ahora bien, como $f'(x) = 3x^2 - 6x = 3x(x - 2)$, tenemos que
\[
f'(c) = 0 \ \Longleftrightarrow \ c(c - 2) = 0 \ \Longleftrightarrow \ c \in\{0,2\}.
\]

En resumen, los números $0$y $2$ son los puntos críticos buscados, puesto que $\{0,2\}$ está
incluido en $]-\sqrt{3} + 1,\ \sqrt{3} + 1[$.
\end{exemplo}

\begin{teocal}[Teorema del valor intermedio o de los incrementos finitos]
Si $f $ es continua en el intervalo $[a,b]$ y derivable en $(a,b)$, entonces existe un número $c\in
(a,b)$ tal que:
\begin{equation*}
	f'(c)=\frac{f(b)-f(a)}{b-a},
\end{equation*}
o también:
\[
   f(b) - f(a) = f'(c)(b - a).
\]
\end{teocal}

\begin{exemplo}[]{}
Sea $f$ la función real definida por $f(x) = x^3 - 3x^2 + 3$. Veamos cómo se aplica el teorema del
valor intermedio en el intervalo $[1,3]$.

Como $f$ es un polinomio, es una función continua y derivable en $\Rbb$ y, por ende, es continua en
el intervalo $[1,3]$ y derivable en el intervalo $]1,3[$. Entonces, se verifican las hipótesis del
teorema del valor intermedio, el que nos garantiza que existe $c\in\ ]1,3[$ tal que
\[
f'(c) = \frac{f(3) - f(1)}{2 - 1} = \frac{3 - 1}{3 - 1} = 1.
\]
Como $f'(x) = 3x^2 - 6x$, el número $c$ satisface la ecuación
\[
3c^2 - 6c = 1.
\]

Las raíces de esta ecuación son $c_1 = 1 - \frac{2}{\sqrt{3}}$ y $c_2 = 1 + \frac{2}{\sqrt{3}}$.
Puesto que $c_1\not\in\ ]1,3[$, entonces el número $c$ del teorema del valor intermedio es $c_2$.
\end{exemplo}

Supongamos que una función $\funcjc{f}{I}{\Rbb}$, donde $I$ es un intervalo de extremos $a$ y $b$
tales que $a < b$, tiene derivada nula en $]a,b[$; es decir, supongamos que $f'(x) = 0$ para todo
$x\in\ ]a,b[$. Tomemos dos elementos en $I$, $x_1$ y $x_2$ tales que $x_1 < x_2$. Si sucediera que
$f(x_1) \neq f(x_2)$, por el teorema del valor intermedio, existe $c\in\ ]x_1,x_2[$ tal que
\[
f'(c) = \frac{f(x_2) - f(x_1)}{x_2 - x_1}.
\]
Por un lado, $f'(c) = 0$; por otro, como $x_1 \neq x_2$ y, hemos supuesto que $f(x_1)\neq f(x_2)$,
tenemos que
\[
0 = f'(c) = \frac{f(x_2) - f(x_1)}{x_2 - x_1} \neq 0.
\]
Esta contradicción nos indica que no es posible que $f(x_1) \neq f(x_2)$; por lo tanto, la función
$f$ debe ser constante en $]a,b[$.

Resumamos este resultado en el siguiente teorema.

\begin{teocal}\label{teo:daDerivadaCero}
Si $f$ está definida y es continua en un intervalo $I$, y si $f$ es derivable en el interior de $I$ y su derivada es igual a $0$, entonces $f$ es constante en $I$.
\end{teocal}

Finalmente, el siguiente es una generalización del teorema del valor intermedio. En los ejercicios
de esta sección, se provee una sugerencia para la demostración del teorema general.

\begin{teocal}[Teorema general del valor intermedio]
Sean $f\colon \mathbb{R} \rightarrow  \mathbb{R}$ y $g\colon \mathbb{R} \rightarrow  \mathbb{R}$
dos funciones continuas en el intervalo $[a,b]$ y derivables en $(a,b)$. Supongamos que $g'(x)\neq
0$ para todo $x\in (a,b)$ y que $g(a)\neq g(b)$. Entonces existe $c\in (a,b)$ tal que:
\begin{equation*}
	 \frac{f(b)-f(a)}{g(b)-g(a)}=\frac{f'(c)}{g'(c)}.
\end{equation*}
\end{teocal}

\begin{multicols}{2}[\section{Ejercicios}]
\begingroup
\small
\begin{enumerate}[leftmargin=*]
\item Verifique si en el intervalo $I$ se puede aplicar el teorema de Rolle para la $f$ dada.
    De ser así, halle los $c$ cuya existencia garantiza el mencionado teorema.
    \begin{enumerate}[leftmargin=*]
    \item $\displaystyle f(x) = x^2 - 5;\ I = [-2,2]$.
    \item $\displaystyle f(x) =
          \begin{cases}
            x^3 + 2 & \text{si } -3 \leq x < 0, \\
            -x^3 + 2 & \text{si } 0 \leq x \leq 3;
          \end{cases}\\[6pt]
          I = [-3,3]$.
    \item $\displaystyle f(x) = x^3 - 3x^2 + 2x + 4;\ I = [0,1]$.
    \item $\displaystyle f(x) =
          \begin{cases}
            x^2 + 3 & \text{si } -2 \leq x \leq 0, \\
            8 - x^3 & \text{si } 0 < x \leq 1;
          \end{cases}\\[6pt]
          I = [-2,1]$.
    \item $\displaystyle f(x) = 3\sen x + 4\cos x;\ I =
        \left[\frac{-\pi}{2},\frac{7\pi}{2}\right]$.
    \item $\displaystyle f(x) = \sec x;\ I = [0,2\pi]$.
    \end{enumerate}

\item Verifique si en el intervalo $I$ la función $f$ satisface las hipótesis del teorema del
    valor intermedio. De ser así, halle los valores de $c$ cuya existencia nos es garantizada
    por dicho teorema.
    \begin{enumerate}[leftmargin=*]
    \item $\displaystyle f(x) = x^3 - x + 1;\ I = [0,3]$.
    \item $\displaystyle f(x) =
          \begin{cases}
            x^3 + 2 & \text{si } -2\leq x < 0,\\
            2 - x^3 & \text{si } 0 \leq x \leq 3;
          \end{cases}\\[6pt]
          I = [-2,3]$.
    \item $\displaystyle f(x) = x^3 - 3x^2 + 2x;\ I = [0,1]$.
    \item $\displaystyle f(x) =
          \begin{cases}
            x^2 + 3 & \text{si } -2 \leq x \leq 0, \\
            3 - x^3 & \text{si } 0 \leq x \leq 1;
          \end{cases}\\[6pt]
          I = [-2,1]$.
    \item $\displaystyle f(x) =
          \begin{cases}
            x^3 + 2 & \text{si } -3 \leq x \leq 1, \\
            2x^2 + 3x - 2 & \text{si } 1 \leq x \leq 2;
          \end{cases}\\[6pt]
          I = [-3,2]$.
    \item $\displaystyle f(x) = x - \sqrt[3]{x};\ I = [-1,27]$.
    \end{enumerate}

\item Un camionero entra en una autopista y recibe un talón que marca la hora de ingreso, las
    $7$ horas con $55$ minutos. Antes de salir, $250$ kilómetros después, paga el peaje y el
    recibo marca el valor pagado y la hora del pago: $10$ horas con $35$ minutos. Un policía le
    revisa los documentos y le multa por exceso de velocidad. ?`Cuál cree que era el límite de
    velocidad para el transporte pesado en ese tramo? Justifique su respuesta sustentándose en
    el teorema del valor intermedio.

\item Mediante el teorema del valor intermedio, demuestre el siguiente teorema (de los
    incrementos finitos). Sean $I$ un intervalo abierto, $\funcjc{f}{I}{\Rbb}$ derivable en
    $I$, $x_0\in I$ y $\Delta x$ tales que $x_0 + \Delta x \in I$. Entonces existe $\theta\in\
    ]0,1[$ tal que
    \[
      \Delta y = f(x_0 + \Delta x) - f(x_0) = f'(x_0 + \theta\Delta x)\Delta x.
    \]

\item Usando el teorema de Rolle, pruebe el teorema general del valor medio.
    \textsc{Sugerencia:} Defina $\funcjc{F}{[a,b]}{\Rbb}$ de la siguiente manera:
    \[
      F(x) = f(x) - \frac{f(b) - f(a)}{g(b) - g(a)}[g(x) - g(a)].
    \]

\item Sea $f(x) = ax^5 + bx^3 + cx + d$. Use el teorema de Rolle para probar que la ecuación
    $f(x) = 0$ no puede tener dos raíces reales si $a > 0$, $b > 0$, $c > 0$ y si $20ac>9b^2$.

\item Use el teorema de Rolle y el método de inducción para probar que un polinomio de grado
    $n\in\Nbb$ y coeficientes reales tiene a lo sumo $n$ raíces reales.
\end{enumerate}
\endgroup
\end{multicols}

\section{Convexidad}

\subsection{Punto intermedio}

Sean $P_{0}$ y  $P_{1}$ dos puntos cualesquiera, distintos entre sí y situados sobre una recta
real. Sean  $x_{0}$ y  $x_{1}$ sus respectivas coordenadas. Si tomamos un punto arbitrario $P$
situado entre $P_{0}$ y  $P_{1}$, de coordenada $x$, resulta que $x$ puede expresarse de una manera
sencilla en función de $x_{0}$  y  $x_{1}$. En efecto, tenemos el siguiente lema:

\begin{lemacal}
Sean $x_{0}$  y  $x_{1}$ en $\mathbb{R}$ dos números reales distintos entre sí. Entonces, para
cualquier $x\in \mathbb{R}$ ``situado'' entre ellos; es decir, tal que $x_0 < x < x_1$ o $x_1 < x <
x_0$, existe $t\in (0,1)$ tal que:
\begin{equation}
\label{eq:algeom001}
x=tx_{1}+(1-t)x_{0}.
\end{equation}
\end{lemacal}

Tenemos dos casos.
\begin{enumerate}[leftmargin=*]
\item Supongamos que $x_0 < x < x_1$: \quad
      %
      \begin{pspicture}(0,0)(4.5,1)
        \psaxes[arrows=->,ticks=none,yAxis=false,labels=none]%
          (0,0)(0,0.5)(4.5,0.5)%
        \pstGeonode[PointSymbol=|,PosAngle=90]%
          (0.5,0){P_0}(2.5,0){P}(3.5,0){P_1}%
        \uput[-90](P_0){$x_0$}%
        \uput[-90](P_1){$x_1$}%
        \uput[-90](P){$x$}%
      \end{pspicture}
      %

      \vspace{0.6\baselineskip}%
      Vemos que las distancias entre $P$ y $P_0$ y entre $P_0$ y $P_1$ son
      \[
        PP_0 = x - x_0 \yjc P_0P_1 = x_1 - x_0.
      \]
      Además, como
      \[
        0 < P_0P < P_0P_1,
      \]
      existe $t \in\ ]0,1[$ tal que
      \[
        P_0P = tP_0P_1 = t(x_1 - x_0).
      \]
      Por lo tanto:
      \[
        x = x_0 + P_0P = x_0 + t(x_1 - x_0) = tx_1 + (1 - t)x_0.
      \]

\item Supongamos que $x_1 < x < x_0$: \quad
      %
      \begin{pspicture}(0,0)(4.5,1)
        \psaxes[arrows=->,ticks=none,yAxis=false,labels=none]%
          (0,0)(0,0.5)(4.5,0.5)%
        \pstGeonode[PointSymbol=|,PosAngle=90]%
          (0.5,0){P_1}(2.5,0){P}(3.5,0){P_0}%
        \uput[-90](P_1){$x_1$}%
        \uput[-90](P_0){$x_0$}%
        \uput[-90](P){$x$}%
      \end{pspicture}
      %

      \vspace{0.6\baselineskip}%
      Vemos que las distancias entre $P$ y $P_0$ y entre $P_0$ y $P_1$ son
      \[
        PP_0 = x_0 - x \yjc P_0P_1 = x_0 - x_1.
      \]
      Además, como
      \[
        0 < P_0P < P_0P_1,
      \]
      existe $t \in\ ]0,1[$ tal que
      \[
        P_0P = tP_0P_1 = t(x_0 - x_1).
      \]
      Por lo tanto:
      \[
        x = x_0 - P_0P = x_0 - t(x_0 - x_1) = tx_1 + (1 - t)x_0.
      \]
\end{enumerate}

\subsection{Segmento que une dos puntos}

El conjunto de puntos de la recta real situados entre $P_{0}$ y $P_{1}$, incluidos estos, es
representado por $[P_{0},P_{1}]$; es decir:
\begin{equation}
\label{eq:algeom003}
[P_{0},P_{1}]= \{P_{t} : t\in [0,1] \},
\end{equation}
donde $x_{t}$, la coordenada de $P_{t}$, se calcula con la fórmula
\begin{equation}
\label{eq:algeom004}
x_{t} = tx_{1}+(1-t)x_{0}.
\end{equation}
Esta fórmula es una notación ``feliz'', ya que, para $t = 0$ y $t = 1$, se obtienen $x_0$ y $x_1$,
respectivamente.

Las fórmulas~(\ref{eq:algeom003}) y~(\ref{eq:algeom004}) se generalizan a dimensiones superiores.
Así, por ejemplo, si $P_{0},\ P_{1}$ están en $\mathbb{E}^{3}$, y si $(x_{0},y_{0},z_{0})$ y
$(x_{1},y_{1},z_{1})$ son las coordenadas de los puntos $P_{0}$ y $P_{1}$, respectivamente, la
expresión~(\ref{eq:algeom003}) define el conjunto de puntos del segmento de recta de extremos
$P_{0}$ y $P_{1}$ y las coordenadas de $P_{t}$ son $(x_{t},y_{t},z_{t})$, donde:
\begin{equation}
\label{eq:algeom005}
x_{t}=tx_{1}+(1-t)x_{0}, \quad
y_{t}=ty_{1}+(1-t)y_{0}, \quad
z_{t}=tz_{1}+(1-t)z_{0}.
\end{equation}

Si definimos
\[
\overrightarrow{x_{t}} = \left(\begin{array}{c}
x_{t} \\
y_{t} \\
z_{t}
\end{array}\right)
\]
para cada $t\in [0,1]$, la expresión~(\ref{eq:algeom005}) se escribe de la siguiente manera:
\begin{equation}
\label{eq:algeom006}
\overrightarrow{x_{t}} = t\overrightarrow{x_{1}}+(1-t)\overrightarrow{x_{0}}.
\end{equation}
Visualicemos este resultado en $\mathbb{E}^{2}$.

\subsection{Ecuación de la recta que pasa por dos puntos dados en $\mathbb{E}^{2}$}

Sean $P_{0}(x_{0}, y_{0})$ y $P_{1}(x_{1},y_{1})$ dos puntos distintos del plano $\mathbb{E}^{2}$
con $x_{0}\neq x_{1}$. Si la ecuación de la recta que pasa por estos dos puntos es:
\begin{equation}
\label{eq:algeom007}
y = y_{0}+ \frac{y_{1}-y_{0}}{x_{1}-x_{0}} (x-x_{0})= l(x).
\end{equation}
Si $x$ está entre $x_{0}$ y $x_{1}$; es decir, si $x = x_t$ con
\[
x_{t}=tx_{1}+(1-t)x_{0}
\]
y $t\in (0,1)$ o, lo que es lo mismo, si
\begin{equation}
\label{eq:algeom008}
x_{t} = x_{0} + t(x_{1}-x_{0}),
\end{equation}
tendremos que, si
\begin{equation}
\label{eq:algeom009}
y_{t}=l(x_{t}),
\end{equation}
entonces:
\begin{equation}
\label{eq:algeom010}
y_{t} = y_{0}+ \frac{y_{1}-y_{0}}{x_{0}-x_{1}} (x_{0}+t(x_{1}-x_{0})-x_{0})=y_{0}+t(y_{1}-y_{0}).
\end{equation}

Cuando $x_{0}=x_{1}$, como $P_{0}\neq P_{1}$, tendremos que $y_{0}\neq y_{1}$, la recta que pasa
por $P_{0}$ y $P_{1}$ es vertical y se reduce al caso $\mathbb{E}^{1}$ con $x_{t}=x_{0}=x_{1}$, y
con $y_{t}=ty_{1}+(1-t)y_{0}$. Se tendrá entonces, que si $P_{t}$ es un punto del segmento
$[P_{0},P_{1}]$ y, si $(x_t,y_t)$ son sus coordenadas, se verificará la siguiente igualdad:

\begin{equation}
\label{eq:algeom011}
 \left(\begin{array}{c}
x_{t} \\
y_{t}
\end{array}\right)=
t\left(\begin{array}{c}
x_{1} \\
y_{1}
\end{array}\right)+
(1-t) \left(\begin{array}{c}
x_{0} \\
y_{0}
\end{array}\right)
\end{equation}
que es la expresión~(\ref{eq:algeom006}) para $\mathbb{E}^{2}$.

\subsection{Funciones convexas y cóncavas}

\begin{defical}[Funciones convexa y cóncava]
Sean $I\subset \mathbb{R}$ un intervalo y $f\colon I
\rightarrow \mathbb{R}$ una función definida en $I$. La función \emph{$f$ es convexa
(\emph{respectivamente} cóncava) en $I$} si para cualquier para $x_{0} \in I$, $x_{1}\in I$ tales
que $x_{0}<x_{1}$, el gráfico de $f$ correspondiente al intervalo $(x_{0},x_{1})$; es decir, el
conjunto de $\mathbb{R}^2$
\begin{equation}
\label{eq:algeom012}
\{(x,f(x)) : x\in (x_{0},x_{1})\}=\{(x_{t},f(x_{t})) :  x_{t}=tx_{1}+(1-t)x_{0},\ 0<t<1 \}
\end{equation}
queda por debajo (respectivamente por encima) del segmento de recta $[P_{0},P_{1}]$, con
$P_{0}(x_{0},y_{0})$ y  $P_{1}(x_{1},y_{1})$, donde $y_{0}=f(x_{0})$ y $y_{1}=f(x_{1})$.
\end{defical}

Teniendo en cuenta la igualdad~(\ref{eq:algeom010}), la ecuación de la recta que une los puntos
$P_{0}$ y $P_{1}$ es:
\begin{equation}
\label{eq:algeom013}
y = l(x)= y_{0}+ \frac{y_{1}-y_{0}}{x_{1}-x_{0}} (x-x_{0}),
\end{equation}
y si $x= x_{t} = tx_{1}+(1-t)x_{0}$, obtenemos que:
\begin{equation}
\label{eq:algeom014}
y = y_{t}=l(x_{t})= y_{0} + t(y_1 - y_0) = ty_{1}+(1-t)y_{0}.
\end{equation}
Entonces, $f$ es convexa (respectivamente cóncava) en el intervalo $I$ si para todo $x_{0}$,
$x_{1}$ en $I$ tales que $x_{0}<x_{1}$, se verifica que
\begin{equation}
\label{eq:algeom015}
f(x_{t})\leq l(x_{t}) \ (\text{respectivamente } f(x_{t})\geq l(x_{t}))
\end{equation}
para todo $t\in\ ]0,1[$. Es decir, $f$ es convexa (respectivamente cóncava) si
\begin{equation}
\label{eq:algeom016}
f(x_{t})\leq y_{t} \ (\text{respectivamente } f(x_{t})\geq y_{t})
\end{equation}
para todo $t\in\ ]0,1[$.

En resumen, hemos demostrado el siguiente teorema.

\begin{teocal}\label{teo:daCaracterizacionConvexidad}
Una función $\funcjc{f}{I}{\Rbb}$ es convexa (respectivamente cóncava) si para todo $t \in\ ]0,1[$
se verifica que
\[
f(x_t) \leq y_t \ (\text{respectivamente } f(x_t) \geq y_t),
\]
donde
\[
y_t = ty_1 + (1 - t)y_0,
\]
con $y_0 = f(x_0)$ y $y_1 = f(x_1)$ y $x_0 < x_1$ elementos de $I$.
\end{teocal}

\begin{exemplo}[]{}
Sea $f$ una función definida por
\[
f(x) = ax^2 + bx + c.
\]
Mediante el teorema~\ref{teo:daCaracterizacionConvexidad}, probemos que $f$ es una función convexa
cuando $a > 0$ y es cóncava si $a < 0$.

\begin{enumerate}[leftmargin=*]
\item Supongamos que $a > 0$. Sean $x_0\in\Rbb$ y $x_1\in\Rbb$ tales que $x_0 < x_1$. Definamos
    $y_0 = f(x_0)$ y $y_1 = f(x_1)$. Sea $t\in\ ]0,1[$; entonces:
    \[
      x_t = (1 - t)x_0 + tx_1 \yjc y_t = (1 - t)y_0 + ty_1.
    \]
    Vamos a probar que $f(x_t) < y_t$. Por ello, calculemos $y_t$ y $f(x_t)$.

    En primer lugar, tenemos que:
    \begin{align*}
      y_t &= (1 - t)y_0 + ty_1 \\
          &= (1 - t)f(x_0) + tf(x_1) \\
          &= (1 - t)(ax_0^2 + bx_0 + c) + t(ax_1^2 + bx_1 + c) \\
          &= a[(1 - t)x_0^2 + tx_1^2] + b[(1-t)x_0 + tx_1] + c.
    \end{align*}
    En segundo lugar, tenemos que:
    \begin{align*}
    f(x_t) &= a[(1-t)x_0 + tx_1]^2 + b[(1-t)x_0 + tx_1] + c \\
           &= a[(1-t)^2x_0^2 + t^2x_1^2 + 2t(1-t)x_0x_1] + b[(1-t)x_0 + tx_1] + c.
    \end{align*}
    Por lo tanto:
    \begin{align*}
    y_t - f(x_t) &= a\left\{[(1-t) - (1-t)^2]x_0^2 - 2(1-t)tx_0x_1 + (t - t^2)x_1^2\right\} \\
                 &= a(1-t)t(x_0^2 - 2x_0x_1 + x_1^2) = a(1-t)t(x_0 - x_1)^2.
    \end{align*}
    Como $a > 0$, $1 - t > 0$, pues $0 < t < 1$ y $(x_0 - x_1)^2 > 0$, ya que $x_0 \neq x_1$,
    entonces
    \[
      y_t - f(x_t) \geq 0,
    \]
    de donde $f(x_t) \leq y_t$. Por lo tanto, por el
    teorema~\ref{teo:daCaracterizacionConvexidad}, $f$ es convexa.

\item Si $a < 0$, un procedimiento similar al anterior muestra que $y_t - f(x_t) < 0$.
\end{enumerate}
\end{exemplo}

Cuando una función es derivable, su convexidad o concavidad está relacionada con la monotonía de la
derivada. En el siguiente teorema, se muestra esta relación.

\begin{teocal}[Criterio de la derivada para la convexidad]\label{teo:daConvexidadDerivabilidad}
Sea $f\colon [a,b] \rightarrow \mathbb{R}$ continua en $[a,b]$ y derivable en $(a,b)$. Si $f'$ es
creciente (respectivamente decreciente) en $(a,b)$, entonces $f$ es convexa (respectivamente
cóncava) en $[a,b]$.
\end{teocal}

\begin{exemplo}[]{}
Sea $f(x) = x^4 + x^2 + 1$. Mediante la monotonía de $f'$, determinemos si $f$ es o no convexa.

En primer lugar, tenemos que $f'(x) = 4x^3 + 2x$. Sean $x_1$ y $x_2$ en $\Rbb$ tales que $x_1 <
x_2$. Entonces:
\begin{align*}
f'(x_2) - f'(x_1) &= 4x_2^3 - 2x_2 - 4x_1^3 + 2x_1 \\
                  &= 4(x_2^3 - x_1^3) + 2(x_2 - x_1).
\end{align*}
Como $x_2 > x_1$, entonces $x_2^3 - x_1^3 > 0$ (la función cúbica es creciente) y $x_2 - x_1 > 0$.
Por lo tanto
\[
f'(x_2) - f'(x_1) = 4(x_2^3 - x_1^3) + 2(x_2 - x_1) > 0.
\]
Entonces $f'(x_1) < f'(x_2)$; es decir, $f'$ es una función creciente. Entonces, por el
teorema~\ref{teo:daConvexidadDerivabilidad}, $f$ es una función convexa.
\end{exemplo}

Del teorema~\ref{teo:daConvexidadDerivabilidad} y por el teorema~(\ref{teo:daDerivadaPositiva}), es
inmediato el siguiente teorema.

\begin{teocal}\label{teo:daConvexidadSegundaDerivada}
Si $f$ es continua en $[a,b]$ y existe $f''$ y es positiva (respectivamente negativa) en $(a,b)$,
entonces $f$ es convexa (respectivamente cóncava) en $[a,b]$.
\end{teocal}

\begin{exemplo}[]{}
Sea $f(x) = 3x^4 - 4x^3 + 29x^2 + 2x - 7$. Probemos que esta función es convexa. Para ello,
mostremos que la segunda derivada de $f$ es siempre positiva. Así, por el
teorema~\ref{teo:daConvexidadSegundaDerivada}, $f$ debe ser convexa.

Calculemos $f''(x)$. En primer lugar, tenemos que
\[
  f'(x) = 12x^3 - 12x^2 + 58x + 2.
\]
En segundo lugar:
\[
  f''(x) = 36x^2 - 24x + 58.
\]
Ahora bien, el discriminante del polinomio de segundo grado $36x^2 - 24x + 58$ es negativo, pues
\[
(24)^2 - 4(36)(58) = -7\,776.
\]
Por lo tanto, para todo $x\in\Rbb$ $f''(x)$ tiene el mismo signo. Así, el signo de $f''(0)$ es el
signo de todos los $f''(x)$. Pues, como $f''(0) = 58 > 0$, tenemos que $f''(x) > 0$ para todo
$x\in\Rbb$. Es decir, $f$ es convexa.
\end{exemplo}

\section{Puntos de inflexión}
En el dibujo:
\begin{center}
\begin{pspicture}[showgrid=false](0,-1)(4.5,4)
\psset{plotpoints=200}
\psaxes[ticks=none,labels=none]{->}(0,0)(-0.5,-0.5)(4.3,4)[$x$,-90][$y$,180]
\pscurve[xunit=1.2cm](0.1,-0.7)(0.8,0.8)(2,1.7)(2.5,3)(3,3.5)
\psline(-0.5,0.6)(4,2)
\psset{linestyle=dashed,xunit=1.2}
\psline(1.4,0)(1.4,1.3)(0,1.3)
\psline(2.2,0)(2.2,2.1)
\rput[br](-0.1,-0.3){\footnotesize{$O$}}
\rput[b](1.4,-0.2){\footnotesize{$x_0$}}
\rput[b](2.2,-0.2){\footnotesize{$x$}}
\rput[r](0,1.3){\footnotesize{$f(x_0)$}}
\rput[U](1.4,1.5){\footnotesize{$P_0$}}
\rput[U](3,3.7){\footnotesize{$f$}}
\rput[U](3.5,2){\footnotesize{$g$}}
\end{pspicture}
\end{center}
$f$ es una función definida en un intervalo abierto $I$ y $x_0\in I$ tal que existe $f'(x_0)$. El
punto $P_0(x_0,f(x_0))$ tiene la propiedad de que la gráfica de $f$ pasa de un lado a otro de la
recta tangente a la gráfica de $f$ en $P_0$, cuya ecuación es
\[
y = g(x) = f(x_0) + f'(x_0)(x - x_0).
\]
El punto $P_0$ es denominado \emph{punto de inflexión} de la gráfica de $f$.

He aquí la definición precisa.

\begin{defical}[Punto de inflexión]
Sean $f\colon [a,b]\rightarrow\mathbb{R}$ y $x_{0}\in (a,b)$. Si $f$ es convexa en $[a,x_{0}]$ y
cóncava en $[x_{0},b]$, o viceversa (cóncava en $[a,x_{0}]$ y convexa en $[x_{0},b]$), se dice que
el punto de coordenadas $(x_{0}, f(x_{0}))$ es un \emph{punto de inflexión} del gráfico de $f$.
\end{defical}

\subsection*{Observaciones}
\begin{enumerate}
\item Si ponemos $h(x)=f(x)-g(x)$, $P_0$ es punto de inflexión de la gráfica de $f$ si y solo
    si $h(x)$ cambia de signo en $x_0$, es decir, si y solo si el punto de coordenadas
    $(x_0,0)$ es punto de inflexión de la gráfica de $h$.

\item Si $f\colon [a,b]\rightarrow\mathbb{R}$ es derivable en $(a,b)$ y si $f''$ cambia de signo
    en $x_{0}\in (a,b)$, entonces el punto de coordenadas $(x_{0}, f(x_{0}))$ es un punto de
    inflexión del gráfico de $f$.

\item Sea $f\colon [a,b]\rightarrow\mathbb{R}$. Si para $x_{0}\in (a,b)$, el punto de
    coordenadas $(x_{0}, f(x_{0}))$ es un punto de inflexión del gráfico de $f$, y si $f$ es
    derivable en $x_{0}$, existirá $r>0$ tal que el gráfico de $f$; es decir, de los puntos de
    coordenadas $(x,f(x))$, estarán de un lado de la recta tangente en el punto $(x_{0},
    f(x_{0}))$ para $x\in (x_{0}-r,x_{0})$ y del otro lado para $x\in (x_0,x_0+r)$.
\end{enumerate}

La siguiente es una condición necesaria para la existencia de un punto de inflexión.

\begin{teocal}
Si existe $f''(x_0)$ y si $P_0(x_0,f(x_0))$ es un punto de inflexión de la gráfica de $f$, entonces
$f''(x_0)=0$.
\end{teocal}

Esta condición no es suficiente; es decir, el recíproco de este teorema no es cierto. Por ejemplo,
si $y=f(x)=x^4$, el punto $P(0,0)$ no es punto de inflexión de la gráfica de $f$ a pesar de que
$f''(0)=0$.

\begin{exemplo}[Solución]{%
Analizar la función $f$ y su gráfico si $f\colon\mathbb{R}\to \mathbb{R}$ está definida por
\[
f(x) = x^3-px
\]
con $p\in \mathbb{R}$, una constante dada. A la gráfica de $f$ se le llama parábola cúbica.
}%
En primer lugar, $f$ es una función impar, pues:
\begin{equation*}
    f(-x)=(- x)^3-p(-x)=-(x^3-px)=-f(x).
\end{equation*}
Por lo tanto, la gráfica de $f$ es simétrica respecto al punto de coordenadas $(0,0)$. Esto
significa que bastará con analizar $f$ solo en uno de los dos intervalos: $[0,+\infty[$ o
$]-\infty,0]$, porque en el otro intervalo el gráfico se obtiene del primero por medio de un
reflexión y una rotación, y los valores $f(x)$ de la identidad $f(x) = -f(-x)$.

Ahora, busquemos los cortes de la gráfica de $f$ con el eje $x$; es decir, busquemos los $x$ tales
que $f(x)=0$:
\begin{equation}
\label{eq:da001}
   x^3-px=0
\end{equation}
La solución de esta ecuación depende del signo de $p$. Por lo tanto, hay tres casos que analizar.

\begin{enumerate}
\item  Si $p<0$, $x=0$ es la única raíz real de~(\ref{eq:da001}); las otras dos son complejas.
\item  Si $p=0$, $x=0$ es una raíz triple de la ecuación.
\item  Si $p>0$, $x_0=0$, $x_1=-\sqrt{p}$ y $x_2=\sqrt{p}$ son las tres raíces reales de la
    ecuación~(\ref{eq:da001}).
\end{enumerate}

Ahora calculemos la derivada de $f$:
\begin{equation}
\label{eq:da002}
    f'(x)= 3x^2-p.
\end{equation}
Estudiemos el signo de $f'(x)$, para lo cual es conveniente buscar los valores de $x$ para los
cuales $f'(x)=0$, en los tres casos ya considerados.

\begin{enumerate}
\item Si $p<0$, $f'(x)>0$ para todo $x\in \mathbb{R}$. Entonces $f$ será estrictamente
    creciente en $\mathbb{R}$.

\item Si $p=0$, $f'(x)>0$  para todo $x\in \mathbb{R} - \{0\}$; además, $f'(0)=0$. Entonces $f$
    es estrictamente creciente en $(-\infty,0)$ y en $(0,+\infty)$.

    Como en este caso $f(x)<0$ si $x<0$ y $f(x)>0$ si $x>0$, y $f(0) = 0$, se concluye que
    también $f$ es estrictamente creciente en $\mathbb{R}$.

\item Si $p>0$, la ecuación~(\ref{eq:da002}) tiene dos raíces:

   \[
      \bar{x}_1=-\sqrt{\frac{p}{3}} \yjc \bar{x}_2=\sqrt{\frac{p}{3}}.
   \]
    Se tiene que $f'(x)>0$ en $(-\infty,\bar{x}_1)$ y en $(\bar{x}_2,+\infty)$, mientras que
    $f'(x)<0$ en $(\bar{x}_1,\bar{x}_2)$, por lo que $f$ es creciente en los dos primeros
    intervalos y decreciente en el tercer intervalo.

    Adicionalmente, esto implica que, en $(\bar{x}_1,f(\bar{x_1}))$, la función $f$ tiene un
    máximo local y, en $(\bar{x}_2,f(\bar{x_2}))$, un mínimo local.
\end{enumerate}

Ahora estudiemos los puntos de inflexión de la gráfica de $f$. Como $f'(0)=-p$, la recta $g$ de
ecuación
\[
y=g(x)=-px
\]
es tangente a la gráfica de $f$ en $(0,0)$. La recta $g$ divide al plano en 3 partes: $g_+$ el
semiplano superior, la propia recta $g$ y $g_-$ el semiplano inferior:
\begin{gather*}
g_+=\{(x,y)\in \mathbb{R}\ |\ y>g(x)=-px\}\\
g=\{(x,y)\in \mathbb{R}\ |\ y=g(x)=-px\}\\
g_-=\{(x,y)\in \mathbb{R}\ |\ y<g(x)=-px\}.
\end{gather*}

\begin{center}
\psset{unit=0.65}
\begin{pspicture}[showgrid=false](-3,-2.5)(3,3.5)
\psaxes[ticks=none,labels=none]{->}(0,0)(-3,-2)(3,2.5)[$x$,-90][$y$,180]
\pstGeonode[PointSymbol=none,PointName={g,none},PointNameSep=0.5em,PosAngle=45]
            (2.5,1.3){A}(-2.5,-1.3){B}
\pstLineAB{A}{B}
\rput[B](1,1.2){$g_+$}
\rput[B](2,0.4){$g_-$}
\rput[bl](0,-0.4){\footnotesize{$O$}}
\end{pspicture}
\end{center}

Para ver la ubicación de los puntos $(x,f(x))$ de la gráfica de $f$ respecto a la recta
$y=g(x)=-px$, basta analizar el signo de
\begin{equation*}
    h(x)= f(x)-g(x)= (x^3-px)-(-px)=x^3.
\end{equation*}
El signo de $h(x)$ coincide con el de $x$, por lo que la gráfica de $f$ estará en el semiplano
$g_+$ para $x>0$ y en el semiplano $g_-$ para $x<0$.

El punto  $(0,0)$ es pues un punto de inflexión de la gráfica de $f$.

Como $f$ es dos veces derivable para todo $x$ y
\begin{equation*}
    f''(x)=6x,
\end{equation*}
tenemos que
\begin{equation*}.
    f''(x)=0 \quad \Leftrightarrow \quad x=0
\end{equation*}
Por lo tanto, como $(0,0)$ es un punto de inflexión, éste será el único (de haber otros para $x\neq
0$, se tendría que $f''(x)=0$; es decir, $6x =0$, lo cual es imposible).

Los resultados que sirven para analizar la función se pueden resumir en una tabla donde se los
anota a medida que se los obtiene y, finalmente, sirve para tener una idea completa del gráfico de
la función que luego, con la ayuda de un número adecuado de puntos de la forma $(a, f(a))$, podrá
ser realizado. En el ejemplo, para el tercer caso, $p>0$, tenemos:
\begin{center}
\setlength\extrarowheight{5pt}
\begin{tabular}{ c | c | c | c | c | c | c}

      % after \\: \hline or \cline{col1-col2} \cline{col3-col4} ...
      $x$   &   $\left (-\infty,-\sqrt{p}\right )$   &   $-\sqrt{p}$   &
      $\left (-\sqrt{p},-\frac{\sqrt{p}}{3}\right )$   &   $-\frac{\sqrt{p}}{3}$   &
      $\left (-\frac{\sqrt{p}}{3},0\right )$   &   $0$ \\
      \hline
     $f(x)$ &  $-$  &  0  &  +  &   + &  +  &    0    \\
      \hline
    $ f'(x) $&  +  &  +  &  +  &  0  & $-$   &  $-$    \\
      \hline
    Resultados &    &    &    & Máx. local   &    &   Punto infl.     \\ \hline

       $x$ &$0$   &   $\left (0,\frac{\sqrt{p}}{3}\right )$   &   $\frac{\sqrt{p}}{3}$   &   %
      $\left(\frac{\sqrt{p}}{3},\sqrt{p}\right )$   &   $\sqrt{p}$   &   $\left (\sqrt{p},+\infty\right )$ \\
      \hline
     $f(x)$ & 0   & $-$   & $-$   & $-$   &   0   &  + \\
      \hline
    $ f'(x) $&  $-$  & $-$   &  0  & +   &   +   &  +  \\
      \hline
    Resultados & Punto infl.   &    &  Mín local  &    &     &  \\ \hline
\end{tabular}
\end{center}

\begin{center}
\psset{unit=0.55}
\begin{pspicture}[showgrid=false](-3,-3.8)(3,3.5)
\psaxes[labels=none,ticks=none]{->}(0,0)(-3,-3.3)(3,3.5)[$x$,-90][$y$,180]
\psplot[yunit=0.2]{-2.3}{2.3}{x 3 exp x add}
\rput[bl](0.1,-0.4){\footnotesize{$O$}}
\rput[b](0,-3.8){$p<0$}
\end{pspicture}
\hspace{0.5cm}
\begin{pspicture}[showgrid=false](-3,-3.8)(3,3.5)
\psaxes[labels=none,ticks=none]{->}(0,0)(-3,-3.3)(3,3.5)[$x$,-90][$y$,180]
\psplot[yunit=0.2]{-2.3}{2.3}{x 3 exp}
\rput[bl](0.1,-0.4){\footnotesize{$O$}}
\rput[b](0,-3.8){$p=0$}
\end{pspicture}
\hspace{0.5cm}
\def\fun{x 3 exp x 5 mul sub}
\begin{pspicture}[showgrid=false](-3,-3.8)(3,3.5)
\psaxes[labels=none,ticks=none]{->}(0,0)(-3,-3.3)(3,3.5)[$x$,-90][$y$,180]
\psplot[yunit=0.2]{-2.5}{2.5}{\fun}
\pstGeonode%
[PointSymbol=none,PointName={none,none,\overline{x_1},\overline{x_2}},PosAngle={0,0,90,-90},PointNameSep=0.5em,yunit=0.2]%
          (! /x 1.29099 def x \fun){A}(! /x 1.29099 neg def x \fun){B}
          (1.29099,0){C}(-1.29099,0){D}
\pstLineAB[linestyle=dashed]{A}{C}
\pstLineAB[linestyle=dashed]{B}{D}
\rput[bl](2.23606,-0.4){\footnotesize{$x_1$}}
\rput[br](-2.23606,0){\footnotesize{$x_2$}}
\rput[br](0.1,-0.4){\footnotesize{$O$}}
\rput[b](0,-3.8){$p>0$}
\end{pspicture}
\end{center}

?`Cómo se corta la gráfica de $f$ con rectas horizontales de ecuación $y=c$? Para responder a esta
pregunta en los tres casos considerados, podemos analizar las soluciones de la ecuación $f(x)=0$:
\begin{equation}
\label{eq:da003}
    x^3-px=c.
\end{equation}

\begin{enumerate}
\item Si $p<0$, la ecuación~(\ref{eq:da003}) tiene una raíz real simple para todo $c$.

\item Si $p=0$, la ecuación tiene una raíz real simple para todo $c\neq 0$. Para $c=0$, $x=0$
    es una raíz real triple.

\item Si $p>0$, hay, a su vez, cuatro casos:

    \begin{enumerate}
    \item Si $c<f(\bar{x_2})$ o $c>f(\bar{x_1})$, la ecuación~(\ref{eq:da003}) tiene una
        sola raíz real simple.

    \item Si $f(\bar{x_2})<c<f(\bar{x_1})$, la ecuación tiene tres  raíces reales
        distintas.

    \item Si $c=f(\bar{x_2})$, $\bar{x_2}$ es raíz doble de la ecuación~(\ref{eq:da003}), y
        se tiene otra raíz simple y negativa.

    \item Si $c=f(\bar{x_1})$, $\bar{x_1}$ es raíz doble de la ecuación, y se tiene otra
        raíz simple y negativa.
    \end{enumerate}

\end{enumerate}
\end{exemplo}

\begin{multicols}{2}[\section{Ejercicios}]
\begingroup
\small
\begin{enumerate}[leftmargin=*]
\item Utilice la segunda derivada de $f$ para determinar los intervalos en los cuales $f$ es
    cóncava y en los cuales es convexa, así como los puntos de inflexión.
    \begin{enumerate}[leftmargin=*]
    \item $\displaystyle f(x) = 14x - x^2$.
    \item $\displaystyle f(x) = 2x^3 - 15x^2 + 24x$.
    \item $\displaystyle f(x) = (x + 2)^3 + 1$.
    \item $\displaystyle f(x) = (x + 1)(x - 1)^3$.
    \item $\displaystyle f(x) = 6x^4 + 2x^3 - 12x^2 + 4$.
    \item $\displaystyle f(x) = \sqrt[3]{x} + x$.
    \item $\displaystyle f(x) = x + \frac{16}{x}$.
    \item $\displaystyle f(x) = \sqrt{x^2 + 26}$.
    \item $\displaystyle f(x) = \frac{1}{x^2 + 1}$.
    \item $\displaystyle f(x) = \frac{x - 3}{x + 2}$.
    \item $\displaystyle f(x) = x^{\frac{7}{3}} + x$.
    \item $\displaystyle f(x) = \sen x$.
    \item $\displaystyle f(x) = x - \cos x + 1$.
    \item $\displaystyle f(x) = \tan x$.
    \end{enumerate}

\item Para la función $f$ dada, utilice la segunda derivada para hallar los extremos locales y
    los puntos de inflexión. Dibuje, aproximadamente, el gráfico de $f$.
    \begin{enumerate}[leftmargin=*]
    \item $\displaystyle f(x) = (2 - 3x)^2$.
    \item $\displaystyle f(x) = \frac{1}{3}x^3 - 2x^2 + 3x$.
    \item $\displaystyle f(x) = 5x^4 + x^2$.
    \item $\displaystyle f(x) = \frac{x^4 + 4}{x^2}$.
    \item $\displaystyle f(x) = x^{\frac{1}{3}}(x + 2)$.
    \item $\displaystyle f(x) = x\sqrt{x + 1}$.
    \item $\displaystyle f(x) = \sqrt{x} - \frac{x}{9}$.
    \item $\displaystyle f(x) = 3\sen x + 4\cos x$.
    \item $\displaystyle f(x) = x^4 - 6x^3 + 12x^2$.
    \item $\displaystyle f(x) = \sen(3x)$.
    \end{enumerate}

\item Dibuje, aproximadamente, el gráfico de la función $f$ que tenga las propiedades dadas.
    \begin{enumerate}[leftmargin=*]
    \item $f'(-1) = f'(2) = 0$; $f''(x) < 0$ para todo $x < 1$ y $f'(x) > 0$ para todo $x
        > 1$; y $f(0) = 1$.

    \item $f'(-2) = f'(-1) = f'(1) = f'(3) = 0$; $f''(x) > 0$ para todo $x < -\frac{3}{2}$
        o $0 < x < 2$ y $f''(x) < 0$ para todo $-\frac{3}{2} < x < 0$ o $x > 2$; y $f(0) =
        0$.
    \end{enumerate}

\item Halle los puntos de inflexión de $f$ si
    \[
      f(x) =
      \begin{cases}
      x^4 & \text{si } x \geq 0, \\
      -x^4 & \text{si } x < 0.
      \end{cases}
    \]

\item Dibuje, aproximadamente, el gráfico de $f$.
    \begin{enumerate}[leftmargin=*]
    \item $\displaystyle f(x) =
        \begin{cases}
          |x - 1| & \text{si } x \leq 1, \\
          2x^2 - 5x + 3 & \text{si } x > 1.
        \end{cases}$
    \item $\displaystyle f(x) =
        \begin{cases}
          -x^2 + 1 & \text{si } x^2 - 3x + 2 \geq 0, \\
          -3(x - 1) & \text{si no}.
        \end{cases}$
    \end{enumerate}

%\item Pruebe con un cambio de variable
%\begin{equation*}
%    x=\alpha x_1, \quad y=\beta y_1
%\end{equation*}
% (cambio de escala), que los tres casos de parábola cúbica se pueden reducir a
%\begin{equation*}
%    y_1=x_1^3-x_1, \quad y_1=x_1^3, \quad y_1=x_1^3+x_1.
%\end{equation*}
%\item Analice la función $f$ dada por
%\begin{equation}
%\label{eq:da004}
%    f(x)= a_0+a_1x+a_2x^2+x^3,
%\end{equation}
%con $a_0,a_1,a_2$ en $\mathbb{R}$. Halle los intervalos donde $f$ es monótona. Luego halle
%$\alpha, \beta$ en $\mathbb{R}$ tales que con un traslado paralelo dado por el cambio de
%variables
%\begin{equation*}
%    x=x_1+\alpha, \quad y=y_1+\beta
%\end{equation*}
%la ecuación (\ref{eq:da004}) sea equivalente a
%\begin{equation*}
%    y_1=x_1^3-x_1.
%\end{equation*}
%
%Lo puede hacer mediante cálculos directos o teniendo en cuenta que $(\alpha, \beta)$ es el
%punto de inflexión de la gráfica de $f$. Exprese entonces $p$ en función de $a_0,a_1\text{ y
%}a_2$
%\begin{equation*}
%   p = p(a_0,a_1,a_2).
%\end{equation*}
%Del análisis hecho anteriormente para la parábola cúbica, deduzca en qué casos (para qué
%relaciones entre $a_0,a_1\text{ y }a_2$) dados por $p<0$,  $p=0$ y  $p>0$, se tendrán para $f$
%una raíz real simple, una raíz real triple o tres raíces reales distintas.
%\item Analice la función  $f$ si
%\begin{equation*}
%    f(x)=b_0+b_1x+b_2x^2+b_3x^3+x^4,
%\end{equation*}
%usando para $f'$ los resultados obtenidos en el ejercicio anterior. Saque conclusiones respecto
%al número posible de máximos y mínimos en función de los coeficientes $b_0$, $b_1$, $b_2$ y
%$b_3$, y halle los puntos de inflexión en función de $b_0$, $b_1$, $b_2$ y $b_3$.
\end{enumerate}
\endgroup
\end{multicols}

\section{Graficación de funciones}
Los resultados estudiados hasta ahora, a más de proveernos de la herramienta necesaria para
resolver los problemas de optimización (encontrar el máximo o el mínimo de una función), se
utilizan para trazar el gráfico de funciones de manera aproximada, como ya se ha propuesto en los
ejercicios de la sección anterior.

En esta sección, vamos a establecer un procedimiento bastante general para dibujar el gráfico de
una función. Empecemos con un ejemplo.

\begin{exemplo}[Solución]{%
Dibujar el gráfico de la función $f$ definida por $y=f(x)=
\frac{1}{10}x^{3}-\frac{3}{10}x^{2}-\frac{8}{10}x$.
}%
La primera y la segunda derivadas de $f$ son:
\begin{eqnarray*}
  f'(x) &=& \frac{3}{10}x^{2}-\frac{3}{5}x-\frac{9}{10}, \\
  f''(x) &=& \frac{3}{5}x-\frac{3}{5}.
\end{eqnarray*}

Busquemos los puntos en los cuales se anulan $f$, $f'$ y $f''$; es decir, sus raíces. Para
\begin{equation*}
f(x_{k})=0,
\end{equation*}
tenemos que las raíces son
\[
x_{1}=-\frac{3}{2}(\sqrt{5}-1)\approx -1.85,\  x_{2}=0 \text{ y }
x_{3}=\frac{3}{2}(\sqrt{5}+1)\approx 4.85.
\]
Para
\begin{equation*}
	f'(x_{k})=0,
\end{equation*}
las raíces son:
\[
x_{4}=-1 \yjc  x_{5}=3.
\]
Finalmente, para
\begin{equation*}
	f''(x_{k})=0,
\end{equation*}
obtenemos una sola raíz:
\[
x_{6}=1.
\]

En $x_{1},x_{2}$ y $x_{3}$ el gráfico de $f$ interseca con el eje horizontal. En $x_{4}$ y $x_{5}$
es posible que $f$ tenga extremos locales,  en $x_{5}$, un punto de inflexión. La siguiente tabla
muestra los signos de  $f$, $f'$ y $f''$ en los intervalos comprendidos entre las raíces arriba
encontradas:

\[
\setlength\extrarowheight{4pt}
\begin{array}{c|c|c|c|c|c|c|c|c|c|c|c|c|c}
x & \multicolumn{1}{r|}{\hspace*{2em} -2} & -1.85 & & -1 & & 0 & & 1 &
\multicolumn{1}{r|}{2} & 3 & \multicolumn{1}{r|}{4} & \multicolumn{1}{r|}{4.85} &
\multicolumn{1}{l}{5} \\ \hline
f(x) & - & 0 & + & & + & 0 & - & & - & & - & 0 & + \\ \hline
f'(x) & + & & & 0 & - & - & & & - & 0 & + & & + \\ \hline
f''(x) & - & & - & & - & & - & 0 & + & & + & & +
\end{array}
\]

De la información contenida en esta tabla, colegimos que:
\begin{enumerate}
\item La función $f$ alcanza un máximo local en $-1$, ya que antes de este punto, la derivada
    es positiva y luego, negativa.
\item La función $f$ alcanza un mínimo local en $3$, ya que antes de este punto, la derivada es
    negativa y luego, positiva.
\item En el punto $1$, hay un punto de inflexión, ya que antes de este punto, la segunda
    derivada es negativa y luego, positiva.
\end{enumerate}

Dado que
\[
\setlength\extrarowheight{4pt}
\begin{array}{c|cccccc}
x & -1.85 & -1 & 0 & 1 & 3 & 4.85 \\ \hline
f(x) & 0 & 0.4 & 0 & -1 & -2.4 & 0
\end{array}
\]
podemos concluir que la gráfica de $f$:
\begin{enumerate}
\item corta el eje horizontal en $-1.85$, $0$ y $4.85$;
\item alcanza un máximo local igual a $0.4$ en $-1$;
\item alcanza un mínimo local igual a $-2.4$ en $3$;
\item es creciente en $]-\infty,-1]$ y $[3,+\infty[$;
\item es decreciente en $[-1,3]$;
\item es cóncava en $]-\infty, 0]$; y
\item es convexa en $[0,+\infty[$.
\end{enumerate}

Con esta información, podemos realizar el dibujo del gráfico de $f$ y obtener algo similar al
siguiente dibujo:
\begin{center}
\psset{algebraic}%
\def\psvlabel#1{\footnotesize $#1$}
\def\pshlabel#1{\footnotesize $#1$}
\begin{pspicture}(-4.5,-3.25)(6,2.5)
  \psplot[plotpoints=500]{-3}{5}{0.1*x^3-(3/10)*x^2-(8/10)*x}
  \psaxes[Dx=1,Dy=2,arrows=->](0,0)(-4.5,-3.25)(6,2.5)
  \psset{dotscale=0.8}
  \psdot[dotstyle=Bar,dotangle=-90](0,0.4)\uput[0](0,0.4){\footnotesize$0.4$}%
  \psdot(-1,0.4)\uput[90](-1,0.4){\footnotesize$(-1,0.4)$}%
  \psdot(-1.85,0)\uput[-45](-1.85,0){\footnotesize$x_1$}%
  \psdot[dotstyle=Bar,dotangle=-90](0,-2.4)\uput[0](0,-2.4){\footnotesize$-2.4$}%
  \psdot(3,-2.4)\uput[-90](3,-2.4){\footnotesize$(3,-2.4)$}%
  \psdot(4.85,0)\uput[-90](4.85,0){\footnotesize$x_2$}%
\end{pspicture}
\end{center}
\end{exemplo}


\begin{exemplo}[Solución]{Sea $f$ una función real definida por:
\begin{equation*}
	f(x)=
\begin{cases}
-2x-3 & \text{si $x\leq -1$}\\
x^3 & \text{si $|x|<1$}\\
\frac{1}{x^2}& \text{si $x\geq 1$}.
\end{cases}
\end{equation*}
Analice y dibuje el gráfico de $f$.
}%
La función $f$ está definida en $\mathbb{R}$. Hay cambio de definición en los puntos $x=-1$ y
$x=1$.

La función $f$ es continua en $\mathbb{R} - \{-1,1\}$ pues las partes en las cuales está definida
son dos polinomios y una función racional, que son continuos en sus dominios. Averigüemos si es
continua en $x=-1$ y $x=1$.

En primer lugar, $f$ está definida en dichos puntos: $f(-1)=-1$ y $f(1)=1$. Veamos ahora si existen
los límites en esos puntos. Calculamos primero los correspondientes límites laterales:
\begin{align*}
\lim_{x\to -1^-}f(x)&=\lim_{x\to -1^-}(-2x-3)= -1 & \lim_{x\to -1^+}f(x)&=\lim_{x\to -1^-}(-x^3)= -1\\
\lim_{x\to 1^-}f(x)&=\lim_{x\to -1^-}x^3= 1 & \lim_{x\to 1^+}f(x)&=\lim_{x\to -1^-}\frac{1}{x^2}= 1.
\end{align*}
Como los correspondientes límites laterales son iguales, concluimos la verdad de las siguientes
igualdades:
\begin{gather*}
\lim_{x\to -1}f(x)=-1,\\
 \lim_{x\to1}f(x)=1.
\end{gather*}
Finalmente, podemos decir que la función es continua en $x=-1$ y $x=1$ porque
\begin{gather*}
\lim_{x\to -1}f(x)=-1= f(-1),\\
 \lim_{x\to1}f(x)=1=f(1).
\end{gather*}

En resumen, la función $f$ es continua en su dominio $\mathbb{R}$.

Como no existen puntos de discontinuidad no hay posibilidad de que existan asíntotas verticales.
Veamos si existen asíntotas horizontales. Para ello, calculemos los límites al infinito:
\begin{gather*}
\lim_{x\to -\infty} f(x) =\lim_{x\to -\infty} (-2x-3) = -\infty,\\
\lim_{x\to +\infty} f(x) =\lim_{x\to +\infty} \frac{1}{x^2} = 0.
\end{gather*}
Entonces $y=0$ es una asíntota horizontal de la gráfica de $f$ cuando $x$ tiende a $+\infty$.

Para continuar con el análisis de la función $f$, necesitaremos su primera y su segunda derivada:
\begin{equation*}
	f'(x)=
\begin{cases}
-2 & \text{si $x< -1$}\\
3x^2 & \text{si $|x|<1$}\\
-\frac{2}{x^3}& \text{si $x> 1$}.
\end{cases}
\end{equation*}

\begin{equation*}
	f''(x)=
\begin{cases}
0 & \text{si $x< -1$}\\
6x & \text{si $|x|<1$}\\
\frac{6}{x^4}& \text{si $x>1$}.
\end{cases}
\end{equation*}
Veamos si existe la primera derivada de $f$ en $x=-1$ y $x=1$. Como en esos puntos la función
cambia de definición, hallemos, primeramente, las derivadas laterales correspondientes:
\begin{align*}
f_-'(-1) & =\lim_{x\to -1^-}\frac{f(x)-f(-1)}{x-(-1)}= \lim_{x\to -1^-}\frac{(-2x -3) - (-1)} {x-(-1)}\\
&=\lim_{x\to -1^-}\frac{-2(x+1)}{x+1} =\lim_{x\to -1^-}(-2)=-2,
\end{align*}

\begin{align*}
f_+'(-1) & =\lim_{x\to -1^+}\frac{f(x)-f(-1)}{x-(-1)} =\lim_{x\to -1^+}\frac{x^3 -(- 1)}{x-(-1)} \\
& =\lim_{x\to -1^+}(x^2-x+1)) = 3,
\end{align*}

\begin{equation*}
f_-'(-1) = -2\neq 3=f_+'(-1).
\end{equation*}
Entonces no existe $f'(-1)$.

\begin{align*}
f_-'(1) & =\lim_{x\to 1^-}\frac{f(x)-f(1)}{x-1}= \lim_{x\to 1^-}\frac{x^3-1}{x-1}\\
& = \lim_{x\to 1^-}(x^2+x+1)= 3,
\end{align*}

\begin{align*}
f_+'(1) & =\lim_{x\to 1^+}\frac{f(x)-f(1)}{x-1} = \lim_{x\to 1^+}\frac{\frac{1}{x^2}-1}{x-1}\\
& = \lim_{x\to 1^+}\left(-\frac{x+1}{x^2}\right) = -2,
\end{align*}
\begin{equation*}
f_-'(1) = 3\neq -2=f_+'(1).
\end{equation*}
Entonces no existe $f'(1)$.

Como en los puntos $x=-1$ y  $x=1$ no existe la derivada, la gráfica de $f$ no tiene tangente en
esos puntos. Además, existe la posibilidad de que en ellos existan extremos locales. Claramente,
$f'(0)=0$; entonces hay la posibilidad de que en $x=0$ exista un extremo local.

Como el $D(f')= \mathbb{R} - \{-1,1\}$, no existe la segunda derivada  en $x=-1$ y  $x=1$ y, como
en dichos puntos no existe recta tangente, no puede haber en ellos puntos de inflexión de la
gráfica de $f$.

Ya solo falta averiguar en qué puntos la segunda derivada es igual a cero; ello sucede en $x=0$. En
$x=0$ podría haber un punto de inflexión.

En resumen, en $x=-1$, $x=0$ y $x=1$ buscaremos extremos locales. En $x=0$ además buscaremos un
punto de inflexión. Esta búsqueda la realizamos con ayuda de la siguiente tabla:

\begin{center}
\begin{tabular}{|c|c|c|c|c|c|c|c|}
  \hline
  % after \\: \hline or \cline{col1-col2} \cline{col3-col4} ...
  $x$    & $]-\infty,-1[$ & $-1$ & $]-1,0[$ & $0$ & $]0,+1[$ & $+1$ & $]+1,+ \infty[$ \\
  \hline
  $f(x)$ & $$                 & $-1$ & $$           & $0$ & $$            & $1$ & $$ \\
  \hline
  $f'(x)$ & $-$ & n. e. & $+$ & $0$ & $+$ & n. e. & $-$ \\
  \hline
  Monotonía & $\searrow$ & $$ & $\nearrow$ & $$ & $\nearrow$ & $$ & $\searrow$ \\
  \hline
  $f''(x)$ & $0$ & n. e. & $-$ & $0$ & $+$ & n. e & $+$ \\
  \hline
  Concavidad & $$ & $$ & $\frown$ & $$ & $\smile$ & $$ & $\smile$ \\
  \hline
  Resultados & $$ & m. l. & $$ & p. i. & $$ & M. l. & $$ \\
   \hline
\end{tabular}
\end{center}

\begin{center}
\begin{tabular}{|c|}
\hline
n. e. = no existe; m. l. = mínimo local; M. l. = máximo local; p. i. = punto de inflexión \\
\hline
\end{tabular}
\end{center}

A partir de la tabla podemos dibujar la forma de la gráfica de la función $f$ como se muestra en la
figura.

\begin{center}
   \psset{xAxisLabel={},yAxisLabel={}}
   \begin{psgraph}[ticks=none](0,0)(-2.1,-1.25)(3,1.5){0.75\textwidth}{5.5cm}
      \psplot[algebraic,plotpoints=1000]%
         {-2}{3}{IfTE(x<-1,-2*x-3,IfTE(x<1,x^3,1/x^2))}%
      \psaxes{->}%
         (0,0)(-2.1,-1.25)(3.25,1.5)%
      \uput[-90](3.25,0){$x$}%
      \uput[0](0,1.5){$y$}%
   \end{psgraph}
\end{center}

\end{exemplo}

De los dos ejemplos, podemos proponer el siguiente procedimiento, bastante general, para analizar
el gráfico de una función real.
\begin{enumerate}[leftmargin=*]
\item Si la función es par o impar, basta considerar el intervalo $[0,+\infty[$.
\item Se identifican los ``valores especiales'' para $f$, que son sus raíces. También sus
    puntos de discontinuidad. Digamos que estos son $x_1 < x_2 < \cdots < x_N$.

\item En cada uno de los intervalos $]-\infty, x_1[$, $]x_k, x_{k+1}[$ y $]x_N, +\infty[$ la
    función $f$ no cambia de signo. Para determinar el signo en cada intervalo, basta calcular
    $f(x)$ para un $x$ en el intervalo correspondiente.

\item El paso anterior se aplica también a $f'$ y a $f''$.

\item Se elabora una tabla con los puntos especiales de $f$, $f'$ y $f''$.
\end{enumerate}
Con la tabla obtenida, podemos determinar los extremos locales, puntos de inflexión, intervalos de
monotonía, concavidad y convexidad.

\begin{multicols}{2}[\section{Ejercicios}]
\begingroup
\small
\begin{enumerate}[leftmargin=*]
\item Dibuje, aproximadamente, el gráfico de la función $f$.
    \begin{enumerate}
    \item $\displaystyle f(x) = x^3 + 2x$.
    \item $\displaystyle f(x) = (x + 2)^3(x - 1)^2$.
    \item $\displaystyle f(x) = x^3 - 6x^2 + 9x + 5$.
    \item $\displaystyle f(x) = x + \frac{1}{x^2}$.
    \item $\displaystyle f(x) = \frac{x - 1}{(x + 2)(x - 3)}$.
    \item $\displaystyle f(x) = \frac{6x + 1}{3x - 1}$.
    \item $\displaystyle f(x) = -x + \cos x$.
    \item $\displaystyle f(x) = \sen^3 x + \cos^3 x$.
    \item $\displaystyle f(x) = x^2\ln x$.
    \item $\displaystyle f(x) = x^2e^x$.
    \item $\displaystyle f(x) = 
    \begin{cases} 
    |x^2-5x+4| & \text{si $x\geq 0$,}\\
    4-x^2 & \text{si $x< 0$.}
    \end{cases}$
    \end{enumerate}
\end{enumerate}
\endgroup
\end{multicols}

\section{Problemas de extremos}

Una de las principales aplicaciones de las derivadas consiste en proveer de un método para resolver
problemas de optimización; es decir, problemas en los que hay que escoger de entre varias opciones
para la solución la mejor o la más conveniente, o la menos inconveniente. Por ejemplo, en una
empresa se busca que las pérdidas sean mínimas y que las utilidades sean máximas; un transportista
busca la ruta más corta o la más rápida; un bodeguero trata de guardar más objetos en cada bodega,
etcétera. El problema de Romeo y Julieta con el que iniciamos este capítulo, es un ejemplo de un
problema de optimización.

El modelo matemático de una gran variedad de problemas de optimización se ajustan al siguiente
problema matemático, denominado, \textbf{problema de optimización}:

\begin{probcal}[Básico de optimización]
Dados $I$ un intervalo y $\funcjc{f}{I}{\Rbb}$ continua en $I$, se debe determinar si existen el
máximo y el mínimo de $f$ en $I$. De existir, se debe hallar $x_m\in I$ y $x_M\in I$ tales que
\[
\underline{y} = f(x_m) = \min_{x\in I}f(x) \yjc \overline{y} = f(x_M) = \max_{x\in I}f(x).
\]
\end{probcal}

A continuación vamos a enunciar los resultados desarrollados en las secciones anteriores y que nos
permitirán establecer un ``algoritmo'' para el problema de optimización.

\begin{teocal}
Sean $I$ un intervalo y $\overline{x} \in I^\circ$ tal que $f$ alcanza un extremo local en
$\overline{x}$. Entonces $\overline{x}$ es un punto crítico de $f$; es decir, $f'(\overline{x}) =
0$ o no existe la derivada de $f$ en $\overline{x}$.
\end{teocal}

Al combinar estos dos resultados, tenemos el siguiente algoritmo para el problema de optmización
cuando $I = [a,b]$ y $f$ tiene un número finito de puntos críticos en el interior de $I$.

\begin{algocal}
Sea $\funcjc{f}{[a,b]}{\Rbb}$ una función continua en $[a,b]$.
\begin{enumerate}
\item Hallar $K_0 = \{x \in \ ]a,b[ \ : f'(x) = 0\}$ y $K_1 = \{x \in \ ]a,b[ \ : \not\exists
    f'(x)\}$. Si $K_0$ y $K_1$ son finitos, vaya al siguiente paso.

\item Sea $K = \{a,b\} \cup K_0 \cup K_1$. Este es el conjunto de los ``candidatos'' a ser los
    puntos en los que se alcancen los extremos globales. Obligatoriamente $x_m$ y $x_M$, cuya
    existencia está garantizada por el primer teorema, son elementos de $K$.

    Escribamos $K = \{a_1 = a < a_2 < a_3 < \cdots < a_N = b\}$.

\item Calcular la tabla de valores $b_k = f(a_k)$ con $1 \leq k \leq N$, ordenando los valores
    de manera que $b_{k_1} \leq b_{k_2} \leq \cdots \leq b_{k_N}$.

\item Entonces $x_m = a_{k_1}$, $x_M = a_{k_N}$, $\underline{y} = b_{k_1}$ y $\overline{y} = b
    _{k_N}$.
\end{enumerate}
\end{algocal}

Los siguientes teoremas, en cambio, nos permiten establecer un algoritmo cuando el intervalo $I$ no
es cerrado y acotado.

\begin{teocal}
Sean $I$ un intervalo y $\funcjc{f}{I}{\mathbb{R}}$ una función continua. Si existe un único extremo local en $I$, entonces este extremo es también es global.
\end{teocal}

\begin{teocal}
Sean $I$ un intervalo y $\funcjc{f}{I}{\Rbb}$ una función continua en $I$. Entonces la imagen de
$f$, el conjunto $\Img(f)$ es un intervalo. En particular, si $I=[a,b]$, entonces $\Img(f) =
[f(x_m),f(x_M)]$.
\end{teocal}

\begin{teocal}
Si $\displaystyle I = \bigcup_{k=1}^n A_n$ y $\funcjc{f}{I}{\Rbb}$, entonces:
\[
\Img(f) = \bigcup_{k=1}^n f(A_k).
\]
\end{teocal}

Y, finalmente:

\begin{teocal}
Sean $I = ]a,b[$ y $\funcjc{f}{I}{\Rbb}$ una función continua en $I$. Entonces:
\begin{enumerate}
\item Si $f$ es creciente en $I$, entonces $\displaystyle \Img(f) =
    \left]\limjc{f(x)}{x}{a^+},\limjc{f(x)}{x}{b^-}\right[$.

\item Si $f$ es decreciente en $I$, entonces $\displaystyle \Img(f) =
    \left]\limjc{f(x)}{x}{b^-},\limjc{f(x)}{x}{a^+}\right[$.

\item Resultados análogos a los anteriores son verdaderos si $I$ es uno de los siguientes
    intervalos: $]-\infty,b[$, $]a,+\infty[$ y $]-\infty,\infty[$.
\end{enumerate}
\end{teocal}

Al combinar los resultados precedentes, se tiene el siguiente algoritmo para resolver el problema
de optimización cuando $I$ no es un intervalo cerrado y acotado, pero sí es la unión de $3$ subintervalos de $I$, digamos $I_1$, $I_2$ e $I_3$, de modo que en $I_1$ y en $I_3$ $f$ es monótona y que $I_2=[a,b]$, con $a$, $b\in I$. El
algoritmo permite, además, hallar la imagen de $f$.

\begin{algocal}
Sean $I$ un intervalo que no es cerrado ni acotado y $\funcjc{f}{I}{\Rbb}$ una función continua en $I$.

\begin{enumerate}
\item Determinar $a$, $b\in I$ y los subintervalos $I_1$ e $I_3$ de $I$ en los cuales $f$ es
    monótona y que $I=I_1\cup [a,b]\cup I_3$.

\item Calcular $J_k = f(I_k)$ para todo $k \in \{1,2,3\}$ (Para $J_2$ se usa el Algoritmo $1$, mientras que para $J_1$ y $J_3$ se aplica el Teorema precedente).

\item Calcular $\displaystyle J = \Img(f) = \bigcup_{k=1}^3 J_k$.

\item El conjunto $J$ es un intervalo. Si $J$ es uno de los intervalos:
    \begin{enumerate}
    \item $\Rbb$, $]-\infty,\beta[$ o $]-\infty,\beta]$, entonces $\displaystyle \inf_{x\in
        I}f(x) = -\infty$ y $f$ no alcanza el mínimo en $I$.

    \item $]\alpha,\beta[$ o $]\alpha,\beta]$, entonces $\displaystyle \inf_{x\in I}f(x) =
        \alpha$ y $f$ no alcanza el mínimo en $I$.

    \item $[\alpha,\beta[$, $[\alpha,\beta]$ o $[\alpha,+\infty [$, entonces $\displaystyle
        \min_{x\in I}f(x) = \alpha$.

    \item $\Rbb$, $[\alpha,\infty[$ o $]\alpha,\infty[$, entonces $\displaystyle \sup_{x\in
        I}f(x) = +\infty$ y $f$ no alcanza el máximo en $I$.

    \item $]-\infty,\beta[$ o $]\alpha,\beta[$ o $[\alpha,\beta[$, entonces $\displaystyle
        \sup_{x\in I}f(x) = \beta$ y $f$ no alcanza el máximo en $I$.

    \item $]\alpha,\beta]$, $[\alpha,\beta]$ o $]-\infty,\beta]$, entonces $\displaystyle
        \max_{x\in I}f(x) = \beta$.
    \end{enumerate}
\end{enumerate}
\end{algocal}

Este algoritmo sirve también si el dominio de la función se puede expresar como la unión finita de
intervalos, pues se lo puede aplicar a cada uno de esos intervalos y luego se calcula la imagen de
$f$ con lo que, si existen, se determinan los extremos de la función.

Ahora resolvamos el problema de Romeo y Julieta.

Empecemos calculando la derivada de $f$ para todo $x\in I$ ya que $f$ es derivable en su dominio:
\begin{align*}
f'(x) &= \frac{2x}{2\sqrt{x^2 + 4}} + \frac{-2(3 - x)}{2\sqrt{}(3 - x)^2 + 1} \\
      &= \frac{x}{\sqrt{x^2 + 4}} - \frac{3 - x}{\sqrt{}(3 - x)^2 + 1}.
\end{align*}
Por lo tanto:
\begin{align*}
f'(x) = 0 &\Leftrightarrow \frac{x}{\sqrt{x^2 + 4}} - \frac{3 - x}{\sqrt{}(3 - x)^2 + 1} \\
    &\Leftrightarrow \frac{x}{\sqrt{x^2 + 4}} = \frac{3 - x}{\sqrt{}(3 - x)^2 + 1}.
\end{align*}

Pero:
\begin{align*}
\frac{x}{\sqrt{x^2 + 4}} = \frac{3 - x}{\sqrt{}(3 - x)^2 + 1} &\Rightarrow
x^2\big[(3 - x)^2 + 1\big] = (3 - x)^2(x^2 + 4) \\
&\Rightarrow 3x^2 - 24x + 36 = 0 \\
&\Rightarrow x \in\{2,6\}.
\end{align*}
Es decir, si $f'(x) = 0$, necesariamente $x = 2$ ó $x = 6$.

Dado que $6\not\in I$ y
\begin{align*}
f'(2) &= \frac{2}{\sqrt{2^2 + 4}} - \frac{3 - 2}{\sqrt{(3 - 2)^2 + 1}} \\
    &= \frac{2}{2\sqrt{2}} - \frac{1}{2\sqrt{2}} = 0,
\end{align*}
tenemos que $2$ es un punto crítico de $f$ en $I$. Además, es el único punto crítico, por que si
hubiera otro, diferente de $2$, debería ser el número $6$, lo que no es posible.

Ahora calculemos la segunda derivada de $f$ en $2$ para saber si, en este número, la función $f$
alcanza un máximo o un mínimo local.

Para ello, luego de un cálculo sencillo, vemos que, para cualquier $x\in I$, tenemos que:
\[
f''(x) = \frac{4}{(x^2 + 4)^{\frac{3}{2}}} +
\frac{1}{\big[(3 - x)^2 + 1\big]^{\frac{3}{2}}}.
\]

Entonces, $f''(x) > 0$ para todo $x\in I$. En particular, $f''(2) > 0$.  Esto significa que $f(2)$
es un mínimo local de $f$ en $I$. Por lo tanto, el mínimo de $f$ es:
\[
f(2) = \min_{x\in I}f(x) = 3\sqrt{2} \approx 4.24.
\]

\subsubsection{Solución del problema original (interpretación de la solución del problema matemático)}
Romeo irá donde Julieta, llevándole rosas recogidas a $2$ kilómetros del extremo $A$ del lago.
Cuando el joven enamorado llegue, habrá remado aproximadamente un recorrido de $4$ kilómetros con
$240$ metros aproximadamente. A su edad, y con el deseo enorme de encontrarse con su amada, ese
recorrido no representará para él ningún problema.

\subsubsection{Epílogo}
Se cuenta que Romeo llegó con un bello ramo de rosas donde Julieta cuando ella estaba a punto de
retirarse de su escondite creyendo que su amado no vendría. Las flores la llenaron de alegría y
justificaron plenamente su impaciente espera.

Para terminar, examinemos algunos ejemplos adicionales de optimización.

\begin{exemplo}[Solución]{%
Sea $f\colon ]0, \infty[\ \to \mathbb{R} $, definida por
\[
f(x) = x^2-\frac{8}{x}.
\]
Halle el punto en el cual la gráfica de $f$ tiene la recta tangente de mínima pendiente.
}%
El dominio de $f$ es el conjunto $]0,+\infty[$.

La pendiente de la recta tangente $m_x$ en un punto de coordenada $x$ viene dada por la derivada de
$f(x)$. Entonces:
\begin{equation*}
    m_x = g(x) = f'(x) = 2x + \frac{8}{x^2}.
\end{equation*}
El dominio de $g$ es el conjunto $]0,+\infty[$.

El problema a resolver es hallar los extremos de la función $g$. Para ello necesitamos su derivada:
\begin{equation*}
    m_x' = g'(x) =2 - \frac{16}{x^3}= 2\frac{x^3-8}{x^3}.
\end{equation*}
La función $g$ no es derivable en $x=0$ pero $0\notin D(g)$.

Por otro lado, $g'(2)=0$. Posiblemente, existe extremo local en $x=2$. Para averiguarlo,
necesitamos la segunda derivada de $g$ (es decir, la tercera de $f$):
\begin{equation*}
    m_x'' = g''(x) = \frac{48}{x^4}.
\end{equation*}
Tenemos que $g''(2)= 3>0$.

Por el criterio de la segunda derivada, podemos concluir que, en $x=2$, la función $g$ tiene un
mínimo local. Debido a que es el único extremo en $D(g)$, es, a la vez, un mínimo absoluto y su
valor es:
\begin{equation*}
    m_x\Big |_{x=2}= g(2) = 2(2) + \frac{8}{2^2}= 6.
\end{equation*}

Así, pues, la pendiente será mínima e igual a 6 para la recta tangente a la gráfica de $f$ en el
punto de coordenadas $(2,0)$.

\begin{center}
   %\psset{xAxisLabel={},yAxisLabel={}}
   \begin{psgraph}[ticks=none](0,0)(-0.5,-30)(4,30){0.75\textwidth}{6cm}
      \psplot[algebraic,plotpoints=1000]%
         {0.27}{4}{x^2-8/x}%
      \psaxes[ticks=none,labels=x]{->}%
         (0,0)(-0.5,-30)(4.25,30.5)%
      \uput[-90](4.25,0){$x$}%
      \uput[0](0,30){$y$}%

      \psplotTangent[algebraic,linewidth=0.75\pslinewidth,linecolor=red,arrows=<->]%
         {2}{1.5}{x^2-8/x}%
      \psplotTangent[algebraic,linewidth=0.75\pslinewidth,linecolor=red,arrows=<->]%
         {1}{0.5}{x^2-8/x}%
      \psplotTangent[algebraic,linewidth=0.75\pslinewidth,linecolor=red,arrows=<->]%
         {0.5}{0.5}{x^2-8/x}%
   \end{psgraph}
\end{center}
\end{exemplo}

\begin{exemplo}[Solución]{%
Halle dos números positivos cuya suma sea 100 y cuyo producto sea máximo.
}%
Sea $x>0$ uno de dichos números. El otro será $100-x>0$ para que su suma sea 100. Tenemos entonces
que $0<x<100$.

Sea $P$ el producto de los dos números. Naturalmente dependerá de $x$:
\begin{equation*}
    P=f(x)=x(100-x)= 100 x-x^2.
\end{equation*}

Debemos, entonces, hallar $x_1\in ]0,100[$ tal que:
\begin{equation*}
  f(x_1)=\max_{x\in (0,100)}f(x).
\end{equation*}

Como $f$ es derivable en $]0,100[$, los puntos críticos de $f$ en ese intervalo serán solución de
\begin{equation*}
  f'(x)=100-2x=0
\end{equation*}
que equivale a $x=50$.

Como $f''(x)=-2<0$, en $x_1=50$, $f$ alcanza un máximo local que por ser el único extremo en el
intervalo de interés $]0,100[$ es, a la vez, un máximo absoluto.

El otro número es $100-x_1= 100-50=50$. Los dos números buscados, tales que su suma es igual a 100
y su producto  2500 es máximo, son iguales a 50.
\end{exemplo}

\begin{exemplo}[Solución]{%
Halle el punto de la parábola que es la gráfica de $y = 1-x^2$ más cercano al punto
$A(2.5,2.75)$. Puede usar el hecho de que
\begin{equation*}
    (4x^3+9x-5)\Big |_{x=0.5}=0.
\end{equation*}
}%
Si $P(x,y)$ es un punto arbitrario de la parábola de ecuación $y = 1-x^2$, tendremos que
$P(x,1-x^2)$.

La distancia entre los puntos $A$ y $P$ está dada por:
\begin{equation*}
    d = f(x) = \sqrt{(x-2.5)^2+[  (1-x^2)-2.75 ]^2}= \sqrt{(x-2.5)^2+ (x^2+1.75)^2}, \ x\in \mathbb{R}.
\end{equation*}

Como la función $h$ definida por $h(x) = x^2$ para todo $x > 0$ es continua y estrictamente
creciente, los extremos de $h\circ f$ coinciden con los extremos de $f$. Entonces, si la distancia
$d$ va a ser mínima, la distancia al cuadrado $d^2$ también será mínima. Sea
\begin{equation*}
    d^2 = g(x) = (x-2.5)^2+ (x^2+1.75)^2, \ x\in \mathbb{R}.
\end{equation*}

Tenemos que resolver el problema:
$\displaystyle
    \text{Hallar} \min_{x\in \mathbb{R}}g(x).
$
Necesitamos la derivada
\begin{equation*}
    g'(x) = 2(x-2.5)+ 4x(x^2+1.75)= 4x^3+9x-5
\end{equation*}
para hallar los puntos críticos de $g$. La ecuación $g'(x)=4x^3+9x-5=0$ puede escribirse como
\begin{equation*}
    4x^3+9x-5=(2x-1)(2x^2+x+5)= 0.
\end{equation*}
Puesto que el segundo factor es un trinomio de discriminante negativo, la única solución real de la
ecuación cúbica es $x = \frac{1}{2}$ la cual es, a la vez, número crítico de $g$.

Ahora, usemos el criterio de la segunda derivada para saber si existe o no extremo en $x =
\frac{1}{2}$. El signo del valor en $x = \frac{1}{2}$ de la segunda derivada  $g''(x)= 12x^2+9$,
que es
\begin{equation*}
  g''\left( \frac{1}{2}\right)= 12\left(\frac{1}{2}\right)^2+9 = 12>0,
\end{equation*}
nos dice que $g$ tiene un mínimo local en $x = \frac{1}{2}$. Como $x = \frac{1}{2}$ es el único
número crítico de $g$ en su dominio dicho extremo es un mínimo absoluto, de modo que
\begin{equation*}
    d^2\Big|_{x=\frac{1}{2}} = \min_{x\in \mathbb{R}}g(x),
\end{equation*}
y
\begin{equation*}
    d\Big|_{x=\frac{1}{2}} = \min_{x\in \mathbb{R}}f(x).
\end{equation*}

El punto buscado es
\begin{equation*}
   P(x,1-x^2)\Big|_{x=\frac{1}{2}}=  P\left(\frac{1}{2},\frac{3}{4}\right),
\end{equation*}
y la distancia de $A$ a $P$ es
\begin{equation*}
 d\Big|_{x=\frac{1}{2}} = f\left(\frac{1}{2}\right) =
 \sqrt{\left(\frac{1}{2}-2.5\right)^2 + \left( \frac{1}{2}+1.75\right)^2}=
 \sqrt{2^2+2^2} = 4\sqrt{2}.
\end{equation*}
\end{exemplo}


%\subsection{Ejemplo}
%\ejemplo{Las residencias de Romeo y Julieta, $R$ y $J$ en el gráfico, están separadas por una
%laguna rectangular de 2 km de ancho. Romeo desea ir en su barca a ver a su amada pero llevándole
%rosas del rosal que se encuentra al extremo del lago. ?`En qué lugar $P$ del rosal tomará las rosas
%para llegar lo más pronto posible donde su amada?
%\begin{center}
%%\includegraphics[scale=0.25]{Derivadas/RomeoJulieta.eps}
%\begin{pspicture}(6,4)
%   \psframe(0.5,0.5)(6,3)%
%
%   \pstGeonode[PosAngle={90,180,-90},PointSymbol=none,PointNameSep={1em,0.5em,1em}]%
%      (2.5,3){R}(0.5,2){P}(1.5,0.5){J}%
%
%   \psline{|<->|}(0.5,3.1)(2.5,3.1)%
%   \uput[90](1.5,3.25){$2$ km}%
%
%   \psline{|<->|}(0.5,0.4)(1.5,0.4)%
%   \uput[-90](1,0.4){$1$ km}%
%
%   \psline{|<->|}(6.1,0.6)(6.1,3)%
%   \uput[0](6.1,1.5){$3$ km}%
%
%   \psline(R)(P)(J)%
%
%   \rput(4,2.5){Lago}%
%   \rput[t](0,1){\pstVerb{
%/vshowdict 4 dict def
%/vshow
%{ vshowdict begin
%/thestring exch def
%/lineskip exch def
%thestring
%{
%/charcode exch def
%/thechar ( ) dup 0 charcode put def
%0 lineskip neg rmoveto
%gsave
%thechar stringwidth pop 2 div neg 0 rmoveto
%thechar show
%grestore
%} forall
%end
%} def
%   64 (lasoR) vshow }}
%
%\end{pspicture}
%\end{center}}{%
%Sea $x=|AP|$ la distancia de $A$ a $P$ en km, $0\leq x\leq3$. Entonces $3-x=|PB|$ la distancia de
%$P$ a $B$ en km. Como $|AR|=2$ y $|BJ|=1$ y utilizando el teorema de Pitágoras obtenemos
%\begin{gather*}
%|RP|= \sqrt{x^2+4},\\
%|PJ|=  \sqrt{(3-x^2)+1} .
%\end{gather*}
%
%Sean $t(RP), t(PJ), \text{ y } t$ el tiempo en horas que demora Romeo en ir en su barca de $R$ a
%$P$,  de $P$ a $J$, y de  $R$ a $J$ pasando por $P$, respectivamente. Suponemos que la barca de
%Romeo avanza con velocidad constante $v$ en km/h. Estos tiempos en términos de las distancias y la
%velocidad son:
%\begin{gather*}
%t(RP)  = \frac{|RP|}{v} =  \frac{\sqrt{x^2+4}}{v}, \\
%t(PJ)  = \frac{|PJ|}{v} =  \frac{\sqrt{(3-x)^2+1}}{v}, \\
%t  = t(RP)+t(PJ)= \frac{\sqrt{x^2+4}}{v} + \frac{\sqrt{(3-x)^2+1}}{v}.
%\end{gather*}
%
%El tiempo total que dura el recorrido de Romeo en su barca es una función de $x$:
%\begin{equation*}
%   t = f(x) = \frac{1}{v}\left( \sqrt{x^2+4} +   \sqrt{(3-x)^2+1} \right), \ x\in [0,3]
%\end{equation*}
%
%Tenemos entonces que resolver el problema
%\begin{equation*}
%    \text{Hallar }\ \min_{x\in [0,3]}f(x).
%\end{equation*}
%
%Procedemos a hallar los puntos críticos de $f$ para lo cual necesitamos la derivada
%\begin{equation*}
%    T'=f'(x)= \frac{1}{v}\left(\frac{x}{\sqrt{x^2+4}} -\frac{3-x}{\sqrt{(x-3)^2+1}}\right).
%\end{equation*}
%La función $f$ es derivable en el intervalo de interés $[0,3]$. Buscamos los puntos críticos
%igualando a cero la derivada:
%\begin{align*}
%f'(x)= 0 & \quad \Leftrightarrow \quad \frac{x}{\sqrt{x^2+4}} = \frac{x-3}{\sqrt{(3-x)^2+1}}\\
% & \quad \Leftrightarrow \quad   x^2[(3-x)^2+1] = (3-x)^2(x^2+4)        \\
%  & \quad \Leftrightarrow \quad  3x^2-24x+36=0,
%\end{align*}
% y como la última ecuación tiene como soluciones $x=x_1=2\in[0,3]$ y $x=x_2=6\notin[0,3]$, el único punto crítico en $[0,3]$ es 2.
%
% Como la función $f$ es continua en el intervalo cerrado $[0,3]$ podemos aplicar el teorema del valor extremo que nos asegura que en dicho intervalo existe 1 mínimo absoluto y 1 máximo absoluto. Dichos extremos pueden estar también en los extremos del intervalo $[0,3]$; para encontrarlos hacemos la siguiente tabla:
% \begin{center}
% \begin{tabular}{c|c}
%    % after \\: \hline or \cline{col1-col2} \cline{col3-col4} ...
%   $x$ & $t=f(x)$ \\
%     \hline
%   0 & $\dfrac{2+\sqrt{10}}{v}\approx \dfrac{5.1}{v}$ \\1
%   2 & $ \dfrac{3\sqrt{2}}{v}\approx \dfrac{4.2}{v}$\\
%  3& $\dfrac{\sqrt{13}+1}{v}\approx \dfrac{4.6}{v}$ \\
% \end{tabular}
%\end{center}
%
%Vemos que $t\Big|_{x=2}= \dfrac{3 \sqrt{2}}{v}$ es el valor mínimo.
%
%Romeo irá entonces donde Julieta llevándole rosas recogidas en el lugar $P$ situado a 2 km del
%extremo $A$ de la laguna y de esa forma estará junto a su amada en el menor tiempo posible,
%$\dfrac{3 \sqrt{2}}{v}\approx \dfrac{4.2}{v}$ h.
%}%Fin de \ejemplo

\begin{multicols}{2}[\section{Ejercicios}]
\begingroup
\small
\begin{enumerate}[leftmargin=*]
\item Resuelva el problema básico de optimización para las siguientes funciones $f$ definidas
    en el intervalo $I$. Adicionalmente, determine el conjunto $\Img(f)$.
    \begin{enumerate}[leftmargin=*]
    \item $\displaystyle f(x) = 2x^3 - 15x^2 + 24x$, \ $I = [0,5]$.
    \item $\displaystyle f(x) = 2x^3 - 15x^2 + 24x$, \ $I = [0,5[$.
    \item $\displaystyle f(x) = 2x^3 - 15x^2 + 24x$, \ $I = ]0,5[$.
    \item $\displaystyle f(x) = 2x^3 - 15x^2 + 24x$, \ $I = ]0,5]$.
    \item $\displaystyle f(x) = 2x^3 - 15x^2 + 24x$, \ $I = [0,3]$.
    \item $\displaystyle f(x) = \frac{x - 3}{x - 2}$, \ $I = [0,+\infty[$.
    \item $\displaystyle f(x) = \frac{1}{\sqrt{x^2 + 25}}$, \ $I = \Rbb$.
    \item $\displaystyle f(x) = 5x^4 + 2x^2 - 7$, \ $I = [-1,1]$.
    \item $\displaystyle f(x) = x^4 + 6x^2 - 7$, \ $I = ]-3,1]$.
    \item $\displaystyle f(x) = x^4 + 6x^2 - 7$, \ $I = [-4,1]$.
    \item $\displaystyle f(x) = 4\sen x + 3\cos x$, \ $I = \Rbb$.
    \item $\displaystyle f(x) = 4\sen x + 3\cos x$, \\ $\displaystyle I =
        \left]-\frac{\pi}{4},\frac{\pi}{2}\right]$.
    \item $\displaystyle f(x) = 4\sen x + 3\cos x$, \\ $\displaystyle I =
        \left[\arctan\frac{3}{4},\pi\right[$.
        \item $\displaystyle f(x) =
        \begin{cases}
          1 - x^2 & \text{si $x < 1$ o $x >2$}, \\
          3 - 3x & \text{si } 1 \leq x \leq 2,
        \end{cases}$ \ $I = [0,3]$.
        \item $\displaystyle f(x) =
        \begin{cases}
          4 - x^2 & \text{si } x \leq 2, \\
          \frac{1}{2}(x - 2) & \text{si } 2 < x,
        \end{cases}$ \ $I = [-1,3]$.
    \item $\displaystyle f(x) = \sqrt{x^2 + x + 1}$, \ $I = [-2,1[$.
    \item $\displaystyle f(x) = (x^2 - 2x + 10)^{\frac{5}{2}}$, \ $I = ]-7,5[$.
    \item $\displaystyle f(x) =
        \begin{cases}
          x - x^2 & \text{si } 0 \leq x \leq 1, \\
          \sen(\pi x) & \text{si no},
        \end{cases}$ \ $I = \Rbb$.
    \end{enumerate}

\item Halle dos números no negativos tales que:
    \begin{enumerate}[leftmargin=*]
    \item Su suma sea igual a $1$ y su producto sea máximo.
    \item Su producto sea igual a $1$ y su suma sea mínima.
    \item Su suma sea igual a $100$ y el producto del cuadrado del primero y el cubo del
        segundo sea máximo.
    \end{enumerate}

\item Halle el punto de $P$ de coordenadas $(a,b)$ de la parábola de ecuación $y = 1 - x^2$ más
    cercano al punto $Q$ de coordenadas $(\frac{5}{2},\frac{11}{4})$.

\item Dadas la recta $l$ de ecuación $y = 4 - x$ y la parábola $p$ de ecuación $y = 1 - x^2$,
    halle los puntos $P\in l$ y $Q \in p$ más cercanos entre sí.

\item Halle el punto del gráfico de $f$ donde la pendiente de la recta tangente sea mínima si
    $f(x) = x^3 - 6x^2$.

\item Se desea cerrar un terreno que es rectangular junto a un río con una pared paralela a la
    orilla del río que cuesta $25$ dólares el metro lineal y dos alambradas perpendiculares al
    río que cuestan $20$ dólares el metro lineal. Si el terreno debe tener diez mil metros
    cuadrados de área, ?`cuáles deben ser sus dimensiones para que los costos sean mínimos?

\item Se quiere construir una caja abierta de un volumen dado $V$ metros cúbicos, cuya base
    rectangular tenga el doble de largo que de ancho. ?`Cuáles deben ser sus dimensiones para
    que el costo de los materiales sea mínimo? ?`Y si la caja es cerrada?

\item Halle las dimensiones del cilindro circular recto de máximo volumen que puede inscribirse
    en un cono circular recto de diámetro $D$ metros y de altura $H$ metros.

\item Halle las dimensiones del cono circular recto de volumen máximo que puede inscribirse en
    una esfera de radio $R$ metros.

\item Se desea elaborar tarros de un litro para conservas de duraznos. ?`Qué dimensiones tendrá
    cada tarro para que el costo por el metal utilizado sea mínimo?

\item De un trozo cuadrado de cartulina se desea elaborar una pirámide de base cuadrada
    desechando la parte rayada y doblando en las líneas punteadas como se muestra en el dibujo:
    \begin{center}
    \begin{pspicture}(0,0)(4,4)
      \psset{PointSymbol=none,PointName=none}
      \pstGeonode[]%
        (2,0){A}(4,2){B}(2,4){C}(0,2){D}%
      \pstGeonode[]%
        (1.5,1.5){E}(2.5,1.5){F}(2.5,2.5){G}(1.5,2.5){H}%

      \psset{fillstyle=vlines,hatchcolor=gray}
      \pspolygon[]%
        (A)(B)(F)%
      \pspolygon[hatchangle=-45]%
        (B)(C)(G)%
      \pspolygon[]%
        (C)(D)(H)%
      \pspolygon[hatchangle=-45]%
        (D)(A)(E)%
      \pspolygon[fillstyle=none,linestyle=dashed]%
        (E)(F)(G)(H)

    \end{pspicture}
    \end{center}
    ?`Qué dimensiones tendrá la pirámide de volumen máximo si la cartulina tiene un metro de
    lado?

\item Dos calles de diez metros de ancho se intersecan. Se desea transportar horizontalmente
    una varilla larga y delgada, pero, en la intersección, se requiere virar a la derecha.
    ?`Cuál es el largo máximo de la varilla que se puede transportar si el grosor de la varilla
    es despreciable?

    \begin{center}
    \begin{pspicture}(0,0)(5,4.5)
      \psset{PointName=none,PointSymbol=none}%
      \pstGeonode[]%
        (1.5,0){A}(3.5,0){B}(5,1.5){C}(5,3.5){D}(3.5,4.5){E}(1.5,4.5){F}(0,3.5){G}(0,1.5){H}%
      \pstGeonode[]%
        (1.5,1.5){I}(3.5,1.5){J}(3.5,3.5){K}(1.5,3.5){L}(1.75,0.25){M}(3.75,3.25){N}%
      \psline[]%
        (A)(I)(H)%
      \psline[]%
        (B)(J)(C)%
      \psline[]%
        (D)(K)(E)%
      \psline[]%
        (F)(L)(G)%
      \psline[linewidth=4\pslinewidth]%
        (M)(N)%
    \end{pspicture}
    \end{center}

\item Halle las dimensiones del cono circular recto de volumen mínimo que circunscribe una
    esfera de radio $R$.

\item Un joven puede remar a razón de dos kilómetros por hora y correr a seis kilómetros por
    hora. Si está en el punto $A$ de la isla que está a dos kilómetros del punto C de la playa y desea llegar lo antes posible al punto $B$, situado en la playa de enfrente, que está a dos kilómetros del punto $C$ en la playa, ?`hasta
    qué punto $P$ de la playa debe remar para luego correr desde $P$ hasta $B$? ?`En qué tiempo
    llegará?
    \begin{center}
    \begin{pspicture}(-0.5,0)(6.5,4)
      \pstGeonode[PosAngle={90,135,45,45}]%
        (1,1){B}(4,1){P}(5,1){C}(5,3){A}%
      \pstGeonode[PointName=none,PointSymbol=none]%
        (0,1){X}(6,1){Y}(5,4){O}%

      \pstLineAB[]%
        {X}{Y}

      \psarc[linewidth=2\pslinewidth,linestyle=solid]%
          (O){1.25}{215}{325}

      \psset{linestyle=dashed}
        \pstLineAB[]%
          {A}{P}%
        \pstLineAB[]%
          {A}{C}%

      \psline[arrows=<->]%
        (! \psGetNodeCenter{B} B.x B.y 0.25 sub)(! \psGetNodeCenter{C} C.x C.y 0.25 sub)

      \pstMiddleAB[PointName=none,PointSymbol=none]%
        {B}{C}{N}%
      \uput[-90](! \psGetNodeCenter{N} N.x N.y 0.2 sub){$4\kilometros$}%

      \pstMiddleAB[PointName=none,PointSymbol=none]%
        {A}{C}{M}%
      \uput[0](! \psGetNodeCenter{M} M.x 0.1 add M.y){$2\kilometros$}%

      \rput[l](0,0.5){Playa}%
      \rput(! \psGetNodeCenter{A} A.x A.y 0.65 add){Isla}
    \end{pspicture}
    \end{center}
\end{enumerate}
\endgroup
\end{multicols}

\section{Regla de L'Hôpital}

\subsection{Formas indeterminadas}

Al tratar de utilizar las propiedades de los límites de la suma, producto, etcétera, se puede
llegar a situaciones en que los resultados no se pueden aplicar por no cumplirse las condiciones
que las garantizan. Y resulta que bajo condiciones similares, los resultados obtenidos no siempre
son iguales. Por ejemplo, el límite del cociente de dos funciones tales que cada una tiende a $0$,
puede ser, en algunos casos, igual a $0$; en otros, $+\infty$; en otros, no existir; etcétera. Se
dice, entonces, que estamos frente a una \emph{forma indeterminada}. Veamos algunos casos.

Sean $f$ y $g$ dos funciones definidas en un intervalo abierto $I$. Sea $x_{0}\in I$:

\begin{enumerate}[leftmargin=*]
\item Cuando se calculan límites de cocientes, como $\displaystyle{\lim_{x\rightarrow a}\frac{f(x)}{g(x)}}$,
    sabemos que si existen los límites
\begin{equation*}
	\lim_{x\rightarrow a}f(x)=L\quad \text{y}\quad \lim_{x\rightarrow a}g(x)=M
\end{equation*}
y si $M\neq 0$ se tiene que
\begin{equation*}
	\lim_{x\rightarrow a}\frac{f(x)}{g(x)}=\frac{\displaystyle{\lim_{x\rightarrow a}f(x)}}{\displaystyle{\lim_{x\rightarrow a}g(x)}}=\frac{L}{M}.
\end{equation*}
Puede suceder, sin embargo, que $L=M=0$, de modo que, al tratar de aplicar este resultado se
obtiene:
\begin{equation*}
	 \displaystyle{\frac{\displaystyle{\lim_{x\rightarrow a}f(x)}}{\displaystyle{\lim_{x\rightarrow a}g(x)}}=\frac{0}{0}.}
\end{equation*}
La expresión $\frac{0}{0}$  es un ejemplo de ``indeterminación''. En este caso no se puede
afirmar nada sobre la existencia o no del límite buscado. A veces existe, a veces no, como se
ve en los siguientes ejemplos.

Por ejemplo, si $f(x)=\sen x$ y $g(x)= x$, entonces
\begin{equation*}
	 \frac{\displaystyle{\lim_{x\rightarrow a}f(x)}}{\displaystyle{\lim_{x\rightarrow a}g(x)}}=1.
\end{equation*}
En cambio, si $f(x)=x\sen \frac{1}{x}$ y $g(x)= 1-\cos x$, el límite
\begin{equation*}
	 \frac{\displaystyle{\lim_{x\rightarrow a}f(x)}}{\displaystyle{\lim_{x\rightarrow a}g(x)}}
\end{equation*}
no existe.

\item Si $\displaystyle{\lim_{x\rightarrow x_{0}}f(x)}=+\infty$ (respectivamente $-\infty$) y si
    $\displaystyle{\lim_{x\rightarrow x_{0}}g(x)}=-\infty$ (respectivamente $+\infty$), las expresiones
    $(f+g)(x)$ y $\left(\frac{f}{g}\right)(x)$ pueden tener diferentes comportamientos posibles
    cuando $x\rightarrow x_{0}$. La primera es una indeterminación del tipo $+\infty -\infty$
    ($-\infty +\infty$, respectivamente), y la segunda, del tipo $\frac{\infty}{\infty}$.

\item Si $\displaystyle\lim_{x\rightarrow x_{0}}f(x)=0$ y si $\displaystyle\lim_{x\rightarrow x_{0}}g(x)=+\infty$ (o
    $-\infty$), la expresión $(fg)(x)$ puede comportarse de diferentes maneras cuando
    $x\rightarrow x_{0}$. Es una indeterminación del tipo $0\cdot \infty$ o $0\cdot (-\infty)$.

\end{enumerate}

Hemos visto algunos casos en los que se pueden hallar los límites buscados como aplicación del
teorema 1.1. Se dice entonces que ``hemos levantado la indeterminación'' Un método muy útil para
ello también es la regla de L'H\^{o}pital que veremos a continuación.


\subsection{Regla de L'H\^{o}pital}

\begin{teocal}[Regla de L'H\^{o}pital]
Sea $a\in \mathbb{R}$. Si $\displaystyle\lim_{x\rightarrow a}\frac{f(x)}{g(x)}$ es una indeterminación de tipo $\frac{0}{0}$ o $\frac{\infty}{\infty}$ y si
\begin{equation*}
	\lim_{x\rightarrow a}\frac{f'(x)}{g'(x)}=L\in \mathbb{R}\cup \{-\infty, \infty\},
\end{equation*}
entonces
\begin{equation*}
	\lim_{x\rightarrow a}\frac{f(x)}{g(x)}=\lim_{x\rightarrow a}\frac{f'(x)}{g'(x)}.
\end{equation*}
\end{teocal}

Se tiene un resultado similar si, en vez de que $x$ tienda al número $a$, lo haga a $a^+$, $a^-$,
$+\infty$ o $-\infty$.

\begin{exemplo}[Solución]{%
Calcular $\displaystyle\lim_{x\rightarrow 0}\frac{\sen x}{x}$.
}%
Como el límite es una indeterminación de tipo $\dfrac{\infty}{\infty}$, y como
\begin{equation*}
	\lim_{x\rightarrow 0}\frac{(\sen x)'}{(x)'}=\lim_{x\rightarrow 0}\frac{\cos x}{1} = 1
\end{equation*}
existe, entonces, por la regla de L'H\^{o}pital:
\begin{equation*}
	\lim_{x\rightarrow 0}\frac{\sen x}{x}=1.
\end{equation*}
\end{exemplo}

\begin{exemplo}[Solución]{
Calcular $\displaystyle\lim_{x\rightarrow \infty}\frac{\ln x}{x}$.
}%
Como el
\begin{equation*}
	\lim_{x\rightarrow \infty}\frac{(\ln x)'}{(x)'}=\lim_{x\rightarrow \infty}\frac{\frac{1}{x}}{1} = 0
\end{equation*}
existe, entonces, por la regla de L'H\^{o}pital:
\begin{equation*}
	\lim_{x\rightarrow \infty}\frac{\ln x}{x}=0.
\end{equation*}
\end{exemplo}

Con ciertas manipulaciones previas, el teorema es también útil para levantar otras
indeterminaciones como son
\begin{equation*}
	\infty -\infty,\ 0\cdot \infty,\  0^{0},\  \infty^{0}\ \text{ y }\ 1^{\infty}.
\end{equation*}
Veamos un ejemplo.

\begin{exemplo}[Solución]{%
Calcular $\displaystyle\lim_{x\rightarrow 0^{+}}\left( \frac{1}{\sqrt{x}} -\frac{1}{x} \right)$.
}%
Este límite conduce a la indeterminación $\infty -\infty$.

Pero como
\begin{equation*}
	\frac{1}{\sqrt{x}} -\frac{1}{x}=\frac{x-\sqrt{x}}{x\sqrt{x}},
\end{equation*}
el límite del segundo miembro de la expresión anterior conduce a una indeterminación del tipo
$\frac{0}{0}$, que puede ser tratada con la regla de L'H\^{o}pital.

El límite
\begin{equation*}
	\lim_{x\rightarrow 0^{+}}\frac{(x-\sqrt{x})'}{(x\sqrt{x})'}=\lim_{x\rightarrow 0^{+}}
  \frac{1-\frac{1}{2\sqrt{x}}}{\frac{3}{2}\sqrt{x}}=\lim_{x\rightarrow 0^{+}}
  \frac{2\sqrt{x}-1}{3x}=-\infty.
\end{equation*}
De donde, por la la regla de L'H\^{o}pital se tiene el límite pedido:
\begin{equation*}
\lim_{x\rightarrow 0^{+}}\left( \frac{1}{\sqrt{x}} -\frac{1}{x}\right)= -\infty.
\end{equation*}
\end{exemplo}

\begin{exemplo}[Solución]{%
Pruebe, utilizando la regla de L'Hôpital, que
\begin{equation*}
	\lim_{x \to 0}\frac{\sen x-x\cos x}{x^2+\tan^2x}=0.
\end{equation*}
}%
Sean $f(x)= \sen x-x\cos x$ y $g(x)=x^2+\tan^2x$. Se pide demostrar que
\begin{equation*}
	\lim_{x \to 0}\frac{f(x)}{g(x)}=0.
\end{equation*}


Aplicando propiedades de los límites se encuentra fácilmente que
\begin{gather*}
\lim_{x \to 0}f(x) = \lim_{x \to 0}(\sen x-x\cos x)=0,\\
\lim_{x \to 0}g(x) = \lim_{x \to 0}(x^2+\tan^2x)=0.
\end{gather*}
Como el límite de la función del denominador $\displaystyle\lim_{x \to 0}g(x)$ es igual a cero no podemos
aplicar el teorema del límite de un cociente de funciones. Pero, por otro lado, tenemos que la
función $\frac{f}{g}$ tiene la forma indeterminada $\frac{0}{0}$ en $x=0$, lo cual nos permitiría
aplicar la regla de L'Hôpital. Para ello debemos encontrar el límite
\begin{equation*}
	\lim_{x \to 0}\frac{f'(x)}{g'(x)}.
\end{equation*}
Si este límite existe la regla de L'Hôpital nos dice que el límite buscado es igual al anterior.
Veamos, pues, si existe dicho límite:
\begin{align*}
	\lim_{x \to 0}\frac{f'(x)}{g'(x)}&= \lim_{x \to 0}\frac{(\sen x-x\cos x)'}{(x^2+\tan^2x)'}\\
&=\lim_{x \to 0}\frac{\cos x-\cos x+x\sen x}{2x+2\tan x\sec^2x}\\
 &= \lim_{x \to 0}\frac{x\sen x}{2x+2\tan x\sec^2x}\\
 &{}=\lim_{x\to 0}\frac{\sen x}{2+2\frac{\sen x}{x}\sec^3 x}=\frac{0}{2+2(1)(1)}=0.
\end{align*}
Una inspección rápida del límite anterior nos dice que la función $\frac{f'}{g'}$ tiene también la
forma indeterminada $\frac{0}{0}$ en $x=0$. Intentamos nuevamente aplicar la regla de L'Hôpital:
\begin{align*}
	\lim_{x \to 0}\frac{(f'(x))'}{(g'(x))'}&= \lim_{x \to 0}\frac{f''(x)}{g''(x)}\\
&=  \lim_{x \to 0}\frac{(x\sen x)'}{(2x+2\tan x\sec^2x)'}\\
 &= \lim_{x \to 0}\frac{\sen x+x\cos x}{2+2\sec^4 x+2\tan^2 x \sec^2 x}\\
 &=0.
\end{align*}
Como el último límite existe podemos aplicar la regla de L'Hôpital:
\begin{equation*}
	\lim_{x \to 0}\frac{f'(x)}{g'(x)}=\lim_{x \to 0}\frac{(f'(x))'}{(g'(x))'}= 0.
\end{equation*}
Esto nos permite aplicar nuevamente la regla de L'Hôpital:
\begin{equation*}
	\lim_{x \to 0}\frac{f(x)}{g(x)}=\lim_{x \to 0}\frac{f'(x)}{g'(x)}= 0.
\end{equation*}
Con lo cual hemos probado que
\begin{equation*}
	\lim_{x \to 0}\frac{f(x)}{g(x)}=\lim_{x \to 0}\frac{\sen x-x\cos x}{x^2+\tan^2x}=0.
\end{equation*}
\end{exemplo}

\begin{multicols}{2}[\section{Ejercicios}]
\begingroup
\small
\begin{enumerate}[leftmargin=*]
\item Calcule los siguientes límites.
    \begin{enumerate}[leftmargin=*]
    \item $\displaystyle \limjc{\frac{1 - \cos x^2}{x^4}}{x}{0}$.
    \item $\displaystyle \limjc{\frac{1 - \cos x^2}{x^3\sin x}}{x}{0}$.
    \item $\displaystyle \limjc{\frac{a^x - x^a}{x - a}}{x}{a}$.
    \item $\displaystyle \limjc{x^\epsilon\ln x}{x}{0^+}$ donde $\epsilon > 0$.
    \item $\displaystyle \limjc{\frac{x}{\sqrt{1 + x^2}}}{x}{+\infty}$.
    \item $\displaystyle \limjc{\frac{\sen x}{\arctan x}}{x}{0}$.
    \item $\displaystyle \limjc{\frac{(2 - x)e^x - x - 2}{x^3}}{x}{0}$.
    \item $\displaystyle \limjc{\frac{\sen(1/x)}{\arctan(1/x)}}{x}{+\infty}$.
    \item $\displaystyle \limjc{\frac{1}{\sqrt{x}}\left(\frac{1}{\sen x} -
        \frac{1}{x}\right)}{x}{0^+}$.
    \item $\displaystyle \limjc{\frac{\tan x - 5}{\sec x + 4}}{x}{\frac{\pi}{2}}$.
    \end{enumerate}
\item Demuestre la regla de L'Hopital si $\displaystyle\lim_{x\to a}\frac{f(x)}{g(x)}$ es una indeterminación de tipo $\frac 00$, para el caso $L\in\Rbb$. \emph{Indicación: }Use el Teorema 4.10 (Teorema General del Valor Intermedio) para $\displaystyle\lim_{x\to a^-}\frac{f(x)}{g(x)}$ y $\displaystyle\lim_{x\to a^+}\frac{f(x)}{g(x)}$.
\end{enumerate}
\endgroup

\end{multicols}

%Como $x_{t}=(1-t)x_{0}+tx_{1}$, se puede obtener que
%\begin{equation}
%\label{eq:algeom027}
%t=\frac{x_{t}-x_{0}}{x_{1}-x_{0}}, \qquad 1-t=\frac{x_{1}-x_{t}}{x_{1}-x_{0}}
%\end{equation}
%Como f es convexa
%\begin{equation}
%\label{eq:algeom028}
%f(x_{t})\leq (1-t)f(x_{0}) + tf(x_{1})
%\end{equation}
%que equivale a
%\begin{equation}
%\label{eq:algeom029}
%f(x_{t})\leq \frac{x_{1}-x_{t}}{x_{1}-x_{0}}f(x_{0}) + \frac{x_{t}-x_{0}}{x_{1}-x_{0}}f(x_{1}).
%\end{equation}
%Se tienen las identidades
%\begin{equation}
%\label{eq:algeom030}
%\frac{x_{1}-x_{t}}{(x_{1}-x_{0})(x_{t}-x_{0})}=\frac{1}{x_{t}-x_{0}}-\frac{1}{x_{1}-x_{0}}
%\end{equation}
%y
%\begin{equation}
%\label{eq:algeom031}
%\frac{x_{t}-x_{0}}{(x_{1}-x_{0})(x_{1}-x_{t})}=\frac{1}{x_{1}-x_{t}}-\frac{1}{x_{1}-x_{0}}
%\end{equation}
%por lo que de \ref{eq:algeom029}, multiplicando previamente por $\frac{1}{x_{t}-x_{0}}$ y
%$\frac{1}{x_{1}-x_{t}}$, respectivamente, se obtienen las desigualdades
%\begin{equation}
%\label{eq:algeom032}
%\frac{f(x_{t})}{x_{t}-x_{0}}\leq \frac{f(x_{0})}{x_{t}-x_{0}} - \frac{f(x_{0})}{x_{1}-x_{0}} + \frac{f(x_{1})}{x_{1}-x_{0}},
%\end{equation}
%y
%\begin{equation}
%\label{eq:algeom033}
%\frac{f(x_{t})}{x_{1}-x_{t}}\leq \frac{f(x_{0})}{x_{1}-x_{0}} + \frac{f(x_{1})}{x_{1}-x_{t}} - \frac{f(x_{1})}{x_{1}-x_{0}},
%\end{equation}
%de donde
%\begin{equation}
%\label{eq:algeom034}
%\frac{f(x_{t})-f(x_{0})}{x_{t}-x_{0}}\leq \frac{f(x_{1})-f(x_{0})}{x_{1}-x_{0}}\leq \frac{f(x_{1})-f(x_{t})}{x_{1}-x_{t}}.\footnote{JC: Poner aquí un "halmos".}
%\end{equation}
