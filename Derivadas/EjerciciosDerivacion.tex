\textbf{Ejercicios de derivaci�n}\\

\begin{enumerate}
%\item
%Halle la derivada de la funci�n dada en el punto indicado, usando la definici�n de derivada
%        \begin{enumerate}
%        \item
%        $f:\mathbb{R}\to \mathbb{R}$, $x\mapsto f(x)= 2x^2+x+1$, en $x=2$.
%        \item
%        $g:\mathbb{R}\to \mathbb{R}$, $t\mapsto g(t)= \frac{2}{t+1}$, en $t=0$.
%        \item
%        $h:\mathbb{R}\to \mathbb{R}$, $s\mapsto h(s)= \frac{s-1}{s}$, en $s=1$.
%        \end{enumerate}
%\item
%Halle la derivada de la expresi�nd dada usando las propiedades algebraicas de la derivada.
%        \begin{enumerate}
%        \item
%        $f(x) = 3x^2-x+1$.
%        \item
%        $h(s) = \frac{s+1}{s-1}$.
%        \item
%        $g(t) = \frac{2}{t^2+1}$.
%        \item
%        $p(x) = x^n-x^{n-1}+1, \ n\geq 2$.
%        \item
%        $g(r) = \sqrt{r}-\sqrt[3]{r^2}+1$.
%        \end{enumerate}
%\item
%Halle la derivada utilizando, de ser necesario, la regla de la cadena.
%        \begin{enumerate}
%        \item
%        $f(x) = (x^2+x+1)^{\frac{1}{2}}$.
%        \item
%        $f(x) = \cos\sqrt{x+1}$.
%        \item
%        $g(s) = \frac{\sen s}{21 + \sen s}$.
%        \item
%        $h(t) = \sqrt[3]{\cos x + x^2}$.
%        \end{enumerate}
\item
Halle la pendiente de la recta tangente a la gr�fica de $f$ en el punto $(x_0, f(x_0))$ y escriba las ecuaciones de dicha recta y de la recta normal en el mismo punto. La pendiente de la recta normal es $-\frac{1}{f'(x_0)}$, si $f'(x_0)\neq 0$.
        \begin{enumerate}
        \item
        $f(x) = -x^2+4$, $x_0=1$.
        \item
        $f(x) = x^3-x+\frac{1}{x}$, $x_0=-1$.
        \item
        $f(x) = 2-(x-2)^4$, $x_0=2$.
        \item
        $f(x) = \sqrt{25-x^2}$, $x_0=3$.
        \end{enumerate}
\item
Halle los puntos de la gr�fica de $g:\mathbb{R}\to \mathbb{R}$ en los cuales la tangente es horizontal.
 \begin{enumerate}
        \item
        $g(x) = 2x^3-9x^2+12x+1$.
        \item
        $g(s) = \frac{s^2+1}{s}$.
        \item
        $g(t) = \sen t + \cos t$.
        \item
        $g(r) = \cos^2 (r+ \pi)$.
        \end{enumerate}
%\item
%Dada  $f:\mathbb{R}\to \mathbb{R}$, notaremos, si existen:
%\begin{align*}
%f_+'(x_0) & = \lim_{x\to x_0^+}\frac{f(x)-f(x_0)}{x-x_0} = \lim_{h\to 0^+}\frac{f(x_0+h)-f(x_0)}{h}\\
%& = \text{``derivada de $f$ en $x_0$ por la derecha''},\\
%f_-'(x_0) & = \lim_{x\to x_0^-}\frac{f(x)-f(x_0)}{x-x_0} = \lim_{h\to 0^-}\frac{f(x_0+h)-f(x_0)}{h} \\
%& = \text{``derivada de $f$ en $x_0$ por la izquierda''}.
%\end{align*}
%Las derivadas $f_+'(x_0) $ y $f_-'(x_0) $ son las pendientes de las rectas tangentes a la gr�fica de $f$ en el punto de coordenadas $(x_0, f(x_0))$, considerando solamente la gr�fica de $f$ a la derecha y a la izquierda, respectivamente, del punto indicado.
%
%En los ejemplos siguientes halle, si existen, $f_+'(x_0) $ y $f_-'(x_0) $.
%        \begin{enumerate}
%        \item
%        $f(x) =
%        \begin{cases}
%        x^2+x-1 & \text{si $x\leq 1$}\\
%        -2x^2+5x-2 & \text{si $x> 1$},
%        \end{cases}
%        \quad x_0 = 1$.
%        \item
%        $f(x) =
%        \begin{cases}
%       (x+1)^\frac{1}{3} & \text{si $x\leq -1$}\\
%        x & \text{si $x> -1$},
%        \end{cases}
%        \quad x_0 = -1$.
%        \item
%        $f(x) = \sqrt{|x-2|}, \quad x_0 = -1$.
%        \item
%         $f(x) =
%        \begin{cases}
%        \sqrt{x-1} & \text{si $x\geq 1$}\\
%        x^2-3x+2 & \text{si $x<
%         1$},
%        \end{cases}
%        \quad x_0 = 1$.
%        \end{enumerate}
%\item
%Demuestre que para $f:\mathbb{R}\to \mathbb{R}$ y $x_0\in \mathbb{R}$
%        \begin{enumerate}
%        \item
%        $\exists f'(x_0)\quad \Leftrightarrow \quad \exists f_-'(x_0) \ \text{ y }\ \exists f_+'(x_0)$.
%        \item
%        En ese caso $f'(x_0) = f_-'(x_0) = f_+'(x_0)$.
%        \end{enumerate}
%\item
%Pruebe que $f$ no es derivable en el punto indicado.
%        \begin{enumerate}
%        \item
%         $f(x) =
%        \begin{cases}
%       \frac{1}{\sqrt{x}} & \text{si $x\geq 1$}\\
%        x +1 & \text{si $x<1$},
%        \end{cases}
%        \quad x = 1$
%        \item
%        $f(x) = \frac{1}{x^2-3x+2}$, $x=1$.
%        \item
%        $f(x) = \sqrt[4]{2x-9}$, $x = 4.5$.
%         \item
%         $f(x) =
%        \begin{cases}
%       2x^2+1 & \text{si $x<0$}\\
%        \sqrt{x +1 }& \text{si $x\geq 0$},
%        \end{cases}
%        \quad x = 0$
%        \end{enumerate}
\item
Halle $f'(x)$ a partir de la gr�fica dada de $f$.\\

[[\\

VER GR�FICAS EN DOCUMENTO EN WORD ADJUNTO.\\

]]\\

\item
Halle las ecuaciones de las rectas tangente y normal a la gr�fica de la ecuaci�n dada en el punto indicado.
        \begin{enumerate}
        \item
        $8x^2-7y+3x = 0$, \ \ en $(1,1)$.
        \item
        $4x-y^2+8y = 3$, \ \ en $(-1 ,1 )$.
        \item
        $y=\frac{x-1}{x+1}$, \ \ en $(0 , -1 )$.
        \item
        $x=\frac{y}{y^2+1}$, \ \ en $(\frac{1}{2} ,1 )$.
        \end{enumerate}
\item
Halle las ecuaciones de las rectas tangente y normal a la gr�fica de $f$ en el punto donde la recta tangente sea paralela a la recta $l$ dada.
        \begin{enumerate}
        \item
        $f(x) = \sqrt{25-x^2}$, \ \ $l:\ 3x+4y+5=0$.
        \item
        $f(x) = 2x^2-3x+2$, \ \ $l:\ 2x-y+8=0$.
        \item
        $f(x) = 1+(x+1)^2$, \ \ $l:\ y=2x-7$.
        \item
        $f(x) = \frac{x+1}{x-1}$, \ \ $l:\ 2y-4x=5$.
        \end{enumerate}
\item
Halle la ecuaci�n de la o las rectas que sean tangentes a las gr�ficas de $f$ y $g$ al mismo tiempo.
        \begin{enumerate}
        \item
        $f(x) = x^2+2x+2$, \ \ $g(x) = \frac{1}{2}(x+1)(3-x)$.
        \item
       $f(x) = 2x^2+6x+1$, \ \ $g(x) = x^2$.\ \ Respuesta: $l_1:\ y = 2x-1$, \ \ $l_2:\ y = -14x-49$.
        \end{enumerate}
\item
%Halle la derivada indicada con $n\in \mathbb{R}$.
%        \begin{enumerate}
%        \item
%        $f''(x)$, si \ $f(x) = \frac{x+1}{x-1}$.
%        \item
%        $f''(x)$, si \ $f(x) = x^5-7x^2+1$.
%        \item
%        $f^{(n)}(x)$, si \ $f(x) = ax^2+bx+c$.
%        \item
%        $f^{(n)}(x)$, si \ $f(x) = ax^3+bx^2+cx+d$.
%        \item
%        $f^{(n)}(x)$, si \ $f(x) = \sen x$.
%        \item
%        $f^{(n)}(x)$, si \ $f(x) = \cos x$.
%        \item
%        $\displaystyle f^{(n)}(x)$, si \ $f(x) = \sum_{k=0}^{m}a_kx^k$ con $m\in \mathbb{N}\setminus \{0\}$.
%        \item
%        $f^{(n)}(x)$, si \ $f(x) = \sen (ax)$.
%        \end{enumerate}
\end{enumerate}
