\chapter{La derivada: definición y propiedades}

\section{Definición}
\begin{defical}[Derivada de una función]\label{et:ddDefinicionDerivada}%
Sea $\funcjc{f}{\Dm(f)}{\mathbb{R}}$ y $a\in\Dm(f)$. Si
existe el límite
\begin{equation}
\label{eq:ddDefinicionDerivada}
\limjc{\frac{f(x) - f(a)}{x-a}}{x}{a}
\end{equation}
se dice que $f$ es derivable en $a$. Este límite se representa por $f'(a)$ o por
$\frac{df}{dx}(a)$, y es denominado \emph{derivada} de $f$ en $a$. La función
\[
\funcionjc{f'}{\Dm(f')}{\mathbb{R}}{x}{f'(x),}
\]
donde $\Dm(f')$ es el subconjunto de $\Dm(f)$ donde $f'$ existe, es denominada la \emph{derivada}
de $f$.
\end{defical}

De esta definición se desprende que si $f$ es derivable en $a$, entonces es continua en $a$. En
efecto, como $a$ está en el dominio de $f$, solo hay que probar que:
\[
\limjc{f(x)}{x}{a}.
\]

Para ello, procedamos de la siguiente manera:
\begin{align*}
f(x) &= [f(x) - f(a)] + f(a) \\[4pt]
  &= \frac{f(x) - f(a)}{x - a}\cdot (x - a) + f(a),
\end{align*}
siempre que $x\neq a$. Entonces, como $f$ es derivable en $a$, existe el límite de la fracción
\[
\frac{f(x) - f(a)}{x - a},
\]
y es igual a $f'(a)$. Entonces:
\[
\limjc{f(x)}{x}{a} = f'(a)\cdot (a - a) + f(a) = f(a),
\]
por las propiedades algebraicas de los límites. En resumen, acabamos de demostrar el siguiente
teorema.

\begin{teocal}[Continuidad de las funciones derivables]\label{et:ddDerivadaContinua}%
Si $f$ es derivable en $a$, entonces $f$ es continua en $a$.
\end{teocal}

El recíproco de este teorema no es verdadero. Por ejemplo, la función $f$ definida en $\mathbb{R}$
por $f(x) = |x|$ es continua en $0$, pero no es derivable en $0$. En efecto, el lector puede
demostrar fácilmente que $f$ es continua en $0$. Veamos por qué no es derivable en $0$.

Si lo fuera, existiría el límite
\[
\limjc{\frac{|x| - |0|}{x - 0}}{x}{0}.
\]
Entonces, existirían los dos límites laterales:
\[
\limjc{\frac{|x|}{x}}{x}{0^+}\yjc \limjc{\frac{|x|}{x}}{x}{0^-},
\]
y serían iguales uno a otro. Sin embargo:
\begin{align*}
\limjc{\frac{|x|}{x}}{x}{0^+} &= \limjc{\frac{x}{x}}{x}{0^+} \\[4pt]
  &= \limjc{1}{x}{0^+} = 1,
\end{align*}
pues $x > 0$, y
\begin{align*}
\limjc{\frac{|x|}{x}}{x}{0^-} &= \limjc{\frac{-x}{x}}{x}{0^-} \\[4pt]
  &= \limjc{-1}{x}{0^-} = -1,
\end{align*}
pues $x < 0$.

Por lo tanto, los límites laterales son diferentes entre sí, lo que significa que la función valor
absoluto no es derivable en $0$.

\begin{teocal}[Derivada de funciones localmente iguales]
Sean:
\begin{itemize}
\item[] $a$ un número real;
\item[] $I$ un intervalo abierto que contiene el número $a$;
\item[] $f$ y $g$ dos funciones definidas en $I$.
\end{itemize}
Si $f(a)=g(a)$ y si $f=g$ cerca de $a$, entonces $f$ es derivable en $a$ si y solo si $g$ es derivable en $a$.
\end{teocal}

De este teorema se desprende el siguiente resultado que es de gran utilidad.

\begin{teocal}[Derivabilidad de funciones iguales en un intervalo abierto]
Sean $I$ un intervalo abierto; y, $f$ y $g$ dos funciones iguales en $I$.

Entonces $f$ es derivable en $I$ si y solo si $g$ es derivable en $I$.
\end{teocal}

\begin{exemplo}[Solución]{%
La función constante es derivable en $\mathbb{R}$ y su derivada es la función constante $0$.
}%
Sean $c\in\mathbb{R}$ y $\funcjc{f}{\mathbb{R}}{\mathbb{R}}$ definida por:
\[
f(x) = c
\]
para todo $x\in\mathbb{R}$. Demostremos que el límite~(\ref{eq:ddDefinicionDerivada}) existe para
cada $a\in\mathbb{R}$.

Para ello, procedamos así. Sean $a\in\mathbb{R}$ y $x\neq a$. Como
\begin{align*}
\frac{f(x) - f(a)}{x - a} &= \frac{c - c}{x - a} \\[4pt]
  &= \frac{0}{x - a} = 0,
\end{align*}
el límite~(\ref{eq:ddDefinicionDerivada}) existe para todo $a\in\mathbb{R}$ y es igual a $0$. Esto
significa que
\[
f'(a) = 0
\]
para todo $a\in \mathbb{R}$. Es decir, $f'$ es la función constante nula.
\end{exemplo}

\begin{exemplo}[Solución]{%
La función identidad es derivable en $\mathbb{R}$ y su derivada es la función constante
$1$.
}%
Sea
\[
\funcionjc{I}{\mathbb{R}}{\mathbb{R}}{x}{x.}
\]
Demostremos que el límite~(\ref{eq:ddDefinicionDerivada}) existe para cada $a\in\mathbb{R}$.

Para ello, procedamos de la siguiente manera. Sean $a\in\mathbb{R}$ y $x\neq a$. Como
\begin{align*}
\razoncambio{I}{x}{a} &= \razoncambio{}{x}{a} = 1,
\end{align*}
el límite~(\ref{eq:ddDefinicionDerivada}) existe para todo $a\in\mathbb{R}$ y es igual a $1$. De
manera que
\[
I'(a) = 1
\]
para todo $a\in\mathbb{R}$. De modo que la derivada de $I$ es la función constante $1$.
\end{exemplo}

\begin{exemplo}[Solución]{%
Calcular la derivada de la función $\tan$ en $0$
}%
Debemos calcular el límite de la fracción:
\[
\razoncambio{\tan}{x}{0}
\]
cuando $x$ se aproxima a $0$.

Como $\tan 0 = 0$, entonces:
\begin{align*}
\razoncambio{\tan}{x}{0} &= \frac{\tan x}{x} \\[4pt]
  &= \frac{\sen x}{x}\cdot\frac{1}{\cos x}.
\end{align*}
Pero
\[
\limjc{\frac{\sen x}{x}}{x}{0} = 1 \yjc \limjc{\frac{1}{\cos x}}{x}{0} = \frac{1}{\cos 0} = 1.
\]
Entonces:
\[
\tan'(0) = \limjc{\razoncambio{\tan}{x}{0}}{x}{0} = 1.
\]
\end{exemplo}

\begin{exemplo}[Solución]{%
Sean $\funcjc{f}{\Dm(f)}{\mathbb{R}}$ y $a\in\Dm(f)$. Entonces $f$ es derivable en $a$ si y
solo si existe el límite
\[
\limjc{\frac{f(a + h) - f(a)}{h}}{h}{0}.
\]
En el caso en que $f$ sea derivable en $a$, se verifica que:
\[
f'(a) = \limjc{\frac{f(a + h) - f(a)}{h}}{h}{0}.
\]
\eijc{-1}
}%
Supongamos que $f$ es derivable en $a$. Sean $x = g(h)=a + h$ y
\[
F(x) = \frac{f(x) - f(a)}{x - a}.
\]
Entonces:
\[
\limjc{g(h)}{h}{0} = \limjc{a + h}{h}{0} = a
\]
y
\[
\limjc{F(x)}{x}{a} = f'(a).
\]
Como $g(h) \neq a$ si $h\neq 0$, por el teorema del cambio de variable de límites, tenemos que
\[
\limjc{\frac{f(a + h) - f(a)}{h}}{h}{0} = \limjc{F(x)}{x}{a} = f'(a).
\]

Recíprocamente, supongamos que el límite
\begin{equation}
\label{eq:ddDefinicionAlternativa}
\limjc{\frac{f(a + h) - f(a)}{h}}{h}{0}
\end{equation}
existe. Sea $h = \varphi(x) = x - a$. Entonces:
\[
\limjc{\varphi(x)}{x}{a} = 0
\]
y $\varphi(x) \neq 0$ si $x \neq a$. Entonces, por el teorema de cambio de variable, tenemos que
\[
\limjc{\frac{f(x) - f(a)}{x - a}}{x}{a} = \limjc{\frac{f(a+h) - f(a)}{h}}{h}{0} = f'(a).
\]
\end{exemplo}

Para demostrar que una función $f$ es derivable en $a$ y calcular $f'(a)$, es común demostrar que
existe el límite~(\ref{eq:ddDefinicionAlternativa}) y calcular su valor.

\begin{exemplo}[Solución]{%
Sea $\funcjc{f}{\mathbb{R}}{\mathbb{R}}$ definida por
\[
f(x) = x^3 - x + 1.
\]
Obtener $f'$.
}%
Sea $x\in\mathbb{R}$. Veamos para qué valores de $x$ existe el
límite~(\ref{eq:ddDefinicionAlternativa}). Para ello, procedamos así. Sea $h\neq 0$:
\begin{align*}
\frac{f(x + h) - f(x)}{h} &= \frac{[(x+h)^3 - (x+h) + 1] - (x^3 - x + 1)}{h} \\[4pt]
  &= \frac{[(x^3 + 3x^2h + 3xh^2 + h^3) - (x+h)] - (x^3 - x)}{h} \\[4pt]
  &= \frac{3x^2 + 3xh^2 + h^2 - h}{h} \\
  &= 3x^2 + 3xh + h^2 - 1.
\end{align*}
Por lo tanto, para todo $x\in\mathbb{R}$, existe el límite~(\ref{eq:ddDefinicionAlternativa})
cuando $h$ se aproxima a $0$, y es igual a
\[
f'(x) = 3x^2 - 1.
\]
Por lo tanto, el dominio de la función $f'$ es $\mathbb{R}$.
\end{exemplo}

\begin{exemplo}[Solución]{%
Obtener la derivada de la función $\sen$.
}%
El dominio de la función $\sen$ es $\mathbb{R}$. Veamos para qué valores $x$ de este dominio existe
$\sen'(x)$. Para ello, procedamos de la siguiente manera:
\begin{align*}
\frac{\sen(x + h) - \sen x}{h} &= \frac{(\sen x\cos h + \sen h\cos x) - \sen x}{h} \\[4pt]
  &= \frac{\cos h - 1}{h}\cdot\sen x + \frac{\sen h}{h}\cdot\cos x,
\end{align*}
con $x\in\mathbb{R}$ y $h\neq 0$. Pero
\[
\limjc{\frac{\sen h}{h}}{h}{0} = 1 \yjc \limjc{\frac{\cos h - 1}{h}}{h}{0} = 0.
\]
Entonces:
\begin{align*}
\sen'(x) &= \limjc{\frac{\sen(x + h) - \sen x}{h}}{h}{0} \\[4pt]
  &= \limjc{\left(\frac{\cos h -1}{h}\cdot\sen x + \frac{\sen h}{h}\cdot\cos x\right)}{h}{0} \\[4pt]
  &= 0\cdot\sen x + 1\cdot\cos x = \cos x
\end{align*}
para todo $x\in\mathbb{R}$. Por lo tanto:
\[
\sen'(x) = \cos x \yjc \sen' = \cos.
\]
\end{exemplo}

\begin{exemplo}[Solución]{%
Obtener la derivada de la función $\cos$
}%
El dominio de la función $\cos$ es $\mathbb{R}$. Veamos para qué valores $x \in\mathbb{R}$ existe
$\cos'(x)$. Para ello, procedamos de la siguiente manera:
\begin{align*}
\frac{\cos(x + h) - \cos x}{h} &= \frac{(\cos x\cos h - \sen h\sen x) - \cos x}{h} \\[4pt]
  &= \frac{\cos h - 1}{h}\cdot\cos x - \frac{\sen h}{h}\cdot\sen x,
\end{align*}
con $x\in\mathbb{R}$ y $h\neq 0$.

Por lo tanto:
\begin{align*}
\cos'(x) &= \limjc{\frac{\cos(x + h) - \cos x}{h}}{h}{0} \\[4pt]
  &= \limjc{\left(\frac{\cos h -1}{h}\cdot\cos x - \frac{\sen h}{h}\cdot\sen x\right)}{h}{0} \\[4pt]
  &= 0\cdot\cos x - 1\cdot\cos x = -\sen x
\end{align*}
para todo $x\in\mathbb{R}$. Por lo tanto:
\[
\cos'(x) = -\sen x \yjc \cos' = -\sen.
\]
\end{exemplo}

\begin{exemplo}[Solución]{%
Sea $\funcjc{f}{[0,+\infty[}{\Rbb}$ definida por $f(x) = \sqrt{x}$. Obtener la derivada de $f$.
}%
Sea $x \in [0,+\infty[$. Veamos para qué valores de $x$ existe $f'(x)$. Para ello, procedamos así:
\begin{align*}
\frac{f(x + h) - f(x)}{h} &= \frac{\sqrt{x + h} - \sqrt{x}}{h} \\
  &= \frac{\left(\sqrt{x + h} - \sqrt{x}\right)\left(\sqrt{x + h} + \sqrt{x}\right)}%
      {h\left(\sqrt{x + h} + \sqrt{x}\right)} \\
  &= \frac{(x + h) - x}{h\left(\sqrt{x + h} + \sqrt{x}\right)} \\
  &= \frac{h}{h\left(\sqrt{x + h} + \sqrt{x}\right)} \\
  &= \frac{1}{\left(\sqrt{x + h} + \sqrt{x}\right)}
\end{align*}
con $h \neq 0$ y $x \geq 0$.

Dado que $f$ no está definida para valores menores que $0$, si $x = 0$, para cualquier valor de $h$
menor que $0$, sucede que $x + h$ es menor que $0$. Entonces, el límite de
\[
\frac{f(x + h) - f(x)}{h}
\]
no existe si $x = 0$.

En cambio, para $x > 0$, se tiene que:
\begin{align*}
f'(x) &= \limjc{\frac{1}{\left(\sqrt{x + h} + \sqrt{x}\right)}}{h}{0} \\
    &= \frac{1}{\sqrt{x + 0} + \sqrt{x}} = \frac{1}{2\sqrt{x}}.
\end{align*}

En resumen, para todo $x > 0$, se verifica que:
\[
(\sqrt{x})' = \frac{1}{2\sqrt{x}} \yjc (\sqrt{\ }') = \frac{1}{2\sqrt{\ }}.
\]
Para $x = 0$, la derivada de la función raíz cuadrada no existe.
\end{exemplo}

\section{Derivadas Unilaterales}

\begin{defical}
Dada  $f:\mathbb{R}\to \mathbb{R}$, notaremos, si existen:
\begin{align*}
f_+'(x_0) &= \lim_{x\to x_0^+}\frac{f(x)-f(x_0)}{x-x_0} \\
          &= \lim_{h\to 0^+}\frac{f(x_0+h)-f(x_0)}{h}\\
f_-'(x_0) &= \lim_{x\to x_0^-}\frac{f(x)-f(x_0)}{x-x_0} \\
          &= \lim_{h\to 0^-}\frac{f(x_0+h)-f(x_0)}{h}.
\end{align*}
Las derivadas $f_+'(x_0) $ y $f_-'(x_0) $ se denominan \emph{derivada de $f$ en $x_0$ por la
derecha} y \emph{derivada de $f$ en $x_0$ por la izquierda}, respectivamente. Representan las
pendientes de las rectas tangentes a la gráfica de $f$ en el punto de coordenadas $(x_0,
f(x_0))$, considerando solamente la gráfica de $f$ a la derecha y a la izquierda,
respectivamente, del punto indicado.
\end{defical}

\begin{teocal}[Derivada unilateral de funciones localmente iguales]
Sean: $a$ un número real; y, $f$ y $g$ dos funciones localmente iguales por la derecha (respectivamente por la izquierda) y tales que $f(a)=g(a)$.

Entonces
\begin{enumerate}
\item Existe $f^{'}_{+}(a)$ (respectivamente $f^{'}_{-}(a)$) si y solo si existe $g^{'}_{+}(a)$ (respectivamente $g^{'}_{-}(a)$).
\item Si estas derivadas unilaterales existen, son iguales.
\end{enumerate}
\end{teocal}


\begin{multicols}{2}[\section{Ejercicios}]
\begingroup
\small
\begin{enumerate}[leftmargin=*]
\item Halle la derivada de la función $\funcjc{f}{D\subseteq \Rbb}{\Rbb}$ dada en el punto $x$
    indicado, usando la definición de derivada:
        \begin{enumerate}[leftmargin=*]
        \item $f(x)= 2x^2+x+1$ en $x=2$.
        \item $f(x) = \frac{2}{x+1}$ en $x=0$.
        \item $f(x)= \frac{x-1}{x}$ en $x=1$.
        \item $f(x) = \sec x$ en $x = 0$.
        \item $f(x) = \csc x$ en $\displaystyle x = \frac{\pi}{2}$.
        \item $f(x) = \lfloor x \rfloor$ en $\displaystyle x = \frac{1}{2}$.
        \item $f(x) = \lfloor x \rfloor$ si $x \not\in\Zbb$.
        \item $f(x) = \lfloor x \rfloor$ si $x = 0$.
        \item $f(x) = \lfloor x \rfloor$ si $x \in\Zbb$.
        \end{enumerate}
\item En los siguientes ejemplos halle, si existen, $f_+'(x_0)$ y $f_-'(x_0)$.
        \begin{enumerate}
        \item $f(x) =
        \begin{cases}
        x^2+x-1 & \text{si $x\leq 1$}\\
        -2x^2+5x-2 & \text{si $x> 1$},
        \end{cases}$\\[5pt]
        $x_0 = 1$.
        \item $f(x) =
        \begin{cases}
       (x+1)^\frac{1}{3} & \text{si $x\leq -1$}\\
        x & \text{si $x> -1$},
        \end{cases}$\\[5pt]
        $x_0 = -1$.
        \item $f(x) = \sqrt{|x-2|}$, $x_0 = -1$.
        \item $f(x) =
        \begin{cases}
        \sqrt{x-1} & \text{si $x\geq 1$}\\
        x^2-3x+2 & \text{si $x< 1$},
        \end{cases}$\\[5pt]
        $x_0 = 1$.
        \end{enumerate}
\item Demuestre que para $\funcjc{f}{\mathbb{R}}{\mathbb{R}}$ y $x_0\in \mathbb{R}$, existe
    $f'(x_0)$ si y solo si:
        \begin{enumerate}
        \item Existen $f_-'(x_0)$ y $f_+'(x_0)$; y
        \item $f_-'(x_0) = f_+'(x_0)$.
        \end{enumerate}
        Además, en el caso de que exista $f'(x_0)$, siempre se verificará que
        \[
            f'(x_0) = f_-'(x_0) = f_+'(x_0).
        \]
\item Pruebe que $f$ no es derivable en el punto indicado.
        \begin{enumerate}
        \item $f(x) =
        \begin{cases}
       \frac{1}{\sqrt{x}} & \text{si $x\geq 1$}\\
        x +1 & \text{si $x<1$},
        \end{cases}
        \quad x = 1$.
        \item $f(x) = \frac{1}{x^2-3x+2}$, $x=1$.
        \item $f(x) = \sqrt[4]{2x-9}$, $x = 4.5$.
         \item $f(x) =
        \begin{cases}
       2x^2+1 & \text{si $x<0$}\\
        \sqrt{x +1 }& \text{si $x\geq 0$},
        \end{cases}
        \quad x = 0$.
        \end{enumerate}
\end{enumerate}
\endgroup
\end{multicols}

\section{Propiedades de la derivada}
Calcular la derivada de una función utilizando únicamente la definición puede resultar muy
engorroso, como el lector puede comprobar en los ejercicios de la sección anterior. Para facilitar
los cálculos, a partir de las propiedades algebraicas de los límites (límite de la suma, del
producto, etcétera), se obtienen propiedades algebraicas de la derivada. Éstas permitirán calcular
la derivada de muchas funciones si se conoce la derivada de funciones ``más simples''.

Por ejemplo, podemos calcular la derivada de $\funcjc{f}{\mathbb{R}}{R}$ definida por
\[
f(x) = x + 2
\]
sabiendo que la derivada de $x$ es $1$, de $2$ es $0$ y que la derivada de la suma de dos funciones
es la suma de las derivadas de cada función. Por lo tanto:
\begin{align*}
f'(x) &= (x + 2)' \\
  &= (x)' + (2)' = 1 + 0 = 1.
\end{align*}
Es decir, la derivada de $f$ es la función constante $1$.

El siguiente teorema enuncia las propiedades de la derivada que se obtienen de la simple aplicación
del concepto de derivada y de las propiedades algebraicas de los límites (ver el teorema
\ref{teol:Algebra} en la página \pageref{teol:Algebra}).

\begin{teocal}[Propiedades algebraicas I]\label{et:ddAlgebraDerivadasI}%
Sean $f$ y $g$ dos funciones reales derivables en $a \in
\mathbb{R}$. Entonces:
\begin{enumerate}
\item \textit{Suma}: $f + g$ es derivable en $a$ y
    \[
      (f + g)'(a) = f'(a) + g'(a).
    \]

\item \textit{Producto}: $fg$ es derivable en $a$ y
    \[
      (fg)'(a) = f'(a)g(a) + f(a)g'(a).
    \]

\item \textit{Inverso multiplicativo}: Si $g(a) \neq 0$, $\frac{1}{g}$ es derivable en $a$ y
    \[
      \left(\frac{1}{g}\right)'(a) = -\frac{g'(a)}{g^2(a)}.
    \]
\end{enumerate}
\end{teocal}

Y de este teorema, el siguiente se obtiene inmediatamente.

\begin{teocal}[Propiedades algebraicas II]\label{et:ddAlgebraDerivadasII}%
Sean $f$ y $g$ dos funciones reales derivables en $a \in \mathbb{R}$ y
$\lambda\in\mathbb{R}$. Entonces:
\begin{enumerate}
\item \textit{Escalar por función}: $\lambda f$ es derivable en $a$ y
  \[
    (\lambda f)'(a) = \lambda f'(a).
  \]

\item \textit{Resta}: $(f - g)$ es derivable en $a$ y
    \[
      (f - g)'(a) = f'(a) - g'(a).
    \]

\item \textit{Cociente}: si $g(a) \neq 0$, $\frac{f}{g}$ es derivable en $a$ y
  \[
    \left(\frac{f}{g}\right)'(a) = \frac{f'(a)g(a) - f(a)g'(a)}{g^2(a)}.
  \]
\end{enumerate}
\end{teocal}

Dada una función $f$, debe estar claro que $f$ es diferente de $f(x)$. El segundo símbolo indica el
valor que toma la función $f$ en $x$. Sin embargo, es común abusar del lenguaje y referirse a la
función $f$ a través de $f(x)$. Si no se especifica a qué conjunto pertenece $x$, se entiende
tácitamente que el dominio de $f$ es el conjunto más grande de $\mathbb{R}$ en el cual $f(x)$ está
definida.

Es así que, si se pide ``calcular la derivada de la función $x^2$'', lo que se pide realizar es el
cálculo de la derivada de la función f definida por $f(x) = x^2$ para todo $x\in\mathbb{R}$.

En general, en vez de decir ``derivar la función $f$'', se puede decir ``derive $f(x)$'', o
``derive $f(t)$'', etcétera. También se podrá decir ``calcule $f'(x)$ o $f'(t)$''.

La ventaja de este abuso del lenguaje es que nos ahorra el introducir un nombre para la función. A
lo largo de este libro, utilizaremos indistintamente ambas formas de referirnos a las funciones,
salvo que pueda haber lugar para confundir la función $f$ con el valor que ésta puede tomar en
algún $x$ en particular.

\begin{exemplo}[Solución]{%
Calcular $(x^2)'$ y $(x^3)'$.
}%
Como $x^2 = x \cdot x$, podemos aplicar la propiedad algebraica de la derivada para la derivada del
producto de dos funciones (teorema~\ref{et:ddAlgebraDerivadasI}).

En este caso, $f(x) = x$ y $g(x) = x$; estas dos funciones son la identidad cuya derivada existe
para todo $x\in\mathbb{R}$ y es igual a $1$. Entonces:
\[
x^2 = f(x)g(x).
\]
Por lo tanto, $x^2$ es derivable en todo $x\in\mathbb{R}$ y su derivada se obtiene de la siguiente
manera:
\begin{align*}
(x^2)' &= (fg)'(x) \\
  &= f'(x)g(x)+ f(x)g'(x)\\
  &= 1\cdot x + x\cdot 1 = x + x,
\end{align*}
  pues $f'(x) = g'(x) = (x)' = 1$. Por lo tanto:
\[
  (x^2)' = 2x
\]
para todo $x\in\mathbb{R}$.

Para calcular la derivada de $x^3$, apliquemos nuevamente la regla de la derivada de un producto,
pero esta vez a $x^2$ y a $x$:
\begin{align*}
(x^3)' &= (x^2 \cdot x)'\\
  &= (x^2)' \cdot x + x^2 \cdot (x)' \\
  &= (2x)(x) + (x^2)(1) = 2x^2 + x^2.
\end{align*}
Por lo tanto:
\[
(x^3)' = 3x^2.
\]
\end{exemplo}

Ahora obtengamos una fórmula general para la derivada de $x^n$ con $n$ un número natural. Para
ello, recordemos el siguiente ``producto notable'':
\[
x^n - a^n = (x - a)(x^{n-1} + x^{n-2}a + x^{n-3}a^2 + \cdots + x^2a^{n-3} + xa^{n-2} + a^{n-1}).
\]

\begin{exemplo}[Solución]{\label{ej:ddDerivadaPotencia}%
Obtener $(x^n)'$ para todo $n\in\mathbb{N}$.
}%
Sean $n\in\mathbb{N}$ y $\funcjc{f}{\mathbb{R}}{\mathbb{R}}$ definida por $f(x) = x^n$ para todo $x
\in\mathbb{R}$. Obtengamos la derivada de $f$ para cualquier $a\in\mathbb{R}$. Sea $x\neq a$:
\begin{align*}
\frac{f(x) - f(a)}{x - a} &= \frac{x^n - a^n}{x - a} \\[4pt]
  &= \frac{(x - a)(x^{n-1} + x^{n-2}a + x^{n-3}a^2 + \cdots +
    x^2a^{n-3} + xa^{n-2} + a^{n-1})}{x - a}\\[4pt]
  &= x^{n-1} + x^{n-2}a + x^{n-3}a^2 + \cdots + x^2a^{n-3} + xa^{n-2} + a^{n-1}.
\end{align*}
Por lo tanto:
\begin{align*}
f'(a) &= \limjc{\frac{f(x) - f(a)}{x - a}}{x}{a} \\
  &= \limjc{\left(x^{n-1} + x^{n-2}a + x^{n-3}a^2 + \cdots + x^2a^{n-3} + xa^{n-2} + a^{n-1}\right)}{x}{a} \\
  &= a^{n-1} + a^{n-2}a + a^{n-3}a^2 + \cdots + a^2a^{n-3} + aa^{n-2} + a^{n-1} \\
  &= a^{n-1} + a^{n-1} + a^{n-1} + \cdots + a^{n-1} + a^{n-1} + a^{n-1} \\
  &= na^{n-1},
\end{align*}
pues la suma tiene $n$ términos. Por lo tanto:
\[
(x^n)' = nx^{n-1},
\]
para todo $x\in\mathbb{R}$ y todo $n\in\mathbb{N}$.

Por ejemplo, $(x^6)' = 6x^5$.
\end{exemplo}

\begin{exemplo}{%
Calcular la derivada de $x^5 + 2x^4 - 3x^3 - 4x^2 + 5x - 1$.
}%
Esta función es la suma y resta de funciones de la forma $ax^n$. Cada una de esas funciones tiene
derivada para cada número real $x$. La propiedad de la derivada para la suma de dos funciones
(teorema~\ref{et:ddAlgebraDerivadasI}), el del producto de un escalar por una función
(teorema~\ref{et:ddAlgebraDerivadasII}) y el de la resta de dos funciones
(teorema~\ref{et:ddAlgebraDerivadasII}) nos permiten proceder de la siguiente manera:
\begin{align*}
(x^5 + 2x^4 - 3x^3 - 4x^2 + 5x - 1)' &= (x^5)' + (2x^4)' - (3x^3)' - (4x^2)' + (5x)' - (1)'\\
  &= 5x^4 + 2(x^4)' - 3(x^3)' - 4(x^2)' + 5(x)' - 0 \\
  &= 5x^4 + 2(4x^3) - 3(3x^2) - 4(2x) + 5(1) \\
  &= 5x^4 + 8x^3 - 9x^2 - 8x + 5.
\end{align*}
Por lo tanto:
\[
(x^5 + 2x^4 - 3x^3 - 4x^2 + 5x - 1)' = 5x^4 + 8x^3 - 9x^2 - 8x + 5
\]
para todo $x\in\mathbb{R}$.
\end{exemplo}%Fin de \ejemplo

\begin{exemplo}[Solución]{%
Sea $\displaystyle f(x) = \frac{1 - x + x^2}{1 + x - x^2}$. Calcular $f'(x)$.
}%
La función $f$ es el cociente de las funciones $g$ y $h$ definidas por
\[
g(x) = 1 - x + x^2 \yjc h(x) = 1 + x - x^2.
\]
El dominio de $f$ es $\Dm(f) = \mathbb{R} - \left\{\frac{1-\sqrt{5}}{2},\frac{1 +
\sqrt{5}}{2}\right\}$, pues las raíces de la ecuación
\[
1 + x - x^2 = 0
\]
son $\frac{1-\sqrt{5}}{2}$ y $\frac{1 + \sqrt{5}}{2}$.

Para todo $x\in\Dm(f)$, se tiene que:
\[
f(x) = \frac{g(x)}{h(x)},
\]
pues $h(x) \neq 0$ para dichos $x$.

Por la propiedad de la derivada de un cociente (teorema~\ref{et:ddAlgebraDerivadasII}) tenemos que:
\begin{align*}
f'(x) &= \left(\frac{g(x)}{h(x)}\right)' \\[4pt]
   &= \frac{g'(x)h(x) - g(x)h'(x)}{h^2(x)} \\[4pt]
   &= \frac{(1 - x + x^2)'(1 + x - x^2) - (1 - x + x^2)(1 + x - x^2)'}{(1 + x - x^2)^2} \\[4pt]
   &= \frac{(0 - 1 + 2x)(1 + x - x^2) - (1 - x + x^2)(0 + 1 - 2x)}{(1 + x - x^2)^2}\\[4pt]
   &= \frac{-(1 - 2x)(1 + x - x^2) - (1 - 2x)(1 - x + x^2)}{(1 + x - x^2)^2} \\[4pt]
   &= \frac{-(1 - 2x)(1 + x - x^2 + 1 - x + x^2)}{(1 + x - x^2)^2}.\\[4pt]
\end{align*}
Por lo tanto:
\[
f'(x) = \frac{2(2x - 1)}{(1 - x + x^2)^2}
\]
para todo $x\in\Dm(f)$.

\begingroup
\renewcommand{\qedsymbol}{}
\begin{proof}[Solución 2]
En este caso, es posible calcular la derivada de $f$ de una manera más simple.

En efecto, la ley de asignación de la función $f$, que es el cociente de dos polinomios de segundo
grado, puede ser expresada de la siguiente manera:
\[
f(x) = \frac{1 - x + x^2}{1 + x - x^2} = -1 + \frac{2}{1 + x - x^2}.
\]
Esta representación se obtiene al dividir el numerador por el denominador y obtener el cociente y
el residuo.

Entonces:
\begin{align*}
f'(x) &= \left(-1 + \frac{2}{1 + x - x^2}\right)' \\[4pt]
   &= 0 + 2\frac{-(1 + x - x^2)'}{(1 +x - x^2)^2} \\[4pt]
   &= 2\frac{-0 - 1 + 2x}{(1 +x - x^2)^2} = \frac{2(2x - 1)}{(1 +x - x^2)^2}.\qedhere
\end{align*}
\end{proof}
\endgroup
\end{exemplo}

\begin{exemplo}[Solución]{%
Sea $f$ una función real definida por
\[
f(x) = \frac{1}{x}\sen x
\]
para $x \neq 0$. Obtener $f'$
}%
El dominio de $f$ es $\Dm(f) = \mathbb{R} - \{0\}$. Para $x \in\Dm(f)$, $f(x)$ es el producto de la
función $\frac{1}{x}$ y $\sen x$. Entonces, por la propiedad de la derivada de un producto de
funciones (teorema~\ref{et:ddAlgebraDerivadasI}) tenemos que
\[
f'(x) = \left(\frac{1}{x}\right)'(\sen x) + \frac{1}{x}(\sen x)'.
\]
Pero
\[
\left(\frac{1}{x}\right)' = -\frac{(x)'}{x^2} = -\frac{1}{x^2}
\]
por la propiedad de la derivada del inverso multiplicativo (teorema~\ref{et:ddAlgebraDerivadasII}),
ya que $x\neq 0$. También tenemos que:
\[
(\sen x)' = \cos x.
\]
Por lo tanto:
\begin{align*}
f'(x) &= -\frac{1}{x^2}\sen x + \frac{1}{x}\cos x \\[4pt]
   &= \frac{x\cos x - \sen x}{x^2}
\end{align*}
para todo $x \neq 0$.
\end{exemplo}

\begin{exemplo}[Solución]{%
Sea $f$ una función polinomial de grado $n = 1$. Es decir,
\[
f(x) = a_0 + a_1x + a_2x^2 + \cdots + a_nx^n = \sum_{k=0}^n a_x^k,
\]
para todo $x\in\mathbb{R}$. Calcular $f'$.
}%
Utilizando inducción matemática sobre el número de términos, se obtiene que la derivada de la suma
de un número arbitrario de funciones es igual a suma de las derivadas de cada una de esas
funciones. Como $f$ puede ser visto como la suma de las $n + 1$ funciones $f_k$ definidas por:
\[
f_k(x) = a_kx^k,
\]
para todo $k\in\{0,1,\ldots,n\}$; entonces:
\[
f'(x) = \left(\sum^n_{k=0} a_kx^k\right)' = \sum_{k=0}^n f_k'(x).
\]
Pero, para $k\in\{1,2,\ldots,n\}$, tenemos que:
\[
f'_k(x) = ka_kx^{k-1},
\]
ya que la derivada de un escalar por una función es igual al producto del escalar por la derivada
de la función (ver teorema~\ref{et:ddAlgebraDerivadasII}) y por el ejemplo
\ref{ej:ddDerivadaPotencia}.

Para $k = 0$, en cambio, tenemos que:
\[
f'_0(x) = (a_0)' = 0,
\]
pues la derivada de una función constante es la función cero.

Por lo tanto:
\[
f'(x) = \sum_{k=1}^n ka_kx^{k-1}
\]
para todo $x\in\mathbb{R}$.
\end{exemplo}

\begin{exemplo}[Solución]{%
Obtener la derivada de $\tan$.
}%
Para todo $x\in\Dm(\tan)$, se tiene que:
\[
\tan x = \frac{\sen x}{\cos x}.
\]
Por lo tanto, para obtener la derivada de tan, podemos aplicar la derivada del cociente de dos
funciones (teorema~\ref{et:ddAlgebraDerivadasII}). Obtendremos lo siguiente:
\begin{align*}
\tan' x &= \left(\frac{\sen x}{\cos x}\right)' \\[4pt]
   &= \frac{\sen' x\cos x - \sen x\cos' x}{\cos^2 x} \\[4pt]
   &= \frac{\cos x \cos x - (\sen x)(-\sen x)}{cos^2 x} \\[4pt]
   &= \frac{\cos^2 x + \sen^2 x}{\cos^2 x} = \frac{1}{\cos^2 x}.
\end{align*}
Por lo tanto:
\[
\tan'(x) = \sec x^{2} = 1 + \tan^2 x,
\]
para todo $x \neq \frac{\pi}{2} + k\pi$ con $k\in\mathbb{Z}$, pues $\Dm(\tan) = \mathbb{R} -
\{\frac{\pi}{2} + k\pi : k\in\mathbb{Z}\}$.
\end{exemplo}

\begin{exemplo}[Solución]{%
Obtener la derivada de $\cot$.
}%
Recordemos que
\[
\cot x = \frac{1}{\tan x}
\]
para todo $x\neq k\pi$ con $k\in\mathbb{Z}$. Entonces, mediante la aplicación de la regla para
derivar el inverso multiplicativo de una función (teorema~\ref{et:ddAlgebraDerivadasII}), tenemos
que:
\begin{align*}
\cot'(x) &= \left(\frac{1}{\tan x}\right)' \\[4pt]
   &= -\frac{\tan' x}{\tan^2 x} \\[4pt]
   &= -\frac{1 + \tan^2 x}{\tan^2 x} = -\left(\frac{1}{\tan^2 x} + 1\right).
\end{align*}
Por lo tanto:
\[
\cot' x = -(1 + \cot^2 x) = -\csc^2 x
\]
para todo $x\neq k\pi$ con $k\in\mathbb{Z}$.
\end{exemplo}

\begin{exemplo}[Solución]{%
Obtener la derivada de $\sec$.
}%
Puesto que
\[
\sec x = \frac{1}{\cos x}
\]
para todo $x \neq \frac{\pi}{2} + k\pi$ con $k\in\mathbb{Z}$, tenemos que:
\begin{align*}
\sec' x &= -\frac{\cos' x}{\cos^2 x} \\[4pt]
   &= -\frac{-\sen x}{\cos^2 x} \\[4pt]
   &= \frac{\sen x}{\cos x}\cdot\frac{1}{\cos x}
\end{align*}
para todo $x \neq \frac{\pi}{2} + k\pi$. Por lo tanto:
\[
\sec' x  = \tan x \sec x
\]
para todo $x \neq \frac{\pi}{2} + k\pi$ con $k\in\mathbb{Z}$.
\end{exemplo}

\begin{exemplo}[Solución]{%
Obtener la derivada de $\csc$.
}%
Para todo $x\neq k\pi$ con $k\in \mathbb{Z}$, se verifica que:
\[
\csc x = \frac{1}{\sen x}.
\]
Por lo tanto:
\begin{align*}
\csc' x &= -\frac{\sen' x}{\sen^2 x} \\[4pt]
   &= -\frac{\cos x}{\sen^2 x} \\[4pt]
   &= -\frac{\cos x}{\sen x}\cdot\frac{1}{\sen x}
\end{align*}
para todo $x \neq k\pi$. Entonces:
\[
\csc' x  = -\cot x \csc x
\]
para todo $x \neq k\pi$ con $k\in\mathbb{Z}$.
\end{exemplo}

\section{Ejercicios}
\begingroup\small
Halle $f'(x)$ usando las propiedades algebraicas de la derivada (teoremas
\ref{et:ddAlgebraDerivadasI} y \ref{et:ddAlgebraDerivadasII}).
\begin{multicols}{2}
\begin{enumerate}[leftmargin=*]
\item $\displaystyle f(x) = 5x^2 + 7$.
\item $\displaystyle f(x) = x^3 + 3x^2 - 5x + 2$.
\item $f(x) = 3x^2-x+1$.
\item $\displaystyle f(x) = 7x^{15} + 8x^{-7}$.
\item $f(x) = \frac{x+1}{x-1}$.
\item $f(x) = \frac{2}{x^2+1}$.
\item $f(x) = x^n-x^{n-1}+1, \ n\geq 2$.
\item $f(x) = \sqrt{x}-\sqrt[3]{x}+1$.
\item $\displaystyle f(x) = \frac{2}{x^2} + \frac{3}{x^3} + \frac{5}{x^5}$.
\item $\displaystyle f(x) = \sqrt{x} + \sqrt[3]{x} + \sqrt[5]{x}$.
\item $\displaystyle f(x) = \frac{2}{3}x^{\frac{3}{2}} - x^{-3} + \frac{5}{x^4}$.
\item $\displaystyle f(x) = x^{\sqrt{3}} - x^{-\sqrt{3}}$.
\item $\displaystyle f(x) = 7x\cos x$.
\item $\displaystyle f(x) = (x^2 + 1)\tan x$.
\item $\displaystyle f(x) = x^2\cot x + 5$.
\item $\displaystyle f(x) = \frac{\sqrt{x}}{\tan x}$.
\item $\displaystyle f(x) = \frac{\sen x + \cos x}{\sen x - \cos x}$.
\end{enumerate}
\end{multicols}
\endgroup

\section{La regla de la cadena o la derivada de la compuesta}
Con la ayuda de las propiedades algebraicas de la derivada se obtienen las derivadas de un
considerable número de funciones. Sin embargo, no todas las funciones se pueden expresar como suma,
resta, producto, división de otras funciones. Muchas funciones se expresan como la composición de
dos o más funciones. La siguiente propiedad de las derivadas nos dice cómo obtener la derivada de
la composición de dos funciones si éstas son derivables. Esta propiedad es conocida como la regla
de la cadena. El porqué de este nombre se verá más adelante.

\begin{teocal}[Regla de la cadena o derivada de la función compuesta]
Sean $f$ y $g$ dos funciones reales tales que existe $f\circ g$, y $a\in\Dm(f\circ g)$. Supongamos
que:
\begin{enumerate}
\item $g$ es derivable en $a$; y
\item $f$ es derivable en $g(a)$.
\end{enumerate}
Entonces la compuesta $f\circ g$ es derivable en $a$. Además:
\[
(f\circ g)'(a) = f'(g(a))\cdot g'(a).
\]
\end{teocal}

\begin{exemplo}[Solución]{%
Calcular la derivada de $\sen^2 x$.
}%
Sea $h$ una función real definida por
\[
h(x) = \sen^2 x
\]
para todo $x\in\mathbb{R}$. Aunque fuera posible expresar $h(x)$ como suma, producto, etcétera de
otras funciones cuyas derivadas conozcamos, no es fácil ver cuáles serían esas funciones. Sin
embargo, mediante la regla de la cadena podemos obtener la derivada de $h$, pues es la composición
de las funciones $f$ y $g$ definidas de la siguiente manera:
\[
f(u) = u^2
\]
para todo $u\in\mathbb{R}$ y
\[
g(x) = \sen x
\]
para todo $x\in\mathbb{R}$.

En efecto, si $x\in\mathbb{R}$, tenemos que:
\begin{align*}
(f\circ g)(x) &= f(g(x)) \\
   &= f(\sen x) = (\sen x)^2 \\
   &= \sen^2 x = h(x).
\end{align*}
Por lo tanto, $h = f\circ g$ y su dominio es $\mathbb{R}$.

Ahora bien, $f$ y $g$ son derivables en todo $x\in\mathbb{R}$. Por lo tanto, $h$ es derivable en
$\mathbb{R}$, y su derivada se calcula de la siguiente manera:
\begin{align*}
h'(x) &= (f\circ g)'(x) \\
   &= f'(g(x))\cdot g'(x).
\end{align*}
Pero
\[
f'(u) = 2u \yjc g'(x) = \cos x.
\]
Por lo tanto
\[
f'(g(x)) = 2g(x) = 2 \sen x.
\]
Entonces:
\[
h'(x) = 2 \sen x \cos x = \sen 2x
\]
para todo $x\in\mathbb{R}$. Por lo tanto:
\[
(\sen^2 x)' = \sen 2x
\]
para todo $x\in\mathbb{R}$.

En otras palabras, la derivada del cuadrado del $\sen$ es dos veces el $\sen$ (como si fuera la
función $x^2$) multiplicada por la derivada del $\sen$, que es el $\cos$.
\end{exemplo}


\begin{multicols}{2}[\section{Ejercicios}]
\begingroup\small
\begin{enumerate}[leftmargin=*]
\item Antes de ejercitarse en la aplicación de la regla de la cadena, conviene llenar los
    casilleros vacíos de la tabla~\ref{tab:dcCompuestas} en la
    página~\pageref{tab:dcCompuestas}.
\item Calcule $f'(x)$.
  \begin{enumerate}[leftmargin=*]
    \item $f(x) = (x^2+x+1)^{\frac{1}{2}}$.
    \item $f(x) = \cos\sqrt{x+1}$.
    \item $\displaystyle f(x) = \frac{\sen x}{21 + \sen x}$.
    \item $\displaystyle f(x) = \sqrt[3]{\cos x + x^2}$.
    \item $\displaystyle f(x) = \sqrt{x} + \sqrt[3]{x^2}$.
    \item $\displaystyle f(x) = \frac{\sqrt{x}}{1 + \sqrt[3]{x^2}}$.
    \item $\displaystyle f(x) = x^2\sqrt[3]{x^4} + x^5\sqrt[6]{x^7}$.
    \item $\displaystyle f(x) = (ax^2 + bx + c)^k$ con $a, b, c, k$ en $\Rbb$.
    \item $\displaystyle f(x) = Ae^{kx}(a\sen x + b\cos x)$ con $A, k, a, b$ en $\Rbb$.
    \item $\displaystyle f(x) = \sqrt{2 + \sqrt{3 + \sqrt{x}}}$.
    \item $\displaystyle f(x) = \sqrt[n]{\frac{ax + b}{ax - b}}$ con $n\in\Nbb$ y $a, b$ en
        $\Rbb$.
    \item $\displaystyle f(x) = \ln\left(\ln\frac{x^2}{5}\right)$.
    \item $\displaystyle f(x) = \frac{x^2 + \tan x}{\sqrt{1 + x^2}}$.
    \item $\displaystyle f(x) = \ln\left|x + e^x\right|$.
    \item $\displaystyle f(x) = \sen(\cos^3 x)\cos(\sen^3 x)$.
    \item $\displaystyle f(x) = \frac{\tan x^2}{\tan^2 x}$.
    \item $\displaystyle f(x) = \frac{\sen^2 x}{\sen x^2}$.
    \item $\displaystyle f(x) = \ln\sqrt{\frac{1 + \cos x}{1 - \cos x}}$.
  \end{enumerate}

\item Resuelva la ecuación $y'(x) = 0$ si
    \begin{enumerate}[leftmargin=*]
    \item $\displaystyle y = x^3 - 4x^2 + 5x - 2$.
    \item $\displaystyle y = \frac{x^2 + x - 6}{x^2 - 10x + 25}$.
    \item $\displaystyle y = \frac{1}{1 + \sen^2 x}$.
    \item $\displaystyle y = x(x + 1)^2(x - 1)^3$.
    \item $\displaystyle y = \frac{e^{|x - 1|}}{x + 1}$.
    \item $\displaystyle y = \max\{|x|^3, 7x - 6x^2\}$.
    \end{enumerate}

\item Si $g$ y $h$ son funciones derivables, calcule $f'(x)$:
    \begin{enumerate}[leftmargin=*]
    \item $\displaystyle f(x) = g\left(h(x)\right) + h\left(g(x)\right)$
    \item $\displaystyle f(x) = \sqrt[n]{[g(x)]^2 + [h(x)^2]}$ si $[g(x)]^2 + [h(x)^2] > 0$
    \item $\displaystyle f(x) = \ln\left|\frac{g(x)}{h(x)}\right|$ si $g(x)h(x)\neq 0$.
    \item $\displaystyle f(x) = g(\sen^2 x) + h(\cos^2 x)$.
    \end{enumerate}

\item Halle los valores de $\alpha$ y $\beta$ en $\Rbb$ de modo que la función $f$ dada a
    continuación sea continua y derivable en $\Rbb$:
    \begin{enumerate}[leftmargin=*]
    \item $\displaystyle f(x) =
    \begin{cases}
      \alpha x + \beta & \text{si $x \leq 1$,} \\
      x^2 & \text{si $x > 1$.}
    \end{cases}$
    \item $\displaystyle f(x) =
    \begin{cases}
        \alpha + \beta x^2 & \text{si |x| < 1,} \\
        \frac{1}{|x|} & \text{si $|x| \geq 1$.}
    \end{cases}$
    \item $\displaystyle f(x) =
    \begin{cases}
      \alpha x^3 + \beta x & \text{si $|x| \leq 2$,} \\
      \frac{1}{\pi}\sen\frac{1}{x} & \text{si $|x| > 2$.}
    \end{cases}$
    \item $\displaystyle f(x) =
    \begin{cases}
        2x - 2 & \text{si $x \leq 1$,} \\
        \alpha(x-1)(x-\beta) & \text{si $x>1$.}
    \end{cases}$
    \end{enumerate}

\item Halle los valores de $\alpha$ y $\beta$ en $\Rbb$ para que $f$ sea derivable en $\Rbb$:
    \begin{enumerate}[leftmargin=*]
    \item $\displaystyle f(x) =
    \begin{cases}
      (x + \alpha)e^{-\beta x} & \text{si $x < 0$,} \\
      \alpha x^2 + \beta x + 1 & \text{si $x \geq 0$.}
    \end{cases}$
    \item $\displaystyle f(x) =
    \begin{cases}
      \alpha x + \beta & \text{si $x < 0$,} \\
      \alpha\cos x + \beta\sen x & \text{si $x\geq 0$.}
    \end{cases}$
    \end{enumerate}

\item ?`Para qué valores de $x\in\Rbb$, $f$ es derivable en $x$?
    \begin{enumerate}[leftmargin=*]
    \item $\displaystyle f(x) = |x^3(x + 1)^2(x + 2)|$.
    \item $\displaystyle f(x) = |\sen x|$.
    \item $\displaystyle f(x) = x|x|$.
    \item $\displaystyle f(x) = |\pi - x|\sen x$.
    \item $\displaystyle f(x) =
    \begin{cases}
      x^2\left|\cos\frac{\pi}{x}\right| & \text{si $x\neq 0$,} \\
      0 & \text{si $x = 0$.}
    \end{cases}$
    \end{enumerate}

\item Halle $f'(x)$ para los $x$ en las que existe:
    \begin{enumerate}[leftmargin=*]
    \item $\displaystyle f(x) =
    \begin{cases}
      x^2 & \text{si $x\in\Qbb$,} \\
      0 & \text{si $x\not\in\Qbb$.}
    \end{cases}$
    \item $\displaystyle f(x) =
    \begin{cases}
      x^2 & \text{si $x\in\Qbb$,} \\
      2|x| - 1 & \text{si $x\not\in\Qbb$.}
    \end{cases}$
    \item $\displaystyle f(x) =
    \begin{cases}
      x & \text{si $x\in\Qbb$,} \\
      0 & \text{si $x\not\in\Qbb$.}
    \end{cases}$
    \item $\displaystyle f(x) = \arccos(\cos x)$.
    \end{enumerate}

\item Dé ejemplos de funciones $f$ y $g$ y $x_0\in\Rbb$ tales que exista $(f\circ g)'(x_0)$
    pero:
    \begin{enumerate}[leftmargin=*]
    \item Existe $f'(g(x_0))$ y no $g'(x_0)$.
    \item No existe $f'(g(x_0))$ y sí $g'(x_0)$.
    \item No existe $f'(g(x_0))$ ni $g'(x_0)$.
    \end{enumerate}

\item Diga si es correcta o no la afirmación siguiente y argumente su respuesta. Sea $I = \
    ]a,b[$
    \begin{enumerate}[leftmargin=*]
    \item Si $f$ y $g$ son derivables en $I$, entonces $f(x) < g(x)$ para todo $x\in I$
        implica que $f'(x) \leq g'(x)$ para todo $x\in I$.
    \item Si $f' < g'$ en $I$, entonces $f < g$ en $I$.
    \item Si $f$ y $g$ son continuas por la derecha en $a$ y si $f(a) = g(a)$ y $f'(x) < g'(x)$ para todo $x\in I$, entonces $f(x) < g(x)$
        para todo $x\in I$.
    \item Si $f$ es derivable en $\Rbb$ y $f$ es par, entonces $f'$ es impar.
    \item Si $f$ es derivable en $\Rbb$ y $f$ es impar, entonces $f'$ es par.
    \item Si $f'$ es par, entonces $f$ es impar.
    \item Si $f'$ es impar, entonces $f$ es par.
    \item Si $f$ es derivable en $I$ y $\displaystyle\limjc{f'(x)}{x}{a^+} = +\infty$,
        entonces $\displaystyle\limjc{f(x)}{x}{a^+} = +\infty$.
    \item Si $f$ es derivable en $I$ y $\displaystyle\limjc{f(x)}{x}{a^+} = +\infty$,
        entonces $\displaystyle\limjc{f'(x)}{x}{a^+} = +\infty$.
    \end{enumerate}
\end{enumerate}
\endgroup
\end{multicols}

\begin{table}[ht]
\[
{\setlength\extrarowheight{5pt}
\begin{array}{|>{\displaystyle}c|>{\displaystyle}c|>{\displaystyle}c|>{\displaystyle}c|} \hline
f(x) & g(x) & f(g(x)) & g(f(x)) \\ \hline
e^x & \cos x & e^{\cos x} & \cos (e^x) \\
x^2 + 2x + 3 & \sqrt[3]{x} & & \\
\sqrt{x} + x + \frac{1}{\sqrt[3]{x}} & 2x + 3 & & \\
2x + \cos x & h(x) + 5 & & \\
& & \sqrt{x^2 + 1} + e^{\sqrt{x^2 + 1}} & \\
& & (\sec 3x + \cos x)^5 & \\
& & & \cos\sqrt{x^2 + 3} \\
& & & \tan\sqrt[3]{(x^4 + 2)^5} \\
& & (x^2 + 1)^{x^2 + 10} & \\
& & \sen(\cos x) & \\
& & \tan\left(\cos\frac{x^2 + 1}{3}\right) & \\
& & & (\sec(x + x^4))^{-5} \\
\hline
\end{array}
}
\]
\caption{Tabla de compuestas para los ejercicios de regla de la cadena}
\label{tab:dcCompuestas}%
\end{table}

\section{Razones de cambio relacionadas}
Supongamos que dos o más magnitudes, digamos $y,z,w$, etcétera, que están relacionadas de alguna
manera entre sí, dependan de una misma variable $x$ (o $t$, o $u$, o $v$, etcétera). Ello hace que
las razones de cambio
\[
\frac{dy}{dx}, \ \frac{dz}{dx},\ \frac{dw}{dx}, \ldots
\]
también estén relacionadas entre sí. Esta última relación puede ser encontrada gracias a la regla
de la cadena, estudiada en la sección anterior. También se puede obtener esa relación con ayuda de
la derivación implícita, que estudiaremos en la siguiente sección.

\begin{exemplo}[Solución]{%
Para inflar un globo esférico inyectamos aire a razón de $500\, \text{litros}/\text{minutos}$. ?`Con qué
razón varía la longitud del radio cuando éste mide $1\metros$?
}%
Si $V$ litros es el volumen del globo cuando su radio mide $r$ decímetros\footnote{Hemos escogido
como unidad de longitud el decímetro para que la unidad de volumen sea el litro, que es un
decímetro cúbico.}, entonces:
\begin{equation}
\label{eq:dd001}
	V=\frac{4}{3}\pi r^{3}.
\end{equation}

Puesto que el aire ingresa al globo a una razón de $500$ litros por minuto, entonces
\[
\frac{dV}{dt}=500.
\]

En este caso, las magnitudes $V$ y $r$, que están relacionadas entre sí por medio de la
igualdad~(\ref{eq:dd001}), dependen, ambas, de una tercera magnitud: el tiempo $t$. Y lo que
queremos calcular, la razón de cambio instantánea del radio cuando su longitud es $10$ decímetros,
es, precisamente, la razón de cambio es $\frac{dr}{dt}$ cuando $r=10$, que está relacionada con la
razón de cambio de $V$ respecto de $t$.

Para ello, derivamos respecto de $t$ ambos lados de la igualdad \ref{eq:dd001}. Obtenemos así la
relación entre las razones de cambio $\frac{dr}{dt}$ y $\frac{dV}{dt}$:
\begin{equation*}
	\frac{dV}{dt}=4\pi r^{2}\frac{dr}{dt}.
\end{equation*}
Entonces:
\begin{equation*}
	\frac{dr}{dt}=\frac{1}{4\pi r^{2}}\frac{dV}{dt}.
\end{equation*}

Por lo tanto, como $r=10$ y  $\frac{dV}{dt}=500$, obtenemos que:
\begin{equation*}
	\frac{dr}{dt}=\frac{500}{4\pi\  10^{2}}=\frac{5}{4\pi}\approx 0.398.
\end{equation*}
Es decir que la longitud del radio $r$ varía a una velocidad de aproximadamente $3.98$ centímetros
por minuto.
\end{exemplo}

\begin{multicols}{2}[\section{Ejercicios}]
\begingroup\small
\begin{enumerate}[leftmargin=*]
\item\label{ex:dcRCCono} El agua escapa del reservorio cónico, mostrado en la figura
    \ref{fig:cono}, a una razón de $20$ litros por minuto: ¿Con qué velocidad disminuye el
    nivel de agua cuando su altura $h$ desde el fondo es de $5$ metros? ¿Cuál es la razón de
    cambio de radio $r$ del espejo del agua en ese instante?

\item\label{ex:dcRCIslote} Un faro ubicado en un islote situado a $3$ kilómetros de la playa
    emite un haz de luz que gira dando una vuelta entera cada minuto (figura \ref{fig:islote}).
    ¿Con qué velocidad se ``mueve'' el punto $P$ de la playa iluminado por el haz de luz $FP$
    emitido por el faro cuando $P$ está situado a $2$ kilómetros del punto $Q$, que es el punto
    de la playa más cercano al faro?

\item Un avión que está a $500$ kilómetros al norte de Quito viene hacia la capital a $400$
    kilómetros por hora, mientras que otro, que está a $600$ kilómetros al este, lo hace a una
    velocidad de $300$ kilómetros por hora. ¿A qué velocidad se acercan el uno al otro?

\item Un rectángulo mide $10$ metros de altura por $20$ metros de base. Si la base aumenta a
    razón constante de un metro por minuto y la altura disminuye a una razón constante de dos
    metros por minuto, ¿con qué velocidad varía el área del rectángulo? El área, ¿aumenta o
    disminuye?

\item Si de un recipiente, que tiene la forma de una pirámide truncada invertida de base
    cuadrada, de $10$ metros de altura, de $10$ metros el lado del cuadrado más grande y de $5$
    metros el lado del cuadrado más pequeño, y que está lleno de agua a media altura, se extrae
    el líquido a una razón constante de un metro cúbico por minuto:
    \begin{enumerate}[leftmargin=*]
    \item ¿Con qué rapidez baja el nivel del agua?
    \item Si, por otro lado, el nivel del agua sube un metro por hora al bombear agua en el
        reservorio cuando está lleno hasta la mitad, ¿cuál es el caudal de agua que se
        introduce?
    \end{enumerate}

\item Un radar que está a $12$ kilómetros de una base militar detecta que un avión sobrevuela
    la base a $9\,000$ metros de altura y que se dirige hacia el radar, manteniendo su altitud
    y velocidad. Si la rapidez con que decrece la distancia entre el avión y el radar es de
    $500$ kilómetros por hora, ¿a qué velocidad vuela el avión?

\item ¿Con qué rapidez varía el área de la corona que queda entre dos circunferencias
    concéntricas de radios $10$ metros y $20$ metros respectivamente, si el diámetro de la más
    pequeña aumenta un metro por hora y el diámetro de la más grande disminuye dos metros por
    hora? ¿Y si los diámetros aumentan un metro por hora? Si el diámetro menor crece dos
    metros, ¿cómo debe variar el otro para que el área de la corona no cambie?

\item Una partícula se mueve en una trayectoria elíptica cuya forma está dada por la ecuación
    \[
      \left(\frac{x}{3}\right)^2 + \left(\frac{y}{2}\right)^2 = 1.
    \]
    Supongamos que las longitudes están expresadas en metros.

    Si se sabe que en un punto de la abscisa $1$, ésta se incrementa a una razón de dos metros
    por segundo, ¿qué sucede con la ordenada del punto con la abscisa dada? Considere los casos
    que se derivan según los cuadrantes en los que se halla la ordenada.

\item Un buque se acerca a un faro de $50$ metros de altura. Se sabe que el ángulo entre la
    horizontal y la recta que une el buque con la punta del faro varía a una razón constante de
    $\alpha^\circ$ por minuto, cuando el buque está a dos kilómetros del faro. ¿Cuál es la
    velocidad del buque? ¿Cuál sería un valor razonable para $\alpha$?
\end{enumerate}
\endgroup
\end{multicols}

\begin{figure}[h]
\begin{center}
\subfloat[Ejercicio \ref{ex:dcRCCono}]{%
    \begin{pspicture}(0,0)(6,6)
    \psset{PointSymbol=none,PointName=none}%
    \footnotesize%

    \pstGeonode[]%
      (3,1){V}(3,5){C}(1.5,5){A}(4.5,5){B}(3,3){D}(0,3){F}%

    \pstLineAB[]%
      {V}{A}%
    \pstLineAB[]%
      {V}{B}%

    \psellipse[]%
      (C)(! \psGetNodeCenter{A} \psGetNodeCenter{C} C.x A.x sub 0.25)

    \pstInterLL%
      {D}{F}{V}{A}{I}%

    \psellipse[]%
      (D)(! \psGetNodeCenter{I} \psGetNodeCenter{D} D.x I.x sub 0.15)%

    \pstLineAB[]%
      {D}{I}%
    \pstMiddleAB[]%
      {D}{I}{M}%
    \uput[90](M){$r$}%

    \psline{|<->|}%
      (! \psGetNodeCenter{A} A.x A.y 0.5 add)(! \psGetNodeCenter{B} B.x B.y 0.5 add)%
    \uput[90](3,5.5){$8\metros$}%

    \psline{|<->|}%
      (! \psGetNodeCenter{V} V.x 1.25 sub V.y)(! \psGetNodeCenter{V} V.x 1.25 sub
      \psGetNodeCenter{I} I.y)%
    \uput[180](1.75,2){$h$}%

    \psline{|<->|}%
      (! \psGetNodeCenter{V} V.x 2 add V.y)(! \psGetNodeCenter{V} V.x 2 add \psGetNodeCenter{B}
      B.y)%
    \uput[0](5,3){$8\metros$}
    \end{pspicture}
    \label{fig:cono}}%
\qquad
\subfloat[Ejercicio \ref{ex:dcRCIslote}]{%
    \begin{pspicture}(0,0)(6,6)
      \psset{PointSymbol=none}%
      \footnotesize%

      \pstGeonode[PosAngle={180,180,0}]%
        (1,5){P}(1,1){Q}(5,1){F}%
      \pstGeonode[PointName=none]%
        (1,0){I}(1,6){S}%

      \pstLineAB[]%
        {I}{S}%
      \psdot[]%
        (F)%

      \pstLineAB[linestyle=dashed,arrows=->]%
        {F}{P}%
      \pstLineAB[arrows=|<->|]%
        {Q}{F}%
      \uput[-90](3,1){$3\kilometros$}%

      \uput[180]{90}(0.75,3){Playa}
    \end{pspicture}
    \label{fig:islote}}%
    \end{center}
\caption{Gráficos de los ejercicios sobre razón de cambio}
\end{figure}
\section{Derivación implícita}

La representación en el plano cartesiano de las parejas ordenadas $(x,y)$ que satisfacen una
ecuación dada, llamada gráfico de esa ecuación, puede consistir en un punto, en una curva,
etcétera.

Cuando ese gráfico es una curva, ésta no es necesariamente el gráfico de una función (sabemos que
para que lo sea, su intersección con cualquier vertical tiene que consistir, a lo más, en un
punto).

Por ejemplo, el gráfico de $x^{2}+y^{2}=25$ es la circunferencia de radio 5 cuyo centro es el
origen del sistema de coordenadas. La recta vertical $x=3$ la corta en dos puntos, $(3,-4)$ y
$(3,4)$, por lo que esta curva no puede ser el gráfico de una función.

Sin embargo, si tomamos solo un pedazo de esa curva, puede suceder que ese trozo sí cumple con el
criterio de los cortes verticales por lo que puede ser el gráfico de una función. En nuestro
ejemplo, si tomamos el trozo de circunferencia que está sobre el eje horizontal, ese pedazo puede
considerarse como el gráfico de la función $f\colon\mathbb{R}  \rightarrow \mathbb{R}$ definida por
\begin{equation*}
	y=f(x)=\sqrt{25-x^{2}}.
\end{equation*}
Lo mismo hubiese sucedido si tomábamos la semicircunferencia que está bajo el eje horizontal, que
puede considerarse como el gráfico de una función $g\colon\mathbb{R}  \rightarrow \mathbb{R}$
definida por:
\begin{equation*}
	y=g(x)=-\sqrt{25-x^{2}}.
\end{equation*}

En este caso, se dice que las funciones $f$ y $g$ están \emph{definidas implícitamente} por la
ecuación:
\begin{equation*}
	x^{2}+y^{2}=25,
\end{equation*}
o simplemente que son \emph{funciones implícitas}.

Podemos ver que
\[
\Dm(f) = \Dm(g) = [-5,5],
\]
y que para todo $x\in [-5,5]$, se verifican las siguientes igualdades:
\begin{equation*}
	x^{2}+[f(x)]^{2}=25
\yjc
	x^{2}+[g(x)]^{2}=25.
\end{equation*}

Por otro lado, se pueden tomar otros pedazos o uniones de pedazos de circunferencia que pueden
considerarse gráficos de funciones definidas implícitamente por la misma ecuación $x^{2}+y^{2}=25$.
En este ejemplo logramos hallar fórmulas para \emph{definir explícitamente} a las funciones $f$ y
$g$.

Si una función $h\colon\mathbb{R}  \rightarrow \mathbb{R}$ puede definirse mediante una fórmula de
la forma $y=h(x)$, se dice que $h$ es una \emph{función explícita}.

En general no siempre es posible explicitar una función implícita. Por ejemplo, para la ecuación
\begin{equation}
\label{eq:MasDer001}
	xy+x^{5}+5x^{2}y^{4}-9y^{5}+2=0,
\end{equation}
vemos que no es posible despejar la $y$ para ponerla en términos de $x$, es decir que no podemos
explicitar las funciones implícitas cuyos gráficos sean subconjuntos del gráfico de esta ecuación.

Sin embargo, gracias a la regla de la cadena, es posible, en estos casos, hallar la derivada de las
funciones implícitas, cuando, en las ecuaciones, cada uno de los miembros es una combinación (que
consiste de sumas, restas, multiplicaciones, divisiones o composiciones finitas), de las funciones
elementales conocidas (polinomios, exponenciales, trigonométricas, o las inversas de las
mencionadas).

En estos casos, se puede derivar ambos miembros respecto a la variable consideradas independiente
($x$ en el ejemplo), y luego se despeja fácilmente la derivada de la variable dependiente ($y$ en
el ejemplo), que aparece al utilizar la regla de la cadena en las expresiones que contienen la
variable dependiente.

Al derivar ambos miembros de la ecuación~(\ref{eq:MasDer001}), tenemos que:
\begin{equation*}
	 y+xy'+5x^{4}+10xy^{4}+20x^{2}y^{3}y'-45y^{4}y'=0,
\end{equation*}
de donde
\begin{equation*}
	 y'=\frac{y+5x^{4}+10xy^{4}}{-x-20x^{2}y^{3}+45y^{4}}.
\end{equation*}

\begin{exemplo}[Solución]{%
Hallar la ecuación de la tangente a la circunferencia $x^{2}+y^{2}=25$ en el punto
$(3,4)$.
}%
La ecuación de la recta será
\begin{equation*}
	y=4+m(x-3),
\end{equation*}
donde $m$ es la pendiente.

Ahora bien, $m$ es la derivada de la función explícita cuyo gráfico es un pedazo de la
circunferencia que contenga al punto $(3,4)$.

Para hallar $y'$, derivemos explícitamente $x^{2}+y^{2}=25$. Obtendremos que se verifica la
igualdad siguiente:
\[
2x+2yy'=0,
\]
de donde se obtiene la derivada de $y$:
\begin{equation*}
	y'=-\frac{x}{y}.
\end{equation*}

Para $(x,y)=(3,4)$ tendremos $m=y'=-\frac{3}{4}$. Entonces, la ecuación de la recta será:
\[
y=4-\frac{3}{4}(x-3),
\]
que es equivalente a:
\begin{equation*}
	3x+4y-25=0.
\end{equation*}
\end{exemplo}

\begin{multicols}{2}[\section{Ejercicios}]
\begingroup
\small
\begin{enumerate}[leftmargin=*]
\item Si una función $\funcjc{f}{x}{y = f(x)}$ es derivable y está definida implícitamente
    por la ecuación dada a continuación, calcule $y' = f'(x)$.
    \begin{enumerate}[leftmargin=*]
    \item $\displaystyle y^5 + y^3 + y - x = 0$.
    \item $\displaystyle y - x = \epsilon\sen y$ con $|\epsilon| < 1$.
    \item $\displaystyle y^2 = 2px$, con $y > 0$.
    \item $\displaystyle \frac{x^2}{a^2} + \frac{y^2}{b^2} = 1$ con $y > 0$.
    \item $\displaystyle \frac{x^2}{a^2} - \frac{y^2}{b^2} = 1$ con $y < 0$.
    \item $\displaystyle (2a - x)y^2 = x^3$, con $a > 0$ y $y < 0$.
    \item $\displaystyle \sqrt{x} + \sqrt{y} = 2$.
    \item $\displaystyle x^{\frac{2}{3}} + y^{\frac{2}{3}} = a^{\frac{2}{3}}$, con $a > 0$,
        $y > 0$.
    \item $\displaystyle 5x^2 + 9y^2 - 30x + 18y + 9 = 0$ con $y < -1$.
    \item $\displaystyle x^2 - 4xy + 4y^2 + 4x - 3y - 7 = 0$ con $x < 2y - 1$.
    \end{enumerate}

\item Si una función $\funcjc{f}{x}{y = f(x)}$ es derivable y está definida implícitamente
    por la ecuación dada a continuación, calcule $f'(a)$.
    \begin{enumerate}[leftmargin=*]
    \item $\displaystyle x^2 + y^2 - 6x + 10y - 2 = 0$, $y > - 5$ y $a = 0$.
    \item $\displaystyle 6xy + 8y^2 - 12x - 26y + 11 = 0$, $y > 0$ y $a = \frac{11}{12}$.
    \item $\displaystyle e^y + xy = e$, $y > 0$ y $a = 0$.
    \item $\displaystyle xy + \ln y = 1$, $y < e^2$ y $a = 0$.
    \end{enumerate}
\end{enumerate}
\endgroup
\end{multicols}

\section{Derivada de la función inversa}
Si $\funcjc{f}{D\subseteq\Rbb}{\Rbb}$ es es inyectiva, existe la función inversa
$\funcjc{f^{-1}}{\Img(f)}{D}$ tal que
\begin{equation*}
	x=f^{-1}(y) \quad \Leftrightarrow \quad y=f(x).
\end{equation*}
Se verifican, entonces, las siguientes igualdades:
\begin{equation*}
	f^{-1}(f(x))=x
\end{equation*}
para todo $x\in D(f)$; y
\begin{equation}
\label{eq:dd002}
f(f^{-1}(y))=y
\end{equation}
para todo $y\in \Dm(f^{-1})$.

Si se conoce la derivada de $f$, podemos hallar la derivada de $f^{-1}$. Esto lo podemos hacer con
ayuda de la regla de la cadena.

En efecto, la igualdad~(\ref{eq:dd002}) puede ser considerada como una ecuación que define
implícitamente $x$ como función de $y$, donde
\[
x = f^{-1}(y).
\]

Podemos, entonces, calcular la derivada de $x$, es decir, la derivada de la inversa de $f$, si
derivamos respecto de $y$ ambos miembros de la igualdad~(\ref{eq:dd002}). Al hacerlo, obtendremos
que:
\[
f'(f^{-1}(y))(f^{-1})'(y) = 1,
\]
de donde:
\begin{equation}
\label{et:ddDerivadaInversa}
	(f^{-1})'(y)= \frac{1}{f'(f^{-1}(y))},
\end{equation}
siempre que $y\in \Dm(f^{-1})$ y que el denominador sea diferente de cero.

\begin{exemplo}[Solución]{%
Hallar la derivada de $\ln$, la función inversa de $f$ definida por $f(x) = e^x$.
}%
En primer lugar:
\[
f'(x) = e^x = f(x).
\]
Por lo tanto:
\[
f'(f^{-1}(y)) = f(f^{-1}(y)) = y.
\]

Entonces, al utilizar la fórmula~(\ref{et:ddDerivadaInversa}), obtenemos que:
\begin{align*}
(f^{-1})'(y) &= \frac{1}{f'(f^{-1}(y))} \\[4pt]
   &= \frac{1}{f(f^{-1}(y))} \\[4pt]
   &= \frac{1}{y}.
\end{align*}

Como $\ln y = f^{-1}(y)$, entonces:
\[
(\ln y)' = \frac{1}{y}.
\]
\end{exemplo}

\begin{exemplo}[Solución]{%
Hallar la derivada de la función $\arcsen$.
}%
En primer lugar, la función $\arcsen$ es la función inversa de la función $\sen$ restringida al
intervalo $[-\frac{\pi}{2},\frac{\pi}{2}]$. Sea, entonces, $f$ definida por:
\[
f(x) = \sen x
\]
para todo $x\in [-\frac{\pi}{2},\frac{\pi}{2}]$.

Entonces $f'(x) = \cos x$ y:
\begin{equation}
\label{eq:dd003}
f'(f^{-1}(y)) = \cos(\arcsen y)
\end{equation}
para todo $y \in [-1,1] = \Dm(\arcsen)$.

Ahora bien, sabemos que
\[
\sen^2 x + \cos^2 x = 1,
\]
de donde
\[
\cos^2 x = 1 - \sen^2 x,
\]
y, dado que $\cos x > 0$ para todo $x\in [-\frac{\pi}{2},\frac{\pi}{2}]$, tenemos que:
\[
\cos x = \sqrt{1 - \sen^2 x}.
\]

Con ayuda de esta última expresión, podemos reescribir la igualdad~(\ref{eq:dd003}) de la siguiente
manera:
\begin{align*}
f'(f^{-1}(y)) &= \cos(\arcsen y) \\
   &= \sqrt{1 - \sen^2(\arcsen y)} \\
   &= \sqrt{1 - \left(\sen(\arcsen y)\right)^2} \\
   &= \sqrt{1 - y^2}.
\end{align*}

Entonces, la fórmula~(\ref{et:ddDerivadaInversa}) nos da:
\begin{align*}
(f^{-1})'(y) &= \frac{1}{f'(f^{-1}(y))} \\[4pt]
   &= \frac{1}{\sqrt{1 - y^2}}.
\end{align*}
Por lo tanto:
\[
(\arcsen y)' = \frac{1}{\sqrt{1 - y^2}}
\]
para todo $y\in [-1,1]$.
\end{exemplo}

\begin{exemplo}[Solución]{%
Las derivadas de $\arccos$, $\arctan$ y $\sqrt{ \ }$.
}%
El procedimiento es similar al del ejemplo anterior. Pero hay que considerar lo siguiente.

En el caso del $\arccos$, su dominio es $[-1,1]$ y su imagen $[0,\pi]$. Esta función es la inversa
de la función $\cos$, restringida al intervalo $[0,\pi]$, cuya derivada es $-\sen$.

Como la función $\sen$ es positiva en el intervalo $[0,\pi]$, entonces:
\[
\sen x = \sqrt{1 - \cos^2 x}.
\]

Al aplicar la fórmula~(\ref{et:ddDerivadaInversa}), obtendremos que:
\[
(\arccos y)' = -\frac{1}{\sqrt{1 - y^2}}
\]
para todo $y\in [-1,1]$.

En el caso de $\arctan$, su dominio es $\mathbb{R}$ y su imagen $[-\frac{\pi}{2},\frac{\pi}{2}]$, y
es la función inversa de $\tan$, restringida a este último intervalo.

La derivada de $\tan$ es $\sec^2$. Con ayuda de la identidad
\[
\sec^2 x = 1 + \tan^2 x,
\]
verdadera para todo $x\in\mathbb{R}$, al aplicar la fórmula~(\ref{et:ddDerivadaInversa}),
obtendremos que
\[
(\arctan y)' = \frac{1}{1 + y^2}
\]
para todo $y\in\mathbb{R}$.

Finalmente, para obtener la derivada de $\sqrt{ \ }$, solo hay que recordar que es la función
inversa de la función $x^2$ restringida al intervalo $[0,+\infty]$. Al aplicar la
fórmula~(\ref{et:ddDerivadaInversa}), obtendremos que:
\[
(\sqrt{y})' = \frac{1}{2\sqrt{y}}
\]
para todo $y > 0$.
\end{exemplo}

\begin{multicols}{2}[\section{Ejercicios}]
\begingroup\small
\begin{enumerate}[leftmargin=*]
\item Calcule la derivada $(f^{-1})'(y_0)$:
    \begin{enumerate}[leftmargin=*]
    \item $\displaystyle f(x) = x + \frac{1}{3}x^3$, $y_0
        \in\{0,\pm\frac{4}{3},\pm\frac{14}{3}\}$.
    \item $\displaystyle f(x) = 2x - \frac{1}{2}\cos x$, $y_0 = -\frac{1}{2}$.
    \item $\displaystyle f(x) = 0.1x + e^{0.01x}$, $y_0 = 1$.
    \item $\displaystyle f(x) = 2x^2 - x^4$, $x > 1$ y $y_0 = 0$.
    \item $\displaystyle f(x) = 2x^2 - x^4$, $0 < x < 1$ y $y_0 = \frac{3}{4}$.
    \item $\displaystyle f(x) = \ln x$, $x > 0$, $y_0=1$.
    \item $\displaystyle f(x) = \exp x$, $y_0=1$.
    \end{enumerate}

\item Si conoce que $(\ln x)' = \frac{1}{x}$ y que $\exp = \ln^{-1}$, calcule $\exp'$.

\item Si conoce que $\exp' = \exp$ y que $\ln = \exp^{-1}$, calcule $\ln'$.

\item Calcule fórmulas para las derivadas de las funciones inversas de $\sinh$, $\cosh$ y
    $\tanh$.
\item Calcule $(f^{-1})'$ y $\Dm((f^{-1})')$:
    \begin{enumerate}[leftmargin=*]
    \item $\displaystyle f(x) = \frac{x^2}{1 + x^2}$, $x < 0$.
    \item $\displaystyle f(x) = \coth x$, $x > 0$.
    \end{enumerate}

\item Si $f(x) = x + \sen x$ para todo $x\in\Rbb$, calcule los valores de $a$ y de $b$ para los
    cuales se verifican las igualdades $(f^{-1})'(a) = +\infty$ y $(f^{-1})'(b) = -\infty$.
\end{enumerate}
\endgroup
\end{multicols}

\section{Derivadas de orden superior}

Recordemos que, dada una función $\funcjc{f}{\Dm(f)}{\mathbb{R}}$, se define la función
$\funcjc{f'}{\Dm(f')}{\mathbb{R}}$, la función derivada de $f$, cuyo dominio es:
\begin{equation*}
	\Dm(f')=\{x\in \Dm(f) : \ \text{existe} \ f'(x)\}.
\end{equation*}

Al ser $f'$ una función, es posible que también sea derivable en algunos elementos de su dominio.
Esta situación conduce a la siguiente definición.

\begin{defical}[Derivadas de segundo orden]
Sea $a\in D(f')$ para el cual existe la derivada de $f'$; es decir, existe $(f')'(a)$. Este número
es denominado \emph{segunda derivada de $f$ en $a$} y es representado por:
\[
f^{\prime\prime}(a) \quad\text{o}\quad \frac{d^{2}}{d x^{2}}f(a).
\]
Es decir:
\begin{equation*}
	f''(a)=\frac{d^{2}}{d x^{2}}f(a)=\frac{d}{d x}\left(\frac{d}{d x}f\right)(a)=(f')'(a).
\end{equation*}
\end{defical}

\begin{exemplo}[Solución]{%
Calcular la segunda derivada de $\sen x$, $\cos x$ y $e^x$.
}%
Sea $f$ definida por $f(x) = \sen x$. Entonces $f$ es derivable en $\mathbb{R}$ y:
\[
f'(x) = \cos x.
\]
Por lo tanto, $f'$ también es derivable en $\mathbb{R}$ y su derivada es igual a la derivada de la
función $\cos$:
\[
(f')'(x) = (\cos x)' = -\sen x.
\]
Por lo tanto:
\[
(\sen x)'' = -\sen x.
\]

De manera similar se establece que $\cos x$ y $e^x$ tienen segunda derivada y:
\[
(\cos x)'' = (-\sen x)' = -\cos x;
\]
y
\[
(e^x)'' = (e^x)' = e^x.
\]
\end{exemplo}

Ahora bien, podría suceder que la función $f''$ también fuera derivable en algunos elementos de su
dominio. Entonces, existiría la función $f'''$, que se definiría por:
\[
f''' = (f'')'.
\]
Esta función es denominada la \emph{derivada de tercer orden}.

En general, podemos hablar de la derivada de orden $n$, que se define inductivamente así:

\begin{defical}[Derivada de orden $n$]
Sea $\funjc{f}{\Dm(f)}{\mathbb{R}}$. Para cada $n\in\mathbb{N}$, se define:
\begin{enumerate}
\item $f^{(0)} = f$.
\item $f^{(n+1)} = (f^{(n)})'$, para $n \geq 0$.
\end{enumerate}
Si existe $f^{(n)}(a)$, este número es denominado la $n$-ésima derivada de $f$ en $a$ y suele ser
representado de la siguiente manera:
\[
f^{(n)}(a) = \frac{d^n}{dx^n}f(a).
\]
\end{defical}

De esta definición, es inmediato que:
\begin{equation*}
	f^{(n)}(a)=\frac{d^{n}}{d x^{n}}f(a)=
   \frac{d}{d x}\left(\frac{d^{n-1}}{d x^{n-1}}f\right)(a) =
   \left(f^{(n-1)}\right)'(a).
\end{equation*}

Tiene sentido definir como la derivada de orden $0$ a la misma función, pues se puede interpretar
esto diciendo que no derivar la función es dejarla intacta.

\begin{exemplo}[Solución]{%
Establecer fórmulas generales para las derivadas de orden $n$ para $\sen x$, $\cos x$ y
$e^x$.
}%
En el ejemplo anterior, calculamos la segunda derivada de estas tres funciones. Calculemos ahora
las derivadas de tercero y cuarto orden.

Para el caso de la función $\sen x$:
\begin{align*}
(\sen x)^{(3)} &= ((\sen x)^{(2)})' = (-\sen x)' = -\cos x. \\
(\sen x)^{(4)} &= ((\sen x)^{(3)})' = (-\cos x)' = \sen x.
\end{align*}

Como podemos ver, la derivada de orden $4$ de $\sen x$ es igual a la función $\sen x$. Esto
significa que la derivada de orden $5$ es igual a la primera derivada; la derivada de orden $6$,
igual a la segunda derivada; la derivada de orden $7$, igual a la tercera; y la de orden $8$, a la
de orden $4$. Es decir, las derivadas se repetirán cada cuatro órdenes. Podemos resumir esto de la
siguiente manera:
\[
\begin{array}{ll}
(\sen x)^{(4n)} &= \sen x. \\[4pt]
(\sen x)^{(4n+1)} &= \cos x. \\[4pt]
(\sen x)^{(4n + 2)} &= -\sen x. \\[4pt]
(\sen x)^{(4n + 3)} &= -\cos x
\end{array}
\]
para todo $n \geq 0$.

De manera similar, se establece para $\cos x$ lo siguiente:
\[
\begin{array}{ll}
(\cos x)^{(4n)} &= -\sen x. \\[4pt]
(\cos x)^{(4n+1)} &= -\cos x. \\[4pt]
(\cos x)^{(4n + 2)} &= \sen x. \\[4pt]
(\cos x)^{(4n + 3)} &= \cos x
\end{array}
\]
para todo $n \geq 0$.

Finalmente, la derivada de $e^x$ siempre es igual a $e^x$, por lo que se verifica la siguiente
fórmula:
\[
(e^x)^{(n)} = e^x
\]
para todo $n\geq 0$.
\end{exemplo}

\section{Ejercicios}
\begingroup\small
Calcule $f^{(n)}$ con $n\in\Nbb$:
\begin{multicols}{2}
\begin{enumerate}[leftmargin=*]
\item $f(x) = \displaystyle\frac{x+1}{x-1}$ si $n = 2$.
\item $f(x) = x^5-7x^2+1$ si $n = 2$.
\item $f(x) = ax^2+bx+c$.
\item $f(x) = ax^3+bx^2+cx+d$.
\item $f(x) = \sen(ax)$.
\item $f(x) = \cos(ax)$.
\item $\displaystyle f(x) = \ln x$.
\item $\displaystyle f(x) = \sum_{k=0}^{m}a_kx^k$ con $m\in \mathbb{N} - \{0\}$.
\item $\displaystyle f(x) = \frac{1}{x}$.
\item $\displaystyle f(x) = \frac{1}{(x + a)}$.
\item $\displaystyle f(x) = \frac{x + a}{x - a}$.
\end{enumerate}
\end{multicols}
\endgroup

\section{Diferenciales}
La derivada de una función $f$ en un punto dado $x_0$ es el valor de la pendiente de la recta
tangente al gráfico de $f$ en el punto de coordenadas $(x_{0},f(x_{0}))$, cuya ecuación es:
\begin{equation*}
	 y=f(x_{0})+f'(x_{0})\, (x-x_{0}).
\end{equation*}

Esta recta suele ser utilizada para aproximar el gráfico de $f$ para  valores cercanos a $x_{0}$,
pues, como se puede apreciar el gráfico que está a continuación, los valores de la recta y de la
función son próximos:
\begin{center}
\def\F{x dup mul}%
\psset{unit=8cm}
\begin{pspicture}(-0.1,-0.1)(0.8,0.65)
   \SpecialCoor

   \psaxes{->}(0,0)(-0.1,-0.1)(0.75,0.6)%
   \uput[-90](0.75,0){$x$}%
   \uput[0](0,0.6){$y$}%

   \psplot[linewidth=3\pslinewidth]{0}{0.7}{\F}%
   \psplotTangent[linecolor=gray,linewidth=1.75\pslinewidth]%
      {0.4}{0.3}{\F}%
   \psline[linestyle=dashed]%
      (! 0 /x 0.4 def \F)(! 0.4 /x 0.4 def \F)(0.4,0)%

   \uput[180](! 0 /x 0.4 def \F){$f(x_0)$}%
   \uput[-90](0.4,0){$x_0$}%
   \uput[0](! 0.7 /x 0.7 def \F){$y = f(x)$}%

\end{pspicture}
\end{center}

Ahora bien, si $x$ varía desde $x_{0}$ hasta $x_0 + \Delta x$, el valor correspondiente de $y$
cambia de $y_{0}=f(x_{0})$ a $y = f(x_{0}+\Delta x)$. Este cambio o variación, representado por
$\Delta y$ se calcula así:
\begin{equation*}
	 \Delta y = y - y_{0}=f(x_{0}+\Delta x)-f(x_{0}).
\end{equation*}

Entonces, vamos a utilizar el valor que toma $(x_{0}+\Delta x)$ en la recta tangente, en lugar del
que toma en $f$; es decir, en lugar de utilizar el valor $y = f(x_{0}+\Delta x)$, utilizamos el
dado por la ecuación de la recta tangente:
\begin{equation*}
	 y = f(x_{0}) +f'(x_{0})((x_{0}+\Delta x)-x_{0})= f(x_{0})+f'(x_{0})\Delta x.
\end{equation*}

Esto implica que al cambio $\Delta y$ lo estamos aproximando con el cambio que se produce en la
recta al variar $x$, de $x_{0}$ a $x_{0}+\Delta x$, que se nota $d y$ y que se calcula así:
\begin{equation*}
	d y=[f(a)+f'(a)\, \Delta x]-f(a)=f'(a)\, \Delta x.
\end{equation*}
Se suele escribir $\Delta x = d x$; a $d x$ se le llama diferencial de $x$. A $d y$ se le llama
\emph{diferencial de $f$ en $a$}. Se tiene entonces
\begin{equation*}
	d y=f'(a)\, d x.
\end{equation*}
Esta igualdad puede interpretarse como la ecuación de la recta tangente en un sistema de
coordenadas $(d x, d y)$ cuyo origen está en $(x_{0},f(x_{0}))$ y que es paralelo al sistema de
coordenadas $(x,y)$.

\begin{exemplo}[Solución]{%
Si $f$ está definida por $f(x) = 3x^2 + 2$, calcular el diferencial de $f$ en $1$.
}%
Sea $y = f(x) = 3x^{2} + 2$. Como $f'(x) = 6x$, para $x_{0}=1$, se tiene que:
\begin{equation*}
	dy=f'(1)\,dx=6dx.
\end{equation*}
\end{exemplo}

\begin{exemplo}[Solución]{%
Si $g$ está definida por $g(t) = t + e^t$, calcular el diferencial de $g$ en $2$.
}%
Sea $z=g(t)=t+e^{t}$. Como $g'(t) = 1+e^{t}$, para $t_{0}=2$ se tiene que:
\begin{equation*}
	dz=g'(2)\,dt=(1+e^{2})dt.
\end{equation*}
\end{exemplo}

Entre las aplicaciones de los diferenciales está la posibilidad de realizar ciertos cálculos
aproximados.

\begin{exemplo}[Solución]{%
Calcular un valor aproximado para $\sqrt{50}$.
}%
Ponemos $f(x)=\sqrt{x}, x_0=49,\Delta x= dx=1$ y aproximamos $\Delta y$ con $dy$:
\begin{equation*}
	 dy=f'(x_0)\,dx=\frac{1}{2\sqrt{49}}\cdot 1=\frac{1}{14}.
\end{equation*}
Entonces
\begin{equation*}
	\sqrt{50}=f(50)=f(49)+\Delta y\approx f(49)+dy=\sqrt{49}+\frac{1}{14}=7+\frac{1}{14}=\frac{99}{14}.
\end{equation*}
Si se tiene en cuenta que $\sqrt{50}\approx 7.071068 \ldots$ y que $\frac{99}{14}\approx 7.071429
\dots$, vemos que la aproximación es aceptable.
\end{exemplo}

\section{Ejercicios}
\begingroup
\small
\begin{multicols}{2}
\begin{enumerate}[leftmargin=*]
\item Halle $\mathrm{d}y$ en función de $x$ y de $\mathrm{d}x$:
  \begin{enumerate}
  \item $y = x^2 - 3x$.
  \item $y = \displaystyle\frac{x + 1}{x - 1}$.
  \item $\displaystyle y = x\sen x + \cos x$.
  \item $\displaystyle y = x^{\frac{2}{3}}$.
  \end{enumerate}

\item Para los valores de $\Delta x$ y de $x$ dados, calcule $\Delta y$ y $\mathrm{d}y$:
  \begin{enumerate}
  \item $\displaystyle y = x^2 + x + 1, \ \Delta x \in \{0.1, 0.5, 1\}, \ x = 1$.
  \item $\displaystyle y = \frac{2x}{x - 1}, \ \Delta x \in \{0.1, 0.5\}, \ x = 1$.
  \item $\displaystyle y = \frac{1}{x}, \ \Delta x \in \{0.1, 0.5\}, \ x = 1$.
  \end{enumerate}

\item Calcule, mediante diferenciales, un valor aproximado de:
  \begin{enumerate}
  \item $\sqrt{26}$.
  \item $\displaystyle\frac{1}{\sqrt{50}}$.
  \item $\displaystyle\frac{0.9^3}{1.9}$.
  \item $\sen(\pi/3 - 0.1)$.
  \item $\sqrt{82}$.
  \item $\displaystyle\frac{1}{\sqrt{37}}$.
  \item $\sen 44^\circ$ (¡Atención: utilice radianes!).
  \item $(2.1)^3 + 3(2.1)^2$.
  \end{enumerate}
\end{enumerate}
\end{multicols}
\endgroup

\section{Cálculo de los ceros de funciones derivables}
Si conocemos que una función $\funcjc{f}{I}{\mathbb{R}}$, continua en un intervalo abierto $I$, posee un cero en dicho intervalo (por ejemplo, si para $a$, $b$ números reales tales que $a < b$, hemos constatado que $f(a)f(b) < 0$, con lo cual se garantizará, por el teorema del valor intermedio de las funciones continuas, la existencia de un cero $\overline{x}$ entre $a$ y $b$), es útil conocer métodos para el cálculo aproximado de los ceros.

\begin{enumerate}
\item [a)]\textit{Método de dicotomía:} Si $f(a)f(b) < 0$, ponemos $c_1 = (a + b)/2$ y $c_1$ es ya una aproximación del cero buscado, y el error de aproximación satisfará la desigualdad
$|\overline{x} - c_1| < (b - a)/2$. Para $f(c_1)$, tenemos tres posibilidades:
\begin{enumerate}
\item [1.]$f(c_1) = 0$, con lo cual $\overline{x} = c_1$.
\item [2.]$f(c_1)f(a) < 0$. Se aplica la misma idea y se pone $c_2 = (c_1 + a)/2$, el punto intermedio entre $a$ y $c_1$, y es una nueva aproximación de $\overline{x}$, y esta vez el error de aproximación satisface la desigualdad $|\overline{x} - c_2| < (b - a)/2^2$.
\item [3.] $f(c_1)f(b) < 0$. Se aplica la misma idea, pero se pone $c_2 = (c_1 + b)/2$.
\end{enumerate}
Y, así sucesivamente, se obtienen aproximaciones $c_1$, $c_2$, $c_3$, etcétera, de $\overline{x}$; en el paso $n$, para la aproximación $c_n$ del cero $\overline{x}$, el error de aproximación satisface la desigualdad $|\overline x - c_n| < (b - a)/2^n$.

Se puede, entonces, calcular el número de pasos necesarios para lograr una aproximación con un error menor a un valor $\epsilon > 0$ pequeño y predeterminado. Si, por ejemplo, se desea obtener una aproximación con $k$ cifras decimales exactas, bastará resolver la desigualdad
\begin{equation}
\label{eq:dc001}
\frac{b - a}{2^n} < \frac{10^{-k}}{2}
\end{equation}
para obtener el número $n$ de pasos necesarios.

\begin{exemplo}[Solución]{Calcule $\sqrt{2}$ con dos cifras decimales exactas mediante el método de dicotomía.}
El número $\sqrt{2}$ es un cero de la función $\funcjc{f}{[1,2]}{\mathbb{R}}$ definida por $f(x) = x^2 - 2$. Si en el $n$-ésimo paso trabajamos con el intervalo $[a_n,b_n]$, poniendo $a_1 = 1$, $b_1 = 2$, $c_1 = (a_1 + b_1)/2$, $c_n = (a_n + b_n)/2$, y teniendo en cuenta que la inecuación~(\ref{eq:dc001}) en este caso es
\[
\frac{2 - 1}{2^n} < \frac{0.01}{2},
\]
se tiene que $n = 7$ es suficiente. Los cálculos se resumen en el cuadro siguiente:
\[
\begin{array}{|c|l|l|l|l|l|l|}\hline
n &
\multicolumn{1}{c|}{a_n} & \multicolumn{1}{c|}{b_n} & \multicolumn{1}{c|}{c_n} &
\multicolumn{1}{c|}{f(a_n)} & \multicolumn{1}{c|}{f(b_n)} & \multicolumn{1}{c|}{f(c_n)} \\ \hline
1	& 1	& 2	& 1.5	& -1	& 2	& 0.25 \\
2	& 1	& 1.5	& 1.25	& -1	& 0.25	 & -0.437\,5 \\
3	& 1.25	& 1.5	& 1.375	& -0.437\,5	& 0.25	& -0.109\,375 \\
4	& 1.375	& 1.5	& 1.437\,5	& -0.109\,375	& 0.25	& 0.066\,406\,25 \\
5	& 1.375	& 1.437\,5	& 1.406\,25	& -0.109\,375	& 0.066\,406\,25	& -0.022\,460\,938 \\
6	& 1.406\,25	& 1.4375	& 1.421\,875	& -0.022\,460\,938	& 0.066\,406\,25	& 0.021\,728\,516 \\
7	& 1.406\,25	& 1.421875	& 1.414\,0625	& -0.022\,460\,938	& 0.021\,728\,516	& -0.000\,427\,246 \\ \hline
\end{array}
\]
Se ve, entonces, que la aproximación buscada con dos decimales exactos es $c_7 = 1.41$.
\end{exemplo}

\item [b)] \textit{Método de Newton:} Cuando la función $f$ es, además, derivable, es más efectivo y rápido el método de \textit{Newton}, siempre y cuando la derivada tome valores distintos de $0$ en $I$. La idea de este método es partir de una aproximación cualquiera $x_0$ del cero $\overline x$ y obtener una nueva aproximación $x_1$, reemplazando el gráfico de $f$ con la tangente a él en el punto de coordenadas $(x_0,f(x_0))$, como se puede observar en el siguiente dibujo:
\begin{center}
\def\F{x dup mul 1.75 mul 0.125 sub}%
\psset{unit=8cm}
\begin{pspicture}(-0.1,-0.1)(0.8,0.65)
   \SpecialCoor

   \psaxes{->}(0,0)(-0.1,-0.15)(0.75,0.6)%
   \uput[-90](0.75,0){$x$}%
   \uput[0](0,0.6){$y$}%

   \psplot[linewidth=3\pslinewidth]{0}{0.6}{\F}%
   \psplot[linecolor=red,linewidth=1.25\pslinewidth]{0.3}{0.6}{1.925 x mul 0.654375 sub}
   \psline[linestyle=dashed,linecolor=blue]%
      (! 0.55 /x 0.55 def \F)(0.55,0)%

   \uput[180](! 0.55 /x 0.55 def \F){$P_0$}%
   \uput[-90](0.55,0){$Q_0(x_0)$}%
   \uput[0](! 0.65 /x 0.65 def \F){$y = f(x)$}%
   \uput[90](0.267261,0){$\overline x$}%
   \uput[-90](0.339935,0){$Q_1(x_1)$}%
   \uput[180](! 0.339935 /x 0.339935 def \F){$P_1$}%
   \psline[linestyle=dashed,linecolor=blue]%
      (! 0.339935 /x 0.339935 def \F)(0.339935,0)%

\end{pspicture}
\end{center}

En el triángulo rectángulo $Q_1Q_0P_0$, tenemos que
\[
\tan\angle P_0Q_1Q_0 = \frac{Q_0P_0}{Q_1Q_0} = \frac{f(x_0)}{x_0 - x_1}.
\]
Como $\tan\angle P_0Q_1Q_0 = f'(x_0)$, tendremos que
\[
x_1 = x_0 - \frac{f(x_0)}{f'(x_0)}.
\]

Si volvemos a aplicar la misma, pero ahora pariendo de $x_1$ en lugar de $x_2$, obtendremos una nueva aproximación $x_2$ de $\overline x$ mediante la fórmula
\[
x_2 = x_1 - \frac{f(x_1)}{f'(x_1)},
\]
y así sucesivamente. Si notamos $y_n = f(x_n)$ y $y'_n = f'(x_n)$ para $n\in\mathbb{N}$, tendremos que
\[
x_n = x_{n-1} - \frac{y_{n-1}}{y'_{n-1}},
\]
que es una fórmula iterativa de fácil aplicación.

Como criterio para suspender el cálculo, se puede exigir que para un valor $\epsilon > 0$, pequeño y predeterminado, se tenga $|y_n| < \epsilon$.

Apliquemos el método de Newton para volver a calcular una aproximación de $\sqrt{2}$, con $x_0 = 1$, $f(x) = x^2 - 2$ y $f'(x) = 2x$. Los cálculos se resumen en el cuadro siguiente:
\[
\begin{array}{|c|l|l|l|}\hline
n & \multicolumn{1}{c|}{x_n} & \multicolumn{1}{c|}{f(x_n)} & \multicolumn{1}{c|}{f'(x_n)} \\ \hline
0	& 1	& -1	& 2 \\
1	& 1.5	& 0.25	& 3 \\
2	& 1.416\,666\,667	& 0.006\,944\,444	& 2.833\,333\,333 \\
3	& 1.414\,215\,686	& 0.000\,006\,007	& 2.828\,431\,373 \\
4	& 1.414\,213\,562	& 4.5\times 10^{-12}	& 2.828\,427\,125 \\
5	& 1.414\,213\,562	& 0	& 2.828\,427\,125 \\ \hline
\end{array}
\]
La aproximación para el cero es $1.414\,213\,562$.

Vemos que en $4$ pasos se tiene ya una aproximación con $8$ cifras decimales que no se va a modificar con más iteraciones.
\end{enumerate}
\section{Ejercicios}
\begingroup
\small
\begin{multicols}{2}
\begin{enumerate}
\item ¿En cuántos pasos el método de dicotomía le permitirá calcular el número dado con $3$ cifras decimales exactas?
    \begin{enumerate}
    \item $\sqrt{3}$.
    \item $1 + \sqrt{2}$.
    \item $\sqrt{2} - 1$.
    \item $\sqrt{2} + \sqrt{3}$.
    \item La raíz de $x^3 + 5 = 0$.
    \item La raíz de $\sinh x + 2 = 0$.
    \end{enumerate}

\item Compare los métodos de dicotomía y de Newton para los cálculos de aproximaciones de los números del ejercicio precedente.
\end{enumerate}
\end{multicols}
\endgroup
%\subsection{Ejercicios}
%
%\section{Ejercicios}
%\begingroup\small
%Suponga conocidos las definiciones de las funciones que se dan en la siguiente tabla de derivadas.
%Con ayuda de esta tabla y de las propiedades algebraicas de las derivadas, calcule la derivada de
%$f(x)$.
%
%Con los conceptos desarrollados hasta este momento, no nos es posible definir satisfactoriamente
%las funciones exponencial y logaritmo. Sin embargo, para que aquellos lectores que conocen estas
%funciones, las hemos incluido en la tabla. Debe recordarse, entonces, que:
%\[
%a^x = e^{x\ln a} \yjc \log_a x = \frac{\ln x}{\ln a}.
%\]
%\[
%\begin{array}{|l|l|l||l|l|c|}\hline
%\multicolumn{1}{|c|}{f(x)} &
%\multicolumn{1}{c|}{f'(x)} &
%\multicolumn{1}{c||}{\Dm(f')} &
%\multicolumn{1}{c|}{f(x)} &
%\multicolumn{1}{c|}{f'(x)} &
%\multicolumn{1}{c|}{\Dm(f')} \\ \hline
%c \in \Rbb & 0 & \Rbb & x^\alpha, \ \alpha \geq 1 & \alpha x^{\alpha - 1} & \Rbb \\
%x^\alpha, \ \alpha < 1 & \alpha x^{\alpha - 1} & \Rbb - \{0\}
%\end{array}
%\]
%\begin{multicols}{2}
%\begin{enumerate}[leftmargin=*]
%\item
%
%\end{enumerate}
%\end{multicols}
%\endgroup
