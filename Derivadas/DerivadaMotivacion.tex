\chapter{La derivada: su motivación}

Veamos algunos problemas que llevan a un mismo concepto: el de derivada.

\section{La recta tangente a una curva}
En la segunda sección del capítulo de límites, estudiamos
el concepto de recta tangente a una curva. Vimos que si ésta es el gráfico de una función
$\funcjc{g}{D\subset\mathbb{R}}{\mathbb{R}}$ y $P$ es un punto de la curva con coordenadas
$(a,g(a))$, para conocer la ecuación de la recta tangente a la curva en el punto $P$, que es
\[
y = m(x - a) + g(a),
\]
bastaba con calcular su pendiente $m$.

Para ello, aproximamos $m$ con la pendiente de una recta que pasa por $P$ y por otro punto $Q$ de
la recta de coordenadas $(x,g(x))$, que la notamos $m_x$, y que es igual a:
\[
m_x = \frac{g(x) - g(a)}{x - a}.
\]

Para obtener $m$, la idea era ``acercar'' $Q$ a $P$, esperando que, al hacerlo, $m_x$ se acercara a
$m$. En el ejemplo estudiado, vimos que eso era así y que, de hecho:
\[
m = \limjc{m_x}{x}{a} = \limjc{\frac{g(x) - g(a)}{x - a}}{x}{a}.
\]

En efecto, teníamos que $g$ está definida por $g(x) = 3x^2$ y que $a = 2$. Por lo tanto, $g(a) =
g(2) = 12$ y:
\begin{align*}
m_x &= \frac{3x^2 - 12}{x - 2} \\[4pt]
m &= \limjc{\frac{3x^2 - 12}{x - 2}}{x}{2} = 12.
\end{align*}

La recta $l$ de ecuación
\[
y = 12 + 12(x - 2) = 12x + 10
\]
es, efectivamente, la tangente al gráfico de $g$ que es una parábola $p$. Más aún, la recta $l$ es
la única recta que pasa por el punto de $P$ de coordenadas $(2,12)$ y tiene un solo punto en común
con la parábola $p$ de ecuación
\[
y = g(x) = 3x^2,
\]
que es, justamente, el punto $P$.

Por otra parte, si exceptuamos el punto $P$, toda la parábola está ``de un solo lado'' de la recta
$l$, y ``ninguna otra recta se interpone en el espacio entre'' la recta $l$ y la parábola $p$, como
lo exige la definición de Euclides.

Mostraremos más tarde que esta afirmación es verdadera, cuando estudiemos otras definiciones de
tangencia y mostremos algunas propiedades notables de la recta tangente. Por ejemplo, probaremos
que de entre todas las rectas que pasan por el punto de tangencia, la recta tangente es la que
mejor aproxima a la curva.

Ahora bien, del ejemplo citado resaltaremos el hecho de que la pendiente $m$ de la recta tangente
al gráfico de $g$ en el punto $(a,g(a))$ está dada por
\[
m = \limjc{\frac{g(x) - g(a)}{x - a}}{x}{a}.
\]

Esta propiedad la cumple la pendiente de la recta tangente en un punto dado de una gran variedad de
curvas que son el gráfico de funciones (como son los polinomios, funciones racionales,
trigonométricas, logarítmicas, exponenciales, etcétera). Esto nos motiva a cambiar la tradicional
definición de recta tangente a una curva por la siguiente:

\begin{defical}[Recta tangente]
Dada una función real $f$ definida en un intervalo abierto $I$ y $a \in I$, diremos que el gráfico
de $f$ tiene una recta tangente en el punto $(a, f(a))$ si existe
\[
m:= \limjc{\frac{f(x) - f(a)}{x - a}}{x}{a}.
\]
En este caso, la ecuación de la recta tangente es:
\[
y - f(a) = m(x - a).
\]
\end{defical}

\begin{exemplo}[Solución]{%
Hallar, si existe, la ecuación de la recta tangente al gráfico de la hipérbola de ecuación
$y = 2 - \frac{1}{x}$ en el punto de coordenadas $(1,1)$.
}%
Si ponemos $a = 1$ y definimos $f$ por
\[
f(x) = 2 - \frac{1}{x},
\]
entonces $f(a) = f(1) = 1.$ Además:
\begin{align*}
m &= \limjc{\frac{f(x) - f(1)}{x - 1}}{x}{1} =%
    \limjc{\displaystyle\frac{\left(2 - \frac{1}{x}\right) - \left(2 - 1\right)}{x - 1}}%
    {x}{1} \\[6pt]
  &= \limjc{\displaystyle\frac{-\frac{1}{x} + 1}{x - 1}}{x}{1} =
  \limjc{\frac{x - 1}{x(x - 1)}}{x}{1} \\[4pt]
  &= \limjc{\frac{1}{x}}{x}{1} = 1.
\end{align*}
Como el límite existe, la recta tangente tendrá por ecuación a
\[
y -1 = 1(x - 1);
\]
es decir, la ecuación de la recta tangente a la hipérbola en el punto de coordenadas $(1,1)$ es $y
= x$.

Una de las aplicaciones más valiosas del concepto de derivada que vamos a estudiar es el proveer
una herramienta para realizar un dibujo aproximado del gráfico de una función. Más adelante,
podremos constatar que el gráfico de $f$ y de la tangente al gráfico de $f$ en el punto de
coordenadas $(1,1)$ es el siguiente:
\begin{center}
\begin{pspicture}(-3,-1)(5,4)
\psset{unit=0.5cm,plotpoints=250}%
\def\pshlabel#1{\footnotesize #1}%
\def\psvlabel#1{\footnotesize #1}%

\psaxes[ticks=none,labels=none]{->}(0,0)(-6,-2)(9.5,7.5)%
\uput[0](0,7.5){$y$}%
\uput[-90](9.5,0){$x$}%

\psplot[linewidth=1.5\pslinewidth]{-6}{-0.2}{2 1 x div sub}%
\psplot[linewidth=1.5\pslinewidth]{0.25}{8}{2 1 x div sub}%
\rput[l](-5,3.5){$\displaystyle y = 2 - \frac{1}{x}$}

\psplot[linestyle=dashed]{-6}{8}{2}%

\psplot{-1.5}{5}{x}%
\rput[l](5.1,5){$y = x$}

\uput[0](0,2.4){$2$}%

\psline[linestyle=dashed,linecolor=gray]%
    (1,0)(1,1)(0,1)%
\psdot[dotscale=0.8](1,1)%
\uput[-90](1,0){$1$}%
\uput[180](0,1){$1$}%

\end{pspicture}
\end{center}
\end{exemplo}

\begin{exemplo}[Solución]{%
Hallar la ecuación de la recta tangente al gráfico de la cúbica de ecuación $y = x^3$ en
el punto de coordenadas $(-1,-1)$.
}%
Si ponemos $a = -1$ y definimos $f(x) = x^3$, entonces $f(-1) = -1$ y
\begin{align*}
m   &= \limjc{\frac{f(x) - f(-1)}{x - (-1)}}{x}{-1} =
        \limjc{\frac{x^3 + 1}{x + 1}}{x}{-1} \\[4pt]
    &= \limjc{\frac{(x+1)(x^2 - x + 1)}{x + 1}}{x}{-1} =
        \limjc{x^2 - x + 1}{x}{-1} = 3.
\end{align*}
Por lo tanto, la recta buscada tiene como ecuación la siguiente:
\[
y - (-1) = 3(x - (-1)) = 3x + 3,
\]
de donde, la ecuación buscada es:
\[
y = 3x + 2.
\]

El gráfico de $f$ y de la tangente en $(-1,-1)$ es el siguiente:
\begin{center}
\begin{pspicture}(-2,-2)(4.2,5)

\SpecialCoor
\psset{yunit=0.5cm,plotpoints=250}%
\def\pshlabel#1{\footnotesize #1}%
\def\psvlabel#1{\footnotesize #1}%

\psaxes[ticks=none,labels=none]{->}(0,0)(-2,-4)(2.75,9.5)%
\uput[0](0,9.5){$y$}%
\uput[-90](2.75,0){$x$}%

\begingroup
    \psset{linewidth=1.5\pslinewidth}%
    \psplot{-1.5}{2.1}{x 3 exp}%
    \rput[l](! -1.4 1.5 3 exp neg){$y= x^3$}%

    \psplot{-1.5}{2.5}{x 3 mul 2 add}%
    \rput[l](2.5,8.3){$y = 3x + 2$}%
\endgroup

\begingroup
    \psset{dotscale=0.8,linestyle=dashed}%
    \psline[linecolor=gray]%
        (-1,0)(-1,-1)(0,-1)%
    \psdot(-1,-1)%
    \uput[90](-1,0){$-1$}%
    \uput[0](0,-1){$-1$}%

    \psline[linecolor=gray]%
        (2,0)(2,8)(0,8)%
    \psdot(2,8)%
\endgroup
\end{pspicture}
\end{center}
\end{exemplo}

Observemos en el dibujo que la recta tangente a la curva no tiene con ésta un único punto en común
sino dos. El un punto, que es el de tangencia, tiene coordenadas $(-1,-1)$. Podemos averiguar las
coordenadas del otro punto resolviendo el sistema de ecuaciones:
\[
\left\{
\begin{matrix}
y & = & x^3 \\
y & = & 3x + 2
\end{matrix}
\right.
\]

Al igualar las ecuaciones, obtenemos que
\[
x^3 = 3x + 2,
\]
lo que equivale a
\[
x^3 - 3x - 2 = 0.
\]

Esta ecuación puede tener hasta tres raíces reales. Sabemos que una de ellas es $-1$, porque la
tangente y la curva se ``encuentran'' en el punto de coordenadas $(-1,-1)$. Entonces, el polinomio
cúbico $x^3 - 3x - 2$ tiene como factor $(x + 1)$.

Por lo tanto, podemos obtener el otro factor si dividimos el cúbico por $(x + 1)$. Al hacerlo,
obtenemos que:
\[
x^3 - 3x - 2 = (x + 1)(x^2 - x - 2) = (x+1)(x + 1)(x - 2).
\]
Entonces, el polinomio tiene las tres raíces son reales e iguales a $-1$, de multiplicidad $2$, y
$2$.

Entonces, ya estamos seguros que la curva y la recta tangente se encuentran exactamente en dos
puntos: $(-1,1)$ y $(2,f(2)) = (2,8)$.

Pero, ?`no se suele decir que la tangente y la curva solo deben tener un punto en común? Sí, eso
suele decirse, sin embargo, tal definición no es del todo precisa, pues, en las ``cercanías'' de
$(-1,-1)$ la tangente y la curva tienen, efectivamente, un comportamiento de tangente en el sentido
tradicional.

Por esto, la propiedad de tangencia es de carácter ``local''; es decir, no importa el
``comportamiento'' de la curva y de la recta ``lejos'' del punto de tangencia sino en sus
``cercanías''.


\section{?`Cómo medir el cambio?}

Tengamos en cuenta que la Matemática desarrollada hasta antes del Renacimiento estudiaba más bien
conceptos ``estáticos'': describir formas geométricas que no cambian, calcular áreas de figuras
simples, llevar cuentas, etcétera. La geometría de Euclides, la aritmética de Diofanto y el álgebra
desarrollada por Al-Khowarizmi fueron instrumentos idóneos para responder a esta visión estática.

Pero cuando la humanidad intenta describir el movimiento que se produce en gran cantidad de
fenómenos estudiados, se hace necesaria la creación de nuevos conceptos matemáticos que permitan
responder adecuadamente a estos requerimientos.

Dado que el movimiento es percibido como el cambio de la posición de un cuerpo en el tiempo, una de
las preguntas más importantes que la matemática enfrentó fue ``?`cómo medir el cambio?''.

Veamos en esta sección cómo la respuesta a esta pregunta nos lleva también al concepto de derivada.

\subsection{?`Cuánto cuesta producir autos?}

\begingroup
\itshape El administrador de una fábrica de autos desea tener información adecuada respecto a los
costos de producción de la fábrica en función del número de vehículos en un mes dado. Así, desearía
saber cómo varían estos costos al variar la producción. ?`Qué información requiere y cómo
presentarla? ?`Cómo medir la variación o cambio de los costos de la producción? ?`Cómo relacionar la
producción con los costos y la dependencia entre la variación de la producción y la de los costos?
\endgroup

Para responder a éstas y a otras preguntas relacionadas, se requerirá de algunos conceptos
matemáticos como son el de ``variable numérica'', ``función real'', ``incremento absoluto'',
``incremento relativo'', ``razón de cambio'', ``derivada'', etcétera, los mismos que expondremos a
lo largo de esta sección.

\subsection{Las variables representan magnitudes}
Cuando estudiamos determinados fenómenos físicos, sociales o económicos ---como el problema
planteado en la sección anterior---, identificamos ciertas magnitudes que consideramos importantes
para dicho estudio.

Por ejemplo, al estudiar un circuito eléctrico, la tensión $V$ voltios y la intensidad $I$ amperios
de la corriente eléctrica, y la resistencia $R$ ohmios de los dispositivos instalados son
importantes. Cuando se monitorea el trabajo de una represa, será de interés la cota del nivel de
agua $h$ metros y la energía $E\MWh$ almacenada. Al dirigir una fábrica de autos se deseará conocer
el costo total $C$ miles de dólares y el correspondiente costo unitario $CU$ miles de dólares al
producir $x$ vehículos, etcétera.

Establecida la unidad de medida que corresponda y luego de realizar las mediciones o cálculos
necesarios, se determinará, para cada magnitud, un determinado valor numérico en un momento dado.
El símbolo que represente a dicho valor numérico, el cual oscila entre un valor mínimo y un valor
máximo, dados por el fenómeno concreto que se está estudiando, se llama \emph{variable}. En los
ejemplos dados, tenemos que $V$, $I$, $R$, $h$, $E$, $C$, $CU$ y $x$ son variables.

Por ejemplo, la fábrica de autos producirá $x$ automóviles por año y $x$ no puede ser menor a
1\,000 porque no sería rentable producir menos, pero tampoco producirá más de 10\,000, dado que la
capacidad instalada de la fábrica no lo permite. Matemáticamente, expresamos estas dos
restricciones ``haciendo que'' la variable $x$ pertenezca al intervalo $[1\,000,10\,000]$.

\subsection{La función como modelo de la dependencia entre magnitudes}
Hemos visto el concepto de función. Ilustremos con tres ejemplos su uso como modelo de la
dependencia entre magnitudes.

\subsubsection{Costo de producción en una fábrica de autos}
Como el costo $C$ de producción de los autos, en miles de dólares, depende del número $x$ de
vehículos producidos, esta dependencia es modelizada con una función
$\funcjc{f}{\Dm(f)\subset\mathbb{R}}{\mathbb{R}}$, donde:
\[
C=f(x).
\]
Como solo nos interesan los valores de $x$ entre $1\,000$ y $10\,000$, el dominio de $f$ será
$\Dm(f)=[1\,000,10\,000]$.

Luego de los estudios correspondientes, los administradores de la empresa determinan que:
\[
C=f(x)=2\,000+11x -0.000\,12x^{2}.
\]
Por lo tanto, el costo unitario $C_U$, es decir el costo promedio de un auto es:
\[
C_U = \frac{f(x)}{x}=\frac{2\,000}{x}+11-0.000\,12x.
\]

A manera de ejemplo, en la siguiente tabla, se muestran los valores de $C$ y del costo unitario
$C_U$ que cuesta producir cada auto si se producen $x$ autos, dados en miles de dólares,
correspondientes al año anterior:

\begingroup
\begin{center}
{\renewcommand\arraystretch{1.5}%
\setlength\extrarowheight{1pt}
\begin{tabular}{|c |c |c|}
\hline
$x$ & $C=f(x)$ & $C_U=\frac{f(x)}{x}$  \\
\hline
1\,000  &  12\,880 &  12.880 \\[-2pt]
\hline
2\,000 & 23\,520 & 11.760 \\
\hline
3\,000 & 33\,920  & 11.307 \\
\hline
5\,000 & 54\,000 & 10.800 \\
\hline
6\,000 & 63\,680 & 10.613 \\
\hline
7\,000 & 73\,120 & 10.446 \\
\hline
8\,000 & 82\,320 & 10.290 \\
\hline
9\,000 & 91\,280 & 10.142 \\
\hline
10\,000 & 100\,000 & 10.000 \\
\hline
\end{tabular}}
\end{center}

\subsubsection{Energía potencial en una represa}
La \emph{energía potencial} $E$, medida en $\MWh$, que está almacenada en una represa depende de la
\emph{cota} $h$, medida en $\metros$, se define como la altura sobre el nivel del mar del espejo de
agua en la represa.

Supongamos que, en una determinada represa, la cota puede estar entre $2\,900$ y $2\,950$ metros,
existe una función $\funcjc{g}{\Dm(g)}{\mathbb{R}}$, con dominio $\Dm(g)=[2\,900,2\,950]$, tal
$g(h)$ es la energía almacenada $E$ cuando la cota es $h$. Es decir:
\[
E=g(h).
\]

\subsubsection{El diámetro y el volumen de un globo}
Un globo, en la parte superior, tiene forma de un hemisferio achatado; en su parte inferior, es un
hemisferio alargado que culmina con una abertura cilíndrica.

Los fabricantes han determinado que el volumen $V$, medido en $\decimetros^3$, de la cámara de gas
del globo depende de su diámetro $d$, medido en $\decimetros$. Entonces, existe una función
$\funcjc{\varphi}{\Dm(\varphi)}{\mathbb{R}}$ tal que:
\[
V = \varphi(d)=\frac{1}{2\,000}d^{3}.
\]

El diseño de los globos de esta fábrica establece que el globo se eleva si el diámetro mide, al
menos, $6\metros$. Además, el mayor diámetro al que el globo puede ser inflado es de $12\metros$.
Por ello, el dominio de $\varphi$ es $\Dm(\varphi)=[60,120]$.

Si el helio costara $40$ centavos de dólar por cada $\metros^3$, podríamos establecer el gasto $G$
en dólares que deberíamos hacer por concepto de gas en función del diámetro del globo. Así, existe
una función $\funcjc{\psi}{\Dm(\psi)}{\mathbb{R}}$ tal que
\[
G=\psi(d) = 0.40\varphi(d)=0.000\,2d^3.
\]

En la siguiente tabla, se muestran valores del volumen del globo y del gasto en el que hay que
incurrir para inflarlo para algunos valores del diámetro:
\begin{center}
\begingroup
\setlength\extrarowheight{4pt}
\begin{tabular}{|c |c |r @{.} l|}
\hline
$d$ & $V=\varphi(d)$ & \multicolumn{2}{|c|}{$G=\psi(d)$}  \\
{\small$\decimetros$} & $\decimetros^3$ & \multicolumn{2}{|c|}{dólares} \\
\hline
60  &  108.0 &  43&20 \\
\hline
70 & 171.5 & 68&60 \\
\hline
80 & 256.0  & 102&40 \\
\hline
90 & 364.5 & 145&50 \\
\hline
100 & 500.0 & 200&00 \\
\hline
110 & 665.5 & 260&20 \\
\hline
120 & 864.0 & 345&60 \\
\hline
\end{tabular}
\endgroup
\end{center}

\subsection{La variaciones absoluta y relativa como medidas del cambio}
Cuando una variable, digamos $x$, toma diversos valores luego de haber tomado un ``valor inicial''
$x_0$, unos dirán que varió mucho, otros lo contrario. Esto es algo subjetivo. Por ello, se hace
necesario establecer medidas objetivas del cambio producido. Para ello, nos serviremos de los
conceptos de \emph{variación absoluta} y \emph{relativa}.

\subsubsection{La variación absoluta o incremento}
En el ejemplo del globo, si trabajáramos con un diámetro de $80\dm$, veríamos que el volumen de gas
necesario sería de $256\metros^3$ y que el costo del gas sería de $102.40$ dólares. Si infláramos
algo más el globo, digamos hasta un diámetro $d$ decímetros, la \emph{variación} o \emph{cambio}
del diámetro se define como la diferencia $d-80$ decímetros, a la que llamaremos \emph{variación
absoluta} o \emph{incremento} del diámetro, la notaremos con $\Delta d$ y será leída ``delta $d$''
y está dada también en decímetros. Así $\Delta d = d-80$.

El incremento puede ser positivo o negativo. Por supuesto que $\Delta d > 0$ significa que el
diámetro creció mientras que $\Delta d < 0$ significa lo contrario. En este caso, a $\Delta d$
también se le llama \emph{decremento}.

Al variar $d$, naturalmente, varían también el volumen del gas $V = \varphi(d)\dm^3$ cúbicos y el
costo que hay que pagar por este gas $G = \psi(d)$ dólares. La \emph{variación absoluta} del
volumen será:
\[
\Delta V = V - 256 = \varphi(d) - \varphi(80)\dm^3
\]
y la \emph{variación absoluta} del costo será:
\[
\Delta G = G - 102.40 = \psi(d) - \psi(80)\ \text{dólares}.
\]

\begin{exemplo}[]{}
Consideremos la situación de la fábrica de autos una vez más. Si su producción, en un mes dado,
fuera de, digamos, $5\,000$ autos y, en el mes siguiente, ésta pasara a ser de $x$ autos, entonces
la variación absoluta de la producción estaría dada por:
\[
\Delta x=x-5\,000.
\]
En este caso, la variación absoluta de los costos, que pasarían de $54\,000$ miles de dólares a $C$
miles de dólares sería igual a:
\[
\Delta C=C-54\,000.
\]

Como
\[
C = f(x)= 2\,000+11x-0.000\,12x^2,
\]
se tendrá que
\[
\Delta C=f(x)-54\,000 =-52\,000 + 11x - 0.000\,12x^2.
\]

Análogamente, la variación absoluta del costo unitario, que pasó de $10.8$ miles de dólares a $C_U$
miles de dólares, estaría dado por:
\[
\Delta C_U=C_U-10.8.
\]
Puesto que:
\[
C_U = \frac{f(x)}{x}= \frac{2\,000}{x}+11-0.000\,12x,
\]
se tendría que:
\[
\Delta C_U = \frac{f(x)}{x}-10.8= \frac{2\,000}{x}+0.2 -0.000\,12x.
\]

Por ejemplo, si $x=6\,000$ autos, entonces:
\[
f(6\,000) \approx 63\,680
\]
y
\[
\frac{f(6\,000)}{6\,000}\approx 10.613.
\]
Además, tendremos que:
\begin{align*}
\Delta x & = 6\,000-5\,000=1\,000,\\
\Delta C & \approx 63\,680-54\,000= 9\,680, \\
\Delta C_U & \approx 10.613-10.8 = -0.187.
\end{align*}
Es decir que, si la producción aumentara en $1\,000$ autos, los costos subirían en $9.68$ millones
de dólares, mientras que ¡el costo unitario bajaría en $187$ dólares por auto!
\end{exemplo}

\subsubsection{La variación relativa}
Si el precio del kilogramo de arroz se incrementara de un precio inicial de $p_{0} = 0.40$ dólares
a un precio de $p = 0.70$ dólares, el incremento del precio se mediría por su variación absoluta:
\[
\Delta p = p - p_{0} = 0.70 - 0.40 = 0.30\ \text{dólares}.
\]

Si el precio de una llanta pasara de $p_{0} = 20.00$ dólares a $p = 20.30$ dólares, la variación
absoluta sería también de $0.30$ dólares.

Obviamente, la importancia del alza de $0.30$ dólares al precio del kilogramo de arroz no es la
misma que la del alza al precio de una llanta. En el primer caso, el precio casi se ha duplicado,
mientras que, en el caso de la llanta, el alza es insignificante.

Para medir este fenómeno, se utiliza el concepto de \emph{variación relativa} que es el cociente de
la variación absoluta dividida por el valor inicial que tiene la variable. Así, en el ejemplo del
arroz, la variación relativa es:
\[
\frac{\Delta p}{p_{0}} = \frac{0.30}{0.40} = 0.75.
\]

La variación relativa suele también expresarse como un porcentaje al que se lo denomina
\emph{variación porcentual}.

Teniendo en cuenta la identidad $\%=\frac{1}{100}$ (o 100\% = 1), podemos decir que la variación
porcentual del precio del kilogramo de arroz es:
\[
\frac{\Delta p}{p_{0}}\times 100 \% = 0.75\times 100\% = 75\%
\]
y que la variación porcentual del precio de la llanta es:
\[
\frac{\Delta p}{p_{0}}\times 100\% = 0.015\times 100\% = 1.5\%.
\]

\begin{exemplo}[]{}
En el caso de la fábrica de autos, considerando las mismas variaciones absolutas de autos,
tendremos que la variación relativa de la producción es:
\[
\frac{\Delta x}{5\,000}= \frac{x-5\,000}{5\,000},
\]
la variación relativa del costo:
\[
\frac{\Delta C}{54\,000}= \frac{C-54\,000}{54\,000}
\]
y la del costo unitario:
\[
\frac{\Delta C_U}{10.8}= \frac{C_U-10.8}{10.8} =
    \frac{\displaystyle\frac{f(x)}{x}-10.8}{10.8}.
\]

Si $x=6\,000$ autos, se tendrá que la variación porcentual de la producción sería:
\[
\frac{\Delta x}{5\,000} \times 100\% = \frac{1\,000}{5\,000}\times 100\% = 20\%,
\]
la variación porcentual del costo sería:
\[
\frac{\Delta C}{54\,000}\times 100\%= \frac{9\,680}{54\,000}\times 100\% \approx 18\%
\]
y la del costo unitario sería:
\[
\frac{\Delta C_U}{10.8} \times 100\% = \frac{-0.187}{10.8}\times 100\% \approx -1.7\%.
\]
En otras palabras, al subir la producción en un $20\%$, los costos aumentan en un $18\%$, mientras
que el costo unitario baja en $1.7\%$.
\end{exemplo}

\section{Razón de cambio, elasticidad y magnitudes marginales}
Vimos que la dependencia entre las dos variables numéricas que describen un fenómeno, digamos una
variable $y$ que depende de otra variable $x$, puede ser descrita mediante una función
$\funcjc{f}{I}{\mathbb{R}}$, donde $I$ es un intervalo en el cual toma sus valores la variable
independiente $x$.

Por ejemplo, $y$ miles de dólares representa el costo de producir $x$ autos en una fábrica durante
un mes, el intervalo $I$ es, digamos, $[1\,000, 10\,000]$ y la función $f$ está definida por:
\[
f(x) = 2\,000 + 10x.
\]

Ahora bien, si, en un mes dado, la producción fue de $x_0 = 4\,800$ autos, los costos de producción
ese mes fueron de $y_0$ miles de dólares. Entonces:
\[
y_0 = f(x_0) = 2\,000 + 10x_0.
\]
Por lo tanto:
\[
y_0 = 2\,000 + 48\,000 = 50\,000.
\]

Supongamos que, en el siguiente mes, la producción se elevó a $x$ autos. Este cambio se puede
medir, según vimos, por la variación absoluta $\Delta x$, llamada también \emph{incremento} de $x$,
y por la variación relativa $\frac{\Delta x}{x_0}$ de la variable $x$. En este caso:
\begin{align}
\label{eq:dm001}
\Delta x &= x - x_0 = x - 4\,800. \\
\label{eq:dm002}
\frac{\Delta x}{x_0} &= \frac{x - x_0}{x_0} = \frac{x - 4\,800}{4\,800}.
\end{align}

Obviamente, si $\Delta x > 0$, implica que la variable $x$ creció; en el caso contrario, decreció.
En la práctica, a un incremento negativo se le llama \emph{decremento}.

Por otra parte, como la dependencia del costo respecto del nivel de producción está descrito por la
función $f$, mediante la igualdad $y = f(x)$, la variable costo sufrirá también un cambio, descrito
por la variación absoluta o incremento $\Delta y$ y por la variación relativa de $y$, que es
$\frac{\Delta y}{y_0}$. Podemos calcular ambas variaciones:
\begin{align}
\Delta y &= y - y_0 = f(x) - f(x_0) = (2\,000 + 10x) - 50\,000 \nonumber \\
\label{eq:dm003}
\Delta y &= -48\,000 + 10x \\
\label{eq:dm004}
\frac{\Delta y}{y_0} &= \frac{-48\,000 + 10x}{50\,000}.
\end{align}

Ahora bien, como $y$ depende de $x$, y esta dependencia está descrita por $y = f(x)$, para un $x_0$
fijo dado, es de esperar que la dependencia del incremento $\Delta y$ respecto del incremento
$\Delta x$ pueda describirse fácilmente, al igual que la dependencia de la variación relativa o
porcentual $\frac{\Delta y}{y_0}$ respecto de $\frac{\Delta x}{x_0}$, la variación relativa o
porcentual de $x$. Veamos que sí es así.

¡Resulta que $\Delta y$ es directamente proporcional a $\Delta x$, así como $\frac{\Delta y}{y_0}$
es directamente proporcional a $\frac{\Delta x}{x_0}$! Es decir, existen constantes $\kappa$ y
$\eta$ tales que
\begin{align}
\Delta y &= \kappa\Delta x \label{eq:dm005}\\
\frac{\Delta y}{y_0} &= \eta\frac{\Delta x}{x_0}. \label{eq:dm006}
\end{align}

A la constante $\kappa$ se le llama \emph{razón de cambio} y a la constante $\eta$,
\emph{elasticidad} de $y$ respecto de $x$. Calculemos estas dos constantes. De (\ref{eq:dm001}),
(\ref{eq:dm003}) y (\ref{eq:dm005}) tenemos que:
\begin{equation}
\label{eq:dm007}
\kappa = \frac{\Delta y}{\Delta x} = \frac{-48\,000 + 10x}{x - 4\,800} = 10.
\end{equation}

De (\ref{eq:dm006}) y (\ref{eq:dm007}) obtenemos:
\begin{equation}
\label{eq:dm008}
\eta = \frac{\displaystyle\frac{\Delta y}{y_0}}{\displaystyle\frac{\Delta x}{x_0}}
    = \frac{x_0}{y_0}\frac{\Delta y}{\Delta x} = \frac{4\,800}{50\,000}\cdot 10 = 0.96.
\end{equation}

Es fácil verificar que esta sencilla dependencia de $\Delta y$ respecto de $\Delta x$ y de
$\frac{\Delta y}{y_0}$ respecto de $\frac{\Delta x}{x_0}$ se da siempre que $f$ sea un polinomio de
grado menor que o igual a $1$:

\begin{teocal}
Si $\funcjc{f}{I\subset\mathbb{R}}{\mathbb{R}}$ definida por $y = f(x) = mx + b$, y para todo
$x_0\in I$ y todo $x\in I$, si:
\[
y_0 = f(x_0), \quad \Delta x = x - x_0, \quad \Delta y = y - y_0 = f(x) - f(x_0),
\]
se tiene que:
\begin{equation}
\label{eq:dm009}
\Delta y = \kappa\Delta x, \quad \frac{\Delta y}{y_0} = \eta\frac{\Delta x}{x_0},
\end{equation}
donde
\begin{equation}
\label{eq:dm010}
\kappa = m \yjc \eta = \frac{x_0}{y_0}m.
\end{equation}
\end{teocal}

Recordemos que el gráfico de $f$ es una recta con pendiente $m$. Por lo tanto, la razón de cambio
$\kappa$ es igual a $m$. Naturalmente, ese caso es excepcional. En general, tenemos las siguientes
definiciones inspiradas en (\ref{eq:dm007}) y (\ref{eq:dm008}).

\begin{defical}[Razón de cambio]
Sean $\funcjc{f}{I\subset\mathbb{R}}{\mathbb{R}}$ y $\lajc{x}{y = f(x)}$. Para $x_0 \in I$ y $x\in
I$ tales que $x\neq x_0$, si
\[
y_0 = f(x_0), \quad \Delta x = x - x_0, \quad \Delta y = y - y_0,
\]
la \emph{razón de cambio de $y$ respecto de $x$ en el intervalo de extremos $x_0$ y $x$} es
\begin{equation}
\label{eq:dm011}
f'(x_0;x) = \frac{\Delta y}{\Delta x} = \frac{f(x) - f(x_0)}{x - x_0}.
\end{equation}
\end{defical}

Obviamente, $\Delta y$ no es, en general, directamente proporcional a $\Delta x$, puesto que, si bien
\begin{equation}
\label{eq:dm012}
\Delta y = f'(x_0;x)\Delta x,
\end{equation}
se tiene que $f'(x_0;x)$ no es constante la mayoría de las veces, sino que depende de $x_0$ y de $x$. Sin
embargo, para muchas funciones, el valor de $f'(x_0;x)$ es muy cercano a una constante para valores
pequeños de $\Delta x$. Ilustremos esto con un ejemplo.

\paragraph{Ejemplo.}
En el caso del globo, el costo $G$ en dólares del gas necesario para inflar el globo de diámetro
$d$ decímetros está dado por:
\[
G = \psi(d) = \frac{1}{2\,000}d^{3},\ d\in [60,120].
\]
Si tomamos $d_{0}=80$ y diferentes valores de $d$ cercanos a 80 (ver la tabla a continuación),
tendremos que los valores que toma la razón de cambio
\[
\psi'(d_{0};d) = \frac{\Delta G}{\Delta d} = \frac{\psi(d)-\psi(d_{0})}{d-d_{0}}
\]
son muy similares entre sí y se acercan cada vez más a $3.84$. En otras palabras, podemos ver que
¡ese número es el límite de la razón de cambio $\psi'(d_{0};d)$ cuando $d$ tiende a $d_0$!

\begin{center}
{\renewcommand\arraystretch{1.5}%
\setlength\extrarowheight{1pt}
\begin{tabular}{|c|c|c|c|}
\hline
$d$ & $\psi(d)$ & $\psi(d)-\psi(d_{0})$ & $\psi'(d_{0},d)$ \\
\hline
78  &  94.91 &  $-$7.49 & 3.74  \\[-2pt]
\hline
79 & 98.61 & $-$3.79 & 3.79 \\
\hline
79.5 &  100.49    & $-$1.91     &  3.82   \\
\hline
79.9 & 102.02     &  $-$0.38    &  3.84      \\
\hline
&  &  &  $\downarrow $\\
80 & 102.40 & 0 & 3.84  \\
&  &  &  $\uparrow $\\
\hline
80.1 &  102.78     &  0.38    &    3.84       \\
\hline
80.5 &  104.33    &   1.93     &   3.86      \\
\hline
81 & 106.29 & 3.89 & 3.89 \\
\hline
82 & 110.27 & 7.87 & 3.94 \\
\hline
\end{tabular}}
\end{center}

Entonces, si definimos $\psi'(d_0)$ por:
\[
\psi'(d_{0})=\lim_{d\rightarrow d_{0}}\psi'(d_{0};d) =
    \lim_{d\rightarrow d_{0}}\frac{\psi(d)-\psi(d_{0})}{d-d_{0}} =
    \lim_{\Delta d \rightarrow 0}\frac{\Delta G}{\Delta d} = 3.84,
\]
podremos escribir
\[
\Delta G \approx 3.84\Delta d \approx \psi'(d_{0}) \Delta d,
\]
si $\Delta d$ es pequeño. Es decir, $\Delta G$ es ``casi'' directamente proporcional a $\Delta d$,
si  $\Delta d$ es pequeño. Al valor $\psi'(d_{0})$ se le llama \emph{razón de cambio instantánea de
$\psi$ en $d_{0}$}.

En general:

\begin{defical}[Razón de cambio instantánea]
Si una variable $y$ depende de una variable $x$ mediante una función $f\colon I\subset\mathbb{R}
\rightarrow \mathbb{R}$ y, para un $x_{0}$ dado existe:
\[
f'(x_{0}) = \lim_{x \rightarrow x_{0}}f'(x_{0},x) =
    \lim_{\Delta x \rightarrow 0}\frac{\Delta y}{\Delta x} =
    \lim_{x \rightarrow x_{0}}\frac{f(x)-f(x_{0})}{x-x_{0}},
\]
diremos que $f'(x_{0})$ es la \emph{razón de cambio instantánea de $y$ con respecto a $x$}.
\end{defical}

Para $\Delta x$ pequeños, tendremos que:
\[
\Delta y \approx f'(x_{0}) \Delta x.
\]
En otras palabras, para variaciones pequeñas de $x$ respecto de $x_0$, las variaciones de $y$
respecto de $y_0$ son ``casi'' directamente proporcionales a las variaciones de $x$, donde
$f'(x_{0})$ es la constante de proporcionalidad.

\begin{exemplo}[]{}
En la fábrica de autos, cuyo costo $C$ miles de dólares está dado por
\[
C = f(x) = 200 + 11x - 0.000\,12 x^2,
\]
obtuvimos que el costo unitario es igual a:
\[
C_U = f_1(x) = \frac{f(x)}{x}= \frac{2\,000}{x}+11-0.000\,12x.
\]
Si la producción pasó de $5\,000$ a $x$ autos, la razón de cambio de los costos $C$ en el intervalo
de extremos $5\,000$ y $x$ está dada por:
\[
f'(5\,000,x)=\frac{f(x)-f(5\,000)}{x-5\,000}=\frac{-52\,000+11x-0.000\,12x^2}{x-5\,000}.
\]

Además, la razón de cambio del costo unitario $C_U$ en el mismo intervalo será:
\[
\frac{f_1(x)-f_1(5\,000)}{x-5\,000}=\frac{\frac{f(x)}{x}-\frac{f(5\,000)}{5\,000}}{x-5\,000}%
=\frac{\frac{2\,000}{x}+0.2-0.000\,12x}{x-5\,000}.
\]

Al tomar, en estos dos casos, el límite cuando $x$ se aproxima a $5\,000$ o, lo que es lo mismo, el
límite cuando $\Delta x$ se aproxima a $0$, obtendremos las respectivas razones de cambio
instantáneas para un nivel de producción igual a $x=5\,000$ autos:
\begin{gather*}
C'=f'(5\,000)=\lim_{x\to 5\,000}\frac{-52\,000+11x-0.000\,12x^2}{x-5\,000}=9.8,\\
C_U'=f_1'(5\,000)=\lim_{x\to 5\,000}\frac{\frac{2\,000}{x}+0.2-0.000\,12x}{x-5\,000}=-0.000\,2.
\end{gather*}

Por la definición de razón de cambio instantánea, si $\Delta x$ es pequeño, a partir de un nivel de
producción igual a $x=5\,000$ autos tendremos que:
\begin{gather*}
\Delta C\approx f'(5\,000)\Delta x  =9.8\Delta x,\\
\Delta C_U\approx f_1'(5\,000)\Delta x   = -0.000\,2\Delta x.
\end{gather*}

Si $\Delta x = 1\,000$, entonces:
\[
f'(5\,000)\Delta x  =9.8(1\,000)=9\,800
\]
y $\Delta C = 9\,600$. Es decir, a pesar de que $\Delta x$ no es tan pequeño, es cercano a $\Delta
C$.

También
\[
f_1'(5\,000)\Delta x  =-0.000\,2(1\,000)=-0.2
\]
es cercano a $\Delta C_U$ que, en este caso, es
\[
\Delta C_U  =-0.187.
\]
\end{exemplo}

\subsection{Magnitudes marginales en Economía}
En el ejemplo de la empresa de autos que produce $5\,000$ autos por mes, los administradores
quisieran saber cuánto le costaría a la fábrica producir $1$ auto más al mes; es decir, quisieran
averiguar el costo de producción del auto número $5\,001$. A este costo se lo conoce con el nombre
de \emph{costo marginal}.

Para averiguar el costo marginal al nivel de producción de $5\,000$ autos mensuales solo hay que
calcular el incremento del costo en el intervalo $[5\,000, 5\,001]$. En este caso, $\Delta x=1$ y:
\[
\Delta C  = f(5\,001)-f(5\,000) \approx f'(5000) = 9.8,
\]
que aproxima al \emph{costo marginal} para una producción de $x=5\,000$ autos, ya que $\Delta x$ es
pequeño. Es decir, producir un auto más cuesta $9\,800$ dólares aproximadamente. El costo marginal es aproximado por la razón de cambio instantánea del costo, por lo cual a esta magnitud se la llama, por simplicidad, costo marginal.

En cambio, el \emph{costo unitario marginal} para este mismo nivel de producción sería:
\[
\Delta C_U  = f_1(5\,001)-f_1(5\,000)\approx f_1'(5\,000) = -0.000\,2.
\]
Es decir, el costo unitario de producir el auto número $5\,001$ es $-0.000\,2$
aproximadamente y, por ser negativo, hay una ganancia de 20 centavos de dólar cuando se produce el
auto número $5\,001$.

Para ésta y otras magnitudes como costo unitario, ingreso, utilidad, demanda, etcétera, a la razón de cambio instantánea
correspondiente se la llama, respectivamente, costo unitario marginal, ingreso marginal, utilidad marginal, demanda
marginal, etcétera.

\subsection{Elasticidad}
En esta sección, se definió la \emph{elasticidad}. Veamos esta cuestión más de cerca.

Si una variable $y$ depende de otra variable $x$ a través de una función $f\colon\mathbb{R}
\rightarrow\mathbb{R}$, y si variamos el valor de $x$ a partir de un valor inicial $x_{0}$, nos
interesa en qué porcentaje varía $y$ a partir del valor inicial $y_{0} = f(x_{0})$, si $x$ varía en
un porcentaje dado.

Conocida la variación porcentual de $x$:
\[
\frac{\Delta x}{x_{0}}100\%
\]
y la variación porcentual de $y$:
\[
\frac{\Delta y}{y_{0}}100\%,
\]
queremos conocer cómo depende $\frac{\Delta y}{y_{0}}$ de $\frac{\Delta x}{x_{0}}$, donde
\[
\Delta x = x - x_{0}; \quad \Delta y = y - y_0 = f(x) - f(x_{0}),
\]
con $x\neq x_{0}$.

En el ejemplo en el que los costos $y$ en miles de dólares vienen dados a través de la función $f$
definida por
\[
y = f(x) = 2000 + 10x,
\]
donde $x\in[1\,000, 10\,000]$, vimos que si, en un momento dado, se estuvieran fabricando $x_{0} =
4\,800$ autos a un costo de $y_{0} = f(4\,800) = 50\,000$ miles de dólares, nos interesaría saber
en qué porcentaje se incrementarían los costos si eleváramos la producción en un $10\%$.

Recordemos que, cuando $f$ es un polinomio de grado menor o igual a $1$ de la forma $f(x) = mx +
b$, ¡la variación relativa de $y$ es directamente proporcional a la variación relativa de $x$! Es
decir, existe una constante $\eta$ tal que
\[
\frac{\Delta y}{y_{0}} = \eta \frac{\Delta x}{x_{0}}.
\]
A esta constante la llamamos, precisamente, \emph{elasticidad de $y$ respecto de $x$}.

Como
\[
  \eta = \frac{\frac{\Delta y}{y_{0}}}{\frac{\Delta x}{x_{0}}} =
    \frac{x_{0}}{f(x_{0})}\frac{\Delta y}{\Delta x},
\]
tendremos:
\begin{align*}
\eta &= \frac{\frac{\Delta y}{y_{0}}}{\frac{\Delta x}{x_{0}}} =
  \frac{\frac{f(x) - f(x_{0})}{f(x_{0})}}{\frac{x - x_{0}}{x_{0}}} \\[4pt]
  &= \frac{x_{0}}{f(x_{0})}\cdot\frac{(mx + b)-(mx_{0}+b)}{x - x_{0}} \\[4pt]
  &= \frac{mx_{0}}{f(x_{0})}\cdot\frac{x - x_{0}}{x - x_{0}}.
\end{align*}
Es decir:
\[
\eta = \frac{mx_{0}}{f(x_{0})}.
\]

Se dice que $y$ es elástica si $|\eta|>1$, inelástica si $|\eta|<1$ y que la elasticidad es
unitaria si $|\eta| = 1$.

En nuestro ejemplo: $y = f(x) = 2000+10x$, $m = 10$, $x_{0}= 4800$, $y_{0}= f(x_{0}) = 5000$, por
lo que:
\[
\eta = \frac{10\times 4800}{5000} = 0.96.
\]
Entonces
\[
 \frac{\Delta y}{y_{0}} = 0.96\frac{\Delta x}{x_{0}},
\]
por lo que, si la producción se incrementara en un 10\%, es decir si $\frac{\Delta x}{x_{0}} =
10\%$, los costos se incrementarán en $\frac{\Delta y}{y_{0}} = 0.96\times 10\%$; es decir, en un
$9.6\%$.\vspace{\baselineskip}

En el caso general, si calculamos $\eta$ de modo que
\[
 \frac{\Delta y}{y_{0}} = \eta \frac{\Delta x}{x_{0}},
\]
obtendremos que
\[
 \eta = \eta(x_{0},x) = \frac{x_{0}}{f(x_{0})}  \frac{\Delta y}{\Delta x} =
 \frac{x_{0}}{f(x_{0})} \frac{f(x) - f(x_{0})}{x - x_{0}} =
 \frac{x_{0}}{f(x_{0})} \frac{f(x_{0} + \Delta x) - f(x_{0})}{\Delta x},
\]
por lo que $\eta$ depende de $x_{0}$ y de $x \neq x_{0}$. Sin embargo, si existe la razón de cambio
instantánea
\[
 f'(x_{0}) = \lim_{x\rightarrow x_{0}}\frac{f(x) - f(x_{0})}{x - x_{0}} ,
\]
se puede observar que para valores  de $x$ cercanos a $x_{0}$; es decir, cuando $\Delta x$ es
pequeño, los valores que toma $\eta(x_{0},x) $ son muy similares entre sí y se acercan cada vez más
a
\[
 \eta(x_{0}) = \lim_{x\rightarrow x_{0}}\eta(x_{0},x)   =
 \lim_{x\rightarrow x_{0}} \frac{x_{0}}{f(x_{0)}}  \frac{f(x) - f(x_{0})}{x - x_{0}},
\]
\[
 \eta(x_{0}) = \frac{x_{0}}{f(x_{0})} f'(x_{0}).
\]
A $\eta(x_{0}) $ se le llama \emph{elasticidad puntual} de $y$ en $x_{0}$, y si $\Delta x$ es
pequeño, tendremos que
\[
 \frac{\Delta y}{y_{0}} \approx \eta(x_{0})  \frac{\Delta x}{x_{0}}.
\]

\begin{exemplo}[]{}
Luego de un estudio de mercado, se determinó que la demanda $D$ de una marca de lavadoras es de
$f(p)$ unidades si el precio es de $p$ dólares. Como su producción no es rentable si $p < 200$ y la
marca del competidor cuesta $500$ dólares, tendremos que $D(f) = [200,500]$. El estudio reveló
también que
\[
 f(p) = 2\,000 - 0.4p + \frac{22\,500}{p}
\]
mide la demanda de lavadoras cuando el precio de cada una es de $p$ dólares.

El precio actual es de $300$ dólares. Se desea conocer aproximadamente en qué porcentaje aumentaría
la demanda si se hiciera un descuento del $5\%$.

En este caso, tendríamos que $p_{0} = 300$, por lo que:
\[
 D_{0} = f(p_{0}) = 1\,955.
\]

Como lo veremos en el siguiente capítulo, se tiene que
\[
 f'(p) = -0.4 - \frac{22\,500}{p^{2}},
\]
por lo que:
\[
f'(p_{0}) = - 0.65.
\]

Se desea conocer el valor aproximado de $\frac{\Delta D}{D_{0}}$. Como:
\[
 \eta (p_{0}) = \frac{p_{0}}{D_{0}} f'(p_{0}) = \frac{300}{1\,955}(-0.65) =
 \frac{39}{391}\approx -0.1,
\]
tenemos que:
\[
\frac{\Delta D}{D_{0}}\approx \eta (p_{0})\frac{\Delta p}{p_{0}} =
\left(-\frac{39}{391}\right)(-5\%)\approx 0.5\%
\]

Es decir, una rebaja del precio en un $5\%$ significaría un incremento en la demanda de apenas el
$0.5\%$.
\end{exemplo}

\begin{exemplo}[]{}
En el caso de la primera fábrica de autos cuyo nivel de producción mensual pasaría de $5\,000$
autos a $x$ autos, como los costos son $C$ miles de dólares y
\[
C = f(x)= 2\,000+11x-0.000\,12x^2,
\]
la \emph{elasticidad media} de los costos en el intervalo de extremos 5000 y $x$ sería:
\[
\eta (5\,000,x)= \frac{5\,000}{f(5\,000)}\frac{\Delta C}{\Delta x}=
\frac{5\,000}{54\,500}\frac{-52\,000+11x-0.000\,12x^2}{x-5\,000}.
\]

La \emph{elasticidad puntual} de los costos para $x_0 =5000$ será :
\[
\eta (5\,000)= \frac{5\,000}{f(5\,000)}f'(5\,000)= \frac{5\,000}{54\,500}9.8\approx 0.899\,1.
\]
Esto quiere decir que si el nivel de producción variara, digamos en $a\%$, los costos variarían en
$0.899\,1 a\%$ aproximadamente.

Así, si la producción subiera en un $5\%$, los costos variarían en $0.899\,1\times 5\%\approx
4.5\%$. Los costos son, pues, ``poco elásticos'' respecto al nivel de producción.

Un cálculo análogo lo podemos realizar para el costo unitario $C_U$, teniendo en cuenta que:
\begin{gather*}
C_U = f_1(x)= \frac{2\,000}{x}+11-0.000\,12x,\\
f_1(5\,000)=10.8, \\
C_U'=f_1'(x)=-\frac{2\,000}{x^2}-0.000\,12,\\
f_1'(5\,000)=-0.000\,2,
\end{gather*}
la elasticidad media del costo unitario en el intervalo de extremos $5\,000$ y $x$ sería:
\[
\eta_1 (5\,000,x)= \frac{5\,000}{f_1(5\,000)}\frac{\Delta C_U}{\Delta x}=
\frac{5\,000}{10.8}\frac{\frac{2\,000}{x}+0.2-0.000\,12x}{x-5\,000},
\]
y la elasticidad puntual del costo unitario para un nivel de producción $x_0=5\,000$ autos sería:
\[
\eta_1 (5\,000)= \frac{5\,000}{f_1(5\,000)}f_1'(5\,000) =\frac{5\,000}{10.8}(-0.000\,2)\approx -0.1.
\]
Esto quiere decir que, si el nivel de producción aumentara, digamos en un $5\%$, el costo unitario
variaría en $(-0.1)5\%$ aproximadamente, es decir, bajaría en $0.5\%$.
\end{exemplo}

\section{La descripción del movimiento}

Uno de los grandes descubrimientos de la humanidad es el concepto de tiempo. Obedeció a la
existencia de ciclos naturales como el día y la noche, las lunas llenas, las estaciones del año, el
movimiento regular de los astros. Esos ciclos naturales dieron lugar a la determinación de unidades
de medir el tiempo: los días y sus fracciones (horas, minutos y segundos), los meses, los años y,
por consiguiente, a la creación de relojes y calendarios.

En la física, para estudiar un fenómeno descrito por uno o más parámetros que varían con el tiempo,
se hace necesario medir a éste. Para ello se adopta una unidad de medida del tiempo $\UT$, por
ejemplo $\UT$ puede ser un segundo, una hora, un año, etcétera, y se escoge un instante de
referencia a partir del cual se inicia la medición del tiempo. Así, si han transcurrido $t \UT$,
diremos que estamos en el instante $t$. La cantidad $t$ es pues un número, por lo cual el tiempo
puede ser representado gráficamente en la recta real, cuyo origen representa el instante en que se
empezó a medir el tiempo y un punto $T$ de abscisa $t$, representará al que llamamos instante $t$.
Hemos establecido así \emph{un sistema de medición del tiempo}. Podemos ahora introducir la
variable $t$ para describir el tiempo en el estudio de los fenómenos en los que esté involucrado.

Por ejemplo, si queremos estudiar el movimiento de una partícula a lo largo de una recta, una vez
que hemos adoptado una unidad de longitud $(\UL)$, podemos escoger un punto de referencia $O$ en la
recta y un sentido a partir de $O$, digamos a la derecha de $O$, que se lo considerará positivo,
por lo que podemos asumir que el movimiento se realiza a lo largo de una recta real. Escogida una
unidad de tiempo $(\UT)$, y si consideramos que $t$ es el número de unidades de tiempo transcurrido
desde el tiempo inicial $\tau_{0}$, el movimiento de la partícula quedará descrito siempre que
podamos establecer la posición $x = f(t)$ de la partícula en cada instante $t\geq \tau_{0}$. La
posición $x = f(t)$ es también denominada \emph{coordenada de la partícula en el instante $t$}.

Si la partícula que estaba en $x_{0} = f(\tau_0)$ en el instante $\tau_0$ pasó a ocupar la posición
$x = f(t)$ en el instante $t$, diremos que su desplazamiento es
\[
\Delta x = \Delta x(\tau_0, t) = f(t) - f(\tau_0).
\]
Esto sucederá en el lapso
\[
\Delta t = \Delta t(\tau_0, t) = t - \tau_0>0.
\]

Notemos que si $\Delta x>0$, la partícula se ha desplazado a la derecha y si $\Delta x<0$, a la
izquierda de donde estaba en el instante $\tau_0$; esto es, a la derecha de $x_{0}= f(\tau_0)$ o a
la izquierda de $x_{0} = f(\tau_0)$, respectivamente.

\subsection{El concepto de velocidad}

Cuando una partícula se mueve a lo largo de la recta, un observador puede opinar que ésta se mueve
rápidamente, otro que se mueve lentamente. El concepto de rapidez tiene pues cierta carga de
subjetividad, pero se la puede medir objetivamente gracias a la cantidad llamada \emph{velocidad}.
Veamos el concepto de velocidad en dos casos.

\subsubsection{El movimiento uniforme}

Supongamos que medimos la longitud en metros y el tiempo en segundos. Si $x=f(t)$, con $\tau_0\leq
t$, es la posición de una partícula, entre un instante $t_0$ y otro instante $t$, $\tau_0\leq t_0<
t$, habrá transcurrido un \emph{lapso} de $\Delta t$ segundos, donde $\Delta t= t-t_0$.

En ese lapso, se habrá producido un desplazamiento de la partícula desde la posición $x_0=f(t_0)$
hasta la posición $x=f(t)$. El desplazamiento será de $\Delta x$ metros, donde $\Delta x=
x-x_0=f(t)-f(t_0)$.

\begin{defical}[Movimiento uniforme]
Diremos que el movimiento es \emph{uniforme} si $\Delta x$ es directamente proporcional a $\Delta
t$, es decir si existe una constante de proporcionalidad $v$ tal que
\begin{equation*}
	\Delta x= v\Delta t.
\end{equation*}
\end{defical}

Esto equivale a decir que el cociente
\begin{equation*}
	\frac{\Delta x}{\Delta t}= \frac{\Delta x(t_0,t)}{\Delta t(t_0,t)} = v
\end{equation*}
es constante.

Evidentemente, en este caso, el valor absoluto de $v$, es decir $|v|$, nos dará una idea clara de
cuán rápido se mueve la partícula, mientras que el signo de $v$ nos indica si la partícula se mueve
de ``izquierda a derecha'', en el caso de que $v>0$ y, en sentido contrario, si $v<0$.

El número $v$ nos sirve entonces para medir la \emph{velocidad} de la partícula. Diremos que $v
\UV$ es la velocidad de la partícula. Como
\begin{equation*}
	v=\frac{\Delta x}{\Delta t},
\end{equation*}
se puede escribir simbólicamente, en este caso, lo siguiente:
\begin{equation*}
	\text{velocidad de la partícula}=\frac{\text{desplazamiento de la partícula}}{\text{lapso transcurrido}}
\end{equation*}
o también
\begin{equation*}
	v \UV= \frac{\Delta x\text{ m}}{\Delta t\text{ s}}=\frac{\Delta x}{\Delta t}\frac{\text{m}}{\text{s}}
\end{equation*}
y como $v=\frac{\Delta x}{\Delta t}$, es cómodo escribir $\frac{\text{m}}{\text{s}}$ en vez de
$\UV$, la unidad de la velocidad. Si se escoge otra unidad $\UL$ del desplazamiento $\Delta x$ y
otra unidad $\UT$ del lapso $\Delta t$, se tendrá otra unidad de la velocidad $\UV$.

En general, si la longitud se mide en $\UL$ y el tiempo en $\UT$, pondremos $\frac{\UL}{\UT}$ en
vez de $\UV$.

Es fácil notar que, si $x=f(t)$ es un polinomio de grado menor o igual a 1, el movimiento es
uniforme.

En efecto, en este caso:
\[
x=f(t)=at+b,
\]
donde $a$ y $b$ son constantes. Por lo tanto:
\begin{equation*}
\frac{\Delta x}{\Delta t}= \frac{\Delta x(t_0,t)}{\Delta t(t_0,t)}=\frac{x-x_0}{t-t_0}=\frac{f(t)-f(t_0)}{t-t_0}%
=\frac{(at+b)-(at_0+b)}{t-t_0}=a.
\end{equation*}
Es decir, $v=a$ es la velocidad y no depende de $t_0$ ni de $t$.

Notemos que, sin importar el valor de $t_0$, para valores iguales de $\Delta t$, se obtiene valores
iguales para el desplazamiento, ya que  $\Delta x= a\Delta t$. Esto justifica el nombre de
\emph{movimiento uniforme}.

Por otro lado, si $\Delta t =1$, tendremos que  $\Delta x= v$, es decir que $v$ m es el
desplazamiento de la partícula en 1 segundo. En general, $v \UL$ es el desplazamiento en el lapso
$1 \UT$.

\begin{exemplo}[]{}
\begingroup
\itshape La empresa de transporte pesado TRANSLIT despachó desde Guayaquil el camión N$^\text{o
}17$ a las 8h00 con una pequeña carga destinada a un cliente en Balzar, y otra muy pesada a un
cliente en Santo Domingo. El chofer del camión es cambiado en Quevedo. El recorrido hacia Santo
Domingo es realizado por otro chofer.

El cliente que vive en Balzar llamó para indicar que recibió el despacho a las 10h00, aunque se le
había prometido que se le entregaría el envío a las 9h30. Se le explicó que el camión estaba
completamente cargado por lo que debió viajar lentamente. La empresa deberá telefonear al chofer
que espera en Quevedo para indicarle la hora aproximada de llegada del camión para el cambio de
conductor. ?`A qué hora llegará el camión a Quevedo? ?`A qué hora se espera que el camión llegue a
Santo Domingo?
\endgroup

\vspace{0.6\baselineskip}%
Para resolver el problema, observemos la información con que contamos
sobre las distancias entre las ciudades:
\begin{itemize}
\item \emph{Guayaquil-Balzar}: 96 km,
\item \emph{Balzar-Quevedo}: 72 km,
\item \emph{Quevedo-Santo Domingo}: 120 km,
\item \emph{Guayaquil-Quevedo}: 168 km,
\item \emph{Guayaquil-Santo Domingo}: 288 km.
\end{itemize}

Ahora, definamos las cantidades involucradas en el problema:\vspace{\baselineskip}\\
{\setlength\tabcolsep{3pt}
\begin{tabular}{r p{0.85\textwidth}}
  $t:$ & \textsl{instante en horas en que se da cada suceso; es el tiempo transcurrido desde las 0 horas
    del día del viaje.}\\
  $\tau_0:$ & \textsl{instante que coincide con las $0$ horas del día del viaje.}\\
  $t_0:$ & \textsl{instante inicial en el que el camión inició su recorrido desde
    Guayaquil.}\\
  $x:$ & \textsl{posición del camión en la carretera Guayaquil-Balzar-Quevedo-Santo Domingo;
    es la distancia recorrida en kilómetros desde Guayaquil al punto en la carretera en el momento $t$.}\\
  $x_0:$ & \textsl{posición inicial del camión  en Guayaquil al tiempo $t_0$.}\\
  $x_1:$ & \textsl{posición  del camión al pasar por Balzar.}\\
  $t_2:$ & \textsl{hora de llegada a Quevedo.}\\
  $t_3:$ & \textsl{hora de llegada a Santo Domingo.}
\end{tabular}}
\vspace{\baselineskip}

Bajo el supuesto de que el movimiento del camión será uniforme aproximadamente, la posición del
camión en cualquier momento $t$ puede ser descrito por la función $f$ definida por:
\begin{equation*}
	x=f(t)= at + b.
\end{equation*}
Para poder calcular la posición del camión, debemos, entonces, determinar $a$ y $b$.

En primer lugar, $t_0 = 8$, pues el camión inicia el recorrido a las ocho de la mañana. Entonces:
$x_0= f(8)$. Como en ese momento, el camión aún no se ha movido, $f(8) = 0$, por lo que $x_0 = 0$.

En segundo lugar, a las diez de la mañana, el camión llegó al Balzar, después de recorrer $96$
kilómetros. Por lo tanto:
\[
x_1 = f(10) = 96.
\]
Entonces:
\begin{equation*}
\left\{
\begin{matrix}
0 & = & 8a & + & b\\
96 & = & 10a& + & b
\end{matrix}
\right.
\end{equation*}
Al restar la primera ecuación de la segunda y, luego de dividir el resultado por $2$, se obtendrá
que $a=48$ y, por consiguiente, que $b=-384$, pues $b = -8a$.

Entonces, tenemos que:
\begin{equation*}
x = f(t)= 48t-384.
\end{equation*}

Ahora hallemos $t_2$. Éste es el momento en que el camión llega a Quevedo. Como esta ciudad está a
$168$ kilómetros de Guayaquil, $t_2$ debe verificar la siguiente igualdad:
\begin{equation*}
x_2 = 168 = f(t_2)= 48t_2 - 384,
\end{equation*}
lo que nos da que:
\begin{equation*}
t_2 =\frac{168+384}{48}=\frac{552}{48}=11.5.
\end{equation*}
En otras palabras, el camión llegará a Quevedo a las 11h30.

Finalmente, para determinar la hora de llegada a Santo Domingo, debemos calcular $t_3$, que
satisface la igualdad
\begin{equation*}
x_3 =288= f(t_3)= 48t_3-384,
\end{equation*}
pues la distancia entre Guayaquil y Santo Domingo es de $288$ kilómetros. Por lo tanto:
\begin{equation*}
t_3 =\frac{228+384}{48}=\frac{672}{48}=14.
\end{equation*}
Es decir, se espera que el camión llegue a Santo Domingo a las 14h00.

Así, pues, la empresa TRANSLIT deberá telefonear al chofer que espera en Quevedo  para indicarle
que el camión N$^\text{o }17$ llegará aproximadamente a las 11h30 para el cambio de conductor. Si
la empresa TRANSLIT es eficiente también telefoneará a sus oficinas en Santo Domingo para que
avisen al cliente que su carga estará llegando alrededor de las 14h00.
\end{exemplo}

\subsection{Caso general: movimiento no-uniforme}

Si el cociente
\[
 v_{m}(t_{0},t) = \frac{\Delta x}{\Delta t} = \frac{\Delta x(t_{0},t)}{\Delta t(t_{0},t)}
\]
no es constante, diremos que el movimiento es \emph{no-uniforme} y a $v_{m}(t_{0},t) $ le
llamaremos \emph{velocidad media de la partícula en el intervalo de tiempo $[t_{0},t]$}.

Obviamente,
\[
\Delta x = v_{m}(t_{0},t)\Delta t,
\]
por lo que mientras mayor sea $|v_{m}(t_{0},t)|$ más rápido será el movimiento de la partícula y viceversa, lo que justifica su nombre.

Si el lapso considerado $\Delta t$ toma valores cada vez más pequeños para ciertas funciones de
posición $f\colon\mathbb{R}\rightarrow\mathbb{R},t\mapsto x=f(t)$, puede suceder que los
correspondientes valores de la velocidad media $v_{m}(t_{0})$ sean muy similares y se acerquen al
límite de $v_{m}(t_{0},t)$ cuando $t$ tiende a $t_{0}$ (o cuando $\Delta t $ tiende a 0) que lo
notaremos $v_{0} = g(t_{0})$. Es decir
\[
v_{0} = g(t_{0}) = \lim_{t \rightarrow t_{0}}v_{m}(t_{0},t) = \lim_{\Delta t \rightarrow 0}\frac{\Delta x}{\Delta t}.
\]
A $v_{0} = g(t_{0})$, que viene dada en $(\UL/ \UT)$, le llamaremos \emph{velocidad instantánea} de la partícula en el instante $t_{0}$.

Dado que para valores pequeños de $\Delta t$,
\[
v_{0} = g(t_{0})\approx v(t_{0},t) = v(t_{0},t_{0} + t),
\]
tendremos que:
\[
\Delta x \approx  v_{0} \Delta t,
\]
lo que justifica que a $v_{0} = g(t_{0})$ se le llame \emph{velocidad instantánea en el instante
$t_{0}$}.

En efecto, para un $\Delta t$ dado, mientras mayor sea $v_{0}$, más grande será el desplazamiento
que se produzca en el lapso $\Delta t$, lo que indica que la partícula se mueve más rápidamente.
Por otra parte, $x_{0}(\UL)$ es el desplazamiento que tendría si no se acelerara o frenara a la
partícula o, lo que es lo mismo, si sobre la partícula no actuara fuerza alguna.

\begin{exemplo}[]{}
\begingroup
\itshape Si se deja caer un objeto desde lo alto de un rascacielos de $122.5\metros$ de altura, sea
$x = f(t)$ metros la altura del objeto medida desde el suelo después de $t$ segundos. Entonces, las
leyes de Newton nos dicen que:
\begin{equation*}
x=f(t) = 122.5-4.9t^2.
\end{equation*}
?`Con qué velocidad choca el objeto con el suelo y en cuánto tiempo lo hace? ?`Cuál es la velocidad
media con la que ha hecho el recorrido?
\endgroup

\vspace{0.5\baselineskip}%
Sea $t_1$ segundos el tiempo que tarda el objeto en llegar al suelo.
Podemos determinar $t_1$ de la igualdad
\begin{equation*}
	x_1=f(t_1) =0,
\end{equation*}
puesto que $0$ es la altura del objeto al llegar al suelo. Entonces:
\begin{equation*}
	x_1=0=f(t_1) = 122.5-4.9t_1^2,
\end{equation*}
lo que nos da que $t_1^2=25$ o $t_1=5$. La otra solución de la ecuación cuadrática es $-5$. Aunque
es solución de la ecuación, no es solución del problema, pues no hay como asignar un significado a
$t_1 = -5$.

En resumen, el objeto llegará al suelo en $5$ segundos.

Como $v_1= x_1'= f'(t_1)$ es la velocidad del objeto en el instante $t_1$, podemos escribir lo
siguiente:
\begin{align*}
v_1=  f'(t_1)&=\lim_{t\to t_1}\frac{f(t)-f(t_1)}{t-t_1}\\
&=\lim_{t\to t_1}\frac{(122.5-4.9t^2)-(122.5-4.9t_1^2)}{t-t_1}\\
&=\lim_{t\to t_1}\frac{-4.9(t^2-t_1^2)}{t-t_1}\\
&=\lim_{t\to t_1}\frac{-4.9(t+t_1)(t-t_1)}{t-t_1}\\
&=\lim_{t\to t_1}(-4.9)(t+t_1)\\
&=-9.8t_1.
\end{align*}
Es decir:
\begin{equation*}
	v_1=  f'(t_1)=-9.8t_1.
\end{equation*}
Para $t_1=5$, tendremos que:
\begin{equation*}
	v_1=  f'(5)=-9.8(5)=-49.
\end{equation*}
Es decir, el objeto llegará al suelo con una velocidad de $-49$ metros por segundo.

El signo negativo de la velocidad indica que el movimiento es hacia abajo, puesto que la altura se
mide desde el suelo hacia arriba.

Finalmente, calculemos la velocidad media en el intervalo $[0,t_1]$:
\begin{equation*}
	v_m(0, t_1)= \frac{x_1-x_0}{t_1-0}= \frac{f(t_1)-f(t_0)}{t_1}= \frac{(122.5-4.9t_1^2)-122.5}{t_1}=-4.9t_1.
\end{equation*}

Para $t_1=5$, tendremos $v_m(0, 5)=-4.9(5)=24.5$.

Es decir, la velocidad media del objeto en el lapso $[0,5]$ será de $24.5$ metros por segundo.
\end{exemplo}

\subsubsection{Notas importantes}

\begin{enumerate}[leftmargin=*]
\item Solo por facilitar la exposición hemos supuesto que $t_{0}<t$. Nada impide que $t<t_{0}$.
    Solo que en este caso $\Delta t = t - t_{0} < 0$. Es por eso que, en la definición de
    velocidad instantánea, pusimos:
\[
v(t_{0}) = \lim_{\Delta t \rightarrow 0}\frac{\Delta x(t_{0},t)}{\Delta t(t_{0},t)} =
      \lim_{t \rightarrow t_{0}}\frac{ x(t) - x(t_{0})}{t - t_{0}}
\]
y no hace falta poner $\lim_{\Delta t \rightarrow 0+}$ o $\lim_{ t \rightarrow t_{0}+}$.

\item $\Delta t$ representa el lapso (de tiempo, por supuesto) o el intervalo $[t_{0},t]$ si $t
    > t_{0}$, o $[t,t_{0}]$ si $t < t_{0}$, en cuyo caso $\Delta t < 0$.

\item ``Acelerar'' equivale a ir cada vez más rápidamente.

\item ``Frenar'' equivale a ir cada vez más lentamente.

\item El signo de $\Delta x$ da el sentido del desplazamiento: si $\Delta x > 0$ quiere decir
    que $x(t)$ está a la derecha de $x(t_{0})$ y viceversa. Por consiguiente, como $v(t_{0}) >
    0$, se tiene que
    \[
    \frac{\Delta x(t_{0},t)}{\Delta t(t_{0},t) }> 0
    \]
    para $|\Delta t|$ pequeño. Esto significa que la partícula se mueve de izquierda a derecha.
    Por el contrario, si $v(t_{0}) < 0$, el movimiento será de derecha a izquierda.
\end{enumerate}

\section{Conclusión}
En los cuatro problemas que hemos tratado en este capítulo, hemos llegado a establecer una función
$\funcjc{f}{I\subset\mathbb{R}}{\mathbb{R}}$, donde $I$ es un intervalo abierto, para la cual, si
$a\in I$, existe el límite
\[
\limjc{\frac{f(x) - f(a)}{x - a}}{x}{a}.
\]
A este límite le hemos representado con $f'(a)$ y se le denomina la \emph{derivada} de $f$ en $a$.
En el siguiente capítulo, será objeto de estudio pues, como lo hemos visto, es una herramienta
poderosa que resuelve una clase amplia de problemas. Un capítulo más adelante, veremos algunas de
sus principales aplicaciones.

\begin{multicols}{2}[\section{Ejercicios}]
\begingroup
\small
\begin{enumerate}[leftmargin=*]
\item Hallar, si existe, la ecuación de la recta tangente al gráfico de la función $f$ en el
    punto $P(a,f(a))$:
    \begin{enumerate}[leftmargin=*]
    \item $f(x) = 2x^2 + 3x - 5$, $a = 2$.
    \item $f(x) = x^3 - x$, $a = -2$.
    \item $\displaystyle f(x) = \frac{3x - 2}{x}$, $a = 1$.
    \item $\displaystyle f(x) = \frac{1}{x^2}$, $a = 1$.
    \item $\displaystyle f(x) = \sqrt{2 + x}$, $a = 0$.
    \item $f(x) = 2(x + 1)$, $a = -2$.
    \item $\displaystyle f(x )= \sqrt[3]{x^2}$, $a = 0$.
    \item $\displaystyle f(x) = 2\sen 3x$, $\displaystyle a = \frac{\pi}{6}$.
    \end{enumerate}

\item ?`Para qué valores de $a$, el gráfico de $f$ tiene una recta tangente en $P(a,f(a))$. Para
    estos valores, ?`cuál es la pendiente $m_a$?
    \begin{enumerate}[leftmargin=*]
    \item $\displaystyle f(x) = x^3 + 2x + 1$.
    \item $\displaystyle f(x) = \frac{1}{x}$.
    \item $\displaystyle f(x) = \sqrt{x}$.
    \item $\displaystyle f(x) = x\sqrt[3]{x}$.
    \item $\displaystyle f(x) = \frac{1}{1 + x^2}$.
    \item $\displaystyle f(x) = \sen 2x$.
    \item $\displaystyle f(x) = \frac{x + 1}{x - 1}$.
    \item $\displaystyle f(x) = \frac{x^3}{x^2 + 3}$.\\
          \textsc{Sugerencia}: primero realice la división; así $f(x)$ se expresará como la
          suma de un polinomio y una expresión racional más sencilla.
    \item $\displaystyle f(x) = \frac{3x^2 - 5}{2x^2 + 1}$.
    \item $\displaystyle f(x) = \cos 2x$.
    \item $\displaystyle f(x) = x\sqrt{x}$.
    \end{enumerate}

\item Para la función $f$ y $x_0\in\mathbb{R}$ dados, y para $\Delta x \in \{0.001,\ 0.01,\
    0.1,\ 1\}$, calcule $\Delta y$ si $y = f(x)$. Para los mismos valores $\Delta x$ y $x_0$,
    calcule los incrementos relativos $\displaystyle \frac{\Delta x}{x_0}$ y
    $\displaystyle\frac{\Delta y}{y_0}$, donde $y_0 = f(x_0)$.

    Resuma los resultados obtenidos en una tabla en la cual los incrementos relativos se
    expresen porcentualmente.
    \begin{enumerate}[leftmargin=*]
    \item $\displaystyle f(x) = 3x + 5$, $x_0 = -3$.
    \item $\displaystyle f(x) = 2x^2 - 3x + 4$, $x_0 = 1$.
    \item $\displaystyle f(x) = x^3 - 3$, $x_0 = 2$.
    \item $\displaystyle f(x) = \frac{1}{x}$, $x_0 = 1$.
    \item $\displaystyle f(x) = \sen \pi x$, $\displaystyle x_0 = \frac{1}{3}$.
    \item $\displaystyle f(x) = \cos \pi x$, $\displaystyle x_0 = \frac{1}{6}$.
    \end{enumerate}

\item Para la función $f$ dada, observe que la razón de cambio promedio en un intervalo de
    extremos $x_0$ y $x_0 + \Delta x$, $f'(x_0, x_0 + \Delta x)$, aproxima a la razón de cambio
    instantánea en $x_0$, $f'(x_0)$, tomando $10$ valores de $\Delta x$ cercanos a $x_0$.
    \begin{enumerate}[leftmargin=*]
    \item $\displaystyle f(x) = 5x - 3$, $x_0 = 2$.
    \item $\displaystyle f(x) = x^2 - x + 1$, $x_0 = 1$.
    \item $\displaystyle f(x) = x^3 + x + 3$, $x_0 = -1$.
    \item $\displaystyle f(x) = \frac{1}{x}$, $x_0 = 1$.
    \end{enumerate}

\item Si los costos de una fábrica de refrigeradoras que produce $x$ cientos de esos
    electrodomésticos por mes es de $C$ miles de dólares, calcule el costo marginal y el costo
    unitario marginal si en un mes dado se produjeron tres mil aparatos. Se conoce que $x \in
    [10,100]$.
    \begin{enumerate}[leftmargin=*]
    \item $f(x) = 10 + 20x$.
    \item $f(x) = 10 + 20x - 0.000\, 1x^2$.
    \end{enumerate}

\item Si la fábrica del ejercicio precedente vende su producción total a un concesionario a un
    precio $p$ miles de dólares por cada cien refrigeradoras, y si $p(x) = 30 - 0.002x$,
    calcule las funciones de ingreso $I$ y de utilidades $U$, así como las de los ingresos
    marginal $I'$ y de utilidades marginales $U'$. Para un nivel de producción $x_0 = 30$,
    interprete los valores que obtenga para $I'(x_0)$ y $U'(x_0)$. Compare estos valores con
    $I(x_0 + 1) - I(x_0)$ y con $U(x_0 + 1) - U(x_0)$, respectivamente.

\item Si para una empresa con un nivel de producción de $x$ unidades por semana, los costos $C$
    y los ingresos $I$ en miles de dólares están dados por
    \[
      C = 2\,000 x + 6\,000 \yjc I = 10x(1\,000 - x),
    \]
    determine la función de utilidades, así como los costos e ingresos marginales cuando el
    nivel de producción es de $x_0 = 50$. Compare la utilidad marginal con el incremento de las
    utilidades al subir la producción en una unidad.

\item Realice los cálculos pedidos en el ejercicio precedente si $x_0 = 100$ y si $x \in
    [20,110]$:
    \begin{enumerate}[leftmargin=*]
    \item $C(x) = 2x^2 + 650$, $I(x) = 1\,000x - 3x^2$.
    \item $C(x) = x^2 + 30x + 100$, $I(x) = 450x - x^2$.
    \item $C(x) = 2x^3 - 50x^2 + 30x + 12\,000$, $I(x) = x(20\,000 - x)$.
    \end{enumerate}

\item Si a un precio de $p$ dólares la demanda de un bien es de $x$ unidades y si $x = D(p)$,
    halle la elasticidad de la demanda para el precio $p_0$ dado. ?`Cómo varía la demanda si el
    precio baja $5\%$? ?`Y si sube $5\%$?
    \begin{enumerate}[leftmargin=*]
    \item $\displaystyle D(p) = 1\,200 - 5p$, $10 \leq p \leq 200$, $p_0 = 100$.
    \item $\displaystyle D(p) = (30 -p)^2$, $1 \leq p < 30$, $p_0 = 20$.
    \item $\displaystyle D(p) = 1\,000\sqrt{40 - p}$, $1 \leq p < 40$, $p_0 = 10$.
    \item $\displaystyle D(p) = 2\,000 + \frac{1\,000}{\sqrt{3p + 4}}$, $0 < p < 100$, $p_0
        = 20$.
    \end{enumerate}

\item Se lanza un proyectil hacia arriba y la altura $h$ metros es alcanzada luego de $t$
    segundos es de $f(t)$ metros aproximadamente, donde
    \[
      f(t) = 10t - 4.9t^2.
    \]
    Determine cuánto tiempo necesita para llegar al punto más alto y cuánto para regresar al
    punto de partida. Note que el instante en que el proyectil deja de subir es cuando alcanza
    el punto más alto para luego descender.

\item Si el conductor no frena, se ha establecido que en las carreras por las fiestas de Quito,
    el más veloz de los carritos de madera que bajan por la calle Las Casas recorre $x$ metros
    luego de $t$ segundos, donde
    \[
      x = f(t) = 2t + t^2
    \]
    hasta que la velocidad es de $10$ metros por segundo. Luego el movimiento es uniforme. Si
    el recorrido que debe hacer es de $154$ metros, escriba $x$ y la velocidad $v$ metros por
    segundo en función de $t$. ?`En cuánto tiempo llega a la meta?
\end{enumerate}
\endgroup
\end{multicols}
\cleartooddpage[\thispagestyle{empty}]
