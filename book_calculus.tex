\documentclass{book}\usepackage[]{graphicx}\usepackage[]{color}
%% maxwidth is the original width if it is less than linewidth
%% otherwise use linewidth (to make sure the graphics do not exceed the margin)
\makeatletter
\def\maxwidth{ %
  \ifdim\Gin@nat@width>\linewidth
    \linewidth
  \else
    \Gin@nat@width
  \fi
}
\makeatother

\definecolor{fgcolor}{rgb}{0.345, 0.345, 0.345}
\newcommand{\hlnum}[1]{\textcolor[rgb]{0.686,0.059,0.569}{#1}}%
\newcommand{\hlstr}[1]{\textcolor[rgb]{0.192,0.494,0.8}{#1}}%
\newcommand{\hlcom}[1]{\textcolor[rgb]{0.678,0.584,0.686}{\textit{#1}}}%
\newcommand{\hlopt}[1]{\textcolor[rgb]{0,0,0}{#1}}%
\newcommand{\hlstd}[1]{\textcolor[rgb]{0.345,0.345,0.345}{#1}}%
\newcommand{\hlkwa}[1]{\textcolor[rgb]{0.161,0.373,0.58}{\textbf{#1}}}%
\newcommand{\hlkwb}[1]{\textcolor[rgb]{0.69,0.353,0.396}{#1}}%
\newcommand{\hlkwc}[1]{\textcolor[rgb]{0.333,0.667,0.333}{#1}}%
\newcommand{\hlkwd}[1]{\textcolor[rgb]{0.737,0.353,0.396}{\textbf{#1}}}%

\usepackage{framed}
\makeatletter
\newenvironment{kframe}{%
 \def\at@end@of@kframe{}%
 \ifinner\ifhmode%
  \def\at@end@of@kframe{\end{minipage}}%
  \begin{minipage}{\columnwidth}%
 \fi\fi%
 \def\FrameCommand##1{\hskip\@totalleftmargin \hskip-\fboxsep
 \colorbox{shadecolor}{##1}\hskip-\fboxsep
     % There is no \\@totalrightmargin, so:
     \hskip-\linewidth \hskip-\@totalleftmargin \hskip\columnwidth}%
 \MakeFramed {\advance\hsize-\width
   \@totalleftmargin\z@ \linewidth\hsize
   \@setminipage}}%
 {\par\unskip\endMakeFramed%
 \at@end@of@kframe}
\makeatother

\definecolor{shadecolor}{rgb}{.97, .97, .97}
\definecolor{messagecolor}{rgb}{0, 0, 0}
\definecolor{warningcolor}{rgb}{1, 0, 1}
\definecolor{errorcolor}{rgb}{1, 0, 0}
\newenvironment{knitrout}{}{} % an empty environment to be redefined in TeX

\usepackage{alltt}

\usepackage{etex}
\usepackage[latin1]{inputenc}
% \usepackage[utf8]{inputenc}
\usepackage[T1]{fontenc}
%\usepackage{exscale}
\usepackage[english,spanish,es-nolayout]{babel}
\usepackage{array}
\usepackage{graphicx}
\usepackage{amsmath,amssymb,bm,amsthm,multicol,ifthen}
%\usepackage{yhmath}
\usepackage{tikz}
\usepackage[svgnames]{pstricks}
\usepackage{multido,pst-plot,pst-func,pst-node,pst-text}
\usepackage{pst-eucl,pst-solides3d,pstricks-add,pst-barcode}
\usepackage{lscape}
\usepackage{makeidx}
\usepackage{paralist}
\usepackage{subfig}
\usepackage{wrapfig}
\usepackage[spanish]{varioref}
\usepackage[scaled]{helvet}

\usepackage{enumitem}
\usepackage{nextpage}

\usepackage[calcwidth]{titlesec}
\usepackage{fancyhdr}

\usepackage{xr}
\usepackage[paperwidth=18cm,paperheight=26cm,twoside,%
    left=2.2cm,right=1.7cm,top=2.25cm,bottom=1.5cm]{geometry} % con 10pt.
%\usepackage[a4paper,twoside,left=3cm,right=2.5cm,top=2.5cm,bottom=2.5cm]{geometry} %con 11pt

\pagestyle{fancy}
%\makeindex

\input Definiciones/yhmath_restricted.sty
\input Definiciones/macmatjc
\input Definiciones/maccaljc

%\externaldocument[L-]{Limites/Limites}
\bibliographystyle{plain}

\title{Cálculo}
\author{Escuela Politécnica Nacional}
\date{Julio 2010}

\includeonly{%
            Prefacio/Prefacio,
            Limites/Limites,
            Derivadas/DerivadaMotivacion,
            Derivadas/DerivadaConcepto,
            Derivadas/DerivadaAplicaciones,
%            Integrales/IntegralIndefinida,
%            Integrales/IntegralDefinida,
%            Integrales/IntegralAplicaciones,
%            Integrales/IntegralLogaritmoExponencial,
%            Sucesiones/SucesionesSeries,
%            Apendices/TablasIntegracion
            }

\setlength{\headheight}{13.70743pt}

\allowdisplaybreaks



%\usepackage[colorlinks,linkcolor=red]{hyperref}
\makeindex
\IfFileExists{upquote.sty}{\usepackage{upquote}}{}
\begin{document}
\frontmatter

\newlength{\drop}
\newcommand*{\titleTMB}{\begingroup% Three Men in a Boat
\drop=0.1\textheight
\centering
\settowidth{\unitlength}{\LARGE CÁLCULO EN UNA VARIABLE}
\vspace*{\baselineskip}
{\Large\scshape Cuadernos de Matemática}\\[\baselineskip]
{\Large\scshape de la Escuela Politécnica Nacional}\\[7.5\baselineskip]
{\large\scshape G. Rojas - J. C. Trujillo - F. Barba}\\[\baselineskip]
%{\large\scshape G. Rojas}\\[\baselineskip]
\rule{\unitlength}{1.6pt}\vspace*{-\baselineskip}\vspace*{2pt}
\rule{\unitlength}{0.4pt}\\[\baselineskip]
{\LARGE CÁLCULO EN UNA VARIABLE}\\[\baselineskip]
{\itshape Cálculo Diferencial}\\[0.2\baselineskip]
%{\itshape Cálculo Integral}\\[0.2\baselineskip]
\rule{\unitlength}{0.4pt}\vspace*{-\baselineskip}\vspace{3.2pt}
\rule{\unitlength}{1.6pt}\\[\baselineskip]
\par
\vfill
\includegraphics[scale=0.15]{Logos/LogoCiencias.eps}\hspace{4em}
\includegraphics[scale=0.15]{Logos/LogodMatematica.eps}\par
\vspace*{\drop}

\newpage

\raggedright
\vspace*{\baselineskip}
{\large\textbf{Cuaderno de Matemática No. 1}}\\[0.6\baselineskip]
{\large\scshape Cálculo en una variable: Cálculo Diferencial} \\[0.5\baselineskip]
{\large\scshape Germán Rojas I. - Juan Carlos Trujillo - Fabián Barba}
\par

\vspace*{4\baselineskip}

\textbf{Responsable de la Edición}: Juan Carlos Trujillo \\
\textbf{Revisión técnica}: Rolando Sáenz \\
%\textbf{Asistentes}: \\
\textbf{Portada}: Byron Reascos\par

\vspace*{3\baselineskip}

Registro de derecho autoral No. 34995\\
ISBN: 978-9978-383-03-2
%\hspace{0.3cm}
%\begin{pspicture}(3,1in)
% \psbarcode{978-9978-383-03-2}{includetext guardwhitespace}{isbn}
%\end{pspicture}


\vspace*{4\baselineskip}

Publicado por la Unidad de Publicaciones de la Facultad de Ciencias de la Escuela Politécnica Nacional, Ladrón de Guevara E11-253, Quito, Ecuador. \par

\vspace*{3\baselineskip}

\copyright \ Escuela Politécnica Nacional 2010


\endgroup}

\pagestyle{empty}
\titleTMB
\cleartooddpage[\thispagestyle{empty}]
%\clearpage

\pagestyle{fancy}

%\begingroup
%\newcommand{\contentsformat}[1]{%
%\parbox[b]{.75\textwidth}{\filleft\bfseries #1}}
%\titleformat{\chapter}% command
%            [block]% shape
%            {\filleft\fontsize{29.86}{29.86}\selectfont\sffamily}% format
%            {}% label
%            {0pt}% sep
%            {\contentsformat}% before code

\def\contentsname{Tabla de contenidos}
\tableofcontents
\cleartooddpage[\thispagestyle{empty}]
%\endgroup

\include{Prefacio/Prefacio}

\begingroup
\mainmatter
%\input Definiciones/FormatoCapitulos.tex
\input Definiciones/FormatoEncabezados.tex
\parindent 15.0pt
\chapter[Límites]{Límites}

\section{Aproximar}

\textit{Aproximar} es la palabra clave de la ciencia y de la tecnología modernas. En efecto,
consciente o no, el trabajo de todo científico ha sido el de elaborar modelos que se
\textit{aproximan} a una realidad compleja que el científico quiere comprender y explicar. Los
números son, quizás, la primera herramienta que creó el ser humano para la elaboración de dichos
modelos.

A pesar de que se han construido conceptos de un alto nivel de abstracción para la noción de
número, la práctica cotidiana se remite casi exclusivamente a la utilización de números decimales.
Más aún, se utilizan frecuentemente una o dos cifras decimales a lo más. Por ejemplo, en la
representación de cantidades de dinero, se utilizan hasta dos cifras decimales que indican los
centavos. O, si se requiere dividir un terreno de 15 hectáreas entre siete herederos y en partes
iguales, se dividirá el terreno en parcelas de aproximadamente $\frac{15}{7}$ de hectárea y, en la
práctica, cada parcela tendrá unas $2.14$ hectáreas.

En realidad, todo lo que se hace con cualquier número a la hora de realizar cálculos es
\textit{aproximarlo} mediante números decimales. Es así que, cuando en los modelos aparecen números
como $\pi$, $\sqrt{2}$ o $\frac{4}{3}$, en su lugar se utilizan números decimales.

Veamos esta situación más de cerca. Si convenimos en utilizar números decimales con solo dos cifras
después del punto, ?`qué número o números decimales deberíamos elegir para aproximar, por ejemplo,
el número $\frac{4}{3}$? Los siguientes son algunos de los números decimales con dos cifras después
del punto que pueden ser utilizados para aproximar a $\frac{4}{3}$:
\[
1.31; \ 1.32; \ 1.33; \ 1.34; \ 1.35.
\]
Dado que ninguno de ellos es exactamente el número $\frac{4}{3}$, el que elijamos como aproximación
debería ser el \textit{menos diferente} del número $\frac{4}{3}$. Ahora, la \textit{resta} entre
números es el modo como se establece la \textit{diferencia} entre dos números. A esta diferencia le
vamos a considerar como el \textit{error} que se comete al aproximar $\frac{4}{3}$ con un número
decimal con dos cifras después del punto.

Así, si se utilizara $1.31$ como aproximación, el error cometido se calcularía de la siguiente
manera:
\begin{align*}
\frac{4}{3} - 1.31 &= \frac{4}{3} - \frac{131}{100} \\
&= \frac{400}{300} - \frac{393}{300} = \frac{7}{300}.
\end{align*}
Es decir, el error que se comete al aproximar $\frac{4}{3}$ con $1.31$ es $\frac{7}{300}$.

En cambio, si se utilizara $1.32$ para aproximar $\frac{4}{3}$, el error cometido sería:
\begin{align*}
\frac{4}{3} - 1.32 &=\frac{4}{3} - \frac{132}{100} \\
&= \frac{400}{300} - \frac{396}{300} = \frac{4}{300}.
\end{align*}

Vemos que $1.32$ es una \emph{mejor aproximación} de $\frac{4}{3}$ que $1.31$ en el sentido de que
el error que se comete al aproximar $\frac{4}{3}$ con $1.32$ es de $\frac{4}{300}$, que es menor
que $\frac{7}{300}$, el error que se comete al aproximar $\frac{4}{3}$ con $1.31$.

Veamos qué sucede si aproximamos $\displaystyle{\frac{4}{3}}$ con $1.33$. En este caso, el error
cometido sería:
\begin{align*}
\frac{4}{3} - 1.33 &=\frac{4}{3} - \frac{133}{100} \\
&= \frac{400}{300} - \frac{399}{300} = \frac{1}{300}.
\end{align*}

Ésta tercera aproximación es mejor que las dos anteriores, porque el error cometido cuando se
aproxima $\frac{4}{3}$ con $1.33$, que es $\frac{1}{300}$, es menor que el error de aproximar
$\frac{4}{3}$ con $1.31$ o $1.32$.

El error cometido al aproximar $\frac{4}{3}$ con $1.34$ es:
\begin{align*}
\frac{4}{3} - 1.34 &=\frac{4}{3} - \frac{134}{100} \\
&= \frac{400}{300} - \frac{402}{300} = -\frac{2}{300}.
\end{align*}
El signo negativo indica que el número utilizado como aproximación es mayor que el número que se
quiere aproximar. Sin embargo, si no nos interesa saber si el número que aproxima a $\frac{4}{3}$
es mayor o menor que éste, podemos tomar siempre la \textit{diferencia positiva}, es decir, la
diferencia entre el número mayor y menor, la misma que es igual al valor absoluto de la resta de
ambos números (sin importar cuál es el mayor). En este caso, la diferencia positiva es:
\[
\left|\frac{4}{3} - 1.34\right| = \left|-\frac{2}{300}\right| = \frac{2}{300}.
\]
A este número le denominamos \emph{error absoluto} cometido al aproximar $\frac{4}{3}$ con el
número $1.34$.

Análogamente, calculemos el error absoluto que se comete al aproximar $\frac{4}{3}$ con $1.35$:
\begin{align*}
\left|\frac{4}{3} - 1.35\right| &= \left|\frac{4}{3} - \frac{135}{100}\right| \\
&= \left|\frac{400}{300} - \frac{405}{300}\right| = \frac{5}{300}.
\end{align*}

Finalmente, si se toma un número decimal con dos cifras después del punto, mayor que $1.35$ o menor
que $1.31$ para aproximar $\frac{4}{3}$, podemos ver que el error absoluto cometido en esta
aproximación será mayor que cualquiera de las ya obtenidas. Por lo tanto, vemos que $1.33$ es,
efectivamente, la mejor aproximación de $\frac{4}{3}$ con un número decimal con dos cifras
decimales después del punto, porque el error que se comete al hacerlo es menor que el cometido con
cualquier otro número con solo dos cifras después del punto.

Podemos, entonces, decir que la \textit{proximidad} entre dos números se mide a través del valor
absoluto de la diferencia entre ellos. De manera más precisa: si $x$ e $y$ son dos números reales,
entonces la cantidad
\[
|x - y|
\]
mide la proximidad entre $x$ e $y$; y, mientras más pequeña sea esta cantidad, consideraremos a los
números $x$ e $y$ más próximos. Al contrario, si esta diferencia es grande, diremos que los números
son menos próximos.

En lo que sigue, usaremos como sinónimo de \textit{aproximar} la palabra \textit{acercar} y de la
palabra \textit{próximo}, la palabra \emph{cerca}. Así, significará lo mismo la frase ``$x$ está
próximo a $y$'' que la frase ``$x$ está cerca de $y$''.

\section{La recta tangente a una curva}

\begin{wrapfigure}{r}{4.5cm}
\begin{center}
\begin{pspicture}(0,.5)(4,3)
\pscircle[linecolor=gray](2,1.5){1.5}%
\psline(2,1.5)(0.9393,2.5607)% radio
\psline(-.1213,1.5)(2,3.6213)% tangente
\rput[lt](2.05,1.5){\footnotesize{$O$}} % etiqueta del centro del círculo
\rput[l](2.05,3.6213){$t$} % etiqueta de la tangente
\rput[rb](0.9,2.6){\footnotesize{$P$}}
\end{pspicture}
\end{center}
\end{wrapfigure}

El dibujar primero y luego el encontrar la ecuación de la recta tangente a una curva en un punto
dado de ella no es, en absoluto, un problema trivial.

Euclides (325 a.C.-265 a.C.) definió la tangente a una circunferencia como la recta que tocándola
no corta a la circunferencia.
%\footnote{Definición 2 del tercer libro de los \emph{Elementos} de Euclides, traducción
%de María Luisa Puertas Castaños, edición de Gredos, Madrid, 1991.}.
Esto significa que la recta tangente tiene con la circunferencia un único punto en común: el punto
de tangencia. Con regla y compás, el geómetra griego construyó la línea recta tangente $t$ a una
circunferencia en el punto $P$ de la siguiente manera: trazó por el punto $P$ la recta
perpendicular al radio de la circunferencia que tiene por uno de sus extremos el punto $P$.
Demostró, además, que ninguna otra recta se interpondrá en el espacio entre la tangente y la
circunferencia.
%\footnote{Proposición 16
%del tercer libro de los \emph{Elementos}.}.

Esta última propiedad se convirtió, más adelante, en la definición de la recta tangente, pues la
definición de Euclides no describe el caso general. En efecto, en las siguientes figuras se muestra
dos ejemplos de lo afirmado:

\begin{figure}[h]
%
\begin{center}
\subfloat[]{
\begin{pspicture}(0,0)(4,5.2) \psset{plotpoints=200}
\psplot{.5}{3.5}{x 2 sub x 2 sub mul 2.5 add}% y = x^2
\rput[br](.5,4.75){\footnotesize{$y = x^2$}}%
\psplot{2.25}{3.6}{x 2 sub 2 mul 1.5 add}% la tangente y = 2x - 1
\rput[bl](3.7,4.7){\footnotesize{$y = 2x - 1$}} %
\psline(3,2)(3,4.5)% la recta x = 1
\rput[tl](3.1,2){\footnotesize{$x = 1$}} %
\psline{->}(0,2.5)(4,2.5)% eje x
\rput[l](4.1,2.5){\footnotesize{$x$}} %
\psline{->}(2,1.5)(2,4.5)% eje y
\rput[b](2,4.6){\footnotesize{$y$}} %
\rput[l](3.1,3.5){\footnotesize{$(1,1)$}}
\end{pspicture}}
%
\hspace{.175\textwidth}
%
\subfloat[]{
\begin{pspicture}(0,0)(4,5.2)
\psset{plotpoints=200}
\psplot{.8}{3.2}{x 2 sub 3 exp 2.5 add}%
\rput[l](3.3,4.228){\footnotesize{$y = x^3$}}
\psline{->}(0,2.5)(4,2.5)% eje x
\rput[l](4.1,2.5){\footnotesize{$x$}} %
\psline{->}(2,.5)(2,4.5)% eje y
\rput[b](2,4.6){\footnotesize{$y$}} %
\psplot{.8}{3.2}{x .75 mul .75 add}%
\rput[l](3.3,3.15){\footnotesize{$y = 0.75x - 0.25x$}}
\end{pspicture}}
\end{center}
\end{figure}
En el primero, la parábola con ecuación $y = x^2$ y la recta con ecuación $x = 1$ tienen un único
punto de contacto: el de coordenadas $(1,1)$; sin embargo, esta recta no es tangente a la curva en
este punto. Es decir, para ser tangente no es suficiente con tener un único punto en común con la
curva. Por otro lado, puede apreciarse, en la figura, que la propiedad de Euclides --de que ninguna
otra recta se interpondrá en el espacio entre la tangente y la curva-- sí es verdadera en este
caso.

En el segundo ejemplo, se puede observar que la recta tangente a la curva cuya ecuación es $y =
x^3$ en el punto $(0.5,1.25)$ tiene por ecuación $y = 0.75x - 0.25$ (esto se probará más adelante).
Sin embargo, esta recta tiene aún otro punto en común con la curva, sin que por ello deje de ser
tangente en el punto $(0.5,1.25)$. Puede apreciarse que la propiedad de Euclides es verdadera pero
solo en una región cercana al punto de tangencia.

En general, la propiedad: ``entre la curva y la recta tangente, alrededor del punto de tangencia,
no se interpone ninguna recta'' pasó a ser la definición de tangente, la misma que ya fue utilizada
por los geómetras griegos posteriores a Euclides.

Aunque el problema de formular una definición de tangente adecuada para cualquier caso fue resuelto
como se indicó, los matemáticos griegos y los de la edad media no encontraron un método general
para obtener esa recta tangente a cualquier curva y en cualquier punto de ella. Este problema fue
uno de los temas centrales de la matemática en la modernidad: uno de los primeros intentos por
resolverlo fue realizado por el francés Pierre Fermat (1601-1665) en 1636. Poco después, se obtuvo
una solución. Pero ésta trajo consigo un nuevo concepto en las matemáticas: el de \emph{derivada}.
Entre los protagonistas de estos descubrimientos estuvieron el matemático inglés Isaac Newton
(1643-1727) y el alemán Gottfried Wilhelm Leibniz (1646-1716). Con el concepto de derivada se
encontró un método general para obtener la recta tangente a una curva de una clase amplia de
curvas.

Ahora bien, el concepto de derivada descansa sobre otro: el de límite. En los tiempos de Newton y
Leibiniz, este concepto fue tratado de una manera informal, lo que provocó serias críticas y dudas
sobre la validez del método. Tuvieron que transcurrir aproximadamente 150 años para que la
comunidad matemática cuente con el fundamento de la derivada: recién en 1823, el matemático francés
Augustin Cauchy (1789-1857) propuso una definición rigurosa de límite, y ésa es la que usamos hasta
hoy en día. El tema de este capítulo es estudiar, precisamente, esa definición de límite como base
para el estudio tanto del concepto de derivada como del concepto de integral, otro de los grandes
descubrimientos de la modernidad.

Pero antes de ello, vamos a presentar una solución no rigurosa del problema de encontrar la
tangente a una curva cualquiera.

\subsection{Formulación del problema}
Dada una curva general $C$, como la de la figura (a), y un punto $P$ en ella, se busca la recta $t$
tangente a $C$ en el punto $P$:
\begin{figure}[h]
\begin{center}
%
\subfloat[]{%
\begin{pspicture}(-.5,0)(6,4.3)

%La curva y = .08e^(x - .75) + .5
\psplot{0}{4.6}{.08 2.71828 x .75 sub exp mul .5 add}%

%La etiqueta de la curva
\rput[l](4.7,4.259){\footnotesize$C$}

%La marca del punto P
\pscircle[fillstyle=solid,fillcolor=black](1.5,.669){.04}%

%La etiqueta del punto P
\rput[b](1.5,.719){\footnotesize{$P$}}%

%La recta tangente a la curva en P: y = .169x + .415
\psplot{-.5}{5}{.169 x mul .415 add}%

%La etiqueta de la tangente buscada
\rput[l](5.1,1.277){$t$}

\end{pspicture}}
%
\subfloat[]{%
\begin{pspicture}(-.5,0)(6,4.3)
%La curva y = .08e^(x - .75) + .5
\psplot{0}{4.6}{.08 2.71828 x .75 sub exp mul .5 add}%

%La marca del punto P
\pscircle[fillstyle=solid,fillcolor=black](1.5,.669){.04}%

%La etiqueta del punto P
\rput[b](1.5,.719){\footnotesize{$P$}}%

%La recta tangente a la curva en P: y = .169x + .415
\psplot{-.5}{5}{.169 x mul .415 add}%

%La etiqueta de la tangente buscada
\rput[l](5.1,1.277){$t$}

%El punto Q
\pscircle[fillstyle=solid,fillcolor=black](4.5,3.902){.04}%
\rput[r](4.4,3.902){\footnotesize{$Q$}}%

%La secante QP
\psplot[linecolor=gray]{1}{4.9}{1.077 x mul -.947 add}%

%El punto Q1
\pscircle[fillstyle=solid,fillcolor=black](4.2,3.020){.04}%
\rput[l](4.35,3.020){\footnotesize{$Q_1$}}%

%La secante PQ1
\psplot[linecolor=gray]{1}{4.9}{.871 x mul -.637 add}%

%El punto Q2
\pscircle[fillstyle=solid,fillcolor=black](3.8,2.189){.04}%
\rput[l](4.1,2.189){\footnotesize{$Q_2$}}%

%La secante PQ2
\psplot[linecolor=gray]{1}{4.9}{.661 x mul -.322 add}%

%El punto Q3
\pscircle[fillstyle=solid,fillcolor=black](3.3,1.535){.04}%
\rput[l](3.55,1.535){\footnotesize{$Q_3$}}%

%La secante PQ3
\psplot[linecolor=gray]{1}{4.9}{.475 x mul -.043 add}%

%El punto adicional sin nombre
\pscircle[fillstyle=solid](2.6,1.009){.04}%

%La secante PQ3
\psplot[linecolor=gray]{1}{4.9}{.309 x mul .207 add}%

\end{pspicture}}
\end{center}
\end{figure}
\\
Procedamos de la siguiente manera. Imaginemos que un móvil puntual se mueve a lo largo de la curva
$C$ hacia el punto $P$ desde un punto $Q$, distinto de $P$. Sean $Q_1$, $Q_2$ y $Q_3$ algunos de
los puntos de la curva por los que el móvil pasa. Las rectas que unen cada uno de esos puntos y el
punto $P$ son rectas secantes, como se puede observar en la figura (b). El dibujo sugiere que, a
medida que el móvil \emph{está más próximo} al punto $P$, la correspondiente recta secante
\emph{está más próxima} a la recta tangente $t$; lo que se espera es que la recta tangente buscada
sea la recta a la cual se aproximan las rectas secantes obtenidas cuando el móvil se acerque al
punto $P$. A esa recta la llamaremos ``recta límite'' de las secantes.

Pero, ?`qué significa ser la ``recta límite''? Para tratar de encontrar un significado, supongamos
que la curva $C$ está en un plano cartesiano y que su ecuación es
\[
y = g(x),
\]
donde $g$ es una función real. Supongamos también que las coordenadas del punto $P$ son $(a,g(a))$.
Entonces, encontrar la recta tangente a la curva $C$ en el punto $P$ significa conocer la ecuación
de dicha recta en el sistema de coordenadas dado.

Ahora, para obtener la ecuación de una recta es suficiente conocer su pendiente (la tangente del
ángulo que forma la recta con el eje horizontal) y un punto por el que la recta pase. Como la
tangente $t$ debe pasar por $P$, ya tenemos el punto. Busquemos, entonces, la pendiente de la recta
$t$. ?`Cómo? Utilizando las pendientes de las rectas secantes que unen los puntos de la curva por
los que se desplaza el móvil desde $Q$ hacia al punto $P$, pues, así como creemos que las rectas
secantes alcanzarán la recta $t$ como una posición límite, tal vez, las pendientes de las rectas
secantes alcancen un valor límite: la pendiente de la recta tangente. Con esto en mente, calculemos
la pendiente de cualquiera de esas rectas secantes.

Sea $R$ cualquier punto de la curva $C$ que indica la posición del móvil en su trayecto desde $Q$
hasta $P$. Aunque el punto $R$ no se mueve (ni ningún otro punto de la curva), diremos que ``el
punto $R$ se mueve hacia $P$'' para indicar que es el móvil el que se está moviendo. Esto nos
permitirá indicar la posición del móvil a través de las coordenadas de los puntos de la curva. En
este sentido $R$ no indica un único punto, sino todos los puntos por dónde está pasando el móvil en
su camino hacia al punto $P$.

Sea $x$ la abscisa de $R$; entonces, sus coordenadas son $(x,g(x))$:
\begin{center}
\begin{pspicture}(-.5,-.5)(5.5,4)
%La curva y = .08e^(x - .75) + .5
\psplot{0}{4.3}{.08 2.71828 x .75 sub exp mul .5 add}%

%La marca del punto P
\pscircle[fillstyle=solid,fillcolor=black](1.5,.669){.04}%

%La etiqueta del punto P
\rput[b](1.5,.8){\footnotesize{$P$}}%

%El punto R
\pscircle[fillstyle=solid,fillcolor=black](4,2.563){.04}%
\rput[l](4.1,2.563){\footnotesize{$R$}}%

%La secante PR
\psplot[linecolor=gray]{1}{4.6}{.758 x mul -.467 add}%

%Ejes coordenadas
\psaxes[ticks=none,labels=none]{->}(5,3.6)%

%x
\rput[l](5.1,0){$x$}

%y
\rput[b](0,3.7){$y$}

%líneas para $P$
\psline[linecolor=gray,linestyle=dashed](1.5,.669)(1.5,0)%
\rput[t](1.5,-.09){\small{$a$}}%
\psline[linecolor=gray,linestyle=dashed](1.5,.669)(0,.669)%
\rput[r](-.09,.669){\small{$g(a)$}}%

%líneas para $R$
\psline[linecolor=gray,linestyle=dashed](4,2.563)(4,0)%
\rput[t](4,-.09){\small{$x$}}%
\psline[linecolor=gray,linestyle=dashed](4,2.563)(0,2.563)%
\rput[r](-.09,2.563){\small{$g(x)$}}%

%línea horizontal:
\psline[linecolor=gray,linestyle=dashed](1.5,.669)(4,.669)%
\rput[l](4,.75){\footnotesize{$S$}}

\end{pspicture}
\end{center}
Sabemos que la pendiente de la recta que pasa por $P$ y $R$ es igual a la tangente del ángulo que
forma la recta con el eje horizontal; este ángulo mide lo mismo que el ángulo $\angle SPR$ del
triángulo rectángulo $\triangle SPR$. Por lo tanto, la tangente de este ángulo es igual al cociente
entre la longitud $RS$ (que, en el caso de la curva de la figura, es igual a la diferencia $g(x) -
g(a)$) y la longitud $PS$ (que, en este caso, es igual a la diferencia $x - a \neq 0$, pues el
punto $R$ no es igual al punto $P$); es decir, si representamos con $m_x$ la tangente del ángulo
$\angle SPR$, podemos afirmar que:
\begin{equation}
\label{eqPenSec}%
m_x = \frac{g(x)-g(a)}{x-a}.
\end{equation}
La pendiente de cualquier secante que une el punto $R$, cuyas coordenadas son $(x,g(x))$ y que se
está moviendo hacia $P$, y el punto $P$ se calculará mediante la
fórmula~(\ref{eqPenSec})\footnote{La fórmula~(\ref{eqPenSec}) es válida no solamente para curvas
crecientes como la de la figura. Su validez será demostrada cuando se presente una definición
general para la pendiente de la recta tangente a una curva.}.

Ahora, notemos que el móvil ubicado en el punto $R$ se mueve hacia $P$ cuando la abscisa $x$ de $R$
se ``acerca'' hacia la abscisa $a$ del punto $P$. Entonces, el problema de obtener la pendiente de
la recta tangente a la curva $C$ en el punto $P$, en el que se concibe a dicha recta tangente como
la ``recta límite'' de las secantes que pasan por $P$ y $R$, que se aproxima a $P$, se sustituye
por el problema de encontrar un número al que los cocientes
\[
\frac{g(x)-g(a)}{x-a}
\]
se aproximan cuando el número $x$ se aproxima al número $a$. A ese número le llamaremos,
provisionalmente, ``límite de los cocientes''.

?`Y cómo se puede hallar este ``límite''? Como un primer acercamiento a la solución de este nuevo
problema, consideremos un ejemplo. Pero, antes, ampliemos un poco más el significado de
``aproximar'' que se discutió en la primera sección.

\subsection{Aproximación numérica al concepto de límite}
Supongamos que la curva $C$ es una parábola cuya ecuación es $y = 3x^2$ y el punto $P$ tiene
coordenadas $(2,12)$. Entonces, $g(x) = 3x^2$. Queremos calcular el ``límite'' de los cocientes
\[
m_x = \frac{g(x) - g(2)}{x - 2} = \frac{3x^2 - 12}{x - 2}
\]
cuando el número $x$ se aproxima al número $2$, pero $x\neq 2$. Si encontramos ese número
``límite'', lo usaremos como la pendiente de la recta tangente a la curva en el punto $(2,12)$.
Obtendremos luego la ecuación de la recta tangente.

Para empezar, observemos que, como $x\neq 2$, entonces:
\[
m_x = \frac{3x^2 - 12}{x - 2} = \frac{3(x - 2)(x + 2)}{x - 2}.
\]
Por lo tanto:
\[
m_x = 3(x + 2),
\]
para $x\neq 2$.

Para ver qué sucede con $m_x$ cuando el número $x$ se aproxima al número 2, construyamos una tabla
con los valores que $m_x$ toma para valores de $x$ próximos a 2, unos mayores y otros menores que
2:

\begin{wrapfigure}[19]{l}{.35\textwidth}
\centering
\begin{tabular}{|r|r|}\hline
\multicolumn{1}{|c|}{$x$} & \multicolumn{1}{c|}{$m_x$}\\\hline%
3 & 15 \\
2.5 & 13.5 \\
2.1 & 12.3 \\
2.01 & 12.03 \\
2.001 & 12.003 \\
2.0001 & 12.0003 \\
2.00001 & 12.00003 \\
2 & \multicolumn{1}{c|}{No existe} \\
1.99999 & 11.99997 \\
1.9999 & 11.9997 \\
1.999 & 11.997 \\
1.99 & 11.97 \\
1.9 & 11.7 \\
1.5 & 10.5 \\
1 & 9 \\ \hline
\end{tabular}
\end{wrapfigure}
La aproximación de $x$ a $2$ por valores mayores que 2 significa que el punto $R$ se aproxima al
punto $P$ desde la derecha, mientras que la aproximación de $x$ a 2 por valores menores que 2
significa que el punto $R$ se aproxima al punto $P$ desde la izquierda. En los dos casos, se puede
observar que, mientras $x$ está más cerca de $2$, $m_x$ está más cerca de 12; es decir, mientras el
punto $R$ está más cerca del punto $P$, la pendiente de la recta secante que pasa por $P$ y por $R$
está más cerca del número 12. Esta primera evidencia nos sugiere y alienta a pensar que la
pendiente de la recta tangente es igual a 12. Sin embargo, ?`cómo podemos estar seguros? Lo
siguiente nos proporciona una evidencia adicional que nos hace pensar que estamos en lo correcto.

Hemos visto que para todo $x\neq 2$, se verifica que
\[
m_x = 3(x + 2).
\]
Por otro lado, si evaluamos la expresión de la derecha en $x = 2$, obtenemos que:
\[
3(x + 2) = 3(2 + 2) = 12;
\]
que es el valor al que parece aproximarse $m_x$ cuando $x$ se aproxima a $2$.

Todo parece indicar, entonces, que el número $12$ es el ``límite'' de $m_x$ cuando $x$ se aproxima
a 2. Sin embargo, ?`podemos asegurar tal cosa?

Para poder responder esta pregunta, antes que nada necesitamos una definición para el ``límite''.
Ésta llegó en el año 1823 de la mano del matemático francés Augustin Cauchy. En la siguiente
sección vamos a estudiarla y, con ella, podremos asegurar que el límite de $m_x$ cuando $x$ se
aproxima a $2$ es, efectivamente, el número $12$.

Aceptando como verdadero este resultado por el momento, obtengamos la ecuación de la recta tangente
a la curva de ecuación $y = 3x^2$ en el punto $(2,12)$.

Ya sabemos, entonces, que la pendiente de dicha recta es igual a $12$. Recordaremos que la ecuación
de una recta de pendiente $m$ que pasa por un punto de coordenadas $(a,b)$ es
\[
y - b = m(x - a).
\]
En este caso $m = 12$ y $(a,b) = (2,12)$. Entonces, la ecuación de la recta tangente a la curva de
ecuación $y = 3x^2$ que pasa por el punto $(2,12)$ es
\[
y - 12 = 12(x - 2),
\]
que puede ser escrita de la siguiente manera:
\[
y = 12x - 12.
\]
En resumen:
\marcojc{.9}{1.5}{black}{black}{white}{%
La pendiente de la recta tangente a la curva $y = 3x^2$ en el punto $(2,12)$ es igual al límite de
\[
m_x = \frac{3x^2 - 12}{x - 2} = 3(x + 2),
\]
cuando $x$, siendo distinto de $2$, se aproxima a 2. Este límite es igual al número $12$.

La ecuación de la recta tangente es:
\[
y = 12x - 12.
\]
\eijc{-1.25} }

Aparte de la definición de límite, persiste aún otro problema: ?`podemos asegurar que la recta
encontrada es la recta tangente? Es decir, ?`cómo podemos estar seguros de que en el espacio entre
la recta de ecuación $y = 12x - 12$ y la curva $y = 3x^2$ no se interpondrá ninguna otra recta,
alrededor del punto $(2,12)?$

Más adelante, en un capítulo posterior, provistos ya con el concepto de límite dado por Agustin
Cauchy, probaremos que el método seguido para la consecución de la recta tangente es correcto y
general.

\begin{multicols}{2}[\subsection{Ejercicios}]
\begingroup
\small
\begin{enumerate}[leftmargin=*]
\item Sean $C$ una curva cuya ecuación es $y = x^3$, $s$ un número real distinto de $1$ y $m_s$
    la pendiente de la recta secante a $C$ en los puntos de coordenadas $(1,1)$ y $(s,s^3)$.
    \begin{enumerate}[leftmargin=*]
    \item Calcule $m_s$.
    \item Elabore una tabla de dos columnas. En la primera, coloque valores de $s$ cercanos
        a $1$; en la segunda, los valores de $m_s$ correspondientes. Con la ayuda de esta
        tabla, determine un candidato para el valor de la pendiente de la recta tangente a
        la curva $C$ en el punto de coordenadas $(1,1)$.
    \item Use el valor de la pendiente hallado en el literal anterior para escribir la
        ecuación de la recta tangente.
    \item Dibuje la curva $C$, las secantes correspondientes para $s\in\{1.1, 1.5, 2\}$ y
        la recta tangente.
    \end{enumerate}
\item Para cada una de las funciones definidas a continuación, elabore una tabla para los
    valores $f(x_i)$ con $x_i = a \pm 10^{-i}$ con $i\in\{1,2,\ldots, 5\}$. ?`Tiene $f(x)$ un
    ``límite'' cuando $x$ se aproxima a $a$? En otras palabras, ?`existe un número al que $f(x)$
    parece acercarse cuando $x$ toma valores cercanos al número $a$?
    \begin{enumerate}[leftmargin=*]
    \item
    \[
    f(x) =
        \begin{cases}
        3 + 2x - x^2 & \text{si } x < 1, \\
        x^2 - 4x + 7 & \text{si } x > 1,
        \end{cases}
        \quad\text{$a = 1$.}
    \]

    \item
    \[
    f(x) =
        \begin{cases}
        x^2 - 4x + 5 & \text{si } x < 1, \\
        3 & \text{si } x = 1, \\
        x + 3 & \text{si } x > 1,
        \end{cases}
        \quad a = 1.
    \]

    \item
    \[
        f(x) = \frac{3x - 15}{\sqrt{x^2 - 10x + 25}},\quad a = 5.
    \]

    \item
    \[
        f(x) = \frac{\sin(3x)}{2x},\quad a = 0.
    \]

    \item
    \[
        f(x) = \frac{1}{(x - 2)^2},\quad a = 2.
    \]


    \item
    \[
        f(x) = \sin\left(\frac{\pi}{x}\right),\quad a = 0.
    \]

    \end{enumerate}

    ?`Guardan alguna relación el hecho de que la función esté o no definida en $a$ y que parezca
    tener ``límite'' cuando $x$ se aproxima al número $a$?

\item El método utilizado en los ejercicios anteriores para encontrar el ``límite'' de una
    función puede sugerir la no necesidad de elaborar un concepto adecuado para la definición
    del límite y el correspondiente desarrollo de técnicas de cálculo. Sin embargo, la función
    $h$ definida por
    \[
      h(x) = \frac{\sqrt[3]{x^3 + 8} - 2}{x^3}
    \]
    nos alerta sobre el método heurístico de la ``aproximación numérica''. Intente determinar
    un número al que parece acercarse $h(x)$ cuando $x$ toma valores cercanos al número $0$
    utilizando el procedimiento propuesto en el ejercicio anterior.
\end{enumerate}
\endgroup
\end{multicols}

\section{La definición de límite}
Vamos a tratar de precisar lo que queremos decir con ``el límite de los cocientes
\[
m_x = \frac{3x^2 - 12}{x-2}
\]
es $12$ cuando $x$ se aproxima a $2$''.

Para empezar, este cociente no está definido en $x = 2$, pero para todo $x\neq 2$:
\[
m_x = 3(x + 2).
\]
?`Existirá algún $x\neq 2$ para el cual $m_x = 12$? Si así fuera, entonces
\[
3(x + 2) = 12.
\]
De aquí, obtendríamos que
\[
x = 2.
\]
Pero esto es absurdo, pues $x\neq 2$. ?`Qué podemos concluir? Que para $x\neq 2$:
\[
m_x \neq 12.
\]

\label{eqLimBeginPLim}Sin embargo, vimos en la sección anterior, que para valores de $x$ cercanos a
$2$, $m_x$ toma valores cercanos a $12$, aunque nunca tomará el valor $12$. Surge, entonces, la
siguiente pregunta: ?`qué tan cerca de 12 puede llegar $m_x$? En otras palabras, ?`podemos encontrar
valores de $x$, cercanos a 2, para los cuales $m_x$ no difiera de $12$ en alguna cantidad dada; por
ejemplo, que no difiera en más de $10^{-2}$ (es decir, en más de $0.01$)? Y ?`qué tan cerca debe
estar $x$ del número $2$ para que ello ocurra? Para responder esta pregunta, formulemos el problema
con mayor precisión.

En primer lugar, ?`qué queremos decir con ``que $m_x$ no difiera de $12$ en más de $10^{-2}$''? Que
el error de aproximar $12$ con $m_x$, es decir, el valor absoluto de la diferencia entre $m_x$ y
$12$ sea menor que $10^{-2}$. En otras palabras, que se verifique la siguiente desigualdad:
\[
|m_x - 12| < 10^{-2}.
\]

En segundo lugar, ?`existen valores de $x$ para los que se cumple esta desigualdad? Y si existen,
?`qué tan cerca de $2$ deberán estar los $x$? Más aún,  Para responder estas preguntas, primero
notemos que podemos medir la cercanía de $x$ a $2$ mediante el valor absoluto de la diferencia
entre $x$ y $2$:
\[
|x - 2|.
\]
En efecto, mientras más pequeño sea este valor absoluto, $x$ estará más cercano a $2$; por el
contrario, mientras más grande sea, $x$ estará más lejos de $2$. Por ello a $|x - 2|$ nos
referiremos también como ``distancia de $x$ a $2$''.

Notemos también que como $x$ es diferente de $2$, entonces se debe cumplir la desigualdad:
\[
0 < |x - 2|.
\]
En todo lo que sigue, supondremos que $x\neq 2$.

Ahora bien, la pregunta:
\begin{quote}
{\bfseries ?`qué tan cerca debe $x$ estar del número $2$ para asegurar que
\[
\bm{|m_x - 12| < 10^{-2}\text{?}}
\]
}
\end{quote}
equivale a la siguiente:
\begin{quote}
{\bfseries ?`a qué distancia debe estar $x$ de $2$ para asegurar que
\[
\bm{|m_x - 12| < 10^{-2}\text{?}}
\]
}
\end{quote}
A su vez, esta segunda pregunta equivale a esta otra:
\begin{quote}
{\bfseries ?`a qué cantidad debe ser inferior $|x - 2|$ para asegurar que
\[
\bm{|m_x - 12| < 10^{-2}\text{?}}
\]
}
\end{quote}
Y esta tercera pregunta equivale a la siguiente:
\begin{quote}
{\bfseries ?`existe un número $\delta > 0$ tal que
\[
\bf{\text{si } \ 0 < |x - 2| < \delta, \ \text{ entonces } \ |m_x - 12| < 10^{-2}\text{?}}
\]
}
\end{quote}

Responder esta pregunta equivale a resolver el siguiente problema:
\marcojc{.9}{1.5}{black}{black}{white}{%
Si para todo valor de $x\neq 2$ se tiene que
\[
m_x = 3(x + 2),
\]
se busca un número real $\delta > 0$ tal que, si las desigualdades
\begin{equation}
\label{eqLim012}
0 < |x - 2| < \delta
\end{equation}
fueran verdaderas, la desigualdad
\begin{equation}
\label{eqLim013}
|m_x - 12| < 10^{-2}
\end{equation}
también sería verdadera.}

\subsection{Solución del problema} Este es un problema de \emph{búsqueda}: debemos encontrar el
número $\delta$. Para ello, el método que vamos a aplicar consiste en suponer temporalmente que
$\delta$ ya ha sido encontrado; es decir, suponer que si $x$ es un número tal que $x\neq 2$ y que
satisface la desigualdad
\begin{equation}
\label{eqLim002} |x - 2| < \delta,
\end{equation}
entonces, se debe cumplir la desigualdad
\begin{equation}
\label{eqLim003} |m_x - 12| = |3(x+2) - 12| < 10^{-2}.
\end{equation}
A partir de esta última desigualdad vamos a tratar de encontrar propiedades del número $\delta$,
aún desconocido, que nos permitan hallarlo.

?`Qué camino seguir? Para no hacerlo a ciegas, el trabajo que realicemos con la
desigualdad~(\ref{eqLim003}), o con una parte de ella, debe llevarnos de alguna manera a $\delta$;
es decir, debe llevarnos a la desigualdad~(\ref{eqLim002}) o a una similar. Con esto en mente,
empecemos el trabajo con el miembro izquierdo de la desigualdad~(\ref{eqLim003}), en el cual
podemos aplicar propiedades conocidas del valor absoluto de un número:
\begin{align*}
|3(x + 2) - 12| &= |3x + 6 - 12| \\
&= |3x - 6| \\
&= 3|x - 2|.
\end{align*}
Es decir,
\begin{equation}
\label{eqLim004}%
|3(x + 2) - 12| = 3|x-2|.
\end{equation}
Pero hemos supuesto que
\[
|x - 2| < \delta.
\]
Entonces:
\[
3|x - 2| < 3\delta.
\]
Por lo tanto, por la propiedad transitiva de la relación ``menor que'', vemos que esta última
desigualdad y la igualdad~(\ref{eqLim004}) implican una nueva desigualdad:
\begin{equation}
\label{eqLim008}%
|3(x + 2) - 12| < 3\delta.
\end{equation}
En resumen:
\marcojc{.9}{1.5}{black}{black}{white}{%
bajo el supuesto de que existe el número $\delta > 0$, si $x\neq 2$ satisficiera la desigualdad
\[
\tag{\ref{eqLim002}}%
|x - 2| < \delta,
\]
se debería satisfacer la desigualdad
\[
\tag{\ref{eqLim008}}%
|3(x + 2) - 12| < 3\delta.
\]
En otras palabras,
\[
\text{si } \ 0 < |x - 2| < \delta, \ \text{ entonces } \ |3(x + 2) - 12| < 3\delta.
\]
\eijc{-.9}}%
Recordemos que queremos que el número $\delta$ que encontremos nos garantice el cumplimiento de la
desigualdad
\[
\tag{\ref{eqLim003}}%
|3(x + 2) - 12| < 10^{-2}.
\]
Para lograrlo, comparemos entre sí las desigualdades~(\ref{eqLim008}) y (\ref{eqLim003}). ?`Qué
podemos observar? Que si el número $3\delta$ fuera menor o igual que $10^{-2}$, entonces
obtendríamos:
\[
|3(x + 2) - 12| < 3\delta \leq 10^{-2};
\]
es decir:
\[
\text{si } \ 3\delta \leq 10^{-2}, \ \text{ entonces } \ |3(x + 2) - 12| < 10^{-2},
\]
donde la desigualdad de la derecha es la que queremos obtener. Por lo tanto, como la desigualdad
\[
3\delta \leq 10^{-2}
\]
es equivalente a la desigualdad
\[
\delta \leq \frac{10^{-2}}{3},
\]
podemos asegurar que:
\marcojc{.9}{1.5}{black}{black}{white}{%
si se elige el número $\delta > 0$ tal que
\[
\delta \leq \frac{10^{-2}}{3}
\]
y si $x\neq 2$ satisface la desigualdad
\[
\tag{\ref{eqLim002}}
|x - 2| < \delta,
\]
la desigualdad requerida
\[
\tag{\ref{eqLim003}}
|3(x + 2) - 12| < 10^{-2}.
\]
es satisfecha. Es decir,
\[ \text{si } 0 < \delta \leq  \frac{10^{-2}}{3}\ \text{ y } \
0 < |x - 2| < \delta, \ \text{ entonces } \ |3(x + 2) - 12| < 10^{-2}.
\]
\eijc{-1.5}\label{eqLimEndPLim}} Y esto es precisamente lo que queríamos hacer.

Resumamos el procedimiento seguido para la búsqueda de $\delta$. Observemos que consiste de dos
etapas: {\label{eqLim017}\begin{enumerate}
\item La \textit{búsqueda} del número $\delta$. En esta etapa se supone encontrado el número
    $\delta$. Bajo esta suposición, se encuentra uno o más valores candidatos para el número
    $\delta$.
\item La \textit{constatación} de que el valor o valores encontrados para $\delta$ satisfacen,
    efectivamente, las condiciones del problema.
\end{enumerate}}

Para nuestro caso, estas etapas se ejemplifican así:
\begin{enumerate}
\item \textit{Búsqueda}: se supone que existe un número $\delta > 0$ tal que para $x\neq 2$:
\[
\text{si } \ |x - 2| < \delta, \ \text{ entonces } \ |3(x + 2) - 12| < 10^{-2}.
\]
Trabajando con la expresión $|3(x + 2) - 12|$ y bajo las suposiciones de que $|x - 2| < \delta$
y $x\neq 2$, se demuestra que
\[
|3(x + 2) - 12| < 3\delta.
\]
Esta última desigualdad sugiere que el número $\delta > 0$ debe satisfacer la desigualdad:
\[
3\delta \leq 10^{-2},
\]
lo que equivale a sugerir que $\delta$ debe cumplir esta otra desigualdad:
\begin{equation}
\label{eqLim009}%
\delta \leq \frac{10^{-2}}{3}.
\end{equation}

\item \textit{Constatación}: con la elección del número $\delta$ que satisface la desigualdad
    (\ref{eqLim009}) y bajo los supuestos de que $x\neq 2$ y $|x - 2| < \delta$, se verifica el
    cumplimiento de la desigualdad
\[
|3(x + 2) - 12| < 10^{-2}.
\]
\end{enumerate}

La solución que acabamos de encontrar al problema planteado nos permite responder la pregunta:
\begin{quote}
{\bfseries ?`qué tan cerca debe estar $\bm{x}$ del número $\bm{2}$ para asegurar que $\bm{m_x}$
difiera de $\bm{12}$ en menos de $\bm{10^{-2}}$?}
\end{quote}
La respuesta es:
\begin{quote}
{\bfseries si $\bm{x}$ difiere de $\bm{2}$ en menos de $\bm{\displaystyle{\frac{10^{-2}}{3}}}$,
$\bm{m_x}$ difiere de $\bm{12}$ en menos de $\bm{10^{-2}}$.}
\end{quote}

?`Podremos encontrar valores de $x$ cercanos a $2$ que garanticen que $m_x$ difiera de $12$ en una
cantidad aún más pequeña que $10^{-2}$? Por ejemplo, ?`qué difiera en menos de $10^{-6}$? La
respuesta es afirmativa, pues, si repasamos el modo cómo se resolvió este mismo problema para el
caso en que queríamos que $m_x$ difiriera de $12$ en menos de $10^{-2}$, descubriremos lo
siguiente:
\begin{quote}
{\bfseries si $\bm{x}$ difiere de $\bm{2}$ en menos de $\bm{\displaystyle{\frac{10^{-6}}{3}}}$,
$\bm{m_x}$ difiere de $\bm{12}$ en menos de $\bm{10^{-6}}$.}
\end{quote}

Si el lector tiene dudas de este resultado, deberá leer una vez más,
\vpagerefrange{eqLimBeginPLim}{eqLimEndPLim}, el procedimiento para resolver el problema cuando la
diferencia entre $m_x$ y $12$ difería en menos de $10^{-2}$. Cada vez que encuentre un $10^{-2}$,
deberá sustituirlo por un $10^{-6}$. Esto lo convencerá del todo.

Y ahora podemos responder a una pregunta más general:
\begin{quote}
{\bfseries ?`qué tan cerca debe estar $x$ del número $2$ para asegurar que $m_x$ difiera de $12$ en
menos de $\epsilon$?}
\end{quote}
donde $\epsilon$ representa cualquier número positivo. Y la respuesta la encontraremos de manera
idéntica a cómo hemos respondido las dos preguntas anteriores; esa respuesta será la siguiente:
\begin{quote}
{\bfseries si $\bm{x}$ difiere de $\bm{2}$ en menos de $\bm{\displaystyle{\frac{\epsilon}{3}}}$,
$\bm{m_x}$ difiere de $\bm{12}$ en menos de $\bm{\epsilon}$.}
\end{quote}

Como este número $\epsilon$ puede ser cualquier número positivo, puede ser elegido tan pequeño como
queramos. Y lo que ya sabemos es que, en esa situación, si $x$ es tal que
\[
|x - 2| < \frac{\epsilon}{3},
\]
garantizamos que
\[
|m_x - 12| < \epsilon.
\]
Es decir, aseguraremos para tales $x$ que $m_x$ estará tan cerca del número $12$ como queramos.

Es más, podremos afirmar que se garantiza que $m_x$ está tan cerca como se desee del número $12$,
si $x$ está lo suficientemente cerca de $2$. En efecto, en este caso, para que la distancia de
$m_x$ a $12$ sea menor que $\epsilon$, bastará que la distancia de $x$ a $2$ sea menor que
$\frac{\epsilon}{3}$.

Esto también puede ser expresado de la siguiente manera:

\begin{quote}
{\bfseries $\bm{12}$ puede ser aproximado por los cocientes $\bm{m_x}$ con la precisión que se
desee, con la condición de que $\bm{x}$, siendo distinto de $\bm{2}$, esté suficientemente cerca de
$\bm{2}$.}
\end{quote}
Y, cuando una situación así ocurre, siguiendo a Cauchy, diremos que
\begin{quote}
{\bfseries $\bm{12}$ es el límite de $\bm{m_x}$ cuando $\bm{x}$ se aproxima al número $\bm{2}$ y
escribiremos:
\[
\bm{12 =} \bm{\limjc{m_x}{x}{2}}.
\]
}
\end{quote}

Al inicio de la sección, nos habíamos propuesto precisar la frase ``el límite de los cocientes
$m_x$ es $12$ cuando $x$ se aproxima a $2$''. De lo mostrado anteriormente, vemos que esta frase
debe ser cambiada por la siguiente: ``12 es aproximado por los cocientes $m_x$ con la precisión que
se desee, con tal que $x$, siendo distinto de $2$, esté lo suficientemente cerca de $2$''. Y ahora
esta frase tiene pleno sentido.

El proceso seguido para afirmar que $12$ es el límite de $m_x$ puede ser resumido de la siguiente
manera:
\marcojc{.9}{1.5}{black}{black}{white}{%
dado cualquier número $\epsilon > 0$, encontramos un número $\delta > 0$, que en nuestro caso fue
$\frac{\epsilon}{3}$, tal que $12$ puede ser aproximado por $m_x$ con un error de aproximación
menor que $\epsilon$, siempre que $x$, siendo distinto de $2$, se aproxime a $2$ a una distancia
menor que $\delta$.} Y este texto puede ser expresado simbólicamente mediante desigualdades de la
siguiente manera:
\marcojc{.9}{1.5}{black}{black}{white}{%
dado cualquier número $\epsilon > 0$, encontramos un número $\delta > 0$ tal que
\[
|m_x - 12| < \epsilon,
\]
siempre que
\[
0 < |x - 2| < \delta.
\]
\eijc{-1.5}} Y toda esta afirmación se expresa de manera simple por:
\[
12 = \limjc{m_x}{x}{2}.
\]

A partir de este ejemplo vamos a formular una definición general de límite de una función real.

\subsection{La definición de límite}
Sean:
\begin{enumerate}
\item $a$ y $L$ dos números reales,
\item $I$ un intervalo abierto que contiene el número $a$, y
\item $f$ una función real definida en $I$, salvo, tal vez, en $a$; es decir, $I \subset \Dm(f)
    \cup \{a\}$.
\end{enumerate}
El número $a$ generaliza a $2$, $L$ a $12$, $I$ a $(-\infty,+\infty)$ y $f(x)$ a $m_x$.

Lo que vamos a definir es:
\begin{quote}
{\bfseries $\bm{L}$ es el límite de $\bm{f(x)}$ cuando $\bm{x}$ se aproxima a $\bm{a}$.}
\end{quote}
Igual que en el ejemplo, esta frase se deberá entender como:
\begin{quote}
{\bfseries $\bm{L}$ puede ser aproximado por los valores de $\bm{f(x)}$ con la precisión que se
desee, con la condición de que $\bm{x}$, siendo distinto de $\bm{a}$, sea lo suficientemente
cercano a $\bm{a}$.}
\end{quote}
O, de forma equivalente, se entenderá como:
\begin{quote}
{\bfseries $\bm{L}$ es el límite de $\bm{f(x)}$ cuando $\bm{x}$ se aproxima a $\bm{a}$, lo que se
representará por:
\[
\bm{L =} \bm{\limjc{f(x)}{x}{a},}
\]
si para cualquier número $\bm{\epsilon > 0}$, existe un número $\bm{\delta > 0}$ tal que $\bm{L}$
puede ser aproximado por $\bm{f(x)}$ con un error de aproximación menor que $\bm{\epsilon}$,
siempre que $\bm{x}$, siendo distinto de $\bm{a}$, se aproxime a $\bm{a}$ a una distancia menor que
$\bm{\delta}$. }
\end{quote}
Finalmente, todo lo anterior nos lleva a la siguiente definición:

\begin{defical}[Límite de una función]\label{def:Limite} Sean:
\begin{enumerate}
\item $a$ y $L$ dos números reales,
\item $I$ un intervalo abierto que contiene el número $a$, y
\item $f$ una función real definida en $I$, salvo, tal vez, en $a$; es decir, $I \subset \Dm(f)
    \cup \{a\}$.
\end{enumerate}
Entonces:
\[
L = \limjc{f(x)}{x}{a}
\]
si y solo si para todo $\epsilon > 0$, existe un $\delta > 0$ tal que
\[
|f(x) - L| < \epsilon,
\]
siempre que
\[
0 < |x - a| < \delta.
\]
\end{defical}

Veamos algunos ejemplos para familiarizarnos con esta definición.

\begin{exemplo}[Solución]{\label{ex:lim001}%
Probemos que
\[
9 = \limjc{x^2}{x}{-3}.
\]
Es decir, probemos que $9$ puede ser aproximado por valores de $x^2$ siempre y cuando elijamos
valores de $x$ lo suficientemente cercanos a $-3$.}%
Para ello, de la definición de límite, sabemos que, dado cualquier $\epsilon
> 0$, debemos hallar un $\delta
> 0$ tal que
\begin{equation}
\label{eqLim018}
|x^2 - 9| < \epsilon
\end{equation}
siempre que $x \neq -3$ y
\begin{equation}
\label{eqLim019}
|x + 3| < \delta.
\end{equation}

\paragraph{Búsqueda de $\delta$:}
Empecemos investigando el miembro izquierdo de la desigualdad~(\ref{eqLim018}), que mide el error
de aproximar $9$ con $x^2$. Podemos expresarlo así:
\[
|x^2 - 9| = |(x - 3)(x + 3)| = |x - 3||x + 3|.
\]
Entonces, debemos encontrar los $x \neq -3$, pero cercanos a $-3$, para los que se verifique la
desigualdad:
\begin{equation}
\label{eqLim020}
|x - 3||x + 3| < \epsilon,
\end{equation}
que es equivalente a la desigualdad~(\ref{eqLim018}). Es decir, debemos encontrar los $x\neq -3$,
pero cercanos a $-3$, que hacen que el producto
\begin{equation}
\label{eqLim021}
|x - 3||x + 3|
\end{equation}
esté tan cerca de $0$ como se quiera.

Ahora bien, como $x$ está cerca a $-3$, el factor $|x+3|$ estará cerca de $0$. Por lo tanto, para
que el producto~(\ref{eqLim021}) esté tan cerca de $0$ como se quiera, será suficiente con que el
otro factor, $|x - 3|$, no supere un cierto límite; es decir, esté acotado por arriba. Veamos si
ése es el caso. Para ello, investiguemos el factor $|x - 3|$.

Como $x$ debe estar cerca a $-3$, consideremos solamente valores de $x$ que estén en un intervalo
que contenga al número $-3$. Por ejemplo, tomemos valores de $x$ distintos de $-3$ y tales que
estén en el intervalo con centro en $-3$ y radio 1, como el que se muestra en la siguiente figura:
\begin{center}
\begin{pspicture}(-4,-.5)(1,.5)
\psaxes[yAxis=false,labelsep=-20pt]{->}(0,0)(-4,0)(1,0)%
\psframe[hatchcolor=gray,fillstyle=hlines,hatchangle=45,linestyle=none,hatchsep=2pt](-4,-.1)(-2,.1)%
\end{pspicture}
\end{center}
Esto significa que $x$ debe cumplir las siguientes desigualdades:
\[
-4 < x < -2,
\]
que son equivalentes a estas otras:
\begin{equation}
\label{eqLim023}
-1 < x + 3 < 1,
\end{equation}
las que, a su vez, son equivalentes a la siguiente:
\begin{equation}
\label{eqLim022}
|x + 3| < 1.
\end{equation}

Bajo la suposición del cumplimiento de estas desigualdades, veamos si el factor $|x-3|$ está
acotado por arriba. Para ello, construyamos el factor $|x - 3|$ a partir de las
desigualdades~(\ref{eqLim023}).

En primer lugar, para obtener la diferencia $x - 3$, podemos sumar el número $-6$ a los miembros de
estas desigualdades. Obtendremos lo siguiente:
\begin{equation}
\label{eqLim024}
-7 < x - 3 < -5.
\end{equation}
Pero esto significa que, para $x \neq -3$ tal que
\[
\tag{\ref{eqLim022}}
|x + 3| < 1,
\]
la diferencia
\[
x - 3
\]
es negativa, y, por lo tanto:
\[
|x - 3| = -(x - 3);
\]
es decir:
\[
x - 3 = -|x - 3|,
\]
con lo cual podemos reescribir las desigualdades~(\ref{eqLim024}) de la siguiente manera:
\[
-7 < -|x - 3| < -5,
\]
que son equivalentes a las siguientes:
\[
7 > |x - 3| > 5.
\]

Hemos probado que, si $x\neq -3$ es tal que $|x + 3| < 1$, se verifica que el factor $|x - 3|$ está
acotado superiormente por el número $7$:
\[
|x - 3| < 7.
\]

Con este resultado, volvamos al producto~(\ref{eqLim021}). Ya podemos afirmar que, si $x\neq -3$ y
$|x + 3| < 1$, se debe verificar lo siguiente:
\[
|x^2 - 9| = |x - 3||x+3| < 7|x + 3|,
\]
es decir:{\bfseries
\begin{equation}
\label{eqLim025}
\bm{|x^2 - 9| < 7|x + 3|,}
\end{equation}
siempre que
\[
\bm{x\neq -3 \quad\text{y}\quad |x + 3| < 1}.
\]}

Ahora, si elegimos los $x\neq -3$ tales que $|x + 3| < 1$ y
\begin{equation}
\label{eqLim026}
7|x + 3| < \epsilon,
\end{equation}
por la desigualdad~(\ref{eqLim025}), obtendríamos la desigualdad~(\ref{eqLim018}):
\[
|x^2 - 9| < \epsilon,
\]
es decir, haríamos que el error que se comete al aproximar $9$ por $x^2$ sea menor que $\epsilon$.

Pero elegir $x$ de modo que se cumpla la desigualdad~(\ref{eqLim026}) equivale a elegir a $x$ de
modo que se cumpla la desigualdad:
\begin{equation}
\label{eqLim027}
|x + 3| < \frac{\epsilon}{7}.
\end{equation}

Por lo tanto: {\bfseries la desigualdad
\[
\tag{\ref{eqLim018}}
\bm{|x^2 - 9| < \epsilon}
\]
será satisfecha si se eligen los $\bm{x\neq -3}$ tales que se verifiquen, simultáneamente, las
desigualdades:
\[
\bm{|x + 3| < 1 \quad\text{y}\quad |x + 3| < \frac{\epsilon}{7}}.
\]
}

Esto quiere decir que para el $\delta > 0$ buscado, la desigualdad $|x + 3| < \delta$ deberá
garantizar el cumplimiento de estas dos desigualdades. ?`Cómo elegir $\delta$? Pues, como el más
pequeño entre los números $1$ y $\frac{\epsilon}{7}$:
\[
\delta = \min\left\{1,\frac{\epsilon}{7}\right\}.
\]
Al hacerlo, garantizamos que
\[
\delta \leq 1\quad\text{y}\quad \delta \leq \frac{\epsilon}{7},
\]
de donde, si $x\neq -3$ tal que
\[
|x + 3| < \delta,
\]
entonces
\[
|x + 3| < \delta \leq 1 \quad\text{y}\quad |x + 3| < \delta \leq \frac{\epsilon}{7},
\]
con lo cual garantizamos que se verifique la desigualadad~(\ref{eqLim018}):
\[
\tag{\ref{eqLim018}}
|x^2 - 9| < \epsilon.
\]

En resumen:
\marcojc{.9}{1.5}{black}{black}{white}{%
dado $\epsilon > 0$, podemos garantizar que
\[
|x^2 - 9| < \epsilon,
\]
siempre que elijamos $x\neq -3$ tal que
\[
|x + 3| < \min\left\{1,\frac{\epsilon}{7}\right\}.
\]
En otras palabras, el número $9$ puede ser aproximado tanto como se quiera por $x^2$ siempre que
$x$ esté lo suficientemente cerca de $-3$. Esto demuestra que $ 9 = \displaystyle\lim_{x \to -3}{x^2}.$}\vspace*{-1.4\baselineskip}
\end{exemplo}

Antes de estudiar otro ejemplo, tratemos de obtener un procedimiento para demostrar que un número
$L$ es el límite $f(x)$ cuando $x$ se aproxima al número $a$, a partir del que acabamos de utilizar
para encontrar el $\delta$ dado el $\epsilon$.

Si leemos la demostración realizada una vez más, podemos resumir, en los siguientes pasos, el
procedimiento seguido:
\begin{enumerate}
\item El valor absoluto de la diferencia entre $x^2$ y el límite $9$ se expresó como el
    producto de dos factores en valor absoluto:
    \[
        |x^2 - 9| = |x-3||x+3|.
    \]
    Uno de los factores es el valor absoluto de la diferencia entre $x$ y el número $-3$, que
    es el número a dónde se aproxima $x$.

\item Como $|x + 3|$ debe ser menor que el número $\delta$, para que la cantidad
\[
|x^2 - 9|
\]
sea tan pequeña como se desee (es decir, menor que $\epsilon$), se busca una cota superior para
el segundo factor $|x - 3|$. En este ejemplo, se encontró que:
\[
|x - 3| < 7,
\]
bajo el supuesto de que $|x+3| < 1$. Esta última suposición es realizada porque los $x$ deben
estar cerca de $-3$, por lo que se decide trabajar con valores de $x$ que estén en un intervalo
con centro en el número $3$. En este caso, un intervalo de radio $1$.

\item El resultado anterior sirve de prueba de la afirmación:
\[
|x^2 - 9| < 7|x + 3|,
\]
siempre que $0 < |x + 3| < \delta$ y $|x + 3| < 1$.

\item Para obtener $|x^2 - 9| < \epsilon$ siempre que $0 < |x + 3| < \delta$, el resultado
    precedente nos dice cómo elegir el número $\delta$:
\[
\delta = \min\left\{1, \frac{\epsilon}{7}\right\}.
\]
La elección de este $\delta$ muestra que:
\begin{align*}
|x^2 - 9| & = |x - 3||x + 3| \\
& < 7|x + 3|,\quad \text{pues }\ |x + 3| < \delta \leq 1, \\
& < 7 \delta, \quad \text{pues }\ |x + 3| < \delta, \\
& \leq 7\frac{\epsilon}{7} = \epsilon, \quad\text{pues }\ \delta \leq \frac{\epsilon}{7}.
\end{align*}
Por lo tanto:
\[
|x^2 - 9| < \epsilon,
\]
siempre que
\[
0 < |x + 3| < \delta = \min\left\{1, \frac{\epsilon}{7}\right\}.
\]
\end{enumerate}

Generalicemos este procedimiento para probar que
\[
L = \limjc{f(x)}{x}{a}.
\]
\begin{enumerate}
\item Hay que tratar de expresar el valor absoluto de la diferencia entre $f(x)$ y el límite
    $L$ para $x\neq a$ de la siguiente manera:
    \[
        |f(x) - L| = |g(x)||x-a|.
    \]

\item Se busca una cota superior para el factor $|g(x)|$. Supongamos que esta cota superior sea
    el número $M > 0$. Entonces, debe cumplirse la siguiente desigualdad:
    \[
    |g(x)| < M.
    \]
    Probablemente, para encontrar este valor $M$ haya que suponer adicionalmente que
    \[
        0 < |x - a| < \delta_1,
    \]
    con cierto $\delta_1 > 0$. Esta suposición puede ser el resultado de trabajar con valores
    cercanos al número $a$, para lo cual se decide trabajar con valores de $x$ en un intervalo
    con centro en el número $a$.

\item El resultado anterior sirve de prueba de la afirmación:
\[
|f(x) - L| < M|x - a| < \epsilon,
\]
siempre que $0 < |x - a| < \frac{\epsilon}{M}$ y $|x - a| < \delta_1$.

\item El resultado precedente nos dice cómo elegir el número $\delta$:
\[
\delta = \min\left\{\delta_1, \frac{\epsilon}{M}\right\}.
\]
La elección de este $\delta$ muestra que:
\begin{align*}
|f(x) - L| & = |g(x)||x - a| \\
& < M|x - a|,\quad \text{pues }\ |g(x)| < M \ \text{debido a que} \ |x - a| < \delta \leq \delta_1, \\
& < M \delta, \quad \text{pues }\ |x - a| < \delta, \\
& \leq M\frac{\epsilon}{M} = \epsilon, \quad\text{pues }\ \delta \leq \frac{\epsilon}{M}.
\end{align*}
Por lo tanto:
\[
|f(x) - L| < \epsilon,
\]
siempre que
\[
0 < |x - a| < \delta = \min\left\{\delta_1, \frac{\epsilon}{M}\right\}.
\]
\end{enumerate}

Apliquemos este procedimiento en el siguiente ejemplo.

\begin{exemplo}[Solución]{Demostremos que
\[
2 = \limjc{\frac{5x - 3}{x + 3}.}{x}{3}
\]
Es decir, probemos que el número $2$ puede ser aproximado por valores de
\[
\frac{5x - 3}{x + 3}
\]
siempre que $x\neq 3$ esté lo suficientemente cerca de $3$.} Dado $\epsilon > 0$, debemos encontrar
un número $\delta > 0$ tal que
\begin{equation}
\label{eqLim028}
\left|\frac{5x-3}{x+3} - 2\right| < \epsilon
\end{equation}
siempre que $x\neq 3$ y
\begin{equation}
\label{eqLim029}
|x - 3| < \delta.
\end{equation}

Vamos a aplicar el procedimiento descrito. Lo primero que tenemos que hacer es expresar el valor
absoluto de la diferencia entre $\frac{5x-3}{x+3}$ y $2$ para $x\neq 2$, de la siguiente manera:
\begin{equation}
\label{eqLim030}
\left|\frac{5x-3}{x+3} - 2\right| = |g(x)||x - 3|.
\end{equation}

Para ello, trabajemos con el lado izquierdo de la desigualdad~(\ref{eqLim028}). Éste puede ser
simplificado de la siguiente manera:
\begin{align*}
\left|\frac{5x-3}{x+3} - 2\right| &= \left|\frac{5x-3 - 2(x+3)}{x+3}\right| \\
&= \left|\frac{3x-9}{x+3}\right| = 3\left|\frac{x-3}{x+3}\right|\\
&= 3\frac{|x-3|}{|x+3|}.
\end{align*}
Es decir:
\begin{equation}
\label{eqLim036}
\left|\frac{5x-3}{x+3} - 2\right| = \frac{3}{|x+3|}|x-3|.
\end{equation}

Entonces, si definimos:
\[
g(x) = \frac{3}{x+3},
\]
ya tenemos la forma~(\ref{eqLim030}).

Lo segundo que hay que hacer es encontrar una cota superior para $g(x)$. Es decir, debemos hallar
un número positivo $M$ y, posiblemente, un número positivo $\delta_1 > 0$ tales que:
\begin{equation}
\label{eqLim033}
\frac{3}{|x + 3|} < M,
\end{equation}
siempre que
\begin{equation*}
0 < |x - 3| < \delta_1.
\end{equation*}

Para ello, como los $x$ deben tomar valores cercanos al número $3$, consideremos valores para $x$
que estén en el intervalo con centro en $3$ y radio $1$:
\begin{center}
\begin{pspicture}(0,-.5)(5,.5)
\psaxes[yAxis=false,labelsep=-20pt]{->}(0,0)(0,0)(5,0)%
\psframe[hatchcolor=gray,fillstyle=hlines,hatchangle=45,linestyle=none,hatchsep=2pt](2,-.1)(4,.1)%
\end{pspicture}
\end{center}
Esto significa que $x$ debe satisfacer las siguientes igualdades:
\begin{equation}
\label{eqLim031}
2 < x < 4,
\end{equation}
que son equivalentes a estas otras:
\[
-1 < x - 3 < 1,
\]
que, a su vez, son equivalentes a la siguiente desigualdad:
\begin{equation}
\label{eqLim032}
|x - 3| < 1.
\end{equation}

Lo que vamos a hacer a continuación es encontrar $M$ reconstruyendo $g(x)$ a partir de las
desigualdades~(\ref{eqLim031}). Para ello, sumemos el número $3$ a los miembros de estas
desigualdades. Obtendremos lo siguiente:
\begin{equation}
\label{eqLim034}
5 < x + 3 < 7.
\end{equation}
Esto significa que $x + 3 > 0$. Por lo tanto:
\[
x + 3 = |x + 3|,
\]
con lo que las desigualdades~(\ref{eqLim034}) se pueden reescribir así:
\begin{equation}
\label{eqLim035}
5 < |x + 3| < 7.
\end{equation}
Además, como $|x + 3| > 0$, existe el cociente
\[
\frac{1}{|x + 3|},
\]
y las desigualdades~(\ref{eqLim035}) son equivalentes a las siguientes:
\[
\frac{1}{5} > \frac{1}{|x + 3|} > \frac{1}{7}.
\]
Ahora, si multiplicamos por $3$ cada miembro de estas desigualdades, obtenemos que:
\[
\frac{3}{5} > \frac{3}{|x + 3|} > \frac{3}{7}.
\]
Es decir:
\[
|g(x)| = \frac{3}{|x + 3|} < \frac{3}{5},
\]
siempre que
\[
|x - 3| < 1.
\]
Acabamos de encontrar el número $M$, que es igual a $\frac{3}{5}$; y el número $\delta_1$, que es
igual a $1$.

Con este resultado, volvamos al producto~(\ref{eqLim036})\vpageref{eqLim036}. Ya podemos afirmar
que, si $x$ es tal que $|x - 3| < 1$, entonces debe satisfacer lo siguiente:
\[
\left|\frac{5x-3}{x+3} - 2\right| = \frac{3}{|x+3|}|x-3| < \frac{3}{5}|x - 3|,
\]
es decir: {\bfseries
\begin{equation}
\label{eqLim037}
\bm{\left|\frac{5x-3}{x+3} - 2\right| < \frac{3}{5}|x - 3| < \epsilon},
\end{equation}
siempre que
\[
\bm{|x - 3| < 1} \text{ \ \textbf{y} \ } \bm{0 < |x - 3| < \frac{5}{3}\epsilon}.
\]
}

La desigualdad~(\ref{eqLim037}) nos dice como elegir el $\delta$ buscado:
\[
\delta = \min\left\{1,\frac{5}{3}\epsilon\right\}.
\]

Esta elección de $\delta$ nos asegura que el valor absoluto de la diferencia entre
\[
\frac{5x - 3}{x + 3}
\]
y el número $2$ es menor que $\epsilon$. En efecto:
\begin{align*}
\left|\frac{5x - 3}{x + 3} - 2\right| &= \frac{3}{|x + 3|}|x - 3| \\
& < \frac{3}{5}|x - 3|, \quad\text{pues, al tener}\ |x - 3| < \delta \leq 1, \ \text{se tiene que }\
   \frac{3}{|x + 3|} < \frac{3}{5}, \\
& < \frac{3}{5}\delta, \quad\text{pues }\ |x - 3| < \delta, \\
& \leq \frac{3}{5}\times\frac{5}{3}\epsilon = \epsilon, \quad\text{pues }\ \delta \leq \frac{5}{3}\epsilon.
\end{align*}
Por lo tanto: {\bfseries la desigualdad
\[
\bm{\left|\frac{5x - 3}{x + 3} - 2\right| < \epsilon}
\]
se verifica si $\bm{x}$ satisface las desigualdades
\[
\bm{0 < |x - 3| < \delta = \min\left\{1,\frac{5}{3}\epsilon\right\}}.
\]}

Hemos demostrado, entonces, que el número $2$ es el límite de
\[
\frac{5x - 3}{x + 3}
\]
cuando $x$ se aproxima al número $3$.
\end{exemplo}

El procedimiento encontrado funcionó. Vamos a seguir aplicándolo en algunos ejemplos adicionales,
los mismos que nos servirán más adelante para obtener un método para calcular límites sin la
necesidad de recurrir todas las veces a la definición.

\begin{exemplo}[Solución]{Sea $a > 0$. Probemos que
\[
\limjc{\sqrt{x}}{x}{a} = \sqrt{a}.
\]
Es decir, demostremos que el número $\sqrt{a}$ puede ser aproximado por valores de $\sqrt{x}$
siempre que $x$ esté lo suficientemente cerca de $a$.} Sea $\epsilon > 0$. Buscamos $\delta > 0$
tal que
\[
|\sqrt{x} - \sqrt{a}| < \epsilon,
\]
siempre que $|x - a| < \delta$ y $x > 0$.

Para empezar, encontremos $g(x)$ tal que, para $x\neq a$ y $x > 0$,
\begin{equation}
\label{eqLim039}
|\sqrt{x} - \sqrt{a}| = |g(x)||x - a|.
\end{equation}

Para obtener el factor $|x - a|$ a partir de la diferencia
\[
|\sqrt{x} - \sqrt{a}|,
\]
utilicemos el factor conjugado de esta diferencia; es decir, el número:
\[
\sqrt{x} + \sqrt{a}.
\]
Como éste es estrictamente mayor que $0$, podemos proceder de la siguiente manera:
\begin{align*}
|\sqrt{x} - \sqrt{a}| &=
|\sqrt{x} - \sqrt{a}|\times\frac{|\sqrt{x} + \sqrt{a}|}{|\sqrt{x} + \sqrt{a}|} \\
&= \frac{|(\sqrt{x})^2 - (\sqrt{a})^2|}{|\sqrt{x} + \sqrt{a}|} \\
&= \frac{|x - a|}{\sqrt{x} + \sqrt{a}}.
\end{align*}
Por lo tanto:
\begin{equation}
\label{eqLim038}
|\sqrt{x} - \sqrt{a}| = \frac{1}{\sqrt{x} + \sqrt{a}}|x - a|.
\end{equation}
Si definimos
\[
g(x) = \frac{1}{\sqrt{x} + \sqrt{a}},
\]
ya tenemos la igualdad~(\ref{eqLim039}).

Ahora debemos acotar superiormente $g(x)$. Esto no es muy difícil, ya que, como $\sqrt{a} > 0$ y
$\sqrt{x} > 0$, entonces
\[
\sqrt{x} + \sqrt{a} > \sqrt{a},
\]
de donde se obtiene que, para todo $x > 0$, se verifica la desigualdad
\[
\frac{1}{\sqrt{x} + \sqrt{a}} < \frac{1}{\sqrt{a}}.
\]
Por lo tanto:
\[
|g(x)| < \frac{1}{\sqrt{a}}
\]
para todo $x > 0$.

Ya tenemos $M$:
\[
M = \frac{1}{\sqrt{a}}.
\]
Entonces:
\[
|\sqrt{x} - \sqrt{a}| < \frac{1}{\sqrt{a}}|x - a|,
\]
para todo $x > 0$. Por ello, si para un $\delta > 0$, se tuviera que
\[
|x - a| < \delta,
\]
entonces se cumpliría que:
\[
|\sqrt{x} - \sqrt{a}| < \frac{1}{\sqrt{a}}|x - a| < \frac{\delta}{\sqrt{a}}.
\]

De esta desigualdad se ve que podemos elegir $\delta$ de la siguiente manera:
\[
\frac{\delta}{\sqrt{a}} = \epsilon.
\]
Es decir:
\[
\delta = \epsilon\sqrt{a}.
\]

En efecto:
\begin{align*}
|\sqrt{x} - \sqrt{a}| &= \frac{1}{\sqrt{x} + \sqrt{a}}|x - a| \\
& < \frac{1}{\sqrt{a}}|x - a| \\
& < \frac{1}{\sqrt{a}}\delta \\
& = \frac{\epsilon\sqrt{a}}{\sqrt{a}} = \epsilon.
\end{align*}
Por lo tanto: {\bfseries la desigualdad
\[
\bm{|\sqrt{x} - \sqrt{a}| < \epsilon}
\]
se verifica si $\bm{x}$ satisface las desigualdades
\[
0 < \bm{|x - a| < \delta = \epsilon\sqrt{a}}.
\]
}
Hemos probado, entonces, que el número $\sqrt{a}$ es el límite de $\sqrt{x}$ cuando $x$ se
aproxima al número $a$.
\end{exemplo}

Observemos que, en este ejemplo, no hemos seguido, exactamente, el procedimiento desarrollado en
los anteriores. Esto se debe a que, para este caso, se cumple que
\begin{equation}
\label{eqLim043}
|f(x) - L| \leq M|x - a|
\end{equation}
para todos los elementos del dominio de la función $f$ (con $M = \frac{1}{\sqrt{a}}$). Eso
significa que la función $g$ del método es menor o igual que la constante $M$ en todo el dominio de
$f$. Esto significa, entonces, que no hay necesidad de buscar un $\delta_1$ para acotar $g$. A su
vez, esto permite que el número $\delta$ sea definido de la siguiente manera:
\[
\delta = \frac{\epsilon}{M}.
\]

Como se puede ver, esta versión del método para hallar $\delta$ dado el $\epsilon$ funcionará
siempre que se verifique la desigualdad~(\ref{eqLim043}).

\subsection{Dos observaciones a la definición de límite}
\subsubsection{Primera observación: delta depende de epsilon}
Tanto en la definición de límite como en los ejemplos desarrollados anteriormente, se puede
observar que el número $\delta$ depende del número $\epsilon$. Es decir, si el valor de $\epsilon$
cambia, también lo hace $\delta$. Por ejemplo, para probar que
\[
9 = \limjc{x^2}{x}{-3},
\]
encontramos que
\[
\delta = \min\left\{1,\frac{\epsilon}{7}\right\}.
\]
Así, si tomamos $\epsilon = 14$, entonces:
\[
\delta = \min\left\{1,\frac{14}{7}\right\} = \min\{1,2\} = 1.
\]

En cambio, si $\epsilon = \frac{7}{2}$, entonces:
\[
\delta = \min\left\{1,\frac{7}{2\times 7}\right\} = \min\{1,\frac{1}{2}\} = \frac{1}{2}.
\]

El significado del primer caso es que para que $9$ pueda ser aproximado por $x^2$ de modo que el
error de aproximación sea menor que $14$, es suficiente con elegir que $x$ esté a una distancia de
$-3$ menor que $1$. En el segundo caso, para lograr que $9$ sea aproximado por $x^2$ con un error
más pequeño que $\frac{7}{2}$ es suficiente con que $x$ esté a una distancia de $-3$ menor que
$\frac{1}{2}$.

Para recordar esta dependencia del número $\delta$ del número $\epsilon$, se suele escribir, en la
definición de límite, $\delta(\epsilon)$ en lugar de solo escribir $\delta$. En este libro, en
aquellas situaciones en las que tener en cuenta esta dependencia sea crítico, escribiremos delta
seguido de epsilon entre paréntesis.

Veamos un ejemplo más donde obtenemos $\delta$ dado un $\epsilon$, en el que se aprecia, una vez
más, cómo $\delta$ depende de $\epsilon$.

\begin{exemplo}[Solución]{%
Sea $f$ una función de $\mathbb{R}$ en $\mathbb{R}$ definida por:
\[
	f(x)=
\begin{cases}
x^2 + 1 & \text{si $x<1$,} \\
-2x^2 + 8x - 4 & \text{si $x>1$}.
\end{cases}
\]
Demuestre que $\displaystyle\lim_{x\to 1}f(x) = 2$.}%
El siguiente es un dibujo del gráfico de $f$ en el intervalo $[-1.5,3]$:
\begin{center}
\psset{xAxisLabel={},yAxisLabel={},plotpoints=1000}%
\def\f{x dup mul 1 add}
\def\g{2 neg x dup mul mul 8 x mul add 4 sub}

\begin{psgraph}[arrows=->](0,0)(-1.75,-0.5)(3.5,4.5){0.5\textwidth}{5cm}
  \uput[-90](3.5,0){$x$}%
  \uput[0](0,4.5){$y$}%

  \psline*[linecolor=lightgray]
    (0.9,0)(! 0.9 /x 0.9 def \f)(! 0 /x 0.9 def \f)(! 0 /x 1.075 def \g)%
    (! 1.075 /x 1.075 def \g)(1.075,0)%

  \psplot{-1.5}{1}{\f}%
  \psplot[arrows=o-]{1}{3}{\g}%

\end{psgraph}
\end{center}
Como se puede observar, $f(x)$ está tan cerca del número $2$ como se quiera si $x$ está lo
suficientemente cerca de $1$. A través de la definición de límite, vamos a demostrar que esta
conjetura es verdadera.

Sea $\epsilon > 0$. Debemos hallar $\delta > 0$ tal que
\begin{equation}
\label{eqLim048}
|f(x)-2|< \epsilon
\end{equation}
siempre que $x \neq 1$ y $|x-1| < \delta$.

Utilicemos el método descrito en esta sección para encontrar $\delta$ dado $\epsilon$. Lo primero
que tenemos que hacer es encontrar una función $g$ para expresar el valor absoluto de la diferencia
entre $f(x)$ y $2$, para $x\neq 1$, de la siguiente manera:
\begin{equation}
\label{eqLim049}
|f(x) - 2| = |g(x)||x - 1|.
\end{equation}
Para ello, debemos trabajar con el lado izquierdo de la desigualdad~(\ref{eqLim048}). Pero, como
$f$ está definida por dos fórmulas ---una para valores menores que $1$ y otra para valores mayores
que $1$---, vamos a dividir el análisis en dos casos: cuando $x < 1$ y cuando $x > 1$.

Antes de estudiar cada caso, como el límite que nos interesa es cuando $x$ tiende a $1$, nos
interesan únicamente valores cercanos a $1$. Por ello, en todo lo que sigue, suponemos que $x$ toma
valores en el intervalo de centro $1$ y radio $1$; es decir, suponemos que $x$ satisface la
desigualdad $|x-1| < 1$, que es equivalente a $0 < x < 2$.

Por lo tanto, los dos casos a analizar son: $0 < x < 1$ y $1 < x < 2$.

\paragraph{Caso 1: $0 < x < 1$.}
Analicemos el lado izquierdo de la desigualdad~(\ref{eqLim048}). Recordemos que para $x < 1$ se
tiene que $f(x) = x^2 + 1$. Por lo tanto:
\begin{align*}
	|f(x)-2| &= |(x^2+1)-2| = |(x^2 + 1) - 2| \\
  &= |x^2 - 1| = |x+1||x-1|.
\end{align*}
Como $x > 0$, entonces $x + 1 > 0$, de donde
\[
|f(x) - 2| = (x + 1)|x - 1|.
\]

Entonces, si definimos $g(x) = x + 1$, ya tenemos la forma~(\ref{eqLim049}).

Lo que ahora debemos hacer es encontrar una cota superior para $g(x)$. En este caso, esto es
sencillo, pues, como $x < 1$, entonces $g(x) = x + 1 < 2$. Por lo tanto, tenemos que
\[
	|f(x)-2| = g(x)|x - 1| < 2|x-1|.
\]

En resumen, si $0 < x < 1$, entonces
\begin{equation}
\label{eqLim050}
|f(x) - 2| < 2|x - 1|.
\end{equation}

Ahora bien, si el miembro de la derecha de la expresión anterior fuera menor que $\epsilon$, el de
la izquierda también lo sería; es decir, se verifica la siguiente implicación lógica:
\[
	2|x-1|<\epsilon \quad \Rightarrow \quad |f(x)-2|< 2|x - 1| < \epsilon,
\]
la misma que puede ser expresada de la siguiente manera:
\[
	|x-1| < \frac{\epsilon}{2} \quad \Rightarrow \quad |f(x)-2| < \epsilon.
\]
Vemos, entonces, que, si $0 < x < 1$, un buen candidato para $\delta$ es $\frac{\epsilon}{2}$.

Sin embargo, recordemos que supusimos que también se verifica la condición $|x - 1| < 1$. Por lo
tanto, el número $1$ también es un candidato para $\delta$. ?`Cuál de los dos debemos elegir? El más
pequeño.

Así, para este primer caso, la elección para $\delta$ es la siguiente:
\begin{equation}
\label{eqLim051}
\delta = \min\left\{1,\frac{\epsilon}{2}\right\}.
\end{equation}

\paragraph{Caso 2: $1 < x < 2$.}
Puesto que, para estos valores de $x$, se tiene que $f(x) = -2x^2 + 8x - 4$, el lado izquierdo de
la desigualdad~(\ref{eqLim048}) puede ser expresado de la siguiente manera:
\begin{align*}
|f(x) - 2| &= |-2x^2 + 8x - 4 - 2| \\
  &= |2x^2 - 8x + 6| \\
  &= 2|x^2 - 4x + 3| = 2|x - 3||x - 1|.
\end{align*}

Si definimos $g(x) = 2|x - 3|$, tenemos que:
\[
|f(x) - 2| < g(x)|x - 1|.
\]

Ahora encontremos una cota superior para $g(x)$. Para ello, recordemos que $1 < x < 2$. De estas
desigualdades, tenemos que:
\[
1 - 3 < x - 3 < 2 - 3.
\]
Es decir, se verifica que
\[
-2 < x - 3 < - 1 < 2.
\]
Por lo tanto:
\[
|x - 3| < 2 \yjc g(x) = 2|x - 3| < 4.
\]

Entonces, para $x$ tal que $1 < x < 2$, se verifica la desigualdad:
\[
|f(x) - 2| < 4|x - 1|.
\]

De esta desigualdad tenemos la siguiente implicación:
\[
	4|x-1|<\epsilon \quad \Rightarrow \quad |f(x)-2|< 4|x - 1| < \epsilon,
\]
la misma que puede ser expresada de la siguiente manera:
\[
	|x-1| < \frac{\epsilon}{4} \quad \Rightarrow \quad |f(x)-2| < \epsilon.
\]

Con un razonamiento similar al caso anterior, la elección de $\delta$ es la siguiente:
\begin{equation}
\label{eqLim052}
\delta = \min\left\{1,\frac{\epsilon}{4}\right\}.
\end{equation}

\paragraph{Conclusión.}
De los dos casos estudiados, podemos concluir que, si $0 < x < 2$, el $\delta$ buscado debe ser
elegido de la siguiente manera:
\[
\delta = \min\left\{1,\frac{\epsilon}{2}, \frac{\epsilon}{4}\right\}.
\]
Pero, como
\[
\frac{\epsilon}{4} < \frac{\epsilon}{2},
\]
si
\[
\delta = \min\left\{1,\frac{\epsilon}{4}\right\},
\]
se verifica la siguiente implicación:
\[
	0 < |x-1| < \delta \quad \Rightarrow \quad |f(x)-2| < \epsilon.
\]
Esto prueba que
\[
2 = \limjc{f(x)}{x}{1}.
\]
\end{exemplo}

\subsubsection{Segunda observación: no importa qué valor tome la función en el punto donde se calcula el límite} En
efecto, la definición de
\[
L = \limjc{f(x)}{x}{a}
\]
no exige que el número $a$ esté en el dominio de la función $f$. De hecho, en los ejemplos
anteriores, se puede ver que los $x$ que se utilizan para aproximar $L$ con $f(x)$ siempre son
distintos de $a$, pues estos $x$ satisfacen la desigualdad:
\[
0 < |x - a|.
\]

Los siguientes ejemplos nos muestran porqué no es necesario tomar en cuenta al número $a$ en la
definición de límite.

\paragraph{1.}Sea $\funcjc{f}{\mathbb{R}}{\mathbb{R}}$ la función definida por
\[
f(x) = 2x + 1.
\]
Entonces:
\[
3 = \limjc{f(x)}{x}{1}.
\]
En efecto: sea $\epsilon > 0$. Debemos encontrar un número $\delta > 0$ tal que se verifique la
igualdad
\begin{equation}
\label{eqLim042}
|f(x) - 3| < \epsilon,
\end{equation}
siempre que
\[
0 < |x - 1| < \delta.
\]

Para hallar el número $\delta$, ya sabemos qué hacer. En primer lugar, tenemos que:
\begin{align*}
|f(x) - 3| &= |(2x + 1) - 3| \\
&= |2x - 2| \\
&= 2|x - 1|.
\end{align*}
Es decir, se verifica que:
\begin{equation*}
|f(x) - 3| < 2|x-1|.
\end{equation*}
Por lo tanto, si existiera el número $\delta$, debería satisfacerse la desigualdad:
\begin{equation*}
|f(x) - 3| < 2|x-1| < 2\delta.
\end{equation*}
De esta última desigualdad, se ve que, para que se verifique la desigualdad~(\ref{eqLim042}), basta
elegir el número $\delta$ tal que
\[
2\delta = \epsilon,
\]
es decir, tal que
\[
\delta = \frac{\epsilon}{2}.
\]

En efecto: si $x$ es tal que
\[
0 < |x - 1| < \delta = \frac{\epsilon}{2},
\]
entonces:
\begin{align*}
|f(x) - 3| &= 2|x - 1| \\
&< 2\delta \\
&= 2\frac{\epsilon}{2} = \epsilon.
\end{align*}

\paragraph{2.} Sea $\funcjc{g}{\mathbb{R}}{\mathbb{R}}$ la función definida por
\[
g(x) =
\begin{cases}
2x + 1 & \text{si } x\neq 1, \\
1 & \text{si } x = 1.
\end{cases}
\]
Entonces:
\[
g(x) = f(x)
\]
para todo $x\neq 1$. Es decir, $g$ y $f$ son casi la misma función, excepto por el valor que cada
una de ellas toma en el número $x = 1$, pues
\[
g(1) = 1\yjc f(1) = 3.
\]
Sin embargo:
\[
3 = \limjc{g(x)}{x}{1}.
\]

En efecto: sea $\epsilon > 0$. Debemos hallar un número $\delta > 0$ tal que
\begin{equation}
\label{eqLim044}
|g(x) - 3| < \epsilon,
\end{equation}
siempre que
\[
0 < |x - 1| < \delta.
\]

Ahora bien, encontrado el número $\delta$, lo que tenemos que probar es que si
\[
0 < |x - 1| < \delta,
\]
se verifica la desigualdad~(\ref{eqLim044}). Es decir, debemos probar que para $x \neq 1$, pues $|x
- 1| > 0$, tal que $|x - 1| < \delta$, se verifica la desigualdad~(\ref{eqLim044}). Pero,
recordemos que
\[
g(x) = f(x)
\]
para todo $x\neq 1$. Entonces, para estos $x$, la desigualdad~(\ref{eqLim044}) se transforma en la
desigualdad:
\[
\tag{\ref{eqLim042}}
|f(x) - 3| < \epsilon.
\]
Y ya probamos, en el ejemplo anterior, que esta desigualdad se cumple siempre que
\[
0 < |x - 1| < \delta = \frac{\epsilon}{2}.
\]
De manera que para todo $x$ que cumpla con estas dos desigualdades se cumple la
desigualdad~(\ref{eqLim044}). Y esto significa que el número $3$ es el límite de $g(x)$ cuando $x$
se aproxima al número $1$.

\paragraph{3.} Sea $\funcjc{h}{\mathbb{R} - \{1\}}{\mathbb{R}}$ la función definida por
\[
h(x) = \frac{2x^2 - x - 1}{x - 1}.
\]
En este caso, $h$ no está definida en $1$ por lo que es distinta tanto de $f$ como de $g$. Sin
embargo, para todo $x\neq 1$, se verifican las igualdades:
\[
h(x) = \frac{2x^2 - x - 1}{x-1} = \frac{(2x+1)(x-1)}{x-1} = 2x + 1 = g(x) = f(x).
\]
De manera análoga al caso de $g$, podemos demostrar que
\[
3 = \limjc{h(x)}{x}{1}.
\]

En estos tres ejemplos podemos ver que, en lo que se refiere al límite de una función en un punto
$a$, \emph{no importa el valor que pueda tomar la función en el punto $a$}; incluso, la función
puede no estar definida en este punto. \emph{Lo que importa realmente es el comportamiento de la
función alrededor del punto $a$}. Las gráficas de las funciones $f$, $g$ y $h$, que están a
continuación, ilustran esta última afirmación:
\begin{center}
\begin{pspicture}(-1,-1)(7,3)
\psset{xunit=.8,yunit=.5}%
\psaxes[ticks=none,labels=none]{->}(0,0)(-1,-1)(3,6)%
\uput[-90](3,0){$x$}%
\uput[180](0,6){$f(x)$}%
\psplot{-1}{2}{2 x mul 1 add}%
\psline[linecolor=gray,linestyle=dashed](1,0)(1,3)(0,3)%
\rput(1,-.5){$1$}%
\rput(-.25,3){$3$}%

\rput[l](4,4){$\funcionjc{f}{\mathbb{R}}{\mathbb{R}}{x}{f(x) = 2x + 1}$}%
\rput[l](4,1.5){$\displaystyle{\limjc{f(x)}{x}{1} = 3 = f(1)}$}
\end{pspicture}
\end{center}
%
\begin{center}
\begin{pspicture}(-1,-1)(7,3)
\psset{xunit=.8,yunit=.5}%
\psaxes[ticks=none,labels=none]{->}(0,0)(-1,-1)(3,6)%
\uput[-90](3,0){$x$}%
\uput[180](0,6){$g(x)$}%
\psplot{-1}{.95}{2 x mul 1 add}%
\psplot{1.05}{2}{2 x mul 1 add}%
\pscircle(1,3){.05}%
\pscircle[fillstyle=solid,fillcolor=black](1,1){.05}%
\psline[linecolor=gray,linestyle=dashed](1,0)(1,1)(0,1)%
\rput(1,-.5){$1$}%
\rput(-.25,3){$3$}%
\rput(-.25,1){$1$}%
\rput[l](4,4){$\funcionjc{g}{\mathbb{R}}{\mathbb{R}}{x}{g(x) = %
\begin{cases}
2x + 1 & \text{si } x \neq 1 \\
1 & \text{si } x = 1
\end{cases}}$}%
\rput[l](4,1){$\displaystyle{\limjc{g(x)}{x}{1} = 3 \neq g(1)}$}
\end{pspicture}
\end{center}
%
\begin{center}
\begin{pspicture}(-1,-1)(7,3)
\psset{xunit=.8,yunit=.5}%
\psaxes[ticks=none,labels=none]{->}(0,0)(-1,-1)(3,6)%
\uput[-90](3,0){$x$}%
\uput[180](0,6){$h(x)$}%
\psplot{-1}{.95}{2 x mul 1 add}%
\psplot{1.05}{2}{2 x mul 1 add}%
\pscircle(1,3){.05}%
\rput(1,-.5){$1$}%
\rput(-.25,3){$3$}%
\rput[l](4,4){$\funcionjc{h}{\mathbb{R}-\{1\}}{\mathbb{R}}{x}{h(x) = 2x + 1}$}%
\rput[l](4,2){$\displaystyle{\limjc{h(x)}{x}{1} = 3}$}%
\rput[l](4,.75){$h(1)$ no existe}%
\end{pspicture}
\end{center}

Podemos resumir la situación que se ilustra en estos ejemplos, diciendo que si dos funciones
solo difieren en un punto de su dominio, o bien las dos tienen límite en dicho punto, y es el mismo
límite, o bien ninguna tiene límite. De manera más precisa se expresa este resultado en el teorema del límite de funciones localmente iguales, que presentaremos a continuación.

\subsection{Límite de funciones localmente iguales}

\begin{defical}[Funciones localmente iguales]
Sean: 
\begin{itemize}
      \item[] $a$ un número real;
      \item[] $I$ un intervalo abierto que contiene al número $a$; y,
      \item[] $f$ y $g$ dos funciones reales definidas en $I$, salvo talvez en $a$ (es decir, $I\subset \Dm(f)\cup \{a\}$; e $I\subset \Dm(g)\cup \{a\}$).
\end{itemize}
Diremos que ``$f=g$ localmente cerca de $a$'' o simplemente que ``$f=g$ cerca de $a$'', si existe $r>0$ tal que para todo $x\in ]a-r, a+r[ \setminus \{a\}$, $f(x)=g(x)$.
\end{defical}

En la definición, no importa cuán pequeño sea el valor de $r$. Esto justifica la expresión ``cerca de a''. En vez de ``$f(x)=g(x)$'' puede tomarse cualquier otra propiedad. Por ejemplo, diremos que ``$f > g$ cerca de $a$'' si en vez de ``$f(x)=g(x)$'' se exige que ``$f(x) > g(x)$''.

\begin{exemplo}[ ]{%
\[
	F(x)=
\begin{cases}
\displaystyle\frac{2x^2-x-1}{x-1} & \text{si $x<2$,} \\
x^2+1 & \text{si $x\geq 2$}.
\end{cases}
\]

\[
	G(x)=
\begin{cases}
1-x^2 & \text{si $x<0$,} \\
2x+1 & \text{si $x\geq 0$}.
\end{cases}
\]
}%
Claramente $F=G$ cerca de $1$. En efecto, si tomamos $r\in ]0,1[$, se puede ver que para todo $x\in ]1-r, 1+r[\setminus \{1\}$, $F(x)=G(x)$.

Podemos ahora enunciar el siguiente teorema.
\end{exemplo}


% \begin{teocal}[Límite de funciones localmente iguales]\label{eq:limitegeneral}%
% Sean $I$ y $J$ dos intervalos abiertos y $\funcjc{f}{I}{\mathbb{R}}$ y $\funcjc{g}{J}{\mathbb{R}}$
% tales que $a \in I \cap J$ y:
% \begin{enumerate}
% \item $f(x) = g(x)$ para todo $x \in I \cap J$ y $x \neq a$; y
% \item existe L tal que:
%    \[
%       L = \limjc{g(x)}{x}{a}.
%    \]
% \end{enumerate}
% Entonces $f$ también tiene límite en a y:
% \[
%    L = \limjc{f(x)}{x}{a}.
% \]
% \end{teocal}%Fin del teo
% 
% Este es el caso de las funciones $f$, $g$ y $h$. Se diferencian únicamente en $a = 1$. Como el
% límite de $f$ existe y es igual a $3$, entonces los límites de $g$ y $h$ existen también y son
% iguales a $3$.

\begin{teocal}[Límite de funciones localmente iguales]\label{eq:limitegeneral}%
Sean:
\begin{enumerate}
\item[] $a$ un número real;
\item[] $I$ un intervalo abierto que contine al número $a$;
\item[] $f$ y $g$ dos funciones definidas en $I$, salvo tal vez en $a$.
\end{enumerate}
Si $f=g$ cerca de $a$, entonces:
\begin{enumerate}
      \item Existe $\displaystyle\limjc{f(x)}{x}{a}$ si y solo si existe $\displaystyle\limjc{g(x)}{x}{a}$.
      \item Si los límites existen, son iguales.
\end{enumerate}
\end{teocal}%Fin del teo

En el ejemplo de las funciones $f$, $g$ y $h$ definidas anteriormente, vemos que $f=g=h$ cerca de $1$, y como existe $\displaystyle\limjc{f(x)}{x}{1}=3$, entonces también existen $\displaystyle\limjc{g(x)}{x}{1}$ y $\displaystyle\limjc{h(x)}{x}{1}$ y son iguales a $3$.

Este teorema es muy útil para el cálculo de límites y se lo utiliza de la siguiente manera.

Supongamos que queremos calcular $\displaystyle\limjc{f(x)}{x}{a}$. Se busca una función $g$ cuyo límite en $a$ se conozca y tal que $f=g$ cerca de $a$.

El teorema nos permite afirmar entonces que
\[
      \limjc{f(x)}{x}{a} = \limjc{g(x)}{x}{a}.
\]

% Este teorema se utiliza de la siguiente manera. Supongamos que queremos calcular
% \[
%    \limjc{f(x)}{x}{a}.
% \]
% Se busca una función $g$ cuyo límite en $a$ se conozca y tal que
% \[
% g(x) = f(x)
% \]
% para todo $x$ en la intersección de los dominios de $f$ y $g$ y que sea diferente de $a$. Entonces,
% lo que podemos afirmar es que
% \[
%    \limjc{f(x)}{x}{a} = \limjc{g(x)}{x}{a}.
% \]

\begin{exemplo}[Solución]{%
Supongamos conocido que
\[
   \limjc{(x + 2)}{x}{1} = 3.
\]
Calcular
\[
   \limjc{\frac{x^2 + x - 2}{x - 1}}{x}{1} .
\]
}%
Sea $\funcjc{f}{\mathbb{R} - {1}}{\mathbb{R}}$ tal que
\[
f(x) = \frac{x^2 + x - 2}{x - 1}.
\]
Debemos hallar una función $g$ que sea igual a $f$, excepto en $1$, y cuyo límite conozcamos.

Esto se puede hacer, pues
\[
   f(x) = \frac{x^2 + x - 2}{x - 1} = \frac{(x - 1)(x + 2)}{x - 1} = x + 2,
\]
para todo $x \neq 1$. Por lo tanto, si se define
\[
   g(x) = x + 2,
\]
sabemos que:
\begin{enumerate}
\item $f(x) = g(x)$ para todo $x \neq 1$; por lo que $f=g$ cerca de $1$; y,
\item $\displaystyle\limjc{g(x)}{x}{1} = 3.$
\end{enumerate}
Por lo tanto, gracias al teorema (\ref{eq:limitegeneral}), podemos afirmar que:
$\displaystyle
   \limjc{\frac{x^2 + x - 2}{x - 1}}{x}{1} = \limjc{g(x)}{x}{1} = 3$.
\end{exemplo}

Para terminar esta sección, presentamos el siguiente teorema, que es una consecuencia inmediata de
la definición de límite, que es muy útil en diferentes aplicaciones del concepto de límite.
%-----> 2008 09 12
%\textcolor{red}{[?`qué? También ameritaría uno o dos ejemplos que muestren cómo se utiliza este
%teorema. Además, un par de ejercicios sobre este teorema para el final de la sección.]}

\begin{teocal}[Caracterizaciones del límite]\label{teol:LEquiv0} Sea $\funcjc{f}{\Dm(f)}{\mathbb{R}}.$ Entonces:
\[
   L = \limjc{f(x)}{x}{a} \Leftrightarrow 0 = \limjc{(f(x) - L)}{x}{a} \Leftrightarrow
   0 = \limjc{|f(x) - L|}{x}{a}.
\]
\end{teocal}%

La demostración es un buen ejercicio para trabajar la definición de límite por lo que se sugiere al
lector la haga por sí mismo.

Un uso típico de este teorema es el siguiente. Queremos demostrar que
\[
3 = \limjc{\frac{x + 3}{x + 1}}{x}{0}.
\]
En lugar de ello, probaremos que
\[
0 = \limjc{\frac{x + 3}{x + 1} - 3}{x}{0}.
\]
Pero, como
\[
\frac{x + 3}{x + 1} - 3 = -\frac{2x}{x + 1},
\]
lo que hay que probar es
\[
0 = \limjc{-2\frac{x}{x + 1}}{x}{0}.
\]
Y, como
\[
\left\lvert-2\frac{x}{x + 1}\right\rvert = 2\left\lvert\frac{x}{x + 1}\right\rvert,
\]
una alternativa es probar que
\[
0 = 2\limjc{\frac{x}{x + 1}}{x}{0}.
\]

En sentido estricto, no hay mayor diferencia en la manera cómo se demuestra cualquiera de estas
igualdades, salvo, en ciertas ocasiones, en las que algunas operaciones algebraicas suelen
simplificarse. El lector debería realizar cada una de estas prueba para que compare y determine
cuáles podrían ser esas simplificaciones.

\subsection{Ejercicios}
\begingroup
\small
\begin{multicols}{2}
\begin{enumerate}[leftmargin=*]
\item Demuestre, usando la definición de límite, que:
\begin{enumerate}[leftmargin=*]
\item $\displaystyle
	\lim_{x\to 8}(\sqrt[3]{x^2} + 2\sqrt[3]{x} + 4) = 12 $

\item $\displaystyle
	\lim_{x\to 3}(8x-15)=9
$
\item $\displaystyle
	\lim_{x\to -2}(5x+14)=4
$

\item $\displaystyle
	\lim_{x\to 9}\dfrac{1}{\sqrt{x} + 3} = \frac{1}{6}
$

\item $\displaystyle
	\lim_{x\to -2}\dfrac{2x^2-8}{x+2}=-8
$
\item $\displaystyle
	\lim_{x\to 1}\dfrac{2x + 3}{x - 2} = -5 $
\item $\displaystyle
	\lim_{x\to 1}(x^2+x+1)=3
$
\item $\displaystyle \lim_{x\to \frac{1}{3}}(-9x^2 + 3x + 1) = 1$

\item $\displaystyle
	\lim_{x\to -1}\dfrac{x + 4}{x - 1} = -\frac{3}{2}$
\end{enumerate}
\item Use el teorema \ref{eq:limitegeneral} para hallar los siguientes límites:
\begin{enumerate}[leftmargin=*]
\item $\displaystyle
	\lim_{x\to 1}\dfrac{2x^2+x-3}{x^2-3x+2}$

\item $\displaystyle
	\lim_{x\to 9}\dfrac{\sqrt{x}-3}{x-9}$

\item $\displaystyle
	\lim_{x\to -1}\dfrac{x^2+5x+4}{x^2-1}$

\item $\displaystyle
	\lim_{x\to 8}\dfrac{x-8}{\sqrt[3]{x}-2}$

\end{enumerate}
\end{enumerate}
\end{multicols}
\endgroup

\section{Continuidad de una función} Las tres funciones utilizadas en la sección precedente tienen
límite en el número $1$. La primera y la segunda también están definidas en $1$; es decir, $1$ está
en el dominio de $f$ y $g$. Sin embargo, en el caso de la primera, como $f(1) = 3$, el valor de $f$
en $1$ es igual al límite; en el caso de la segunda, como $g(1) = 1 \neq 3$, esto no ocurre. Para
la tercera función, el número $1$ no está en el dominio de $h$.

Los dibujos de estas tres funciones muestran que en el gráfico de las funciones $g$ y $h$ hay un
``salto'' al cruzar la recta vertical de ecuación $x = 1$ (en la jerga matemática, se suele decir
que ``hay un salto al pasar por $1$''). Si dibujáramos los gráficos de estas funciones, trazándolo
de izquierda a derecha, en el caso de la segunda y de la tercera, deberíamos ``interrumpir'' o
``discontinuar'' el trazo. Para $f$ eso no ocurrirá. Por esa razón, la función $f$ va a ser una
función ``continua'', mientras que las otras dos no.

Lo que diferencia a $f$ de $g$ y $h$ es el hecho de que, a más de existir el límite en $1$, la
función está definida allí y su valor es igual al límite. Ésta es la definición de continuidad:

\begin{defical}[Función continua]
Una función $\funcjc{f}{\Dm(f)}{\mathbb{R}}$, donde $I\subseteq \Dm(f)$ es un intervalo abierto, es
\emph{continua} en $a$ si y solo si:
\begin{enumerate}
\item $a\in I$;
\item existe $\displaystyle\limjc{f(x)}{x}{a}$; y
\item $\displaystyle f(a) = \limjc{f(x)}{x}{a}$.
\end{enumerate}
Una función es continua en el intervalo abierto $I$ si es continua en todos y cada uno de los
elementos de $I$.
\end{defical}

La continuidad de una función es un tema central en el estudio del Cálculo. A lo largo de este
libro, conoceremos diversas propiedades de las funciones continuas. Por ahora, veamos que la
definición de límite ofrece una definición equivalente de continuidad:

\begin{teocal}[Función continua]%
Sea $\funcjc{f}{\Dm(f)}{\mathbb{R}}$, donde $I\subseteq\Dm(f)$ es un intervalo abierto. Sea $a\in
I$. Entonces, $f$ es continua en $a$ si y solo si para todo $\epsilon > 0$ existe un número $\delta
> 0$ tal que
\[
|f(x) - f(a)| < \epsilon,
\]
siempre que $|x - a| < \delta$ y $x \in I$.
\end{teocal}

Observemos que no hace falta excluir el caso $x = a$, como se hace en la definición de límite,
porque, al ser $f$ continua en $a$, si $x = a$, se verifica que:
\[
|f(x) - f(a)| = |f(a) - f(a)| = 0 < \epsilon.
\]

Veamos un par de ejemplos sencillos de funciones continuas.

\begin{exemplo}[Solución]{\label{ex:lim002}
La función constante es continua en su dominio.}%
Sean $c\in\mathbb{R}$ y $\funcjc{f}{\mathbb{R}}{\mathbb{R}}$ tal que
\[
f(x) = c
\]
para todo $x\in \mathbb{R}$. Probemos que la función $f$ es continua en todo $a\in\mathbb{R}$.

Para ello, sea $a\in\mathbb{R}$. Puesto que $a\in\Dm(f)$, solo nos falta verificar que:
\begin{enumerate}
\item existe $\displaystyle\limjc{f(x)}{x}{a}$; y que
\item $\displaystyle f(a) = c = \limjc{f(x)}{x}{a}.$
\end{enumerate}
Probaremos la segunda condición únicamente, ya que con ello es suficiente para probar la primera.

Sea $\epsilon > 0$. Debemos encontrar un número $\delta > 0$ tal que
\begin{equation}
\label{eq:lim001}
|f(x) - c| < \epsilon
\end{equation}
siempre que
\[
|x - a| < \delta.
\]

Ahora bien, puesto que
\[
|f(x) - c| = |c - c| = 0 < \epsilon
\]
para todo $x\in\mathbb{R}$, la desigualdad~(\ref{eq:lim001}) será verdadera cualquiera que sea el
$\delta > 0$ elegido. Es decir, el número $c$ es el límite de $f(x)$ cuando $x$ se aproxima al
número $a$.

Por lo tanto, este límite existe, lo que prueba que la función constante es continua en $a$ y, como
$a$ es cualquier elemento del dominio de $f$, la función es continua en su dominio.
\end{exemplo}

Este ejemplo nos muestra, además, la veracidad de la siguiente igualdad:
\begin{equation}
\label{eq:LimConstante}
c = \limjc{c}{x}{a}.
\end{equation}
Esta igualdad suele enunciarse de la siguiente manera:
\begin{quote}
\textbf{el límite de una constante es igual a la constante.}
\end{quote}

\begin{exemplo}[Solución]{\label{ex:lim003}%
La función identidad es continua en todo su dominio}%
Sea $\funcjc{f}{\mathbb{R}}{\mathbb{R}}$ tal que
\[
f(x) = x
\]
para todo $x\in \mathbb{R}$. Probemos que la función $f$ es continua en todo $a\in\mathbb{R}$.

Para ello, sea $a\in\mathbb{R}$. Puesto que $a\in\Dm(f)$, solo nos falta verificar que:
\begin{enumerate}
\item existe $\displaystyle\limjc{f(x)}{x}{a}$; y que
\item $\displaystyle f(a) = a = \limjc{f(x)}{x}{a}.$
\end{enumerate}
Probaremos la segunda condición únicamente, ya que con ello es suficiente para probar la primera.

Sea $\epsilon > 0$. Debemos encontrar un número $\delta > 0$ tal que
\begin{equation}
\label{eq:lim002}
|f(x) - a| < \epsilon
\end{equation}
siempre que
\[
|x - a| < \delta.
\]
Como
\[
|f(x) - a| = |x - a|,
\]
es obvio que el número $\delta$ buscado es igual a $\epsilon$. Si lo elegimos así, habremos probado
que $a$ es el límite de $f(x)$ cuando $x$ se aproxima al número $a$. Por lo tanto, éste límite
existe, lo que prueba que la función identidad es continua en $a$, de donde, es continua en su
dominio.
\end{exemplo}

Este ejemplo nos demuestra que la siguiente igualdad es verdadera:
\begin{equation}
\label{eq:LimIdentidad}
a = \limjc{x}{x}{a}.
\end{equation}
Esta igualdad se enuncia de la siguiente manera:
\begin{quote}
\textbf{el límite de $x$ es igual al número $a$ cuando $x$ se aproxima al número $a$.}
\end{quote}

En una sección posterior, desarrollaremos algunos procedimientos para el cálculo de límites lo que,
a su vez, nos permitirá estudiar la continuidad de una función.

\section{Interpretación geométrica de la definición de límite}
Como hemos podido ver, encontrar el número $\delta$ dado el número $\epsilon$ no siempre es una
tarea fácil. Los ejemplos que hemos trabajado son sencillos relativamente. En general, demostrar
que un cierto número es el límite de una función suele ser una tarea de considerable trabajo. Por
ello, el poder ``visualizar'' la definición es de mucha ayuda para poder manipular luego las
desigualdades con los $\epsilon$ y $\delta$. Esta visualización puede ser realizada de la siguiente
manera.

En primer lugar, recordemos que la desigualdad
\begin{equation}
\label{eqLim045}
|x - x_0| < r,
\end{equation}
donde $x$, $x_0$ y $r$ son números reales y, además, $r > 0$, es equivalente a la desigualdad
\[
x_0 - r < x < x_0 + r.
\]
Por lo tanto, el conjunto de todos los $x$ que satisfacen la desigualdad~(\ref{eqLim045}) puede ser
representado geométricamente por el intervalo $]x_0 - r, x_0 + r[$; es decir, por el intervalo
abierto con centro en $x_0$ y radio $r$:
\begin{center}
\begin{pspicture}(-2,-.5)(2,.5)
\psline{<->}(-2,0)(2,0)%
\psline(-1.5,.15)(-1.5,-.15)%
\psline(0,.15)(0,-.15)%
\psline(1.5,.15)(1.5,-.15)%
\rput(-1.5,-.35){$x_0 - r$}%
\rput(0,-.35){$x_0$}%
\rput(1.5,-.35){$x_0 + r$}%
\psframe[hatchcolor=gray,fillstyle=hlines,hatchangle=45,linestyle=none,hatchsep=2pt]%
      (-1.5,-.1)(1.5,.1)%
\end{pspicture}
\end{center}
La longitud de este intervalo es igual a $2r$.

Supongamos que el número $L$ es el límite de $f(x)$ cuando $x$ se aproxima al número $a$. Esto
significa que, dado cualquier número $\epsilon > 0$, existe un número $\delta > 0$ tal que se
verifica la desigualdad
\begin{equation}
\label{eqLim046}
|f(x) - L| < \epsilon,
\end{equation}
siempre que se satisfagan las desigualdades
\begin{equation}
\label{eqLim047}
0 < |x - a| < \delta.
\end{equation}

Con la interpretación geométrica realizada previamente, esta definición de límite puede expresarse
en términos geométricos de la siguiente manera:
\begin{quote}
{\bfseries dado cualquier número $\bm{\epsilon > 0}$, existe un número $\bm{\delta > 0}$ tal que
$\bm{f(x)}$ se encuentre en el intervalo de centro $\bm{L}$ y radio $\bm{\epsilon}$, siempre que
$\bm{x}$ se encuentre en el intervalo de centro $\bm{a}$ y radio $\bm{\delta}$ y $x\neq a$.}
\end{quote}

Esta formulación puede ser visualizada de la siguiente manera. En un sistema de coordenadas, en el
que se va a representar gráficamente la función $f$, dibujemos los dos intervalos que aparecen en
la definición de límite: $]a - \delta, a + \delta[$ y $]L - \epsilon, L + \epsilon[$. Obtendremos
lo siguiente:
\begin{center}
\psset{unit=0.9}
\begin{pspicture}(-.5,-.5)(5,4.5)
\psaxes[ticks=none,labels=none]{->}(0,0)(-.5,-.5)(5,4)%
\uput[-90](5,0){$x$}%
\uput[180](0,4){$f(x)$}%
\psset{xunit=.35mm,yunit=.35mm,plotpoints=200}%

\psframe[hatchcolor=gray,fillstyle=hlines,hatchangle=45,linestyle=none,hatchsep=2pt]%
      (67.5,-1.5)(97.5,1.5)%
\psline(67.5,-1)(67.5,1)% a - \delta
\rput[Br](67.5,-10){$a - \delta$}%
\psline(82.5,-1)(82.5,1)% a
\rput[B](82.5,-10){$a$}%
\psline(97.5,-1)(97.5,1)% a + \delta
\rput[Bl](97.5,-10){$a + \delta$}%

\psframe[hatchcolor=gray,fillstyle=hlines,hatchangle=45,linestyle=none,hatchsep=2pt]%
      (-1.5,52.36)(1.5,86.67)%
\psline(-1,52.36)(1,52.36)% L - \epsilon
\rput[r](-3,52.36){$L - \epsilon$}%
\psline(-1,71.79)(1,71.79)% L
\rput[r](-3,71.79){$L$}%
\psline(-1,86.67)(1,86.67)% L + \epsilon
\rput[r](-3,86.67){$L + \epsilon$}%

\end{pspicture}
\end{center}
A continuación, dibujemos dos bandas: una horizontal, limitada por las rectas horizontales cuyas
ecuaciones son $y = L - \epsilon$ y $y = L + \epsilon$, y una vertical, limitada por las rectas
verticales cuyas ecuaciones son $x = a - \delta$ y $x = a + \delta$:
\begin{center}
\psset{unit=0.9}
\begin{pspicture}(-.5,-.5)(5,4.5)
\psaxes[ticks=none,,labels=none]{->}(0,0)(-.5,-.5)(5,4)%
\uput[-90](5,0){$x$}%
\uput[180](0,4){$f(x)$}%
\psset{xunit=.35mm,yunit=.35mm,plotpoints=200}%

\psline(67.5,-1)(67.5,1)% a - \delta
\rput[Br](67.5,-10){$a - \delta$}%
\psline(82.5,-1)(82.5,1)% a
\rput[B](82.5,-10){$a$}%
\psline(97.5,-1)(97.5,1)% a + \delta
\rput[Bl](97.5,-10){$a + \delta$}%

\psline(-1,52.36)(1,52.36)% L - \epsilon
\rput[r](-3,52.36){$L - \epsilon$}%
\psline(-1,71.79)(1,71.79)% L
\rput[r](-3,71.79){$L$}%
\psline(-1,86.67)(1,86.67)% L + \epsilon
\rput[r](-3,86.67){$L + \epsilon$}%

\psset{linestyle=dashed}
\psframe[linestyle=none,fillstyle=solid,fillcolor=lightgray](1,52.36)(67.5,86.67)%
\psframe[linestyle=none,fillstyle=solid,fillcolor=lightgray](97.5,52.36)(120,86.67)%
\psline(0,52.36)(120,52.36)%
\psline(0,86.67)(120,86.67)%

\psframe[linestyle=none,fillstyle=solid,fillcolor=lightgray](67.5,1)(97.5,52.36)%
\psframe[linestyle=none,fillstyle=solid,fillcolor=lightgray](67.5,86.67)(97.5,110)%
\psline(67.5,0)(67.5,100)%
\psline(97.5,0)(97.5,100)%

\end{pspicture}
\end{center}

El rectángulo obtenido por la intersección de las dos bandas está representado por el siguiente
conjunto:
\[
C = \{(x,y) \in \mathbb{R}^2 : x \in\ ]a-\delta, a + \delta[, \ \ y\in\ ]L-\epsilon,L+\epsilon[\} = \ ]a-\delta, a+\delta [\times ]L-\epsilon, L+\epsilon [.
\]

La definición de límite afirma que $f(x)$ estará en el intervalo $]L-\epsilon, L+\epsilon[$ siempre
que $x$, siendo distinto de $a$, esté en el intervalo $]a-\delta, a+ \delta[$. Esto significa,
entonces, que la pareja $(x,f(x))$ está en el conjunto $C$. Es decir, todos los puntos de
coordenadas
\[
(x,f(x))
\]
tales que $x\neq a$ pero $x\in\ ]a-\delta, a +\delta[$ están en el interior del rectángulo
producido por la intersección de las dos bandas. Pero todos estos puntos no son más que la gráfica
de la función $f$ en el conjunto $]a-\delta, a+\delta[ - \{a\}$. En otras palabras:
\begin{quote}
{\bfseries dado cualquier número $\bm{\epsilon > 0}$, existe un número $\bm{\delta > 0}$ tal que la
gráfica de $f$ en el conjunto $]a-\delta, a+\delta[ - \{a\}$ está en el interior del conjunto $C$.}
\end{quote}

El siguiente dibujo muestra lo que sucede cuando $L$ es el límite de $f(x)$ cuando $x$ se aproxime
al número $a$:
\begin{center}
\psset{unit=0.9}
\begin{pspicture}(-.5,-.5)(5,4.5)
\psaxes[ticks=none,labels=none]{->}(0,0)(-.5,-.5)(5,4)%
\uput[-90](5,0){$x$}%
\uput[180](0,4){$f(x)$}%
\psset{xunit=.35mm,yunit=.35mm,plotpoints=200}%

\psline(67.5,-1)(67.5,1)% a - \delta
\rput[Br](67.5,-10){$a - \delta$}%
\psline(82.5,-1)(82.5,1)% a
\rput[B](82.5,-10){$a$}%
\psline(97.5,-1)(97.5,1)% a + \delta
\rput[Bl](97.5,-10){$a + \delta$}%

\psline(-1,52.36)(1,52.36)% L - \epsilon
\rput[r](-3,52.36){$L - \epsilon$}%
\psline(-1,71.79)(1,71.79)% L
\rput[r](-3,71.79){$L$}%
\psline(-1,86.67)(1,86.67)% L + \epsilon
\rput[r](-3,86.67){$L + \epsilon$}%

\psset{linestyle=dashed}
\psframe[linestyle=none,fillstyle=solid,fillcolor=lightgray](1,52.36)(67.5,86.67)%
\psframe[linestyle=none,fillstyle=solid,fillcolor=lightgray](97.5,52.36)(120,86.67)%
\psline(0,52.36)(120,52.36)%
\psline(0,86.67)(120,86.67)%

\psframe[linestyle=none,fillstyle=solid,fillcolor=lightgray](67.5,1)(97.5,52.36)%
\psframe[linestyle=none,fillstyle=solid,fillcolor=lightgray](67.5,86.67)(97.5,110)%
\psline(67.5,0)(67.5,100)%
\psline(97.5,0)(97.5,100)%

\psset{linestyle=solid,xunit=.35mm,yunit=.35mm,plotpoints=200}%
\psplot{0}{125}{x dup sin mul 10 sub}%

\pscircle[fillstyle=solid,fillcolor=white](82.5,71.79){.05}
\end{pspicture}
\end{center}

Al conjunto $C$ se le denomina \emph{caja} para el gráfico de $f$ en el punto de coordenadas
$(a,L)$. De manera más general, una \emph{caja para el gráfico de una función en el punto de
coordenadas} $(x,y)$ es el interior de una región rectangular cuyos lados son paralelos a los ejes
coordenados, que contiene al punto de coordenadas $(x,y)$, de modo que ningún punto del gráfico de
la función $f$ que esté en la banda vertical, salvo, tal vez, el punto de coordenadas $(x,f(x))$,
no puede estar ni sobre la región rectangular ni bajo de ella. Si el punto de coordenadas $(x,y)$
es también el centro de la región rectangular (es decir, es la intersección de las diagonales del
rectángulo), entonces la región rectangular es denominada \emph{caja centrada en el punto de
coordenadas} $(x,y)$.

Un ejemplo de una región rectangular que no es una caja para el gráfico de $f$ es el siguiente:
\begin{center}
\psset{unit=0.9}
\begin{pspicture}(-.5,-.5)(5,4.5)
\psaxes[ticks=none,labels=none]{->}(0,0)(-.5,-.5)(5,4)%
\uput[-90](5,0){$x$}%
\uput[180](0,4){$f(x)$}%
\psset{xunit=.35mm,yunit=.35mm,plotpoints=200}%

\psset{linestyle=dashed}
\psframe[linestyle=none,fillstyle=solid,fillcolor=lightgray](1,60)(67.5,80)%
\psframe[linestyle=none,fillstyle=solid,fillcolor=lightgray](97.5,60)(120,80)%
\psline(0,60)(120,60)%
\psline(0,80)(120,80)%

\psframe[linestyle=none,fillstyle=solid,fillcolor=lightgray](67.5,1)(97.5,60)%
\psframe[linestyle=none,fillstyle=solid,fillcolor=lightgray](67.5,80)(97.5,110)%
\psline(67.5,0)(67.5,100)%
\psline(97.5,0)(97.5,100)%

\psset{linestyle=solid,xunit=.35mm,yunit=.35mm,plotpoints=200}%
\psplot{0}{125}{x dup sin mul 10 sub}%

\pscircle[fillstyle=solid,fillcolor=white](82.5,71.79){.05}

\end{pspicture}
\end{center}

Con la definición de caja, podemos decir que el conjunto $C$ es, efectivamente, una caja para el
gráfico de $f$ centrada en el punto de coordenadas $(a,L)$. En este caso, la caja $C$ tiene una
altura igual a $2\epsilon$ y una base igual a $2\delta$.

En términos de cajas, la definición de límite garantiza que:
\begin{quote}
{\bfseries si $L$ es el límite de $f(x)$ cuando $x$ se aproxima a $a$, entonces, el gráfico de $f$
tiene cajas de todas las alturas positivas posibles centradas en el punto $(a,L)$.}
\end{quote}

El siguiente dibujo, muestra dos cajas para el gráfico de $f$, ambas centradas en $(a,L)$:
\begin{center}
\psset{unit=0.9}
\begin{pspicture}(-.5,-.5)(5,4.5)
\psaxes[ticks=none,labels=none]{->}(0,0)(-.5,-.5)(5,4)%
\uput[-90](5,0){$x$}%
\uput[180](0,4){$f(x)$}%
\psset{xunit=.35mm,yunit=.35mm,plotpoints=200}%

\psline(77.5,-1)(77.5,1)% a - \delta
\rput[Br](77.5,-10){$a - \delta$}%
\psline(82.5,-1)(82.5,1)% a
\rput[B](82.5,-10){$a$}%
\psline(87.5,-1)(87.5,1)% a + \delta
\rput[Bl](87.5,-10){$a + \delta$}%

\psline(-1,65.66)(1,65.66)% L - \epsilon
\rput[r](-3,65.66){$L - \epsilon$}%
\psline(-1,71.79)(1,71.79)% L

\psline(-1,77.42)(1,77.42)% L + \epsilon
\rput[r](-3,77.42){$L + \epsilon$}%

\psset{linestyle=dashed}
\psframe[linestyle=none,fillstyle=solid,fillcolor=lightgray](1,65.66)(77.5,77.42)%
\psframe[linestyle=none,fillstyle=solid,fillcolor=lightgray](87.5,65.66)(120,77.42)%
\psline(0,65.66)(120,65.66)%
\psline(0,77.42)(120,77.42)%

\psframe[linestyle=none,fillstyle=solid,fillcolor=lightgray](77.5,1)(87.5,65.66)%
\psframe[linestyle=none,fillstyle=solid,fillcolor=lightgray](77.5,77.42)(87.5,110)%
\psline(77.5,0)(77.5,100)%
\psline(87.5,0)(87.5,100)%

\rput[l](3,71.79){$L$}%

\psset{linestyle=solid,xunit=.35mm,yunit=.35mm,plotpoints=200}%
\psplot{0}{125}{x dup sin mul 10 sub}%

\pscircle[fillstyle=solid,fillcolor=white](82.5,71.79){.05}
\end{pspicture}
\hspace{1cm}
\begin{pspicture}(-.5,-.5)(5,4.5)
\psaxes[ticks=none,labels=none]{->}(0,0)(-.5,-.5)(5,4)%
\uput[-90](5,0){$x$}%
\uput[180](0,4){$f(x)$}%
\psset{xunit=.35mm,yunit=.35mm,plotpoints=200}%

\psline(72.5,-1)(72.5,1)% a - \delta
\rput[Br](72.5,-10){$a - \delta$}%
\psline(82.5,-1)(82.5,1)% a
\rput[B](82.5,-10){$a$}%
\psline(92.5,-1)(92.5,1)% a + \delta
\rput[Bl](92.5,-10){$a + \delta$}%

\psline(-1,59.15)(1,59.15)% L - \epsilon
\rput[r](-3,59.15){$L - \epsilon$}%
\psline(-1,71.79)(1,71.79)% L
\rput[r](-3,71.79){$L$}%
\psline(-1,82.41)(1,82.41)% L + \epsilon
\rput[r](-3,82.41){$L + \epsilon$}%

\psset{linestyle=dashed}
\psframe[linestyle=none,fillstyle=solid,fillcolor=lightgray](1,59.15)(72.5,82.41)%
\psframe[linestyle=none,fillstyle=solid,fillcolor=lightgray](92.5,59.15)(120,82.41)%
\psline(0,59.15)(120,59.15)%
\psline(0,82.41)(120,82.41)%

\psframe[linestyle=none,fillstyle=solid,fillcolor=lightgray](72.5,1)(92.5,59.15)%
\psframe[linestyle=none,fillstyle=solid,fillcolor=lightgray](72.5,82.41)(92.5,110)%
\psline(72.5,0)(72.5,100)%
\psline(92.5,0)(92.5,100)%

\psset{linestyle=solid,xunit=.35mm,yunit=.35mm,plotpoints=200}%
\psplot{0}{125}{x dup sin mul 10 sub}%

\pscircle[fillstyle=solid,fillcolor=white](82.5,71.79){.05}
\end{pspicture}
\end{center}

Estos dos últimos dibujos sugieren que el gráfico de $f$ tiene cajas centradas en $(a,L)$ de todas
las alturas posibles.

A continuación, podemos visualizar por qué las funciones $f$, $g$ y $h$ de la sección anterior, las
que utilizamos para mostrar por qué en la definición de límite no importa el valor que pueda tener
(o no tener), tienen el mismo límite:
\begin{center}
\psset{unit=0.9}
\begin{pspicture}(-1,-1)(3.5,3.5)
\psset{xunit=.8,yunit=.5}%
\psaxes[ticks=none,labels=none]{->}(0,0)(-1,-1)(4,6)%
\uput[-90](4,0){$x$}%
\uput[180](0,6){$f(x)$}%
\psplot{-1}{2}{2 x mul 1 add}%

\psframe[linestyle=none,fillstyle=solid,fillcolor=lightgray](0,2.5)(.75,3.5)%
\psframe[linestyle=none,fillstyle=solid,fillcolor=lightgray](1.25,2.5)(2,3.5)%
\psset{linestyle=dashed}
\psline(0,2.5)(2,2.5)%
\psline(0,3.5)(2,3.5)%

\psframe[linestyle=none,fillstyle=solid,fillcolor=lightgray](.75,0)(1.25,2.5)%
\psframe[linestyle=none,fillstyle=solid,fillcolor=lightgray](.75,3.5)(1.25,5)%
\psset{linestyle=dashed}
\psline(.75,0)(.75,5)%
\psline(1.25,0)(1.25,5)%

\psline[linecolor=gray,linestyle=dashed](1,0)(1,3)(0,3)%
\rput(1,-.5){$1$}%
\rput(-.25,3){$3$}%

\end{pspicture}
%
\begin{pspicture}(-1,-1)(3.5,3)
\psset{xunit=.8,yunit=.5}%
\psaxes[ticks=none,labels=none]{->}(0,0)(-1,-1)(4,6)%
\uput[-90](4,0){$x$}%
\uput[180](0,6){$g(x)$}%
\psplot{-1}{.95}{2 x mul 1 add}%
\psplot{1.05}{2}{2 x mul 1 add}%

\psframe[linestyle=none,fillstyle=solid,fillcolor=lightgray](0,2.5)(.75,3.5)%
\psframe[linestyle=none,fillstyle=solid,fillcolor=lightgray](1.25,2.5)(2,3.5)%
\psset{linestyle=dashed}
\psline(0,2.5)(2,2.5)%
\psline(0,3.5)(2,3.5)%

\psframe[linestyle=none,fillstyle=solid,fillcolor=lightgray](.75,0)(1.25,2.5)%
\psframe[linestyle=none,fillstyle=solid,fillcolor=lightgray](.75,3.5)(1.25,5)%
\psset{linestyle=dashed}
\psline(.75,0)(.75,5)%
\psline(1.25,0)(1.25,5)%

\pscircle(1,3){.05}%
\pscircle[fillstyle=solid,fillcolor=black](1,1){.05}%
\psline[linecolor=gray,linestyle=dashed](1,0)(1,1)(0,1)%
\rput(1,-.5){$1$}%
\rput(-.25,3){$3$}%
\rput(-.25,1){$1$}%

\end{pspicture}
%
\begin{pspicture}(-1,-1)(3.5,3)
\psset{xunit=.8,yunit=.5}%
\psaxes[ticks=none,labels=none]{->}(0,0)(-1,-1)(4,6)%
\uput[-90](4,0){$x$}%
\uput[180](0,6){$h(x)$}%
\psplot{-1}{.95}{2 x mul 1 add}%
\psplot{1.05}{2}{2 x mul 1 add}%

\psframe[linestyle=none,fillstyle=solid,fillcolor=lightgray](0,2.5)(.75,3.5)%
\psframe[linestyle=none,fillstyle=solid,fillcolor=lightgray](1.25,2.5)(2,3.5)%
\psset{linestyle=dashed}
\psline(0,2.5)(2,2.5)%
\psline(0,3.5)(2,3.5)%

\psframe[linestyle=none,fillstyle=solid,fillcolor=lightgray](.75,0)(1.25,2.5)%
\psframe[linestyle=none,fillstyle=solid,fillcolor=lightgray](.75,3.5)(1.25,5)%
\psset{linestyle=dashed}
\psline(.75,0)(.75,5)%
\psline(1.25,0)(1.25,5)%

\pscircle(1,3){.05}%
\rput(1,-.5){$1$}%
\rput(-.25,3){$3$}%
\end{pspicture}
\end{center}

Veamos un ejemplo.

\begin{exemplo}[Solución]{%
Sea $\funcjc{f}{\mathbb{R}}{\mathbb{R}}$ la función definida por $f(x)=x^2$ para todo
$x\in\mathbb{R}$.

\vspace*{0.5\baselineskip}
\begin{enumerate}
\item Encuentre un $\delta>0$ tal que si $0 < |x-3| < \delta$ entonces $|f(x) - 9| < \epsilon =
    \frac{1}{2}$.
\item Construya una caja, tal como está definida en el texto, que ilustre el resultado
    encontrado en el primer punto.
\end{enumerate}
}%
\begin{enumerate}[leftmargin=*]
\item En el ejemplo de la página~\pageref{ex:lim001}, probamos que
      \[
        9 = \limjc{x^2}{x}{3}.
      \]
      Para probar esta igualdad, dado $\epsilon > 0$, encontramos que el número
      \[
        \delta = \min\left\{1,\frac{\epsilon}{7}\right\}
      \]
      es tal que, si $|x - 3| < \delta$, entonces $|x^2 - 9| < \epsilon$.

      Por lo tanto, como $\epsilon = \frac{1}{2}$, el número buscado es:
      \[
        \delta = \min\left\{1,\frac{\frac{1}{2}}{7}\right\} = \frac{1}{14}.
      \]

      En resumen, si $|x - 3| < \frac{1}{14}$, entonces $|x^2 - 9| < \frac{1}{2}$.

\item Como $9 = \limjc{x^2}{x}{3}$, dado $\epsilon = \frac{1}{2}$, el conjunto
      \[
        C = \big\{(x,y) \in \mathbb{R}^2 : x \in \big]3 - \frac{1}{14}, 3 + \frac{1}{14}\big[, \
        y \in \big]9 - \frac{1}{2}, 9 + \frac{1}{2}\big[\big\}
      \]
      es una caja para el gráfico de $f$ en el punto de coordenadas $(3,9)$.

      El siguiente, es un dibujo de esta caja:
      \begin{center}
      \psset{unit=0.8}
      \psset{xAxisLabel={},yAxisLabel={},labelFontSize=\scriptstyle}
      \begin{psgraph}[arrows=->,Dy=3](0,0)(-0.25,-1)(3.75,11){0.8\textwidth}{6cm}
          \uput[0](0,11){$y$}%
          \uput[-90](4,0){$x$}%

          \psset{PointSymbol=none,PointName=none}
          \pstGeonode[]%
            (! 3 1 14 div sub 0){A}(! 3 1 14 div add 0){B}%
            (! 3 1 14 div sub 10.5){A'}%
            (! 3 1 14 div add 10.5){B'}%
            (! 0 9 1 2 div sub){C}(! 0 9 1 2 div add){D}%
            (! 3.5 9 1 2 div sub){C'}(! 3.5 9 1 2 div add){D'}

          \pstInterLL[]%
            {A}{A'}{C}{C'}{U}%
          \pstInterLL[]%
            {B}{B'}{C}{C'}{V}%
          \pstInterLL[]%
            {A}{A'}{D}{D'}{W}%
          \pstInterLL[]%
            {B}{B'}{D}{D'}{X}%

          {\psset{linestyle=none,fillstyle=solid,fillcolor=lightgray}%
          \psframe[]%
            (A)(V)%
          \psframe[]%
            (C)(W)%
          \psframe[]%
            (W)(B')%
          \psframe[]%
            (V)(D')%
          }

          {\psset{linestyle=dashed}
          \pstLineAB[]%
            {A}{A'}%
          \pstLineAB[]%
            {B}{B'}%
          \pstLineAB[]%
            {C}{C'}%
          \pstLineAB[]%
            {D}{D'}%
          }

          \scriptsize%
          \rput[tr](A){$3 - \frac{1}{14}$}%
          \rput[tl](B){$3 + \frac{1}{14}$}%
          \rput[tr](C){$9 - \frac{1}{2}$}%
          \rput[br](D){$9 + \frac{1}{2}$}%

      \end{psgraph}
      \end{center}

      Ahora bien, por la definición de caja, como
      \[
      9 = \limjc{x^2}{x}{3},
      \]
      el gráfico de la función $f$ en el intervalo $]3 - \frac{1}{14}, 3 + \frac{1}{14}[$ debe
      estar contenido plenamente en la caja. Constatemos esto.

      En primer lugar, veamos que $f(x)$ es mayor que $9 - \frac{1}{2}$ cuando $x = 3 -
      \frac{1}{14}$. Por un lado tenemos que:
      \begin{align*}
        f\left(3 - \frac{1}{14}\right) &= f\left(\frac{41}{14}\right) \\
          &= \frac{1\,681}{196} = 8 + \frac{113}{196},
      \end{align*}
      Por otro lado:
      \begin{align*}
      9 - \frac{1}{2} &= 8 + \frac{1}{2} = 8 + \frac{98}{196}.
      \end{align*}
      Por lo tanto:
      \[
          f\left(3 - \frac{1}{14}\right) = 8 + \frac{113}{196} > 8 + \frac{98}{196} = 9 - \frac{1}{2}.
      \]

      En segundo lugar, veamos que $f(x)$ es menor que $9 + \frac{1}{2}$ cuando $x = 3 +
      \frac{1}{14}$. Por un lado tenemos que:
      \begin{align*}
        f\left(3 + \frac{1}{14}\right) &= f\left(\frac{43}{14}\right) \\
          &= \frac{1\,849}{196} = 9 + \frac{85}{196},
      \end{align*}
      Por otro lado:
      \begin{align*}
      9 + \frac{1}{2} &= 9 + \frac{1}{2} = 9 + \frac{98}{196}.
      \end{align*}
      Por lo tanto:
      \[
          f\left(3 - \frac{1}{14}\right) = 9 + \frac{85}{196} < 9 + \frac{98}{196} = 9 + \frac{1}{2}.
      \]

      Finalmente, como la función $f$ es creciente en $[0,+\infty[$, entonces
      \[
        9 - \frac{1}{2} < f(x) < 9 + \frac{1}{2}
      \]
      siempre que
      \[
        3 - \frac{1}{14} < x < 3 + \frac{1}{14}.
      \]
      Por esta razón, al dibujar el gráfico de $f$ en este intervalo, éste deberá estar
      contenido plenamente en la caja, como lo asegura la definición de límite y de caja. El
      siguiente es un dibujo de la situación:
      \begin{center}
      \psset{unit=0.8}
      \psset{xAxisLabel={},yAxisLabel={},labelFontSize=\scriptstyle}
      \begin{psgraph}[arrows=->,Dy=3](0,0)(-0.25,-1)(3.75,11){0.8\textwidth}{6cm}
          \uput[0](0,11){$y$}%
          \uput[-90](4,0){$x$}%

          \psset{PointSymbol=none,PointName=none}
          \pstGeonode[]%
            (! 3 1 14 div sub 0){A}(! 3 1 14 div add 0){B}%
            (! 3 1 14 div sub 10.5){A'}%
            (! 3 1 14 div add 10.5){B'}%
            (! 0 9 1 2 div sub){C}(! 0 9 1 2 div add){D}%
            (! 3.5 9 1 2 div sub){C'}(! 3.5 9 1 2 div add){D'}

          \pstInterLL[]%
            {A}{A'}{C}{C'}{U}%
          \pstInterLL[]%
            {B}{B'}{C}{C'}{V}%
          \pstInterLL[]%
            {A}{A'}{D}{D'}{W}%
          \pstInterLL[]%
            {B}{B'}{D}{D'}{X}%

          {\psset{linestyle=none,fillstyle=solid,fillcolor=lightgray}%
          \psframe[]%
            (A)(V)%
          \psframe[]%
            (C)(W)%
          \psframe[]%
            (W)(B')%
          \psframe[]%
            (V)(D')%
          }

          {\psset{linestyle=dashed}
          \pstLineAB[]%
            {A}{A'}%
          \pstLineAB[]%
            {B}{B'}%
          \pstLineAB[]%
            {C}{C'}%
          \pstLineAB[]%
            {D}{D'}%
          }

          \psplot[plotpoints=200]%
            {0}{3.25}{x dup mul}%
          \uput[0](! 3.25 /x 3.25 def x dup mul){$y = x^2$}%

          \scriptsize%
          \rput[tr](A){$3 - \frac{1}{14}$}%
          \rput[tl](B){$3 + \frac{1}{14}$}%
          \rput[tr](C){$9 - \frac{1}{2}$}%
          \rput[br](D){$9 + \frac{1}{2}$}%

      \end{psgraph}
      \end{center}

\end{enumerate}

\end{exemplo}

Para terminar esta sección, veamos cómo la interpretación geométrica nos puede ser de ayuda para
demostrar que un cierto número $L$ no es límite de $f(x)$ cuando $x$ se aproxima al número $a$.

De la definición de límite, decir que $L$ no es el límite equivale a decir que:
\begin{quote}
{\bfseries existe un número $\bm{\epsilon > 0}$ tal que, para todo número $\bm{\delta > 0}$, existe
un número $\bm{x}$ tal que
\[
\bm{0 < |x - a| < \delta} \yjc \quad \bm{|f(x) - L| \geq \epsilon}.
\]
}
\end{quote}

Ilustremos estas ideas con un ejemplo.

\begin{exemplo}[Solución]{%
Demostrar que
\[
\limjc{f(x)}{x}{1} \neq \frac{3}{2}
\]
si
\[
f(x) =
\begin{cases}
x & \text{si } x \in\ [0,1],\\
x + 1 & \text{si } x \in\ ]1,2].
\end{cases}
\]
}%
Para demostrar que el número $\frac{3}{2}$ no es el límite de $f(x)$ cuando $x$ se aproxima a $1$
si $f$ está definida de la siguiente manera:
\[
f(x) =
\begin{cases}
x & \text{si } x \in\ [0,1],\\
x + 1 & \text{si } x \in\ ]1,2],
\end{cases}
\]
debemos hallar un número $\epsilon > 0$ tal que, para todo $\delta > 0$, encontremos un número
$x\in \Dm(f)$ tal que
\[
0 < |x - 1| < \delta \yjc \left|f(x) - \frac{3}{2}\right| \geq \epsilon.
\]

Para ello, primero dibujemos la función $f$:
\begin{center}
\begin{pspicture}(-.5,-.5)(4,3)
\psset{xunit=1.5,yunit=.8}%
\psaxes[ticks=none,labels=none]{->}(0,0)(-.5,-.5)(2.5,3.5)%
\uput[-90](2.5,0){$x$}%
\uput[180](0,3.5){$f(x)$}%

\psplot{0}{1}{x}%
\psplot[arrows=o-]{1.03}{2}{x 1 add}%
%\pscircle(1,2){.03}%

\psline(0,1)(.1,1)%
\rput[r](-.1,1){\footnotesize{$1$}}%

\psline(0,1.5)(.1,1.5)%
\rput[r](-.1,1.5){\small{$\frac{3}{2}$}}%

\psline(0,2)(.1,2)%
\rput[r](-.1,2){\footnotesize{$2$}}%

\psline(1,0)(1,.12)%
\rput[t](1,-.12){\footnotesize{$1$}}%

\end{pspicture}
\end{center}

Este dibujo nos permite ver que cualquier caja del gráfico de $f$ ubicada entre las rectas
horizontales de ecuaciones $y=1$ y $y = 2$ no contendrá ningún elemento del gráfico de $f$.
Recordemos que el alto de la caja es $2\epsilon$: una distancia de un $\epsilon$ sobre
$\frac{3}{2}$ y una distancia de un $\epsilon$ bajo $\frac{3}{2}$. Por lo tanto, si elegimos
$\epsilon$ igual a $\frac{1}{4}$, la banda horizontal alrededor de $\frac{3}{2}$ se vería así:
\begin{center}
\begin{pspicture}(-.5,-.5)(4,3)
\psset{xunit=1.5,yunit=.8}%
\psaxes[ticks=none,labels=none]{->}(0,0)(-.5,-.5)(2.5,3.5)%
\uput[-90](2.5,0){$x$}%
\uput[180](0,3.5){$f(x)$}%


\psplot{0}{1}{x}%
\psplot{1.03}{2}{x 1 add}%
\pscircle(1,2){.03}%

\psframe[linestyle=none,fillstyle=solid,fillcolor=lightgray](0,1.25)(1.75,1.75)%
\psset{linestyle=dashed}
\psline(0,1.25)(1.75,1.25)%
\psline(0,1.75)(1.75,1.75)%


\psline(0,1)(.1,1)%
\rput[r](-.1,1){\footnotesize{$1$}}%

\psline(0,1.5)(.1,1.5)%
\rput[r](-.1,1.5){\small{$\frac{3}{2}$}}%

\psline(0,2)(.1,2)%
\rput[r](-.1,2){\footnotesize{$2$}}%

\psline(1,0)(1,.12)%
\rput[t](1,-.12){\footnotesize{$1$}}%

\end{pspicture}
\end{center}

A continuación, podemos dibujar tres bandas verticales alrededor del número $1$, lo que nos
sugerirá que ninguna caja del gráfico de $f$ con el alto $2\epsilon$ elegido contendrá elementos
del gráfico:

\begin{center}
\begin{pspicture}(-.5,-.5)(4,3)
\psset{xunit=1.5,yunit=.8}%
\psaxes[ticks=none,labels=none]{->}(0,0)(-.5,-.5)(2.5,3.5)%
\uput[-90](2.5,0){$x$}%
\uput[180](0,3.5){$f(x)$}%

\rput[Br](.6,-.4){\footnotesize{$1 - \delta$}}%
\rput[Bl](1.4,-.4){\footnotesize{$1 + \delta$}}%

\psframe[linestyle=none,fillstyle=solid,fillcolor=lightgray](0,1.25)(.6,1.75)%
\psframe[linestyle=none,fillstyle=solid,fillcolor=lightgray](1.4,1.25)(1.75,1.75)%
\psset{linestyle=dashed}
\psline(0,1.25)(1.75,1.25)%
\psline(0,1.75)(1.75,1.75)%

\psframe[linestyle=none,fillstyle=solid,fillcolor=lightgray](.6,0)(1.4,1.25)%
\psframe[linestyle=none,fillstyle=solid,fillcolor=lightgray](.6,1.75)(1.4,2.5)%
\psset{linestyle=dashed}
\psline(.6,0)(.6,2.5)%
\psline(1.4,0)(1.4,2.5)%

\psline(0,1)(.1,1)%
\rput[r](-.1,1){\footnotesize{$1$}}%

\psline(0,1.5)(.1,1.5)%
\rput[r](-.1,1.5){\small{$\frac{3}{2}$}}%

\psline(0,2)(.1,2)%
\rput[r](-.1,2){\footnotesize{$2$}}%

\psline(1,0)(1,.12)%
\rput[B](1,-.4){\footnotesize{$1$}}%

\psset{linestyle=solid}
\psplot{0}{1}{x}%
\psplot{1.03}{2}{x 1 add}%
\pscircle(1,2){.03}%

\end{pspicture}
%
\begin{pspicture}(-.5,-.5)(4,3)
\psset{xunit=1.5,yunit=.8}%
\psaxes[ticks=none,labels=none]{->}(0,0)(-.5,-.5)(2.5,3.5)%
\uput[-90](2.5,0){$x$}%
\uput[180](0,3.5){$f(x)$}%

\rput[Br](.75,-.4){\footnotesize{$1 - \delta$}}%
\rput[Bl](1.25,-.4){\footnotesize{$1 + \delta$}}%

\psframe[linestyle=none,fillstyle=solid,fillcolor=lightgray](0,1.25)(.75,1.75)%
\psframe[linestyle=none,fillstyle=solid,fillcolor=lightgray](1.25,1.25)(1.75,1.75)%
\psset{linestyle=dashed}
\psline(0,1.25)(1.75,1.25)%
\psline(0,1.75)(1.75,1.75)%

\psframe[linestyle=none,fillstyle=solid,fillcolor=lightgray](.75,0)(1.25,1.25)%
\psframe[linestyle=none,fillstyle=solid,fillcolor=lightgray](.75,1.75)(1.25,2.5)%
\psset{linestyle=dashed}
\psline(.75,0)(.75,2.5)%
\psline(1.25,0)(1.25,2.5)%

\psline(0,1)(.1,1)%
\rput[r](-.1,1){\footnotesize{$1$}}%

\psline(0,1.5)(.1,1.5)%
\rput[r](-.1,1.5){\small{$\frac{3}{2}$}}%

\psline(0,2)(.1,2)%
\rput[r](-.1,2){\footnotesize{$2$}}%

\psline(1,0)(1,.12)%
\rput[B](1,-.3){\footnotesize{$1$}}%

\psset{linestyle=solid}
\psplot{0}{1}{x}%
\psplot{1.03}{2}{x 1 add}%
\pscircle(1,2){.03}%

\end{pspicture}
%
\begin{pspicture}(-.5,-.5)(4,3)
\psset{xunit=1.5,yunit=.8}%
\psaxes[ticks=none,labels=none]{->}(0,0)(-.5,-.5)(2.5,3.5)%
\uput[-90](2.5,0){$x$}%
\uput[180](0,3.5){$f(x)$}%

\rput[Br](.85,-.4){\footnotesize{$1 - \delta$}}%
\rput[Bl](1.15,-.4){\footnotesize{$1 + \delta$}}%

\psframe[linestyle=none,fillstyle=solid,fillcolor=lightgray](0,1.25)(.85,1.75)%
\psframe[linestyle=none,fillstyle=solid,fillcolor=lightgray](1.15,1.25)(1.75,1.75)%
\psset{linestyle=dashed}
\psline(0,1.25)(1.75,1.25)%
\psline(0,1.75)(1.75,1.75)%

\psframe[linestyle=none,fillstyle=solid,fillcolor=lightgray](.85,0)(1.15,1.25)%
\psframe[linestyle=none,fillstyle=solid,fillcolor=lightgray](.85,1.75)(1.15,2.5)%
\psset{linestyle=dashed}
\psline(.85,0)(.85,2.5)%
\psline(1.15,0)(1.15,2.5)%

\psline(0,1)(.1,1)%
\rput[r](-.1,1){\footnotesize{$1$}}%

\psline(0,1.5)(.1,1.5)%
\rput[r](-.1,1.5){\small{$\frac{3}{2}$}}%

\psline(0,2)(.1,2)%
\rput[r](-.1,2){\footnotesize{$2$}}%

\psline(1,0)(1,.12)%
\rput[B](1,-.4){\footnotesize{$1$}}%

\psset{linestyle=solid}
\psplot{0}{1}{x}%
\psplot{1.03}{2}{x 1 add}%
\pscircle(1,2){.03}%

\end{pspicture}
\end{center}
Estos dibujos nos sugieren cuál debe ser el $x$ que buscamos para el cual se verifiquen las
condiciones siguientes:
\[
0 < |x - 1| < \delta\yjc \left|f(x) - \frac{3}{2}\right| \geq \epsilon.
\]
Cualquier $x\neq 1$ que esté en el dominio de $f$ y en el intervalo $]1-\delta,1+\delta[$ satisfará
estas dos condiciones.

Procedamos, entonces, a demostrar que el número $\frac{3}{2}$ no es el límite de $f(x)$ cuando $x$
se aproxima a $1$.

Sean $\epsilon = \frac{1}{4}$ y $\delta > 0$. Si $\delta > 1$, la banda vertical abarcaría más que
el dominio de la función $f$. En este caso, $x = 0$, satisfaría las condiciones. En efecto:
\begin{enumerate}
\item $x$ está en el dominio de $f$;
\item Como $x = 0$, entonces $x\neq 1$ y
\[
|x - 1| = |0 - 1| = 1 < \delta;
\]
\item Como $0 \in\ [0,1]$, entonces $f(0) = 0$ y, por ello:
\begin{align*}
\left|f(x) - \frac{3}{2}\right| &= \left|0 - \frac{3}{2}\right| \\
&= \frac{3}{2} > \frac{1}{4} = \epsilon.
\end{align*}
\end{enumerate}

Ahora, si $\delta \leq 1$, podemos elegir el número $x$ que está en la mitad entre $1$ y
$1-\delta$. Ese número es:
\[
x = \frac{(1-\delta) + 1}{2} = 1 - \frac{\delta}{2}.
\]
Entonces:
\begin{enumerate}
\item $x$ está en el dominio de $f$, pues, como
\begin{enumerate}
\item $1 - \delta \geq 0$, ya que $\delta \leq 1$ y
\item $\frac{\delta}{2} < \delta$, ya que $\delta > 0$,
\end{enumerate}
se tiene que
\[
0 \leq 1 - \delta < 1 - \frac{\delta}{2} < 1;
\]
es decir, $x\in [0,1[$.
\item Como $\delta > 0$, entonces $\frac{\delta}{2} > 0$, entonces
\[
x = 1 - \frac{\delta}{2} \neq 1.
\]
\item Como $x\in\ [0,1]$, entonces
\[
f(x) = x = 1 - \frac{\delta}{2}.
\]
Por lo tanto:
\begin{align*}
\left|f(x) - \frac{3}{2}\right| &= \left|\left(1 - \frac{\delta}{2}\right) - \frac{3}{2}\right|
\\
&= \left|-\frac{1}{2} - \frac{\delta}{2}\right| = \frac{1}{2} + \frac{\delta}{2} > \frac{1}{4} = \epsilon,
\end{align*}
pues
\[
\frac{1}{2} > \frac{1}{4}\yjc \frac{\delta}{2} > 0.
\]
\end{enumerate}

Por lo tanto, el número $\frac{3}{2}$ no puede ser el límite de $f(x)$ cuando $x$ se aproxima a
$1$.
\end{exemplo}
%-----> 2008 09 12
%{\color{red} Quizás aquí amerite también un ejercicio en el que se muestre que ningún número real
%$L$ puede ser límite de la función $f$ de este último ejemplo.}

En la sección próxima, vamos a mostrar un ejemplo de cómo se puede resolver un problema aplicando
el método de encontrar un número $\delta$ dado el número $\epsilon$.

\subsection{Ejercicios}
\begingroup
\small En los ejercicios propuestos a continuación, se busca que el lector adquiera un dominio
básico del aspecto operativo de la definición de límite, a pesar de que luego, en la práctica del
cálculo de límites, no se recurra a esta definición, sino a las propiedades que se obtienen y se
demuestran a partir de esta definición.

\begin{multicols}{2}
\begin{enumerate}[leftmargin=*]
\item Sean $\funcjc{f}{I}{\mathbb{R}}$, $a\in\mathbb{R}$, $L\in\mathbb{R}$ y $\epsilon > 0$:
\begin{enumerate}[leftmargin=*]
\item encuentre un $\delta > 0$ tal que $|f(x) - L| < \epsilon$ siempre que $0 < |x - a| <
    \delta$; y
\item construya una caja de $f(x)$ centrada en $(a,L)$ y de altura $2\epsilon$ que ilustre
    el resultado encontrado en el punto anterior
\end{enumerate}
para cada $f$, $a$, $L$ y $\epsilon$ dados a continuación:
\begin{enumerate}[leftmargin=*]
\item $f(x) = x - 2$; $a = 5$; $L = 3$; $\epsilon = 0.02$.
\item $f(x) = x - 2$; $a = -2$; $L = -4$; $\epsilon = 0.01$.
\item $f(x) = -2x + 1$; $a = -2$; $L = 5$; $\epsilon = 0.05$.
\item $f(x) = 4x - 3$; $a = 1$; $L = 1$; $\epsilon = 0.03$.
\item $f(x) = \frac{2x^2 - 7x + 6}{x - 2}$; $a = 2$; $L = 1$; $\epsilon = 0.04$.
\item $f(x) = \frac{8x^2 + 10x + 3}{2x + 1}$; $a = -\frac{1}{2}$; $L = 1$; $\epsilon = 0.003$.
\item $f(x) = x^2$; $a = 3$; $L = 9$; $\epsilon = 1$.
\item $f(x) = x^2$; $a = 3$; $L = 9$; $\epsilon = 0.1$.
\item $f(x) = x^2$; $a = -2$; $L = 4$; $\epsilon = 0.3$.
\item $f(x) = x^2 + 3x - 5$; $a = 1$; $L = -1$; $\epsilon = 0.4$.
\item $f(x) = x^2 + 3x - 5$; $a = -1$; $L = -7$; $\epsilon = 0.08$.
\end{enumerate}

\item Demuestre, utilizando la definición de límite:
   \begin{enumerate}[leftmargin=*]
   \item $\displaystyle\limjc{\frac{3x^2 + 5x - 2}{x + 2}}{x}{-2} = -7$.
   \item $\displaystyle\limjc{\frac{6x}{x + 2}}{x}{1} = 2$.
   \item $\displaystyle\limjc{\sqrt{x + 7}}{x}{2} = 3$.
   \item $\displaystyle\limjc{\frac{x^3 - 2x^2 - 3x}{x - 3}}{x}{3} = 12$.
   \item $\displaystyle\limjc{\frac{8x^2 + 10x + 3}{2x + 1}}{x}{-\frac{1}{2}} = 1$.
   \item $\displaystyle\limjc{(2x^2 - 3x + 4)}{x}{1} = 3$.
   \item $\displaystyle\limjc{\frac{x^2 - 7x + 5}{-2x + 3}}{x}{1} = -1$.
   \end{enumerate}
\end{enumerate}
\end{multicols}
\endgroup

\section{Energía solar para Intipamba}
Intipamba es una pequeña comunidad asentada en una meseta. Desde allí, se tiene la sensación de
poder acariciar las trenzadas montañas que, más abajo, están rodeadas por un límpido cielo azul la
mayor parte del año. El sol es un visitante asiduo de Intipamba. En algunas ocasiones se ha quedado
tanto tiempo que las tierras alrededor de Intipamba no han podido dar su fruto, y los lugareños han
tenido que ir muy lejos en busca de agua.

Los viejos ya no recuerdan cuando llegó la energía eléctrica a Intipamba y los jóvenes, si no la
tuvieran ya, la extrañarían, pues, desde que tienen memoria, la han tenido. Pero en los últimos
años la energía eléctrica que llega al pueblo ya no es suficiente. El gobierno les ha dicho que
llevar más energía es muy costoso. Sin embargo, una organización no gubernamental ha propuesto a la
comunidad solventar la falta de energía mediante la construcción de una pequeña planta que genere
energía eléctrica a partir de energía solar, para aprovechar, así, ``el sol'' de Intipamba.

Para abaratar los costos, la comunidad entera participará en la construcción de la planta. Entre
las tareas que realizarán los lugareños, está la construcción de los paneles solares. Las
especificaciones técnicas indican que los paneles deben tener una forma rectangular, que el largo
debe medir el doble que el ancho, el cuál no debe superar los tres metros, y que el área debe ser
de ocho metros cuadrados, tolerándose a lo más un error del uno por ciento.

Se ha decidido que cada panel tenga por dimensiones 2 y 4 metros, lo que satisface el requerimiento
de que la longitud del largo sea el doble que la del ancho. El problema en que se encuentran los
encargados de la construcción de los paneles consiste en determinar con qué grado de precisión debe
calibrarse la maquinaria que corta los paneles para que se garantice que el área obtenida no
difiera de los ocho metros cuadrados en más del uno por ciento.

\subsection{Planteamiento del problema}

En primer lugar, ?`qué significa que el área no difiera de los ocho metros cuadrados en más del uno
por ciento? Ninguna máquina cortará los lados de cada panel de dos y cuatro metros exactamente,
sino que, en algunos cortes, por ejemplo, la dimensión del ancho será un poco menos que dos metros;
en otros cortes, podría ser un poco más; una situación similar ocurrirá respecto de la dimensión
del largo. El resultado de esto es que el área de cada panel no será exactamente igual a ocho
metros cuadrados.

La diferencia entre el área real de un panel y los ocho metros cuadrados esperados se denomina
\emph{error}. Esta cantidad puede ser positiva o negativa, según el área real sea mayor o menor a
ocho metros cuadrados. Para cuantificar únicamente la diferencia entre el área real y la esperada,
se define el concepto de \emph{error absoluto} como el valor absoluto del error. El error absoluto
no da cuenta de si el área obtenida es mayor o menor que los ocho metros cuadrados esperados, solo
da cuenta de la diferencia positiva entre estos dos valores.

Si se representa con la letra $a$ el número de metros cuadrados que mide el área real de un panel,
el error absoluto se expresa de la siguiente manera:
\[
\text{error absoluto } = |a - 8|.
\]

La especificación de que ``el área no difiera de los ocho metros cuadrados en más del uno por
ciento'' quiere decir, entonces, que el error absoluto sea menor que el uno por ciento del área
esperada; es decir, que el error absoluto sea menor que el uno por ciento de ocho metros, valor
igual a 0.08 metros cuadrados. Por lo tanto, para que se cumpla con la especificación técnica, la
calibración de la máquina debe asegurar la verificación de la siguiente desigualdad:
\[
|a - 8| < 0.08.
\]

En resumen, el problema a resolver por el equipo de Intipamba es:%
\marcojc{.9}{1.5}{black}{black}{white}{%
calibrar la máquina de los paneles para garantizar que la desigualdad
\begin{equation}
\label{eqLim001}%
|a - 8| < 0.08.
\end{equation}
sea verdadera.}

\subsection{El modelo}

?`Qué se quiere decir con ``calibrar la máquina'' para garantizar que la
desigualdad~(\ref{eqLim001}) sea verdadera? Desde el punto de vista de la máquina, cuando ésta
``dice que va a cortar el ancho de dos metros'', cortará, en realidad, de un poco más de dos
metros, en algunas ocasiones; en otras, de un poco menos que dos metros. Lo mismo sucederá con el
largo. Esto significa que el área del panel obtenida en cada corte no será, con toda seguridad,
igual a ocho metros cuadrados. La calibración consiste, entonces, en decirle a la máquina en cuánto
se puede equivocar a lo más en el corte del largo y del ancho para que asegure a los lugareños de
Intipamba que el ``error absoluto'' en el área del panel no difiera de ocho metros cuadrados en más
del uno por ciento. En otras palabras, los constructores de los paneles quieren saber en cuánto la
máquina puede equivocarse en el corte del largo y del ancho para que el área del panel no difiera
en más del uno por ciento; es decir, quieren saber el valor máximo permitido para el error en las
longitudes de los cortes, de manera que el área no difiera en más del uno por ciento con el área
esperada.

El área real de cada panel no es, entonces, constante; dependerá de las dos dimensiones del panel
que la máquina cortadora produzca realmente. Bajo la suposición de que ésta máquina siempre logra
cortar el largo de una longitud el doble que la del ancho, el área real de cada panel se puede
expresar exclusivamente en función de una de las dimensiones. Por ejemplo, el ancho.

En efecto, los encargados de la construcción de los paneles utilizan la letra $x$ para representar
el número de metros que mide el ancho real de un panel; entonces, $2x$ representa el número de
metros que mide el largo del panel. De esta manera, si el área de cada panel es de $a$ metros
cuadrados, $a$ puede ser expresada en función de $x$ de la siguiente manera:
\[
a = 2x^2.
\]
Vamos a representar con la letra $A$ esta relación funcional entre $a$ y $x$. Así, la función $A$
definida por
\[
A(x) = 2x^2
\]
nos permite escribir las siguientes igualdades:
\[
a = A(x) = 2x^2.
\]

?`Cuál es el dominio de la función $A$? La respuesta se encuentra después de averiguar los valores
que puede tomar la variable $x$. Dentro de las especificaciones técnicas para la construcción de
los paneles, se indica que el ancho del panel no puede superar los tres metros; como, además, el
ancho no puede ser medido por el número cero ni por un número negativo, se puede elegir\footnote{No
hay una sola manera de realizar esta elección; es decir, hay varias alternativas para el dominio de
la función $A$. En efecto, aunque los constructores de los paneles no saben de cuánto exactamente
es el ancho de cada panel, están seguros que no podrá ser muy diferente de dos metros. Esto
significa que se podría elegir como dominio de la función $A$, por ejemplo, el intervalo $[1.5,
2.5]$; también se podría tomar el intervalo $[1,2.5]$.} como dominio de $A$ el intervalo $]0,3]$.
En resumen, $A$ es una función de $]0,3]$ en $\mathbb{R}^+$.

El equipo de Intipamba encargado de los paneles ya pueden representar simbólicamente el error
absoluto que comete la máquina en el corte del ancho de un panel: el valor absoluto de la
diferencia entre $x$, el valor de la longitud real del ancho, y 2, el valor esperado para el ancho:
\[
\text{error absoluto } = |x - 2|.
\]

El problema que tienen los constructores es, entonces, saber de qué valor no debe superar este
error absoluto para asegurar que el área del panel no difiera del uno por ciento del área esperada.
Es decir, los constructores necesitan encontrar un número $\delta > 0$ tal que
\[
\text{si } |x - 2| < \delta, \text{ entonces } |A(x) - 8| < 0.08;
\]
es decir, deben hallar un $\delta > 0$ tal que
\[
\text{si } |x - 2| < \delta, \text{ entonces } |2x^2 - 8| < 0.08.
\]

Ahora los lugareños de Intipamba acaban de formular un \emph{modelo matemático} del problema; es
decir, tienen una representación mediante símbolos que, por un lado, representan elementos del
problema de construcción de los paneles, pero que, por otro lado, representan conceptos matemáticos
que, en su interrelación, plantean un problema ``matemático'' que, al ser resuelto, permita ofrecer
una solución al problema de los paneles solares. El modelo se puede resumir de la siguiente manera:
\marcojc{.92}{1.5}{black}{black}{white}{%
\begin{center}
\textbf{Modelo para el problema de los paneles solares}
\end{center}
Sean
  \eil\eijc{-.5}\\
    {\setlength\tabcolsep{3pt}
    \begin{tabular}{r p{0.9\textwidth}}
    $x:$ & \textsl{el número de metros que mide el ancho real de un panel y que no puede ser
    mayor que tres metros.}\\
    $a:$ & \textsl{el número de metros cuadrados que mide el área real de un panel cuyo ancho
    y largo miden $x$ y $2x$ metros, respectivamente.}
  \end{tabular}}
  \eil\\
$A$ es un función de $]0,3]$ en $\mathbb{R}^+$ definida por
\[
A(x) = a = 2x^2.
\]
Se busca un número $\delta > 0$ tal que si la desigualdad
\[
\tag{\ref{eqLim002}} |x - 2| < \delta
\]
fuera verdadera, la desigualdad
\[
\tag{\ref{eqLim003}} |A(x) - 8| = |2x^2 - 8| < 0.08
\]
también sería verdadera.%
}

\subsection{El problema matemático}
El modelo plantea un problema exclusivamente matemático:
\marcojc{.9}{1.5}{black}{black}{white}{%
Dada la función $\funjc{A}{]0,3]}{\mathbb{R}}$ definida por
\[
A(x) = 2x^2,
\]
se busca un número real $\delta > 0$ tal que si la desigualdad
\[
\tag{\ref{eqLim002}}%
|x - 2| < \delta
\]
fuera verdadera, la desigualdad
\[
\tag{\ref{eqLim003}}%
|A(x) - 8| = |2x^2 - 8| < 0.08
\]
también sería verdadera.}%
En esta formulación, el significado de los símbolos que aparecen ($A$, $x$ y $\delta$) carece de
importancia. Solo importa que estos símbolos representan, respectivamente, una función, cualquier
número real en el intervalo $]0,3]$ y un número real positivo.

El reto en este momento es resolver este problema. Si lo podemos hacer, usaremos la solución
encontrada para responder la pregunta que se formularon los habitantes de Intipamba: ?`cuál es la
calibración de la máquina que corta los paneles solares para garantizar que el área de los mismos
no difiera de los ocho metros cuadrados en más del uno por ciento?

\subsection{Solución del problema matemático}
Podemos aplicar el método aprendido en la sección anterior para encontrar el número $\delta$. Sin
embargo, para resolver este problema, por tratarse de $2x^2$, podemos hacer un atajo en el
método.

Para empezar, tenemos que:
\begin{align*}
|2x^2 - 8| &= |2(x^2 - 4)| \\
&= 2|x^2 - 4| \\
&= 2|(x - 2)(x + 2)| \\
&= 2|x-2||x+2|.
\end{align*}
Es decir,
\begin{equation}
\label{eqLim014}%
|2x^2 - 8| = 2|x-2||x+2|.
\end{equation}

Ahora debemos acotar $g(x) = 2|x+2|$. Para ello, recordemos que $x$ representa un número que
pertenece al dominio de la función $A$, es decir, $x$ está en el intervalo $]0,3]$. Por lo tanto,
se verifican las desigualdades siguientes:
\[
0 < x \leq 3.
\]
Ahora, si sumamos 2 a cada lado de estas desigualdades, obtendremos lo siguiente:
\begin{equation}
\label{eqLim005}%
2 < x + 2 \leq 5.
\end{equation}
Como $x + 2$ es mayor que cero, entonces,
\[
x + 2 = |x + 2|,
\]
de donde la desigualdad de la derecha en~(\ref{eqLim005}) se escribe así:
\begin{equation}
\label{eqLim006}%
|x + 2| \leq 5.
\end{equation}
Por lo tanto:
\[
g(x) \leq 10.
\]

Entonces, de la igualdad~(\ref{eqLim014}), concluimos que:
\begin{equation}
\label{eqLim007}%
|2x^2 - 8| \leq 10|x - 2|.
\end{equation}
para todo $x\in\ (0,3]$.

Ahora, si para el número $\delta$ que buscamos se cumple (\ref{eqLim002}):
\[
|x - 2| < \delta,
\]
entonces:
\begin{equation*}
|2x^2 - 8| < 10\delta,
\end{equation*}
para $x\in\ (0,3]$.

Por lo tanto, para que se verifique la desigualdad~(\ref{eqLim003}), podemos elegir $\delta$ de
modo que:
\[
10\delta = 0.08.
\]
Y de aquí, obtendríamos que
\[
\delta = \frac{0.08}{10} = 0.008.
\]

Si este es el valor del número $\delta$ que estamos buscando, estaremos seguros de lo siguiente: si
$x\in ]0, 3]$ tal que se verifica la desigualdad~(\ref{eqLim002}):
\[
|x - 2| < \delta,
\]
se cumplirá también la desigualdad~(\ref{eqLim003}):
\[
|2x^2 - 8| < 10\delta = 10\times 0.008 = 0.08.
\]
Y esto es precisamente lo que buscábamos.

Antes de seguir, observemos que hay más de un valor para $\delta$ que satisface la condición del
problema. En efecto, si en lugar de elegir $\delta = 0.008$, lo elegimos de tal manera que
\[
10\delta < 0.008,
\]
es decir, tal que
\[
\delta < \frac{0.08}{10} = 0.008,
\]
también se cumple la condición del problema: si $x \in\ ]0, 3]$ tal que se verifica la
desigualdad~(\ref{eqLim002}):
\[
|x - 2| < \delta,
\]
se cumplirá también la desigualdad~(\ref{eqLim003}):
\[
|2x^2 - 8| < 10\delta < 10\times 0.008 = 0.08,
\]
de donde
\[
|2x^2 - 8| < 0.08.
\]
Es decir, todo los números positivos menores que $0.008$ aseguran el cumplimiento de la
desigualdad~(\ref{eqLim003}).

El problema matemático ha sido resuelto. La solución puede resumirse así:%
\marcojc{.9}{1.5}{black}{black}{white}{%
si se elige el número $\delta > 0$ tal que
\[
\delta \leq 0.008,
\]
entonces, si $x\in\ ]0,3]$ y $|x - 2| < \delta$, necesariamente se cumplirá la siguiente
desigualdad:
\[
|2x^2 - 8| < 0.08.
\]
\eijc{-1.5} }

Observemos también que hemos hecho algo más que resolver este problema particular. Si en lugar de
$0.08$, tuviéramos cualquier número positivo $\epsilon$, lo que hemos demostrado es lo siguiente:
\begin{quote}
{\bfseries dado $\bm{\epsilon > 0}$, el número
\[
\bm{\delta = \frac{\epsilon}{10}}
\]
es tal que, si $\bm{x\in\ ]0,3]}$ y $\bm{|x - 2| < \delta}$, se verifica que
\[
\bm{|2x^2 - 8| < \epsilon}.
\]
}

En otras palabras, el número $8$ es el límite de $2x^2$ cuando $x$ se aproxima a $2$.
\end{quote}

\subsection{Solución del problema}
A partir de la solución matemática, es decir, de la solución del problema matemático, vamos a
proponer una solución al problema de los constructores de los paneles solares. Para ello, debemos
\textit{interpretar} los elementos matemáticos del problema, para lo cual debemos recordar el
significado que estos elementos tienen.

La letra $x$ representa el valor de la longitud del ancho de cualquier panel medida en metros. La
desigualdad
\[
|x - 2| < \delta
\]
expresa la cota superior admisible para el error absoluto en la longitud del ancho del panel.
Finalmente, la desigualdad
\[
|2x^2 - 8| < 0.08
\]
indica la cota superior admisible para el error absoluto en el área del panel solar cuyo ancho mide
$x$ metros y cuyo largo es $2x$ metros.

De la solución matemática, podemos asegurar que, si se elige $\delta < 0.008$, es decir, si se
asegura que el error cometido en el corte del ancho es menor que 8 milímetros, el área obtenida
para ese panel no difiere de ocho metros cuadrados en más de 0.08 metros cuadrados. En resumen:
\marcojc{.9}{1.5}{black}{black}{white}{%
\textsl{si se calibra la máquina cortadora de paneles de modo que se asegure que el error absoluto
cometido en el corte del largo de cada panel no supere los 8 milímetros, entonces se asegurará que
el área obtenida para cada panel no diferirá de ocho metros cuadrados en más del uno por ciento}.}

\subsection{Epílogo}
Luego de resolver el problema, el equipo encargado de la fabricación de los paneles procedió a la
calibración de la máquina cortadora. En las especificaciones del equipo, se indicaba que podía ser
calibrado para que el error absoluto en el corte de una cierta longitud no supere los 5 milímetros.
Dado que 5 es menor que 8, los constructores de los paneles solares procedieron a realizar dicha
calibración; estaban seguros que con ella el área de cada panel satisfaría las especificaciones
técnicas para asegurar el correcto funcionamiento de la planta de energía solar.

La comunidad entera trabajó para completar la construcción de la planta. Luego de mucho esfuerzo,
Intipamba ya tiene energía eléctrica para satisfacer las necesidades de su gente. El esfuerzo
hubiera sido mayor, si la gente de Intipamba no hubiera resuelto el problema, con lo que su
adelanto se habría detenido, y si no se hubieran sentado antes a pensar en el problema y buscar una
solución que, con la ayuda de la poderosa matemática, encontraron.

\subsection{Ejercicios}
\begingroup
\small
\begin{multicols}{2}
\begin{enumerate}[leftmargin=*]
\item PRODAUTO es una empresa productora de automóviles. La gerencia ha establecido que el
    costo de producción mensual, expresado en miles de dólares, de n vehículos puede ser
    obtenido aproximadamente a través de la siguiente fórmula:
    \[
    C = 2 000 + 11n - 0.000\,12n^2.
    \]
    Actualmente, el nivel de producción mensual es de 5 000 vehículos. El equipo de la gerencia
    ha previsto que, dadas las condiciones variables del mercado, habrán variaciones del nivel
    de producción. Esto significa que los correspondientes costos de producción variarán.

    Luego de un análisis de los flujos de caja y teniendo en cuenta la predisposición de los
inversionistas, se ha establecido que las variaciones del monto que se puede destinar a
financiar los costos de producción deberán ser menores a 4 millones de dólares por mes. Como la
planificación del trabajo, compra de insumos, planes de ventas, entre otras actividades de
PRODAUTO dependen del nivel de producción, el equipo de la gerencia debe establecer un margen
para el nivel de producción actual, sabiendo que este nivel no puede variar en más de 500 autos
mensualmente. Determinar el margen para el nivel de producción actual.

\item Los constructores de una central hidroeléctrica de reserva han establecido que la
    relación existente entre la altura $h$ metros del espejo de agua del reservorio respecto al
    nivel mínimo de operación y la energía acumulada $E$ Mwh que se puede generar si se activan
    una o más turbinas, según las necesidades planteadas por la demanda, viene dada por la
    siguiente fórmula:
    \[
    E = 75h + 10h^2 - 0.2h^3.
    \]
    A pesar de que el nivel máximo es de 35 m, han recomendado mantener el nivel en valores
    cercanos a 25 metros.

    En la época de estiaje es recomendable tener disponible la energía acumulada de este
    embalse, pero, al ser necesaria la operación de la central hidroeléctrica, se ha decidido
    permitir que las variaciones de la reserva de energía diarias sean menores a 500 MWh. Una
    manera simple de controlar el cumplimiento de esta restricción es evitar variaciones muy
    grandes del nivel de agua, de modo que se garantice que se respetarán las restricciones a
    las variaciones de la reserva de energía. Determinar la variación de la altura del agua
    permitida para asegurar que la variación de la energía disponible esté siempre por debajo
    de los 500 MWh.
\end{enumerate}
\end{multicols}
\endgroup

\section{Propiedades de los límites}
La definición de límite nos permite verificar si un número es el límite o no de una función. Sin
embargo, no nos permite encontrar el límite de una función. Las propiedades de los límites que
vamos a estudiar en esta sección son, en realidad, un conjunto de reglas de cálculo de límites a
partir del conocimiento de los límites de algunas funciones.

Por ejemplo, demostramos que el límite de una constante es la misma constante y que $a$ es el
límite de $x$ cuando $x$ se aproxima al número $a$. Probaremos más adelante que, cuando $x$ se
aproxima al número $a$, si $L$ es el límite de $f(x)$ y $M$ es el límite de $g(x)$, entonces $LM$
es el límite de $f(x)g(x)$. De estos tres resultados, podremos concluir que---sin más trámite que
su aplicación inmediata---, para una constante cualquiera $k$, el número $ka^2$ es el límite de
$kx^2$ cuando $x$ se aproxima al número $a$.

Vamos a ver que un conjunto relativamente pequeño de reglas de cálculo nos permitirán obtener los
límites de una gran variedad de funciones.

\begin{teocal}[Unicidad del límite]\label{teol:Unicidad} Si existe $\limjc{f(x)}{x}{a}$, entonces hay un único número $L$ tal que
\[
L = \limjc{f(x)}{x}{a}.
\]
\end{teocal}

Los dos teoremas siguientes dan las herramientas necesarias para poder calcular el límite de
cualquier función racional
\[
f(x) = \frac{P(x)}{Q(x)},
\]
donde $P$ y $Q$ son polinomios, cuando $x$ se aproxima al número $a$ y este número no es un cero de
$Q$.

El primer teorema ya lo demostramos. La prueba del anterior y de la mayoría de los siguientes, se
presentarán en el siguiente capítulo.

\begin{teocal}[Límites básicos]\label{teol:ConstanteIdentidad} Sean $a\in\mathbb{R}$ y
$k\in\mathbb{R}$. Supongamos que $k$ no depende de $x$. Entonces:
\begin{enumerate}
\item \emph{Límite de una constante}: $\displaystyle{\limjc{k}{x}{a} = k}$.

   Es decir, la función constante es continua.

\item \emph{Límite de la función identidad}: $\displaystyle{\limjc{x}{x}{a} = a}$.

   Es decir, la función identidad es continua.
\end{enumerate}\end{teocal}

El siguiente teorema muestra que el límite de una función ``preserva'' las operaciones de suma,
producto e inverso multiplicativo de funciones, pues el límite de la suma, del producto y del
inverso multiplicativo siempre es igual a la suma, producto e inverso multiplicativo de los
límites, siempre y cuando estos límites existan y, en el caso del inverso multiplicativo, el límite
sea diferente de cero.

\begin{teocal}[Propiedades algebraicas]\label{teol:Algebra} Si existen $L = \displaystyle
\limjc{f(x)}{x}{a}$ y $M = \displaystyle \limjc{g(x)}{x}{a}$, entonces:
\begin{enumerate}
\item \emph{Límite de la suma}: existe el límite de $f(x) + g(x)$ y:
\[
    L + M = \limjc{(f(x)+ g(x))}{x}{a}.
\]
\item \emph{Límite del producto}: existe el límite de $f(x)g(x)$ y:
\[
LM = \limjc{(f(x)g(x))}{x}{a}.
\]
\item \emph{Límite del inverso multiplicativo}: si $M\neq 0$, existe el límite de
    $\displaystyle{\frac{1}{g(x)}}$ y:
\[
\frac{1}{M} = \limjc{\frac{1}{g(x)}}{x}{a}.
\]
\end{enumerate}
\end{teocal}

Por ejemplo, si $a\neq 0$, por la tercera parte de este teorema, podemos afirmar que
\[
\frac{1}{a} = \limjc{\frac{1}{x}}{x}{a},
\]
pues, de la segunda parte del teorema del límite de la función identidad
(teorema~\ref{teol:ConstanteIdentidad}), podemos asegurar que
\[
a = \limjc{x}{x}{a}.
\]

Más adelante, calcularemos una gran variedad de límites con el uso de estos teoremas. Ahora hagamos
la demostración del primer numeral.

\begin{proof}[Demostración]
Sea $\epsilon > 0$. Debemos probar que existe un número
    $\delta > 0$ tal que
\begin{equation}
\label{eq:pl003}
|(f(x) + g(x)) - (L + M)| < \epsilon,
\end{equation}
siempre que
\[
0 < |x - a| < \delta\yjc x\in\Dm(f+g) = \Dm(f)\cap\Dm(g).
\]

Para todo $x\in\Dm(f+g)$, tenemos que:
\begin{align*}
|(f(x) + g(x)) - (L + M)| &= |(f(x) - L) + (g(x) - M)| \\
&\leq |f(x) - L| + |g(x) - M|.
\end{align*}
Por lo tanto, para hacer tan pequeño como queramos a
\[
|(f(x) + g(x)) - (L + M)|
\]
es suficiente que hagamos tan pequeño como queramos a
\[
|f(x) - L| \yjc |g(x) - M|.
\]
para el $x$ adecuado.

Ahora bien, estas dos diferencias sí pueden hacerse tan pequeñas como se quiera, pues $L$ es el
límite de $f(x)$ y $M$ el de $g(x)$ cuando $x$ se aproxima al número $a$. De manera más
precisa, para obtener la desigualdad~(\ref{eq:pl003}), es suficiente encontrar el $x$ adecuado
para que
\[
|f(x) - L| \yjc |g(x) - M|
\]
sean, cada uno, menores que $\frac{\epsilon}{2}$. Procedamos.

Como $L$ es el límite de $f(x)$ y $\frac{\epsilon}{2} > 0$, existe $\delta_1 > 0$ tal que
\begin{equation}
\label{eq:pl004}
|f(x) - L| < \frac{\epsilon}{2},
\end{equation}
siempre que
\[
0 < |x - a| < \delta_1 \yjc x\in\Dm(f).
\]
Además, por ser $M$ el límite de $g(x)$ y $\frac{\epsilon}{2} > 0$, existe $\delta_2 > 0$ tal
que
\begin{equation}
\label{eq:pl005}
|g(x) - M| < \frac{\epsilon}{2},
\end{equation}
siempre que
\[
0 < |x - a| < \delta_2 \yjc x\in\Dm(g).
\]

?`Qué $x$ satisfará simultáneamente las desigualdades~(\ref{eq:pl004}) y (\ref{eq:pl005})? Aquel
que satisfaga simultáneamente las siguientes condiciones:
\[
0 < |x - a| < \delta_1, \quad 0 < |x - a| < \delta_2, \quad x\in\Dm(f)\yjc x\in \Dm(g).
\]
?`Cómo podemos elegir, entonces, el número $\delta$? Así:
\[
\delta = \min\{\delta_1,\delta_2\},
\]
?`Y cómo elegir $x$? Así:
\[
0 < |x - a| < \delta\yj x\in\Dm(f)\cap \Dm(g).
\]
Tenemos entonces que:
\begin{align*}
|(f(x) + g(x)) - (L + M)| &\leq |f(x) - L| + |g(x) - M| \\
< \frac{\epsilon}{2} + \frac{\epsilon}{2} = \epsilon.
\end{align*}
En otras palabras, $L + M$ es el límite de $f(x) + g(x)$ cuando $x$ se aproxima al número $a$.

\end{proof}

De este teorema, es consecuencia inmediata el siguiente corolario.

\begin{corocal}[Propiedades algebraicas]\label{cor:limPropiedadesAlgebraicas}%
Sean $\alpha\in\mathbb{R}$, $\beta\in\mathbb{R}$. Si
existen
\[
L = \limjc{f(x)}{x}{a} \yjc M = \limjc{g(x)}{x}{a},
\]
entonces:
\begin{enumerate}
\item \emph{Límite del producto de un escalar por una función}: existe el límite de $\alpha
    f(x)$ y:
    \[
        \alpha L = \limjc{(\alpha f(x))}{x}{a}.
    \]
\item \emph{Límite de la resta}: existe el límite de $f(x) - g(x)$ y:
    \[
        L - M = \limjc{(f(x) - g(x))}{x}{a}.
    \]
\item \emph{Límite de una combinación lineal}: existe el límite de $\alpha f(x) + \beta g(x)$
    y:
    \[
        \alpha L + \beta M = \limjc{(\alpha f(x) + \beta g(x))}{x}{a}.
    \]
\item \emph{Límite del cociente}: existe el límite de $\frac{f(x)}{g(x)}$, siempre que $M\neq
    0$. Entonces:
    \[
        \frac{L}{M} = \limjc{\frac{f(x)}{g(x)}}{x}{a}.
    \]

\end{enumerate}
\end{corocal}

De este corolario, y su correspondiente teorema, se deduce que las operaciones algebraicas entre
funciones continuas producen funciones continuas. De manera precisa:

\begin{teocal}[Propiedades algebraicas de la continuidad]%
Si $\funcjc{f}{\Dm(f)}{\mathbb{R}}$ y $\funcjc{g}{\Dm(g)}{\mathbb{R}}$ son continuas en $a$.
Entonces
\begin{quote}
la \emph{suma} $(f + g)$, la \emph{resta} $(f - g)$, el \emph{producto} $(fg)$ y el \emph{cociente}
$(f/g)$ (siempre que $g(a) \neq 0$)
\end{quote}
son funciones continuas en a.
\end{teocal}

\begin{exemplo}[Solución]{%
Calcular
\[
\limjc{\frac{x + 3}{x - 5}}{x}{2}.
\]
}%
Sea $\funcjc{F}{\mathbb{R}}{\mathbb{R}}$ definida por
\[
F(x) = \frac{x + 3}{x - 5}.
\]
Entonces, el límite pedido es:
\[
\limjc{F(x)}{x}{2}.
\]

La función $F$ puede ser expresada como el cociente de las funciones
$\funcjc{\varphi}{\mathbb{R}}{\mathbb{R}}$ y $\funcjc{\psi}{\mathbb{R}}{\mathbb{R}}$, definidas por
\[
\varphi(x) = x + 3 \yjc \psi(x) = x - 5.
\]
Es decir,
\[
F(x) = \frac{\varphi(x)}{\psi(x)}.
\]
Entonces, para calcular el límite de $F(x)$, podemos utilizar el teorema del ``límite del
cociente'', es decir, el corolario~\ref{cor:limPropiedadesAlgebraicas} del teorema
\ref{teol:Algebra}. Para poder hacerlo, tenemos que constatar que se verifiquen las condiciones de
este teorema. Estas son tres:
\begin{enumerate}
\item Existe el límite
$\displaystyle
\limjc{\varphi(x)}{x}{2}$.

\item Existe el límite
$\displaystyle
\limjc{\psi(x)}{x}{2}$.

\item Si el límite anterior existe, éste debe ser diferente de cero.
\end{enumerate}
Verifiquemos que se cumplen estas tres condiciones.
\begin{enumerate}
\item La función $\varphi$ es la suma de la función identidad y una función constante. Entonces
    su límite existe y es igual a:
   \[
    \limjc{\varphi(x)}{x}{2} = \limjc{x}{x}{2} + \limjc{3}{x}{2} = 2 + 3 = 5.
   \]
\item La función $\psi$ es la resta de la función identidad y una función constante. Entonces
    su límite existe y es igual a:
   \[
    \limjc{\psi(x)}{x}{2} = \limjc{x}{x}{2} - \limjc{5}{x}{2} = 2 - 5 = -3.
   \]
   Como se puede observar, el límite de la función $\psi$ es diferente de $0$.
\end{enumerate}

Podemos, entonces, aplicar el teorema del ``límite del cociente''. Éste afirma que el límite del
cociente de dos funciones cuyo límites existen y el del denominador es diferente de $0$, es igual
al cociente de ambos límites. Por lo tanto, podemos proceder de la siguiente manera:
\begin{align*}
\limjc{\frac{\varphi(x)}{\psi(x)}}{x}{2} &= \frac{\displaystyle\limjc{\varphi(x)}{x}{2}}{\displaystyle\limjc{\psi(x)}{x}{2}} \\
   &= \frac{5}{-3}.
\end{align*}
Es decir:
$\displaystyle
\limjc{\frac{x + 3}{x - 5}}{x}{2} = -\frac{5}{3}$.
\end{exemplo}

\begin{exemplo}[Solución]{%
Calcule $\displaystyle\limjc{\frac{s^2 - 21}{s + 2}}{s}{7} $ si existe}%
Para aplicar el teorema del límite del cociente necesitamos saber si existen los límites de las
funciones numerador y denominador, y si el límite del denominador es distinto de $0$. Empecemos con
el numerador:
\begin{align*}
\limjc{(s^2 - 21)}{s}{7} &= \limjc{s^2}{s}{2} + \limjc{(-21)}{s}{2} \\
&= \limjc{s}{s}{2} \times \limjc{s}{s}{2} - 21 \\
&= 7 × 7 - 21 = 28.
\end{align*}
Ahora el límite del denominador:
\begin{align*}
\limjc{(s + 2)}{s}{7} &= \limjc{s}{s}{7} + \limjc{2}{s}{7}\\
  &= 7 + 2 = 9 \neq 0.
\end{align*}
Podemos, entonces, aplicar el teorema del límite del cociente:
\begin{align*}
\limjc{\frac{s^2 - 21}{s + 2}}{s}{7} &= \frac{\displaystyle\limjc{(s^2 - 21)}{s}{7}}{\displaystyle\limjc{(s + 2)}{s}{7}} = \frac{28}{9}.
\end{align*}
\end{exemplo}

\begin{exemplo}[Solución]{%
Calcule $\displaystyle\limjc{(2t + 3)(3t^2 - 5)}{t}{1}$ si existe.}%
\begin{align*}
\limjc{(2t + 3)(3t^2 - 5)}{t}{1} &= \left(\limjc{(2t + 3)}{t}{1}\right)\left(\limjc{(3t^2 -
5)}{t}{1}\right) \\
  &= \left(\limjc{2t}{t}{1} + \limjc{3}{t}{1}\right)\left(\limjc{3t^2}{t}{1} + \limjc{(-5)}{t}{1}\right)\\
  &= \left(2\limjc{t}{t}{1} + \limjc{3}{t}{1}\right)\left(3\limjc{t^2}{t}{1} + \limjc{(-5)}{t}{1}\right)\\
  &= \left(2\limjc{t}{t}{1} +
  \limjc{3}{t}{1}\right)\left(3\left[\limjc{t}{t}{1}\right]\left[\limjc{t}{t}{1}\right] +
  \limjc{(-5)}{t}{1}\right) \\
  &= (2\cdot 1 + 3)(3[1\cdot 1] + (-5)) \\
  &= (2 + 3)(3 - 5) = -10.
\end{align*}
Entonces
\[
\limjc{(2t + 3)(3t^2 - 5)}{t}{1} = -10.
\]
\end{exemplo}

\begin{exemplo}[Solución]{%
Calcule $\displaystyle\limjc{y\left(\frac{4}{y} - 1\right)}{y}{0}$ si existe.}%
Si intentamos calcular el límite como producto de límites no podríamos hacerlo, ya que no tenemos
aún a disposición ninguna herramienta que nos diga como obtener el límite del segundo factor. En
situaciones como ésta, es necesario efectuar una manipulación algebraica previa al cálculo del
límite:
\begin{align*}
\limjc{y\left(\frac{4}{y} - 1\right)}{y}{0}
  &=\limjc{\left(\frac{4}{y}y - y\right)}{y}{0}\\
  &= \limjc{(4 - y)}{y}{0} = 4 - 0 = 4.
\end{align*}
De donde:
\[
\limjc{y\left(\frac{4}{y} - 1\right)}{y}{0} = 4.
\]
\end{exemplo}

\subsection{Ejercicios}
\begingroup
\small
\begin{multicols}{2}
\begin{enumerate}[leftmargin=*]
\item En los siguientes ejercicios, calcule el límite dado si él existe. De ser necesario,
    realice primero una manipulación algebraica.
\begin{enumerate}[leftmargin=*]
\item $\displaystyle\limjc{(2x^2 - 3x + 2)}{x}{-2}$.
\item $\displaystyle\limjc{\frac{-x^2 + x - 1}{-3x + 1}}{x}{2}$.
\item $\displaystyle\limjc{\frac{-x + 3}{2x - 7}}{x}{3}$.
\item $\displaystyle\limjc{\frac{t^3 - 1}{t - 1}}{t}{1}$.
\item $\displaystyle\limjc{\frac{3z + 9}{36 - 4z^2}}{z}{-3}$.
\item $\displaystyle\limjc{\left(x - \frac{1}{x - 1}\right)}{x}{0}$.
\item $\displaystyle\limjc{\frac{t^3+1}{t^2 - 1}}{t}{-1}$.
\item $\displaystyle\limjc{2x^3(-x^4 + 3x^3)^{-1}}{x}{0}$.
\item $\displaystyle\limjc{\left(\frac{1}{x} + \frac{2x^2 - 5x - 1}{x}\right)}{x}{0}$.
\item $\displaystyle\limjc{(ar^2 - br)^3}{r}{1}$ con $a$ y $b$ constantes. No desarrolle el
    cubo.
\item $\displaystyle\limjc{\frac{1}{h}\left((x + h)^3 - x^3\right)}{h}{0}$.
\end{enumerate}

\item Calcule los límites dados usando las propiedades de los límites. Indique qué propiedades
    usó.
\begin{enumerate}[leftmargin=*]
\item $\displaystyle
	\lim_{x\to 3}7.9.
$
\item $\displaystyle
	\lim_{x\to 5}(4x-8).
$
\item $\displaystyle
	\lim_{x\to 2}(x^5-7x+1).
$
\item $\displaystyle
	\lim_{x\to 0}\dfrac{3x-7}{7x^2+8}.
$
\end{enumerate}

\end{enumerate}
\end{multicols}
\endgroup

\subsection{Generalizaciones}
Los teoremas y el corolario sobre las propiedades algebraicas de los límites y de la continuidad
están formulados para las propiedades de dos funciones: la suma de dos funciones, el producto,
etcétera. En el siguiente teorema se generalizan los anteriores para cualquier número de funciones.

\begin{corocal}[Generalización de las propiedades algebraicas]\label{teol:AlgebraGeneralizada}%
Sean $n\in\mathbb{N}$, $f_1, f_2, \ldots, f_n$ tales que
\[
L_i = \limjc{f_i(x)}{x}{a} \yjc M_i = \limjc{g_i(x)}{x}{a}
\]
para todo $i$ tal que $1\leq i \leq n$. Entonces:
\begin{enumerate}
\item \label{teol:SumaGeneralizada} $\displaystyle{\limjc{\sum_{i=1}^nf_i(x)}{x}{a} =
    \sum_{i=1}^n \limjc{f_i(x)}{x}{a} = \sum_{i=1}^n L_i}.$
\item \label{teol:ProductoGeneralizado} $\displaystyle{\limjc{\prod_{i=1}^nf_i(x)}{x}{a} =
    \prod_{i=1}^n \limjc{f_i(x)}{x}{a} = \prod_{i=1}^n L_i}.$
\item Si $f_i$ es continua en $a$ para todo $i$ tal que $1\leq i \leq n$, entonces
    $\sum_{i=1}^nf_i$ y $\prod_{i=1}^nf_i$ son continuas en $a$.
\end{enumerate}
\end{corocal}

La demostración es sencilla si se aplica el método de inducción matemática sobre $n$.

Ahora es fácil probar que los polinomios y las funciones racionales son continuas:

\begin{teocal}[Continuidad de un polinomio y una función
racional]\label{teol:PolinRacionContinuas}%
Todo polinomio es una función continua en $\mathbb{R}$ y toda función racional es continua en
$\mathbb{R}$, excepto en aquellos números en los que el denominador es igual a cero. De manera más
precisa: si $P$ y $Q$ son dos polinomios, entonces:
\begin{enumerate}
\item \label{teol:PolinContinua} $\displaystyle{\limjc{P(x)}{x}{a} = P(a).}$
\item \label{teol:RacionContinua} $\displaystyle{\limjc{\frac{P(x)}{Q(x)}}{x}{a} =
    \frac{P(a)}{Q(a)}}$, siempre que $Q(a) \neq 0$.
\end{enumerate}
\end{teocal}

Del límite del producto generalizado, se obtiene que si $f(x)$ tiene $L$ como límite cuando $x$ se
aproxima al número $a$, entonces $L^n$ es el límite de $f^n(x)$. Esto también es verdadero para el
caso de la raíz $n$-ésima, aunque la demostración de esta propiedad, que no se deriva del teorema
anterior, no es elemental.

\begin{teocal}[Límite de la raíz $n$-ésima]\label{teol:RaizGeneralizada}%
Si $L = \displaystyle{\limjca{f(x)}}$, entonces para todo $n\in\mathbb{N}$ impar, se verifica que
\[
\limjca{\sqrt[n]{f(x)}} = \sqrt[n]{\limjca{f(x)}} = \sqrt[n]{L}.
\]
Si $n$ es par, es necesario que $L > 0$.
\end{teocal}

Para la demostración de este teorema es necesario antes demostrar que la composición de funciones
también preserva el límite y la continuidad, pues, probaremos que la función $h_n$, definida por
\[
h_n(x) = \sqrt[n]{x}
\]
es continua en $\mathbb{R}$ si $n$ es impar, mientras que es continua en $[0,+\infty[$ si $n$ es
par. Eso lo haremos en el siguiente capítulo. Ahora veamos algunos ejemplos de límites que
involucran raíces.

\begin{exemplo}[Solución]{%
Calcule $\displaystyle \lim_{x\to 4}\sqrt{2x^2 -7}$ si existe.
}%
El límite pedido existirá si existe el $\displaystyle \lim_{x\to 4}(2x^2 -7)$ y si éste es
positivo. Entonces, calculemos primero dicho límite:
\begin{align*}
\lim_{x\to 4}(2x^2 -7) &= 2\lim_{x\to 4}x^2 + \lim_{x\to 4}(-7) \\
&= 2(4)(4)-7 = 25>0.
\end{align*}
Entonces, el límite buscado existe y se lo calcula así:
\begin{align*}
\lim_{x\to 4}\sqrt{2x^2 -7} &= \sqrt{\lim_{x\to 4}(2x^2 -7)} \\
&= \sqrt{25}=5.
\end{align*}
Por lo tanto:
$\displaystyle
\lim_{x\to 4}\sqrt{2x^2 -7}=5$.
\end{exemplo}

\begin{exemplo}[Solución]{%
Calcule $\displaystyle \lim_{x\to 3}\sqrt[3]{x -7}$ si existe.
}%
A diferencia del ejercicio anterior, y por ser raíz impar, basta que exista el $\displaystyle
\lim_{x\to 3}(x -7)$. Como
\[
\lim_{x\to 3}(x-7)=-4,
\]
entonces
\begin{align*}
\lim_{x\to 3}\sqrt[3]{x -7} &= \sqrt[3]{\lim_{x\to 3}(x-7)} \\
&= \sqrt[3]{-4} = -\sqrt[3]{4}.
\end{align*}
Por lo tanto:
$\displaystyle
\lim_{x\to 3}\sqrt[3]{x -7}=-\sqrt[3]{4}$.

\end{exemplo}

\subsection{Ejercicios}
\begingroup
\small
Si los límites existen, calcularlos:
\begin{multicols}{2}
\begin{enumerate}[leftmargin=*]
\item $\displaystyle \lim_{x\to 10}\sqrt{\frac{10x}{2x+5}}$.
%
\item $\displaystyle \lim_{x\to 2}\frac{\sqrt[3]{x^2-10}}{\sqrt{x^3-3}}$.
%
\item $\displaystyle \lim_{t\to 2}(t+2)^\frac{3}{2}(2t+4)^\frac{1}{3}$.
%
\item $\displaystyle \lim_{h\to 0}\frac{\sqrt{x+h}-\sqrt{x}}{h}, x>0$.
%
\item $\displaystyle \lim_{t\to 1}\frac{\sqrt{t}-1}{t-1}$.
%
\item $\displaystyle
	\lim_{x\to 1}\sqrt{\dfrac{2x^2-x+3}{x^2+x+1}}.
$
%
\item $\displaystyle
	\lim_{x\to 2}\sqrt[3]{\dfrac{3x-5}{2x^2-x+1}}.
$
%
\item $\displaystyle \limjc{\frac{\sqrt[4]{x^3 - 3x + 2}}{\sqrt[3]{1 - 2x + x^4}}}{x}{1}$.
\end{enumerate}
\end{multicols}
\endgroup

\section{Continuidad de funciones localmente iguales}

Una consecuencia inmediata del teorema del límite de funciones localmente iguales es el siguiente:

\begin{teocal}[Continuidad de funciones localmente iguales]
Sean:
\begin{itemize}
\item[] $a$ un número real;
\item[] $I$ un intervalo abierto que contiene al número $a$;
\item[] $f$ y $g$ dos funciones definidas en $I$ (es decir, $I\subset \Dm(f)$; e $I\subset \Dm(g)$).
\end{itemize}
Si $f(a)=g(a)$ y si $f=g$ cerca de $a$, entonces $f$ es continua en $a$ si y solo si $g$ es continua en $a$.
\end{teocal}%

Este teorema tiene como corolario el muy útil teorema siguiente:

\begin{teocal}[Continuidad de funciones iguales en un intervalo abierto]
Sean $I$ un intervalo abierto; y, $f$ y $g$ dos funciones reales que son iguales en $I$ (es decir que para todo $x\in I$, $f(x)=g(x)$, que es lo mismo, que $f|_{I}=g|_{I}$).

Entonces $f$ es continua en $I$ si y solo si $g$ es continua en $I$.
\end{teocal}%

\begin{exemplo}[Solución]{%
Sea
\[
f(x) = \begin{cases}
g(x) = -2x^2+8x-4 & \text{, si $x\geq 1$,}\\
h(x) = x^2 + 1 & \text{, si $x<1$.}
\end{cases}
\]
Pruebe que $f$ es continua
}%
\begin{enumerate}
\item[a)] Vemos que $f|_{]-\infty, 1[}=g|_{]-\infty, 1[}$ y como $g$ es continua por ser un polinomio, aplicando el teorema precedente tenemos que $f$ es continua en $]-\infty, 1[$.
\item[b)] Análogamente, como $f|_{]1, +\infty[}=h|_{]1, +\infty[}$ y como $h$ es continua por ser un polinomio, podemos concluir que $f$ es continua en $]1, +\infty[$.
\item[c)] Queda por verificar la continuidad en $1$. Como $f(1)=g(1)=2$ y, teniendo en cuenta lo calculado en el Ejemplo 1.4, se obtiene que
\[
\limjc{f(x)}{x}{1}=2
\]
En consecuencia, $f$ es continua en $1$.
\end{enumerate}
\end{exemplo}


\section{El límite de una composición: cambio de variable}
Más adelante, en este capítulo, probaremos que la función $\sen$ es continua en $\mathbb{R}$. Por
lo tanto, se cumplirá que:
\[
\limjc{\sen x}{x}{0} = \sen 0 = 0.
\]
Por otra parte, sabemos que
\[
\limjc{(x^2 - 1)}{x}{1} = 0.
\]
?`Cómo podemos utilizar estos dos resultados para calcular
\[
\limjc{\sen{(x^2 - 1)}}{x}{1}
\]
si tener que recurrir a la definición de límite?

Como puede verse, este último es el límite de la composición de las dos funciones presentes en los
dos primeros límites. El siguiente teorema, nos dice cómo se puede calcular el límite de una
composición a partir de los límites de las funciones que conforman la composición.

\begin{teocal}[Límite de una composición]
Supongamos que existe el límite $\displaystyle\lim_{x\to a}g(x)= b$ y que $f$ sea continua en $b$. Entonces
existe $\displaystyle\lim_{x\to a}f(g(x))$ y
\[
\lim_{x\to a}f(g(x)) = f\left(\limjc{g(x)}{x}{a}\right) = f(b).
\]
\end{teocal}%

\begin{corocal}[]
La composición de funciones continuas es continua.
\end{corocal}


Ahora, las hipótesis de este teorema no siempre se verifican. Por ejemplo, si queremos calcular
el siguiente límite
\[
\limjc{\frac{\sen(2x)}{x}}{x}{0},
\]
deberemos recurrir al límite
\[
\limjc{\frac{\sen y}{y}}{y}{0} = 1.
\]
Como se puede ver, la función $f$, definida por
\[
f(y) = \frac{\sen y}{y}
\]
para todo $y\neq 0$, no es continua en $0$, pues no está definida en este número. Sin embargo, aún
se puede calcular el límite de una composición si se modifican la hipótesis del teorema anterior.

De manera más precisa, probaremos que la hipótesis de que $f$ sea continua en $b$ puede
\emph{debilitarse}; es decir, puede ser sustituida por una condición que no exige la continuidad de
$f$ en $b$, sino solamente la propiedad de que exista un intervalo alrededor del número $a$ en el
cual, a excepción tal vez de $a$, la función $g$ es diferente de $b$. En los ejemplos que veremos a
continuación, así como en los ejercicios propuestos, es necesario verificar que la función $g$
satisface esta condición en los casos de que la función $f$ no sea continua en $b$.

En la práctica, este teorema se utiliza bajo la siguiente formulación:

\begin{teocal}[Cambio de variable para límites]\label{teo:LimCambioVariable}%
Para calcular el límite de una composición como el siguiente
\[
\limjc{f(g(x))}{x}{a},
\]
se puede usar el \emph{cambio de variable}
$
y = g(x)
$
cuando existan $\displaystyle b = \limjc{g(x)}{x}{a}$ y $\displaystyle L = \limjc{f(y)}{y}{b}$, y siempre que se satisfaga una de las tres condiciones siguientes:
\begin{enumerate}
\item $f$ es continua en $b$;
\item $f$ no está definida en $b$; y,
\item $L \neq f(b)$ y existe $r > 0$ tal que para todo $x \in\ ]a-r,a+r[-\{a\}$, $g(x) \neq b$ (es decir, $g\new b$ cerca de $a$).
\end{enumerate}
Se tiene, entonces, que:
\[
\limjc{f(g(x))}{x}{a} = \limjc{f(y)}{y}{b} = L.
\]
\end{teocal}

El utilizar la función $g$ y la composición con $f$ para calcular el límite de $f(g(x))$ se
denomina \emph{método del cambio de variable}. Veamos cómo se utiliza este teorema en los dos
ejemplos con los que se abre esta subsección.

\begin{exemplo}[Solución]{%
Calcular
\[
\limjc{\sen(x^2 - 1)}{x}{1}.
\]
}%
Sean $f$ y $g$ definidas por $f(x) = \sen x$ y $g(x) = x^2 - 1$. Además, hagamos $a = 1$. Entonces:
\[
\limjc{\sen(x^2 - 1)}{x}{1} = \limjc{f(g(x))}{x}{1}.
\]
Veamos si las funciones $f$ y $g$ satisfacen las condiciones del
teorema~\ref{teo:LimCambioVariable}. En primer lugar, determinemos si el límite de $g(x)$ existe
cuando $x$ tiende a $1$:
\begin{align*}
\limjc{g(x)}{x}{1} &= \limjc{(x^2 - 1)}{x}{1} \\
  &= (\limjc{x}{x}{1})^2 - \limjc{1}{x}{1} \\
  &= 1^2 - 1 = 0.
\end{align*}
Entonces:
\[
0 = \limjc{g(x)}{x}{1}.
\]
Por lo tanto, $b = 0$.

Ahora debemos determinar si $f$ es continua en $0$, o si existen $\displaystyle\limjc{f(y)}{y}{0}$ y un
intervalo centrado en $1$ en el cual $g$ sea diferente de $0$ (salvo, tal vez, en $1$).

Como la función $f$ es la función $\sen$, entonces $f$ es continua en $0$ y, además:
\[
\limjc{f(y)}{y}{0} = \limjc{\sen y}{y}{0} = \sen 0 = 0.
\]

Podemos, entonces, aplicar el teorema~\ref{teo:LimCambioVariable}. Al hacerlo, obtendremos que:
\begin{align*}
\limjc{f(g(x))}{x}{1} &= \limjc{f(y)}{y}{0} \\
  &= \limjc{\sen y}{y}{0} = 0.
\end{align*}
Es decir:
\[
\limjc{\sen(x^2 - 1)}{x}{1} = 0.
\]
\end{exemplo}

Veamos qué sucede en el segundo ejemplo.

\begin{exemplo}[Solución]{%
Calcular
\[
\limjc{\frac{\sen(2x)}{x}}{x}{0}.
\]
}%
Si hacemos el cambio de variable $y = g(x) = 2x$, tenemos que
\[
\frac{\sen(2x)}{x} = \frac{\sen y}{\frac{y}{2}} = 2\frac{\sen y}{y}.
\]
Si definimos $f$ por:
\[
f(y)= 2\frac{\sen y}{y},
\]
lo que debemos calcular es:
\[
\limjc{\frac{\sen(2x)}{x}}{x}{0} = \limjc{f(g(x))}{x}{0}.
\]

Apliquemos el teorema de cambio de variable. En este caso $a = 0$. Busquemos $b$. Para ello,
debemos calcular $\displaystyle\limjc{g(x)}{x}{0}$:
\[
\limjc{g(x)}{x}{0} = \limjc{2x}{x}{0} = 2\limjc{x}{x}{0} = 2\times 0 = 0.
\]
Por lo tanto, $b = 0$.

Para poder aplicar el teorema, nos falta determinar si $f$ o bien es continua en $0$ o si existe su
límite en $b = 0$ y, en un intervalo centrado en $a = 0$, la función $g$ es diferente de $b = 0$
(salvo, tal vez, en $0$).

En primer lugar, sí existe el límite de $f$ en $0$, pues
\[
\limjc{f(y)}{y}{0} = \limjc{2\frac{\sen y}{y}}{y}{0} = 2\limjc{\frac{\sen y}{y}}{y}{0} = 2\times 1 = 2.
\]
Ahora bien, dado que $f$ no está definida en $0$, no puede ser continua en $0$. Es decir, la
primera condición no se verifica, pero si se cumple la segunda.

También $g(x) = 2x$ es diferente de $0$ en todos los puntos de un intervalo centrado en $0$, salvo
en $0$, por lo que, en este caso, también se cumple la tercera condición.

Entonces, el teorema del cambio de variable es aplicable, con lo que obtenemos que:
\begin{align*}
\limjc{\frac{\sen(2x)}{x}}{x}{0} &= \limjc{f(g(x))}{x}{0} \\
&= \limjc{f(y)}{y}{0} = 2.
\end{align*}
Es decir:
\[
\limjc{\frac{\sen(2x)}{x}}{x}{0} = 2.
\]

En este segundo ejemplo, la función $f$ del teorema~\ref{teo:LimCambioVariable} satisface las
condiciones segunda y tercera, pero no la primera. Sin embargo, el límite de este ejemplo puede ser resuelto
aplicando la primera condición si se define $f$ de tal manera que sí sea continua en $0$.

En efecto, si definimos
\[
f(y) =
\begin{cases}
2\frac{\sen y}{y} & \text{si} \ y \neq 0, \\
2 & \text{si} \ y = 0,
\end{cases}
\]
entonces la función $f$ sí es continua en $0$, pues está definida en dicho punto, y su valor allí
es $2$, y el límite de $f$ en $0$ es $2$. Hemos ``extendido o prolongado de una manera continua la función $f$ al punto $0$''.

Obviamente, al aplicar el teorema de cambio de variable a esta $f$, obtenemos el mismo resultado
que con la definición anterior.
\end{exemplo}

En la práctica, muchos cambios de variable se realizan sin explicitar la función $f$. Por ejemplo,
el caso del último ejemplo, se suele proceder así:
\begin{align*}
\limjc{\frac{\sen(2x)}{x}}{x}{0} &= \limjc{2\frac{\sen(2x)}{2x}}{x}{0} \\
&= 2\limjc{\frac{\sen(2x)}{2x}}{x}{0} \\
&= 2\limjc{\frac{\sen(y)}{y}}{y}{0} = 2\times 1 = 2,
\end{align*}
donde $y = 2x$ y $\displaystyle\limjc{y}{x}{0} = \limjc{2x}{x}{0} = 0$.

Si bien se puede proceder como en este ejemplo, el lector debe verificar que se satisfagan las
hipótesis del teorema del cambio de variable, pues, en el caso contrario, sus
conclusiones podrían no son correctas, como nos lo muestra el ejemplo siguiente.

\begin{exemplo}[Solución]{%
Sean
\begin{displaymath}
f(t)=
\begin{cases}
4 & \text{si $t\neq 1$}\\
3 & \text{si $t= 1$},
\end{cases} \yjc
g(x)=
\begin{cases}
1 & \text{si $x\neq 2$}\\
0 & \text{si $x= 2$}.
\end{cases}
\end{displaymath}
Demostrar que
\[
\limjc{f(g(x))}{x}{2} \neq \limjc{f(t)}{t}{1}.
\]
Es decir, el teorema~\ref{teo:LimCambioVariable} del cambio de variable no es aplicable para
calcular el límite de la compuesta de $f$ con $g$ en $2$.
}%
En primer lugar, tenemos que $a = 2$ y, como se verifica que
\[
\limjc{g(x)}{x}{2} = 1,
\]
entonces $b = 1$. Además, si hacemos el cambio $t = g(x)$, entonces
\[
\limjc{f(t)}{t}{1} = 4.
\]

En segundo lugar, tenemos que
\begin{align*}
f(g(x)) &=
\begin{cases}
4 & \text{si} \ g(x) \neq 1 \\
3 & \text{si} \ g(x) = 1
\end{cases}
\\
&=
\begin{cases}
4 & \text{si} \ x = 2 \\
3 & \text{si} \ x \neq 2.
\end{cases}
\end{align*}
Por lo tanto:
\[
\limjc{f(g(x))}{x}{2} = 3.
\]
Y, como $3 \neq 4$, entonces:
\[
\limjc{f(g(x))}{x}{2} \neq \limjc{f(t)}{t}{1}.
\]

Esto significa que el teorema de cambio de variable no es aplicable. ?`Por qué?

Porque aunque $f$ sí está definida en $b = 1$, no es continua en este punto $b = 1$ y $g(x)$ es igual a $b = 1$ en cualquier intervalo centrado en
$a = 2$, salvo en $2$.
\end{exemplo}

Aplicar el teorema del cambio de variable (o cualquier teorema) sin verificar que las hipótesis
requeridas se satisfacen puede conducirnos a un error. Y, aunque no lo hiciera, hacerlo es un
error, porque no sabemos si el resultado obtenido es correcto o no.

Veamos algunos ejemplos adicionales en los que sí se puede aplicar el teorema del cambio de
variable.

\begin{exemplo}[Solución]{%
Sean
\begin{displaymath}
f(x)=
\begin{cases}
3 & \text{si $x\neq 1$}\\
4 & \text{si $x= 1$},
\end{cases} \yjc
g(x)=
\begin{cases}
0 & \text{si $x\neq 2$}\\
1 & \text{si $x= 2$}.
\end{cases}
\end{displaymath}

Utilizar el teorema del cambio de variable, si es aplicable, para mostrar que
\[
\limjc{f(g(x))}{x}{2} = f(\limjc{g(x)}{x}{2})
\]
\eijc{-1}}

En primer lugar, calculemos el límite de la composición sin utilizar el teorema del cambio de
variable. Luego lo aplicamos y miramos que el resultado obtenido es el mismo.

Puesto que
\[
f(g(x))=
\begin{cases}
f(0)&\text{si $x\neq 2$}\\
f(1) & \text{si $x= 2$}
\end{cases}
\quad =\quad
\begin{cases}
3 &\text{si $x\neq 2$}\\
4 & \text{si $x= 2$},
\end{cases}
\]
entonces
\[
\lim_{x\to 2}f(g(x))=\lim_{x\to 2}3=3.
\]

Por otra parte:
\[
\lim_{x\to 2}g(x)=0.
\]
Por lo tanto:
\[
f\left(\lim_{x\to 2}g(x)\right) = f(0) = 3.
\]
Esto prueba que:
\[
\lim_{x\to 2}f(g(x))= f\left( \lim_{x\to 2}g(x)  \right).
\]

Podemos llegar a la misma conclusión al aplicar el teorema de cambio de variable, pues existe el
limite de $f(y)$ cuando $y$ se aproxima a $0$ y es igual a $3$; existe el límite de $g(x)$ cuando
$x$ se aproxima a $2$ y es igual a $0$. Como $f$ es continua en $0$, el teorema es aplicable con $a
= 2$ y $b = 0$.
\end{exemplo}

\begin{exemplo}[Solución]{%
Probar que, si $\displaystyle f(x)=\frac{2\tan (x^2-1)}{x^4-x^2}$, entonces
\[
\lim_{x\to 1}f(x)=2.
\]
}%
Sean
\[
k(x) = \frac{2}{x^2\cos (x^2-1)} \yjc h(x)= \frac{\sen(x^2-1)}{x^2-1}.
\]
Como $f(x) = k(x)h(x)$ tenemos que:
\[
\lim_{x\to 1}f(x)=\left( \lim_{x\to 1}k(x) \right)  \left(\lim_{x\to 1}h(x)  \right),
\]
si los límites del lado derecho existen.

Probemos que sí existen. Para el primero, constatamos que
\begin{align*}
\lim_{x\to 1} \cos (x^2-1) & = \cos \left(\lim_{x\to 1}(x^2-1) \right)  \\
& =\cos (0) \\
&=1,
\end{align*}
pues podemos aplicar el teorema del límite de una composición, dado que la función $\cos$ es continua en
$\mathbb{R}$, como lo probaremos más adelante, y:
\[
\lim_{x\to 1}(x^2-1)= 1^2-1 = 0.
\]

Por consiguiente, usando el teorema de las propiedades algebraicas de los límites se tiene:
\[
\lim_{x\to 1}k(x)= \frac{2}{\left( \lim_{x\to 1}x^2 \right)
\left(\lim_{x\to 1}\cos (x^2-1)  \right)} = \frac{2}{(1)(1)} =2.
\]

Para calcular $\displaystyle\lim_{x\to 1}h(x)$, notemos que $h(x)=j(x^2-1)$. En este caso, utilicemos el cambio
de variable $t=g(x)= x^2-1$, para la cual conocemos que
\[
  \lim_{x\to 1}g(x) =\lim_{x\to 1}(x^2-1) =0.
\]
Como la función $j$ no es continua en 0, para utilizar el teorema de cambio de variable, debemos
probar que $g$ es distinta de $0$ en un intervalo centrado en $1$, excepto quizás en $1$. El
intervalo puede ser $[0,2]$, pues, allí, tenemos que
\[
g(x) = x^2 - 1 \neq 0
\]
si $x\neq 1$ (pues $g$ se hace $0$ únicamente en $1$ y en $-1$). Por el teorema de cambio de
variable tenemos que:
\[
\lim_{x\to 1}h(x)= \lim_{x\to 1}j(x^2-1)= \lim_{t\to 0}j(t)=1.
\]
Por lo tanto:
\[
\lim_{x\to 1}f(x)=\lim_{x\to 1}k(x)\cdot \lim_{x\to 1}h(x) =2\cdot 1= 2.
\]
\end{exemplo}

\subsection{Ejercicios}
\begingroup
\small
\begin{multicols}{2}
\begin{enumerate}[leftmargin=*]
\item Use el teorema del cambio de variable para calcular los límites dados, justificando su
    uso en cada caso.
            \begin{enumerate}[leftmargin=*]
            \item $\displaystyle \lim_{x\to 2}\frac{\sen(x^2-4)}{x^4-16}$.
            \item $\displaystyle \lim_{x\to 1}\sqrt{x^2-3x+4}$.
            \item $\displaystyle \lim_{x\to -1}\sqrt[3]{x^2-2x+2}$.
            \item $\displaystyle \lim_{x\to 0}\sqrt{\frac{-x+8}{x+2}}$.
            \end{enumerate}

\item Pruebe que si existe $\displaystyle\lim_{y\to 0}f(y)\neq f(0)$, no se puede usar el teorema del cambio de variable para
    calcular el límite
\[
\lim_{x\to 0}f\left(x\sen \frac{1}{x}\right),
\]
%aunque se conozca que existe $\displaystyle \lim_{y\to 0}f(y)=L$.
\item Supongamos que existen
\[
\lim_{y\to b}f(y) = l \quad \text{y} \quad \lim_{x\to a}g(x) = b.
\]
Pruebe que se puede aplicar el teorema de cambio de variable para calcular
$\displaystyle\limjc{f(g(x))}{x}{a}$, si poniendo $h(x)=g(x)-b$, la función $h$ cambia de signo en $a$.

\item Si existe, halle el valor del límite dado. Haga uso de las siguientes igualdades, cuya
    validez será demostrada más adelante:
    \[
    \lim_{x\to 0}\frac{\sen x}{x}=1 \yjc
    \lim_{x\to 0}\frac{1-\cos x}{x}=0.
    \]
    De ser el caso, realice un cambio de variable adecuado.

\begin{enumerate}[leftmargin=*]
\item $\displaystyle \lim_{s\to 0}\frac{\sen 3s}{5s}$.
\item $\displaystyle \lim_{t\to 0}\frac{\tan 8t}{7t}$.
\item $\displaystyle \lim_{x\to 0}\frac{\sen x+\cos x}{3+\cos x}$.
\item $\displaystyle \lim_{x\to 1}\frac{\sen (x^2-1)}{x^2-3x+2}$.
\item $\displaystyle \lim_{x\to 2}\frac{2x^2+x-10}{\tan (2-x)}$.
\item $\displaystyle \lim_{x\to 0}\frac{x}{\sen\sqrt[3]{x}}$.
\item $\displaystyle \lim_{x\to 0}\frac{1-\cos^2x}{\sen x}$.
\item $\displaystyle \lim_{x\to 0}\frac{x^2+x}{\sen x+\tan x}$.
\item $\displaystyle \lim_{x\to 0}\frac{\arcsen x}{x}$.
\item $\displaystyle \lim_{x\to 0}\frac{\arctan x}{x}$.
\end{enumerate}
\end{enumerate}
\end{multicols}
\endgroup

\section{El teorema del ``sandwich''}
En las últimas unidades, hemos venido utilizando el siguiente límite:
\[
1 = \limjc{\frac{\sen x}{x}}{x}{0}.
\]
Con las propiedades sobre los límites que hemos desarrollado hasta ahora, no es posible que
probemos que esta igualdad es verdadera.

En efecto, no podemos aplicar el límite de un cociente, porque el límite del denominador es igual a
$0$. Por otro lado, no se vislumbra un cambio de variable que nos conduzca a algún límite ya
calculado.

En esta sección, vamos a enunciar un teorema, conocido como el teorema del ``sandwich'', que nos
proveerá de una herramienta muy útil para el cálculo de límites. En el siguiente capítulo,
demostraremos la validez de este resultado.

Antes de enunciar el teorema, veamos cuál es la idea subyacente a él. Para ello, analicemos el caso
del límite
\[
1 = \limjc{\frac{\sen x}{x}}{x}{0}.
\]

La interpretación geométrica de las funciones $\sen$, $\cos$ y $\tan$ nos permiten establecer las
siguientes desigualdades: para todo $x \in \ ]-\frac{\pi}{2},\frac{\pi}{2}[ - \{0\}$, se verifica

\begin{equation*}
	0<\cos x< \frac{\sen x}{x}<1.
\end{equation*}

Para demostrarlas, consideremos un círculo de centro $O$ y radio igual a $1$, como los que se
muestran en la siguiente figura:
\begin{center}
\begin{pspicture}(-0.5,-2.5)(4.5,2.5)
\scriptsize

\psset{PointSymbol=none}

\pstGeonode[PosAngle={-90,0},PointNameSep=0.8em]%
  (0,0){O}(4,0){A}%

\pstCurvAbsNode[]%
  {O}{A}{P}{\pstDistVal{2}}%
\pstCurvAbsNode[]%
  {O}{A}{P'}{\pstDistVal{-2}}%

\pstArcOAB[]%
  {O}{P'}{P}%

\pstLineAB[]%
  {O}{P}%
\pstLineAB[]%
  {O}{P'}%
\pstLineAB[]%
  {P}{P'}%
\pstLineAB[]%
  {O}{A}%

\pstInterLL[PosAngle=225]%
  {P}{P'}{O}{A}{C}%

\pstMarkAngle[MarkAngleRadius=0.8]%
  {A}{O}{P}{}%
\pstMarkAngle[MarkAngleRadius=0.8]%
  {P'}{O}{A}{}%
  
\rput(4.1,1){$x$}
\rput(4.1,-1){$x$}

\pstMiddleAB[PointName=none]%
  {P}{C}{M}%
\uput[180](M){$\sen x$}%

\pstMiddleAB[PointName=none]%
  {O}{C}{N}%
\uput[-90](N){$\cos x$}%

\pstRightAngle[RightAngleSize=0.21]%
  {O}{C}{P}
\end{pspicture}
\hspace{5em}
\begin{pspicture}(-0.5,-0.5)(4.5,4.5)
\scriptsize

\psset{PointSymbol=none}

\pstGeonode[PosAngle={-90,0},PointNameSep=0.8em,PointName={default,default,none}]%
  (0,0){O}(4,0){A}(4,1){B}%

\pstCurvAbsNode[]%
  {O}{A}{P}{\pstDistVal{2}}%
\pstCurvAbsNode[]%
  {O}{A}{R}{\pstDistVal{4}}%

\pstLineAB[]%
  {O}{A}%

\pstMarkAngle[MarkAngleRadius=0.8]%
  {A}{O}{P}{}%
\pstMarkAngle[MarkAngleRadius=0.8]%
  {P}{O}{R}{}%
\pstRightAngle[RightAngleSize=0.21]%
  {O}{A}{B}%
 
\rput(3.5,1){$x$}
\rput(2.8,2.5){$x$}

\pstLineAB[]%
  {O}{A}%
\pstLineAB[]%
  {O}{R}%

\pstInterLL[]%
  {A}{B}{O}{P}{Q}%

\pstLineAB[]%
  {A}{Q}%
\pstLineAB[]%
  {Q}{R}%
\pstLineAB[]%
  {O}{Q}%
\pstRightAngle[RightAngleSize=0.21]%
  {Q}{R}{O}%

\pstArcOAB[]%
  {O}{A}{R}%

\pstMiddleAB[PointName=none]%
  {A}{Q}{S}%

\uput[0](S){$\tan x$}
\end{pspicture}
\end{center}
Entonces, si $x$ es la medida en radianes de un ángulo agudo; es decir, si $x \in\ ]0,\pi/2[$, entonces $\sen x$, $\cos x$ y $\tan
x$ son las medidas de los segmentos $\sjc{CP}$, $\sjc{CO}$ y $\sjc{AQ}$, respectivamente.

Adicionalmente, tenemos que:

\begin{enumerate}
\item $OA = OP = OP' = OR = 1$.
\item La longitud de la cuerda $\sjc{PP'}$ es menor que la longitud del arco $\wideparen{PAP'\
    }$. Puesto que $CP = CP' = \sen x$, la desigualdad entre la cuerda y el arco se traduce en
    la siguiente desigualdad:
    \begin{equation}
    \label{eq:Lim003}
    2\sen x < \ell(\wideparen{PAP'\ }),
    \end{equation}
    donde $\ell(\wideparen{PAP'\ })$ indica la longitud del arco $\wideparen{PAP'\ }$.
\item La longitud de la línea quebrada $\widehat{AQR\ }$ es mayor que la longitud del arco
    $\wideparen{ APR\ }$. Como $AQ = RQ = \tan x$, la desigualdad se traduce en:
    \begin{equation}
    \label{eq:Lim004}
    \ell(\wideparen{APR\ }) < 2\tan x.
    \end{equation}
\item Puesto que la longitud del arco $\wideparen{AP\ }$ es la mitad de la longitud del arco
    $\wideparen{PAP'\ }$ y la longitud del arco $\wideparen{AP}$ es la mitad de la longitud del
    arco $\wideparen{APR\ }$, las desigualdades~(\ref{eq:Lim004}) y (\ref{eq:Lim003}) implican
    las siguientes desigualdades:
    \[
        \ell(\wideparen{AP\ }) < \frac{\sen x}{\cos x} \yjc \sen x < \ell(\wideparen{AP\ }).
    \]
    Por lo tanto, se deben verificar las siguientes desigualdades:
    \[
        \ell(\wideparen{AP})\cos x < \sen x < \ell(\wideparen{AP}).
    \]

    Y, si suponemos que $x\neq 0$, entonces $\ell(\wideparen{AP}) \neq 0$. Entonces, estas dos
últimas desigualdades implican las siguientes:
    \begin{equation}
    \label{eq:Lim005}
    \cos x < \frac{\sen x}{\ell(\wideparen{AP})} < 1.
    \end{equation}

\item La longitud del arco $\wideparen{AP\ }$ es igual a $x$, pues, por la definición de
    radián, se verifica que:
    \[
     \ell(\wideparen{AP\ }) = OA x = 1\times x = x.
    \]
    Por lo tanto, las desigualdades~(\ref{eq:Lim005}) se escriben de la siguiente manera:
    \begin{equation}
    \label{eq:Lim006}
    \cos x < \frac{\sen x}{x} < 1.
    \end{equation}
\end{enumerate}

En resumen, como $x$ es un ángulo agudo, entonces $x\in ]0,\frac{\pi}{2}[$. Y es para estos valores
de $x$ que las igualdades~(\ref{eq:Lim006}) se verifican.

El mismo procedimiento realizado para $x\in ]0,\frac{\pi}{2}[$ puede ser realizado para $x\in
]-\frac{\pi}{2}, 0[$. Lo que prueba que las desigualdades~(\ref{eq:Lim006}) son verdaderas para
todo $x$ distinto de $0$ tal que $-\frac{\pi}{2} < x < \frac{\pi}{2}$.

Ahora bien, cuando $x$ tiende a $0$, probaremos más adelante que $\cos x$ tiende a $\cos 0$; es
decir, tiende a $1$. Lo mismo sucede con la constante $1$. Es decir, las dos cotas de la fracción
\[
\frac{\sen x}{x},
\]
la mayor y la menor, tienden a $1$ cuando $x$ tiende a $0$. ?`Podría suceder, entonces, que la
fracción no tendiera a $0$? El teorema del ``sandwich'' nos asegura que eso no puede suceder. Es
decir, este teorema nos asegura que es verdad que
\[
1 = \limjc{\frac{\sen x}{x}}{x}{0}.
\]

Este es el enunciado del teorema:

\begin{teocal}[Teorema del sandwich o de los dos gendarmes, o principio de intercalación]
Sean $a\in\mathbb{R}$, $f:\Dm(f) \rightarrow \mathbb{R}$, $g:\Dm(g) \rightarrow \mathbb{R} $,
$h:\Dm(h) \rightarrow \mathbb{R}$ tres funciones tales que sus dominios contienen un intervalo
abierto $I$ centrado en $a$, excepto quizás el punto $a$.
Supongamos que:
$
	f(x)\leq g(x)\leq h(x) \ \text{ para todo }\ x\in I - \{a\}
$
y que
$\displaystyle
	L = \lim_{x \rightarrow a}f(x)= \lim_{x \rightarrow a}h(x)
$.
Entonces
$\displaystyle
	L = \lim_{x \rightarrow a}g(x).
$
\end{teocal}

Para el caso del límite
\[
\limjc{\frac{\sen x}{x}}{x}{0},
\]
tenemos que $f(x) = \cos x$, $g(x) = \frac{\sen x}{x}$ y $h(x) = 1$ para todo $x \in D =
]-\frac{\pi}{2},\frac{\pi}{2}[ - \{0\}$ y que
\[
L = 1 = \limjc{f(x)}{x}{0} = \limjc{h(x)}{x}{0}.
\]

El nombre de ``sandwich'' --o en español ``emparedado''-- se debe al hecho de que la función $g$
(el queso) se ``coloca'' entre $f$ y $h$ (los panes). Los franceses lo llaman teorema ``de los dos
gendarmes'' ($f$ y $h$) que ``llevan preso a $g$ entre ellos''.

\begin{exemplo}[Solución]{%
Calcular
\[
\lim_{x\to 0}x^2\sen^2 \frac{\pi}{x} \yjc  \lim_{x\to 0}x\cos \frac{1}{x}.
\]
}
\begin{enumerate}[leftmargin=*]
\item Si definimos $f$, $g$ y $h$ tales que
\[
g(x)=0, \quad f(x)= x^2\sen^2 \frac{\pi}{x} \yjc h(x)=x^2,
\]
tenemos que, para todo $x\neq 0$:
\[
g(x)\leq f(x)\leq h(x).
\]
Como
\[
\lim_{x\to 0}g(x)= \lim_{x\to 0}h(x)=0,
\]
por el teorema del sandwich tenemos que:
\[
\lim_{x\to 0}f(x)=0.
\]

\item Definamos:
\[
g(x)=0, \quad f(x)=\left| x\cos \frac{1}{x}  \right| \yjc h(x)=|x|.
\]
Tenemos que, para todo $x\neq 0$:
\[
g(x)\leq f(x)\leq h(x).
\]
Por consiguiente:
\[
\lim_{x\to 0}|f(x)|= \lim_{x\to 0}\left|x\cos \frac{1}{x}\right|=0.
\]

Puesto que (ver el teorema~\ref{teol:LEquiv0})
\[
\lim_{x\to a} \varphi (x)=0\quad \Leftrightarrow \quad \lim_{x\to a} |\varphi (x)|=0,
\]
tendremos entonces que
\[
\lim_{x\to 0}x\cos \frac{1}{x} = 0.
\]
\end{enumerate}
\end{exemplo}

Como otro ejemplo, demostremos que las funciones $\sen$ y $\cos$ son continuas. El teorema del
sandwich será de ayuda para ello.

\begin{exemplo}[Solución]{
Las funciones $\sen$ y $\cos$ son continuas en $\mathbb{R}$. Y, por lo tanto, las otras cuatro
funciones trigonométricas son también continuas en sus respectivos dominios. }

\begin{enumerate}[leftmargin=*]
\item En primer lugar, probemos que $\sen$ es continua en 0. Para ello, probemos que
$\displaystyle
\lim_{x\to 0}\sen x=0.
$

Para todo $x\in \mathbb{R}$, se tiene que
$
0 \leq |\sen x| \leq |x|.
$
La certeza de esta desigualdad puede ser obtenida del procedimiento seguido para probar que el
cociente $\frac{\sen x}{x}$ está acotado entre $\cos x$ y $1$, desarrollado en páginas
anteriores.
Por el teorema del sandwich y el teorema~(\ref{teol:LEquiv0}) tenemos que:
\[
\lim_{x\to 0}|\sen x| = \limjc{0}{x}{0} = \limjc{|x|}{x}{0} = 0.
\]
Por lo tanto, otra vez por el teorema~\ref{teol:LEquiv0}, podemos afirmar que
$\displaystyle
\lim_{x\to 0}\sen x=0.
$

\item Ahora probemos que $\cos$ es continua en 0. Para ello, debemos probar que
$\displaystyle
\lim_{x\to 0}\cos x=1.
$
Esto equivale a probar que
$\displaystyle
\lim_{x\to 0}(1-\cos x) = 0.
$
Como $1-\cos x= 2 \sen^2\frac{x}{2}$ para todo $x\in \mathbb{R}$ y como
\[
\lim_{x\to 0}\sen\frac{x}{2}= \lim_{t\to 0}\sen t =0
\]
(usando el cambio de variable $t=\frac{x}{2}$, lo que se puede ya que $x\neq 0$ implica  $t\neq
0$), tenemos que
\[
\lim_{x\to 0}(1-\cos x)= 2 \left(\lim_{x\to 0} \sen \frac{x}{2}  \right)^2 =2(0)=0
\]
(por las propiedades algebraicas de los límites).

\item Sea $a\neq 0$. Vamos a probar que
$\displaystyle
\lim_{x\to a}\sen x=\sen a.
$
Para ello, probemos que
$\displaystyle
\lim_{x\to a}(\sen x- \sen a)=0.
$
Hagamos el cambio de variable $x=a+t$. Esto es posible, ya que $\displaystyle\lim_{x\to a}t=0$ y que $x\neq
a$ implica $t\neq 0$. Entonces, como $t=x-a$, tenemos que
\begin{align*}
\lim_{x\to a}(\sen x- \sen a) & = \lim_{t\to 0}[\sen (t+a)- \sen a] \\
& = \lim_{t\to 0}[\sen t \cos a +\cos t \sen a- \sen a]  \\
& = \lim_{t\to 0}[\sen t \cos a +\sen a(\cos t -1) ]  \\
& = \cos a \lim_{t\to 0}\sen t  +\sen a \lim_{t\to 0}(\cos t -1) ] \\
&= (\cos a)(0)+ (\sen a)(0) = 0.
\end{align*}

\item Sea $a\neq 0$. Probemos que
$\displaystyle
\lim_{x\to a}\cos x=\cos a.
$
Para ello, probemos que
$\displaystyle
\lim_{x\to a}(\cos x- \cos a)=0.
$
Con el mismo cambio de variable que en el numeral anterior, tenemos que:
\begin{align*}
\lim_{x\to a}(\cos x- \cos a) & = \lim_{t\to 0}[\cos (t+a)- \cos a] \\
& = \lim_{t\to 0}[\cos t \cos a -\sen t \sen a- \cos a]  \\
& = \lim_{t\to 0}[(\cos t-1) \cos a -\sen a\sen t ]  \\
& = \cos a \lim_{t\to 0}(\cos t -1) -\sen a \lim_{t\to 0}\sen t \\
&= (\cos a)(0)-(\sen a)(0) = 0.
\end{align*}

\end{enumerate}
\end{exemplo}

\subsection{Ejercicios}
\begingroup
\small
\begin{multicols}{2}
\begin{enumerate}[leftmargin=*]
\item Use el teorema del sandwich para calcular:
            \begin{enumerate}
            \item $\displaystyle \lim_{x\to 1}(x-1)^2\cos\frac{\pi}{x-1}$.
             \item $\displaystyle \lim_{x\to 1}f(x)$, si se sabe que $1-|x-1|\leq f(x)\leq
                 x^2-2x+2$.
             \item $\displaystyle \lim_{x\to -1}(x+1)^2g(x)$, si se conoce que existe $M>0$
                 tal que para todo $x$, $|g(x)|<M$.
             \end{enumerate}
\item Diga si se puede aplicar el teorema del sandwich para calcular $\displaystyle\limjc{g(x)}{x}{1}$ si se
    conoce que para todo $x$
\[
2-|x-1|\leq g(x)\leq x^2-2x+4.
\]
\item Diga si se puede aplicar el teorema del sandwich para calcular $\displaystyle\limjc{g(x)}{x}{1}$ si se conoce que para todo $x$, se verifica la desigualdad siguiente:
\[
|g(x)+3|\leq (x-1)^4.
\]

\item Calcule $\displaystyle \limjc{\frac{\tan x}{x}}{x}{0}$.
\end{enumerate}
\end{multicols}
\endgroup

\section{Límites unilaterales}
Como una aplicación de la propiedad arquimediana de los números reales, sabemos que para todo
$x\in\mathbb{R}$, existe un único número entero $n$ tal que
\begin{equation}
\label{prp:ExistenciaSuelo}
n \leq x < n + 1.
\end{equation}
A este número $n$ se le denomina el \emph{suelo} de $x$. Por ejemplo, el suelo de $32.45$ es $32$,
pues
\[
32 \leq 32.45 < 33.
\]
En este caso, $n = 32$.

En el caso de que $x$ sea mayor que $0$, el suelo de $x$ será la parte entera de su representación
decimal. Por ello, al número $n$ también se le conoce como la \emph{parte entera de $x$}  y se le
suele representar\footnote{La notación $[x]$ también suele ser utilizada para representar la parte
entera del número $x$.} con $\lfloor x \rfloor$.

Dado un $x\in\mathbb{R}$, la unicidad del número $n$ que satisface las
desigualdades~(\ref{prp:ExistenciaSuelo}), nos permite definir la función \emph{suelo} de la
siguiente manera:
\[
\funcionjc{\lfloor \ \rfloor}{\mathbb{R}}{\mathbb{Z}}{x}{\lfloor x \rfloor = n,}
\]
donde $n$ satisface las desigualdades~(\ref{prp:ExistenciaSuelo}).

El lector puede constatar por sí mismo que dibujar el gráfico de la función suelo es muy sencillo.
Deberá obtener algo similar al siguiente dibujo:
\begin{center}
\psset{unit=0.75}
\begin{pspicture}(-3.5,-3.5)(3.75,3.75)
\psset{labelFontSize=\scriptstyle}%

\psaxes[arrows=->,linecolor=gray]%
  (0,0)(-3.5,-3.5)(3.5,3.5)%
\uput[-90](3.5,0){$x$}%
\uput[0](0,3.5){$y$}%

\multido{\ii=-3+1,\is=-2+1}{6}{%
\psline[linewidth=\pslinewidth]%
  {*-o}(\ii,\ii)(\is,\ii)%
}%
\end{pspicture}
\end{center}

A pesar de la sencillez de la función suelo, es muy útil a la hora de cuantificar ciertas
magnitudes en situaciones en las que los números reales no pueden captar la esencia del problema.
Por ejemplo, un problema que aparece frecuentemente en el campo de las ciencias de la computación
es el de contar el número de veces que se ejecutan las instrucciones de un algoritmo.

Para concretar, imaginemos el caso de tener que buscar un número de cédula en una lista de un
millón de números de cédulas (como puede suceder en un padrón electoral). Supongamos que, cada vez
que se registra un número de cédula en la lista, la lista es ordenada en forma ascendente. A pesar
de este orden, a la hora de requerir la información relacionada a uno de estos números de cédula,
es necesario buscar dicho número en la lista. El algoritmo de búsqueda denominada \emph{búsqueda
binaria} realiza, cuando el número buscado no está en la lista o es localizado en la última
comparación,
\[
W(n) = \lfloor \lg(n + 1) \rfloor + 1
\]
comparaciones del valor que busca con los valores que están en la lista, donde $n$ es el número de
elementos en la lista y $\lg$ es la función logaritmo en base $2$.

En el caso del ejemplo, $n = 10^6$. Entonces
\[
W(10^6) = \lfloor \lg(10^6 + 1) \rfloor + 1 = 20.
\]

Es decir, el algoritmo de búsqueda binaria solo deberá realizar $20$ comparaciones para indicar que
el número de cédula buscado no está en la lista.

Como puede observarse, la función $\lg$ retorna un número real. La función suelo ``transforma''
este número real en un entero, positivo en este ejemplo, y que refleja correctamente la naturaleza
del problema.

En los ejercicios se presentará una guía para calcular $W(10^6)$ sin recurrir a una calculadora
electrónica para calcular el valor de $\lg(10^6 + 1)$.

Del gráfico de la función suelo, se puede observar que esta función es continua en todo número real
que no es un entero. Utilizando la definición de límite, se puede probar que la función suelo no es
continua en cada entero al mostrar que no existe el límite allí.

?`Por qué no existe ese límite, por ejemplo en el número $1$? Porque $\lfloor x \rfloor$ tiene un
comportamiento diferente antes del número $1$, pero cerca de él, y otro luego del número $1$, pero
también cerca de él.

En casos como el de este ejemplo, resulta conveniente introducir la noción de límites unilaterales
en el sentido de que $x$ se aproxima al número $1$ bajo la condición de que $x > 1$ o bajo la
condición de que $x < 1$. Si una función tiene límite en el punto $1$, la aproximación bajo
cualquiera de las dos condiciones deberá producir el mismo límite; es decir, deberá suceder que la
función tenga los dos límites unilaterales y, además, sean iguales.

Este principio es adecuado para probar también que el límite de una función no existe en un punto:
se prueba que o no existe uno de los unilaterales, o se prueba que son diferentes. Por ejemplo, es
fácil probar que
\[
1 = \limjc{\lfloor x \rfloor}{x}{1} \quad \text{cuando} \ x > 1
\]
y que
\[
0 = \limjc{\lfloor x \rfloor}{x}{1} \quad \text{cuando} \ x < 1.
\]

Las definiciones de límites unilaterales son similares a la definición general de límite, lo mismo
que las propiedades de estos límites.

% -----> 2008 09 12
%\newpage

\begin{defical}[Límites unilaterales]
$L$ es el límite de $f(x)$ cuando $x$ se aproxima a $a$ \emph{por la derecha}, y se escribe:
\begin{equation*}
	L=\lim_{x \rightarrow a^+}f(x),
\end{equation*}
si y solo si para todo $\epsilon > 0$, existe $\delta > 0$ tal que
\[
|f(x) - L| < \epsilon,
\]
siempre que $0 < x - a < \delta$.

Análogamente: $L$ es el límite de $f(x)$ cuando $x$ se aproxima a $a$ \emph{por la izquierda}, y se
escribe:
\begin{equation*}
	L=\lim_{x \rightarrow a^-}f(x),
\end{equation*}
si y solo si para todo $\epsilon > 0$, existe $\delta > 0$ tal que
\[
|f(x) - L| < \epsilon,
\]
siempre que $0 < a - x < \delta$.
\end{defical}

Con estas definiciones, podemos afirmar que:
\[
1 = \limjc{\lfloor x \rfloor}{x}{1^+} \yjc 0 = \limjc{\lfloor x \rfloor}{x}{1^-}.
\]
Entonces:
\[
\limjc{\lfloor x \rfloor}{x}{1^+} \neq \limjc{\lfloor x \rfloor}{x}{1^-},
\]
Esto es suficiente para afirmar que no existe el límite de $\lfloor x \rfloor$ cuando $x$ se
aproxima a $1$. De hecho, se tiene el siguiente teorema:

\begin{teocal}$L$ es el límite de $f(x)$ cuando $x$ se aproxima al número $a$ si y solo si
existen los dos límites unilaterales y son iguales a $L$.
\end{teocal}

Las propiedades de los límites que enunciamos en teoremas anteriores se verifican también para los
límites unilaterales con las evidentes modificaciones en cada caso.

Con la noción de límites unilaterales se puede definir la \emph{continuidad por la derecha} y por
\emph{la izquierda} mediante la siguiente modificación de la definición de continuidad: en lugar de
que $f(a)$ sea el límite de $f(x)$ cuando $x$ se aproxima a $a$, hay que cambiar, en el caso de la
continuidad por la derecha, que $x$ se aproxima a $a$ por la derecha; lo mismo para el caso de la
continuidad por la izquierda. Es inmediato de esta definición que una función será continua en un
punto si y solo si es continua por la derecha y por la izquierda del punto. Este hecho lo
expresamos en el siguiente teorema.

\begin{teocal}Una función es continua en $a$ si y solo si es continua en $a$ por la derecha y es
continua en $a$ por la izquierda.
\end{teocal}

\begin{exemplo}[Solución]{%
Sea $\funcjc{f}{[0,+\infty[}{[0,+\infty[}$ definida por $f(x)= \sqrt{x}$. Entonces $f$ es continua
en $0$ por la derecha.} Puesto que
\[
\lim_{x\to a}\sqrt{x}= \sqrt{a},
\]
para $a > 0$, $f$ es continua en $\mathbb{R}^+$. Como $f(x)$ no está definida para $x<0$, no existe
\[
\lim_{x\to 0}\sqrt{x}.
\]
Sin embargo, tenemos que \emph{$f$ es continua en $0$ por la derecha}, es decir:
\[
\lim_{x\to 0^+}\sqrt{x}=0=\sqrt{0}.
\]

En efecto: sea $\epsilon>0$. Debemos hallar $\delta >0$ tal que
\[
|\sqrt{x}-0|<\epsilon,
\]
siempre que $0<x<\delta$.

Sea $x>0$. Como $|\sqrt{x}- \sqrt{0}|=\sqrt{x}$, tenemos que
\[
|\sqrt{x}-0|<\epsilon \quad \Leftrightarrow \quad \sqrt{x}<\epsilon \quad \Leftrightarrow \quad x<\epsilon^2.
\]
Tomemos, entonces, $\delta=\epsilon^2$. En ese caso, tenemos:
\[
0<x<\delta \quad \Rightarrow \quad x<\epsilon^2 \quad \Rightarrow \quad
\sqrt{x}<\epsilon \quad \Rightarrow \quad |\sqrt{x}- \sqrt{0}|<\epsilon.
\]
Hemos probado entonces que $\displaystyle 0=\sqrt{0}= \lim_{x\to 0^+}\sqrt{x}$; es decir, hemos probado que $f$
es continua en $0$ por la derecha.

Es claro que la función $f$ no puede ser continua en $0$ por la izquierda, pues $f$ no está
definida para ningún $x < 0$.
\end{exemplo}

\begin{exemplo}[Solución]{%
Sea
\begin{equation*}
f(x) =
\begin{cases}
2x^2-1& $si $x>1$$\\
2& $si $x=1$$\\
-x^2+3x& $si $x<1$$
\end{cases}
\end{equation*}
?`Es $f$ continua en $1$, en $1$ por la derecha, en $1$ por la izquierda?}%
Para saber si es continua en
$1$, veamos si es continua por la izquierda y por la derecha. Para ello, calculemos $\displaystyle\lim_{x\to
1^-}f(x)$ y $\displaystyle\lim_{x\to 1^+}f(x)$. En primer lugar:
\[
\lim_{x\to 1^-}f(x)  = \lim_{x\to 1^-}(-x^2+3x),
\]
pues $f(x)=-x^2+3x$ si $x < 1$. Entonces:
\begin{align*}
\lim_{x\to 1^-}f(x) &= \lim_{x\to 1}(-x^2+3x) \\
&= -(1)^2 + 3\times 1 = 2.
\end{align*}

En segundo lugar, como $f(x) = 2x^2-1$ si $x > 1$, entonces:
\begin{align*}
\lim_{x\to 1^+}f(x) &= \lim_{x\to 1}(2x^2-1) \\
&= 2(1)^2 - 1 = 1.
\end{align*}

Por lo tanto:
\[
\lim_{x\to 1^-}f(x)  = f(1) = 2\neq 1 =\lim_{x\to 1^+}f(x) .
\]

En conclusión, $f$ es continua en 1 por la izquierda, pero no lo es por la derecha, ni tampoco es
continua en $1$.
\end{exemplo}

\begin{exemplo}[Solución]{%
Sea $\funcjc{f}{\mathbb{R}}{\mathbb{R}}$ definida por $f(x) = \lfloor x \rfloor$. Entonces $f$ es
discontinua en $x$ si y solo si $x\in \mathbb{Z}$.}

Sean $x$ y $n$ tales que $f(x) = n$. Entonces:
\[
n \leq x < n +1.
\]
Por lo tanto, $f$ es constante en $]n,n+1[$, por lo que es continua en cualquier intervalo entre
dos enteros consecutivos.

Veamos ahora que no es continua en ningún entero. Para ello calculemos los límites laterales en
$n$. Para empezar:
\[
\lim_{x\to n^+}f(x) = \lim_{x\to n}n = n = f(n).
\]
Por otra parte, si $x < n$, $f(x) = n - 1$. Entonces:
\[
\lim_{x\to n^-}f(x) = \lim_{x\to n}(n-1) = n-1 \neq f(n).
\]
Por lo tanto, para $n\in \mathbb{Z}$:
\[
\lim_{x\to n^-}f(x) = n-1\neq n= f(n) = \lim_{x\to n^+}f(x).
\]
Entonces: $f$ es discontinua en $n$ para todo $n\in \mathbb{Z}$.
\end{exemplo}

\begin{exemplo}[Solución]{%
Sea $\funcjc{g}{\mathbb{R}}{\mathbb{R}}$ definida por $g(x)=x-\lfloor x \rfloor$. El número $g(x)$
es la ``parte fraccionaria de $x$''. Entonces $g$ es discontinua en $x$ si y solo si $x\in
\mathbb{Z}$.}

Sean $f$ definida por $f(x) = \lfloor x\rfloor$ y $h$ definida por $h(x) = x$ para todo
$x\in\mathbb{R}$. Entonces $g = h - f$, pues
\[
g(x) = h(x) - f(x) = x - \lfloor x \rfloor.
\]

Si $g$ fuera continua en $n\in\mathbb{Z}$, como $h$ es continua en $\mathbb{R}$, es continua en
$n$. Por lo tanto: $g + h$ sería continua en $n$. Pero $f = h - g$ sería continua en $n$, ya que
$h$ y $g$ lo son. Pero sabemos por el ejercicio anterior que $f$ no es continua en ningún entero.
Por lo tanto, $g$ no puede ser continua en $n$.

?`Por qué el número $g(x)$ es llamado la ``parte fraccionaria de $x$''?
\end{exemplo}

\subsection{Límites unilaterales de funciones localmente iguales}

El concepto de funciones localmente iguales puede aplicarse también unilateralmente.

\begin{defical}[Funciones localmente iguales por la derecha o por la izquierda]
Sean: $a$ un número real; y, $f$ y $g$ dos funciones reales.

Diremos que $f=g$ cerca de $a$ por la derecha (respectivamente por la izquierda), si existe un número $r>0$ tal que para todo $x\in ]a,a+r[$ (respectivamente $x\in ]a-r,a[$), se tiene que $f(x)=g(x)$.
\end{defical}

Con esta definición se puede fácilmente demostrar el siguiente resultado.

\begin{teocal}[Límite unilaterales de funciones localmente iguales]%
Sean $a$ un número real; y $f$ y $g$ dos funciones reales localmente iguales por la derecha (respectivamente por la izquierda). Entonces:
\begin{enumerate}
\item Existe $\displaystyle\limjc{f(x)}{x}{a{+}}$ (respectivamente $\displaystyle\limjc{f(x)}{x}{a{-}}$) si y solo si existe $\displaystyle\limjc{g(x)}{x}{a{-}}$ (respectivamente $\displaystyle\limjc{g(x)}{x}{a{-}}$).
\item Si los límites existen, son iguales.
\end{enumerate}
\end{teocal}


\subsection{Ejercicios}
\begingroup
\small
\begin{multicols}{2}
\begin{enumerate}[leftmargin=*]
\item Dibuje la gráfica de $f$ y determine, si existen, los límites para el valor de $a$ dado:
\begin{equation*}
	\lim_{x\to a{-}}f(x), \quad \lim_{x\to a{+}}f(x), \quad \lim_{x\to a}f(x).
\end{equation*}
\begin{enumerate}[leftmargin=*]
\item
\begin{equation*}
	f(x)=
\begin{cases}
2x-1& \text{si $x<2$}\\
3 & \text{si $x=2$}\\
x+1 & \text{si $x>2$}.
\end{cases}
\quad a=2
\end{equation*}
\item
\begin{equation*}
	f(x)=
\begin{cases}
x^2-1 & \text{si $x\leq 2$}\\
\frac{1}{3}(11-x)& \text{si $x>2$}.
\end{cases}
\quad  a=2
\end{equation*}
\item
\begin{equation*}
	f(x)=
\begin{cases}
|x+1| & \text{si $x<1$}\\
\sqrt{2-x}& \text{si $x\geq 1$}.
\end{cases}
\quad a=1
\end{equation*}
\item
\begin{equation*}
	f(x)=
\begin{cases}
\sqrt{9-x^2}& \text{si $|x|< 3$}\\
x+1 & \text{si $|x|\geq 3$}.
\end{cases}
\quad a=3
\end{equation*}
\item $f(x) =1+\lfloor x\rfloor$; $a\in \mathbb{Z}$.
\end{enumerate}

\item Mostrar que
  \[
      \lfloor x \rfloor = \max\{n\in\mathbb{Z} : n \leq x\}
  \]
  para todo $x\in\mathbb{R}$.

\item Estudie la continuidad de la función $f$ cuyo dominio es $\mathbb{R}$ y definida por
    $f(x) = \lfloor x \rfloor - \lfloor x + 1 \rfloor$.

\item Calcule los límites siguientes, si existen. En los casos en que no exista el límite,
    demuéstrelo:
    \begin{enumerate}
    \item $\displaystyle{\limjc{\frac{\sqrt{x - 1}}{x^2 - 3x + 2}}{x}{1^+}}$.
    \item $\displaystyle{\limjc{\frac{\sqrt{1 - x}}{x^2 - 3x + 2}}{x}{1^+}}$.
    \item $\displaystyle{\limjc{\frac{x+1}{x^2 - 3x + 2}}{x}{2^-}}$.
    \item $\displaystyle{\limjc{\frac{2x^2 - x + 6}{x^2 - 3x + 2}}{x}{2^-}}$.
    \end{enumerate}

\item Diga si la función $f$ es continua en $1$ por la derecha o por la izquierda:
\begin{enumerate}
\item $\displaystyle{f(x) = \begin{cases} x^2 - 3x + 1 & \text{si} \ x > 1 \\
-1 & \text{si} \ x = 1 \\
2x^2 + x + 1 & \text{si} \ x < 1.
\end{cases}}$

\item $\displaystyle{f(x) = \begin{cases} \sqrt{x - 1} & \text{si} \ x > 1 \\
1 & \text{si} \ x = 1 \\
x^2 - 3x + 3 & \text{si} \ x < 1.
\end{cases}}$

\item $\displaystyle{f(x) = \begin{cases} \frac{1}{x - 1} & \text{si} \ x > 1 \\
0 & \text{si} \ x = 1 \\
\frac{x^2 - 3x + 2}{\sqrt{1 - x}} & \text{si} \ x < 1.
\end{cases}}$

\item $\displaystyle{f(x) = \begin{cases} \sqrt{x^2 + x - 2} & \text{si} \ x > 1 \\
\sqrt{1 - x^2} & \text{si} \ x \leq 1.
\end{cases}}$

\end{enumerate}
\item Use las definiciones de límites laterales para probar que:

\begin{enumerate}
\item $\displaystyle \lim_{x\to 2^-}f(x)=-1$, $\displaystyle \lim_{x\to 2^+}f(x)=5$, si
\[
    f(x)=
\begin{cases}
x^2+1 & \text{si $x>2$}\\
-x^2+3 & \text{si $x<2$}.
\end{cases}
\]
\item $\displaystyle \lim_{x\to 1^-}\sqrt{-x^3+1}=0.$
\item $\displaystyle \lim_{x\to \frac{2}{3}^+}\sqrt[4]{3x-2}=0$.
\end{enumerate}

\item Pruebe que no existe el límite dado.

\begin{enumerate}
\item $\displaystyle \lim_{x\to -1}f(x)$, si
\[
f(x)=
\begin{cases}
x-3 & \text{si $x<-1$}\\
2x+1 & \text{si $x>-1$}.
\end{cases}
\]
\item $\displaystyle \lim_{x\to 2}\frac{2x+1}{x^4-4}$.
\item $\displaystyle \lim_{x\to 1}\frac{1}{x^2-5x+4}$.
\item $\displaystyle \lim_{x\to 0}f(x)$, si $\displaystyle f(x)=
\begin{cases}
x+1 & \text{si $x<0$}\\
2x-3& \text{si $x>0$}.
\end{cases}
$
\end{enumerate}

\item Con ayuda de las propiedades de la función $\lg$ se puede probar, sin recurrir al uso de
    una calculadora electrónica, el valor de $W(10^6)$, dado por:
    \[
      W(10^6) = \lfloor\lg(10^6  + 1)\rfloor + 1.
    \]
    En primer lugar, el lector debe constar mediante un cálculo directo que
    \[
        2^{19} < 10^6 + 1 < 2^{20}.
    \]
    A continuación, debe recordar que la función $\lg$ es estrictamente creciente y $\lg(x)$ es
    el número real al que hay que elevar $2$ para obtener $x$. Es decir:
    \[
      y = \lg(x) \Longleftrightarrow x = 2^y.
    \]
    Esta información es suficiente para obtener el valor exacto de $W(10^6)$.

\item Sea 
\[ f(x)=
\begin{cases}
x^2+1 & \text{si $x\leq 0$}\\
1-x^2 & \text{si $0<x\leq 1$; y,}\\
\alpha (x-1) & \text{si $x>1$.}
\end{cases}
\]
?`Existen valores de $\alpha$ para los cuales $f$ sea derivable ?
\end{enumerate}
\end{multicols}
\endgroup

\section{Límites infinitos y al infinito}
Cuando una función tiene límite en un punto, se dice que la función \emph{converge} al límite en
ese punto. Cuando la función no tiene límite, se dice que \emph{diverge}. Por ejemplo, la función
suelo diverge en todo número entero.

Hay varias maneras de divergir. Una como la de la función suelo. Otra, cuando la función crece
indefinidamente o decrece indefinidamente. En ese caso se dice que la divergencia es al infinito, o
que el límite es ``infinito''. Por ejemplo, probaremos que $f(x) = \frac{1}{x^2}$ crece
indefinidamente cuando $x$ tiende a $0$.

Por otra parte, puede suceder que un número esté tan cerca como se quiera de $f(x)$ cuando $x$
crece indefinidamente o cuando decrece indefinidamente; en ese caso, no hay divergencia, pues el
límite existe. En estas circunstancias, se dice que los ``límites son al infinito''. Por ejemplo,
mostraremos que la $f(x) = \frac{1}{x^2}$ tiende a $0$ cuando $x$ crece indefinidamente.

Hay numerosas situaciones en las que surgen estos dos tipos de límites. En las siguientes
subsecciones, vamos a ver un ejemplo de cada uno.

\subsection{Límites infinitos}
En $1905$, Albert Einstein corrigió un error que encontró en las leyes del movimiento enunciadas
por Newton doscientos años atrás.

En efecto, la segunda ley de Newton asume implícitamente que la masa de un cuerpo es constante. Sin
embargo, Einstein descubrió que la masa de un cuerpo varía con su velocidad\footnote{Un tratamiento
conceptual, profundo, pero al mismo tiempo sencillo, se encuentra en el libro de Richard Feynman,
"The Feynman Lectures on Physics", en el capítulo 15 del primer volumen.}.

De manera más precisa, Einstein estableció que la masa de un cuerpo es una función de su velocidad,
que puede ser calculada a través de la siguiente igualdad:
\[
m(v) = \frac{m_0}{\sqrt{1 - \frac{v^2}{c^2}}},
\]
donde $m_0$ es la masa que el cuerpo tiene cuando está en reposo, $v\kilometros/\segundos$ es su
velocidad y $c\kilometros/\segundos$ es la velocidad de la luz (aproximadamente $3 \times 10^5
\kilometros/\segundos$). Se supone, además, que la velocidad del cuerpo es menor que la velocidad
de la luz.

?`Qué sucedería si la velocidad del cuerpo se acercara a la velocidad de la luz?

Esta pregunta puede ser expresada en términos de límites de la siguiente manera: ?`a qué es igual el
límite de $m(v)$ cuando $v$ tiende a $c$?

Si damos a $v$ algunos valores cercanos a $c$, veremos que $m(v)$ crece. Probaremos en esta sección
que esto es, efectivamente, así.

La idea subyacente de que $f(x)$ crece indefinidamente cuando $x$ se acerca a un número $a$
consiste en que, dado cualquier número positivo $R$, por más grande que éste sea, siempre hay un
intervalo alrededor de $a$ en donde $f(x)$ es más grande que $R$. En otras palabras, no es posible
acotar superiormente el conjunto de valores de la función $f$. Se dice que $f(x)$ tiende a ``más
infinito'' para representar este ``crecimiento indefinido'' y se utiliza el símbolo $+\infty$ para
representar este comportamiento de la función $f$ alrededor del punto $a$. En la siguiente
definición, precisamos esta idea.

\lteocal[Límites infinitos]{defi}{%
$\displaystyle{\lim_{x \to a}f(x) =+\infty}$ si y solo si para todo $R > 0$, existe $\delta
    > 0$ tal que
    \[
    f(x) > R,
    \]
    siempre que $x\in\Dm(f)$ y $0 < |x - a| < \delta$.

Análogamente: $\displaystyle{\lim_{x \to a}f(x) =-\infty}$ si y solo si para todo $R < 0$, existe
$\delta
    > 0$ tal que
    \[
    f(x) < R,
    \]
    siempre que $x\in\Dm(f)$ y $0 < |x - a| < \delta$.}

En los ejemplos y en los ejercicios de esta sección, veremos definiciones análogas a los límites
laterales para el caso de límites infinitos.

Veamos algunos ejemplos.

\begin{exemplo}[Solución]{%
Probar que $\displaystyle\limjc{\frac{1}{x^2}}{x}{0} = +\infty$.} Sea $R > 0$. Debemos encontrar un
número $\delta > 0$ tal que
\begin{equation}
\label{eq:Lim007}
\frac{1}{x^2} > R,
\end{equation}
siempre que $0 < |x| < \delta$.

Para encontrar el número $\delta$, analicemos, en primer lugar, la desigualdad~(\ref{eq:Lim007})
con miras a establecer una relación de esta desigualdad con las desigualdades $0 < |x| < \delta$.

Para $x \neq 0$, las siguientes equivalencias son verdaderas:
\begin{align*}
\frac{1}{x^2} > R & \ \Leftrightarrow \ 0 < x^2 < \frac{1}{R} \\
  & \ \Leftrightarrow \ 0 < |x| < \frac{1}{\sqrt{R}}.
\end{align*}
Por lo tanto:
\begin{equation}
\label{eq:Lim008}
\frac{1}{x^2} > R \ \Leftrightarrow \ 0 < |x| < \frac{1}{\sqrt{R}}.
\end{equation}

Esta última equivalencia nos garantiza que el número $\delta$ buscado es:
\[
\delta = \frac{1}{\sqrt{R}},
\]
pues, si se elige $x$ de modo que $0 < |x| < \delta$, por las equivalencia~(\ref{eq:Lim008}), se
verifica la desigualdad~(\ref{eq:Lim007}). Por lo tanto, hemos probado que:
\[
\limjc{\frac{1}{x^2}}{x}{0} = +\infty.
\]
\end{exemplo}

En el siguiente ejemplo, se requiere un trabajo mayor para encontrar el número $\delta$ que exige
la definición de límite infinito. Para encontrarlo, las siguientes reflexiones sobre este concepto
son de mucha ayuda.

Supongamos que
\[
\limjc{f(x)}{x}{a} = +\infty.
\]
Entonces, dado un número real $R > 0$, por la definición de límite, podemos asegurar la existencia
de un número $\delta > 0$ tal que
\[
f(x) > R
\]
para todo $x \in A = ]a - \delta, a + \delta[ - \{a\}$.

Como $R > 0$, podemos asegurar que $f(x) > 0$ para todo $x \in A$.

En resumen, si $f(x)$ tiende a $+\infty$ cuando $x$ tiende al número $a$, entonces $f(x)$ es
positiva, excepto, quizás, en $a$, en todos los puntos de un intervalo abierto centrado en $a$.

De manera similar, podemos concluir que, si $f(x)$ tiende a $-\infty$ cuando $x$ tiende al
número $a$, entonces $f(x)$ es negativa, excepto, quizás, en $a$, en todos los puntos de un
intervalo abierto centrado en $a$.

Resumamos estos resultados en el siguiente teorema:

\lteocal{teo}{\label{teo:LimInfPosNeg}%
Si
\[
\limjc{f(x)}{x}{a} = +\infty,
\]
existe $\delta > 0$ tal que
\[
f(x) > 0
\]
para todo $x \in ]a - \delta, a + \delta[ - \{a\}$.

Si
\[
\limjc{f(x)}{x}{a} = -\infty,
\]
existe $\delta > 0$ tal que
\[
f(x) < 0
\]
para todo $x \in ]a - \delta, a + \delta[ - \{a\}$.
}%fin de \lteocal

Esta característica de una función cuando tiende a más o menos infinito, reduce el espacio de
búsqueda del número $\delta$.

En efecto, en el caso de que se quiera demostrar que $f(x)$ tiende a $+\infty$, el análisis para la
búsqueda del número $\delta$ solo debe reducirse al subconjunto del dominio de $f$ en la que ésta
es positiva. En el caso de que sea $-\infty$, el análisis se realiza en el conjunto donde $f$ sea
negativa. En otras palabras, conocer los valores de $x$ donde $f(x)$ es positiva o es negativa es
de mucha ayuda a la hora de buscar el número $\delta$. Veamos cómo se puede hacer esto en el
siguiente ejemplo.


\begin{exemplo}[Solución]{%
Pruebe que $\displaystyle \lim_{x\to 2}\frac{(5-x)(1 - x)}{(x-2)^2}=-\infty$.}%
\def\f(x){\dfrac{(5-x)(1 - x)}{(x-2)^2}}

Sean $f(x)=\dfrac{(5-x)(1 - x)}{(x-2)^2}$ y $R<0$. Debemos hallar $\delta>0$ tal que
\begin{equation}
\label{eq:Lim009}
	f(x)<R,
\end{equation}
siempre que $0<|x-2|<\delta$ y $x\neq 2$.

Para encontrar el valor de $\delta$, analicemos la desigualdad~\ref{eq:Lim009}:
\[
\f(x) < R.
\]
Observemos que, como el factor $(x - 2)$ está en el denominador, si se verificara la condición
\[
0 < |x - 2| < \delta,
\]
entonces se verificaría la condición
\[
0 < (x - 2)^2 < \delta^2,
\]
de donde también se verificaría que
\begin{equation}
\label{eq:Lim010}
\frac{1}{(x - 2)^2} > \frac{1}{\delta^2}.
\end{equation}

Ahora, busquemos un intervalo centrado en $2$ donde $f(x)$ sea un número negativo. Como $(x - 2)^2
> 0$, entonces, en dicho intervalo, deberá ocurrir que
\[
(5 - x)(1 - x) < 0.
\]
Y, si encontráramos una constante $M < 0$ tal que
\[
(5 - x)(1 - x) < M
\]
en dicho entorno, la desigualdad~(\ref{eq:Lim010}) implicaría que
\begin{equation}
\label{eq:Lim011}
f(x) = \f(x) < \frac{M}{\delta^2},
\end{equation}
siempre que $0 < |x - 2| < \delta$ y $f(x) < 0$.

Entonces, si comparamos la desigualdad~(\ref{eq:Lim011}) con la desigualdad~(\ref{eq:Lim009}),
vemos que el número $\delta$ que hay que elegir es aquel garantice que
\[
\frac{M}{\delta^2} = R \yjc f(x) < 0.
\]

En resumen, lo que nos queda por hacer es encontrar:
\begin{enumerate}
\item un intervalo centrado en $2$ en el que $f(x) < 0$; y
\item una constante $M < 0$ tal que $(5 - x)(1 - x) < M$ en dicho entorno.
\end{enumerate}

Empecemos por encontrar los valores de $x \neq 2$ para los cuales $f(x) < 0$. Para ello
consideremos las siguientes equivalencias:
\begin{align*}
\frac{(5 - x)(1 - x)}{(x - 2)^2} < 0 &\Longleftrightarrow
(5 - x)(1 - x) < 0, \quad\text{pues}\ (x - 2)^2 > 0 \ \text{para todo}\ x\in\mathbb{R} \\
&\Longleftrightarrow
(x - 5)(x - 1) < 0 \\
&\Longleftrightarrow x \in \ ]1,5[.
\end{align*}
La verdad de la última equivalencia se puede determinar si se considera que el gráfico del
polinomio $(x -5)(x - 1)$ es una parábola convexa que corta el eje $x$ en los abscisas $1$ y $5$,
por lo que la parte de la parábola que está bajo el eje $x$ está entre $1$ y $5$, como se muestra
en el siguiente dibujo:
\begin{center}
\psset{xAxisLabel={},yAxisLabel={},plotpoints=1000}%
\def\f{x dup neg 5 add exch neg 1 add mul}

\begin{psgraph}[arrows=->,ticks=x](0,0)(-0.5,-4.5)(6,5.5){0.4\textwidth}{5cm}
  \uput[-90](6,0){$x$}%
  \uput[0](0,5.5){$y$}%

  \psplot{0}{5.5}{\f}%

\end{psgraph}

{\small El gráfico de $y = (5 - x)(1 - x)$}
\end{center}

En resumen, la función $f$ es negativa en el intervalo $]1, 5[ - \{2\}$. Es decir, vemos que
\[
f(x) < 0,
\]
siempre que $1 < x < 5$ y $x \neq 2$.

Ahora, encontremos la constante $M$. Como el límite es en $2$, elijamos valores de $x$ cercanos a
$2$. Por ejemplo, tomemos $x\neq 2$ tales que $|x-2| < \frac{1}{2}$. Esto equivale a
\[
-\frac{1}{2} < x - 2 < \frac{1}{2},
\]
de donde
\begin{equation}
\label{eq:Lim012}
\frac{3}{2} < x < \frac{5}{2}.
\end{equation}
Como para estos valores de $x$, $f(x) < 0$, si $x\neq 2$, entonces, busquemos $M$ en el intervalo
$\left]\frac{3}{2},\frac{5}{2}\right[$.

Para ello, definamos $g(x) = (5- x)(1 - x)$. Sabemos que, en el intervalo
$\left]\frac{3}{2},\frac{5}{2}\right[$, el gráfico de $g$ es una parábola, como se mostró
anteriormente. El mínimo de $g$ en este intervalo está en el punto medio de las dos raíces; es
decir, en $x = 3$. Por lo tanto, en el intervalo $\left]\frac{3}{2},3\right[$, y por ende en
$\left]\frac{3}{2},3\right[$, la función $g$ es decreciente, por lo que
\[
g(x) < g\left(\frac{3}{2}\right) = \left(5 - \frac{3}{2}\right)\left(1 - \frac{3}{2}\right) =
-\frac{7}{4}
\]
para todo $x \in\ \left]\frac{3}{2},\frac{5}{2}\right[$. Esto significa, entonces, que $M =
-\frac{7}{4}$.

Resumamos: si $x$ es tal que $|x - 2| < \frac{1}{2}$, entonces $(5 - x)(1 - x) < M = -\frac{7}{4}$.

Por otro lado, $\delta$ debe cumplir con
\[
\frac{M}{\delta^2} = R,
\]
de donde
\[
\delta^2 = \frac{M}{R},
\]
de donde, como $M < 0$ y $R < 0$, tenemos que $\delta$ puede ser elegido así:
\[
\delta = \sqrt{\frac{M}{R}} = \frac{1}{2}\sqrt{\frac{-7}{R}}.
\]

Por lo tanto, si elegimos $\delta$ tal que:
\begin{equation*}
	\delta = \min\left\{\frac{1}{2},\frac{1}{2}\sqrt{\frac{-7}{R}}\right\},
\end{equation*}
se verifica que
\[
\f(x) < R
\]
siempre que $0 < |x - 2| < \delta$.

Hemos probado, entonces, que
\[
\limjc{\f(x)}{x}{2} = -\infty.
\]
\end{exemplo}

La definición de límite lateral infinito es similar a la de límite infinito con la inclusión de la
condición de que $x$ sea o menor que $a$ o mayor que $a$. Así, la expresión
\[
\limjc{f(x)}{x}{a^+} = +\infty
\]
quiere decir que, dado cualquier número real positivo $R$, siempre existe un número $\delta > 0$
tal que
\[
f(x) > R
\]
siempre que $0 < x - a < \delta$.

Definiciones similares se tienen para el caso de que $x$ tiende al número $a$ por la izquierda.

Con estas definiciones, podemos expresar el hecho de que la masa de un cuerpo, según la teoría
especial de la relatividad de Einstein, crezca indefinidamente cuando su velocidad se acerca a la
de la luz de la siguiente manera:
\begin{equation}
\label{eq:LimMasaEinstein}
\limjc{\frac{m_0}{\sqrt{1 - \frac{v^2}{c^2}}}}{v}{c^-} = +\infty.
\end{equation}

Para probar que esta igualdad es verdadera, podríamos recurrir a la definición directamente. Sin
embargo, igual que ocurre con los límites ``comunes y corrientes'', vamos a utilizar propiedades de
los límites infinitos que nos permitirán calcular muchos límites, entre ellos el de la masa de un
cuerpo, conociendo algunos límites infinitos solamente.

\begin{teocal}\label{teo:LimUnoSobreCero}%
\[
\limjc{\frac{1}{x}}{x}{0^+} = +\infty \yjc \limjc{\frac{1}{x}}{x}{0^-} = -\infty.
\]
\end{teocal}

La demostración de este teorema es sencilla y se la deja para que el lector la realice.

El siguiente teorema es una generalización del anterior, y es la fuente de cálculo de muchos
límites, entre los que se encuentra el límite~(\ref{eq:LimMasaEinstein}). Pero, para una
formulación más compacta, necesitamos ampliar la idea de límites laterales al valor del límite.

De manera más precisa, si ocurre que
\[
L = \limjc{f(x)}{x}{a},
\]
y adicionalmente $f>L$ localmente cerca de $a$; es decir, si existe un número $r > 0$
tal que $f(x) > L$ para todo $x \in ]a-r,a+r[-{a}$, entonces diremos que ``$f(x)$ tiende a $L$ por la
derecha cuando $x$ tiende al número $a$'', y lo expresaremos simbólicamente así:
\[
\limjc{f(x)}{x}{a} = L^+\quad\text{o así}\quad f(x) \rightarrow L^+.
\]

Como un ejemplo, se tiene que
\[
\limjc{x^2}{x}{0} = 0^+,
\]
pues $x^2 > 0 $ para todo $x\neq 0$.

De manera similar hablaremos de que una función $f(x)$ ``tiende a $L$ por la izquierda'', y
escribiremos
\[
\limjc{f(x)}{x}{a} = L^-\quad\text{o así}\quad f(x) \rightarrow L^-,
\]
cuando $\displaystyle L = \limjc{f(x)}{x}{a}$ y exista $r > 0$ tal que $f(x) < L$ siempre que $x
\in ]a-r,a+r[-\{a\}$.

\begin{teocal}\label{teo:LimGeneralInfUnoSobreCero}%
Sean $I$, un intervalo abierto, $a\in I$ y $f$ una función real tal que $I \subset
\Dm(f) \cup \{a\}$. Entonces:
\begin{enumerate}
\item $\displaystyle\limjc{f(x)}{x}{a} = 0^+$ \ si y solo si \
    $\displaystyle\limjc{\frac{1}{f(x)}}{x}{a} = +\infty$.
\item $\displaystyle\limjc{f(x)}{x}{a} = 0^-$ \ si y solo si \
    $\displaystyle\limjc{\frac{1}{f(x)}}{x}{a} = -\infty$.
\end{enumerate}
\end{teocal}

Este teorema también es válido si los límites son laterales. En los ejercicios de esta sección, el
lector tendrá la oportunidad de probar esta afirmación.

Antes de estudiar la demostración de este teorema, vamos a utilizarlo para calcular el
límite~(\ref{eq:LimMasaEinstein}).

\begin{exemplo}[Solución]{%
Cálculo del límite
\[
\limjc{\frac{m_0}{\sqrt{1 - \frac{v^2}{c^2}}}}{v}{c^-}.
\]}%
Para poder utilizar la primera parte del teorema~(\ref{teo:LimUnoSobreCero}), primeramente
expresemos $m(v)$ de la siguiente manera:
\[
m(v) = \frac{1}{\frac{1}{m_0}\sqrt{1 - \frac{v^2}{c^2}}}.
\]
Ahora, podemos definir la función $f$ así:
\[
f(v) = \frac{1}{m_0}\sqrt{1 - \frac{v^2}{c^2}}
\]
para todo $v \in [0,c[$.

Lo que ahora tenemos que probar es que $f(v)$ tiende a $0^+$ cuando $x$ tiende a $c$ por la
izquierda.

Por un lado, tenemos que
\[
f(v) > 0
\]
para todo $v \in [0,c[$.

Por otro lado, cuando $v$ tiende a $c$ por la izquierda, entonces tenemos que:
\[
\limjc{\frac{v}{c}}{v}{c^-} = 1^-,
\]
pues $v < c$. Por lo tanto, podemos concluir que:
\[
\limjc{1 - \frac{v^2}{c^2}}{v}{c^-} = 0^+.
\]
Así que, obtenemos que:
\[
\limjc{\frac{1}{m_0}\sqrt{1 - \frac{v^2}{c^2}}}{v}{c^-} = 0^+.
\]

Por lo tanto, por el teorema~\ref{teo:LimUnoSobreCero}, podemos concluir que
\[
\limjc{m(v)}{v}{c^-} =\limjc{\frac{1}{\frac{1}{m_0}\sqrt{1 - \frac{v^2}{c^2}}}}{v}{c^-} =
+\infty.
\]
Es decir, cuando la velocidad de un cuerpo es cercana a la velocidad de la luz, su masa crece
indefinidamente.
\end{exemplo}

Probemos a continuación el primer numeral del teorema~\ref{teo:LimGeneralInfUnoSobreCero}. El
segundo se deja a que el lector lo realice como un ejercicio.
\begin{proof}
Supongamos que
\begin{equation}
\label{eq:Lim013}
\limjc{f(x)}{x}{a} = 0^+.
\end{equation}
Vamos a demostrar que
\[
\limjc{\frac{1}{f(x)}}{x}{a} = +\infty.
\]

Para ello, sea $R > 0$. Debemos hallar $\delta > 0$ tal que
\begin{equation}
\label{eq:Lim014}
\frac{1}{f(x)} > R
\end{equation}
siempre que $0 < |x - a| < \delta$.

La igualdad~(\ref{eq:Lim013}) implica, por un lado, que existe $\delta_1 > 0$ tal que
\begin{equation}
\label{eq:Lim015}
f(x) > 0
\end{equation}
siempre que $0 < |x - a| < \delta_1$.

Por otro lado, la igualdad~(\ref{eq:Lim013}) también implica que existe $\delta_2 > 0$ tal que
\begin{equation}
\label{eq:Lim016}
|f(x) - 0| = |f(x)| < \frac{1}{R}
\end{equation}
siempre que $0 < |x - a| < \delta_2$, ya que $\frac{1}{R} > 0$.

Por lo tanto, si tomamos $\delta = \min\{\delta_1,\delta_2\}$, entonces tenemos que las
desigualdades~(\ref{eq:Lim015}) y (\ref{eq:Lim016}) se verifican simultáneamente si $0 < |x - a| <
\delta$. Es decir, se verifica que
\[
0 < f(x) = |f(x)| < \frac{1}{R}
\]
siempre que $0 < |x - a| < \delta$.

Pero estas dos desigualdades últimas implican que
\[
\frac{1}{f(x)} > R
\]
siempre que $0 < |x - a| < \delta$.

En resumen, hemos probado que
$\displaystyle
\limjc{\frac{1}{f(x)}}{x}{a} = +\infty.
$
La demostración de que esta última igualdad implica la igualdad~(\ref{eq:Lim013}) es similar, por
lo que se deja al lector que la haga como un ejercicio.
\end{proof}

\subsection{Propiedades de los límites infinitos}
La definición de límites infinitos nos permite si un límite es $+\infty$ o $-\infty$, pero no nos
permite saber de antemano si un límite es o no infinito. Los límites infinitos poseen propiedades
que nos permiten determinar si una función diverge a partir de saber que ciertas funciones
divergen. En el teorema~\ref{teo:LimGeneralInfUnoSobreCero} se presentan dos de esas propiedades.
En el siguiente teorema, se reúnen algunas de las propiedades más útiles, que también son
verdaderas si los límites son laterales únicamente. Pero antes de enunciarlo, es necesario hablar
del concepto de \emph{función acotada} y su relación con el concepto de límite.

\begin{defical}[Función acotada]%
Sean $\funcjc{f}{\Dm(f)}{\mathbb{R}}$ y $A\subset \Dm(f)$. La función $f$ está:
\begin{enumerate}[leftmargin=*]
\item \emph{acotada superiormente en} $A$ si existe una constante $M$ tal que $f(x) < M$ para
    todo $x\in A$;

\item \emph{está acotada inferiormente en} $A$ si $f(x) > M$ para todo $x\in A$.

\item \emph{está acotada en $A$} si está acotada superiormente e inferiormente en $A$. Esto es equivalente a
    que exista un intervalo $]K,M[$ tal que $f(x) \in\ ]K,M[$. También es equivalente a la
    afirmación de que exista un número $P > 0$ tal que $|f(x)| < P$ para todo $x\in A$.
\end{enumerate}
Si $A = \mathbb{R}$, se dice, por ejemplo, ``$f$ está acotada \textit{siempre}'', en lugar de ``acotada en $\mathbb{R}$''.
\end{defical}

Veamos algunos ejemplos. La función $\funcjc{f}{\mathbb{R}}{\mathbb{R}}$ definida por
\[
f(x) = x
\]
\begin{enumerate}
\item está acotada en cualquier intervalo finito $[a,b]$ o $]a,b[$;
\item está acotada inferiormente, pero no superiormente en cualquier intervalo $[a,+\infty[$ o
    $]a,+\infty[$;
\item está acotada superiormente, pero no inferiormente en cualquier intervalo $]\!-\infty, a]$
    o $]\!-\infty,a[$.
\end{enumerate}

Las funciones $\sen$ y $\cos$ están siempre acotadas (es decir, están acotadas en $\mathbb{R}$), pues
\[
|\sen x| \leq 1 \yjc |\cos x| \leq 1
\]
para todo $x\in\mathbb{R}$.

A continuación, estudiemos la conexión entre el concepto de acotación y el de límite. Supongamos
que existe $L\in\mathbb{R}$ tal que $L = \displaystyle\limjc{f(x)}{x}{a}$. Sabemos, entonces, que
$f(x)$ puede estar tan cerca de $L$ como se quiera siempre que $x$ esté lo suficientemente cerca de
$a$. Esto significa que existirá un intervalo $I$ centrado en $a$ tal que $f$ esté acotada en $I -
\{a\}$. Probemos esta afirmación.

Sea $\epsilon > 0$. Existe, entonces, $\delta > 0$ tal que $|f(x) - L| < \epsilon$ siempre que $0 <
|x - a| < \delta$. Por lo tanto
\[
L -\epsilon < f(x) < L + \epsilon
\]
para todo $x\neq a$ tal que $a - \delta < x < a + \delta$.

Si definimos $K = L - \epsilon$, $M = L + \epsilon$ e $I = ]a - \delta, a + \delta[$, entonces
$f(x) \in\ ]K,M[$ para todo $x\in I - \{a\}$.

Resumamos estos razonamientos en el siguiente teorema.

\begin{teocal}\label{teo:LimLimAcotada}%
Si $L = \displaystyle\limjc{f(x)}{x}{a}$, entonces existe un intervalo abierto $I$, centrado en $a$ tal
que $f$ está acotada en $I - \{a\}$. Es decir, $f$ está acotada localmente cerca de $a$.
\end{teocal}

En el caso de que una función tienda a $+\infty$ o a $-\infty$, la función no está acotada
superiormente, en el primer caso, e inferiormente en el segundo. La demostración, que es similar a
la del caso de los límites finitos, la dejamos como un ejercicio para el lector. Enunciemos este
resultado como el siguiente teorema.

\begin{teocal}
Si $\displaystyle\limjc{f(x)}{x}{a} = +\infty$, entonces $f$ no está acotada superiormente en
ningún intervalo centrado en $a$. En cambio, si $\displaystyle\limjc{f(x)}{x}{a} = -\infty$, la
función $f$ no está acotada inferiormente en ningún intervalo centrado en $a$.
\end{teocal}

Ahora ya podemos formular algunas de las propiedades de los límites infinitos.

\begin{teocal}[Propiedades de límites infinitos]\label{teo:LimPropLimInf}%
Sean $f$ y $g$ dos funciones reales. Entonces:
\begin{enumerate}[leftmargin=*]
\item Si $\displaystyle\limjc{f(x)}{x}{a} = +\infty$ y $\displaystyle\limjc{g(x)}{x}{a} = +\infty $, entonces:
    \[
      \limjc{[f(x) + g(x)]}{x}{a} = +\infty \yjc
      \limjc{[f(x)g(x)]}{x}{a} = +\infty.
    \]
\item Si $\displaystyle\limjc{f(x)}{x}{a} = +\infty$ y $\lambda < 0$, entonces:
    \[
      \limjc{[\lambda f(x)]}{x}{a} = -\infty.
    \]
\item Si $\displaystyle\limjc{f(x)}{x}{a} = +\infty$ y $g$ está acotada inferiormente localmente cerca de $a$, entonces:
    \[
      \limjc{[f(x) + g(x)]}{x}{a} = +\infty
    \]
\item Si $\displaystyle\limjc{f(x)}{x}{a} = +\infty$ y $g$ está acotada inferiormente por un número positivo localmente cerca de $a$, entonces:
    \[
      \limjc{[f(x)g(x)]}{x}{a} = +\infty.
    \]
\item Si $f$ está acotada inferiormente por un número positivo localmente cerca de $a$ y $\displaystyle\limjc{g(x)}{x}{a} = 0^+$, entonces:
    \[
      \limjc{\frac{f(x)}{g(x)}}{x}{a} = +\infty.
    \]
\item Si $f$ está acotada localmente cerca de $a$ y $\displaystyle\limjc{g(x)}{x}{a} = +\infty$,
    entonces:
    \[
      \limjc{\frac{f(x)}{g(x)}}{x}{a} = 0.
    \]
\end{enumerate}
\end{teocal}

Se pueden formular propiedades similares cuando las funciones tienden a $-\infty$. Esto se hará en
el siguiente capítulo. En ese capítulo también se estudiarán algunas de las demostraciones. Sin
embargo, el lector interesado en adquirir una comprensión más profunda del concepto de límite,
debería realizar por sí mismo estas demostraciones.

A continuación, veamos algunos ejemplos del uso de las propiedades enunciadas.

\begin{exemplo}[Solución]{%
Calcular $\displaystyle\limjc{\left(\frac{1}{x^2} + \sen x\right)}{x}{0}$.}%
Sean $f(x) = \displaystyle \frac{1}{x^2}$ y $g(x) = \sen x$. Entonces, por el
teorema~\ref{teo:LimGeneralInfUnoSobreCero}, tenemos que
\[
\limjc{f(x)}{x}{0} = +\infty
\]
ya que
\[
\limjc{x^2}{x}{0} = 0^+.
\]
Por otro lado, para todo $x$, se tiene que $\sen x \geq -1$; es decir, la función $g$ está acotada
inferiormente en cualquier intervalo que contenga a $0$. Por lo tanto, por la tercera propiedad del
teorema~\ref{teo:LimPropLimInf}, podemos afirmar que:
\[
\limjc{\left(\frac{1}{x^2} + \sen x\right)}{x}{0} = \limjc{f(x) + g(x)}{x}{0} = +\infty.
\]
\end{exemplo}

\begin{exemplo}[Solución]{%
Calcular $\displaystyle\limjc{\frac{x^2 + 1}{x^3 - 1}}{x}{1^+}$.}%
Sean $f(x) = x^2 + 1$ y $\displaystyle g(x) = \frac{1}{x^3 - 1}$.

Por un lado, para todo $x\in\mathbb{R}$, se verifica que:
\[
f(x) = x^2 + 1 \geq 1.
\]
Por lo tanto, $f$ está acotada inferiormente por un número positivo en cualquier intervalo cuyo
extremo inferior sea el número $1$.

Por otro lado, dado que
\[
\limjc{x^3 - 1}{x}{1^+} = 0^+,
\]
entonces
\[
\limjc{g(x)}{x}{1^+} = \limjc{\frac{1}{x^3 - 1}}{x}{1^+} = +\infty.
\]

De modo que, si aplicamos la cuarta propiedad enunciada en el teorema~\ref{teo:LimPropLimInf},
podemos concluir que
\[
\limjc{\frac{x^2 + 1}{x^3 - 1}}{x}{1^+} = \limjc{f(x)g(x)}{x}{1^+} = +\infty.
\]

Podemos llegar a la misma conclusión de otra manera. En efecto, dado que $x\neq 0$, podemos
escribir lo siguiente:
\[
\frac{x^2 + 1}{x^3 - 1} = \frac{\displaystyle\frac{x^2}{x^2} + \frac{1}{x^2}}%
{\displaystyle\frac{x^3}{x^2} - \frac{1}{x^2}} =
\frac{\displaystyle 1 + \frac{1}{x^2}}{\displaystyle x - \frac{1}{x^2}},
\]
podemos utilizar la cuarta propiedad del teorema~\ref{teo:LimPropLimInf} para calcular el límite.

Para ello, definamos $f(x) = 1 + \frac{1}{x^2}$ y $g(x) = x - \frac{1}{x^2}$. Entonces tenemos que
$f$ está acotada inferiormente por el número $1$ en un intervalo abierto cuyo extremo inferior es
el número $1$, ya que $\displaystyle f(x) = 1 + \frac{1}{x^2} > 1$ pues $\displaystyle\frac{1}{x^2}
> 0$. Adicionalmente, tenemos que
\[
\limjc{g(x)}{x}{1^+} = 1 - 1 = 0
\]
y, como
\[
x - \frac{1}{x^2} > 0,
\]
pues
\begin{align*}
x > 1 &\Longrightarrow x^2 > 1 \\
&\Longrightarrow \frac{1}{x^2} < 1 \\
&\Longrightarrow -\frac{1}{x^2} > - 1 \\
&\Longrightarrow x - \frac{1}{x^2} > 0,
\end{align*}
tenemos que
\[
\limjc{g(x)}{x}{1^+} = 0^+.
\]

Entonces, si aplicamos la cuarta propiedad obtenemos que:
\[
\limjc{\frac{x^2 + 1}{x^3 - 1}}{x}{1^+} = \limjc{\frac{f(x)}{g(x)}}{x}{1^+} = +\infty.
\]
\end{exemplo}

Cuando $\displaystyle\limjc{f(x)}{x}{a} = \limjc{g(x)}{x}{a} = + \infty$, el
teorema~\ref{teo:LimPropLimInf} no dice nada sobre $\displaystyle\limjc{[f(x) - g(x)]}{x}{a}$ ni
sobre $\displaystyle\limjc{\frac{f(x)}{g(x)}}{x}{a}$. Los siguientes ejemplos nos van a decir por
qué.

\begin{exemplo}[Solución]{%
Calcular los límites
\[
\limjc{f(x)}{x}{a^+}, \quad \limjc{g(x)}{x}{a^+} \yjc \limjc{[f(x) - g(x)]}{x}{a^+}
\]
si:
\begin{enumerate}
\item $f(x) = \displaystyle\sqrt{\frac{x + 1}{x}}$ y $g(x) = \displaystyle\sqrt{\frac{1}{x}}$;
    $a = 0$.
\item $f(x) = \displaystyle\frac{1}{(x - 1)^3}$ y $g(x) = \displaystyle\frac{1}{(x - 1)^2}$; $a
    = 1$.
\end{enumerate}}
\begin{enumerate}[leftmargin=*]
\item Tenemos que $\displaystyle f(x) = \sqrt{\frac{x + 1}{x}} = \sqrt{1 + \frac{1}{x}}$, que
    $\displaystyle\limjc{\frac{1}{x}}{x}{0^+} = + \infty$ y que $\displaystyle\limjc{1 +
    \frac{1}{x}}{x}{0^+} = + \infty$ (por el numeral tres del teorema~\ref{teo:LimPropLimInf}).
    Es decir:
    \begin{equation}
    \label{eq:Lim017}
    \limjc{\frac{x + 1}{x}}{x}{o^+} = +\infty.
    \end{equation}

    Por otro lado, es fácil demostrar que si $\displaystyle\limjc{\phi(x)}{x}{a} = +\infty$,
    entonces\footnote{En los ejercicios de esta sección, se propone al lector la demostración
    de esta propiedad.} $\displaystyle\limjc{\sqrt{\phi(x)}}{x}{a} = +\infty$. Si aplicamos
    esta propiedad en~(\ref{eq:Lim017}), entonces tenemos que
    \[
      \limjc{f(x)}{x}{0^+} = \limjc{\sqrt{\frac{x+1}{x}}}{x}{0^+} = +\infty.
    \]

    Un procedimiento similar nos permite afirmar que
    \[
      \limjc{g(x)}{x}{0^+} = \limjc{\sqrt{\frac{1}{x}}}{x}{0^+} = +\infty.
    \]

    Ahora calculemos el límite de la diferencia $[f(x) - g(x)]$.

    Para empezar:
    \[
    f(x) - g(x) = \sqrt{1 + \frac{1}{x}} - \sqrt{\frac{1}{x}}
    = \frac{1}{\sqrt{1 + \frac{1}{x}} + \sqrt{\frac{1}{x}}}.
    \]
    Puesto que
    \[
      \limjc{\sqrt{1 + \frac{1}{x}}}{x}{0^+} = +\infty \yjc
      \limjc{\sqrt{\frac{1}{x}}}{x}{0^+} = +\infty,
    \]
    por el primer numeral del teorema~\ref{teo:LimPropLimInf}, tenemos que
    \[
      \limjc{\left(\sqrt{1 + \frac{1}{x}} + \sqrt{\frac{1}{x}}\right)}{x}{0^+} = +\infty,
    \]
    de donde, por el teorema~\ref{teo:LimGeneralInfUnoSobreCero} (página
    \pageref{teo:LimGeneralInfUnoSobreCero}), podemos concluir que:
    \[
      \limjc{[f(x) - g(x)]}{x}{0^+} =
      \limjc{\left(\frac{1}{\sqrt{1 + \frac{1}{x}} + \sqrt{\frac{1}{x}}}\right)}{x}{0^+} = 0.
    \]

\item Dado que, para $x > 1$, tenemos que $(x - 1)^3 > 0$, entonces $\displaystyle\limjc{(x -
    1)^3}{x}{1^+} = 0^+$. Por lo tanto, por el teorema~\ref{teo:LimGeneralInfUnoSobreCero},
    podemos concluir que:
    \[
      \limjc{f(x)}{x}{1^+} = \limjc{\frac{1}{(x - 1)^3}}{x}{1^+} = +\infty.
    \]

    Un razonamiento similar nos conduce a concluir que:
    \[
      \limjc{g(x)}{x}{1^+} = \limjc{\frac{1}{(x - 1)^2}}{x}{1^+} = +\infty.
    \]

    Por último, puesto que
    \[
      f(x) - g(x) = \frac{1}{(x - 1)^3} - \frac{1}{(x - 1)^2} = (2 - x)\frac{1}{(x - 1)^3},
    \]
    tenemos que, por la cuarta propiedad del teorema~\ref{teo:LimPropLimInf}, tenemos que
    \[
      \limjc{[f(x) - g(x)]}{x}{1^+} = \limjc{(2 - x)\frac{1}{(x - 1)^3}}{x}{1^+} = +\infty,
    \]
    ya que, para $x \in\ ]1, \frac{3}{2}[$, tenemos que
    \[
        2 - x > \frac{1}{2},
    \]
    es decir, $(2 - x)$ está acotado inferiormente por un número positivo.
\end{enumerate}
\end{exemplo}

En el primer límite de este ejemplo, obtenemos que el límite de la diferencia de dos funciones que
tienden a $+\infty$ es $0$; en el segundo, en cambio, el límite es $+\infty$. Y hay casos en que
ese límite puede ser un número real diferente de $0$, $-\infty$, etcétera. En otras palabras, no
podemos concluir nada sobre el límite de la diferencia; cada caso deberá ser analizado con sus
particularidades.

Ocurre algo similar para el caso de la división de dos funciones que tienden a $+\infty$. El
siguiente ejemplo ilustra lo dicho.

\begin{exemplo}[Solución]{%
Calcular el límite $\displaystyle\limjc{\frac{f(x)}{g(x)}}{x}{0^+}$ si:
\begin{multicols}{3}
\begin{enumerate}
\item $f(x) = g(x) = \displaystyle\frac{1}{x}$.
\item $f(x) = \displaystyle\frac{1}{x}$, $g(x) = \displaystyle\frac{1}{x^2}$.
\item $f(x) = \displaystyle\frac{1}{x^2}$, $g(x) = \displaystyle\frac{1}{x}$.
\end{enumerate}
\end{multicols}
}%
En los tres ejemplos, tenemos que:
\[
\limjc{f(x)}{x}{0^+} = \limjc{g(x)}{x}{0^+} = +\infty.
\]
Veamos lo que sucede con el límite del cociente entre $f(x)$ y $g(x)$ en cada caso.
\begin{enumerate}[leftmargin=*]
\item Tenemos que $\displaystyle\frac{f(x)}{g(x)} = 1$. Por lo tanto,
    $\displaystyle\limjc{\frac{f(x)}{g(x)}}{x}{0^+} = 1.$

\item Tenemos que $\displaystyle\frac{f(x)}{g(x)} = \dfrac{\dfrac{1}{x}}{\dfrac{1}{x^2}} = x.$
    Por lo tanto: $\displaystyle\limjc{\frac{f(x)}{g(x)}}{x}{0^+} = 0.$

\item Tenemos que $\displaystyle\frac{f(x)}{g(x)} = \dfrac{\dfrac{1}{x^2}}{\dfrac{1}{x}} =
    \frac{1}{x}$. Por lo tanto, $\displaystyle\limjc{\frac{f(x)}{g(x)}}{x}{0^+} = +\infty$.
\end{enumerate}
\end{exemplo}

Los límites de diferencias y cocientes de funciones que tienden a $+\infty$ se suelen representar
por $+\infty - \infty$ y $\frac{+\infty}{+\infty}$. Puesto que según las funciones, estos límites
pueden tomar cualquier número real, o ser infinitos, se los denomina \emph{formas indeterminadas},
y cada caso deberá ser resuelto de modo particular.

Estas formas indeterminadas ya aparecieron anteriormente; es el caso del cociente de dos funciones
que convergen cada una a $0$. Por ejemplo, tenemos los límites:
\[
\limjc{\frac{\sen x}{x}}{x}{0} = 1 \yjc \limjc{\frac{1 - \cos x}{x^2}}{x}{0} = \frac{1}{2}.
\]
Uno de los resultados estudiados en este capítulo que ayuda a calcular los límites del tipo
$\frac{0}{0}$ es el teorema~\ref{eq:limitegeneral} enunciado \vpageref{eq:limitegeneral}.

\subsection{Límites al infinito}
En las ciencias de la computación, para un mismo problema, se han desarrollado diversos algoritmos
para resolverlo. Por ejemplo, para ordenar un conjunto de datos (los números de cédula de los
ciudadanos empadronados para una elección), existen el ordenamiento por inserción, Quicksort,
Mergesort, entre otros.

La complejidad computacional es una rama de las ciencias de la computación dedicada a determinar
aproximadamente el número de instrucciones que un algoritmo realiza para una determinada tarea, y a
desarrollar nuevos algoritmos que hagan las mismas tareas en un número menor de instrucciones.

Por ejemplo, para multiplicar dos matrices cuadradas de orden $n$, existen al menos dos algoritmos:
el que utiliza la definición de multiplicación y que realiza
\[
  f(n) = 2n^3 - n^2
\]
operaciones (sumas, restas y multiplicaciones), y el algoritmo de Winograd (1970), que utiliza
\[
  g(n) = 2n^3 + 3n^2 - 2n
\]
operaciones.

La complejidad computacional busca determinar cuál de los dos algoritmos es más eficiente, para lo
cual se determina la relación entre $f(n)$ y $g(n)$ cuando $n$ es un valor grande. De manera más
precisa, el problema se plantea en términos de límites: ?`a dónde tiende el cociente $\displaystyle
\frac{f(n)}{g(n)}
$
cuando $n$ crece indefinidamente?

En otras palabras, la pregunta que se formula es: ?`existe un número real tan cerca como se quiera
del cociente $\displaystyle\frac{f(n)}{g(n)}$ para todo $n$ a partir de un cierto $n_0$?

El concepto de límite al infinito permite responder esta pregunta.

\begin{defical}[Límites al infinito]
$\displaystyle L=\lim_{x \to +\infty}f(x)$ si y solo si para todo $\epsilon > 0$, existe $R > 0$
tal que
\[
|f(x) - L| < \epsilon,
\]
siempre que $x\in\Dm(f)$ y $x > R$.

Análogamente: $\displaystyle L=\lim_{x \to -\infty}f(x)$ si y solo si para todo $\epsilon > 0$,
existe $M < 0$ tal que
\[
|f(x) - L| < \epsilon,
\]
siempre que $x\in\Dm(f)$ y $x < M$.
\end{defical}

De la definición se desprende que
\[
\limjc{x}{x}{+\infty} = +\infty \yjc \limjc{x}{x}{-\infty} = -\infty.
\]
Y de estos resultados, podemos demostrar que:
\[
\limjc{x^n}{x}{+\infty} = +\infty
\]
y
\[
\limjc{x^n}{x}{-\infty} =
\begin{cases}
+\infty & \text{si $n$ es par,} \\
-\infty & \text{si $n$ es impar}.
\end{cases}
\]

En el caso de que $x$ tienda a $+\infty$, estamos exigiendo que exista un intervalo del tipo
$]R,+\infty[$ con $R > 0$ que esté contenido en el dominio de la función $f$; esto se traduce en
que podemos excluir el caso $x \leq 0$ cuando analicemos un límite donde $x$ tiende a $+\infty$. De
manera similar, cuando $x$ tiende a $-\infty$, el dominio de la función debe contener un intervalo
del tipo $]-\infty,M[$ con $M < 0$. Por lo tanto, podemos excluir el caso $x\geq 0$ cuando se
analice un límite donde $x$ tiende a $-\infty$.

\begin{exemplo}[Solución]{%
Mostrar que $\displaystyle\limjc{\frac{1}{x}}{x}{+\infty} = 0$ y
$\displaystyle\limjc{\frac{1}{x}}{x}{-\infty} = 0$.
}%
\begin{enumerate}[leftmargin=*]
\item Sea $\epsilon > 0$. Debemos encontrar $R > 0$ tal que
      \begin{equation}
      \label{eq:Lim018}
        \left\lvert\frac{1}{x} - 0\right\rvert < \epsilon
      \end{equation}
      para todo $x > R$.

      Para ello, procedamos de manera similar al caso de los límites finitos. Analicemos, pues,
      la desigualdad~(\ref{eq:Lim018}).

      Como $x$ tiende a $+\infty$, supongamos que $x > 0$. Entonces, se tienen las siguientes
      equivalencias:
      \begin{align*}
      \left|\frac{1}{x} - 0\right| < \epsilon &\Longleftrightarrow
      \frac{1}{x} < \epsilon \\
      &\Longleftrightarrow x > \frac{1}{\epsilon}.
      \end{align*}
      Por lo tanto, si definimos $R = \frac{1}{\epsilon}$, obtendremos la
      desigualdad~(\ref{eq:Lim018}) siempre que $x > R$.

\item Sea $\epsilon > 0$. Debemos hallar $M < 0$ tal que
      \begin{equation}
      \label{eq:Lim019}
        \left\lvert\frac{1}{x} - 0\right\rvert < \epsilon
      \end{equation}
      para todo $x < M$.

      Como $x$ tiende a $-\infty$, supongamos que $x < 0$. Dado que
      \[
        \left\lvert\frac{1}{x} - 0\right\rvert < \epsilon \Longleftrightarrow
        -\frac{1}{x} < \epsilon \Longleftrightarrow x > -\frac{1}{\epsilon},
      \]
      si definimos $M = -\frac{1}{\epsilon}$, la desigualdad~(\ref{eq:Lim019}) es verdadera
      para todo $x > M$.
\end{enumerate}
\end{exemplo}

En la sección anterior se definió $\displaystyle\limjc{f(x)}{x}{a} = L^+$ y
$\displaystyle\limjc{f(x)}{x}{a} = L^-$. Podemos extender estas definiciones cuando, en lugar de
$a$ tenemos $+\infty$ o $-\infty$. La única diferencia está que, $f(x)$ debe ser mayor que $L$ o
menor que $L$ en un intervalo del tipo $]R,+\infty[$ con $R > 0$ cuando $x$ tiende a $+\infty$, y
en un intervalo del tipo $]-\infty,M[$ con $M < 0$ cuando $x$ tiende a $-\infty$.

Con estas extensiones, podemos reformular los dos límites del ejemplo de la siguiente manera:
\[
\limjc{\frac{1}{x}}{x}{+\infty} = 0^+ \yjc \limjc{\frac{1}{x}}{x}{-\infty} = 0^-.
\]

Examinemos un ejemplo más.

\begin{exemplo}[Solución]{%
Demostrar que:
\[
1 = \lim_{x \to +\infty}\frac{x+1}{x-1}.
\]
}%

Sea $\epsilon > 0$. Buscamos un número $R > 0$ tal que, si definimos
\[
f(x) = \frac{x+1}{x-1},
\]
se verifique que
\begin{equation}
\label{eq:Lim020}
\left|f(x)-1\right| < \epsilon
\end{equation}
siempre que $x > R$.

Podemos suponer que $x > 0$; más aún, podemos suponer que $x > 1$, así $f(x) > 1$, con lo cual
restringiremos el análisis a un intervalo donde $f$ es positiva.

Por otro lado, dado que:
\begin{equation*}
	f(x) - 1 = \frac{x+1}{x-1}-1 = \frac{x+1-x+1}{x-1} = \frac{2}{x - 1} > 1,
\end{equation*}
pues $x > 1$, tenemos que
\[
|f(x) - 1| = \frac{2}{x - 1}.
\]

Por lo tanto, para que se verifique la desigualdad~(\ref{eq:Lim020}), es suficiente que se cumpla
lo siguiente:
\[
\frac{2}{x - 1} < \epsilon.
\]
Pero:
\begin{align*}
\frac{2}{x - 1} < \epsilon &\Longleftrightarrow \frac{x - 1}{2} > \frac{1}{\epsilon} \\
&\Longleftrightarrow x - 1 > \frac{2}{\epsilon} \\
&\Longleftrightarrow x > 1 + \frac{2}{\epsilon}.
\end{align*}
Es decir, para que se verifique la desigualdad~(\ref{eq:Lim020}), es suficiente que $x > 1$ y que
\[
x > 1 + \frac{2}{\epsilon}.
\]

Dado que $\epsilon > 0$, entonces
\[
1 + \frac{2}{\epsilon} > 1,
\]
por lo que podemos elegir $R = 1 + \frac{2}{\epsilon}$ para estar seguros de que la
desigualdad~(\ref{eq:Lim020}) sea verdadera para todo $x > R$.

Podemos también afirmar que
\[
\limjc{\frac{x + 1}{x - 1}}{x}{+\infty} = 1^+,
\]
ya que $f(x) > 1$ para $x > 1$.
\end{exemplo}

Igual a lo que sucede con los límites finitos y los infinitos, existen propiedades de los límites
al infinito que nos permiten calcular los límites de muchas funciones a partir de ciertos límites
que ya nos son conocidos. Estas propiedades son similares a las de los límites finitos: el límite
de una suma, resta, multiplicación, división, raíz, composición, es la suma, resta, etcétera, de
los límites cuando estos existen con las correspondientes restricciones para el caso de la división
y las raíces pares. En otras palabras, en todos los teoremas que se enuncian las propiedades de los
límites finitos se pueden sustituir el número $a$ (a donde tiende $x$) por $+\infty$ o $-\infty$.
En aquellas propiedades en las que se exige un comportamiento de $f$ en un intervalo centrado en $a$,
en el caso de los límites al infinito, ese comportamiento deberá suceder en un intervalo del tipo
$]R,+\infty[$ con $R > 0$ o $]-\infty,M[$ con $M < 0$.

Las demostraciones son dejadas para que el lector las realice, pues no ofrecen ninguna dificultad y
son, más bien, un buen entrenamiento para su comprensión del concepto de límite. A continuación, se
ofrecen algunos ejemplos del uso de las propiedades mencionadas.

\begin{exemplo}[Solución]{%
Calcular $\displaystyle\limjc{\frac{3x + 1}{2x - 5}}{x}{-\infty}$.
}%
Puesto que $x\neq 0$, tenemos que
\[
\frac{3x + 1}{2x - 5} = \dfrac{3 + \dfrac{1}{x}}{2 - \dfrac{5}{x}}
\]
y $\displaystyle\limjc{\frac{1}{x}}{x}{-\infty} = 0$, entonces:
\begin{align*}
\limjc{\frac{3x + 1}{2x - 5}}{x}{-\infty} &=
\limjc{\dfrac{3 + \dfrac{1}{x}}{2 - \dfrac{5}{x}}}{x}{-\infty} \\
&= \frac{3 + 0}{2 - 0} = \frac{3}{2}.
\end{align*}
\end{exemplo}

\begin{exemplo}[Solución]{%
Estudiemos la relación entre las complejidades computacionales de los dos algoritmos para
multiplicar matrices dadas por las funciones
\[
f(x) = 2x^3 - x^2 \yjc g(x) = 2x^3 + 3x^2 - 2x.
\]
}%
Para ello, vamos a calcular el límite
\[
\limjc{\frac{f(x)}{g(x)}}{x}{+\infty} = \limjc{\frac{2x^3 - x^2}{2x^3 + 3x^2 - 2x}}{x}{+\infty}.
\]

Puesto que:
\[
\frac{2x^3 - x^2}{2x^3 + 3x^2 - 2x} = \dfrac{2 - \dfrac{1}{x}}{2 + 3\dfrac{1}{x} - 2\dfrac{1}{x^2}}
\]
y que
\[
\limjc{\frac{1}{x}}{x}{+\infty} = \limjc{\frac{1}{x^2}}{x}{+\infty} = 0,
\]
podemos concluir que:
\[
\limjc{\frac{2x^3 - x^2}{2x^3 + 3x^2 - 2x}}{x}{+\infty} = \frac{2 - 0}{2 + 3\times 0 - 2\times 0}
= 1.
\]
Es decir: $\displaystyle\limjc{\frac{f(x)}{g(x)}}{x}{+\infty} = 1$.

?`Cómo se interpreta este resultado?
Sea $\epsilon > 0$. Entonces, existe $R > 0$ tal que
\[
\left\lvert\frac{f(x)}{g(x)} - 1\right\rvert < \epsilon
\]
para todo $x > R$. Es decir,
$
\displaystyle
1 - \epsilon < \frac{f(x)}{g(x)} < 1 + \epsilon
$
siempre que $x > R$. Más aún, dado que $g(x) > 0$ para todo $x > 0$, se verifica para todo $x > R$:
\[
(1 - \epsilon)g(x) < f(x) < (1 + \epsilon)g(x).
\]


Estas dos desigualdades nos dicen algo importante: a partir de un cierto orden para las matrices
($x$ representa lo que $n$ en el ejemplo de inicio de esta sección), $f(x)$ está acotada por arriba
y por abajo por $g(x)$ multiplicada por sendas constantes que, si elegimos $\epsilon$ muy pequeño
(algo que podemos hacer a nuestro gusto), dichas constantes son cercanas a $1$. En otras palabras,
los valores de $f(x)$ y de $g(x)$ son similares, por lo que, cuando $x$ es grande, el número de
operaciones que realizan ambos algoritmos no difieren de manera significativa. En este sentido,
ambos algoritmos son o igual de buenos o igual de malos.
\end{exemplo}

Para terminar esta sección, podemos reunir los límites infinitos y al infinito. De manera más
precisa, podemos definir los siguientes límites:
\begin{align*}
\lim_{x \to +\infty}f(x) & = +\infty & \lim_{x \to +\infty}f(x) & = -\infty \\
\lim_{x \to -\infty}f(x) & = +\infty & \lim_{x \to -\infty}f(x) & = -\infty
\end{align*}

Por ejemplo: $\displaystyle \lim_{x \to -\infty}f(x)$ se tiene cuando para todo $R > 0$, existe $M
> 0$ tal que
$\displaystyle
f(x) > R
$
para todo $x\in\Dm(f)$ y $x < -M$.

Como un buen ejercicio, el lector debería formular las restantes definiciones.


La propiedades de estos límites se pueden expresar a través de las propiedades de los límites
infinitos y los límites al infinito. Por ejemplo, si
\[
\limjc{f(x)}{x}{+\infty} = -\infty,
\]
entonces
\[
\limjc{\frac{1}{f(x)}}{x}{+\infty} = 0^-.
\]
O, si
\[
\limjc{f(x)}{x}{-\infty} = +\infty \yjc \limjc{g(x)}{x}{-\infty} = +\infty,
\]
entonces:
\[
\limjc{[f(x)+g(x)]}{x}{-\infty} = +\infty \yjc \limjc{f(x)g(x)}{x}{-\infty} = +\infty.
\]

Por supuesto, también son indeterminados límites del tipo
\[
\limjc{[f(x) - g(x)]}{x}{+\infty} \yjc \limjc{\frac{f(x)}{g(x)}}{x}{+\infty}
\]
si tanto $f(x)$ como $g(x)$ tienden, por ejemplo, a $-\infty$.

En los ejercicios de esta sección, se pide demostrar algunas de las propiedades de estos límites.

\newpage
\subsection{Ejercicios}
\begingroup
\small
\begin{multicols}{2}
\begin{enumerate}[leftmargin=*]
\item Mediante la definición correspondiente, demuestre que:
            \begin{enumerate}[leftmargin=*]
            \item $\displaystyle \lim_{x\to -\infty}\frac{2x-1}{x+1}=2$.
            \item $\displaystyle \lim_{x\to +\infty}\frac{x+2}{x-1}=1$.
            \item $\displaystyle \lim_{x\to 1}\frac{(2-x)(2x-1)}{|x-1|}=+\infty$.
            \item $\displaystyle \lim_{x\to 0}\frac{x-2}{x^2+|x|}=-\infty$.
            \item $\displaystyle \limjc{\frac{(2x - 1)(7 - 2x)}{|x^2 - 3x + 2|}}{x}{2} =
                +\infty$.
            \item $\displaystyle \limjc{\frac{\sen x}{x}}{x}{-\infty} = 0$.
            \item $\displaystyle \limjc{\frac{x^2 - 2x}{|x^2 - 1|}}{x}{1} = -\infty$.
            \item $\displaystyle \limjc{\frac{2x^2 - 5x + 1}{x^2 + 1}}{x}{+\infty} = 2$.
            \item $\displaystyle \limjc{(1 - x^2)}{x}{+\infty} = -\infty$.
            \item $\displaystyle \limjc{\frac{-5 + 4x}{x^2 - 5x + 4}}{x}{1^+} =
                +\infty$.
             \end{enumerate}

\item Calcule los límites dados
            \begin{enumerate}
            \item $\displaystyle \lim_{x\to 1^+}\frac{1}{x^2-3x+2}$.
            \item $\displaystyle \lim_{x\to 1^-}\frac{x}{x^2-3x+2}$.
            \item $\displaystyle \lim_{x\to -\infty}\frac{x-2}{x^2+3}$.
            \item $\displaystyle \lim_{x\to +\infty}\frac{x-2}{x^2+3}$.
            \item $\displaystyle \lim_{x\to -\infty}\frac{x+1}{|x+1|}$.
            \item $\displaystyle \lim_{x\to +\infty}\frac{x+1}{|x+1|}$.
            \item $\displaystyle \limjc{\frac{(2x - 1)(4 - 3x)}{|x - 1|}}{x}{1}$.
            \item $\displaystyle \limjc{\frac{2x - 5}{|x - 2||x - 1|}}{x}{2}$.
            \end{enumerate}

\item Demostrar las siguientes propiedades de límites infinitos y al infinito:
      \begin{enumerate}[leftmargin=*]
      \item Si $\displaystyle \limjc{f(x)}{x}{+\infty} = L > 0$, entonces
      \[
          \limjc{\sqrt{f(x)}}{x}{+\infty} = \sqrt{L}.
      \]

      \item Si $\displaystyle \limjc{f(x)}{x}{-\infty} = -\infty$, entonces
      \[
          \limjc{\frac{1}{f(x)}}{x}{-\infty} = 0^-.
      \]

      \item Si $f > M > 0$ y $\displaystyle \limjc{g(x)}{x}{+\infty} = 0^+$, entonces
      \[
          \limjc{\frac{f(x)}{g(x)}}{x}{+\infty} = +\infty.
      \]

      \item Si $f$ está acotada y $\displaystyle \limjc{g(x)}{x}{+\infty} = +\infty$,
          entonces
      \[
            \limjc{\frac{f(x)}{g(x)}}{x}{+\infty} = 0.
      \]
      \end{enumerate}

  \item Una recta no vertical de ecuación $y = ax + b$ es una \emph{asíntota}\footnote{Cuando
      se estudie el concepto de derivada y se aplique a la obtención del gráfico de una curva,
      se hará un tratamiento más profundo del concepto de asíntota que el dado en este
      ejercicio} de la curva de ecuación $y = f(x)$ si una de las dos igualdades siguientes es
      verdadera (o ambas):
      \[
          \limjc{[f(x) - (ax + b)]}{x}{+\infty} = 0
      \]
      o
      \[
      \limjc{[f(x) - (ax + b)]}{x}{-\infty} = 0.
      \]

      Una recta vertical de ecuación $x = a$ es una \emph{asíntota vertical} de la curva de
      ecuación $y = f(x)$ si una de las dos igualdades siguientes es verdadera (o ambas):
      \[
          \limjc{|f(x)|}{x}{a^+} = +\infty \quad\text{o}\quad
          \limjc{|f(x)|}{x}{a^-} = +\infty,
      \]
      y si $f$ es monótona cerca de $a$ por la derecha o por la izquierda, según el caso.
      Determine si la curva de ecuación $y = f(x)$ tiene asíntotas verticales u horizontales.
            \begin{enumerate}
            \item $\displaystyle f(x)=\frac{1}{x^2+x+1}$.
            \item $\displaystyle f(x)=\frac{2x^2+1}{x^2-3x+2}$.
            \item $\displaystyle f(x)=\frac{x^2}{x+2}$.
            \item $\displaystyle f(x)=\frac{3x^2-1}{2x^2+1}$.
            \item $\displaystyle f(x)=\frac{2}{1- |x+1|}$.
            \item $\displaystyle f(x)=\sqrt{\frac{x+1}{x-1}}$.
            \end{enumerate}

\item En cada uno de los siguientes dibujos se muestra el gráfico de la función $f$. Con la
    información contenida en él, encuentre, si existen, los siguientes límites de $f(x)$ cuando
    $x$ tiende a $+\infty$, $-\infty$, $a^+$ y $a^-$.

    \begin{enumerate}[leftmargin=*]
    \item
  \begin{center}
  \psset{xAxisLabel={},yAxisLabel={},plotpoints=1000}%
  \begin{psgraph}[arrows=->,ticks=none,labels=none](0,0)(-4,-1)(4,8){0.4\textwidth}{5cm}
  \uput[-90](4,0){$x$}%
  \uput[0](0,8){$y$}%

  \psplot{-4}{0.85}{1 x 1 sub abs div 0.25 add}%

  \psline[linestyle=dashed,linecolor=gray]%
    (1,0)(1,7)%
  \uput[-90](1,0){\footnotesize$a$}%

  \psplot[arrows=o-]{1}{2.5}{x 3 mul 1 sub}%

  \psline[linewidth=0.1pt]%
    (-0.1,1)(0.1,1)%
  \uput[0](0,1){\footnotesize$a$}
  \psline[linewidth=0.1pt]%
    (-0.1,2)(0.1,2)%
  \uput[180](0,2){\footnotesize$2a$}
\end{psgraph}
\end{center}

\item
\begin{center}
\psset{xAxisLabel={},yAxisLabel={},plotpoints=1000}%
\begin{psgraph}[arrows=->,ticks=none,labels=none](0,0)(-2.5,-1)(5.5,8){0.4\textwidth}{5cm}
  \uput[-90](5.5,0){$x$}%
  \uput[0](0,8){$y$}%

  \psplot{1.15}{5.25}{1 x 1 sub abs div 0.25 add}%

  \psline[linestyle=dashed,linecolor=gray]%
    (1,0)(1,7)%
  \uput[-90](1,0){\footnotesize$a$}%

  \psplot[arrows=-o]{-2.5}{1}{x neg 0.75 mul 1.75 add}%

  \psline[linewidth=0.1pt]%
    (-0.1,1)(0.1,1)%
  \uput[0](0,1){\footnotesize$a$}
  \psline[linewidth=0.1pt]%
    (-0.1,2)(0.1,2)%
  \uput[0](0,2){\footnotesize$2a$}
\end{psgraph}
\end{center}

\item

\begin{center}
\psset{xAxisLabel={},yAxisLabel={},plotpoints=1000}%
\begin{psgraph}[arrows=->,ticks=none,labels=none](0,0)(-3,-4)(5,3){0.4\textwidth}{5cm}
  \uput[-90](5,0){$x$}%
  \uput[0](0,3){$y$}%

  \psplot{-3}{0.8}{x dup 1 sub div}%

  \psline[linestyle=dashed,linecolor=gray]%
    (1,0)(1,-4)%
  \uput[-90](1,0){\footnotesize$a$}%

  \psplot[arrows=o-]{1}{4}%
      {x 1 sub dup mul 2 div neg 2.71828182 exch exp x 1 sub mul 1.5 div 4 neg mul 2 add}%

  \psline[linestyle=dashed,linecolor=gray]%
    (-3,1)(0,1)%
  \psline[linewidth=0.1pt]%
    (-0.1,1)(0.1,1)%
  \uput[0](0,1){\footnotesize$a$}

  \psline[linestyle=dashed,linecolor=gray]%
    (0,2)(4,2)%
  \psline[linewidth=0.1pt]%
    (-0.1,2)(0.1,2)%
  \uput[180](0,2){\footnotesize$2a$}
\end{psgraph}
\end{center}

\item
\begin{center}
\psset{xAxisLabel={},yAxisLabel={},plotpoints=1000}%
\begin{psgraph}[arrows=->,ticks=none,labels=none](0,0)(-4,-2)(4.5,3){0.4\textwidth}{5cm}
  \uput[-90](4.5,0){$x$}%
  \uput[0](0,3){$y$}%

  \psplot[algebraic]{1}{4}{sin(5*(x-1)) + 1}%

  \psline[linestyle=dashed,linecolor=gray]%
    (1,0)(1,2)%
  \uput[-90](1,0){\footnotesize$a$}%

  \psplot[arrows=-o]{-4}{1}%
      {x 0.5 sub dup mul 2 div neg 2.71828182 exch exp x 0.5 sub mul 1.5 div 4 mul 0.176662537 sub}%

  \psline[linestyle=dashed,linecolor=gray]%
    (0,2)(4,2)%
  \psline[linewidth=0.1pt]%
    (-0.1,1)(0.1,1)%
  \uput[180](0,1){\footnotesize$a$}

  \psline[linestyle=dashed,linecolor=gray]%
    (0,1)(4,1)%
  \psline[linewidth=0.1pt]%
    (-0.1,2)(0.1,2)%
  \uput[180](0,2){\footnotesize$2a$}
\end{psgraph}
\end{center}

    \end{enumerate}

\item Dada $\funcjc{f}{\mathbb{R}}{\mathbb{R}}$, si existen los límites $\displaystyle \limjc{\frac{f(x)}{x}}{x}{+\infty} = m$ y $\displaystyle \limjc{f(x) - mx}{x}{+\infty} = b$, pruebe que la recta de ecuación $y = mx + b$ es asíntota horizontal si $m = 0$, u oblicua si $m \neq 0$ de $f$ del gráfico de $f$ por la derecha. El resultado es análogo por la izquierda, tomando en cada caso el límite cuando $x$ tiende a $-\infty$. Aplique este resultado para hallar las asíntotas horizontales u oblicuas de $f$ si:
    \begin{enumerate}[leftmargin=*]
    \item $\displaystyle f(x) = \frac{x}{x^2 - 4x + 3}$.
    \item $\displaystyle f(x) = \frac{x}{\sqrt{x^2 + 3}}$.
    \item $\displaystyle f(x) = e^{-x^2} + 2$.
    \item $\displaystyle f(x) = \frac{3x^2 - 2x + \cos(x + 1)}{x - 1}$.
    \item $\displaystyle f(x) = \frac{2x^3 - 7e^{-x}}{x + 2}$.
    \item $\displaystyle f(x) = \frac{4x^3 - 7x + 2}{5x^3 + 2x + \cos x}$.
    \end{enumerate}
\end{enumerate}
\end{multicols}
\endgroup

\chapter{La derivada: su motivación}

Veamos algunos problemas que llevan a un mismo concepto: el de derivada.

\section{La recta tangente a una curva}
En la segunda sección del capítulo de límites, estudiamos
el concepto de recta tangente a una curva. Vimos que si ésta es el gráfico de una función
$\funcjc{g}{D\subset\mathbb{R}}{\mathbb{R}}$ y $P$ es un punto de la curva con coordenadas
$(a,g(a))$, para conocer la ecuación de la recta tangente a la curva en el punto $P$, que es
\[
y = m(x - a) + g(a),
\]
bastaba con calcular su pendiente $m$.

Para ello, aproximamos $m$ con la pendiente de una recta que pasa por $P$ y por otro punto $Q$ de
la recta de coordenadas $(x,g(x))$, que la notamos $m_x$, y que es igual a:
\[
m_x = \frac{g(x) - g(a)}{x - a}.
\]

Para obtener $m$, la idea era ``acercar'' $Q$ a $P$, esperando que, al hacerlo, $m_x$ se acercara a
$m$. En el ejemplo estudiado, vimos que eso era así y que, de hecho:
\[
m = \limjc{m_x}{x}{a} = \limjc{\frac{g(x) - g(a)}{x - a}}{x}{a}.
\]

En efecto, teníamos que $g$ está definida por $g(x) = 3x^2$ y que $a = 2$. Por lo tanto, $g(a) =
g(2) = 12$ y:
\begin{align*}
m_x &= \frac{3x^2 - 12}{x - 2} \\[4pt]
m &= \limjc{\frac{3x^2 - 12}{x - 2}}{x}{2} = 12.
\end{align*}

La recta $l$ de ecuación
\[
y = 12 + 12(x - 2) = 12x + 10
\]
es, efectivamente, la tangente al gráfico de $g$ que es una parábola $p$. Más aún, la recta $l$ es
la única recta que pasa por el punto de $P$ de coordenadas $(2,12)$ y tiene un solo punto en común
con la parábola $p$ de ecuación
\[
y = g(x) = 3x^2,
\]
que es, justamente, el punto $P$.

Por otra parte, si exceptuamos el punto $P$, toda la parábola está ``de un solo lado'' de la recta
$l$, y ``ninguna otra recta se interpone en el espacio entre'' la recta $l$ y la parábola $p$, como
lo exige la definición de Euclides.

Mostraremos más tarde que esta afirmación es verdadera, cuando estudiemos otras definiciones de
tangencia y mostremos algunas propiedades notables de la recta tangente. Por ejemplo, probaremos
que de entre todas las rectas que pasan por el punto de tangencia, la recta tangente es la que
mejor aproxima a la curva.

Ahora bien, del ejemplo citado resaltaremos el hecho de que la pendiente $m$ de la recta tangente
al gráfico de $g$ en el punto $(a,g(a))$ está dada por
\[
m = \limjc{\frac{g(x) - g(a)}{x - a}}{x}{a}.
\]

Esta propiedad la cumple la pendiente de la recta tangente en un punto dado de una gran variedad de
curvas que son el gráfico de funciones (como son los polinomios, funciones racionales,
trigonométricas, logarítmicas, exponenciales, etcétera). Esto nos motiva a cambiar la tradicional
definición de recta tangente a una curva por la siguiente:

\begin{defical}[Recta tangente]
Dada una función real $f$ definida en un intervalo abierto $I$ y $a \in I$, diremos que el gráfico
de $f$ tiene una recta tangente en el punto $(a, f(a))$ si existe
\[
m:= \limjc{\frac{f(x) - f(a)}{x - a}}{x}{a}.
\]
En este caso, la ecuación de la recta tangente es:
\[
y - f(a) = m(x - a).
\]
\end{defical}

\begin{exemplo}[Solución]{%
Hallar, si existe, la ecuación de la recta tangente al gráfico de la hipérbola de ecuación
$y = 2 - \frac{1}{x}$ en el punto de coordenadas $(1,1)$.
}%
Si ponemos $a = 1$ y definimos $f$ por
\[
f(x) = 2 - \frac{1}{x},
\]
entonces $f(a) = f(1) = 1.$ Además:
\begin{align*}
m &= \limjc{\frac{f(x) - f(1)}{x - 1}}{x}{1} =%
    \limjc{\displaystyle\frac{\left(2 - \frac{1}{x}\right) - \left(2 - 1\right)}{x - 1}}%
    {x}{1} \\[6pt]
  &= \limjc{\displaystyle\frac{-\frac{1}{x} + 1}{x - 1}}{x}{1} =
  \limjc{\frac{x - 1}{x(x - 1)}}{x}{1} \\[4pt]
  &= \limjc{\frac{1}{x}}{x}{1} = 1.
\end{align*}
Como el límite existe, la recta tangente tendrá por ecuación a
\[
y -1 = 1(x - 1);
\]
es decir, la ecuación de la recta tangente a la hipérbola en el punto de coordenadas $(1,1)$ es $y
= x$.

Una de las aplicaciones más valiosas del concepto de derivada que vamos a estudiar es el proveer
una herramienta para realizar un dibujo aproximado del gráfico de una función. Más adelante,
podremos constatar que el gráfico de $f$ y de la tangente al gráfico de $f$ en el punto de
coordenadas $(1,1)$ es el siguiente:
\begin{center}
\begin{pspicture}(-3,-1)(5,4)
\psset{unit=0.5cm,plotpoints=250}%
\def\pshlabel#1{\footnotesize #1}%
\def\psvlabel#1{\footnotesize #1}%

\psaxes[ticks=none,labels=none]{->}(0,0)(-6,-2)(9.5,7.5)%
\uput[0](0,7.5){$y$}%
\uput[-90](9.5,0){$x$}%

\psplot[linewidth=1.5\pslinewidth]{-6}{-0.2}{2 1 x div sub}%
\psplot[linewidth=1.5\pslinewidth]{0.25}{8}{2 1 x div sub}%
\rput[l](-5,3.5){$\displaystyle y = 2 - \frac{1}{x}$}

\psplot[linestyle=dashed]{-6}{8}{2}%

\psplot{-1.5}{5}{x}%
\rput[l](5.1,5){$y = x$}

\uput[0](0,2.4){$2$}%

\psline[linestyle=dashed,linecolor=gray]%
    (1,0)(1,1)(0,1)%
\psdot[dotscale=0.8](1,1)%
\uput[-90](1,0){$1$}%
\uput[180](0,1){$1$}%

\end{pspicture}
\end{center}
\end{exemplo}

\begin{exemplo}[Solución]{%
Hallar la ecuación de la recta tangente al gráfico de la cúbica de ecuación $y = x^3$ en
el punto de coordenadas $(-1,-1)$.
}%
Si ponemos $a = -1$ y definimos $f(x) = x^3$, entonces $f(-1) = -1$ y
\begin{align*}
m   &= \limjc{\frac{f(x) - f(-1)}{x - (-1)}}{x}{-1} =
        \limjc{\frac{x^3 + 1}{x + 1}}{x}{-1} \\[4pt]
    &= \limjc{\frac{(x+1)(x^2 - x + 1)}{x + 1}}{x}{-1} =
        \limjc{x^2 - x + 1}{x}{-1} = 3.
\end{align*}
Por lo tanto, la recta buscada tiene como ecuación la siguiente:
\[
y - (-1) = 3(x - (-1)) = 3x + 3,
\]
de donde, la ecuación buscada es:
\[
y = 3x + 2.
\]

El gráfico de $f$ y de la tangente en $(-1,-1)$ es el siguiente:
\begin{center}
\begin{pspicture}(-2,-2)(4.2,5)

\SpecialCoor
\psset{yunit=0.5cm,plotpoints=250}%
\def\pshlabel#1{\footnotesize #1}%
\def\psvlabel#1{\footnotesize #1}%

\psaxes[ticks=none,labels=none]{->}(0,0)(-2,-4)(2.75,9.5)%
\uput[0](0,9.5){$y$}%
\uput[-90](2.75,0){$x$}%

\begingroup
    \psset{linewidth=1.5\pslinewidth}%
    \psplot{-1.5}{2.1}{x 3 exp}%
    \rput[l](! -1.4 1.5 3 exp neg){$y= x^3$}%

    \psplot{-1.5}{2.5}{x 3 mul 2 add}%
    \rput[l](2.5,8.3){$y = 3x + 2$}%
\endgroup

\begingroup
    \psset{dotscale=0.8,linestyle=dashed}%
    \psline[linecolor=gray]%
        (-1,0)(-1,-1)(0,-1)%
    \psdot(-1,-1)%
    \uput[90](-1,0){$-1$}%
    \uput[0](0,-1){$-1$}%

    \psline[linecolor=gray]%
        (2,0)(2,8)(0,8)%
    \psdot(2,8)%
\endgroup
\end{pspicture}
\end{center}
\end{exemplo}

Observemos en el dibujo que la recta tangente a la curva no tiene con ésta un único punto en común
sino dos. El un punto, que es el de tangencia, tiene coordenadas $(-1,-1)$. Podemos averiguar las
coordenadas del otro punto resolviendo el sistema de ecuaciones:
\[
\left\{
\begin{matrix}
y & = & x^3 \\
y & = & 3x + 2
\end{matrix}
\right.
\]

Al igualar las ecuaciones, obtenemos que
\[
x^3 = 3x + 2,
\]
lo que equivale a
\[
x^3 - 3x - 2 = 0.
\]

Esta ecuación puede tener hasta tres raíces reales. Sabemos que una de ellas es $-1$, porque la
tangente y la curva se ``encuentran'' en el punto de coordenadas $(-1,-1)$. Entonces, el polinomio
cúbico $x^3 - 3x - 2$ tiene como factor $(x + 1)$.

Por lo tanto, podemos obtener el otro factor si dividimos el cúbico por $(x + 1)$. Al hacerlo,
obtenemos que:
\[
x^3 - 3x - 2 = (x + 1)(x^2 - x - 2) = (x+1)(x + 1)(x - 2).
\]
Entonces, el polinomio tiene las tres raíces son reales e iguales a $-1$, de multiplicidad $2$, y
$2$.

Entonces, ya estamos seguros que la curva y la recta tangente se encuentran exactamente en dos
puntos: $(-1,1)$ y $(2,f(2)) = (2,8)$.

Pero, ?`no se suele decir que la tangente y la curva solo deben tener un punto en común? Sí, eso
suele decirse, sin embargo, tal definición no es del todo precisa, pues, en las ``cercanías'' de
$(-1,-1)$ la tangente y la curva tienen, efectivamente, un comportamiento de tangente en el sentido
tradicional.

Por esto, la propiedad de tangencia es de carácter ``local''; es decir, no importa el
``comportamiento'' de la curva y de la recta ``lejos'' del punto de tangencia sino en sus
``cercanías''.


\section{?`Cómo medir el cambio?}

Tengamos en cuenta que la Matemática desarrollada hasta antes del Renacimiento estudiaba más bien
conceptos ``estáticos'': describir formas geométricas que no cambian, calcular áreas de figuras
simples, llevar cuentas, etcétera. La geometría de Euclides, la aritmética de Diofanto y el álgebra
desarrollada por Al-Khowarizmi fueron instrumentos idóneos para responder a esta visión estática.

Pero cuando la humanidad intenta describir el movimiento que se produce en gran cantidad de
fenómenos estudiados, se hace necesaria la creación de nuevos conceptos matemáticos que permitan
responder adecuadamente a estos requerimientos.

Dado que el movimiento es percibido como el cambio de la posición de un cuerpo en el tiempo, una de
las preguntas más importantes que la matemática enfrentó fue ``?`cómo medir el cambio?''.

Veamos en esta sección cómo la respuesta a esta pregunta nos lleva también al concepto de derivada.

\subsection{?`Cuánto cuesta producir autos?}

\begingroup
\itshape El administrador de una fábrica de autos desea tener información adecuada respecto a los
costos de producción de la fábrica en función del número de vehículos en un mes dado. Así, desearía
saber cómo varían estos costos al variar la producción. ?`Qué información requiere y cómo
presentarla? ?`Cómo medir la variación o cambio de los costos de la producción? ?`Cómo relacionar la
producción con los costos y la dependencia entre la variación de la producción y la de los costos?
\endgroup

Para responder a éstas y a otras preguntas relacionadas, se requerirá de algunos conceptos
matemáticos como son el de ``variable numérica'', ``función real'', ``incremento absoluto'',
``incremento relativo'', ``razón de cambio'', ``derivada'', etcétera, los mismos que expondremos a
lo largo de esta sección.

\subsection{Las variables representan magnitudes}
Cuando estudiamos determinados fenómenos físicos, sociales o económicos ---como el problema
planteado en la sección anterior---, identificamos ciertas magnitudes que consideramos importantes
para dicho estudio.

Por ejemplo, al estudiar un circuito eléctrico, la tensión $V$ voltios y la intensidad $I$ amperios
de la corriente eléctrica, y la resistencia $R$ ohmios de los dispositivos instalados son
importantes. Cuando se monitorea el trabajo de una represa, será de interés la cota del nivel de
agua $h$ metros y la energía $E\MWh$ almacenada. Al dirigir una fábrica de autos se deseará conocer
el costo total $C$ miles de dólares y el correspondiente costo unitario $CU$ miles de dólares al
producir $x$ vehículos, etcétera.

Establecida la unidad de medida que corresponda y luego de realizar las mediciones o cálculos
necesarios, se determinará, para cada magnitud, un determinado valor numérico en un momento dado.
El símbolo que represente a dicho valor numérico, el cual oscila entre un valor mínimo y un valor
máximo, dados por el fenómeno concreto que se está estudiando, se llama \emph{variable}. En los
ejemplos dados, tenemos que $V$, $I$, $R$, $h$, $E$, $C$, $CU$ y $x$ son variables.

Por ejemplo, la fábrica de autos producirá $x$ automóviles por año y $x$ no puede ser menor a
1\,000 porque no sería rentable producir menos, pero tampoco producirá más de 10\,000, dado que la
capacidad instalada de la fábrica no lo permite. Matemáticamente, expresamos estas dos
restricciones ``haciendo que'' la variable $x$ pertenezca al intervalo $[1\,000,10\,000]$.

\subsection{La función como modelo de la dependencia entre magnitudes}
Hemos visto el concepto de función. Ilustremos con tres ejemplos su uso como modelo de la
dependencia entre magnitudes.

\subsubsection{Costo de producción en una fábrica de autos}
Como el costo $C$ de producción de los autos, en miles de dólares, depende del número $x$ de
vehículos producidos, esta dependencia es modelizada con una función
$\funcjc{f}{\Dm(f)\subset\mathbb{R}}{\mathbb{R}}$, donde:
\[
C=f(x).
\]
Como solo nos interesan los valores de $x$ entre $1\,000$ y $10\,000$, el dominio de $f$ será
$\Dm(f)=[1\,000,10\,000]$.

Luego de los estudios correspondientes, los administradores de la empresa determinan que:
\[
C=f(x)=2\,000+11x -0.000\,12x^{2}.
\]
Por lo tanto, el costo unitario $C_U$, es decir el costo promedio de un auto es:
\[
C_U = \frac{f(x)}{x}=\frac{2\,000}{x}+11-0.000\,12x.
\]

A manera de ejemplo, en la siguiente tabla, se muestran los valores de $C$ y del costo unitario
$C_U$ que cuesta producir cada auto si se producen $x$ autos, dados en miles de dólares,
correspondientes al año anterior:

\begingroup
\begin{center}
{\renewcommand\arraystretch{1.5}%
\setlength\extrarowheight{1pt}
\begin{tabular}{|c |c |c|}
\hline
$x$ & $C=f(x)$ & $C_U=\frac{f(x)}{x}$  \\
\hline
1\,000  &  12\,880 &  12.880 \\[-2pt]
\hline
2\,000 & 23\,520 & 11.760 \\
\hline
3\,000 & 33\,920  & 11.307 \\
\hline
5\,000 & 54\,000 & 10.800 \\
\hline
6\,000 & 63\,680 & 10.613 \\
\hline
7\,000 & 73\,120 & 10.446 \\
\hline
8\,000 & 82\,320 & 10.290 \\
\hline
9\,000 & 91\,280 & 10.142 \\
\hline
10\,000 & 100\,000 & 10.000 \\
\hline
\end{tabular}}
\end{center}

\subsubsection{Energía potencial en una represa}
La \emph{energía potencial} $E$, medida en $\MWh$, que está almacenada en una represa depende de la
\emph{cota} $h$, medida en $\metros$, se define como la altura sobre el nivel del mar del espejo de
agua en la represa.

Supongamos que, en una determinada represa, la cota puede estar entre $2\,900$ y $2\,950$ metros,
existe una función $\funcjc{g}{\Dm(g)}{\mathbb{R}}$, con dominio $\Dm(g)=[2\,900,2\,950]$, tal
$g(h)$ es la energía almacenada $E$ cuando la cota es $h$. Es decir:
\[
E=g(h).
\]

\subsubsection{El diámetro y el volumen de un globo}
Un globo, en la parte superior, tiene forma de un hemisferio achatado; en su parte inferior, es un
hemisferio alargado que culmina con una abertura cilíndrica.

Los fabricantes han determinado que el volumen $V$, medido en $\decimetros^3$, de la cámara de gas
del globo depende de su diámetro $d$, medido en $\decimetros$. Entonces, existe una función
$\funcjc{\varphi}{\Dm(\varphi)}{\mathbb{R}}$ tal que:
\[
V = \varphi(d)=\frac{1}{2\,000}d^{3}.
\]

El diseño de los globos de esta fábrica establece que el globo se eleva si el diámetro mide, al
menos, $6\metros$. Además, el mayor diámetro al que el globo puede ser inflado es de $12\metros$.
Por ello, el dominio de $\varphi$ es $\Dm(\varphi)=[60,120]$.

Si el helio costara $40$ centavos de dólar por cada $\metros^3$, podríamos establecer el gasto $G$
en dólares que deberíamos hacer por concepto de gas en función del diámetro del globo. Así, existe
una función $\funcjc{\psi}{\Dm(\psi)}{\mathbb{R}}$ tal que
\[
G=\psi(d) = 0.40\varphi(d)=0.000\,2d^3.
\]

En la siguiente tabla, se muestran valores del volumen del globo y del gasto en el que hay que
incurrir para inflarlo para algunos valores del diámetro:
\begin{center}
\begingroup
\setlength\extrarowheight{4pt}
\begin{tabular}{|c |c |r @{.} l|}
\hline
$d$ & $V=\varphi(d)$ & \multicolumn{2}{|c|}{$G=\psi(d)$}  \\
{\small$\decimetros$} & $\decimetros^3$ & \multicolumn{2}{|c|}{dólares} \\
\hline
60  &  108.0 &  43&20 \\
\hline
70 & 171.5 & 68&60 \\
\hline
80 & 256.0  & 102&40 \\
\hline
90 & 364.5 & 145&50 \\
\hline
100 & 500.0 & 200&00 \\
\hline
110 & 665.5 & 260&20 \\
\hline
120 & 864.0 & 345&60 \\
\hline
\end{tabular}
\endgroup
\end{center}

\subsection{La variaciones absoluta y relativa como medidas del cambio}
Cuando una variable, digamos $x$, toma diversos valores luego de haber tomado un ``valor inicial''
$x_0$, unos dirán que varió mucho, otros lo contrario. Esto es algo subjetivo. Por ello, se hace
necesario establecer medidas objetivas del cambio producido. Para ello, nos serviremos de los
conceptos de \emph{variación absoluta} y \emph{relativa}.

\subsubsection{La variación absoluta o incremento}
En el ejemplo del globo, si trabajáramos con un diámetro de $80\dm$, veríamos que el volumen de gas
necesario sería de $256\metros^3$ y que el costo del gas sería de $102.40$ dólares. Si infláramos
algo más el globo, digamos hasta un diámetro $d$ decímetros, la \emph{variación} o \emph{cambio}
del diámetro se define como la diferencia $d-80$ decímetros, a la que llamaremos \emph{variación
absoluta} o \emph{incremento} del diámetro, la notaremos con $\Delta d$ y será leída ``delta $d$''
y está dada también en decímetros. Así $\Delta d = d-80$.

El incremento puede ser positivo o negativo. Por supuesto que $\Delta d > 0$ significa que el
diámetro creció mientras que $\Delta d < 0$ significa lo contrario. En este caso, a $\Delta d$
también se le llama \emph{decremento}.

Al variar $d$, naturalmente, varían también el volumen del gas $V = \varphi(d)\dm^3$ cúbicos y el
costo que hay que pagar por este gas $G = \psi(d)$ dólares. La \emph{variación absoluta} del
volumen será:
\[
\Delta V = V - 256 = \varphi(d) - \varphi(80)\dm^3
\]
y la \emph{variación absoluta} del costo será:
\[
\Delta G = G - 102.40 = \psi(d) - \psi(80)\ \text{dólares}.
\]

\begin{exemplo}[]{}
Consideremos la situación de la fábrica de autos una vez más. Si su producción, en un mes dado,
fuera de, digamos, $5\,000$ autos y, en el mes siguiente, ésta pasara a ser de $x$ autos, entonces
la variación absoluta de la producción estaría dada por:
\[
\Delta x=x-5\,000.
\]
En este caso, la variación absoluta de los costos, que pasarían de $54\,000$ miles de dólares a $C$
miles de dólares sería igual a:
\[
\Delta C=C-54\,000.
\]

Como
\[
C = f(x)= 2\,000+11x-0.000\,12x^2,
\]
se tendrá que
\[
\Delta C=f(x)-54\,000 =-52\,000 + 11x - 0.000\,12x^2.
\]

Análogamente, la variación absoluta del costo unitario, que pasó de $10.8$ miles de dólares a $C_U$
miles de dólares, estaría dado por:
\[
\Delta C_U=C_U-10.8.
\]
Puesto que:
\[
C_U = \frac{f(x)}{x}= \frac{2\,000}{x}+11-0.000\,12x,
\]
se tendría que:
\[
\Delta C_U = \frac{f(x)}{x}-10.8= \frac{2\,000}{x}+0.2 -0.000\,12x.
\]

Por ejemplo, si $x=6\,000$ autos, entonces:
\[
f(6\,000) \approx 63\,680
\]
y
\[
\frac{f(6\,000)}{6\,000}\approx 10.613.
\]
Además, tendremos que:
\begin{align*}
\Delta x & = 6\,000-5\,000=1\,000,\\
\Delta C & \approx 63\,680-54\,000= 9\,680, \\
\Delta C_U & \approx 10.613-10.8 = -0.187.
\end{align*}
Es decir que, si la producción aumentara en $1\,000$ autos, los costos subirían en $9.68$ millones
de dólares, mientras que ¡el costo unitario bajaría en $187$ dólares por auto!
\end{exemplo}

\subsubsection{La variación relativa}
Si el precio del kilogramo de arroz se incrementara de un precio inicial de $p_{0} = 0.40$ dólares
a un precio de $p = 0.70$ dólares, el incremento del precio se mediría por su variación absoluta:
\[
\Delta p = p - p_{0} = 0.70 - 0.40 = 0.30\ \text{dólares}.
\]

Si el precio de una llanta pasara de $p_{0} = 20.00$ dólares a $p = 20.30$ dólares, la variación
absoluta sería también de $0.30$ dólares.

Obviamente, la importancia del alza de $0.30$ dólares al precio del kilogramo de arroz no es la
misma que la del alza al precio de una llanta. En el primer caso, el precio casi se ha duplicado,
mientras que, en el caso de la llanta, el alza es insignificante.

Para medir este fenómeno, se utiliza el concepto de \emph{variación relativa} que es el cociente de
la variación absoluta dividida por el valor inicial que tiene la variable. Así, en el ejemplo del
arroz, la variación relativa es:
\[
\frac{\Delta p}{p_{0}} = \frac{0.30}{0.40} = 0.75.
\]

La variación relativa suele también expresarse como un porcentaje al que se lo denomina
\emph{variación porcentual}.

Teniendo en cuenta la identidad $\%=\frac{1}{100}$ (o 100\% = 1), podemos decir que la variación
porcentual del precio del kilogramo de arroz es:
\[
\frac{\Delta p}{p_{0}}\times 100 \% = 0.75\times 100\% = 75\%
\]
y que la variación porcentual del precio de la llanta es:
\[
\frac{\Delta p}{p_{0}}\times 100\% = 0.015\times 100\% = 1.5\%.
\]

\begin{exemplo}[]{}
En el caso de la fábrica de autos, considerando las mismas variaciones absolutas de autos,
tendremos que la variación relativa de la producción es:
\[
\frac{\Delta x}{5\,000}= \frac{x-5\,000}{5\,000},
\]
la variación relativa del costo:
\[
\frac{\Delta C}{54\,000}= \frac{C-54\,000}{54\,000}
\]
y la del costo unitario:
\[
\frac{\Delta C_U}{10.8}= \frac{C_U-10.8}{10.8} =
    \frac{\displaystyle\frac{f(x)}{x}-10.8}{10.8}.
\]

Si $x=6\,000$ autos, se tendrá que la variación porcentual de la producción sería:
\[
\frac{\Delta x}{5\,000} \times 100\% = \frac{1\,000}{5\,000}\times 100\% = 20\%,
\]
la variación porcentual del costo sería:
\[
\frac{\Delta C}{54\,000}\times 100\%= \frac{9\,680}{54\,000}\times 100\% \approx 18\%
\]
y la del costo unitario sería:
\[
\frac{\Delta C_U}{10.8} \times 100\% = \frac{-0.187}{10.8}\times 100\% \approx -1.7\%.
\]
En otras palabras, al subir la producción en un $20\%$, los costos aumentan en un $18\%$, mientras
que el costo unitario baja en $1.7\%$.
\end{exemplo}

\section{Razón de cambio, elasticidad y magnitudes marginales}
Vimos que la dependencia entre las dos variables numéricas que describen un fenómeno, digamos una
variable $y$ que depende de otra variable $x$, puede ser descrita mediante una función
$\funcjc{f}{I}{\mathbb{R}}$, donde $I$ es un intervalo en el cual toma sus valores la variable
independiente $x$.

Por ejemplo, $y$ miles de dólares representa el costo de producir $x$ autos en una fábrica durante
un mes, el intervalo $I$ es, digamos, $[1\,000, 10\,000]$ y la función $f$ está definida por:
\[
f(x) = 2\,000 + 10x.
\]

Ahora bien, si, en un mes dado, la producción fue de $x_0 = 4\,800$ autos, los costos de producción
ese mes fueron de $y_0$ miles de dólares. Entonces:
\[
y_0 = f(x_0) = 2\,000 + 10x_0.
\]
Por lo tanto:
\[
y_0 = 2\,000 + 48\,000 = 50\,000.
\]

Supongamos que, en el siguiente mes, la producción se elevó a $x$ autos. Este cambio se puede
medir, según vimos, por la variación absoluta $\Delta x$, llamada también \emph{incremento} de $x$,
y por la variación relativa $\frac{\Delta x}{x_0}$ de la variable $x$. En este caso:
\begin{align}
\label{eq:dm001}
\Delta x &= x - x_0 = x - 4\,800. \\
\label{eq:dm002}
\frac{\Delta x}{x_0} &= \frac{x - x_0}{x_0} = \frac{x - 4\,800}{4\,800}.
\end{align}

Obviamente, si $\Delta x > 0$, implica que la variable $x$ creció; en el caso contrario, decreció.
En la práctica, a un incremento negativo se le llama \emph{decremento}.

Por otra parte, como la dependencia del costo respecto del nivel de producción está descrito por la
función $f$, mediante la igualdad $y = f(x)$, la variable costo sufrirá también un cambio, descrito
por la variación absoluta o incremento $\Delta y$ y por la variación relativa de $y$, que es
$\frac{\Delta y}{y_0}$. Podemos calcular ambas variaciones:
\begin{align}
\Delta y &= y - y_0 = f(x) - f(x_0) = (2\,000 + 10x) - 50\,000 \nonumber \\
\label{eq:dm003}
\Delta y &= -48\,000 + 10x \\
\label{eq:dm004}
\frac{\Delta y}{y_0} &= \frac{-48\,000 + 10x}{50\,000}.
\end{align}

Ahora bien, como $y$ depende de $x$, y esta dependencia está descrita por $y = f(x)$, para un $x_0$
fijo dado, es de esperar que la dependencia del incremento $\Delta y$ respecto del incremento
$\Delta x$ pueda describirse fácilmente, al igual que la dependencia de la variación relativa o
porcentual $\frac{\Delta y}{y_0}$ respecto de $\frac{\Delta x}{x_0}$, la variación relativa o
porcentual de $x$. Veamos que sí es así.

¡Resulta que $\Delta y$ es directamente proporcional a $\Delta x$, así como $\frac{\Delta y}{y_0}$
es directamente proporcional a $\frac{\Delta x}{x_0}$! Es decir, existen constantes $\kappa$ y
$\eta$ tales que
\begin{align}
\Delta y &= \kappa\Delta x \label{eq:dm005}\\
\frac{\Delta y}{y_0} &= \eta\frac{\Delta x}{x_0}. \label{eq:dm006}
\end{align}

A la constante $\kappa$ se le llama \emph{razón de cambio} y a la constante $\eta$,
\emph{elasticidad} de $y$ respecto de $x$. Calculemos estas dos constantes. De (\ref{eq:dm001}),
(\ref{eq:dm003}) y (\ref{eq:dm005}) tenemos que:
\begin{equation}
\label{eq:dm007}
\kappa = \frac{\Delta y}{\Delta x} = \frac{-48\,000 + 10x}{x - 4\,800} = 10.
\end{equation}

De (\ref{eq:dm006}) y (\ref{eq:dm007}) obtenemos:
\begin{equation}
\label{eq:dm008}
\eta = \frac{\displaystyle\frac{\Delta y}{y_0}}{\displaystyle\frac{\Delta x}{x_0}}
    = \frac{x_0}{y_0}\frac{\Delta y}{\Delta x} = \frac{4\,800}{50\,000}\cdot 10 = 0.96.
\end{equation}

Es fácil verificar que esta sencilla dependencia de $\Delta y$ respecto de $\Delta x$ y de
$\frac{\Delta y}{y_0}$ respecto de $\frac{\Delta x}{x_0}$ se da siempre que $f$ sea un polinomio de
grado menor que o igual a $1$:

\begin{teocal}
Si $\funcjc{f}{I\subset\mathbb{R}}{\mathbb{R}}$ definida por $y = f(x) = mx + b$, y para todo
$x_0\in I$ y todo $x\in I$, si:
\[
y_0 = f(x_0), \quad \Delta x = x - x_0, \quad \Delta y = y - y_0 = f(x) - f(x_0),
\]
se tiene que:
\begin{equation}
\label{eq:dm009}
\Delta y = \kappa\Delta x, \quad \frac{\Delta y}{y_0} = \eta\frac{\Delta x}{x_0},
\end{equation}
donde
\begin{equation}
\label{eq:dm010}
\kappa = m \yjc \eta = \frac{x_0}{y_0}m.
\end{equation}
\end{teocal}

Recordemos que el gráfico de $f$ es una recta con pendiente $m$. Por lo tanto, la razón de cambio
$\kappa$ es igual a $m$. Naturalmente, ese caso es excepcional. En general, tenemos las siguientes
definiciones inspiradas en (\ref{eq:dm007}) y (\ref{eq:dm008}).

\begin{defical}[Razón de cambio]
Sean $\funcjc{f}{I\subset\mathbb{R}}{\mathbb{R}}$ y $\lajc{x}{y = f(x)}$. Para $x_0 \in I$ y $x\in
I$ tales que $x\neq x_0$, si
\[
y_0 = f(x_0), \quad \Delta x = x - x_0, \quad \Delta y = y - y_0,
\]
la \emph{razón de cambio de $y$ respecto de $x$ en el intervalo de extremos $x_0$ y $x$} es
\begin{equation}
\label{eq:dm011}
f'(x_0;x) = \frac{\Delta y}{\Delta x} = \frac{f(x) - f(x_0)}{x - x_0}.
\end{equation}
\end{defical}

Obviamente, $\Delta y$ no es, en general, directamente proporcional a $\Delta x$, puesto que, si bien
\begin{equation}
\label{eq:dm012}
\Delta y = f'(x_0;x)\Delta x,
\end{equation}
se tiene que $f'(x_0;x)$ no es constante la mayoría de las veces, sino que depende de $x_0$ y de $x$. Sin
embargo, para muchas funciones, el valor de $f'(x_0;x)$ es muy cercano a una constante para valores
pequeños de $\Delta x$. Ilustremos esto con un ejemplo.

\paragraph{Ejemplo.}
En el caso del globo, el costo $G$ en dólares del gas necesario para inflar el globo de diámetro
$d$ decímetros está dado por:
\[
G = \psi(d) = \frac{1}{2\,000}d^{3},\ d\in [60,120].
\]
Si tomamos $d_{0}=80$ y diferentes valores de $d$ cercanos a 80 (ver la tabla a continuación),
tendremos que los valores que toma la razón de cambio
\[
\psi'(d_{0};d) = \frac{\Delta G}{\Delta d} = \frac{\psi(d)-\psi(d_{0})}{d-d_{0}}
\]
son muy similares entre sí y se acercan cada vez más a $3.84$. En otras palabras, podemos ver que
¡ese número es el límite de la razón de cambio $\psi'(d_{0};d)$ cuando $d$ tiende a $d_0$!

\begin{center}
{\renewcommand\arraystretch{1.5}%
\setlength\extrarowheight{1pt}
\begin{tabular}{|c|c|c|c|}
\hline
$d$ & $\psi(d)$ & $\psi(d)-\psi(d_{0})$ & $\psi'(d_{0},d)$ \\
\hline
78  &  94.91 &  $-$7.49 & 3.74  \\[-2pt]
\hline
79 & 98.61 & $-$3.79 & 3.79 \\
\hline
79.5 &  100.49    & $-$1.91     &  3.82   \\
\hline
79.9 & 102.02     &  $-$0.38    &  3.84      \\
\hline
&  &  &  $\downarrow $\\
80 & 102.40 & 0 & 3.84  \\
&  &  &  $\uparrow $\\
\hline
80.1 &  102.78     &  0.38    &    3.84       \\
\hline
80.5 &  104.33    &   1.93     &   3.86      \\
\hline
81 & 106.29 & 3.89 & 3.89 \\
\hline
82 & 110.27 & 7.87 & 3.94 \\
\hline
\end{tabular}}
\end{center}

Entonces, si definimos $\psi'(d_0)$ por:
\[
\psi'(d_{0})=\lim_{d\rightarrow d_{0}}\psi'(d_{0};d) =
    \lim_{d\rightarrow d_{0}}\frac{\psi(d)-\psi(d_{0})}{d-d_{0}} =
    \lim_{\Delta d \rightarrow 0}\frac{\Delta G}{\Delta d} = 3.84,
\]
podremos escribir
\[
\Delta G \approx 3.84\Delta d \approx \psi'(d_{0}) \Delta d,
\]
si $\Delta d$ es pequeño. Es decir, $\Delta G$ es ``casi'' directamente proporcional a $\Delta d$,
si  $\Delta d$ es pequeño. Al valor $\psi'(d_{0})$ se le llama \emph{razón de cambio instantánea de
$\psi$ en $d_{0}$}.

En general:

\begin{defical}[Razón de cambio instantánea]
Si una variable $y$ depende de una variable $x$ mediante una función $f\colon I\subset\mathbb{R}
\rightarrow \mathbb{R}$ y, para un $x_{0}$ dado existe:
\[
f'(x_{0}) = \lim_{x \rightarrow x_{0}}f'(x_{0},x) =
    \lim_{\Delta x \rightarrow 0}\frac{\Delta y}{\Delta x} =
    \lim_{x \rightarrow x_{0}}\frac{f(x)-f(x_{0})}{x-x_{0}},
\]
diremos que $f'(x_{0})$ es la \emph{razón de cambio instantánea de $y$ con respecto a $x$}.
\end{defical}

Para $\Delta x$ pequeños, tendremos que:
\[
\Delta y \approx f'(x_{0}) \Delta x.
\]
En otras palabras, para variaciones pequeñas de $x$ respecto de $x_0$, las variaciones de $y$
respecto de $y_0$ son ``casi'' directamente proporcionales a las variaciones de $x$, donde
$f'(x_{0})$ es la constante de proporcionalidad.

\begin{exemplo}[]{}
En la fábrica de autos, cuyo costo $C$ miles de dólares está dado por
\[
C = f(x) = 200 + 11x - 0.000\,12 x^2,
\]
obtuvimos que el costo unitario es igual a:
\[
C_U = f_1(x) = \frac{f(x)}{x}= \frac{2\,000}{x}+11-0.000\,12x.
\]
Si la producción pasó de $5\,000$ a $x$ autos, la razón de cambio de los costos $C$ en el intervalo
de extremos $5\,000$ y $x$ está dada por:
\[
f'(5\,000,x)=\frac{f(x)-f(5\,000)}{x-5\,000}=\frac{-52\,000+11x-0.000\,12x^2}{x-5\,000}.
\]

Además, la razón de cambio del costo unitario $C_U$ en el mismo intervalo será:
\[
\frac{f_1(x)-f_1(5\,000)}{x-5\,000}=\frac{\frac{f(x)}{x}-\frac{f(5\,000)}{5\,000}}{x-5\,000}%
=\frac{\frac{2\,000}{x}+0.2-0.000\,12x}{x-5\,000}.
\]

Al tomar, en estos dos casos, el límite cuando $x$ se aproxima a $5\,000$ o, lo que es lo mismo, el
límite cuando $\Delta x$ se aproxima a $0$, obtendremos las respectivas razones de cambio
instantáneas para un nivel de producción igual a $x=5\,000$ autos:
\begin{gather*}
C'=f'(5\,000)=\lim_{x\to 5\,000}\frac{-52\,000+11x-0.000\,12x^2}{x-5\,000}=9.8,\\
C_U'=f_1'(5\,000)=\lim_{x\to 5\,000}\frac{\frac{2\,000}{x}+0.2-0.000\,12x}{x-5\,000}=-0.000\,2.
\end{gather*}

Por la definición de razón de cambio instantánea, si $\Delta x$ es pequeño, a partir de un nivel de
producción igual a $x=5\,000$ autos tendremos que:
\begin{gather*}
\Delta C\approx f'(5\,000)\Delta x  =9.8\Delta x,\\
\Delta C_U\approx f_1'(5\,000)\Delta x   = -0.000\,2\Delta x.
\end{gather*}

Si $\Delta x = 1\,000$, entonces:
\[
f'(5\,000)\Delta x  =9.8(1\,000)=9\,800
\]
y $\Delta C = 9\,600$. Es decir, a pesar de que $\Delta x$ no es tan pequeño, es cercano a $\Delta
C$.

También
\[
f_1'(5\,000)\Delta x  =-0.000\,2(1\,000)=-0.2
\]
es cercano a $\Delta C_U$ que, en este caso, es
\[
\Delta C_U  =-0.187.
\]
\end{exemplo}

\subsection{Magnitudes marginales en Economía}
En el ejemplo de la empresa de autos que produce $5\,000$ autos por mes, los administradores
quisieran saber cuánto le costaría a la fábrica producir $1$ auto más al mes; es decir, quisieran
averiguar el costo de producción del auto número $5\,001$. A este costo se lo conoce con el nombre
de \emph{costo marginal}.

Para averiguar el costo marginal al nivel de producción de $5\,000$ autos mensuales solo hay que
calcular el incremento del costo en el intervalo $[5\,000, 5\,001]$. En este caso, $\Delta x=1$ y:
\[
\Delta C  = f(5\,001)-f(5\,000) \approx f'(5000) = 9.8,
\]
que aproxima al \emph{costo marginal} para una producción de $x=5\,000$ autos, ya que $\Delta x$ es
pequeño. Es decir, producir un auto más cuesta $9\,800$ dólares aproximadamente. El costo marginal es aproximado por la razón de cambio instantánea del costo, por lo cual a esta magnitud se la llama, por simplicidad, costo marginal.

En cambio, el \emph{costo unitario marginal} para este mismo nivel de producción sería:
\[
\Delta C_U  = f_1(5\,001)-f_1(5\,000)\approx f_1'(5\,000) = -0.000\,2.
\]
Es decir, el costo unitario de producir el auto número $5\,001$ es $-0.000\,2$
aproximadamente y, por ser negativo, hay una ganancia de 20 centavos de dólar cuando se produce el
auto número $5\,001$.

Para ésta y otras magnitudes como costo unitario, ingreso, utilidad, demanda, etcétera, a la razón de cambio instantánea
correspondiente se la llama, respectivamente, costo unitario marginal, ingreso marginal, utilidad marginal, demanda
marginal, etcétera.

\subsection{Elasticidad}
En esta sección, se definió la \emph{elasticidad}. Veamos esta cuestión más de cerca.

Si una variable $y$ depende de otra variable $x$ a través de una función $f\colon\mathbb{R}
\rightarrow\mathbb{R}$, y si variamos el valor de $x$ a partir de un valor inicial $x_{0}$, nos
interesa en qué porcentaje varía $y$ a partir del valor inicial $y_{0} = f(x_{0})$, si $x$ varía en
un porcentaje dado.

Conocida la variación porcentual de $x$:
\[
\frac{\Delta x}{x_{0}}100\%
\]
y la variación porcentual de $y$:
\[
\frac{\Delta y}{y_{0}}100\%,
\]
queremos conocer cómo depende $\frac{\Delta y}{y_{0}}$ de $\frac{\Delta x}{x_{0}}$, donde
\[
\Delta x = x - x_{0}; \quad \Delta y = y - y_0 = f(x) - f(x_{0}),
\]
con $x\neq x_{0}$.

En el ejemplo en el que los costos $y$ en miles de dólares vienen dados a través de la función $f$
definida por
\[
y = f(x) = 2000 + 10x,
\]
donde $x\in[1\,000, 10\,000]$, vimos que si, en un momento dado, se estuvieran fabricando $x_{0} =
4\,800$ autos a un costo de $y_{0} = f(4\,800) = 50\,000$ miles de dólares, nos interesaría saber
en qué porcentaje se incrementarían los costos si eleváramos la producción en un $10\%$.

Recordemos que, cuando $f$ es un polinomio de grado menor o igual a $1$ de la forma $f(x) = mx +
b$, ¡la variación relativa de $y$ es directamente proporcional a la variación relativa de $x$! Es
decir, existe una constante $\eta$ tal que
\[
\frac{\Delta y}{y_{0}} = \eta \frac{\Delta x}{x_{0}}.
\]
A esta constante la llamamos, precisamente, \emph{elasticidad de $y$ respecto de $x$}.

Como
\[
  \eta = \frac{\frac{\Delta y}{y_{0}}}{\frac{\Delta x}{x_{0}}} =
    \frac{x_{0}}{f(x_{0})}\frac{\Delta y}{\Delta x},
\]
tendremos:
\begin{align*}
\eta &= \frac{\frac{\Delta y}{y_{0}}}{\frac{\Delta x}{x_{0}}} =
  \frac{\frac{f(x) - f(x_{0})}{f(x_{0})}}{\frac{x - x_{0}}{x_{0}}} \\[4pt]
  &= \frac{x_{0}}{f(x_{0})}\cdot\frac{(mx + b)-(mx_{0}+b)}{x - x_{0}} \\[4pt]
  &= \frac{mx_{0}}{f(x_{0})}\cdot\frac{x - x_{0}}{x - x_{0}}.
\end{align*}
Es decir:
\[
\eta = \frac{mx_{0}}{f(x_{0})}.
\]

Se dice que $y$ es elástica si $|\eta|>1$, inelástica si $|\eta|<1$ y que la elasticidad es
unitaria si $|\eta| = 1$.

En nuestro ejemplo: $y = f(x) = 2000+10x$, $m = 10$, $x_{0}= 4800$, $y_{0}= f(x_{0}) = 5000$, por
lo que:
\[
\eta = \frac{10\times 4800}{5000} = 0.96.
\]
Entonces
\[
 \frac{\Delta y}{y_{0}} = 0.96\frac{\Delta x}{x_{0}},
\]
por lo que, si la producción se incrementara en un 10\%, es decir si $\frac{\Delta x}{x_{0}} =
10\%$, los costos se incrementarán en $\frac{\Delta y}{y_{0}} = 0.96\times 10\%$; es decir, en un
$9.6\%$.\vspace{\baselineskip}

En el caso general, si calculamos $\eta$ de modo que
\[
 \frac{\Delta y}{y_{0}} = \eta \frac{\Delta x}{x_{0}},
\]
obtendremos que
\[
 \eta = \eta(x_{0},x) = \frac{x_{0}}{f(x_{0})}  \frac{\Delta y}{\Delta x} =
 \frac{x_{0}}{f(x_{0})} \frac{f(x) - f(x_{0})}{x - x_{0}} =
 \frac{x_{0}}{f(x_{0})} \frac{f(x_{0} + \Delta x) - f(x_{0})}{\Delta x},
\]
por lo que $\eta$ depende de $x_{0}$ y de $x \neq x_{0}$. Sin embargo, si existe la razón de cambio
instantánea
\[
 f'(x_{0}) = \lim_{x\rightarrow x_{0}}\frac{f(x) - f(x_{0})}{x - x_{0}} ,
\]
se puede observar que para valores  de $x$ cercanos a $x_{0}$; es decir, cuando $\Delta x$ es
pequeño, los valores que toma $\eta(x_{0},x) $ son muy similares entre sí y se acercan cada vez más
a
\[
 \eta(x_{0}) = \lim_{x\rightarrow x_{0}}\eta(x_{0},x)   =
 \lim_{x\rightarrow x_{0}} \frac{x_{0}}{f(x_{0)}}  \frac{f(x) - f(x_{0})}{x - x_{0}},
\]
\[
 \eta(x_{0}) = \frac{x_{0}}{f(x_{0})} f'(x_{0}).
\]
A $\eta(x_{0}) $ se le llama \emph{elasticidad puntual} de $y$ en $x_{0}$, y si $\Delta x$ es
pequeño, tendremos que
\[
 \frac{\Delta y}{y_{0}} \approx \eta(x_{0})  \frac{\Delta x}{x_{0}}.
\]

\begin{exemplo}[]{}
Luego de un estudio de mercado, se determinó que la demanda $D$ de una marca de lavadoras es de
$f(p)$ unidades si el precio es de $p$ dólares. Como su producción no es rentable si $p < 200$ y la
marca del competidor cuesta $500$ dólares, tendremos que $D(f) = [200,500]$. El estudio reveló
también que
\[
 f(p) = 2\,000 - 0.4p + \frac{22\,500}{p}
\]
mide la demanda de lavadoras cuando el precio de cada una es de $p$ dólares.

El precio actual es de $300$ dólares. Se desea conocer aproximadamente en qué porcentaje aumentaría
la demanda si se hiciera un descuento del $5\%$.

En este caso, tendríamos que $p_{0} = 300$, por lo que:
\[
 D_{0} = f(p_{0}) = 1\,955.
\]

Como lo veremos en el siguiente capítulo, se tiene que
\[
 f'(p) = -0.4 - \frac{22\,500}{p^{2}},
\]
por lo que:
\[
f'(p_{0}) = - 0.65.
\]

Se desea conocer el valor aproximado de $\frac{\Delta D}{D_{0}}$. Como:
\[
 \eta (p_{0}) = \frac{p_{0}}{D_{0}} f'(p_{0}) = \frac{300}{1\,955}(-0.65) =
 \frac{39}{391}\approx -0.1,
\]
tenemos que:
\[
\frac{\Delta D}{D_{0}}\approx \eta (p_{0})\frac{\Delta p}{p_{0}} =
\left(-\frac{39}{391}\right)(-5\%)\approx 0.5\%
\]

Es decir, una rebaja del precio en un $5\%$ significaría un incremento en la demanda de apenas el
$0.5\%$.
\end{exemplo}

\begin{exemplo}[]{}
En el caso de la primera fábrica de autos cuyo nivel de producción mensual pasaría de $5\,000$
autos a $x$ autos, como los costos son $C$ miles de dólares y
\[
C = f(x)= 2\,000+11x-0.000\,12x^2,
\]
la \emph{elasticidad media} de los costos en el intervalo de extremos 5000 y $x$ sería:
\[
\eta (5\,000,x)= \frac{5\,000}{f(5\,000)}\frac{\Delta C}{\Delta x}=
\frac{5\,000}{54\,500}\frac{-52\,000+11x-0.000\,12x^2}{x-5\,000}.
\]

La \emph{elasticidad puntual} de los costos para $x_0 =5000$ será :
\[
\eta (5\,000)= \frac{5\,000}{f(5\,000)}f'(5\,000)= \frac{5\,000}{54\,500}9.8\approx 0.899\,1.
\]
Esto quiere decir que si el nivel de producción variara, digamos en $a\%$, los costos variarían en
$0.899\,1 a\%$ aproximadamente.

Así, si la producción subiera en un $5\%$, los costos variarían en $0.899\,1\times 5\%\approx
4.5\%$. Los costos son, pues, ``poco elásticos'' respecto al nivel de producción.

Un cálculo análogo lo podemos realizar para el costo unitario $C_U$, teniendo en cuenta que:
\begin{gather*}
C_U = f_1(x)= \frac{2\,000}{x}+11-0.000\,12x,\\
f_1(5\,000)=10.8, \\
C_U'=f_1'(x)=-\frac{2\,000}{x^2}-0.000\,12,\\
f_1'(5\,000)=-0.000\,2,
\end{gather*}
la elasticidad media del costo unitario en el intervalo de extremos $5\,000$ y $x$ sería:
\[
\eta_1 (5\,000,x)= \frac{5\,000}{f_1(5\,000)}\frac{\Delta C_U}{\Delta x}=
\frac{5\,000}{10.8}\frac{\frac{2\,000}{x}+0.2-0.000\,12x}{x-5\,000},
\]
y la elasticidad puntual del costo unitario para un nivel de producción $x_0=5\,000$ autos sería:
\[
\eta_1 (5\,000)= \frac{5\,000}{f_1(5\,000)}f_1'(5\,000) =\frac{5\,000}{10.8}(-0.000\,2)\approx -0.1.
\]
Esto quiere decir que, si el nivel de producción aumentara, digamos en un $5\%$, el costo unitario
variaría en $(-0.1)5\%$ aproximadamente, es decir, bajaría en $0.5\%$.
\end{exemplo}

\section{La descripción del movimiento}

Uno de los grandes descubrimientos de la humanidad es el concepto de tiempo. Obedeció a la
existencia de ciclos naturales como el día y la noche, las lunas llenas, las estaciones del año, el
movimiento regular de los astros. Esos ciclos naturales dieron lugar a la determinación de unidades
de medir el tiempo: los días y sus fracciones (horas, minutos y segundos), los meses, los años y,
por consiguiente, a la creación de relojes y calendarios.

En la física, para estudiar un fenómeno descrito por uno o más parámetros que varían con el tiempo,
se hace necesario medir a éste. Para ello se adopta una unidad de medida del tiempo $\UT$, por
ejemplo $\UT$ puede ser un segundo, una hora, un año, etcétera, y se escoge un instante de
referencia a partir del cual se inicia la medición del tiempo. Así, si han transcurrido $t \UT$,
diremos que estamos en el instante $t$. La cantidad $t$ es pues un número, por lo cual el tiempo
puede ser representado gráficamente en la recta real, cuyo origen representa el instante en que se
empezó a medir el tiempo y un punto $T$ de abscisa $t$, representará al que llamamos instante $t$.
Hemos establecido así \emph{un sistema de medición del tiempo}. Podemos ahora introducir la
variable $t$ para describir el tiempo en el estudio de los fenómenos en los que esté involucrado.

Por ejemplo, si queremos estudiar el movimiento de una partícula a lo largo de una recta, una vez
que hemos adoptado una unidad de longitud $(\UL)$, podemos escoger un punto de referencia $O$ en la
recta y un sentido a partir de $O$, digamos a la derecha de $O$, que se lo considerará positivo,
por lo que podemos asumir que el movimiento se realiza a lo largo de una recta real. Escogida una
unidad de tiempo $(\UT)$, y si consideramos que $t$ es el número de unidades de tiempo transcurrido
desde el tiempo inicial $\tau_{0}$, el movimiento de la partícula quedará descrito siempre que
podamos establecer la posición $x = f(t)$ de la partícula en cada instante $t\geq \tau_{0}$. La
posición $x = f(t)$ es también denominada \emph{coordenada de la partícula en el instante $t$}.

Si la partícula que estaba en $x_{0} = f(\tau_0)$ en el instante $\tau_0$ pasó a ocupar la posición
$x = f(t)$ en el instante $t$, diremos que su desplazamiento es
\[
\Delta x = \Delta x(\tau_0, t) = f(t) - f(\tau_0).
\]
Esto sucederá en el lapso
\[
\Delta t = \Delta t(\tau_0, t) = t - \tau_0>0.
\]

Notemos que si $\Delta x>0$, la partícula se ha desplazado a la derecha y si $\Delta x<0$, a la
izquierda de donde estaba en el instante $\tau_0$; esto es, a la derecha de $x_{0}= f(\tau_0)$ o a
la izquierda de $x_{0} = f(\tau_0)$, respectivamente.

\subsection{El concepto de velocidad}

Cuando una partícula se mueve a lo largo de la recta, un observador puede opinar que ésta se mueve
rápidamente, otro que se mueve lentamente. El concepto de rapidez tiene pues cierta carga de
subjetividad, pero se la puede medir objetivamente gracias a la cantidad llamada \emph{velocidad}.
Veamos el concepto de velocidad en dos casos.

\subsubsection{El movimiento uniforme}

Supongamos que medimos la longitud en metros y el tiempo en segundos. Si $x=f(t)$, con $\tau_0\leq
t$, es la posición de una partícula, entre un instante $t_0$ y otro instante $t$, $\tau_0\leq t_0<
t$, habrá transcurrido un \emph{lapso} de $\Delta t$ segundos, donde $\Delta t= t-t_0$.

En ese lapso, se habrá producido un desplazamiento de la partícula desde la posición $x_0=f(t_0)$
hasta la posición $x=f(t)$. El desplazamiento será de $\Delta x$ metros, donde $\Delta x=
x-x_0=f(t)-f(t_0)$.

\begin{defical}[Movimiento uniforme]
Diremos que el movimiento es \emph{uniforme} si $\Delta x$ es directamente proporcional a $\Delta
t$, es decir si existe una constante de proporcionalidad $v$ tal que
\begin{equation*}
	\Delta x= v\Delta t.
\end{equation*}
\end{defical}

Esto equivale a decir que el cociente
\begin{equation*}
	\frac{\Delta x}{\Delta t}= \frac{\Delta x(t_0,t)}{\Delta t(t_0,t)} = v
\end{equation*}
es constante.

Evidentemente, en este caso, el valor absoluto de $v$, es decir $|v|$, nos dará una idea clara de
cuán rápido se mueve la partícula, mientras que el signo de $v$ nos indica si la partícula se mueve
de ``izquierda a derecha'', en el caso de que $v>0$ y, en sentido contrario, si $v<0$.

El número $v$ nos sirve entonces para medir la \emph{velocidad} de la partícula. Diremos que $v
\UV$ es la velocidad de la partícula. Como
\begin{equation*}
	v=\frac{\Delta x}{\Delta t},
\end{equation*}
se puede escribir simbólicamente, en este caso, lo siguiente:
\begin{equation*}
	\text{velocidad de la partícula}=\frac{\text{desplazamiento de la partícula}}{\text{lapso transcurrido}}
\end{equation*}
o también
\begin{equation*}
	v \UV= \frac{\Delta x\text{ m}}{\Delta t\text{ s}}=\frac{\Delta x}{\Delta t}\frac{\text{m}}{\text{s}}
\end{equation*}
y como $v=\frac{\Delta x}{\Delta t}$, es cómodo escribir $\frac{\text{m}}{\text{s}}$ en vez de
$\UV$, la unidad de la velocidad. Si se escoge otra unidad $\UL$ del desplazamiento $\Delta x$ y
otra unidad $\UT$ del lapso $\Delta t$, se tendrá otra unidad de la velocidad $\UV$.

En general, si la longitud se mide en $\UL$ y el tiempo en $\UT$, pondremos $\frac{\UL}{\UT}$ en
vez de $\UV$.

Es fácil notar que, si $x=f(t)$ es un polinomio de grado menor o igual a 1, el movimiento es
uniforme.

En efecto, en este caso:
\[
x=f(t)=at+b,
\]
donde $a$ y $b$ son constantes. Por lo tanto:
\begin{equation*}
\frac{\Delta x}{\Delta t}= \frac{\Delta x(t_0,t)}{\Delta t(t_0,t)}=\frac{x-x_0}{t-t_0}=\frac{f(t)-f(t_0)}{t-t_0}%
=\frac{(at+b)-(at_0+b)}{t-t_0}=a.
\end{equation*}
Es decir, $v=a$ es la velocidad y no depende de $t_0$ ni de $t$.

Notemos que, sin importar el valor de $t_0$, para valores iguales de $\Delta t$, se obtiene valores
iguales para el desplazamiento, ya que  $\Delta x= a\Delta t$. Esto justifica el nombre de
\emph{movimiento uniforme}.

Por otro lado, si $\Delta t =1$, tendremos que  $\Delta x= v$, es decir que $v$ m es el
desplazamiento de la partícula en 1 segundo. En general, $v \UL$ es el desplazamiento en el lapso
$1 \UT$.

\begin{exemplo}[]{}
\begingroup
\itshape La empresa de transporte pesado TRANSLIT despachó desde Guayaquil el camión N$^\text{o
}17$ a las 8h00 con una pequeña carga destinada a un cliente en Balzar, y otra muy pesada a un
cliente en Santo Domingo. El chofer del camión es cambiado en Quevedo. El recorrido hacia Santo
Domingo es realizado por otro chofer.

El cliente que vive en Balzar llamó para indicar que recibió el despacho a las 10h00, aunque se le
había prometido que se le entregaría el envío a las 9h30. Se le explicó que el camión estaba
completamente cargado por lo que debió viajar lentamente. La empresa deberá telefonear al chofer
que espera en Quevedo para indicarle la hora aproximada de llegada del camión para el cambio de
conductor. ?`A qué hora llegará el camión a Quevedo? ?`A qué hora se espera que el camión llegue a
Santo Domingo?
\endgroup

\vspace{0.6\baselineskip}%
Para resolver el problema, observemos la información con que contamos
sobre las distancias entre las ciudades:
\begin{itemize}
\item \emph{Guayaquil-Balzar}: 96 km,
\item \emph{Balzar-Quevedo}: 72 km,
\item \emph{Quevedo-Santo Domingo}: 120 km,
\item \emph{Guayaquil-Quevedo}: 168 km,
\item \emph{Guayaquil-Santo Domingo}: 288 km.
\end{itemize}

Ahora, definamos las cantidades involucradas en el problema:\vspace{\baselineskip}\\
{\setlength\tabcolsep{3pt}
\begin{tabular}{r p{0.85\textwidth}}
  $t:$ & \textsl{instante en horas en que se da cada suceso; es el tiempo transcurrido desde las 0 horas
    del día del viaje.}\\
  $\tau_0:$ & \textsl{instante que coincide con las $0$ horas del día del viaje.}\\
  $t_0:$ & \textsl{instante inicial en el que el camión inició su recorrido desde
    Guayaquil.}\\
  $x:$ & \textsl{posición del camión en la carretera Guayaquil-Balzar-Quevedo-Santo Domingo;
    es la distancia recorrida en kilómetros desde Guayaquil al punto en la carretera en el momento $t$.}\\
  $x_0:$ & \textsl{posición inicial del camión  en Guayaquil al tiempo $t_0$.}\\
  $x_1:$ & \textsl{posición  del camión al pasar por Balzar.}\\
  $t_2:$ & \textsl{hora de llegada a Quevedo.}\\
  $t_3:$ & \textsl{hora de llegada a Santo Domingo.}
\end{tabular}}
\vspace{\baselineskip}

Bajo el supuesto de que el movimiento del camión será uniforme aproximadamente, la posición del
camión en cualquier momento $t$ puede ser descrito por la función $f$ definida por:
\begin{equation*}
	x=f(t)= at + b.
\end{equation*}
Para poder calcular la posición del camión, debemos, entonces, determinar $a$ y $b$.

En primer lugar, $t_0 = 8$, pues el camión inicia el recorrido a las ocho de la mañana. Entonces:
$x_0= f(8)$. Como en ese momento, el camión aún no se ha movido, $f(8) = 0$, por lo que $x_0 = 0$.

En segundo lugar, a las diez de la mañana, el camión llegó al Balzar, después de recorrer $96$
kilómetros. Por lo tanto:
\[
x_1 = f(10) = 96.
\]
Entonces:
\begin{equation*}
\left\{
\begin{matrix}
0 & = & 8a & + & b\\
96 & = & 10a& + & b
\end{matrix}
\right.
\end{equation*}
Al restar la primera ecuación de la segunda y, luego de dividir el resultado por $2$, se obtendrá
que $a=48$ y, por consiguiente, que $b=-384$, pues $b = -8a$.

Entonces, tenemos que:
\begin{equation*}
x = f(t)= 48t-384.
\end{equation*}

Ahora hallemos $t_2$. Éste es el momento en que el camión llega a Quevedo. Como esta ciudad está a
$168$ kilómetros de Guayaquil, $t_2$ debe verificar la siguiente igualdad:
\begin{equation*}
x_2 = 168 = f(t_2)= 48t_2 - 384,
\end{equation*}
lo que nos da que:
\begin{equation*}
t_2 =\frac{168+384}{48}=\frac{552}{48}=11.5.
\end{equation*}
En otras palabras, el camión llegará a Quevedo a las 11h30.

Finalmente, para determinar la hora de llegada a Santo Domingo, debemos calcular $t_3$, que
satisface la igualdad
\begin{equation*}
x_3 =288= f(t_3)= 48t_3-384,
\end{equation*}
pues la distancia entre Guayaquil y Santo Domingo es de $288$ kilómetros. Por lo tanto:
\begin{equation*}
t_3 =\frac{228+384}{48}=\frac{672}{48}=14.
\end{equation*}
Es decir, se espera que el camión llegue a Santo Domingo a las 14h00.

Así, pues, la empresa TRANSLIT deberá telefonear al chofer que espera en Quevedo  para indicarle
que el camión N$^\text{o }17$ llegará aproximadamente a las 11h30 para el cambio de conductor. Si
la empresa TRANSLIT es eficiente también telefoneará a sus oficinas en Santo Domingo para que
avisen al cliente que su carga estará llegando alrededor de las 14h00.
\end{exemplo}

\subsection{Caso general: movimiento no-uniforme}

Si el cociente
\[
 v_{m}(t_{0},t) = \frac{\Delta x}{\Delta t} = \frac{\Delta x(t_{0},t)}{\Delta t(t_{0},t)}
\]
no es constante, diremos que el movimiento es \emph{no-uniforme} y a $v_{m}(t_{0},t) $ le
llamaremos \emph{velocidad media de la partícula en el intervalo de tiempo $[t_{0},t]$}.

Obviamente,
\[
\Delta x = v_{m}(t_{0},t)\Delta t,
\]
por lo que mientras mayor sea $|v_{m}(t_{0},t)|$ más rápido será el movimiento de la partícula y viceversa, lo que justifica su nombre.

Si el lapso considerado $\Delta t$ toma valores cada vez más pequeños para ciertas funciones de
posición $f\colon\mathbb{R}\rightarrow\mathbb{R},t\mapsto x=f(t)$, puede suceder que los
correspondientes valores de la velocidad media $v_{m}(t_{0})$ sean muy similares y se acerquen al
límite de $v_{m}(t_{0},t)$ cuando $t$ tiende a $t_{0}$ (o cuando $\Delta t $ tiende a 0) que lo
notaremos $v_{0} = g(t_{0})$. Es decir
\[
v_{0} = g(t_{0}) = \lim_{t \rightarrow t_{0}}v_{m}(t_{0},t) = \lim_{\Delta t \rightarrow 0}\frac{\Delta x}{\Delta t}.
\]
A $v_{0} = g(t_{0})$, que viene dada en $(\UL/ \UT)$, le llamaremos \emph{velocidad instantánea} de la partícula en el instante $t_{0}$.

Dado que para valores pequeños de $\Delta t$,
\[
v_{0} = g(t_{0})\approx v(t_{0},t) = v(t_{0},t_{0} + t),
\]
tendremos que:
\[
\Delta x \approx  v_{0} \Delta t,
\]
lo que justifica que a $v_{0} = g(t_{0})$ se le llame \emph{velocidad instantánea en el instante
$t_{0}$}.

En efecto, para un $\Delta t$ dado, mientras mayor sea $v_{0}$, más grande será el desplazamiento
que se produzca en el lapso $\Delta t$, lo que indica que la partícula se mueve más rápidamente.
Por otra parte, $x_{0}(\UL)$ es el desplazamiento que tendría si no se acelerara o frenara a la
partícula o, lo que es lo mismo, si sobre la partícula no actuara fuerza alguna.

\begin{exemplo}[]{}
\begingroup
\itshape Si se deja caer un objeto desde lo alto de un rascacielos de $122.5\metros$ de altura, sea
$x = f(t)$ metros la altura del objeto medida desde el suelo después de $t$ segundos. Entonces, las
leyes de Newton nos dicen que:
\begin{equation*}
x=f(t) = 122.5-4.9t^2.
\end{equation*}
?`Con qué velocidad choca el objeto con el suelo y en cuánto tiempo lo hace? ?`Cuál es la velocidad
media con la que ha hecho el recorrido?
\endgroup

\vspace{0.5\baselineskip}%
Sea $t_1$ segundos el tiempo que tarda el objeto en llegar al suelo.
Podemos determinar $t_1$ de la igualdad
\begin{equation*}
	x_1=f(t_1) =0,
\end{equation*}
puesto que $0$ es la altura del objeto al llegar al suelo. Entonces:
\begin{equation*}
	x_1=0=f(t_1) = 122.5-4.9t_1^2,
\end{equation*}
lo que nos da que $t_1^2=25$ o $t_1=5$. La otra solución de la ecuación cuadrática es $-5$. Aunque
es solución de la ecuación, no es solución del problema, pues no hay como asignar un significado a
$t_1 = -5$.

En resumen, el objeto llegará al suelo en $5$ segundos.

Como $v_1= x_1'= f'(t_1)$ es la velocidad del objeto en el instante $t_1$, podemos escribir lo
siguiente:
\begin{align*}
v_1=  f'(t_1)&=\lim_{t\to t_1}\frac{f(t)-f(t_1)}{t-t_1}\\
&=\lim_{t\to t_1}\frac{(122.5-4.9t^2)-(122.5-4.9t_1^2)}{t-t_1}\\
&=\lim_{t\to t_1}\frac{-4.9(t^2-t_1^2)}{t-t_1}\\
&=\lim_{t\to t_1}\frac{-4.9(t+t_1)(t-t_1)}{t-t_1}\\
&=\lim_{t\to t_1}(-4.9)(t+t_1)\\
&=-9.8t_1.
\end{align*}
Es decir:
\begin{equation*}
	v_1=  f'(t_1)=-9.8t_1.
\end{equation*}
Para $t_1=5$, tendremos que:
\begin{equation*}
	v_1=  f'(5)=-9.8(5)=-49.
\end{equation*}
Es decir, el objeto llegará al suelo con una velocidad de $-49$ metros por segundo.

El signo negativo de la velocidad indica que el movimiento es hacia abajo, puesto que la altura se
mide desde el suelo hacia arriba.

Finalmente, calculemos la velocidad media en el intervalo $[0,t_1]$:
\begin{equation*}
	v_m(0, t_1)= \frac{x_1-x_0}{t_1-0}= \frac{f(t_1)-f(t_0)}{t_1}= \frac{(122.5-4.9t_1^2)-122.5}{t_1}=-4.9t_1.
\end{equation*}

Para $t_1=5$, tendremos $v_m(0, 5)=-4.9(5)=24.5$.

Es decir, la velocidad media del objeto en el lapso $[0,5]$ será de $24.5$ metros por segundo.
\end{exemplo}

\subsubsection{Notas importantes}

\begin{enumerate}[leftmargin=*]
\item Solo por facilitar la exposición hemos supuesto que $t_{0}<t$. Nada impide que $t<t_{0}$.
    Solo que en este caso $\Delta t = t - t_{0} < 0$. Es por eso que, en la definición de
    velocidad instantánea, pusimos:
\[
v(t_{0}) = \lim_{\Delta t \rightarrow 0}\frac{\Delta x(t_{0},t)}{\Delta t(t_{0},t)} =
      \lim_{t \rightarrow t_{0}}\frac{ x(t) - x(t_{0})}{t - t_{0}}
\]
y no hace falta poner $\lim_{\Delta t \rightarrow 0+}$ o $\lim_{ t \rightarrow t_{0}+}$.

\item $\Delta t$ representa el lapso (de tiempo, por supuesto) o el intervalo $[t_{0},t]$ si $t
    > t_{0}$, o $[t,t_{0}]$ si $t < t_{0}$, en cuyo caso $\Delta t < 0$.

\item ``Acelerar'' equivale a ir cada vez más rápidamente.

\item ``Frenar'' equivale a ir cada vez más lentamente.

\item El signo de $\Delta x$ da el sentido del desplazamiento: si $\Delta x > 0$ quiere decir
    que $x(t)$ está a la derecha de $x(t_{0})$ y viceversa. Por consiguiente, como $v(t_{0}) >
    0$, se tiene que
    \[
    \frac{\Delta x(t_{0},t)}{\Delta t(t_{0},t) }> 0
    \]
    para $|\Delta t|$ pequeño. Esto significa que la partícula se mueve de izquierda a derecha.
    Por el contrario, si $v(t_{0}) < 0$, el movimiento será de derecha a izquierda.
\end{enumerate}

\section{Conclusión}
En los cuatro problemas que hemos tratado en este capítulo, hemos llegado a establecer una función
$\funcjc{f}{I\subset\mathbb{R}}{\mathbb{R}}$, donde $I$ es un intervalo abierto, para la cual, si
$a\in I$, existe el límite
\[
\limjc{\frac{f(x) - f(a)}{x - a}}{x}{a}.
\]
A este límite le hemos representado con $f'(a)$ y se le denomina la \emph{derivada} de $f$ en $a$.
En el siguiente capítulo, será objeto de estudio pues, como lo hemos visto, es una herramienta
poderosa que resuelve una clase amplia de problemas. Un capítulo más adelante, veremos algunas de
sus principales aplicaciones.

\begin{multicols}{2}[\section{Ejercicios}]
\begingroup
\small
\begin{enumerate}[leftmargin=*]
\item Hallar, si existe, la ecuación de la recta tangente al gráfico de la función $f$ en el
    punto $P(a,f(a))$:
    \begin{enumerate}[leftmargin=*]
    \item $f(x) = 2x^2 + 3x - 5$, $a = 2$.
    \item $f(x) = x^3 - x$, $a = -2$.
    \item $\displaystyle f(x) = \frac{3x - 2}{x}$, $a = 1$.
    \item $\displaystyle f(x) = \frac{1}{x^2}$, $a = 1$.
    \item $\displaystyle f(x) = \sqrt{2 + x}$, $a = 0$.
    \item $f(x) = 2(x + 1)$, $a = -2$.
    \item $\displaystyle f(x )= \sqrt[3]{x^2}$, $a = 0$.
    \item $\displaystyle f(x) = 2\sen 3x$, $\displaystyle a = \frac{\pi}{6}$.
    \end{enumerate}

\item ?`Para qué valores de $a$, el gráfico de $f$ tiene una recta tangente en $P(a,f(a))$. Para
    estos valores, ?`cuál es la pendiente $m_a$?
    \begin{enumerate}[leftmargin=*]
    \item $\displaystyle f(x) = x^3 + 2x + 1$.
    \item $\displaystyle f(x) = \frac{1}{x}$.
    \item $\displaystyle f(x) = \sqrt{x}$.
    \item $\displaystyle f(x) = x\sqrt[3]{x}$.
    \item $\displaystyle f(x) = \frac{1}{1 + x^2}$.
    \item $\displaystyle f(x) = \sen 2x$.
    \item $\displaystyle f(x) = \frac{x + 1}{x - 1}$.
    \item $\displaystyle f(x) = \frac{x^3}{x^2 + 3}$.\\
          \textsc{Sugerencia}: primero realice la división; así $f(x)$ se expresará como la
          suma de un polinomio y una expresión racional más sencilla.
    \item $\displaystyle f(x) = \frac{3x^2 - 5}{2x^2 + 1}$.
    \item $\displaystyle f(x) = \cos 2x$.
    \item $\displaystyle f(x) = x\sqrt{x}$.
    \end{enumerate}

\item Para la función $f$ y $x_0\in\mathbb{R}$ dados, y para $\Delta x \in \{0.001,\ 0.01,\
    0.1,\ 1\}$, calcule $\Delta y$ si $y = f(x)$. Para los mismos valores $\Delta x$ y $x_0$,
    calcule los incrementos relativos $\displaystyle \frac{\Delta x}{x_0}$ y
    $\displaystyle\frac{\Delta y}{y_0}$, donde $y_0 = f(x_0)$.

    Resuma los resultados obtenidos en una tabla en la cual los incrementos relativos se
    expresen porcentualmente.
    \begin{enumerate}[leftmargin=*]
    \item $\displaystyle f(x) = 3x + 5$, $x_0 = -3$.
    \item $\displaystyle f(x) = 2x^2 - 3x + 4$, $x_0 = 1$.
    \item $\displaystyle f(x) = x^3 - 3$, $x_0 = 2$.
    \item $\displaystyle f(x) = \frac{1}{x}$, $x_0 = 1$.
    \item $\displaystyle f(x) = \sen \pi x$, $\displaystyle x_0 = \frac{1}{3}$.
    \item $\displaystyle f(x) = \cos \pi x$, $\displaystyle x_0 = \frac{1}{6}$.
    \end{enumerate}

\item Para la función $f$ dada, observe que la razón de cambio promedio en un intervalo de
    extremos $x_0$ y $x_0 + \Delta x$, $f'(x_0, x_0 + \Delta x)$, aproxima a la razón de cambio
    instantánea en $x_0$, $f'(x_0)$, tomando $10$ valores de $\Delta x$ cercanos a $x_0$.
    \begin{enumerate}[leftmargin=*]
    \item $\displaystyle f(x) = 5x - 3$, $x_0 = 2$.
    \item $\displaystyle f(x) = x^2 - x + 1$, $x_0 = 1$.
    \item $\displaystyle f(x) = x^3 + x + 3$, $x_0 = -1$.
    \item $\displaystyle f(x) = \frac{1}{x}$, $x_0 = 1$.
    \end{enumerate}

\item Si los costos de una fábrica de refrigeradoras que produce $x$ cientos de esos
    electrodomésticos por mes es de $C$ miles de dólares, calcule el costo marginal y el costo
    unitario marginal si en un mes dado se produjeron tres mil aparatos. Se conoce que $x \in
    [10,100]$.
    \begin{enumerate}[leftmargin=*]
    \item $f(x) = 10 + 20x$.
    \item $f(x) = 10 + 20x - 0.000\, 1x^2$.
    \end{enumerate}

\item Si la fábrica del ejercicio precedente vende su producción total a un concesionario a un
    precio $p$ miles de dólares por cada cien refrigeradoras, y si $p(x) = 30 - 0.002x$,
    calcule las funciones de ingreso $I$ y de utilidades $U$, así como las de los ingresos
    marginal $I'$ y de utilidades marginales $U'$. Para un nivel de producción $x_0 = 30$,
    interprete los valores que obtenga para $I'(x_0)$ y $U'(x_0)$. Compare estos valores con
    $I(x_0 + 1) - I(x_0)$ y con $U(x_0 + 1) - U(x_0)$, respectivamente.

\item Si para una empresa con un nivel de producción de $x$ unidades por semana, los costos $C$
    y los ingresos $I$ en miles de dólares están dados por
    \[
      C = 2\,000 x + 6\,000 \yjc I = 10x(1\,000 - x),
    \]
    determine la función de utilidades, así como los costos e ingresos marginales cuando el
    nivel de producción es de $x_0 = 50$. Compare la utilidad marginal con el incremento de las
    utilidades al subir la producción en una unidad.

\item Realice los cálculos pedidos en el ejercicio precedente si $x_0 = 100$ y si $x \in
    [20,110]$:
    \begin{enumerate}[leftmargin=*]
    \item $C(x) = 2x^2 + 650$, $I(x) = 1\,000x - 3x^2$.
    \item $C(x) = x^2 + 30x + 100$, $I(x) = 450x - x^2$.
    \item $C(x) = 2x^3 - 50x^2 + 30x + 12\,000$, $I(x) = x(20\,000 - x)$.
    \end{enumerate}

\item Si a un precio de $p$ dólares la demanda de un bien es de $x$ unidades y si $x = D(p)$,
    halle la elasticidad de la demanda para el precio $p_0$ dado. ?`Cómo varía la demanda si el
    precio baja $5\%$? ?`Y si sube $5\%$?
    \begin{enumerate}[leftmargin=*]
    \item $\displaystyle D(p) = 1\,200 - 5p$, $10 \leq p \leq 200$, $p_0 = 100$.
    \item $\displaystyle D(p) = (30 -p)^2$, $1 \leq p < 30$, $p_0 = 20$.
    \item $\displaystyle D(p) = 1\,000\sqrt{40 - p}$, $1 \leq p < 40$, $p_0 = 10$.
    \item $\displaystyle D(p) = 2\,000 + \frac{1\,000}{\sqrt{3p + 4}}$, $0 < p < 100$, $p_0
        = 20$.
    \end{enumerate}

\item Se lanza un proyectil hacia arriba y la altura $h$ metros es alcanzada luego de $t$
    segundos es de $f(t)$ metros aproximadamente, donde
    \[
      f(t) = 10t - 4.9t^2.
    \]
    Determine cuánto tiempo necesita para llegar al punto más alto y cuánto para regresar al
    punto de partida. Note que el instante en que el proyectil deja de subir es cuando alcanza
    el punto más alto para luego descender.

\item Si el conductor no frena, se ha establecido que en las carreras por las fiestas de Quito,
    el más veloz de los carritos de madera que bajan por la calle Las Casas recorre $x$ metros
    luego de $t$ segundos, donde
    \[
      x = f(t) = 2t + t^2
    \]
    hasta que la velocidad es de $10$ metros por segundo. Luego el movimiento es uniforme. Si
    el recorrido que debe hacer es de $154$ metros, escriba $x$ y la velocidad $v$ metros por
    segundo en función de $t$. ?`En cuánto tiempo llega a la meta?
\end{enumerate}
\endgroup
\end{multicols}
\cleartooddpage[\thispagestyle{empty}]

\chapter{La derivada: definición y propiedades}

\section{Definición}
\begin{defical}[Derivada de una función]\label{et:ddDefinicionDerivada}%
Sea $\funcjc{f}{\Dm(f)}{\mathbb{R}}$ y $a\in\Dm(f)$. Si
existe el límite
\begin{equation}
\label{eq:ddDefinicionDerivada}
\limjc{\frac{f(x) - f(a)}{x-a}}{x}{a}
\end{equation}
se dice que $f$ es derivable en $a$. Este límite se representa por $f'(a)$ o por
$\frac{df}{dx}(a)$, y es denominado \emph{derivada} de $f$ en $a$. La función
\[
\funcionjc{f'}{\Dm(f')}{\mathbb{R}}{x}{f'(x),}
\]
donde $\Dm(f')$ es el subconjunto de $\Dm(f)$ donde $f'$ existe, es denominada la \emph{derivada}
de $f$.
\end{defical}

De esta definición se desprende que si $f$ es derivable en $a$, entonces es continua en $a$. En
efecto, como $a$ está en el dominio de $f$, solo hay que probar que:
\[
\limjc{f(x)}{x}{a}.
\]

Para ello, procedamos de la siguiente manera:
\begin{align*}
f(x) &= [f(x) - f(a)] + f(a) \\[4pt]
  &= \frac{f(x) - f(a)}{x - a}\cdot (x - a) + f(a),
\end{align*}
siempre que $x\neq a$. Entonces, como $f$ es derivable en $a$, existe el límite de la fracción
\[
\frac{f(x) - f(a)}{x - a},
\]
y es igual a $f'(a)$. Entonces:
\[
\limjc{f(x)}{x}{a} = f'(a)\cdot (a - a) + f(a) = f(a),
\]
por las propiedades algebraicas de los límites. En resumen, acabamos de demostrar el siguiente
teorema.

\begin{teocal}[Continuidad de las funciones derivables]\label{et:ddDerivadaContinua}%
Si $f$ es derivable en $a$, entonces $f$ es continua en $a$.
\end{teocal}

El recíproco de este teorema no es verdadero. Por ejemplo, la función $f$ definida en $\mathbb{R}$
por $f(x) = |x|$ es continua en $0$, pero no es derivable en $0$. En efecto, el lector puede
demostrar fácilmente que $f$ es continua en $0$. Veamos por qué no es derivable en $0$.

Si lo fuera, existiría el límite
\[
\limjc{\frac{|x| - |0|}{x - 0}}{x}{0}.
\]
Entonces, existirían los dos límites laterales:
\[
\limjc{\frac{|x|}{x}}{x}{0^+}\yjc \limjc{\frac{|x|}{x}}{x}{0^-},
\]
y serían iguales uno a otro. Sin embargo:
\begin{align*}
\limjc{\frac{|x|}{x}}{x}{0^+} &= \limjc{\frac{x}{x}}{x}{0^+} \\[4pt]
  &= \limjc{1}{x}{0^+} = 1,
\end{align*}
pues $x > 0$, y
\begin{align*}
\limjc{\frac{|x|}{x}}{x}{0^-} &= \limjc{\frac{-x}{x}}{x}{0^-} \\[4pt]
  &= \limjc{-1}{x}{0^-} = -1,
\end{align*}
pues $x < 0$.

Por lo tanto, los límites laterales son diferentes entre sí, lo que significa que la función valor
absoluto no es derivable en $0$.

\begin{teocal}[Derivada de funciones localmente iguales]
Sean:
\begin{itemize}
\item[] $a$ un número real;
\item[] $I$ un intervalo abierto que contiene el número $a$;
\item[] $f$ y $g$ dos funciones definidas en $I$.
\end{itemize}
Si $f(a)=g(a)$ y si $f=g$ cerca de $a$, entonces $f$ es derivable en $a$ si y solo si $g$ es derivable en $a$.
\end{teocal}

De este teorema se desprende el siguiente resultado que es de gran utilidad.

\begin{teocal}[Derivabilidad de funciones iguales en un intervalo abierto]
Sean $I$ un intervalo abierto; y, $f$ y $g$ dos funciones iguales en $I$.

Entonces $f$ es derivable en $I$ si y solo si $g$ es derivable en $I$.
\end{teocal}

\begin{exemplo}[Solución]{%
La función constante es derivable en $\mathbb{R}$ y su derivada es la función constante $0$.
}%
Sean $c\in\mathbb{R}$ y $\funcjc{f}{\mathbb{R}}{\mathbb{R}}$ definida por:
\[
f(x) = c
\]
para todo $x\in\mathbb{R}$. Demostremos que el límite~(\ref{eq:ddDefinicionDerivada}) existe para
cada $a\in\mathbb{R}$.

Para ello, procedamos así. Sean $a\in\mathbb{R}$ y $x\neq a$. Como
\begin{align*}
\frac{f(x) - f(a)}{x - a} &= \frac{c - c}{x - a} \\[4pt]
  &= \frac{0}{x - a} = 0,
\end{align*}
el límite~(\ref{eq:ddDefinicionDerivada}) existe para todo $a\in\mathbb{R}$ y es igual a $0$. Esto
significa que
\[
f'(a) = 0
\]
para todo $a\in \mathbb{R}$. Es decir, $f'$ es la función constante nula.
\end{exemplo}

\begin{exemplo}[Solución]{%
La función identidad es derivable en $\mathbb{R}$ y su derivada es la función constante
$1$.
}%
Sea
\[
\funcionjc{I}{\mathbb{R}}{\mathbb{R}}{x}{x.}
\]
Demostremos que el límite~(\ref{eq:ddDefinicionDerivada}) existe para cada $a\in\mathbb{R}$.

Para ello, procedamos de la siguiente manera. Sean $a\in\mathbb{R}$ y $x\neq a$. Como
\begin{align*}
\razoncambio{I}{x}{a} &= \razoncambio{}{x}{a} = 1,
\end{align*}
el límite~(\ref{eq:ddDefinicionDerivada}) existe para todo $a\in\mathbb{R}$ y es igual a $1$. De
manera que
\[
I'(a) = 1
\]
para todo $a\in\mathbb{R}$. De modo que la derivada de $I$ es la función constante $1$.
\end{exemplo}

\begin{exemplo}[Solución]{%
Calcular la derivada de la función $\tan$ en $0$
}%
Debemos calcular el límite de la fracción:
\[
\razoncambio{\tan}{x}{0}
\]
cuando $x$ se aproxima a $0$.

Como $\tan 0 = 0$, entonces:
\begin{align*}
\razoncambio{\tan}{x}{0} &= \frac{\tan x}{x} \\[4pt]
  &= \frac{\sen x}{x}\cdot\frac{1}{\cos x}.
\end{align*}
Pero
\[
\limjc{\frac{\sen x}{x}}{x}{0} = 1 \yjc \limjc{\frac{1}{\cos x}}{x}{0} = \frac{1}{\cos 0} = 1.
\]
Entonces:
\[
\tan'(0) = \limjc{\razoncambio{\tan}{x}{0}}{x}{0} = 1.
\]
\end{exemplo}

\begin{exemplo}[Solución]{%
Sean $\funcjc{f}{\Dm(f)}{\mathbb{R}}$ y $a\in\Dm(f)$. Entonces $f$ es derivable en $a$ si y
solo si existe el límite
\[
\limjc{\frac{f(a + h) - f(a)}{h}}{h}{0}.
\]
En el caso en que $f$ sea derivable en $a$, se verifica que:
\[
f'(a) = \limjc{\frac{f(a + h) - f(a)}{h}}{h}{0}.
\]
\eijc{-1}
}%
Supongamos que $f$ es derivable en $a$. Sean $x = g(h)=a + h$ y
\[
F(x) = \frac{f(x) - f(a)}{x - a}.
\]
Entonces:
\[
\limjc{g(h)}{h}{0} = \limjc{a + h}{h}{0} = a
\]
y
\[
\limjc{F(x)}{x}{a} = f'(a).
\]
Como $g(h) \neq a$ si $h\neq 0$, por el teorema del cambio de variable de límites, tenemos que
\[
\limjc{\frac{f(a + h) - f(a)}{h}}{h}{0} = \limjc{F(x)}{x}{a} = f'(a).
\]

Recíprocamente, supongamos que el límite
\begin{equation}
\label{eq:ddDefinicionAlternativa}
\limjc{\frac{f(a + h) - f(a)}{h}}{h}{0}
\end{equation}
existe. Sea $h = \varphi(x) = x - a$. Entonces:
\[
\limjc{\varphi(x)}{x}{a} = 0
\]
y $\varphi(x) \neq 0$ si $x \neq a$. Entonces, por el teorema de cambio de variable, tenemos que
\[
\limjc{\frac{f(x) - f(a)}{x - a}}{x}{a} = \limjc{\frac{f(a+h) - f(a)}{h}}{h}{0} = f'(a).
\]
\end{exemplo}

Para demostrar que una función $f$ es derivable en $a$ y calcular $f'(a)$, es común demostrar que
existe el límite~(\ref{eq:ddDefinicionAlternativa}) y calcular su valor.

\begin{exemplo}[Solución]{%
Sea $\funcjc{f}{\mathbb{R}}{\mathbb{R}}$ definida por
\[
f(x) = x^3 - x + 1.
\]
Obtener $f'$.
}%
Sea $x\in\mathbb{R}$. Veamos para qué valores de $x$ existe el
límite~(\ref{eq:ddDefinicionAlternativa}). Para ello, procedamos así. Sea $h\neq 0$:
\begin{align*}
\frac{f(x + h) - f(x)}{h} &= \frac{[(x+h)^3 - (x+h) + 1] - (x^3 - x + 1)}{h} \\[4pt]
  &= \frac{[(x^3 + 3x^2h + 3xh^2 + h^3) - (x+h)] - (x^3 - x)}{h} \\[4pt]
  &= \frac{3x^2 + 3xh^2 + h^2 - h}{h} \\
  &= 3x^2 + 3xh + h^2 - 1.
\end{align*}
Por lo tanto, para todo $x\in\mathbb{R}$, existe el límite~(\ref{eq:ddDefinicionAlternativa})
cuando $h$ se aproxima a $0$, y es igual a
\[
f'(x) = 3x^2 - 1.
\]
Por lo tanto, el dominio de la función $f'$ es $\mathbb{R}$.
\end{exemplo}

\begin{exemplo}[Solución]{%
Obtener la derivada de la función $\sen$.
}%
El dominio de la función $\sen$ es $\mathbb{R}$. Veamos para qué valores $x$ de este dominio existe
$\sen'(x)$. Para ello, procedamos de la siguiente manera:
\begin{align*}
\frac{\sen(x + h) - \sen x}{h} &= \frac{(\sen x\cos h + \sen h\cos x) - \sen x}{h} \\[4pt]
  &= \frac{\cos h - 1}{h}\cdot\sen x + \frac{\sen h}{h}\cdot\cos x,
\end{align*}
con $x\in\mathbb{R}$ y $h\neq 0$. Pero
\[
\limjc{\frac{\sen h}{h}}{h}{0} = 1 \yjc \limjc{\frac{\cos h - 1}{h}}{h}{0} = 0.
\]
Entonces:
\begin{align*}
\sen'(x) &= \limjc{\frac{\sen(x + h) - \sen x}{h}}{h}{0} \\[4pt]
  &= \limjc{\left(\frac{\cos h -1}{h}\cdot\sen x + \frac{\sen h}{h}\cdot\cos x\right)}{h}{0} \\[4pt]
  &= 0\cdot\sen x + 1\cdot\cos x = \cos x
\end{align*}
para todo $x\in\mathbb{R}$. Por lo tanto:
\[
\sen'(x) = \cos x \yjc \sen' = \cos.
\]
\end{exemplo}

\begin{exemplo}[Solución]{%
Obtener la derivada de la función $\cos$
}%
El dominio de la función $\cos$ es $\mathbb{R}$. Veamos para qué valores $x \in\mathbb{R}$ existe
$\cos'(x)$. Para ello, procedamos de la siguiente manera:
\begin{align*}
\frac{\cos(x + h) - \cos x}{h} &= \frac{(\cos x\cos h - \sen h\sen x) - \cos x}{h} \\[4pt]
  &= \frac{\cos h - 1}{h}\cdot\cos x - \frac{\sen h}{h}\cdot\sen x,
\end{align*}
con $x\in\mathbb{R}$ y $h\neq 0$.

Por lo tanto:
\begin{align*}
\cos'(x) &= \limjc{\frac{\cos(x + h) - \cos x}{h}}{h}{0} \\[4pt]
  &= \limjc{\left(\frac{\cos h -1}{h}\cdot\cos x - \frac{\sen h}{h}\cdot\sen x\right)}{h}{0} \\[4pt]
  &= 0\cdot\cos x - 1\cdot\cos x = -\sen x
\end{align*}
para todo $x\in\mathbb{R}$. Por lo tanto:
\[
\cos'(x) = -\sen x \yjc \cos' = -\sen.
\]
\end{exemplo}

\begin{exemplo}[Solución]{%
Sea $\funcjc{f}{[0,+\infty[}{\Rbb}$ definida por $f(x) = \sqrt{x}$. Obtener la derivada de $f$.
}%
Sea $x \in [0,+\infty[$. Veamos para qué valores de $x$ existe $f'(x)$. Para ello, procedamos así:
\begin{align*}
\frac{f(x + h) - f(x)}{h} &= \frac{\sqrt{x + h} - \sqrt{x}}{h} \\
  &= \frac{\left(\sqrt{x + h} - \sqrt{x}\right)\left(\sqrt{x + h} + \sqrt{x}\right)}%
      {h\left(\sqrt{x + h} + \sqrt{x}\right)} \\
  &= \frac{(x + h) - x}{h\left(\sqrt{x + h} + \sqrt{x}\right)} \\
  &= \frac{h}{h\left(\sqrt{x + h} + \sqrt{x}\right)} \\
  &= \frac{1}{\left(\sqrt{x + h} + \sqrt{x}\right)}
\end{align*}
con $h \neq 0$ y $x \geq 0$.

Dado que $f$ no está definida para valores menores que $0$, si $x = 0$, para cualquier valor de $h$
menor que $0$, sucede que $x + h$ es menor que $0$. Entonces, el límite de
\[
\frac{f(x + h) - f(x)}{h}
\]
no existe si $x = 0$.

En cambio, para $x > 0$, se tiene que:
\begin{align*}
f'(x) &= \limjc{\frac{1}{\left(\sqrt{x + h} + \sqrt{x}\right)}}{h}{0} \\
    &= \frac{1}{\sqrt{x + 0} + \sqrt{x}} = \frac{1}{2\sqrt{x}}.
\end{align*}

En resumen, para todo $x > 0$, se verifica que:
\[
(\sqrt{x})' = \frac{1}{2\sqrt{x}} \yjc (\sqrt{\ }') = \frac{1}{2\sqrt{\ }}.
\]
Para $x = 0$, la derivada de la función raíz cuadrada no existe.
\end{exemplo}

\section{Derivadas Unilaterales}

\begin{defical}
Dada  $f:\mathbb{R}\to \mathbb{R}$, notaremos, si existen:
\begin{align*}
f_+'(x_0) &= \lim_{x\to x_0^+}\frac{f(x)-f(x_0)}{x-x_0} \\
          &= \lim_{h\to 0^+}\frac{f(x_0+h)-f(x_0)}{h}\\
f_-'(x_0) &= \lim_{x\to x_0^-}\frac{f(x)-f(x_0)}{x-x_0} \\
          &= \lim_{h\to 0^-}\frac{f(x_0+h)-f(x_0)}{h}.
\end{align*}
Las derivadas $f_+'(x_0) $ y $f_-'(x_0) $ se denominan \emph{derivada de $f$ en $x_0$ por la
derecha} y \emph{derivada de $f$ en $x_0$ por la izquierda}, respectivamente. Representan las
pendientes de las rectas tangentes a la gráfica de $f$ en el punto de coordenadas $(x_0,
f(x_0))$, considerando solamente la gráfica de $f$ a la derecha y a la izquierda,
respectivamente, del punto indicado.
\end{defical}

\begin{teocal}[Derivada unilateral de funciones localmente iguales]
Sean: $a$ un número real; y, $f$ y $g$ dos funciones localmente iguales por la derecha (respectivamente por la izquierda) y tales que $f(a)=g(a)$.

Entonces
\begin{enumerate}
\item Existe $f^{'}_{+}(a)$ (respectivamente $f^{'}_{-}(a)$) si y solo si existe $g^{'}_{+}(a)$ (respectivamente $g^{'}_{-}(a)$).
\item Si estas derivadas unilaterales existen, son iguales.
\end{enumerate}
\end{teocal}


\begin{multicols}{2}[\section{Ejercicios}]
\begingroup
\small
\begin{enumerate}[leftmargin=*]
\item Halle la derivada de la función $\funcjc{f}{D\subseteq \Rbb}{\Rbb}$ dada en el punto $x$
    indicado, usando la definición de derivada:
        \begin{enumerate}[leftmargin=*]
        \item $f(x)= 2x^2+x+1$ en $x=2$.
        \item $f(x) = \frac{2}{x+1}$ en $x=0$.
        \item $f(x)= \frac{x-1}{x}$ en $x=1$.
        \item $f(x) = \sec x$ en $x = 0$.
        \item $f(x) = \csc x$ en $\displaystyle x = \frac{\pi}{2}$.
        \item $f(x) = \lfloor x \rfloor$ en $\displaystyle x = \frac{1}{2}$.
        \item $f(x) = \lfloor x \rfloor$ si $x \not\in\Zbb$.
        \item $f(x) = \lfloor x \rfloor$ si $x = 0$.
        \item $f(x) = \lfloor x \rfloor$ si $x \in\Zbb$.
        \end{enumerate}
\item En los siguientes ejemplos halle, si existen, $f_+'(x_0)$ y $f_-'(x_0)$.
        \begin{enumerate}
        \item $f(x) =
        \begin{cases}
        x^2+x-1 & \text{si $x\leq 1$}\\
        -2x^2+5x-2 & \text{si $x> 1$},
        \end{cases}$\\[5pt]
        $x_0 = 1$.
        \item $f(x) =
        \begin{cases}
       (x+1)^\frac{1}{3} & \text{si $x\leq -1$}\\
        x & \text{si $x> -1$},
        \end{cases}$\\[5pt]
        $x_0 = -1$.
        \item $f(x) = \sqrt{|x-2|}$, $x_0 = -1$.
        \item $f(x) =
        \begin{cases}
        \sqrt{x-1} & \text{si $x\geq 1$}\\
        x^2-3x+2 & \text{si $x< 1$},
        \end{cases}$\\[5pt]
        $x_0 = 1$.
        \end{enumerate}
\item Demuestre que para $\funcjc{f}{\mathbb{R}}{\mathbb{R}}$ y $x_0\in \mathbb{R}$, existe
    $f'(x_0)$ si y solo si:
        \begin{enumerate}
        \item Existen $f_-'(x_0)$ y $f_+'(x_0)$; y
        \item $f_-'(x_0) = f_+'(x_0)$.
        \end{enumerate}
        Además, en el caso de que exista $f'(x_0)$, siempre se verificará que
        \[
            f'(x_0) = f_-'(x_0) = f_+'(x_0).
        \]
\item Pruebe que $f$ no es derivable en el punto indicado.
        \begin{enumerate}
        \item $f(x) =
        \begin{cases}
       \frac{1}{\sqrt{x}} & \text{si $x\geq 1$}\\
        x +1 & \text{si $x<1$},
        \end{cases}
        \quad x = 1$.
        \item $f(x) = \frac{1}{x^2-3x+2}$, $x=1$.
        \item $f(x) = \sqrt[4]{2x-9}$, $x = 4.5$.
         \item $f(x) =
        \begin{cases}
       2x^2+1 & \text{si $x<0$}\\
        \sqrt{x +1 }& \text{si $x\geq 0$},
        \end{cases}
        \quad x = 0$.
        \end{enumerate}
\end{enumerate}
\endgroup
\end{multicols}

\section{Propiedades de la derivada}
Calcular la derivada de una función utilizando únicamente la definición puede resultar muy
engorroso, como el lector puede comprobar en los ejercicios de la sección anterior. Para facilitar
los cálculos, a partir de las propiedades algebraicas de los límites (límite de la suma, del
producto, etcétera), se obtienen propiedades algebraicas de la derivada. Éstas permitirán calcular
la derivada de muchas funciones si se conoce la derivada de funciones ``más simples''.

Por ejemplo, podemos calcular la derivada de $\funcjc{f}{\mathbb{R}}{R}$ definida por
\[
f(x) = x + 2
\]
sabiendo que la derivada de $x$ es $1$, de $2$ es $0$ y que la derivada de la suma de dos funciones
es la suma de las derivadas de cada función. Por lo tanto:
\begin{align*}
f'(x) &= (x + 2)' \\
  &= (x)' + (2)' = 1 + 0 = 1.
\end{align*}
Es decir, la derivada de $f$ es la función constante $1$.

El siguiente teorema enuncia las propiedades de la derivada que se obtienen de la simple aplicación
del concepto de derivada y de las propiedades algebraicas de los límites (ver el teorema
\ref{teol:Algebra} en la página \pageref{teol:Algebra}).

\begin{teocal}[Propiedades algebraicas I]\label{et:ddAlgebraDerivadasI}%
Sean $f$ y $g$ dos funciones reales derivables en $a \in
\mathbb{R}$. Entonces:
\begin{enumerate}
\item \textit{Suma}: $f + g$ es derivable en $a$ y
    \[
      (f + g)'(a) = f'(a) + g'(a).
    \]

\item \textit{Producto}: $fg$ es derivable en $a$ y
    \[
      (fg)'(a) = f'(a)g(a) + f(a)g'(a).
    \]

\item \textit{Inverso multiplicativo}: Si $g(a) \neq 0$, $\frac{1}{g}$ es derivable en $a$ y
    \[
      \left(\frac{1}{g}\right)'(a) = -\frac{g'(a)}{g^2(a)}.
    \]
\end{enumerate}
\end{teocal}

Y de este teorema, el siguiente se obtiene inmediatamente.

\begin{teocal}[Propiedades algebraicas II]\label{et:ddAlgebraDerivadasII}%
Sean $f$ y $g$ dos funciones reales derivables en $a \in \mathbb{R}$ y
$\lambda\in\mathbb{R}$. Entonces:
\begin{enumerate}
\item \textit{Escalar por función}: $\lambda f$ es derivable en $a$ y
  \[
    (\lambda f)'(a) = \lambda f'(a).
  \]

\item \textit{Resta}: $(f - g)$ es derivable en $a$ y
    \[
      (f - g)'(a) = f'(a) - g'(a).
    \]

\item \textit{Cociente}: si $g(a) \neq 0$, $\frac{f}{g}$ es derivable en $a$ y
  \[
    \left(\frac{f}{g}\right)'(a) = \frac{f'(a)g(a) - f(a)g'(a)}{g^2(a)}.
  \]
\end{enumerate}
\end{teocal}

Dada una función $f$, debe estar claro que $f$ es diferente de $f(x)$. El segundo símbolo indica el
valor que toma la función $f$ en $x$. Sin embargo, es común abusar del lenguaje y referirse a la
función $f$ a través de $f(x)$. Si no se especifica a qué conjunto pertenece $x$, se entiende
tácitamente que el dominio de $f$ es el conjunto más grande de $\mathbb{R}$ en el cual $f(x)$ está
definida.

Es así que, si se pide ``calcular la derivada de la función $x^2$'', lo que se pide realizar es el
cálculo de la derivada de la función f definida por $f(x) = x^2$ para todo $x\in\mathbb{R}$.

En general, en vez de decir ``derivar la función $f$'', se puede decir ``derive $f(x)$'', o
``derive $f(t)$'', etcétera. También se podrá decir ``calcule $f'(x)$ o $f'(t)$''.

La ventaja de este abuso del lenguaje es que nos ahorra el introducir un nombre para la función. A
lo largo de este libro, utilizaremos indistintamente ambas formas de referirnos a las funciones,
salvo que pueda haber lugar para confundir la función $f$ con el valor que ésta puede tomar en
algún $x$ en particular.

\begin{exemplo}[Solución]{%
Calcular $(x^2)'$ y $(x^3)'$.
}%
Como $x^2 = x \cdot x$, podemos aplicar la propiedad algebraica de la derivada para la derivada del
producto de dos funciones (teorema~\ref{et:ddAlgebraDerivadasI}).

En este caso, $f(x) = x$ y $g(x) = x$; estas dos funciones son la identidad cuya derivada existe
para todo $x\in\mathbb{R}$ y es igual a $1$. Entonces:
\[
x^2 = f(x)g(x).
\]
Por lo tanto, $x^2$ es derivable en todo $x\in\mathbb{R}$ y su derivada se obtiene de la siguiente
manera:
\begin{align*}
(x^2)' &= (fg)'(x) \\
  &= f'(x)g(x)+ f(x)g'(x)\\
  &= 1\cdot x + x\cdot 1 = x + x,
\end{align*}
  pues $f'(x) = g'(x) = (x)' = 1$. Por lo tanto:
\[
  (x^2)' = 2x
\]
para todo $x\in\mathbb{R}$.

Para calcular la derivada de $x^3$, apliquemos nuevamente la regla de la derivada de un producto,
pero esta vez a $x^2$ y a $x$:
\begin{align*}
(x^3)' &= (x^2 \cdot x)'\\
  &= (x^2)' \cdot x + x^2 \cdot (x)' \\
  &= (2x)(x) + (x^2)(1) = 2x^2 + x^2.
\end{align*}
Por lo tanto:
\[
(x^3)' = 3x^2.
\]
\end{exemplo}

Ahora obtengamos una fórmula general para la derivada de $x^n$ con $n$ un número natural. Para
ello, recordemos el siguiente ``producto notable'':
\[
x^n - a^n = (x - a)(x^{n-1} + x^{n-2}a + x^{n-3}a^2 + \cdots + x^2a^{n-3} + xa^{n-2} + a^{n-1}).
\]

\begin{exemplo}[Solución]{\label{ej:ddDerivadaPotencia}%
Obtener $(x^n)'$ para todo $n\in\mathbb{N}$.
}%
Sean $n\in\mathbb{N}$ y $\funcjc{f}{\mathbb{R}}{\mathbb{R}}$ definida por $f(x) = x^n$ para todo $x
\in\mathbb{R}$. Obtengamos la derivada de $f$ para cualquier $a\in\mathbb{R}$. Sea $x\neq a$:
\begin{align*}
\frac{f(x) - f(a)}{x - a} &= \frac{x^n - a^n}{x - a} \\[4pt]
  &= \frac{(x - a)(x^{n-1} + x^{n-2}a + x^{n-3}a^2 + \cdots +
    x^2a^{n-3} + xa^{n-2} + a^{n-1})}{x - a}\\[4pt]
  &= x^{n-1} + x^{n-2}a + x^{n-3}a^2 + \cdots + x^2a^{n-3} + xa^{n-2} + a^{n-1}.
\end{align*}
Por lo tanto:
\begin{align*}
f'(a) &= \limjc{\frac{f(x) - f(a)}{x - a}}{x}{a} \\
  &= \limjc{\left(x^{n-1} + x^{n-2}a + x^{n-3}a^2 + \cdots + x^2a^{n-3} + xa^{n-2} + a^{n-1}\right)}{x}{a} \\
  &= a^{n-1} + a^{n-2}a + a^{n-3}a^2 + \cdots + a^2a^{n-3} + aa^{n-2} + a^{n-1} \\
  &= a^{n-1} + a^{n-1} + a^{n-1} + \cdots + a^{n-1} + a^{n-1} + a^{n-1} \\
  &= na^{n-1},
\end{align*}
pues la suma tiene $n$ términos. Por lo tanto:
\[
(x^n)' = nx^{n-1},
\]
para todo $x\in\mathbb{R}$ y todo $n\in\mathbb{N}$.

Por ejemplo, $(x^6)' = 6x^5$.
\end{exemplo}

\begin{exemplo}{%
Calcular la derivada de $x^5 + 2x^4 - 3x^3 - 4x^2 + 5x - 1$.
}%
Esta función es la suma y resta de funciones de la forma $ax^n$. Cada una de esas funciones tiene
derivada para cada número real $x$. La propiedad de la derivada para la suma de dos funciones
(teorema~\ref{et:ddAlgebraDerivadasI}), el del producto de un escalar por una función
(teorema~\ref{et:ddAlgebraDerivadasII}) y el de la resta de dos funciones
(teorema~\ref{et:ddAlgebraDerivadasII}) nos permiten proceder de la siguiente manera:
\begin{align*}
(x^5 + 2x^4 - 3x^3 - 4x^2 + 5x - 1)' &= (x^5)' + (2x^4)' - (3x^3)' - (4x^2)' + (5x)' - (1)'\\
  &= 5x^4 + 2(x^4)' - 3(x^3)' - 4(x^2)' + 5(x)' - 0 \\
  &= 5x^4 + 2(4x^3) - 3(3x^2) - 4(2x) + 5(1) \\
  &= 5x^4 + 8x^3 - 9x^2 - 8x + 5.
\end{align*}
Por lo tanto:
\[
(x^5 + 2x^4 - 3x^3 - 4x^2 + 5x - 1)' = 5x^4 + 8x^3 - 9x^2 - 8x + 5
\]
para todo $x\in\mathbb{R}$.
\end{exemplo}%Fin de \ejemplo

\begin{exemplo}[Solución]{%
Sea $\displaystyle f(x) = \frac{1 - x + x^2}{1 + x - x^2}$. Calcular $f'(x)$.
}%
La función $f$ es el cociente de las funciones $g$ y $h$ definidas por
\[
g(x) = 1 - x + x^2 \yjc h(x) = 1 + x - x^2.
\]
El dominio de $f$ es $\Dm(f) = \mathbb{R} - \left\{\frac{1-\sqrt{5}}{2},\frac{1 +
\sqrt{5}}{2}\right\}$, pues las raíces de la ecuación
\[
1 + x - x^2 = 0
\]
son $\frac{1-\sqrt{5}}{2}$ y $\frac{1 + \sqrt{5}}{2}$.

Para todo $x\in\Dm(f)$, se tiene que:
\[
f(x) = \frac{g(x)}{h(x)},
\]
pues $h(x) \neq 0$ para dichos $x$.

Por la propiedad de la derivada de un cociente (teorema~\ref{et:ddAlgebraDerivadasII}) tenemos que:
\begin{align*}
f'(x) &= \left(\frac{g(x)}{h(x)}\right)' \\[4pt]
   &= \frac{g'(x)h(x) - g(x)h'(x)}{h^2(x)} \\[4pt]
   &= \frac{(1 - x + x^2)'(1 + x - x^2) - (1 - x + x^2)(1 + x - x^2)'}{(1 + x - x^2)^2} \\[4pt]
   &= \frac{(0 - 1 + 2x)(1 + x - x^2) - (1 - x + x^2)(0 + 1 - 2x)}{(1 + x - x^2)^2}\\[4pt]
   &= \frac{-(1 - 2x)(1 + x - x^2) - (1 - 2x)(1 - x + x^2)}{(1 + x - x^2)^2} \\[4pt]
   &= \frac{-(1 - 2x)(1 + x - x^2 + 1 - x + x^2)}{(1 + x - x^2)^2}.\\[4pt]
\end{align*}
Por lo tanto:
\[
f'(x) = \frac{2(2x - 1)}{(1 - x + x^2)^2}
\]
para todo $x\in\Dm(f)$.

\begingroup
\renewcommand{\qedsymbol}{}
\begin{proof}[Solución 2]
En este caso, es posible calcular la derivada de $f$ de una manera más simple.

En efecto, la ley de asignación de la función $f$, que es el cociente de dos polinomios de segundo
grado, puede ser expresada de la siguiente manera:
\[
f(x) = \frac{1 - x + x^2}{1 + x - x^2} = -1 + \frac{2}{1 + x - x^2}.
\]
Esta representación se obtiene al dividir el numerador por el denominador y obtener el cociente y
el residuo.

Entonces:
\begin{align*}
f'(x) &= \left(-1 + \frac{2}{1 + x - x^2}\right)' \\[4pt]
   &= 0 + 2\frac{-(1 + x - x^2)'}{(1 +x - x^2)^2} \\[4pt]
   &= 2\frac{-0 - 1 + 2x}{(1 +x - x^2)^2} = \frac{2(2x - 1)}{(1 +x - x^2)^2}.\qedhere
\end{align*}
\end{proof}
\endgroup
\end{exemplo}

\begin{exemplo}[Solución]{%
Sea $f$ una función real definida por
\[
f(x) = \frac{1}{x}\sen x
\]
para $x \neq 0$. Obtener $f'$
}%
El dominio de $f$ es $\Dm(f) = \mathbb{R} - \{0\}$. Para $x \in\Dm(f)$, $f(x)$ es el producto de la
función $\frac{1}{x}$ y $\sen x$. Entonces, por la propiedad de la derivada de un producto de
funciones (teorema~\ref{et:ddAlgebraDerivadasI}) tenemos que
\[
f'(x) = \left(\frac{1}{x}\right)'(\sen x) + \frac{1}{x}(\sen x)'.
\]
Pero
\[
\left(\frac{1}{x}\right)' = -\frac{(x)'}{x^2} = -\frac{1}{x^2}
\]
por la propiedad de la derivada del inverso multiplicativo (teorema~\ref{et:ddAlgebraDerivadasII}),
ya que $x\neq 0$. También tenemos que:
\[
(\sen x)' = \cos x.
\]
Por lo tanto:
\begin{align*}
f'(x) &= -\frac{1}{x^2}\sen x + \frac{1}{x}\cos x \\[4pt]
   &= \frac{x\cos x - \sen x}{x^2}
\end{align*}
para todo $x \neq 0$.
\end{exemplo}

\begin{exemplo}[Solución]{%
Sea $f$ una función polinomial de grado $n = 1$. Es decir,
\[
f(x) = a_0 + a_1x + a_2x^2 + \cdots + a_nx^n = \sum_{k=0}^n a_x^k,
\]
para todo $x\in\mathbb{R}$. Calcular $f'$.
}%
Utilizando inducción matemática sobre el número de términos, se obtiene que la derivada de la suma
de un número arbitrario de funciones es igual a suma de las derivadas de cada una de esas
funciones. Como $f$ puede ser visto como la suma de las $n + 1$ funciones $f_k$ definidas por:
\[
f_k(x) = a_kx^k,
\]
para todo $k\in\{0,1,\ldots,n\}$; entonces:
\[
f'(x) = \left(\sum^n_{k=0} a_kx^k\right)' = \sum_{k=0}^n f_k'(x).
\]
Pero, para $k\in\{1,2,\ldots,n\}$, tenemos que:
\[
f'_k(x) = ka_kx^{k-1},
\]
ya que la derivada de un escalar por una función es igual al producto del escalar por la derivada
de la función (ver teorema~\ref{et:ddAlgebraDerivadasII}) y por el ejemplo
\ref{ej:ddDerivadaPotencia}.

Para $k = 0$, en cambio, tenemos que:
\[
f'_0(x) = (a_0)' = 0,
\]
pues la derivada de una función constante es la función cero.

Por lo tanto:
\[
f'(x) = \sum_{k=1}^n ka_kx^{k-1}
\]
para todo $x\in\mathbb{R}$.
\end{exemplo}

\begin{exemplo}[Solución]{%
Obtener la derivada de $\tan$.
}%
Para todo $x\in\Dm(\tan)$, se tiene que:
\[
\tan x = \frac{\sen x}{\cos x}.
\]
Por lo tanto, para obtener la derivada de tan, podemos aplicar la derivada del cociente de dos
funciones (teorema~\ref{et:ddAlgebraDerivadasII}). Obtendremos lo siguiente:
\begin{align*}
\tan' x &= \left(\frac{\sen x}{\cos x}\right)' \\[4pt]
   &= \frac{\sen' x\cos x - \sen x\cos' x}{\cos^2 x} \\[4pt]
   &= \frac{\cos x \cos x - (\sen x)(-\sen x)}{cos^2 x} \\[4pt]
   &= \frac{\cos^2 x + \sen^2 x}{\cos^2 x} = \frac{1}{\cos^2 x}.
\end{align*}
Por lo tanto:
\[
\tan'(x) = \sec x^{2} = 1 + \tan^2 x,
\]
para todo $x \neq \frac{\pi}{2} + k\pi$ con $k\in\mathbb{Z}$, pues $\Dm(\tan) = \mathbb{R} -
\{\frac{\pi}{2} + k\pi : k\in\mathbb{Z}\}$.
\end{exemplo}

\begin{exemplo}[Solución]{%
Obtener la derivada de $\cot$.
}%
Recordemos que
\[
\cot x = \frac{1}{\tan x}
\]
para todo $x\neq k\pi$ con $k\in\mathbb{Z}$. Entonces, mediante la aplicación de la regla para
derivar el inverso multiplicativo de una función (teorema~\ref{et:ddAlgebraDerivadasII}), tenemos
que:
\begin{align*}
\cot'(x) &= \left(\frac{1}{\tan x}\right)' \\[4pt]
   &= -\frac{\tan' x}{\tan^2 x} \\[4pt]
   &= -\frac{1 + \tan^2 x}{\tan^2 x} = -\left(\frac{1}{\tan^2 x} + 1\right).
\end{align*}
Por lo tanto:
\[
\cot' x = -(1 + \cot^2 x) = -\csc^2 x
\]
para todo $x\neq k\pi$ con $k\in\mathbb{Z}$.
\end{exemplo}

\begin{exemplo}[Solución]{%
Obtener la derivada de $\sec$.
}%
Puesto que
\[
\sec x = \frac{1}{\cos x}
\]
para todo $x \neq \frac{\pi}{2} + k\pi$ con $k\in\mathbb{Z}$, tenemos que:
\begin{align*}
\sec' x &= -\frac{\cos' x}{\cos^2 x} \\[4pt]
   &= -\frac{-\sen x}{\cos^2 x} \\[4pt]
   &= \frac{\sen x}{\cos x}\cdot\frac{1}{\cos x}
\end{align*}
para todo $x \neq \frac{\pi}{2} + k\pi$. Por lo tanto:
\[
\sec' x  = \tan x \sec x
\]
para todo $x \neq \frac{\pi}{2} + k\pi$ con $k\in\mathbb{Z}$.
\end{exemplo}

\begin{exemplo}[Solución]{%
Obtener la derivada de $\csc$.
}%
Para todo $x\neq k\pi$ con $k\in \mathbb{Z}$, se verifica que:
\[
\csc x = \frac{1}{\sen x}.
\]
Por lo tanto:
\begin{align*}
\csc' x &= -\frac{\sen' x}{\sen^2 x} \\[4pt]
   &= -\frac{\cos x}{\sen^2 x} \\[4pt]
   &= -\frac{\cos x}{\sen x}\cdot\frac{1}{\sen x}
\end{align*}
para todo $x \neq k\pi$. Entonces:
\[
\csc' x  = -\cot x \csc x
\]
para todo $x \neq k\pi$ con $k\in\mathbb{Z}$.
\end{exemplo}

\section{Ejercicios}
\begingroup\small
Halle $f'(x)$ usando las propiedades algebraicas de la derivada (teoremas
\ref{et:ddAlgebraDerivadasI} y \ref{et:ddAlgebraDerivadasII}).
\begin{multicols}{2}
\begin{enumerate}[leftmargin=*]
\item $\displaystyle f(x) = 5x^2 + 7$.
\item $\displaystyle f(x) = x^3 + 3x^2 - 5x + 2$.
\item $f(x) = 3x^2-x+1$.
\item $\displaystyle f(x) = 7x^{15} + 8x^{-7}$.
\item $f(x) = \frac{x+1}{x-1}$.
\item $f(x) = \frac{2}{x^2+1}$.
\item $f(x) = x^n-x^{n-1}+1, \ n\geq 2$.
\item $f(x) = \sqrt{x}-\sqrt[3]{x}+1$.
\item $\displaystyle f(x) = \frac{2}{x^2} + \frac{3}{x^3} + \frac{5}{x^5}$.
\item $\displaystyle f(x) = \sqrt{x} + \sqrt[3]{x} + \sqrt[5]{x}$.
\item $\displaystyle f(x) = \frac{2}{3}x^{\frac{3}{2}} - x^{-3} + \frac{5}{x^4}$.
\item $\displaystyle f(x) = x^{\sqrt{3}} - x^{-\sqrt{3}}$.
\item $\displaystyle f(x) = 7x\cos x$.
\item $\displaystyle f(x) = (x^2 + 1)\tan x$.
\item $\displaystyle f(x) = x^2\cot x + 5$.
\item $\displaystyle f(x) = \frac{\sqrt{x}}{\tan x}$.
\item $\displaystyle f(x) = \frac{\sen x + \cos x}{\sen x - \cos x}$.
\end{enumerate}
\end{multicols}
\endgroup

\section{La regla de la cadena o la derivada de la compuesta}
Con la ayuda de las propiedades algebraicas de la derivada se obtienen las derivadas de un
considerable número de funciones. Sin embargo, no todas las funciones se pueden expresar como suma,
resta, producto, división de otras funciones. Muchas funciones se expresan como la composición de
dos o más funciones. La siguiente propiedad de las derivadas nos dice cómo obtener la derivada de
la composición de dos funciones si éstas son derivables. Esta propiedad es conocida como la regla
de la cadena. El porqué de este nombre se verá más adelante.

\begin{teocal}[Regla de la cadena o derivada de la función compuesta]
Sean $f$ y $g$ dos funciones reales tales que existe $f\circ g$, y $a\in\Dm(f\circ g)$. Supongamos
que:
\begin{enumerate}
\item $g$ es derivable en $a$; y
\item $f$ es derivable en $g(a)$.
\end{enumerate}
Entonces la compuesta $f\circ g$ es derivable en $a$. Además:
\[
(f\circ g)'(a) = f'(g(a))\cdot g'(a).
\]
\end{teocal}

\begin{exemplo}[Solución]{%
Calcular la derivada de $\sen^2 x$.
}%
Sea $h$ una función real definida por
\[
h(x) = \sen^2 x
\]
para todo $x\in\mathbb{R}$. Aunque fuera posible expresar $h(x)$ como suma, producto, etcétera de
otras funciones cuyas derivadas conozcamos, no es fácil ver cuáles serían esas funciones. Sin
embargo, mediante la regla de la cadena podemos obtener la derivada de $h$, pues es la composición
de las funciones $f$ y $g$ definidas de la siguiente manera:
\[
f(u) = u^2
\]
para todo $u\in\mathbb{R}$ y
\[
g(x) = \sen x
\]
para todo $x\in\mathbb{R}$.

En efecto, si $x\in\mathbb{R}$, tenemos que:
\begin{align*}
(f\circ g)(x) &= f(g(x)) \\
   &= f(\sen x) = (\sen x)^2 \\
   &= \sen^2 x = h(x).
\end{align*}
Por lo tanto, $h = f\circ g$ y su dominio es $\mathbb{R}$.

Ahora bien, $f$ y $g$ son derivables en todo $x\in\mathbb{R}$. Por lo tanto, $h$ es derivable en
$\mathbb{R}$, y su derivada se calcula de la siguiente manera:
\begin{align*}
h'(x) &= (f\circ g)'(x) \\
   &= f'(g(x))\cdot g'(x).
\end{align*}
Pero
\[
f'(u) = 2u \yjc g'(x) = \cos x.
\]
Por lo tanto
\[
f'(g(x)) = 2g(x) = 2 \sen x.
\]
Entonces:
\[
h'(x) = 2 \sen x \cos x = \sen 2x
\]
para todo $x\in\mathbb{R}$. Por lo tanto:
\[
(\sen^2 x)' = \sen 2x
\]
para todo $x\in\mathbb{R}$.

En otras palabras, la derivada del cuadrado del $\sen$ es dos veces el $\sen$ (como si fuera la
función $x^2$) multiplicada por la derivada del $\sen$, que es el $\cos$.
\end{exemplo}


\begin{multicols}{2}[\section{Ejercicios}]
\begingroup\small
\begin{enumerate}[leftmargin=*]
\item Antes de ejercitarse en la aplicación de la regla de la cadena, conviene llenar los
    casilleros vacíos de la tabla~\ref{tab:dcCompuestas} en la
    página~\pageref{tab:dcCompuestas}.
\item Calcule $f'(x)$.
  \begin{enumerate}[leftmargin=*]
    \item $f(x) = (x^2+x+1)^{\frac{1}{2}}$.
    \item $f(x) = \cos\sqrt{x+1}$.
    \item $\displaystyle f(x) = \frac{\sen x}{21 + \sen x}$.
    \item $\displaystyle f(x) = \sqrt[3]{\cos x + x^2}$.
    \item $\displaystyle f(x) = \sqrt{x} + \sqrt[3]{x^2}$.
    \item $\displaystyle f(x) = \frac{\sqrt{x}}{1 + \sqrt[3]{x^2}}$.
    \item $\displaystyle f(x) = x^2\sqrt[3]{x^4} + x^5\sqrt[6]{x^7}$.
    \item $\displaystyle f(x) = (ax^2 + bx + c)^k$ con $a, b, c, k$ en $\Rbb$.
    \item $\displaystyle f(x) = Ae^{kx}(a\sen x + b\cos x)$ con $A, k, a, b$ en $\Rbb$.
    \item $\displaystyle f(x) = \sqrt{2 + \sqrt{3 + \sqrt{x}}}$.
    \item $\displaystyle f(x) = \sqrt[n]{\frac{ax + b}{ax - b}}$ con $n\in\Nbb$ y $a, b$ en
        $\Rbb$.
    \item $\displaystyle f(x) = \ln\left(\ln\frac{x^2}{5}\right)$.
    \item $\displaystyle f(x) = \frac{x^2 + \tan x}{\sqrt{1 + x^2}}$.
    \item $\displaystyle f(x) = \ln\left|x + e^x\right|$.
    \item $\displaystyle f(x) = \sen(\cos^3 x)\cos(\sen^3 x)$.
    \item $\displaystyle f(x) = \frac{\tan x^2}{\tan^2 x}$.
    \item $\displaystyle f(x) = \frac{\sen^2 x}{\sen x^2}$.
    \item $\displaystyle f(x) = \ln\sqrt{\frac{1 + \cos x}{1 - \cos x}}$.
  \end{enumerate}

\item Resuelva la ecuación $y'(x) = 0$ si
    \begin{enumerate}[leftmargin=*]
    \item $\displaystyle y = x^3 - 4x^2 + 5x - 2$.
    \item $\displaystyle y = \frac{x^2 + x - 6}{x^2 - 10x + 25}$.
    \item $\displaystyle y = \frac{1}{1 + \sen^2 x}$.
    \item $\displaystyle y = x(x + 1)^2(x - 1)^3$.
    \item $\displaystyle y = \frac{e^{|x - 1|}}{x + 1}$.
    \item $\displaystyle y = \max\{|x|^3, 7x - 6x^2\}$.
    \end{enumerate}

\item Si $g$ y $h$ son funciones derivables, calcule $f'(x)$:
    \begin{enumerate}[leftmargin=*]
    \item $\displaystyle f(x) = g\left(h(x)\right) + h\left(g(x)\right)$
    \item $\displaystyle f(x) = \sqrt[n]{[g(x)]^2 + [h(x)^2]}$ si $[g(x)]^2 + [h(x)^2] > 0$
    \item $\displaystyle f(x) = \ln\left|\frac{g(x)}{h(x)}\right|$ si $g(x)h(x)\neq 0$.
    \item $\displaystyle f(x) = g(\sen^2 x) + h(\cos^2 x)$.
    \end{enumerate}

\item Halle los valores de $\alpha$ y $\beta$ en $\Rbb$ de modo que la función $f$ dada a
    continuación sea continua y derivable en $\Rbb$:
    \begin{enumerate}[leftmargin=*]
    \item $\displaystyle f(x) =
    \begin{cases}
      \alpha x + \beta & \text{si $x \leq 1$,} \\
      x^2 & \text{si $x > 1$.}
    \end{cases}$
    \item $\displaystyle f(x) =
    \begin{cases}
        \alpha + \beta x^2 & \text{si |x| < 1,} \\
        \frac{1}{|x|} & \text{si $|x| \geq 1$.}
    \end{cases}$
    \item $\displaystyle f(x) =
    \begin{cases}
      \alpha x^3 + \beta x & \text{si $|x| \leq 2$,} \\
      \frac{1}{\pi}\sen\frac{1}{x} & \text{si $|x| > 2$.}
    \end{cases}$
    \item $\displaystyle f(x) =
    \begin{cases}
        2x - 2 & \text{si $x \leq 1$,} \\
        \alpha(x-1)(x-\beta) & \text{si $x>1$.}
    \end{cases}$
    \end{enumerate}

\item Halle los valores de $\alpha$ y $\beta$ en $\Rbb$ para que $f$ sea derivable en $\Rbb$:
    \begin{enumerate}[leftmargin=*]
    \item $\displaystyle f(x) =
    \begin{cases}
      (x + \alpha)e^{-\beta x} & \text{si $x < 0$,} \\
      \alpha x^2 + \beta x + 1 & \text{si $x \geq 0$.}
    \end{cases}$
    \item $\displaystyle f(x) =
    \begin{cases}
      \alpha x + \beta & \text{si $x < 0$,} \\
      \alpha\cos x + \beta\sen x & \text{si $x\geq 0$.}
    \end{cases}$
    \end{enumerate}

\item ¿Para qué valores de $x\in\Rbb$, $f$ es derivable en $x$?
    \begin{enumerate}[leftmargin=*]
    \item $\displaystyle f(x) = |x^3(x + 1)^2(x + 2)|$.
    \item $\displaystyle f(x) = |\sen x|$.
    \item $\displaystyle f(x) = x|x|$.
    \item $\displaystyle f(x) = |\pi - x|\sen x$.
    \item $\displaystyle f(x) =
    \begin{cases}
      x^2\left|\cos\frac{\pi}{x}\right| & \text{si $x\neq 0$,} \\
      0 & \text{si $x = 0$.}
    \end{cases}$
    \end{enumerate}

\item Halle $f'(x)$ para los $x$ en las que existe:
    \begin{enumerate}[leftmargin=*]
    \item $\displaystyle f(x) =
    \begin{cases}
      x^2 & \text{si $x\in\Qbb$,} \\
      0 & \text{si $x\not\in\Qbb$.}
    \end{cases}$
    \item $\displaystyle f(x) =
    \begin{cases}
      x^2 & \text{si $x\in\Qbb$,} \\
      2|x| - 1 & \text{si $x\not\in\Qbb$.}
    \end{cases}$
    \item $\displaystyle f(x) =
    \begin{cases}
      x & \text{si $x\in\Qbb$,} \\
      0 & \text{si $x\not\in\Qbb$.}
    \end{cases}$
    \item $\displaystyle f(x) = \arccos(\cos x)$.
    \end{enumerate}

\item Dé ejemplos de funciones $f$ y $g$ y $x_0\in\Rbb$ tales que exista $(f\circ g)'(x_0)$
    pero:
    \begin{enumerate}[leftmargin=*]
    \item Existe $f'(g(x_0))$ y no $g'(x_0)$.
    \item No existe $f'(g(x_0))$ y sí $g'(x_0)$.
    \item No existe $f'(g(x_0))$ ni $g'(x_0)$.
    \end{enumerate}

\item Diga si es correcta o no la afirmación siguiente y argumente su respuesta. Sea $I = \
    ]a,b[$
    \begin{enumerate}[leftmargin=*]
    \item Si $f$ y $g$ son derivables en $I$, entonces $f(x) < g(x)$ para todo $x\in I$
        implica que $f'(x) \leq g'(x)$ para todo $x\in I$.
    \item Si $f' < g'$ en $I$, entonces $f < g$ en $I$.
    \item Si $f$ y $g$ son continuas por la derecha en $a$ y si $f(a) = g(a)$ y $f'(x) < g'(x)$ para todo $x\in I$, entonces $f(x) < g(x)$
        para todo $x\in I$.
    \item Si $f$ es derivable en $\Rbb$ y $f$ es par, entonces $f'$ es impar.
    \item Si $f$ es derivable en $\Rbb$ y $f$ es impar, entonces $f'$ es par.
    \item Si $f'$ es par, entonces $f$ es impar.
    \item Si $f'$ es impar, entonces $f$ es par.
    \item Si $f$ es derivable en $I$ y $\displaystyle\limjc{f'(x)}{x}{a^+} = +\infty$,
        entonces $\displaystyle\limjc{f(x)}{x}{a^+} = +\infty$.
    \item Si $f$ es derivable en $I$ y $\displaystyle\limjc{f(x)}{x}{a^+} = +\infty$,
        entonces $\displaystyle\limjc{f'(x)}{x}{a^+} = +\infty$.
    \end{enumerate}
\end{enumerate}
\endgroup
\end{multicols}

\begin{table}[ht]
\[
{\setlength\extrarowheight{5pt}
\begin{array}{|>{\displaystyle}c|>{\displaystyle}c|>{\displaystyle}c|>{\displaystyle}c|} \hline
f(x) & g(x) & f(g(x)) & g(f(x)) \\ \hline
e^x & \cos x & e^{\cos x} & \cos (e^x) \\
x^2 + 2x + 3 & \sqrt[3]{x} & & \\
\sqrt{x} + x + \frac{1}{\sqrt[3]{x}} & 2x + 3 & & \\
2x + \cos x & h(x) + 5 & & \\
& & \sqrt{x^2 + 1} + e^{\sqrt{x^2 + 1}} & \\
& & (\sec 3x + \cos x)^5 & \\
& & & \cos\sqrt{x^2 + 3} \\
& & & \tan\sqrt[3]{(x^4 + 2)^5} \\
& & (x^2 + 1)^{x^2 + 10} & \\
& & \sen(\cos x) & \\
& & \tan\left(\cos\frac{x^2 + 1}{3}\right) & \\
& & & (\sec(x + x^4))^{-5} \\
\hline
\end{array}
}
\]
\caption{Tabla de compuestas para los ejercicios de regla de la cadena}
\label{tab:dcCompuestas}%
\end{table}

\section{Razones de cambio relacionadas}
Supongamos que dos o más magnitudes, digamos $y,z,w$, etcétera, que están relacionadas de alguna
manera entre sí, dependan de una misma variable $x$ (o $t$, o $u$, o $v$, etcétera). Ello hace que
las razones de cambio
\[
\frac{dy}{dx}, \ \frac{dz}{dx},\ \frac{dw}{dx}, \ldots
\]
también estén relacionadas entre sí. Esta última relación puede ser encontrada gracias a la regla
de la cadena, estudiada en la sección anterior. También se puede obtener esa relación con ayuda de
la derivación implícita, que estudiaremos en la siguiente sección.

\begin{exemplo}[Solución]{%
Para inflar un globo esférico inyectamos aire a razón de $500\, \text{litros}/\text{minutos}$. ¿Con qué
razón varía la longitud del radio cuando éste mide $1\metros$?
}%
Si $V$ litros es el volumen del globo cuando su radio mide $r$ decímetros\footnote{Hemos escogido
como unidad de longitud el decímetro para que la unidad de volumen sea el litro, que es un
decímetro cúbico.}, entonces:
\begin{equation}
\label{eq:dd001}
	V=\frac{4}{3}\pi r^{3}.
\end{equation}

Puesto que el aire ingresa al globo a una razón de $500$ litros por minuto, entonces
\[
\frac{dV}{dt}=500.
\]

En este caso, las magnitudes $V$ y $r$, que están relacionadas entre sí por medio de la
igualdad~(\ref{eq:dd001}), dependen, ambas, de una tercera magnitud: el tiempo $t$. Y lo que
queremos calcular, la razón de cambio instantánea del radio cuando su longitud es $10$ decímetros,
es, precisamente, la razón de cambio es $\frac{dr}{dt}$ cuando $r=10$, que está relacionada con la
razón de cambio de $V$ respecto de $t$.

Para ello, derivamos respecto de $t$ ambos lados de la igualdad \ref{eq:dd001}. Obtenemos así la
relación entre las razones de cambio $\frac{dr}{dt}$ y $\frac{dV}{dt}$:
\begin{equation*}
	\frac{dV}{dt}=4\pi r^{2}\frac{dr}{dt}.
\end{equation*}
Entonces:
\begin{equation*}
	\frac{dr}{dt}=\frac{1}{4\pi r^{2}}\frac{dV}{dt}.
\end{equation*}

Por lo tanto, como $r=10$ y  $\frac{dV}{dt}=500$, obtenemos que:
\begin{equation*}
	\frac{dr}{dt}=\frac{500}{4\pi\  10^{2}}=\frac{5}{4\pi}\approx 0.398.
\end{equation*}
Es decir que la longitud del radio $r$ varía a una velocidad de aproximadamente $3.98$ centímetros
por minuto.
\end{exemplo}

\begin{multicols}{2}[\section{Ejercicios}]
\begingroup\small
\begin{enumerate}[leftmargin=*]
\item\label{ex:dcRCCono} El agua escapa del reservorio cónico, mostrado en la figura
    \ref{fig:cono}, a una razón de $20$ litros por minuto: ¿Con qué velocidad disminuye el
    nivel de agua cuando su altura $h$ desde el fondo es de $5$ metros? ¿Cuál es la razón de
    cambio de radio $r$ del espejo del agua en ese instante?

\item\label{ex:dcRCIslote} Un faro ubicado en un islote situado a $3$ kilómetros de la playa
    emite un haz de luz que gira dando una vuelta entera cada minuto (figura \ref{fig:islote}).
    ¿Con qué velocidad se ``mueve'' el punto $P$ de la playa iluminado por el haz de luz $FP$
    emitido por el faro cuando $P$ está situado a $2$ kilómetros del punto $Q$, que es el punto
    de la playa más cercano al faro?

\item Un avión que está a $500$ kilómetros al norte de Quito viene hacia la capital a $400$
    kilómetros por hora, mientras que otro, que está a $600$ kilómetros al este, lo hace a una
    velocidad de $300$ kilómetros por hora. ¿A qué velocidad se acercan el uno al otro?

\item Un rectángulo mide $10$ metros de altura por $20$ metros de base. Si la base aumenta a
    razón constante de un metro por minuto y la altura disminuye a una razón constante de dos
    metros por minuto, ¿con qué velocidad varía el área del rectángulo? El área, ¿aumenta o
    disminuye?

\item Si de un recipiente, que tiene la forma de una pirámide truncada invertida de base
    cuadrada, de $10$ metros de altura, de $10$ metros el lado del cuadrado más grande y de $5$
    metros el lado del cuadrado más pequeño, y que está lleno de agua a media altura, se extrae
    el líquido a una razón constante de un metro cúbico por minuto:
    \begin{enumerate}[leftmargin=*]
    \item ¿Con qué rapidez baja el nivel del agua?
    \item Si, por otro lado, el nivel del agua sube un metro por hora al bombear agua en el
        reservorio cuando está lleno hasta la mitad, ¿cuál es el caudal de agua que se
        introduce?
    \end{enumerate}

\item Un radar que está a $12$ kilómetros de una base militar detecta que un avión sobrevuela
    la base a $9\,000$ metros de altura y que se dirige hacia el radar, manteniendo su altitud
    y velocidad. Si la rapidez con que decrece la distancia entre el avión y el radar es de
    $500$ kilómetros por hora, ¿a qué velocidad vuela el avión?

\item ¿Con qué rapidez varía el área de la corona que queda entre dos circunferencias
    concéntricas de radios $10$ metros y $20$ metros respectivamente, si el diámetro de la más
    pequeña aumenta un metro por hora y el diámetro de la más grande disminuye dos metros por
    hora? ¿Y si los diámetros aumentan un metro por hora? Si el diámetro menor crece dos
    metros, ¿cómo debe variar el otro para que el área de la corona no cambie?

\item Una partícula se mueve en una trayectoria elíptica cuya forma está dada por la ecuación
    \[
      \left(\frac{x}{3}\right)^2 + \left(\frac{y}{2}\right)^2 = 1.
    \]
    Supongamos que las longitudes están expresadas en metros.

    Si se sabe que en un punto de la abscisa $1$, ésta se incrementa a una razón de dos metros
    por segundo, ¿qué sucede con la ordenada del punto con la abscisa dada? Considere los casos
    que se derivan según los cuadrantes en los que se halla la ordenada.

\item Un buque se acerca a un faro de $50$ metros de altura. Se sabe que el ángulo entre la
    horizontal y la recta que une el buque con la punta del faro varía a una razón constante de
    $\alpha^\circ$ por minuto, cuando el buque está a dos kilómetros del faro. ¿Cuál es la
    velocidad del buque? ¿Cuál sería un valor razonable para $\alpha$?
\end{enumerate}
\endgroup
\end{multicols}

\begin{figure}[h]
\begin{center}
\subfloat[Ejercicio \ref{ex:dcRCCono}]{%
    \begin{pspicture}(0,0)(6,6)
    \psset{PointSymbol=none,PointName=none}%
    \footnotesize%

    \pstGeonode[]%
      (3,1){V}(3,5){C}(1.5,5){A}(4.5,5){B}(3,3){D}(0,3){F}%

    \pstLineAB[]%
      {V}{A}%
    \pstLineAB[]%
      {V}{B}%

    \psellipse[]%
      (C)(! \psGetNodeCenter{A} \psGetNodeCenter{C} C.x A.x sub 0.25)

    \pstInterLL%
      {D}{F}{V}{A}{I}%

    \psellipse[]%
      (D)(! \psGetNodeCenter{I} \psGetNodeCenter{D} D.x I.x sub 0.15)%

    \pstLineAB[]%
      {D}{I}%
    \pstMiddleAB[]%
      {D}{I}{M}%
    \uput[90](M){$r$}%

    \psline{|<->|}%
      (! \psGetNodeCenter{A} A.x A.y 0.5 add)(! \psGetNodeCenter{B} B.x B.y 0.5 add)%
    \uput[90](3,5.5){$8\metros$}%

    \psline{|<->|}%
      (! \psGetNodeCenter{V} V.x 1.25 sub V.y)(! \psGetNodeCenter{V} V.x 1.25 sub
      \psGetNodeCenter{I} I.y)%
    \uput[180](1.75,2){$h$}%

    \psline{|<->|}%
      (! \psGetNodeCenter{V} V.x 2 add V.y)(! \psGetNodeCenter{V} V.x 2 add \psGetNodeCenter{B}
      B.y)%
    \uput[0](5,3){$8\metros$}
    \end{pspicture}
    \label{fig:cono}}%
\qquad
\subfloat[Ejercicio \ref{ex:dcRCIslote}]{%
    \begin{pspicture}(0,0)(6,6)
      \psset{PointSymbol=none}%
      \footnotesize%

      \pstGeonode[PosAngle={180,180,0}]%
        (1,5){P}(1,1){Q}(5,1){F}%
      \pstGeonode[PointName=none]%
        (1,0){I}(1,6){S}%

      \pstLineAB[]%
        {I}{S}%
      \psdot[]%
        (F)%

      \pstLineAB[linestyle=dashed,arrows=->]%
        {F}{P}%
      \pstLineAB[arrows=|<->|]%
        {Q}{F}%
      \uput[-90](3,1){$3\kilometros$}%

      \uput[180]{90}(0.75,3){Playa}
    \end{pspicture}
    \label{fig:islote}}%
    \end{center}
\caption{Gráficos de los ejercicios sobre razón de cambio}
\end{figure}
\section{Derivación implícita}

La representación en el plano cartesiano de las parejas ordenadas $(x,y)$ que satisfacen una
ecuación dada, llamada gráfico de esa ecuación, puede consistir en un punto, en una curva,
etcétera.

Cuando ese gráfico es una curva, ésta no es necesariamente el gráfico de una función (sabemos que
para que lo sea, su intersección con cualquier vertical tiene que consistir, a lo más, en un
punto).

Por ejemplo, el gráfico de $x^{2}+y^{2}=25$ es la circunferencia de radio 5 cuyo centro es el
origen del sistema de coordenadas. La recta vertical $x=3$ la corta en dos puntos, $(3,-4)$ y
$(3,4)$, por lo que esta curva no puede ser el gráfico de una función.

Sin embargo, si tomamos solo un pedazo de esa curva, puede suceder que ese trozo sí cumple con el
criterio de los cortes verticales por lo que puede ser el gráfico de una función. En nuestro
ejemplo, si tomamos el trozo de circunferencia que está sobre el eje horizontal, ese pedazo puede
considerarse como el gráfico de la función $f\colon\mathbb{R}  \rightarrow \mathbb{R}$ definida por
\begin{equation*}
	y=f(x)=\sqrt{25-x^{2}}.
\end{equation*}
Lo mismo hubiese sucedido si tomábamos la semicircunferencia que está bajo el eje horizontal, que
puede considerarse como el gráfico de una función $g\colon\mathbb{R}  \rightarrow \mathbb{R}$
definida por:
\begin{equation*}
	y=g(x)=-\sqrt{25-x^{2}}.
\end{equation*}

En este caso, se dice que las funciones $f$ y $g$ están \emph{definidas implícitamente} por la
ecuación:
\begin{equation*}
	x^{2}+y^{2}=25,
\end{equation*}
o simplemente que son \emph{funciones implícitas}.

Podemos ver que
\[
\Dm(f) = \Dm(g) = [-5,5],
\]
y que para todo $x\in [-5,5]$, se verifican las siguientes igualdades:
\begin{equation*}
	x^{2}+[f(x)]^{2}=25
\yjc
	x^{2}+[g(x)]^{2}=25.
\end{equation*}

Por otro lado, se pueden tomar otros pedazos o uniones de pedazos de circunferencia que pueden
considerarse gráficos de funciones definidas implícitamente por la misma ecuación $x^{2}+y^{2}=25$.
En este ejemplo logramos hallar fórmulas para \emph{definir explícitamente} a las funciones $f$ y
$g$.

Si una función $h\colon\mathbb{R}  \rightarrow \mathbb{R}$ puede definirse mediante una fórmula de
la forma $y=h(x)$, se dice que $h$ es una \emph{función explícita}.

En general no siempre es posible explicitar una función implícita. Por ejemplo, para la ecuación
\begin{equation}
\label{eq:MasDer001}
	xy+x^{5}+5x^{2}y^{4}-9y^{5}+2=0,
\end{equation}
vemos que no es posible despejar la $y$ para ponerla en términos de $x$, es decir que no podemos
explicitar las funciones implícitas cuyos gráficos sean subconjuntos del gráfico de esta ecuación.

Sin embargo, gracias a la regla de la cadena, es posible, en estos casos, hallar la derivada de las
funciones implícitas, cuando, en las ecuaciones, cada uno de los miembros es una combinación (que
consiste de sumas, restas, multiplicaciones, divisiones o composiciones finitas), de las funciones
elementales conocidas (polinomios, exponenciales, trigonométricas, o las inversas de las
mencionadas).

En estos casos, se puede derivar ambos miembros respecto a la variable consideradas independiente
($x$ en el ejemplo), y luego se despeja fácilmente la derivada de la variable dependiente ($y$ en
el ejemplo), que aparece al utilizar la regla de la cadena en las expresiones que contienen la
variable dependiente.

Al derivar ambos miembros de la ecuación~(\ref{eq:MasDer001}), tenemos que:
\begin{equation*}
	 y+xy'+5x^{4}+10xy^{4}+20x^{2}y^{3}y'-45y^{4}y'=0,
\end{equation*}
de donde
\begin{equation*}
	 y'=\frac{y+5x^{4}+10xy^{4}}{-x-20x^{2}y^{3}+45y^{4}}.
\end{equation*}

\begin{exemplo}[Solución]{%
Hallar la ecuación de la tangente a la circunferencia $x^{2}+y^{2}=25$ en el punto
$(3,4)$.
}%
La ecuación de la recta será
\begin{equation*}
	y=4+m(x-3),
\end{equation*}
donde $m$ es la pendiente.

Ahora bien, $m$ es la derivada de la función explícita cuyo gráfico es un pedazo de la
circunferencia que contenga al punto $(3,4)$.

Para hallar $y'$, derivemos explícitamente $x^{2}+y^{2}=25$. Obtendremos que se verifica la
igualdad siguiente:
\[
2x+2yy'=0,
\]
de donde se obtiene la derivada de $y$:
\begin{equation*}
	y'=-\frac{x}{y}.
\end{equation*}

Para $(x,y)=(3,4)$ tendremos $m=y'=-\frac{3}{4}$. Entonces, la ecuación de la recta será:
\[
y=4-\frac{3}{4}(x-3),
\]
que es equivalente a:
\begin{equation*}
	3x+4y-25=0.
\end{equation*}
\end{exemplo}

\begin{multicols}{2}[\section{Ejercicios}]
\begingroup
\small
\begin{enumerate}[leftmargin=*]
\item Si una función $\funcjc{f}{x}{y = f(x)}$ es derivable y está definida implícitamente
    por la ecuación dada a continuación, calcule $y' = f'(x)$.
    \begin{enumerate}[leftmargin=*]
    \item $\displaystyle y^5 + y^3 + y - x = 0$.
    \item $\displaystyle y - x = \epsilon\sen y$ con $|\epsilon| < 1$.
    \item $\displaystyle y^2 = 2px$, con $y > 0$.
    \item $\displaystyle \frac{x^2}{a^2} + \frac{y^2}{b^2} = 1$ con $y > 0$.
    \item $\displaystyle \frac{x^2}{a^2} - \frac{y^2}{b^2} = 1$ con $y < 0$.
    \item $\displaystyle (2a - x)y^2 = x^3$, con $a > 0$ y $y < 0$.
    \item $\displaystyle \sqrt{x} + \sqrt{y} = 2$.
    \item $\displaystyle x^{\frac{2}{3}} + y^{\frac{2}{3}} = a^{\frac{2}{3}}$, con $a > 0$,
        $y > 0$.
    \item $\displaystyle 5x^2 + 9y^2 - 30x + 18y + 9 = 0$ con $y < -1$.
    \item $\displaystyle x^2 - 4xy + 4y^2 + 4x - 3y - 7 = 0$ con $x < 2y - 1$.
    \end{enumerate}

\item Si una función $\funcjc{f}{x}{y = f(x)}$ es derivable y está definida implícitamente
    por la ecuación dada a continuación, calcule $f'(a)$.
    \begin{enumerate}[leftmargin=*]
    \item $\displaystyle x^2 + y^2 - 6x + 10y - 2 = 0$, $y > - 5$ y $a = 0$.
    \item $\displaystyle 6xy + 8y^2 - 12x - 26y + 11 = 0$, $y > 0$ y $a = \frac{11}{12}$.
    \item $\displaystyle e^y + xy = e$, $y > 0$ y $a = 0$.
    \item $\displaystyle xy + \ln y = 1$, $y < e^2$ y $a = 0$.
    \end{enumerate}
\end{enumerate}
\endgroup
\end{multicols}

\section{Derivada de la función inversa}
Si $\funcjc{f}{D\subseteq\Rbb}{\Rbb}$ es es inyectiva, existe la función inversa
$\funcjc{f^{-1}}{\Img(f)}{D}$ tal que
\begin{equation*}
	x=f^{-1}(y) \quad \Leftrightarrow \quad y=f(x).
\end{equation*}
Se verifican, entonces, las siguientes igualdades:
\begin{equation*}
	f^{-1}(f(x))=x
\end{equation*}
para todo $x\in D(f)$; y
\begin{equation}
\label{eq:dd002}
f(f^{-1}(y))=y
\end{equation}
para todo $y\in \Dm(f^{-1})$.

Si se conoce la derivada de $f$, podemos hallar la derivada de $f^{-1}$. Esto lo podemos hacer con
ayuda de la regla de la cadena.

En efecto, la igualdad~(\ref{eq:dd002}) puede ser considerada como una ecuación que define
implícitamente $x$ como función de $y$, donde
\[
x = f^{-1}(y).
\]

Podemos, entonces, calcular la derivada de $x$, es decir, la derivada de la inversa de $f$, si
derivamos respecto de $y$ ambos miembros de la igualdad~(\ref{eq:dd002}). Al hacerlo, obtendremos
que:
\[
f'(f^{-1}(y))(f^{-1})'(y) = 1,
\]
de donde:
\begin{equation}
\label{et:ddDerivadaInversa}
	(f^{-1})'(y)= \frac{1}{f'(f^{-1}(y))},
\end{equation}
siempre que $y\in \Dm(f^{-1})$ y que el denominador sea diferente de cero.

\begin{exemplo}[Solución]{%
Hallar la derivada de $\ln$, la función inversa de $f$ definida por $f(x) = e^x$.
}%
En primer lugar:
\[
f'(x) = e^x = f(x).
\]
Por lo tanto:
\[
f'(f^{-1}(y)) = f(f^{-1}(y)) = y.
\]

Entonces, al utilizar la fórmula~(\ref{et:ddDerivadaInversa}), obtenemos que:
\begin{align*}
(f^{-1})'(y) &= \frac{1}{f'(f^{-1}(y))} \\[4pt]
   &= \frac{1}{f(f^{-1}(y))} \\[4pt]
   &= \frac{1}{y}.
\end{align*}

Como $\ln y = f^{-1}(y)$, entonces:
\[
(\ln y)' = \frac{1}{y}.
\]
\end{exemplo}

\begin{exemplo}[Solución]{%
Hallar la derivada de la función $\arcsen$.
}%
En primer lugar, la función $\arcsen$ es la función inversa de la función $\sen$ restringida al
intervalo $[-\frac{\pi}{2},\frac{\pi}{2}]$. Sea, entonces, $f$ definida por:
\[
f(x) = \sen x
\]
para todo $x\in [-\frac{\pi}{2},\frac{\pi}{2}]$.

Entonces $f'(x) = \cos x$ y:
\begin{equation}
\label{eq:dd003}
f'(f^{-1}(y)) = \cos(\arcsen y)
\end{equation}
para todo $y \in [-1,1] = \Dm(\arcsen)$.

Ahora bien, sabemos que
\[
\sen^2 x + \cos^2 x = 1,
\]
de donde
\[
\cos^2 x = 1 - \sen^2 x,
\]
y, dado que $\cos x > 0$ para todo $x\in [-\frac{\pi}{2},\frac{\pi}{2}]$, tenemos que:
\[
\cos x = \sqrt{1 - \sen^2 x}.
\]

Con ayuda de esta última expresión, podemos reescribir la igualdad~(\ref{eq:dd003}) de la siguiente
manera:
\begin{align*}
f'(f^{-1}(y)) &= \cos(\arcsen y) \\
   &= \sqrt{1 - \sen^2(\arcsen y)} \\
   &= \sqrt{1 - \left(\sen(\arcsen y)\right)^2} \\
   &= \sqrt{1 - y^2}.
\end{align*}

Entonces, la fórmula~(\ref{et:ddDerivadaInversa}) nos da:
\begin{align*}
(f^{-1})'(y) &= \frac{1}{f'(f^{-1}(y))} \\[4pt]
   &= \frac{1}{\sqrt{1 - y^2}}.
\end{align*}
Por lo tanto:
\[
(\arcsen y)' = \frac{1}{\sqrt{1 - y^2}}
\]
para todo $y\in [-1,1]$.
\end{exemplo}

\begin{exemplo}[Solución]{%
Las derivadas de $\arccos$, $\arctan$ y $\sqrt{ \ }$.
}%
El procedimiento es similar al del ejemplo anterior. Pero hay que considerar lo siguiente.

En el caso del $\arccos$, su dominio es $[-1,1]$ y su imagen $[0,\pi]$. Esta función es la inversa
de la función $\cos$, restringida al intervalo $[0,\pi]$, cuya derivada es $-\sen$.

Como la función $\sen$ es positiva en el intervalo $[0,\pi]$, entonces:
\[
\sen x = \sqrt{1 - \cos^2 x}.
\]

Al aplicar la fórmula~(\ref{et:ddDerivadaInversa}), obtendremos que:
\[
(\arccos y)' = -\frac{1}{\sqrt{1 - y^2}}
\]
para todo $y\in [-1,1]$.

En el caso de $\arctan$, su dominio es $\mathbb{R}$ y su imagen $[-\frac{\pi}{2},\frac{\pi}{2}]$, y
es la función inversa de $\tan$, restringida a este último intervalo.

La derivada de $\tan$ es $\sec^2$. Con ayuda de la identidad
\[
\sec^2 x = 1 + \tan^2 x,
\]
verdadera para todo $x\in\mathbb{R}$, al aplicar la fórmula~(\ref{et:ddDerivadaInversa}),
obtendremos que
\[
(\arctan y)' = \frac{1}{1 + y^2}
\]
para todo $y\in\mathbb{R}$.

Finalmente, para obtener la derivada de $\sqrt{ \ }$, solo hay que recordar que es la función
inversa de la función $x^2$ restringida al intervalo $[0,+\infty]$. Al aplicar la
fórmula~(\ref{et:ddDerivadaInversa}), obtendremos que:
\[
(\sqrt{y})' = \frac{1}{2\sqrt{y}}
\]
para todo $y > 0$.
\end{exemplo}

\begin{multicols}{2}[\section{Ejercicios}]
\begingroup\small
\begin{enumerate}[leftmargin=*]
\item Calcule la derivada $(f^{-1})'(y_0)$:
    \begin{enumerate}[leftmargin=*]
    \item $\displaystyle f(x) = x + \frac{1}{3}x^3$, $y_0
        \in\{0,\pm\frac{4}{3},\pm\frac{14}{3}\}$.
    \item $\displaystyle f(x) = 2x - \frac{1}{2}\cos x$, $y_0 = -\frac{1}{2}$.
    \item $\displaystyle f(x) = 0.1x + e^{0.01x}$, $y_0 = 1$.
    \item $\displaystyle f(x) = 2x^2 - x^4$, $x > 1$ y $y_0 = 0$.
    \item $\displaystyle f(x) = 2x^2 - x^4$, $0 < x < 1$ y $y_0 = \frac{3}{4}$.
    \item $\displaystyle f(x) = \ln x$, $x > 0$, $y_0=1$.
    \item $\displaystyle f(x) = \exp x$, $y_0=1$.
    \end{enumerate}

\item Si conoce que $(\ln x)' = \frac{1}{x}$ y que $\exp = \ln^{-1}$, calcule $\exp'$.

\item Si conoce que $\exp' = \exp$ y que $\ln = \exp^{-1}$, calcule $\ln'$.

\item Calcule fórmulas para las derivadas de las funciones inversas de $\sinh$, $\cosh$ y
    $\tanh$.
\item Calcule $(f^{-1})'$ y $\Dm((f^{-1})')$:
    \begin{enumerate}[leftmargin=*]
    \item $\displaystyle f(x) = \frac{x^2}{1 + x^2}$, $x < 0$.
    \item $\displaystyle f(x) = \coth x$, $x > 0$.
    \end{enumerate}

\item Si $f(x) = x + \sen x$ para todo $x\in\Rbb$, calcule los valores de $a$ y de $b$ para los
    cuales se verifican las igualdades $(f^{-1})'(a) = +\infty$ y $(f^{-1})'(b) = -\infty$.
\end{enumerate}
\endgroup
\end{multicols}

\section{Derivadas de orden superior}

Recordemos que, dada una función $\funcjc{f}{\Dm(f)}{\mathbb{R}}$, se define la función
$\funcjc{f'}{\Dm(f')}{\mathbb{R}}$, la función derivada de $f$, cuyo dominio es:
\begin{equation*}
	\Dm(f')=\{x\in \Dm(f) : \ \text{existe} \ f'(x)\}.
\end{equation*}

Al ser $f'$ una función, es posible que también sea derivable en algunos elementos de su dominio.
Esta situación conduce a la siguiente definición.

\begin{defical}[Derivadas de segundo orden]
Sea $a\in D(f')$ para el cual existe la derivada de $f'$; es decir, existe $(f')'(a)$. Este número
es denominado \emph{segunda derivada de $f$ en $a$} y es representado por:
\[
f^{\prime\prime}(a) \quad\text{o}\quad \frac{d^{2}}{d x^{2}}f(a).
\]
Es decir:
\begin{equation*}
	f''(a)=\frac{d^{2}}{d x^{2}}f(a)=\frac{d}{d x}\left(\frac{d}{d x}f\right)(a)=(f')'(a).
\end{equation*}
\end{defical}

\begin{exemplo}[Solución]{%
Calcular la segunda derivada de $\sen x$, $\cos x$ y $e^x$.
}%
Sea $f$ definida por $f(x) = \sen x$. Entonces $f$ es derivable en $\mathbb{R}$ y:
\[
f'(x) = \cos x.
\]
Por lo tanto, $f'$ también es derivable en $\mathbb{R}$ y su derivada es igual a la derivada de la
función $\cos$:
\[
(f')'(x) = (\cos x)' = -\sen x.
\]
Por lo tanto:
\[
(\sen x)'' = -\sen x.
\]

De manera similar se establece que $\cos x$ y $e^x$ tienen segunda derivada y:
\[
(\cos x)'' = (-\sen x)' = -\cos x;
\]
y
\[
(e^x)'' = (e^x)' = e^x.
\]
\end{exemplo}

Ahora bien, podría suceder que la función $f''$ también fuera derivable en algunos elementos de su
dominio. Entonces, existiría la función $f'''$, que se definiría por:
\[
f''' = (f'')'.
\]
Esta función es denominada la \emph{derivada de tercer orden}.

En general, podemos hablar de la derivada de orden $n$, que se define inductivamente así:

\begin{defical}[Derivada de orden $n$]
Sea $\funjc{f}{\Dm(f)}{\mathbb{R}}$. Para cada $n\in\mathbb{N}$, se define:
\begin{enumerate}
\item $f^{(0)} = f$.
\item $f^{(n+1)} = (f^{(n)})'$, para $n \geq 0$.
\end{enumerate}
Si existe $f^{(n)}(a)$, este número es denominado la $n$-ésima derivada de $f$ en $a$ y suele ser
representado de la siguiente manera:
\[
f^{(n)}(a) = \frac{d^n}{dx^n}f(a).
\]
\end{defical}

De esta definición, es inmediato que:
\begin{equation*}
	f^{(n)}(a)=\frac{d^{n}}{d x^{n}}f(a)=
   \frac{d}{d x}\left(\frac{d^{n-1}}{d x^{n-1}}f\right)(a) =
   \left(f^{(n-1)}\right)'(a).
\end{equation*}

Tiene sentido definir como la derivada de orden $0$ a la misma función, pues se puede interpretar
esto diciendo que no derivar la función es dejarla intacta.

\begin{exemplo}[Solución]{%
Establecer fórmulas generales para las derivadas de orden $n$ para $\sen x$, $\cos x$ y
$e^x$.
}%
En el ejemplo anterior, calculamos la segunda derivada de estas tres funciones. Calculemos ahora
las derivadas de tercero y cuarto orden.

Para el caso de la función $\sen x$:
\begin{align*}
(\sen x)^{(3)} &= ((\sen x)^{(2)})' = (-\sen x)' = -\cos x. \\
(\sen x)^{(4)} &= ((\sen x)^{(3)})' = (-\cos x)' = \sen x.
\end{align*}

Como podemos ver, la derivada de orden $4$ de $\sen x$ es igual a la función $\sen x$. Esto
significa que la derivada de orden $5$ es igual a la primera derivada; la derivada de orden $6$,
igual a la segunda derivada; la derivada de orden $7$, igual a la tercera; y la de orden $8$, a la
de orden $4$. Es decir, las derivadas se repetirán cada cuatro órdenes. Podemos resumir esto de la
siguiente manera:
\[
\begin{array}{ll}
(\sen x)^{(4n)} &= \sen x. \\[4pt]
(\sen x)^{(4n+1)} &= \cos x. \\[4pt]
(\sen x)^{(4n + 2)} &= -\sen x. \\[4pt]
(\sen x)^{(4n + 3)} &= -\cos x
\end{array}
\]
para todo $n \geq 0$.

De manera similar, se establece para $\cos x$ lo siguiente:
\[
\begin{array}{ll}
(\cos x)^{(4n)} &= -\sen x. \\[4pt]
(\cos x)^{(4n+1)} &= -\cos x. \\[4pt]
(\cos x)^{(4n + 2)} &= \sen x. \\[4pt]
(\cos x)^{(4n + 3)} &= \cos x
\end{array}
\]
para todo $n \geq 0$.

Finalmente, la derivada de $e^x$ siempre es igual a $e^x$, por lo que se verifica la siguiente
fórmula:
\[
(e^x)^{(n)} = e^x
\]
para todo $n\geq 0$.
\end{exemplo}

\section{Ejercicios}
\begingroup\small
Calcule $f^{(n)}$ con $n\in\Nbb$:
\begin{multicols}{2}
\begin{enumerate}[leftmargin=*]
\item $f(x) = \displaystyle\frac{x+1}{x-1}$ si $n = 2$.
\item $f(x) = x^5-7x^2+1$ si $n = 2$.
\item $f(x) = ax^2+bx+c$.
\item $f(x) = ax^3+bx^2+cx+d$.
\item $f(x) = \sen(ax)$.
\item $f(x) = \cos(ax)$.
\item $\displaystyle f(x) = \ln x$.
\item $\displaystyle f(x) = \sum_{k=0}^{m}a_kx^k$ con $m\in \mathbb{N} - \{0\}$.
\item $\displaystyle f(x) = \frac{1}{x}$.
\item $\displaystyle f(x) = \frac{1}{(x + a)}$.
\item $\displaystyle f(x) = \frac{x + a}{x - a}$.
\end{enumerate}
\end{multicols}
\endgroup

\section{Diferenciales}
La derivada de una función $f$ en un punto dado $x_0$ es el valor de la pendiente de la recta
tangente al gráfico de $f$ en el punto de coordenadas $(x_{0},f(x_{0}))$, cuya ecuación es:
\begin{equation*}
	 y=f(x_{0})+f'(x_{0})\, (x-x_{0}).
\end{equation*}

Esta recta suele ser utilizada para aproximar el gráfico de $f$ para  valores cercanos a $x_{0}$,
pues, como se puede apreciar el gráfico que está a continuación, los valores de la recta y de la
función son próximos:
\begin{center}
\def\F{x dup mul}%
\psset{unit=8cm}
\begin{pspicture}(-0.1,-0.1)(0.8,0.65)
   \SpecialCoor

   \psaxes{->}(0,0)(-0.1,-0.1)(0.75,0.6)%
   \uput[-90](0.75,0){$x$}%
   \uput[0](0,0.6){$y$}%

   \psplot[linewidth=3\pslinewidth]{0}{0.7}{\F}%
   \psplotTangent[linecolor=gray,linewidth=1.75\pslinewidth]%
      {0.4}{0.3}{\F}%
   \psline[linestyle=dashed]%
      (! 0 /x 0.4 def \F)(! 0.4 /x 0.4 def \F)(0.4,0)%

   \uput[180](! 0 /x 0.4 def \F){$f(x_0)$}%
   \uput[-90](0.4,0){$x_0$}%
   \uput[0](! 0.7 /x 0.7 def \F){$y = f(x)$}%

\end{pspicture}
\end{center}

Ahora bien, si $x$ varía desde $x_{0}$ hasta $x_0 + \Delta x$, el valor correspondiente de $y$
cambia de $y_{0}=f(x_{0})$ a $y = f(x_{0}+\Delta x)$. Este cambio o variación, representado por
$\Delta y$ se calcula así:
\begin{equation*}
	 \Delta y = y - y_{0}=f(x_{0}+\Delta x)-f(x_{0}).
\end{equation*}

Entonces, vamos a utilizar el valor que toma $(x_{0}+\Delta x)$ en la recta tangente, en lugar del
que toma en $f$; es decir, en lugar de utilizar el valor $y = f(x_{0}+\Delta x)$, utilizamos el
dado por la ecuación de la recta tangente:
\begin{equation*}
	 y = f(x_{0}) +f'(x_{0})((x_{0}+\Delta x)-x_{0})= f(x_{0})+f'(x_{0})\Delta x.
\end{equation*}

Esto implica que al cambio $\Delta y$ lo estamos aproximando con el cambio que se produce en la
recta al variar $x$, de $x_{0}$ a $x_{0}+\Delta x$, que se nota $d y$ y que se calcula así:
\begin{equation*}
	d y=[f(a)+f'(a)\, \Delta x]-f(a)=f'(a)\, \Delta x.
\end{equation*}
Se suele escribir $\Delta x = d x$; a $d x$ se le llama diferencial de $x$. A $d y$ se le llama
\emph{diferencial de $f$ en $a$}. Se tiene entonces
\begin{equation*}
	d y=f'(a)\, d x.
\end{equation*}
Esta igualdad puede interpretarse como la ecuación de la recta tangente en un sistema de
coordenadas $(d x, d y)$ cuyo origen está en $(x_{0},f(x_{0}))$ y que es paralelo al sistema de
coordenadas $(x,y)$.

\begin{exemplo}[Solución]{%
Si $f$ está definida por $f(x) = 3x^2 + 2$, calcular el diferencial de $f$ en $1$.
}%
Sea $y = f(x) = 3x^{2} + 2$. Como $f'(x) = 6x$, para $x_{0}=1$, se tiene que:
\begin{equation*}
	dy=f'(1)\,dx=6dx.
\end{equation*}
\end{exemplo}

\begin{exemplo}[Solución]{%
Si $g$ está definida por $g(t) = t + e^t$, calcular el diferencial de $g$ en $2$.
}%
Sea $z=g(t)=t+e^{t}$. Como $g'(t) = 1+e^{t}$, para $t_{0}=2$ se tiene que:
\begin{equation*}
	dz=g'(2)\,dt=(1+e^{2})dt.
\end{equation*}
\end{exemplo}

Entre las aplicaciones de los diferenciales está la posibilidad de realizar ciertos cálculos
aproximados.

\begin{exemplo}[Solución]{%
Calcular un valor aproximado para $\sqrt{50}$.
}%
Ponemos $f(x)=\sqrt{x}, x_0=49,\Delta x= dx=1$ y aproximamos $\Delta y$ con $dy$:
\begin{equation*}
	 dy=f'(x_0)\,dx=\frac{1}{2\sqrt{49}}\cdot 1=\frac{1}{14}.
\end{equation*}
Entonces
\begin{equation*}
	\sqrt{50}=f(50)=f(49)+\Delta y\approx f(49)+dy=\sqrt{49}+\frac{1}{14}=7+\frac{1}{14}=\frac{99}{14}.
\end{equation*}
Si se tiene en cuenta que $\sqrt{50}\approx 7.071068 \ldots$ y que $\frac{99}{14}\approx 7.071429
\dots$, vemos que la aproximación es aceptable.
\end{exemplo}

\section{Ejercicios}
\begingroup
\small
\begin{multicols}{2}
\begin{enumerate}[leftmargin=*]
\item Halle $\mathrm{d}y$ en función de $x$ y de $\mathrm{d}x$:
  \begin{enumerate}
  \item $y = x^2 - 3x$.
  \item $y = \displaystyle\frac{x + 1}{x - 1}$.
  \item $\displaystyle y = x\sen x + \cos x$.
  \item $\displaystyle y = x^{\frac{2}{3}}$.
  \end{enumerate}

\item Para los valores de $\Delta x$ y de $x$ dados, calcule $\Delta y$ y $\mathrm{d}y$:
  \begin{enumerate}
  \item $\displaystyle y = x^2 + x + 1, \ \Delta x \in \{0.1, 0.5, 1\}, \ x = 1$.
  \item $\displaystyle y = \frac{2x}{x - 1}, \ \Delta x \in \{0.1, 0.5\}, \ x = 1$.
  \item $\displaystyle y = \frac{1}{x}, \ \Delta x \in \{0.1, 0.5\}, \ x = 1$.
  \end{enumerate}

\item Calcule, mediante diferenciales, un valor aproximado de:
  \begin{enumerate}
  \item $\sqrt{26}$.
  \item $\displaystyle\frac{1}{\sqrt{50}}$.
  \item $\displaystyle\frac{0.9^3}{1.9}$.
  \item $\sen(\pi/3 - 0.1)$.
  \item $\sqrt{82}$.
  \item $\displaystyle\frac{1}{\sqrt{37}}$.
  \item $\sen 44^\circ$ (¡Atención: utilice radianes!).
  \item $(2.1)^3 + 3(2.1)^2$.
  \end{enumerate}
\end{enumerate}
\end{multicols}
\endgroup

\section{Cálculo de los ceros de funciones derivables}
Si conocemos que una función $\funcjc{f}{I}{\mathbb{R}}$, continua en un intervalo abierto $I$, posee un cero en dicho intervalo (por ejemplo, si para $a$, $b$ números reales tales que $a < b$, hemos constatado que $f(a)f(b) < 0$, con lo cual se garantizará, por el teorema del valor intermedio de las funciones continuas, la existencia de un cero $\overline{x}$ entre $a$ y $b$), es útil conocer métodos para el cálculo aproximado de los ceros.

\begin{enumerate}
\item [a)]\textit{Método de dicotomía:} Si $f(a)f(b) < 0$, ponemos $c_1 = (a + b)/2$ y $c_1$ es ya una aproximación del cero buscado, y el error de aproximación satisfará la desigualdad
$|\overline{x} - c_1| < (b - a)/2$. Para $f(c_1)$, tenemos tres posibilidades:
\begin{enumerate}
\item [1.]$f(c_1) = 0$, con lo cual $\overline{x} = c_1$.
\item [2.]$f(c_1)f(a) < 0$. Se aplica la misma idea y se pone $c_2 = (c_1 + a)/2$, el punto intermedio entre $a$ y $c_1$, y es una nueva aproximación de $\overline{x}$, y esta vez el error de aproximación satisface la desigualdad $|\overline{x} - c_2| < (b - a)/2^2$.
\item [3.] $f(c_1)f(b) < 0$. Se aplica la misma idea, pero se pone $c_2 = (c_1 + b)/2$.
\end{enumerate}
Y, así sucesivamente, se obtienen aproximaciones $c_1$, $c_2$, $c_3$, etcétera, de $\overline{x}$; en el paso $n$, para la aproximación $c_n$ del cero $\overline{x}$, el error de aproximación satisface la desigualdad $|\overline x - c_n| < (b - a)/2^n$.

Se puede, entonces, calcular el número de pasos necesarios para lograr una aproximación con un error menor a un valor $\epsilon > 0$ pequeño y predeterminado. Si, por ejemplo, se desea obtener una aproximación con $k$ cifras decimales exactas, bastará resolver la desigualdad
\begin{equation}
\label{eq:dc001}
\frac{b - a}{2^n} < \frac{10^{-k}}{2}
\end{equation}
para obtener el número $n$ de pasos necesarios.

\begin{exemplo}[Solución]{Calcule $\sqrt{2}$ con dos cifras decimales exactas mediante el método de dicotomía.}
El número $\sqrt{2}$ es un cero de la función $\funcjc{f}{[1,2]}{\mathbb{R}}$ definida por $f(x) = x^2 - 2$. Si en el $n$-ésimo paso trabajamos con el intervalo $[a_n,b_n]$, poniendo $a_1 = 1$, $b_1 = 2$, $c_1 = (a_1 + b_1)/2$, $c_n = (a_n + b_n)/2$, y teniendo en cuenta que la inecuación~(\ref{eq:dc001}) en este caso es
\[
\frac{2 - 1}{2^n} < \frac{0.01}{2},
\]
se tiene que $n = 7$ es suficiente. Los cálculos se resumen en el cuadro siguiente:
\[
\begin{array}{|c|l|l|l|l|l|l|}\hline
n &
\multicolumn{1}{c|}{a_n} & \multicolumn{1}{c|}{b_n} & \multicolumn{1}{c|}{c_n} &
\multicolumn{1}{c|}{f(a_n)} & \multicolumn{1}{c|}{f(b_n)} & \multicolumn{1}{c|}{f(c_n)} \\ \hline
1	& 1	& 2	& 1.5	& -1	& 2	& 0.25 \\
2	& 1	& 1.5	& 1.25	& -1	& 0.25	 & -0.437\,5 \\
3	& 1.25	& 1.5	& 1.375	& -0.437\,5	& 0.25	& -0.109\,375 \\
4	& 1.375	& 1.5	& 1.437\,5	& -0.109\,375	& 0.25	& 0.066\,406\,25 \\
5	& 1.375	& 1.437\,5	& 1.406\,25	& -0.109\,375	& 0.066\,406\,25	& -0.022\,460\,938 \\
6	& 1.406\,25	& 1.4375	& 1.421\,875	& -0.022\,460\,938	& 0.066\,406\,25	& 0.021\,728\,516 \\
7	& 1.406\,25	& 1.421875	& 1.414\,0625	& -0.022\,460\,938	& 0.021\,728\,516	& -0.000\,427\,246 \\ \hline
\end{array}
\]
Se ve, entonces, que la aproximación buscada con dos decimales exactos es $c_7 = 1.41$.
\end{exemplo}

\item [b)] \textit{Método de Newton:} Cuando la función $f$ es, además, derivable, es más efectivo y rápido el método de \textit{Newton}, siempre y cuando la derivada tome valores distintos de $0$ en $I$. La idea de este método es partir de una aproximación cualquiera $x_0$ del cero $\overline x$ y obtener una nueva aproximación $x_1$, reemplazando el gráfico de $f$ con la tangente a él en el punto de coordenadas $(x_0,f(x_0))$, como se puede observar en el siguiente dibujo:
\begin{center}
\def\F{x dup mul 1.75 mul 0.125 sub}%
\psset{unit=8cm}
\begin{pspicture}(-0.1,-0.1)(0.8,0.65)
   \SpecialCoor

   \psaxes{->}(0,0)(-0.1,-0.15)(0.75,0.6)%
   \uput[-90](0.75,0){$x$}%
   \uput[0](0,0.6){$y$}%

   \psplot[linewidth=3\pslinewidth]{0}{0.6}{\F}%
   \psplot[linecolor=red,linewidth=1.25\pslinewidth]{0.3}{0.6}{1.925 x mul 0.654375 sub}
   \psline[linestyle=dashed,linecolor=blue]%
      (! 0.55 /x 0.55 def \F)(0.55,0)%

   \uput[180](! 0.55 /x 0.55 def \F){$P_0$}%
   \uput[-90](0.55,0){$Q_0(x_0)$}%
   \uput[0](! 0.65 /x 0.65 def \F){$y = f(x)$}%
   \uput[90](0.267261,0){$\overline x$}%
   \uput[-90](0.339935,0){$Q_1(x_1)$}%
   \uput[180](! 0.339935 /x 0.339935 def \F){$P_1$}%
   \psline[linestyle=dashed,linecolor=blue]%
      (! 0.339935 /x 0.339935 def \F)(0.339935,0)%

\end{pspicture}
\end{center}

En el triángulo rectángulo $Q_1Q_0P_0$, tenemos que
\[
\tan\angle P_0Q_1Q_0 = \frac{Q_0P_0}{Q_1Q_0} = \frac{f(x_0)}{x_0 - x_1}.
\]
Como $\tan\angle P_0Q_1Q_0 = f'(x_0)$, tendremos que
\[
x_1 = x_0 - \frac{f(x_0)}{f'(x_0)}.
\]

Si volvemos a aplicar la misma, pero ahora pariendo de $x_1$ en lugar de $x_2$, obtendremos una nueva aproximación $x_2$ de $\overline x$ mediante la fórmula
\[
x_2 = x_1 - \frac{f(x_1)}{f'(x_1)},
\]
y así sucesivamente. Si notamos $y_n = f(x_n)$ y $y'_n = f'(x_n)$ para $n\in\mathbb{N}$, tendremos que
\[
x_n = x_{n-1} - \frac{y_{n-1}}{y'_{n-1}},
\]
que es una fórmula iterativa de fácil aplicación.

Como criterio para suspender el cálculo, se puede exigir que para un valor $\epsilon > 0$, pequeño y predeterminado, se tenga $|y_n| < \epsilon$.

Apliquemos el método de Newton para volver a calcular una aproximación de $\sqrt{2}$, con $x_0 = 1$, $f(x) = x^2 - 2$ y $f'(x) = 2x$. Los cálculos se resumen en el cuadro siguiente:
\[
\begin{array}{|c|l|l|l|}\hline
n & \multicolumn{1}{c|}{x_n} & \multicolumn{1}{c|}{f(x_n)} & \multicolumn{1}{c|}{f'(x_n)} \\ \hline
0	& 1	& -1	& 2 \\
1	& 1.5	& 0.25	& 3 \\
2	& 1.416\,666\,667	& 0.006\,944\,444	& 2.833\,333\,333 \\
3	& 1.414\,215\,686	& 0.000\,006\,007	& 2.828\,431\,373 \\
4	& 1.414\,213\,562	& 4.5\times 10^{-12}	& 2.828\,427\,125 \\
5	& 1.414\,213\,562	& 0	& 2.828\,427\,125 \\ \hline
\end{array}
\]
La aproximación para el cero es $1.414\,213\,562$.

Vemos que en $4$ pasos se tiene ya una aproximación con $8$ cifras decimales que no se va a modificar con más iteraciones.
\end{enumerate}
\section{Ejercicios}
\begingroup
\small
\begin{multicols}{2}
\begin{enumerate}
\item ¿En cuántos pasos el método de dicotomía le permitirá calcular el número dado con $3$ cifras decimales exactas?
    \begin{enumerate}
    \item $\sqrt{3}$.
    \item $1 + \sqrt{2}$.
    \item $\sqrt{2} - 1$.
    \item $\sqrt{2} + \sqrt{3}$.
    \item La raíz de $x^3 + 5 = 0$.
    \item La raíz de $\sinh x + 2 = 0$.
    \end{enumerate}

\item Compare los métodos de dicotomía y de Newton para los cálculos de aproximaciones de los números del ejercicio precedente.
\end{enumerate}
\end{multicols}
\endgroup
%\subsection{Ejercicios}
%
%\section{Ejercicios}
%\begingroup\small
%Suponga conocidos las definiciones de las funciones que se dan en la siguiente tabla de derivadas.
%Con ayuda de esta tabla y de las propiedades algebraicas de las derivadas, calcule la derivada de
%$f(x)$.
%
%Con los conceptos desarrollados hasta este momento, no nos es posible definir satisfactoriamente
%las funciones exponencial y logaritmo. Sin embargo, para que aquellos lectores que conocen estas
%funciones, las hemos incluido en la tabla. Debe recordarse, entonces, que:
%\[
%a^x = e^{x\ln a} \yjc \log_a x = \frac{\ln x}{\ln a}.
%\]
%\[
%\begin{array}{|l|l|l||l|l|c|}\hline
%\multicolumn{1}{|c|}{f(x)} &
%\multicolumn{1}{c|}{f'(x)} &
%\multicolumn{1}{c||}{\Dm(f')} &
%\multicolumn{1}{c|}{f(x)} &
%\multicolumn{1}{c|}{f'(x)} &
%\multicolumn{1}{c|}{\Dm(f')} \\ \hline
%c \in \Rbb & 0 & \Rbb & x^\alpha, \ \alpha \geq 1 & \alpha x^{\alpha - 1} & \Rbb \\
%x^\alpha, \ \alpha < 1 & \alpha x^{\alpha - 1} & \Rbb - \{0\}
%\end{array}
%\]
%\begin{multicols}{2}
%\begin{enumerate}[leftmargin=*]
%\item
%
%\end{enumerate}
%\end{multicols}
%\endgroup

\chapter{La derivada: aplicaciones}

\section{Romeo y Julieta: la modelización matemática}
\begingroup
\itshape%
Romeo estaba feliz. Acababa de recibir noticias de su amada Julieta, quien logró enviarle
clandestinamente un corto recado, escrito apuradamente en un trozo de papel:

---Te esperaré, luego de la cena, en el muelle de mi residencia. Estaré oculta tras el farol
que queda junto a los sauces.

El dichoso Romeo, quien descansaba ese momento en sus alojamientos, decidió ir lo antes posible en
su bote a la cita. Recordó que, a un costado del lago, crecía un hermoso jardín lleno de bellas
rosas en esta época del año, que él sabía que eran de sumo agrado para Julieta. ¡Cómo no llevarle
un ramo de esas rosas! Estaba, sin embargo, apurado, porque estaba a punto de atardecer. Debía
escoger el camino más corto posible para llegar cuanto antes donde su amada llevándole el más bello
ramo de rosas.
\endgroup

\vspace{\baselineskip}%
Desde el inicio del pensamiento inteligente, éste consistió, fundamentalmente, en la búsqueda de
modelos abstractos, teóricos, de la realidad. Pero, simultáneamente, desde el uso de piedrecillas
(cálculos) y luego del más sofisticado ábaco, para hacer cuentas, seguido del de la ingeniosa regla
de cálculo, y de las maravillosas máquinas electrónicas de que disponemos hoy en día, objetos
físicos y máquinas inventadas y construidas por el hombre se constituyen en modelos físicos de
objetos abstractos, como son los números y las operaciones que se realizan con ellos, las figuras y
objetos geométricos y su representación gráfica.

Las Ciencias Naturales y, cada vez más, las Ciencias Sociales utilizan modelos matemáticos como
instrumento fundamental para describir y entender lo más relevante de los objetos o fenómenos
estudiados. La tecnología de la producción de bienes y servicios utiliza cada vez más problemas
matemáticos como modelos de problemas reales que se presentan en el quehacer creativo. El
advenimiento de las computadoras, cada día más poderosas, ha favorecido este fenómeno, porque
problemas matemáticos cuya solución ``teórica'' es demasiado compleja, y a veces prácticamente
imposible, es ahora abordable, con excelentes resultados, mediante métodos numéricos que proveen de
soluciones aproximadas de calidad.

Es, pues, de vital importancia fortalecer la habilidad del matemático para tender puentes entre la
realidad y la matemática, esto es la de utilizar adecuadamente los modelos matemáticos. La
modelización matemática consiste, generalmente, en un proceso que puede ser descompuesto en las
siguientes etapas, que las ilustraremos con el ejemplo de Romeo y Julieta.

\begin{enumerate}[leftmargin=*]
\item \emph{Identificación clara del problema, objeto o fenómeno que se quiere modelizar}. Como
    resultado de esta fase, se obtiene el ``enunciado'' del problema o la descripción precisa
    del objeto o fenómeno estudiado.

\item \emph{Elaboración del modelo}. Consiste en la representación con símbolos y entes
    matemáticos (números constantes o variables, figuras geométricas, funciones, ecuaciones
    algebraicas o diferenciales, matrices y otros), los aspectos más relevantes del problema,
    objeto o fenómeno estudiado. Un aspecto fundamental en esta etapa es la del uso de un
    sistema coherente de unidades de medida, que permita escoger un modelo matemático que
    abstraiga este tipo de información, es decir que no contenga ya unidades de medida sino
    solo números, funciones numéricas, matrices numéricas, etcétera. Como resultado de esta
    fase, se obtiene el modelo matemático a utilizarse.

\item \emph{Solución del problema matemático}. Estudio del modelo matemático y solución del
    problema matemático, de ser el caso. Al final de esta etapa, se obtiene una solución
    matemática del problema.

\item \emph{Interpretación del modelo matemático}. Se interpretan los resultados del problema
    matemático para dar respuestas al problema original o se da la adecuada interpretación al
    estudio del modelo matemático escogido.

\end{enumerate}

En este capítulo, vamos a conocer los métodos matemáticos que permiten resolver problemas como el
de Romeo, entre otros. A manera de ejemplo, ilustremos las dos primeras etapas del proceso de
modelización con el problema de Romeo y Julieta. Luego de desarrollar las técnicas apropiadas para
resolver estos problemas mediante el cálculo diferencial, completaremos las otras dos etapas.

\subsection{Identificación del problema}
Sabemos que las residencias de Romeo y Julieta, representadas con los puntos $R$ y $J$
respectivamente, están separadas por una laguna rectangular de $3$ kilómetros de ancho como se
ilustra en el siguiente dibujo:
\begin{center}
\begin{pspicture}(6,4)
   \psframe(0.5,0.5)(6,3)%
   \uput[180](0.5,3){$A$}%
   \uput[180](0.5,0.5){$B$}%

   \pstGeonode[PosAngle={90,180,-90},PointSymbol=none,PointNameSep={1em,0.5em,1em}]%
      (2.5,3){R}(0.5,2){P}(1.5,0.5){J}%

   \psline{|<->|}(0.5,3.1)(2.5,3.1)%
   \uput[90](1.5,3.25){$2$ km}%

   \psline{|<->|}(0.5,0.4)(1.5,0.4)%
   \uput[-90](1,0.4){$1$ km}%

   \psline{|<->|}(6.1,0.5)(6.1,3)%
   \uput[0](6.1,1.5){$3$ km}%

   \psline(R)(P)(J)%

   \rput(4,2.5){Laguna}%
   \rput[t](0,1){\pstVerb{
/vshowdict 4 dict def /vshow { vshowdict begin /thestring exch def /lineskip exch def thestring {
/charcode exch def /thechar ( ) dup 0 charcode put def 0 lineskip neg rmoveto gsave thechar
stringwidth pop 2 div neg 0 rmoveto thechar show grestore } forall end } def
   64 (lasoR) vshow }}

\end{pspicture}
\end{center}

El jardín se halla entre el extremo $A$ del lago, situado a $2$ kilómetros de la residencia de
Romeo, y el extremo $B$ del lago, situado a $1$ kilómetro de la de Julieta. Lo que debemos hallar
es el punto $P$ situado en algún lugar de la orilla $AB$ del lago, para que el recorrido total, $RP
+ PJ$, sea lo más corto posible.

\subsection{Elaboración del modelo matemático}
Este es un problema de extremos, cuya solución se obtiene a través del concepto de derivada. Más
adelante estudiaremos el siguiente modelo matemático que es adecuado para el problema de Romeo y
Julieta.

\begin{quote}
Dado un intervalo $I$ y una función $f\colon I \longrightarrow \mathbb{R}$ continua en $I$, se
busca $x_m\in I$ tal que
\[
f(x_m) = \min_{x\in I} f(x).
\]
\end{quote}

Determinemos la que sería la variable $x$, la función objetivo $f$ y el intervalo $I$ en el caso
del problema de Romeo y Julieta.

Para ello, recordemos que lo que debemos minimizar es la longitud del recorrido de Romeo en su
barca. Nombremos con $d$ el número de kilómetros de dicha distancia. Entonces, tenemos que
\[
d = RP + PJ,
\]
donde la longitud del recorrido desde la casa de Romeo, situada en $R$ hasta el punto $P$ ---allí
el joven recogerá las rosas para su amada---, es igual a $RP$ kilómetros, y la distancia recorrida
desde $P$ hasta el punto $J$ ---lugar en el que Julieta se encuentra--- es igual a $PJ$ kilómetros.

Sea $x = AP$, donde $AP$ kilómetros es la distancia recorrida en línea recta desde $A$ hasta $P$.

Como datos, tenemos que $AR = 2$, $AB = 3$, $BJ = 1$ y que los ángulos $\angle RAP$ y $\angle ABJ$
son rectos. En el triángulo $\triangle RAP$, $RP$ es la hipotenusa, por lo que
\[
RP = \sqrt{x^2 + 4}.
\]
En el triángulo rectángulo $\triangle PBJ$, se tiene que $PB = 3 - x$. Entonces, por el teorema de
Pitágoras, otra vez, se tiene que:
\[
PJ = \sqrt{(3 - x)^2 + 1}.
\]
Finalmente, tenemos que:
\[
d = \sqrt{x^2 + 4} + \sqrt{(3 - x)^2 + 1}.
\]

Puesto que se busca que la distancia $d$ recorrida por Romeo sea la mínima, la función cuyo mínimo
hay que hallar es $f$ definida por:
\[
f(x) = \sqrt{x^2 + 4} + \sqrt{(3 - x)^2 + 1}.
\]

De la definición de $x$ ---la distancia entre $A$ y $P$---, se tiene que $x\in (0,3) = I$.
Entonces, $f\colon I \longrightarrow \mathbb{R}$ está definida para todo $x\in I$.

Lo que resta del capítulo, lo vamos a dedicar a aprender cómo encontrar dicho $x$ en $I$ que
optimice la función $f$.

\section{Extremos globales o absolutos}

\begin{defical}[Conjunto acotado]
Un conjunto no vacío $A\subset \Rbb$ es \textbf{acotado por abajo (arriba)} si existe $c_1\in\Rbb$
($c_2\in\Rbb$) tal que $x \geq c_1$ ($x \leq c_2$) para todo $x\in A$. A $c_1$ se le llama
\textbf{cota inferior de $A$} y a $c_2$, \textbf{cota superior de $A$}.
\end{defical}

Sea $A\subset\Rbb$ y no vacío. Si $A$ es acotado por abajo, como consecuencia del axioma de
completitud del conjunto de los números reales $\Rbb$, existe la más grande de las cotas inferiores
de $A$, a la que se le llama \textbf{ínfimo de $A$} y se la representa por $\inf A$.

Análogamente, si $A$ es acotado por arriba, existe la más pequeña de las cotas superiores de $A$,
que se llama \textbf{supremo de $A$} y se la representa por $\sup A$.

Evidentemente, en estos casos, se verifican las siguientes desigualdades:
\[
x \geq \inf A \yjc x \leq \sup A
\]
para todo $x\in A$.

Puede ocurrir que $\inf A \in A$. En este caso, al ínfimo se le llama \textbf{mínimo de $A$} y se
lo representa por $\min A$.

De manera similar, puede suceder que $\sup A\in A$, en cuyo caso, se lo denomina \textbf{máximo de
$A$} y se lo representa por $\max A$.

Si $A$ no es acotado por abajo, se dice que el ínfimo es $-\infty$ y se escribe $\inf A = -\infty$.
Si $A$ no es acotado por arriba, se dice que el supremo es $\infty$ y se escribe $\sup A = \infty$.

Usando la notación de los ``infinitos'', cuando $A$ es acotado por abajo, se suele escribir $\inf A
> -\infty$, y cuando es acotado por arriba, $\sup A < \infty$.

Resumamos estos conceptos en la siguiente definición:

\begin{defical}[Ínfimo, supremo, mínimo y máximo]
Sea $A \subset \Rbb$. Entonces, se define
\[
\begin{array}{l}
\inf A =
\begin{cases}
\text{más grande de las cotas inferiores de $A$} & \text{si $A$ es acotado por abajo,} \\
-\infty & \text{si $A$ no es acotado por abajo};
\end{cases}
\\[16pt]
\sup A =
\begin{cases}
\text{más pequeña de las cotas superiores de $A$} & \text{si $A$ es acotado por arriba,} \\
\infty & \text{si $A$ no es acotado por arriba};
\end{cases}
\\[16pt]
\min A = \inf A \ \text{si $\inf A \in A$ y $\inf A > -\infty$}; \\[4pt]
\max A = \sup A \ \text{si $\sup A \in A$ y $\sup A < \infty$}.
\end{array}
\]
\end{defical}

En el siguiente dibujo se ilustran estas definiciones:
\begin{center}
\begin{pspicture}(0,0)(8,7)
\footnotesize%
\psset{}%

\pstGeonode[PointName={c_1,c_2}]%
  (2,0.5){A}(! \psGetNodeCenter{A} A.x 4 add A.y){A'}
\pstGeonode[PointSymbol={default,o},PointName=none]%
  (2,1.5){B}(! \psGetNodeCenter{B} B.x 4 add B.y){B'}
\uput[0](B){$\min A$}%
\uput[0](B'){$\not\exists\min A$}%
\uput[180](B){$\inf A$}%
\uput[180](B'){$\inf A$}%

\pstGeonode[PointName=none]%
  (2,2){C}
\pstGeonode[PointName=none,PointSymbol={o,o}]%
  (2,2.5){D}(! \psGetNodeCenter{D} D.x 4 add D.y 0.5 add){D'}

\pstGeonode[PointName=none,PointSymbol={default,o}]%
  (2,3.5){E}(! \psGetNodeCenter{E} E.x 4 add E.y){E'}

\pstGeonode[PointName=none]%
  (2,4){F}(! \psGetNodeCenter{F} F.x 4 add F.y 1 add){F'}
\uput[0](F'){$\max A$}%
\uput[180](F'){$\sup A$}%

\pstGeonode[PointName=none,PointSymbol=o]%
  (2,5){H}
\uput[0](H){$\not\exists\max A$}%
\uput[180](H){$\sup A$}%

\pstGeonode[PointName={c_2,c_2}]%
  (2,6){G}(! \psGetNodeCenter{G} G.x 4 add G.y){G'}

\uput[0](! \psGetNodeCenter{D} D.x D.y 0.5 sub){$A$}%
\uput[0](D'){$A$}%
\psaxes[arrows=->,ticks=none,xAxis=false,labels=none]%
  (2,0)(2.5,0)(2.5,6.75)%
\psaxes[arrows=->,ticks=none,xAxis=false,labels=none]%
  (6,0)(6.5,0)(6.5,6.75)%
\end{pspicture}
\end{center}

En el dibujo, representamos, sobre sendas rectas reales verticales, dos casos cuando $A\subset\Rbb$
es acotado por arriba y por abajo. En casos así, se dice, simplemente, que $A$ es acotado. Se
tiene, entonces, la siguiente definición.

\begin{defical}[Conjunto acotado]
Un conjunto $A\subset\Rbb$ es acotado si y solo si $A$ es acotado por arriba y por abajo.
\end{defical}

En otras palabras, un conjunto $A$ es acotado si y solo existen dos constantes reales $c_1$ y $c_2$
tales que
\[
c_1 \leq x \leq c_2
\]
para todo $x \in A$. Si tomamos $R = |\max\{c_1,c_2\}|$, entonces, $A$ es acotado si y solo si
existe $R > 0$ tal que
\[
|x| < R
\]
para todo $x \in A$.

La siguiente es una caracterización del supremo y del ínfimo, muy útil a la hora de trabajar con
estos dos números.

\begin{teocal}\label{teo:daCaracterizacionSupInf}
Sea $A\subset\Rbb$. Entonces:
\begin{enumerate}
\item Si $\inf A > -\infty$, entonces para todo $\epsilon > 0$, existe $a \in A$ tal que $\inf
    A \leq a < \inf A + \epsilon$.
\item Si $\sup A < \infty$, entonces para todo $\epsilon > 0$, existe $a \in A$ tal que $\sup A
    - \epsilon < a \leq \sup A$.
\end{enumerate}
\end{teocal}

Es decir, el $\inf A$ ($\sup A$) puede ser aproximado por un elemento de $A$ con la precisión que
queramos.

Gracias a estos conceptos, podemos introducir otros análogos para las funciones reales, que son
aquellas cuyo conjunto de llegada es $\Rbb$.

\begin{defical}[Función acotada]
Sean $\Omega\neq\emptyset$ y $\funcjc{f}{\Omega}{\Rbb}$. Entonces, las función $f$ es:
\begin{enumerate}
\item \textbf{acotada por abajo (arriba)} si y solo si el conjunto $\Img(f)$ es acotado por
    abajo (arriba).
\item \textbf{acotada} si y solo si el conjunto $\Img(f)$ es acotado.
\end{enumerate}
\end{defical}

A partir de las definiciones de conjuntos acotados, podemos caracterizar de la siguiente manera a
las funciones acotadas.

\begin{teocal}\label{teo:daCaracterizacionFuncAcotadas}
Sean $\Omega\neq\emptyset$ y $\funcjc{f}{\Omega}{\Rbb}$. Entonces, las función $f$ es:
\begin{enumerate}
\item \textbf{acotada por abajo} si y solo si el conjunto existe $c_1\in\Rbb$ tal que
    \[
      f(x) \geq c_1,
    \]
\item \textbf{acotada por arriba} si y solo si el conjunto existe $c_2\in\Rbb$ tal que
    \[
      f(x) \leq c_2,
    \]
\item \textbf{acotada} si y solo si el conjunto existen $c_1\in\Rbb$ y $c_2\in\Rbb$ tales que
    \[
      c_1 \leq f(x) \leq c_2,
    \]
\item \textbf{acotada} si y solo si el conjunto existe $R > 0$ tal que
    \[
      |f(x)| < R,
    \]
\end{enumerate}
para todo $x\in\Omega$.
\end{teocal}

Tenemos las siguientes definiciones.

\begin{defical}
Sean $\Omega\neq\emptyset$ y $\funcjc{f}{\Omega}{\Rbb}$. Definimos:
\begin{enumerate}
\item $\displaystyle\inf_{x\in\Omega} f(x) = \inf\Img(f)$.
\item $\displaystyle\sup_{x\in\Omega} f(x) = \sup\Img(f)$.
\item $\displaystyle\min_{x\in\Omega} f(x) = \min\Img(f)$.
\item $\displaystyle\max_{x\in\Omega} f(x) = \max\Img(f)$.
\end{enumerate}
\end{defical}

En el caso de que exista $\min\Img(f)$, existe $x_m\in\Omega$ tal que
\[
f(x_m) = \min_{x\in\Omega} f(x).
\]
Se dice, entonces, que \textbf{la función $f$ alcanza su mínimo en $x_m$}.

De manera similar, si existiera $\max\Img(f)$, existiría $x_M\in\Omega$ tal que
\[
f(x_M) = \max_{x\in\Omega} f(x).
\]
Se dice, entonces, que \textbf{la función $f$ alcanza su máximo en $x_M$}.

Los números $x_m$ y $x_M$ podrían no ser únicos; es decir, una función podría alcanzar o su máximo
o su mínimo en varios elementos de su dominio. Por ejemplo, la función $f$ definida por $f(x) =
|\sen(x)|$ para todo $x \in [-\frac{\pi}{2},\frac{\pi}{2}]$, alcanza su valor máximo $1$ tanto en
$-\frac{\pi}{2}$ como en $\frac{\pi}{2}$.

A los números $\displaystyle\min_{x\in\Omega} f(x)$, que se lee ``mínimo de $f$ en $\Omega$'', y
$\displaystyle\max_{x\in\Omega} f(x)$, que se lee ``máximo de $f$ en $\Omega$'', se les llama
\textbf{extremos globales o absolutos de $f$}.

Ilustremos estos conceptos con un ejemplo sencillo.

\begin{exemplo}[]{}
Sean $\Omega\subset\Rbb$ y $\funcjc{f}{\Omega}{\Rbb}$ definida por
\[
f(x) = 1 + (x - 2)^2.
\]
\begin{enumerate}[leftmargin=*]
\item Supongamos que $\Omega = \Rbb$. Entonces, el gráfico de $f$ es la parábola cuyo vértice
    es el punto de coordenadas $(2,1)$, como se puede ver en el siguiente dibujo:
    \begin{center}
    \begin{pspicture}(-1,-0.5)(5,6.5)
      \psaxes[arrows=->,ticks=none,labels=none]%
        (0,0)(-1,-0.5)(4.75,6)%
      \uput[-90](4.75,0){\footnotesize$x$}%
      \uput[0](0,5.75){\footnotesize$y$}

      \psplot[]%
        {-0.225}{4.225}{x 2 sub dup mul 1 add}%

      \uput[180](0,1){$1$}%
      \uput[-90](2,0){$2$}%
      \psline[linestyle=dashed,linecolor=gray]%
        (0,1)(2,1)(2,0)
    \end{pspicture}
    \end{center}
    Se puede ver, entonces, que la función $f$ es:
    \begin{enumerate}[leftmargin=*]
    \item no acotada por arriba, por lo que $\displaystyle\sup_{x\in\Rbb} f(x) = +\infty$;
    \item acotada por abajo y:
        \[
          \inf_{x\in\Rbb} f(x) = \min_{x\in\Rbb} f(x) = f(2) = 1.
        \]
    \end{enumerate}

\item Supongamos que $\Omega = [1,4[$. Entonces la función $f$ es:
    \begin{enumerate}[leftmargin=*]
    \item acotada por arriba, $\displaystyle\sup_{x\in\Omega} f(x) = 5$ y no existe
        $\displaystyle\max_{x\in\Omega} f(x)$;
    \item acotada por abajo, $\displaystyle\inf_{x\in\Omega} f(x) = 1 = \min_{x\in\Omega}
        f(x) = f(2)$;
    \item acotada.
    \end{enumerate}

\item Supongamos que $\Omega = \ ]2,4]$. Entonces la función $f$ es:
    \begin{enumerate}[leftmargin=*]
    \item acotada por arriba, $\displaystyle\sup_{x\in\Omega} f(x) = 5 = \max_{x\in\Omega}
        f(x) = f(4)$;
    \item acotada por abajo, $\displaystyle\inf_{x\in\Omega} f(x) = 1$, no existe
        $\min_{x\in\Omega} f(x)$;
    \item acotada.
    \end{enumerate}

\item Supongamos que $\Omega = [1,4]$. Entonces la función $f$ es:
    \begin{enumerate}[leftmargin=*]
    \item acotada;
    \item $\displaystyle\sup_{x\in\Omega} f(x) = 5 = \max_{x\in\Omega} f(x) = f(4)$;
    \item $\displaystyle\inf_{x\in\Omega} f(x) = 1 = \min_{x\in\Omega} f(x) = f(2)$.
    \end{enumerate}
\end{enumerate}
\end{exemplo}

El último numeral del ejemplo es un caso particular del resultado notable sobre la existencia de
los extremos globales:

\begin{teocal}[Existencia de los extremos globales para funciones
continuas]\label{teo:daExistenciaExtremosGlobales} Sean $a$ y $b$ en $\Rbb$ tales que $a < b$ y
$\funcjc{f}{[a,b]}{\Rbb}$, una función continua en $[a,b]$. Entonces existen $x_m$ y $x_M$ en
$[a,b]$ tales que
\[
f(x_m) = \min_{x\in [a,b]} f(x) \yjc f(x_M) = \max_{x\in [a,b]} f(x).
\]

Además, se tiene que $\Img(f) \subset [f(x_m),f(x_M)]$; es decir:
\[
f(x_m) \leq f(x) \leq f(x_M)
\]
para todo $x\in [a,b]$.
\end{teocal}

Como se verá más adelante, en realidad, se tiene que $\Img(f) = [f(x_m),f(x_M)]$.

El último ejemplo muestra lo indispensable que resulta exigir que $\Omega$ sea un intervalo cerrado
y acotado. Por ejemplo, si $\Omega$ es uno de los siguientes intervalos: $]-\infty,b]$, $\Rbb$,
$[a,+\infty[$, $]a,b]$, $]a,b[$, etcétera, el resultado podría no darse.

Es también de suma importancia que la función $f$ sea continua. Esto se ilustra en el siguiente
ejemplo.

\begin{exemplo}[]{}
Sean $\Omega = [-1,3]$ y $\funcjc{f}{\Omega}{\Rbb}$, definida por
\[
f(x) =
\begin{cases}
(x + 1)^2 & \text{si } -1 \leq x < 1, \\
1 & \text{si } x = 1, \\
-2 + x & \text{si } 1 < x \leq 3.
\end{cases}
\]
El siguiente es el gráfico de $f$:
\begin{center}
\begin{pspicture}(-2,-2)(4.5,5)
\psaxes[arrows=->,Dy=4]%
  (0,0)(-2,-2)(4,5)%
\uput[-90](4,0){$x$}%
\uput[0](0,5){$y$}%

\psplot[arrows=*-o]%
  {-1}{1}{x 1 add dup mul}%
\psplot[arrows=o-*]%
  {1}{3}{x 2 sub}%
\psdot[](1,1)

\end{pspicture}
\end{center}
Como se puede ver, $f$ no es continua en $x = 1$, por lo que $f$ no es continua en $\Omega$.
También se puede ver que
\[
\sup_{x\in\Omega} f(x) = 4 \yjc \not\exists \max_{x\in\Omega} f(x)
\]
y que
\[
\inf_{x\in\Omega} f(x) = -1 \yjc \not\exists \min_{x\in\Omega} f(x).
\]
\end{exemplo}

Si $I$ es un intervalo y $\funcjc{f}{I}{\Rbb}$ es una función continua, el cálculo de $\Img(f)$
puede no ser tan simple como el dado en el teorema~\ref{teo:daExistenciaExtremosGlobales} para el
caso $I = [a,b]$. Es muy útil, en estos casos, expresar $I$ (cuando sea posible) como la unión
finita de subintervalos de $I$, digamos $\displaystyle I = \bigcup_{k=1}^N I_k$, de modo que $f$
sea monótona en cada uno de los subintervalos $I_k$, porque se puede aplicar el resultado que sigue
y el hecho de que $\displaystyle \Img(f) = \bigcup_{k=1}^N f(I_k)$, donde $f(A) = \{f(x) : x\in
A\}$ y $A$ es un subconjunto del dominio de $f$.

\begin{lemacal}\label{teo:daImagenMonotonas}
Sean $I = ]a,b[$ y $\funcjc{f}{I}{\Rbb}$, continua y monótona en $I$. Entonces $f(I)$ es el
intervalo abierto de extremos $\displaystyle\limjc{f(x)}{x}{a^+}$ y $\displaystyle\limjc{f(x)}{x}{b^-}$.
\end{lemacal}

De manera más precisa, si $A = \displaystyle\limjc{f(x)}{x}{a^+}$ y $B = \displaystyle\limjc{f(x)}{x}{b^-}$, entonces
\[
\Img(f) = \ ]A,B[
\]
si $f$ es creciente; en cambio:
\[
\Img(f) = \ ]B,A[
\]
si $f$ es decreciente.

Si alguno de los límites laterales es infinito, digamos que $A = -\infty$; entonces tenemos que
\[
\Img(f) = \ ]-\infty, B[.
\]

El lema sigue siendo verdadero si en lugar de intervalos finitos se tienen intervalos infinitos.
Por otro lado, si $I$ es cerrado en un de los dos extremos, o ambos, al ser $f$ una función
continua en $I$, el limite se reemplaza con el correspondiente valor de la función. Por ejemplo, si
$I = [a,b[$ y $f$ es decreciente, entonces
\[
\Img(f) = \ ]B, f(a)].
\]

\begin{exemplo}[]{}
Si $\funcjc{f}{I}{\Rbb}$ está definida por
\[
f(x) = x^2 - 4x + 3,
\]
donde $I = \ ]-1,4]$, vamos a encontrar su imagen.

Sabemos que el gráfico de $f$ es una parábola de vértice el punto de coordenadas $(2,-1)$ tal que
la parábola es decreciente en $]-\infty, 2]$ y creciente en $[2,+\infty[$. En particular, $f$ es
decreciente en $I_1 = \ ]-1,2]$ y creciente en $I_2 = [2,4]$. Además sabemos que $I = I_1 \cup
I_2$.

Sabemos que
\[
f(I_1) = [f(2),\limjc{f(x)}{x}{-1^+}[ = [f(2),f(-1)[,
\]
ya que $f$ es continua en $I$. Por lo tanto:
\[
f(I_1) = [-1,8[.
\]

Por otro lado, tenemos que:
\[
f(I_2) = [f(2),f(4)] = [-1,3].
\]

Entonces:
\[
\Img(f) = f(I_1) \cup f(I_2) = [-1,8[ \cup [-1,3] = [-1,8[.
\]
Además:
\[
\min_{x\in I} f(x) = -1 \yjc \sup_{x\in I} f(x) = 8
\]
y no existe el máximo de $f$ en $I$.
\end{exemplo}

\section{Extremos locales o relativos}

\begin{wrapfigure}[14]{r}{0pt}
\def\f{x RadtoDeg dup dup sin exch 2 mul sin add exch 3 mul sin add}%
\psset{plotpoints=1000}
\begin{pspicture}(-3.5,-5)(4,3)
   %\psgrid[subgriddiv=0,gridlabels=7pt,griddots=10]
   \psaxes[ticks=none,labels=none]{->}%
      (0,0)(-3.5,-2.5)(3.75,3)%
   \uput[-90](3.75,0){$x$}%
   \uput[0](0,3){$y$}%
%
   \psplot[linecolor=gray]%
      {-\psPi}{\psPi}{\f}%
\end{pspicture}
\end{wrapfigure}

Dada una función real $f$ definida en un intervalo $I$, el gráfico de $f$ puede tener ``montes y
valles'', como se ilustra en gráfico de la derecha. Es interesante poder hallar las cimas ---punto
más alto de los montes, cerros y collados--- y simas ---cavidad grande y muy profunda en la
tierra--- que corresponderán a ``máximos locales'' y a ``mínimos locales'', respectivamente. ¿Qué
queremos decir con ``locales''? Que, como se aprecia en la figura, son los puntos más altos o los
más bajos pero solamente en una ``zona próxima a ellos'' y no con respecto a todo el intervalo $I$.
En otras palabras, ``localmente'' son los máximos o los mínimos. A continuación, hagamos precisas
estas nociones y cómo los extremos locales nos permiten hallar los extremos globales para el caso
de las funciones continuas.

\begin{defical}[Extremos locales]
Sean $I$ un intervalo y $\funcjc{f}{I}{\Rbb}$. Sean $\underline{x}$ y $\overline{x}$ dos elementos
de $I$. Entonces:
\begin{enumerate}
\item la función $f$ \textbf{alcanza un mínimo local o relativo en $\underline{x}$} si y solo
    si existe $r > 0$ tal que
    \[
    f(\underline{x}) \leq f(x)
    \]
    para todo $x \in I \cap \ ]\underline{x} - r, \underline{x} + r[$.

\item la función $f$ \textbf{alcanza un máximo local o relativo en $\overline{x}$} si y solo si
    existe $r > 0$ tal que
    \[
    f(\overline{x}) \geq f(x)
    \]
    para todo $x \in I \cap \ ]\underline{x} - r, \underline{x} + r[$.
\end{enumerate}
\end{defical}

En otras palabras, un mínimo local en $\underline{x}$ es el valor más pequeño que toma una función
en un cierto intervalo centrado en $\underline{x}$. De manera similar, un máximo local en
$\overline{x}$ es el valor más grande que toma una función en un cierto intervalo centrado en
$\overline{x}$.

\begin{defical}[Interior de un intervalo]
Sea $I$ un intervalo abierto. El interior de $I$, representado con $I^\circ$, es el intervalo
abierto más grande que está contenido en $I$.
\end{defical}

\begin{exemplo}[]{}
El interior de $I = [a,b]$ es $I^\circ = ]a,b[$. El interior de $I = [a,b[$ es $I^\circ = ]a,b[$.
Para $I = ]-\infty,b[$, el interior $I^\circ$ es $]-\infty, b[$.
\end{exemplo}

\begin{wrapfigure}{r}{0pt}
\def\f{x RadtoDeg dup dup sin exch 2 mul sin add exch 3 mul sin add}%
\psset{plotpoints=1000}
\begin{pspicture}(-3.5,-3)(4,3)
   %\psgrid[subgriddiv=0,gridlabels=7pt,griddots=10]
   \psaxes[ticks=none,labels=none]{->}%
      (0,0)(-3.5,-3)(3.75,3)%
   \uput[-90](3.75,0){$x$}%
   \uput[0](0,3){$y$}%

   \psplot[linecolor=gray]%
      {-\psPi}{\psPi}{\f}%

   \psset{linewidth=0.7pt}
   \psplotTangent{0.667291072}{0.5}{\f}%
   \psplotTangent{1.81519841}{0.5}{\f}%
   \psplotTangent{2.640083649}{0.5}{\f}%
   \psplotTangent{-0.66729107}{0.5}{\f}%
   \psplotTangent{-1.81519841}{0.5}{\f}%
   \psplotTangent{-2.640083649}{0.5}{\f}%
\end{pspicture}
\end{wrapfigure}

Lo que ahora nos interesa es determinar un método para la obtención de los extremos locales de una
función; es decir, para la localización de los máximos y mínimos locales de una función. En el
dibujo de la derecha, podemos observar que las rectas tangentes en los extremos de la función lucen
como rectas horizontales, es decir, con pendiente igual a $0$. Si esto es así, la derivada de la
función debería ser igual a $0$ en estos puntos. Y esto es, efectivamente, así. Es decir, la recta
pendiente en un extremo es horizontal. Por ello, se introduce el concepto de punto crítico. Luego
de la siguiente definición se enuncia el teorema ilustrado por el dibujo.

\begin{defical}[Punto crítico]
Dada una función real $f$, continua en un intervalo $I$ de extremos $a$ y $b$ ($a < b$), a un número $c$ del interior de $I$ se lo llama \textbf{punto
crítico de $f$} si y solo si $f'(c) = 0$ o si no existe $f'(c)$, es decir, si $f$ no es derivable
en $c$.
\end{defical}

\begin{exemplo}[]{}
Sea $\funcjc{f}{[-2,3]}{\Rbb}$ definida por
\[
f(x) =
\begin{cases}
1 - x^2 & \text{si } -2 \leq x \leq 1, \\
2x^3 - 3x^2 - 12x + 13 & \text{si } 1 < x \leq 3.
\end{cases}
\]
Encontremos los puntos críticos de $f$.

Definamos $p_1$ y $p_2$ de la siguiente manera:
\[
p_1(x) = 1 - x^2 \yjc p_2(x) = 2x^3 - 3x^2 - 12x + 13.
\]
Entonces, $f$ es continua en $[-2,3]$ pues:
\begin{enumerate}[leftmargin=*,listparindent=\parindent]
\item En el intervalo $[-2,1]$, la función $f = p_1$, por lo que $f$ es continua en el abierto
    $]-2,1[$, continua en $-2$ por la derecha y continua en $1$ por la izquierda.

\item En el intervalo $]1,3]$, la función $f = p_2$, por lo que $f$ es continua en el abierto
    $]1,3[$ y continua en $3$ por la izquierda.

\item En el punto $x = 1$, tenemos que:
    \[
      f(1) = p_1(1) = 0 = p_2(1) = \limjc{p_2(x)}{x}{1^+},
    \]
    por lo que $f$ es continua en $1$.
\end{enumerate}

Por otro lado, tenemos que
\[
f'(x) =
\begin{cases}
-2x & \text{si } -2 \leq x < 1, \\
6x^2 - 6x - 12 & \text{si } 1 < x \leq 3,
\end{cases}
\]
y para $x = 1$, no existe $f'(x)$, pues
\[
 f_-'(1) = -2 \neq -12 = f_+'(1).
\]
Entonces
\[
f'(x) = 0 \ \Longleftrightarrow \
\begin{cases}
-2x = 0 & \text{si } -2 \leq x < 1, \\
6x^2 - 6x - 12 = 0 & \text{si } 1 < x \leq 3,
\end{cases}
\]
de donde
\[
f'(x) = 0 \ \Longleftrightarrow \
x \in\{0,2\}.
\]

Por lo tanto, de acuerdo a la definición de punto crítico, a más de $0$ y de $2$, el número $1$
también es un punto crítico de $f$.
\end{exemplo}

El siguiente teorema es el mecanismo que tenemos para la determinación de los extremos relativos.

\begin{teocal}[Extremos locales]\label{teo:daExtremosLocalesPuntosCriticos}
Sean $I$ un intervalo, $\funcjc{f}{I}{\Rbb}$ y $c\in I^\circ$. Si $f$ alcanza un extremo local en
$c$, entonces $c$ es un valor crítico de $f$.
\end{teocal}

Usaremos este teorema para la obtención de los extremos absolutos de una función real continua
definida en un intervalo cerrado y acotado.

En efecto, supongamos que $\funcjc{f}{[a,b]}{\Rbb}$ es continua. Entonces, existen $x_m$ y $x_M$ en
$[a,b]$ en los cuales $f$ alcanza el mínimo y el máximo globales, respectivamente. Sea $K$ el
conjunto de todos los puntos críticos de $f$; es decir:
\[
K = \{c \in \ ]a,b[ : c \ \text{ es un punto crítico de } \ f\}.
\]
Se tiene entonces que $x_m\in \{a,b\}\cup K$ y que $x_M\in \{a,b\}\cup K$, pues todo extremo global
también es un extremo local.

Ahora bien, si $K$ es finito, por ejemplo, $K = \{c_1, c_2,\ldots, c_N\}$, y es conocido, para
encontrar los extremos globales de $f$ es suficiente que hagamos una tabla de valores con los
elementos del conjunto
\[
\{f(a), f(c_1), f(c_2), \ldots, f(c_N), f(b)\}.
\]
Es obvio que el mayor de estos números corresponderá al máximo global de $f$ y el menor, al mínimo
global. En el caso de que $K = \emptyset$, los únicos extremos son $f(a)$ y $f(b)$. Veamos un
ejemplo.

\begin{exemplo}[]{}
Sea $I = [0,\frac{5}{2}]$ y $\funcjc{f}{I}{\Rbb}$ definida por
\[
f(x) = 2x^3 - 9x^2 + 12x.
\]
Determinemos el conjunto $K$.

Como $f'(x) = 6x^2 - 18x + 12 = 6(x - 2)(x-1)$, tenemos que los $c\in I$ que satisfacen la igualdad
$f'(c) = 0$ son $c = 1$ y $c = 2$. Además, como $f$ es derivable en el interior de $I$, el conjunto
$K$ es, entonces $K = \{1,2\}$.

Ahora construyamos la tabla de valores de $f$ en $K \cup \{f(0),f(\frac{5}{2})\}$:
\[
\setlength\extrarowheight{4pt}
\begin{array}{c|cccc}
x & 0 & 1 & 2 & \frac{5}{2} \\[4pt] \hline
f(x) & 0 & 5 & 4 & 5
\end{array}
\]
Por lo tanto, la función $f$ alcanza el mínimo en $0$ y el máximo en $\frac{5}{2}$. Además:
\[
\min_{x\in I} f(x) = 0 = f(0) \yjc \max_{x\in I} f(x) = 5 = f\left(\frac{5}{2}\right)=f(1).
\]
\end{exemplo}

\section{Monotonía}

Dada una función $\funcjc{f}{I}{\Rbb}$, donde $I$ es un intervalo, se dice que $f$ es creciente si
al aumentar el valor de la variable independiente $x$, lo hace también el valor $f(x)$. Si $f(x)$
decrece, entonces se dice que $f$ es decreciente. Precisemos estos conceptos.

\begin{defical}[Funciones creciente y decreciente]
Sean $I$ un intervalo y $\funcjc{f}{I}{\Rbb}$. Entonces la función $f$ es
\begin{enumerate}
\item \textbf{creciente en $I$} si y solo si
    $
      f(x_1) < f(x_2)
    $; para todo $x_1 \in I$ y todo $x_2 \in I$ tales que $x_1 < x_2$; y, 
      
\item \textbf{no decreciente en $I$} si y solo si
    $
      f(x_1) \leq f(x_2)
    $; para todo $x_1 \in I$ y todo $x_2 \in I$ tales que $x_1 < x_2$; y,

\item \textbf{decreciente en $I$} si y solo si
    $
      f(x_1) > f(x_2)
    $; para todo $x_1 \in I$ y todo $x_2 \in I$ tales que $x_1 < x_2$; y,

\item \textbf{no creciente en $I$} si y solo si
    $
      f(x_1) \geq f(x_2)
    $; para todo $x_1 \in I$ y todo $x_2 \in I$ tales que $x_1 < x_2$; y,

\item \textbf{monótona} si y solo si es creciente o decreciente.
\end{enumerate}
\end{defical}

Es claro de la definición que una función puede ser creciente en un subintervalo pero creciente en
otro.

\begin{exemplo}[]{}
Sea $\funcjc{f}{]\frac{3}{2},+\infty [}{\Rbb}$ definida por $f(x) = x^2 - 3x + 2$. Determinemos si es
creciente o decreciente. Para ello, sean $x_1$ y $x_2$ tales que sean mayores que $\frac{3}{2}$ y
$x_1 < x_2$. Ahora estudiemos el signo de $f(x_2) - f(x_1)$.

Para ello, observemos que:
\begin{align*}
f(x_2) - f(x_1) &= (x_2^2 - 3x_2 + 2) - (x_1^2 - 3x_1 + 2) \\
  &= (x_2^2 - x_1^2) - 3(x_2 - x_1) \\
  &= (x_2 - x_1)(x_2 + x_1 - 3) \\
  &= (x_2 - x_1)[(x_2 - \frac{3}{2}) + (x_1 - \frac{3}{2})] > 0,
\end{align*}
pues $x_1 < x_2$, $x_1 > \frac{3}{2}$ y $x_2 > \frac{3}{2}$. Por lo tanto, $f(x_1) < f(x_2)$; es
decir, $f$ es creciente.
\end{exemplo}

Probar que una función es creciente o decreciente no siempre es una tarea sencilla como la que
muestra el último ejemplo. El lector podrá convencerse de esta afirmación si estudia la monotonía
de la función $f$ definida por $f(x) = 2x^3 - 9x^2 + 12x$ con el método utilizado en el ejemplo.
Constataría lo engorrosos que pueden llegar a ser los cálculos.

\begin{wrapfigure}[14]{r}{0pt}
\def\f{x RadtoDeg dup dup sin exch 2 mul sin add exch 3 mul sin add}%
\psset{plotpoints=1000}
\begin{pspicture}(-3.5,-3)(4,3)
   %\psgrid[subgriddiv=0,gridlabels=7pt,griddots=10]
   \psaxes[ticks=none,labels=none]{->}%
      (0,0)(-3.5,-3)(3.75,3)%
   \uput[-90](3.75,0){$x$}%
   \uput[0](0,3){$y$}%
%
   \psplot[linecolor=gray]%
      {-\psPi}{\psPi}{\f}%
%
   \psset{linewidth=0.6pt}
   \psplotTangent{-2.8}{0.5}{\f}%
   \psplotTangent{-2}{0.5}{\f}%
   \psplotTangent{-0.7}{0.5}{\f}%
   \psplotTangent{0.5}{0.5}{\f}%
   \psplotTangent{1.6}{0.5}{\f}%
   \psplotTangent{2.5}{0.5}{\f}%
\end{pspicture}
\end{wrapfigure}

Viene entonces en nuestra ayuda un resultado que se ilustra, previamente, en el dibujo de la
derecha. Podemos observar que la recta tangente en cualquier punto de la curva en los intervalos en
los que es creciente tiene una pendiente positiva. En cambio, en los intervalos en que la curva es
decreciente, la pendiente es negativa.

Esta observación, y el hecho de que la derivada representa la pendiente de la tangente, inducen a
pensar que si la derivada de una función fuera positiva en $(a,b)$, esa función debería ser
creciente en ese intervalo y, en el caso contrario, si la derivada fuera negativa, la función
debería ser decreciente.

\begin{teocal}\label{teo:daDerivadaPositiva}%
Sean $I$ un intervalo de extremos $a < b$ $f\colon I \rightarrow  \mathbb{R}$ continua en $I$. Si,
además, $f$ es derivable en $]a,b[$, entonces:
\begin{enumerate}
\item si $f'(x) > 0$ para todo $x\in\ ]a,b[$, entonces $f$ es creciente en $I$; y
\item si $f'(x) < 0$ para todo $x\in\ ]a,b[$, entonces $f$ es decreciente en $I$.
\end{enumerate}
\end{teocal}

Veamos un ejemplo.

\begin{exemplo}[]{}
Sea $\funcjc{f}{\Rbb}{\Rbb}$ definida por $f(x) = 2x^3 - 9x^2 + 12x$. Su derivada es $f'(x) = 6x^2
- 18x + 12$. Como $f'(x) = 6(x - 1)(x - 2)$, los signos de $f$ son los siguientes:
\[
\begin{array}{c|c|c|c|c|c}
x & & 1 & & 2 & \\ \hline
(x - 1) & - & 0 & + & + & + \\
(x - 2) & - & - & - & 0 & + \\
f'(x) & + & 0 & - & 0 & +
\end{array}
\]

Vemos, entonces, que $f'(x) > 0$ si $x \in \ ]-\infty, 1[ \cup \ ]2, +\infty[$ y que $f'(x) < 0$ si
$x \in \ ]1,2[$. Por lo tanto, la función $f$ es creciente en $]-\infty, 1[ \cup \ ]2, +\infty[$ y
es decreciente en $]1,2[$.
\end{exemplo}

\begin{corocal}[Extremos locales]
Sean $I$ un intervalo abierto, $c \in I$ y $\funcjc{f}{I}{\Rbb}$ una función continua en $I$ y
derivable en $I - \{c\}$. Si $c$ es un punto crítico de $f$ y $f'(x)$ cambia de signo en $c$,
entonces $f$ alcanza un extremo local en $c$. De manera más precisa: si $x\in I$ y
\begin{enumerate}
\item si $f'(x) < 0$ para todo $x < c$; y
\item si $f'(x) > 0$ para todo $x > c$,
\end{enumerate}
entonces $f$ alcanza un mínimo local en $c$; y
\begin{enumerate}
\item si $f'(x) > 0$ para todo $x < c$; y
\item si $f'(x) < 0$ para todo $x > c$,
\end{enumerate}
entonces $f$ alcanza un máximo local en $c$.
\end{corocal}

\begin{exemplo}[]{}
En el ejemplo precedente, si $f$ está definida por $f(x) = 2x^3 - 9x^2 + 12x$, la derivada $f'(x)$
de esta función cambia de signo en $1$ y en $2$. Antes de $1$ es creciente, luego de $1$ es
decreciente; por lo tanto, $f$ alcanza un máximo local en $1$. Antes de $2$ es decreciente y luego
es creciente. Entonces, $f$ alcanza un mínimo local en $2$.
\end{exemplo}

\begin{multicols}{2}[\section{Ejercicios}]
\begingroup
\small
\begin{enumerate}[leftmargin=*]
\item Para la función $f$, halle los puntos críticos y los intervalos de monotonía. Distinga
    los puntos críticos que son extremos locales:
    \begin{enumerate}[leftmargin=*]
    \item $\displaystyle f(x) = 3x^2 - 5x + 4$.
    \item $\displaystyle f(x) = 6x^4 - 8x^3 + 2$.
    \item $\displaystyle f(x) = \frac{x + 1}{x^2 + 1}$.
    \item $\displaystyle f(x) = x^{\frac{4}{2}} - x^{\frac{1}{3}}$.
    \item $\displaystyle f(x) = \frac{x - 1}{(x + 7)^{\frac{1}{3}}}$.
    \item $\displaystyle f(x) = x^3 + 2x^2 - 7x + 4$.
    \item $\displaystyle f(x) = \frac{x - 2}{\sqrt{1 - x}}$.
    \item $\displaystyle f(x) = (5x + 2)^{\frac{1}{3}}$.
    \item $\displaystyle f(x) = x^2(x + 7)^{\frac{1}{3}}$.
    \item $\displaystyle f(x) = 3\sen x + 4\cos x$.
    \end{enumerate}

\item Halle los extremos absolutos de $f$ en el intervalo $I$:
    \begin{enumerate}[leftmargin=*]
    \item $\displaystyle f(x) = x^3 - 2x^2 + x; \ I = [-1,4]$.
    \item $\displaystyle f(x) = 3x^2 - 5x + 2; \ I = [-1,3]$.
    \item $\displaystyle f(x) = (x - 1)^{\frac{2}{3}}; \ I = [0,10]$.
    \item $\displaystyle f(x) = (x - 1)^{\frac{2}{3}}(x^2 - 2x); \ I = [0,2]$.
    \item $\displaystyle f(x) = x^3 + x^2 - 5x + 3; \ I = [-2,4]$.
    \item $\displaystyle f(x) = (x + 1)^4(x - 2)^2; \ I = [0,5]$.
    \item $\displaystyle f(x) = \frac{\sqrt{x - 1}}{x^2 + 2}; \ I =
        \left[\frac{5}{4},5\right]$.
    \end{enumerate}

\item Halle los valores críticos y los extremos absolutos y relativos de la función $f$ en el
    intervalo $I$:
    \begin{enumerate}[leftmargin=*]
    \item $\displaystyle f(x) = x^4 - 4 - |x + 2|; \ I = [-3,3]$.
    \item $\displaystyle f(x) =
          \begin{cases}
            2x + 1 & \text{si } -2 \leq x < 1, \\
            x^2 + 2 & \text{si } 1 \leq x \leq 3;
          \end{cases}\\[6pt]
          I = [-2,3]$.
    \item $\displaystyle f(x) = 2x^3 - 3x^2 - 12x + 13; \ I = [-2,3]$.
    \item $\displaystyle f(x) =
          \begin{cases}
            x^2 + x - 2 & \text{si } -2 \leq x < 1, \\
            x^3 - 3x^2 + 2 & \text{si } 1 \leq x \leq 3;
          \end{cases}\\[6pt]
          I = [-2,3]$.
    \item $\displaystyle f(x) = x + \frac{1}{x - 1} - 1; \ I = [-10,10]$.
    \item $\displaystyle f(x) = \lfloor x \rfloor; \ I = [-10,2]$.
    \item $\displaystyle f(x) = \frac{ax + b}{cx + d}; \ I = [-L,L]$, con $L > 0$, y si $ad
        \neq bc$ o si $ad = bc$.
    \end{enumerate}
\end{enumerate}
\endgroup
\end{multicols}

\section{Teoremas del valor intermedio}

\begin{wrapfigure}[11]{r}{0pt}
\def\f{2*cos(x-4.5)-2*sin(2*x - 9)+2*cos(2*x - 9)-2*sin(x - 4.5)+ 2.5 }%
\def\g{x RadtoDeg dup dup dup 4.5 RadtoDeg sub cos 2 mul exch
   4.5 RadtoDeg sub 2 mul sin 2 mul sub exch
   4.5 RadtoDeg sub 2 mul cos 2 mul add exch
   4.5 RadtoDeg sub sin 2 mul sub 2.5 add }

\begin{pspicture}(-0.5,-0)(5,4.25)
   %\psgrid[subgriddiv=0,gridlabels=7pt,griddots=10]
   \psaxes[ticks=none,labels=none]{->}%
      (0,0)(-0.5,-0.5)(4.75,4)%
   \uput[-90](4.75,0){$x$}%
   \uput[0](0,4){$y$}%

   \begingroup
      \psset{linecolor=gray}
      \psline(2.712185027,0)(! 2.712185027 /x 2.712185027 def \g)%
      \psline(0.5,0)(! 0.5 /x 0.5 def \g)%
      \psline[linestyle=dashed,linecolor=gray]%
         (!0.5 /x 0.5 def \g)(! 2.712185027 /x 2.712185027 def \g)%

      \uput[-90](0.5,0){$a$}%
      \uput[180](! 0.5 /x 0.5 def \g 2 div){$f(a)$}%
      \uput[-90](2.712185027,0){$b$}%
      \uput[0](! 2.712185027 /x 2.712185027 def \g 2 div){$f(b)$}%
   \endgroup

   \psplot[linewidth=1.5\pslinewidth]%
      {0.5}{2.712185027}{\g}%

   \psplotTangent[plotpoints=1000]%
      {1.090732918}{1}{\g}

   \pnode(! 1.090732918 /x 1.090732918 def \g){M}

   \psline[linestyle=dashed,linecolor=gray,linewidth=0.75\pslinewidth]%
         (!1.090732918 0)(! 1.090732918 /x 1.090732918 def \g)%
   \uput[-90](1.090732918,0){$c$}%

   \pnode(2,3.2){A}%
   \psline[linecolor=gray]{->}(A)(M)%
   \uput[0](A){$f'(c) = 0$}

\end{pspicture}
\end{wrapfigure}
En el dibujo de la derecha, se muestra el gráfico de una función $f\colon [a,b] \rightarrow
\mathbb{R}$ que es continua en $[a,b]$, derivable en $(a,b)$ y tal que $f(a)=f(b)$. Parece que, en
el punto $c\in (a,b)$, punto en el que la función $f$ alcanza un máximo local, la tangente al
gráfico de $f$ es horizontal; es decir, $f'(c) = 0$, de donde $c$ es un punto crítico. Aunque no
está dibujado, el punto dónde se alcanza el mínimo local, la recta tangente es también paralela al
eje horizontal; esto significa también que ese punto es un punto crítico.

Y esta situación es siempre verdadera bajo las hipótesis adecuadas, como se expresa en el siguiente
teorema.

\begin{teocal}[Teorema de Rolle]
Si $f $ es continua en el intervalo $[a,b]$ y derivable en $(a,b)$, entonces existe un número $c\in
(a,b)$ tal que:
\[
f(a) = f(b).
\]
Entonces, existe un número $c\in (a,b)$ tal que $f'(c) = 0$.
\end{teocal}

\begin{exemplo}[]{}
Vamos a utilizar el teorema de Rolle para averiguar si la función $f$ definida por
\[
f(x) = x^3 - 3x^2 + 2
\]
tiene puntos críticos en el intervalo $]-\sqrt{3} + 1,\ \sqrt{3} + 1[$.

Como $f$ es derivable en $\Rbb$ por ser un polinomio, es continua en $\Rbb$ y, por consiguiente,
también es continua en $]-\sqrt{3} + 1,\ \sqrt{3} + 1[$. Por lo tanto, los puntos críticos son
todos los $x\in\ ]-\sqrt{3} + 1,\ \sqrt{3} + 1[$ para los cuales $f'(x) = 0$.

Como $f(-\sqrt{3} + 1) = f(\sqrt{3} - 1) = 0$, el teorema de Rolle nos garantiza que existe $c$ en
$]-\sqrt{3} + 1,\ \sqrt{3} + 1[$ tal que $f'(c) = 0$. Para hallar tales números $c$, debemos
resolver la ecuación $f'(c) = 0$.

Ahora bien, como $f'(x) = 3x^2 - 6x = 3x(x - 2)$, tenemos que
\[
f'(c) = 0 \ \Longleftrightarrow \ c(c - 2) = 0 \ \Longleftrightarrow \ c \in\{0,2\}.
\]

En resumen, los números $0$y $2$ son los puntos críticos buscados, puesto que $\{0,2\}$ está
incluido en $]-\sqrt{3} + 1,\ \sqrt{3} + 1[$.
\end{exemplo}

\begin{teocal}[Teorema del valor intermedio o de los incrementos finitos]
Si $f $ es continua en el intervalo $[a,b]$ y derivable en $(a,b)$, entonces existe un número $c\in
(a,b)$ tal que:
\begin{equation*}
	f'(c)=\frac{f(b)-f(a)}{b-a},
\end{equation*}
o también:
\[
   f(b) - f(a) = f'(c)(b - a).
\]
\end{teocal}

\begin{exemplo}[]{}
Sea $f$ la función real definida por $f(x) = x^3 - 3x^2 + 3$. Veamos cómo se aplica el teorema del
valor intermedio en el intervalo $[1,3]$.

Como $f$ es un polinomio, es una función continua y derivable en $\Rbb$ y, por ende, es continua en
el intervalo $[1,3]$ y derivable en el intervalo $]1,3[$. Entonces, se verifican las hipótesis del
teorema del valor intermedio, el que nos garantiza que existe $c\in\ ]1,3[$ tal que
\[
f'(c) = \frac{f(3) - f(1)}{2 - 1} = \frac{3 - 1}{3 - 1} = 1.
\]
Como $f'(x) = 3x^2 - 6x$, el número $c$ satisface la ecuación
\[
3c^2 - 6c = 1.
\]

Las raíces de esta ecuación son $c_1 = 1 - \frac{2}{\sqrt{3}}$ y $c_2 = 1 + \frac{2}{\sqrt{3}}$.
Puesto que $c_1\not\in\ ]1,3[$, entonces el número $c$ del teorema del valor intermedio es $c_2$.
\end{exemplo}

Supongamos que una función $\funcjc{f}{I}{\Rbb}$, donde $I$ es un intervalo de extremos $a$ y $b$
tales que $a < b$, tiene derivada nula en $]a,b[$; es decir, supongamos que $f'(x) = 0$ para todo
$x\in\ ]a,b[$. Tomemos dos elementos en $I$, $x_1$ y $x_2$ tales que $x_1 < x_2$. Si sucediera que
$f(x_1) \neq f(x_2)$, por el teorema del valor intermedio, existe $c\in\ ]x_1,x_2[$ tal que
\[
f'(c) = \frac{f(x_2) - f(x_1)}{x_2 - x_1}.
\]
Por un lado, $f'(c) = 0$; por otro, como $x_1 \neq x_2$ y, hemos supuesto que $f(x_1)\neq f(x_2)$,
tenemos que
\[
0 = f'(c) = \frac{f(x_2) - f(x_1)}{x_2 - x_1} \neq 0.
\]
Esta contradicción nos indica que no es posible que $f(x_1) \neq f(x_2)$; por lo tanto, la función
$f$ debe ser constante en $]a,b[$.

Resumamos este resultado en el siguiente teorema.

\begin{teocal}\label{teo:daDerivadaCero}
Si $f$ está definida y es continua en un intervalo $I$, y si $f$ es derivable en el interior de $I$ y su derivada es igual a $0$, entonces $f$ es constante en $I$.
\end{teocal}

Finalmente, el siguiente es una generalización del teorema del valor intermedio. En los ejercicios
de esta sección, se provee una sugerencia para la demostración del teorema general.

\begin{teocal}[Teorema general del valor intermedio]
Sean $f\colon \mathbb{R} \rightarrow  \mathbb{R}$ y $g\colon \mathbb{R} \rightarrow  \mathbb{R}$
dos funciones continuas en el intervalo $[a,b]$ y derivables en $(a,b)$. Supongamos que $g'(x)\neq
0$ para todo $x\in (a,b)$ y que $g(a)\neq g(b)$. Entonces existe $c\in (a,b)$ tal que:
\begin{equation*}
	 \frac{f(b)-f(a)}{g(b)-g(a)}=\frac{f'(c)}{g'(c)}.
\end{equation*}
\end{teocal}

\begin{multicols}{2}[\section{Ejercicios}]
\begingroup
\small
\begin{enumerate}[leftmargin=*]
\item Verifique si en el intervalo $I$ se puede aplicar el teorema de Rolle para la $f$ dada.
    De ser así, halle los $c$ cuya existencia garantiza el mencionado teorema.
    \begin{enumerate}[leftmargin=*]
    \item $\displaystyle f(x) = x^2 - 5;\ I = [-2,2]$.
    \item $\displaystyle f(x) =
          \begin{cases}
            x^3 + 2 & \text{si } -3 \leq x < 0, \\
            -x^3 + 2 & \text{si } 0 \leq x \leq 3;
          \end{cases}\\[6pt]
          I = [-3,3]$.
    \item $\displaystyle f(x) = x^3 - 3x^2 + 2x + 4;\ I = [0,1]$.
    \item $\displaystyle f(x) =
          \begin{cases}
            x^2 + 3 & \text{si } -2 \leq x \leq 0, \\
            8 - x^3 & \text{si } 0 < x \leq 1;
          \end{cases}\\[6pt]
          I = [-2,1]$.
    \item $\displaystyle f(x) = 3\sen x + 4\cos x;\ I =
        \left[\frac{-\pi}{2},\frac{7\pi}{2}\right]$.
    \item $\displaystyle f(x) = \sec x;\ I = [0,2\pi]$.
    \end{enumerate}

\item Verifique si en el intervalo $I$ la función $f$ satisface las hipótesis del teorema del
    valor intermedio. De ser así, halle los valores de $c$ cuya existencia nos es garantizada
    por dicho teorema.
    \begin{enumerate}[leftmargin=*]
    \item $\displaystyle f(x) = x^3 - x + 1;\ I = [0,3]$.
    \item $\displaystyle f(x) =
          \begin{cases}
            x^3 + 2 & \text{si } -2\leq x < 0,\\
            2 - x^3 & \text{si } 0 \leq x \leq 3;
          \end{cases}\\[6pt]
          I = [-2,3]$.
    \item $\displaystyle f(x) = x^3 - 3x^2 + 2x;\ I = [0,1]$.
    \item $\displaystyle f(x) =
          \begin{cases}
            x^2 + 3 & \text{si } -2 \leq x \leq 0, \\
            3 - x^3 & \text{si } 0 \leq x \leq 1;
          \end{cases}\\[6pt]
          I = [-2,1]$.
    \item $\displaystyle f(x) =
          \begin{cases}
            x^3 + 2 & \text{si } -3 \leq x \leq 1, \\
            2x^2 + 3x - 2 & \text{si } 1 \leq x \leq 2;
          \end{cases}\\[6pt]
          I = [-3,2]$.
    \item $\displaystyle f(x) = x - \sqrt[3]{x};\ I = [-1,27]$.
    \end{enumerate}

\item Un camionero entra en una autopista y recibe un talón que marca la hora de ingreso, las
    $7$ horas con $55$ minutos. Antes de salir, $250$ kilómetros después, paga el peaje y el
    recibo marca el valor pagado y la hora del pago: $10$ horas con $35$ minutos. Un policía le
    revisa los documentos y le multa por exceso de velocidad. ¿Cuál cree que era el límite de
    velocidad para el transporte pesado en ese tramo? Justifique su respuesta sustentándose en
    el teorema del valor intermedio.

\item Mediante el teorema del valor intermedio, demuestre el siguiente teorema (de los
    incrementos finitos). Sean $I$ un intervalo abierto, $\funcjc{f}{I}{\Rbb}$ derivable en
    $I$, $x_0\in I$ y $\Delta x$ tales que $x_0 + \Delta x \in I$. Entonces existe $\theta\in\
    ]0,1[$ tal que
    \[
      \Delta y = f(x_0 + \Delta x) - f(x_0) = f'(x_0 + \theta\Delta x)\Delta x.
    \]

\item Usando el teorema de Rolle, pruebe el teorema general del valor medio.
    \textsc{Sugerencia:} Defina $\funcjc{F}{[a,b]}{\Rbb}$ de la siguiente manera:
    \[
      F(x) = f(x) - \frac{f(b) - f(a)}{g(b) - g(a)}[g(x) - g(a)].
    \]

\item Sea $f(x) = ax^5 + bx^3 + cx + d$. Use el teorema de Rolle para probar que la ecuación
    $f(x) = 0$ no puede tener dos raíces reales si $a > 0$, $b > 0$, $c > 0$ y si $20ac>9b^2$.

\item Use el teorema de Rolle y el método de inducción para probar que un polinomio de grado
    $n\in\Nbb$ y coeficientes reales tiene a lo sumo $n$ raíces reales.
\end{enumerate}
\endgroup
\end{multicols}

\section{Convexidad}

\subsection{Punto intermedio}

Sean $P_{0}$ y  $P_{1}$ dos puntos cualesquiera, distintos entre sí y situados sobre una recta
real. Sean  $x_{0}$ y  $x_{1}$ sus respectivas coordenadas. Si tomamos un punto arbitrario $P$
situado entre $P_{0}$ y  $P_{1}$, de coordenada $x$, resulta que $x$ puede expresarse de una manera
sencilla en función de $x_{0}$  y  $x_{1}$. En efecto, tenemos el siguiente lema:

\begin{lemacal}
Sean $x_{0}$  y  $x_{1}$ en $\mathbb{R}$ dos números reales distintos entre sí. Entonces, para
cualquier $x\in \mathbb{R}$ ``situado'' entre ellos; es decir, tal que $x_0 < x < x_1$ o $x_1 < x <
x_0$, existe $t\in (0,1)$ tal que:
\begin{equation}
\label{eq:algeom001}
x=tx_{1}+(1-t)x_{0}.
\end{equation}
\end{lemacal}

Tenemos dos casos.
\begin{enumerate}[leftmargin=*]
\item Supongamos que $x_0 < x < x_1$: \quad
      %
      \begin{pspicture}(0,0)(4.5,1)
        \psaxes[arrows=->,ticks=none,yAxis=false,labels=none]%
          (0,0)(0,0.5)(4.5,0.5)%
        \pstGeonode[PointSymbol=|,PosAngle=90]%
          (0.5,0){P_0}(2.5,0){P}(3.5,0){P_1}%
        \uput[-90](P_0){$x_0$}%
        \uput[-90](P_1){$x_1$}%
        \uput[-90](P){$x$}%
      \end{pspicture}
      %

      \vspace{0.6\baselineskip}%
      Vemos que las distancias entre $P$ y $P_0$ y entre $P_0$ y $P_1$ son
      \[
        PP_0 = x - x_0 \yjc P_0P_1 = x_1 - x_0.
      \]
      Además, como
      \[
        0 < P_0P < P_0P_1,
      \]
      existe $t \in\ ]0,1[$ tal que
      \[
        P_0P = tP_0P_1 = t(x_1 - x_0).
      \]
      Por lo tanto:
      \[
        x = x_0 + P_0P = x_0 + t(x_1 - x_0) = tx_1 + (1 - t)x_0.
      \]

\item Supongamos que $x_1 < x < x_0$: \quad
      %
      \begin{pspicture}(0,0)(4.5,1)
        \psaxes[arrows=->,ticks=none,yAxis=false,labels=none]%
          (0,0)(0,0.5)(4.5,0.5)%
        \pstGeonode[PointSymbol=|,PosAngle=90]%
          (0.5,0){P_1}(2.5,0){P}(3.5,0){P_0}%
        \uput[-90](P_1){$x_1$}%
        \uput[-90](P_0){$x_0$}%
        \uput[-90](P){$x$}%
      \end{pspicture}
      %

      \vspace{0.6\baselineskip}%
      Vemos que las distancias entre $P$ y $P_0$ y entre $P_0$ y $P_1$ son
      \[
        PP_0 = x_0 - x \yjc P_0P_1 = x_0 - x_1.
      \]
      Además, como
      \[
        0 < P_0P < P_0P_1,
      \]
      existe $t \in\ ]0,1[$ tal que
      \[
        P_0P = tP_0P_1 = t(x_0 - x_1).
      \]
      Por lo tanto:
      \[
        x = x_0 - P_0P = x_0 - t(x_0 - x_1) = tx_1 + (1 - t)x_0.
      \]
\end{enumerate}

\subsection{Segmento que une dos puntos}

El conjunto de puntos de la recta real situados entre $P_{0}$ y $P_{1}$, incluidos estos, es
representado por $[P_{0},P_{1}]$; es decir:
\begin{equation}
\label{eq:algeom003}
[P_{0},P_{1}]= \{P_{t} : t\in [0,1] \},
\end{equation}
donde $x_{t}$, la coordenada de $P_{t}$, se calcula con la fórmula
\begin{equation}
\label{eq:algeom004}
x_{t} = tx_{1}+(1-t)x_{0}.
\end{equation}
Esta fórmula es una notación ``feliz'', ya que, para $t = 0$ y $t = 1$, se obtienen $x_0$ y $x_1$,
respectivamente.

Las fórmulas~(\ref{eq:algeom003}) y~(\ref{eq:algeom004}) se generalizan a dimensiones superiores.
Así, por ejemplo, si $P_{0},\ P_{1}$ están en $\mathbb{E}^{3}$, y si $(x_{0},y_{0},z_{0})$ y
$(x_{1},y_{1},z_{1})$ son las coordenadas de los puntos $P_{0}$ y $P_{1}$, respectivamente, la
expresión~(\ref{eq:algeom003}) define el conjunto de puntos del segmento de recta de extremos
$P_{0}$ y $P_{1}$ y las coordenadas de $P_{t}$ son $(x_{t},y_{t},z_{t})$, donde:
\begin{equation}
\label{eq:algeom005}
x_{t}=tx_{1}+(1-t)x_{0}, \quad
y_{t}=ty_{1}+(1-t)y_{0}, \quad
z_{t}=tz_{1}+(1-t)z_{0}.
\end{equation}

Si definimos
\[
\overrightarrow{x_{t}} = \left(\begin{array}{c}
x_{t} \\
y_{t} \\
z_{t}
\end{array}\right)
\]
para cada $t\in [0,1]$, la expresión~(\ref{eq:algeom005}) se escribe de la siguiente manera:
\begin{equation}
\label{eq:algeom006}
\overrightarrow{x_{t}} = t\overrightarrow{x_{1}}+(1-t)\overrightarrow{x_{0}}.
\end{equation}
Visualicemos este resultado en $\mathbb{E}^{2}$.

\subsection{Ecuación de la recta que pasa por dos puntos dados en $\mathbb{E}^{2}$}

Sean $P_{0}(x_{0}, y_{0})$ y $P_{1}(x_{1},y_{1})$ dos puntos distintos del plano $\mathbb{E}^{2}$
con $x_{0}\neq x_{1}$. Si la ecuación de la recta que pasa por estos dos puntos es:
\begin{equation}
\label{eq:algeom007}
y = y_{0}+ \frac{y_{1}-y_{0}}{x_{1}-x_{0}} (x-x_{0})= l(x).
\end{equation}
Si $x$ está entre $x_{0}$ y $x_{1}$; es decir, si $x = x_t$ con
\[
x_{t}=tx_{1}+(1-t)x_{0}
\]
y $t\in (0,1)$ o, lo que es lo mismo, si
\begin{equation}
\label{eq:algeom008}
x_{t} = x_{0} + t(x_{1}-x_{0}),
\end{equation}
tendremos que, si
\begin{equation}
\label{eq:algeom009}
y_{t}=l(x_{t}),
\end{equation}
entonces:
\begin{equation}
\label{eq:algeom010}
y_{t} = y_{0}+ \frac{y_{1}-y_{0}}{x_{0}-x_{1}} (x_{0}+t(x_{1}-x_{0})-x_{0})=y_{0}+t(y_{1}-y_{0}).
\end{equation}

Cuando $x_{0}=x_{1}$, como $P_{0}\neq P_{1}$, tendremos que $y_{0}\neq y_{1}$, la recta que pasa
por $P_{0}$ y $P_{1}$ es vertical y se reduce al caso $\mathbb{E}^{1}$ con $x_{t}=x_{0}=x_{1}$, y
con $y_{t}=ty_{1}+(1-t)y_{0}$. Se tendrá entonces, que si $P_{t}$ es un punto del segmento
$[P_{0},P_{1}]$ y, si $(x_t,y_t)$ son sus coordenadas, se verificará la siguiente igualdad:

\begin{equation}
\label{eq:algeom011}
 \left(\begin{array}{c}
x_{t} \\
y_{t}
\end{array}\right)=
t\left(\begin{array}{c}
x_{1} \\
y_{1}
\end{array}\right)+
(1-t) \left(\begin{array}{c}
x_{0} \\
y_{0}
\end{array}\right)
\end{equation}
que es la expresión~(\ref{eq:algeom006}) para $\mathbb{E}^{2}$.

\subsection{Funciones convexas y cóncavas}

\begin{defical}[Funciones convexa y cóncava]
Sean $I\subset \mathbb{R}$ un intervalo y $f\colon I
\rightarrow \mathbb{R}$ una función definida en $I$. La función \emph{$f$ es convexa
(\emph{respectivamente} cóncava) en $I$} si para cualquier para $x_{0} \in I$, $x_{1}\in I$ tales
que $x_{0}<x_{1}$, el gráfico de $f$ correspondiente al intervalo $(x_{0},x_{1})$; es decir, el
conjunto de $\mathbb{R}^2$
\begin{equation}
\label{eq:algeom012}
\{(x,f(x)) : x\in (x_{0},x_{1})\}=\{(x_{t},f(x_{t})) :  x_{t}=tx_{1}+(1-t)x_{0},\ 0<t<1 \}
\end{equation}
queda por debajo (respectivamente por encima) del segmento de recta $[P_{0},P_{1}]$, con
$P_{0}(x_{0},y_{0})$ y  $P_{1}(x_{1},y_{1})$, donde $y_{0}=f(x_{0})$ y $y_{1}=f(x_{1})$.
\end{defical}

Teniendo en cuenta la igualdad~(\ref{eq:algeom010}), la ecuación de la recta que une los puntos
$P_{0}$ y $P_{1}$ es:
\begin{equation}
\label{eq:algeom013}
y = l(x)= y_{0}+ \frac{y_{1}-y_{0}}{x_{1}-x_{0}} (x-x_{0}),
\end{equation}
y si $x= x_{t} = tx_{1}+(1-t)x_{0}$, obtenemos que:
\begin{equation}
\label{eq:algeom014}
y = y_{t}=l(x_{t})= y_{0} + t(y_1 - y_0) = ty_{1}+(1-t)y_{0}.
\end{equation}
Entonces, $f$ es convexa (respectivamente cóncava) en el intervalo $I$ si para todo $x_{0}$,
$x_{1}$ en $I$ tales que $x_{0}<x_{1}$, se verifica que
\begin{equation}
\label{eq:algeom015}
f(x_{t})\leq l(x_{t}) \ (\text{respectivamente } f(x_{t})\geq l(x_{t}))
\end{equation}
para todo $t\in\ ]0,1[$. Es decir, $f$ es convexa (respectivamente cóncava) si
\begin{equation}
\label{eq:algeom016}
f(x_{t})\leq y_{t} \ (\text{respectivamente } f(x_{t})\geq y_{t})
\end{equation}
para todo $t\in\ ]0,1[$.

En resumen, hemos demostrado el siguiente teorema.

\begin{teocal}\label{teo:daCaracterizacionConvexidad}
Una función $\funcjc{f}{I}{\Rbb}$ es convexa (respectivamente cóncava) si para todo $t \in\ ]0,1[$
se verifica que
\[
f(x_t) \leq y_t \ (\text{respectivamente } f(x_t) \geq y_t),
\]
donde
\[
y_t = ty_1 + (1 - t)y_0,
\]
con $y_0 = f(x_0)$ y $y_1 = f(x_1)$ y $x_0 < x_1$ elementos de $I$.
\end{teocal}

\begin{exemplo}[]{}
Sea $f$ una función definida por
\[
f(x) = ax^2 + bx + c.
\]
Mediante el teorema~\ref{teo:daCaracterizacionConvexidad}, probemos que $f$ es una función convexa
cuando $a > 0$ y es cóncava si $a < 0$.

\begin{enumerate}[leftmargin=*]
\item Supongamos que $a > 0$. Sean $x_0\in\Rbb$ y $x_1\in\Rbb$ tales que $x_0 < x_1$. Definamos
    $y_0 = f(x_0)$ y $y_1 = f(x_1)$. Sea $t\in\ ]0,1[$; entonces:
    \[
      x_t = (1 - t)x_0 + tx_1 \yjc y_t = (1 - t)y_0 + ty_1.
    \]
    Vamos a probar que $f(x_t) < y_t$. Por ello, calculemos $y_t$ y $f(x_t)$.

    En primer lugar, tenemos que:
    \begin{align*}
      y_t &= (1 - t)y_0 + ty_1 \\
          &= (1 - t)f(x_0) + tf(x_1) \\
          &= (1 - t)(ax_0^2 + bx_0 + c) + t(ax_1^2 + bx_1 + c) \\
          &= a[(1 - t)x_0^2 + tx_1^2] + b[(1-t)x_0 + tx_1] + c.
    \end{align*}
    En segundo lugar, tenemos que:
    \begin{align*}
    f(x_t) &= a[(1-t)x_0 + tx_1]^2 + b[(1-t)x_0 + tx_1] + c \\
           &= a[(1-t)^2x_0^2 + t^2x_1^2 + 2t(1-t)x_0x_1] + b[(1-t)x_0 + tx_1] + c.
    \end{align*}
    Por lo tanto:
    \begin{align*}
    y_t - f(x_t) &= a\left\{[(1-t) - (1-t)^2]x_0^2 - 2(1-t)tx_0x_1 + (t - t^2)x_1^2\right\} \\
                 &= a(1-t)t(x_0^2 - 2x_0x_1 + x_1^2) = a(1-t)t(x_0 - x_1)^2.
    \end{align*}
    Como $a > 0$, $1 - t > 0$, pues $0 < t < 1$ y $(x_0 - x_1)^2 > 0$, ya que $x_0 \neq x_1$,
    entonces
    \[
      y_t - f(x_t) \geq 0,
    \]
    de donde $f(x_t) \leq y_t$. Por lo tanto, por el
    teorema~\ref{teo:daCaracterizacionConvexidad}, $f$ es convexa.

\item Si $a < 0$, un procedimiento similar al anterior muestra que $y_t - f(x_t) < 0$.
\end{enumerate}
\end{exemplo}

Cuando una función es derivable, su convexidad o concavidad está relacionada con la monotonía de la
derivada. En el siguiente teorema, se muestra esta relación.

\begin{teocal}[Criterio de la derivada para la convexidad]\label{teo:daConvexidadDerivabilidad}
Sea $f\colon [a,b] \rightarrow \mathbb{R}$ continua en $[a,b]$ y derivable en $(a,b)$. Si $f'$ es
creciente (respectivamente decreciente) en $(a,b)$, entonces $f$ es convexa (respectivamente
cóncava) en $[a,b]$.
\end{teocal}

\begin{exemplo}[]{}
Sea $f(x) = x^4 + x^2 + 1$. Mediante la monotonía de $f'$, determinemos si $f$ es o no convexa.

En primer lugar, tenemos que $f'(x) = 4x^3 + 2x$. Sean $x_1$ y $x_2$ en $\Rbb$ tales que $x_1 <
x_2$. Entonces:
\begin{align*}
f'(x_2) - f'(x_1) &= 4x_2^3 - 2x_2 - 4x_1^3 + 2x_1 \\
                  &= 4(x_2^3 - x_1^3) + 2(x_2 - x_1).
\end{align*}
Como $x_2 > x_1$, entonces $x_2^3 - x_1^3 > 0$ (la función cúbica es creciente) y $x_2 - x_1 > 0$.
Por lo tanto
\[
f'(x_2) - f'(x_1) = 4(x_2^3 - x_1^3) + 2(x_2 - x_1) > 0.
\]
Entonces $f'(x_1) < f'(x_2)$; es decir, $f'$ es una función creciente. Entonces, por el
teorema~\ref{teo:daConvexidadDerivabilidad}, $f$ es una función convexa.
\end{exemplo}

Del teorema~\ref{teo:daConvexidadDerivabilidad} y por el teorema~(\ref{teo:daDerivadaPositiva}), es
inmediato el siguiente teorema.

\begin{teocal}\label{teo:daConvexidadSegundaDerivada}
Si $f$ es continua en $[a,b]$ y existe $f''$ y es positiva (respectivamente negativa) en $(a,b)$,
entonces $f$ es convexa (respectivamente cóncava) en $[a,b]$.
\end{teocal}

\begin{exemplo}[]{}
Sea $f(x) = 3x^4 - 4x^3 + 29x^2 + 2x - 7$. Probemos que esta función es convexa. Para ello,
mostremos que la segunda derivada de $f$ es siempre positiva. Así, por el
teorema~\ref{teo:daConvexidadSegundaDerivada}, $f$ debe ser convexa.

Calculemos $f''(x)$. En primer lugar, tenemos que
\[
  f'(x) = 12x^3 - 12x^2 + 58x + 2.
\]
En segundo lugar:
\[
  f''(x) = 36x^2 - 24x + 58.
\]
Ahora bien, el discriminante del polinomio de segundo grado $36x^2 - 24x + 58$ es negativo, pues
\[
(24)^2 - 4(36)(58) = -7\,776.
\]
Por lo tanto, para todo $x\in\Rbb$ $f''(x)$ tiene el mismo signo. Así, el signo de $f''(0)$ es el
signo de todos los $f''(x)$. Pues, como $f''(0) = 58 > 0$, tenemos que $f''(x) > 0$ para todo
$x\in\Rbb$. Es decir, $f$ es convexa.
\end{exemplo}

\section{Puntos de inflexión}
En el dibujo:
\begin{center}
\begin{pspicture}[showgrid=false](0,-1)(4.5,4)
\psset{plotpoints=200}
\psaxes[ticks=none,labels=none]{->}(0,0)(-0.5,-0.5)(4.3,4)[$x$,-90][$y$,180]
\pscurve[xunit=1.2cm](0.1,-0.7)(0.8,0.8)(2,1.7)(2.5,3)(3,3.5)
\psline(-0.5,0.6)(4,2)
\psset{linestyle=dashed,xunit=1.2}
\psline(1.4,0)(1.4,1.3)(0,1.3)
\psline(2.2,0)(2.2,2.1)
\rput[br](-0.1,-0.3){\footnotesize{$O$}}
\rput[b](1.4,-0.2){\footnotesize{$x_0$}}
\rput[b](2.2,-0.2){\footnotesize{$x$}}
\rput[r](0,1.3){\footnotesize{$f(x_0)$}}
\rput[U](1.4,1.5){\footnotesize{$P_0$}}
\rput[U](3,3.7){\footnotesize{$f$}}
\rput[U](3.5,2){\footnotesize{$g$}}
\end{pspicture}
\end{center}
$f$ es una función definida en un intervalo abierto $I$ y $x_0\in I$ tal que existe $f'(x_0)$. El
punto $P_0(x_0,f(x_0))$ tiene la propiedad de que la gráfica de $f$ pasa de un lado a otro de la
recta tangente a la gráfica de $f$ en $P_0$, cuya ecuación es
\[
y = g(x) = f(x_0) + f'(x_0)(x - x_0).
\]
El punto $P_0$ es denominado \emph{punto de inflexión} de la gráfica de $f$.

He aquí la definición precisa.

\begin{defical}[Punto de inflexión]
Sean $f\colon [a,b]\rightarrow\mathbb{R}$ y $x_{0}\in (a,b)$. Si $f$ es convexa en $[a,x_{0}]$ y
cóncava en $[x_{0},b]$, o viceversa (cóncava en $[a,x_{0}]$ y convexa en $[x_{0},b]$), se dice que
el punto de coordenadas $(x_{0}, f(x_{0}))$ es un \emph{punto de inflexión} del gráfico de $f$.
\end{defical}

\subsection*{Observaciones}
\begin{enumerate}
\item Si ponemos $h(x)=f(x)-g(x)$, $P_0$ es punto de inflexión de la gráfica de $f$ si y solo
    si $h(x)$ cambia de signo en $x_0$, es decir, si y solo si el punto de coordenadas
    $(x_0,0)$ es punto de inflexión de la gráfica de $h$.

\item Si $f\colon [a,b]\rightarrow\mathbb{R}$ es derivable en $(a,b)$ y si $f''$ cambia de signo
    en $x_{0}\in (a,b)$, entonces el punto de coordenadas $(x_{0}, f(x_{0}))$ es un punto de
    inflexión del gráfico de $f$.

\item Sea $f\colon [a,b]\rightarrow\mathbb{R}$. Si para $x_{0}\in (a,b)$, el punto de
    coordenadas $(x_{0}, f(x_{0}))$ es un punto de inflexión del gráfico de $f$, y si $f$ es
    derivable en $x_{0}$, existirá $r>0$ tal que el gráfico de $f$; es decir, de los puntos de
    coordenadas $(x,f(x))$, estarán de un lado de la recta tangente en el punto $(x_{0},
    f(x_{0}))$ para $x\in (x_{0}-r,x_{0})$ y del otro lado para $x\in (x_0,x_0+r)$.
\end{enumerate}

La siguiente es una condición necesaria para la existencia de un punto de inflexión.

\begin{teocal}
Si existe $f''(x_0)$ y si $P_0(x_0,f(x_0))$ es un punto de inflexión de la gráfica de $f$, entonces
$f''(x_0)=0$.
\end{teocal}

Esta condición no es suficiente; es decir, el recíproco de este teorema no es cierto. Por ejemplo,
si $y=f(x)=x^4$, el punto $P(0,0)$ no es punto de inflexión de la gráfica de $f$ a pesar de que
$f''(0)=0$.

\begin{exemplo}[Solución]{%
Analizar la función $f$ y su gráfico si $f\colon\mathbb{R}\to \mathbb{R}$ está definida por
\[
f(x) = x^3-px
\]
con $p\in \mathbb{R}$, una constante dada. A la gráfica de $f$ se le llama parábola cúbica.
}%
En primer lugar, $f$ es una función impar, pues:
\begin{equation*}
    f(-x)=(- x)^3-p(-x)=-(x^3-px)=-f(x).
\end{equation*}
Por lo tanto, la gráfica de $f$ es simétrica respecto al punto de coordenadas $(0,0)$. Esto
significa que bastará con analizar $f$ solo en uno de los dos intervalos: $[0,+\infty[$ o
$]-\infty,0]$, porque en el otro intervalo el gráfico se obtiene del primero por medio de un
reflexión y una rotación, y los valores $f(x)$ de la identidad $f(x) = -f(-x)$.

Ahora, busquemos los cortes de la gráfica de $f$ con el eje $x$; es decir, busquemos los $x$ tales
que $f(x)=0$:
\begin{equation}
\label{eq:da001}
   x^3-px=0
\end{equation}
La solución de esta ecuación depende del signo de $p$. Por lo tanto, hay tres casos que analizar.

\begin{enumerate}
\item  Si $p<0$, $x=0$ es la única raíz real de~(\ref{eq:da001}); las otras dos son complejas.
\item  Si $p=0$, $x=0$ es una raíz triple de la ecuación.
\item  Si $p>0$, $x_0=0$, $x_1=-\sqrt{p}$ y $x_2=\sqrt{p}$ son las tres raíces reales de la
    ecuación~(\ref{eq:da001}).
\end{enumerate}

Ahora calculemos la derivada de $f$:
\begin{equation}
\label{eq:da002}
    f'(x)= 3x^2-p.
\end{equation}
Estudiemos el signo de $f'(x)$, para lo cual es conveniente buscar los valores de $x$ para los
cuales $f'(x)=0$, en los tres casos ya considerados.

\begin{enumerate}
\item Si $p<0$, $f'(x)>0$ para todo $x\in \mathbb{R}$. Entonces $f$ será estrictamente
    creciente en $\mathbb{R}$.

\item Si $p=0$, $f'(x)>0$  para todo $x\in \mathbb{R} - \{0\}$; además, $f'(0)=0$. Entonces $f$
    es estrictamente creciente en $(-\infty,0)$ y en $(0,+\infty)$.

    Como en este caso $f(x)<0$ si $x<0$ y $f(x)>0$ si $x>0$, y $f(0) = 0$, se concluye que
    también $f$ es estrictamente creciente en $\mathbb{R}$.

\item Si $p>0$, la ecuación~(\ref{eq:da002}) tiene dos raíces:

   \[
      \bar{x}_1=-\sqrt{\frac{p}{3}} \yjc \bar{x}_2=\sqrt{\frac{p}{3}}.
   \]
    Se tiene que $f'(x)>0$ en $(-\infty,\bar{x}_1)$ y en $(\bar{x}_2,+\infty)$, mientras que
    $f'(x)<0$ en $(\bar{x}_1,\bar{x}_2)$, por lo que $f$ es creciente en los dos primeros
    intervalos y decreciente en el tercer intervalo.

    Adicionalmente, esto implica que, en $(\bar{x}_1,f(\bar{x_1}))$, la función $f$ tiene un
    máximo local y, en $(\bar{x}_2,f(\bar{x_2}))$, un mínimo local.
\end{enumerate}

Ahora estudiemos los puntos de inflexión de la gráfica de $f$. Como $f'(0)=-p$, la recta $g$ de
ecuación
\[
y=g(x)=-px
\]
es tangente a la gráfica de $f$ en $(0,0)$. La recta $g$ divide al plano en 3 partes: $g_+$ el
semiplano superior, la propia recta $g$ y $g_-$ el semiplano inferior:
\begin{gather*}
g_+=\{(x,y)\in \mathbb{R}\ |\ y>g(x)=-px\}\\
g=\{(x,y)\in \mathbb{R}\ |\ y=g(x)=-px\}\\
g_-=\{(x,y)\in \mathbb{R}\ |\ y<g(x)=-px\}.
\end{gather*}

\begin{center}
\psset{unit=0.65}
\begin{pspicture}[showgrid=false](-3,-2.5)(3,3.5)
\psaxes[ticks=none,labels=none]{->}(0,0)(-3,-2)(3,2.5)[$x$,-90][$y$,180]
\pstGeonode[PointSymbol=none,PointName={g,none},PointNameSep=0.5em,PosAngle=45]
            (2.5,1.3){A}(-2.5,-1.3){B}
\pstLineAB{A}{B}
\rput[B](1,1.2){$g_+$}
\rput[B](2,0.4){$g_-$}
\rput[bl](0,-0.4){\footnotesize{$O$}}
\end{pspicture}
\end{center}

Para ver la ubicación de los puntos $(x,f(x))$ de la gráfica de $f$ respecto a la recta
$y=g(x)=-px$, basta analizar el signo de
\begin{equation*}
    h(x)= f(x)-g(x)= (x^3-px)-(-px)=x^3.
\end{equation*}
El signo de $h(x)$ coincide con el de $x$, por lo que la gráfica de $f$ estará en el semiplano
$g_+$ para $x>0$ y en el semiplano $g_-$ para $x<0$.

El punto  $(0,0)$ es pues un punto de inflexión de la gráfica de $f$.

Como $f$ es dos veces derivable para todo $x$ y
\begin{equation*}
    f''(x)=6x,
\end{equation*}
tenemos que
\begin{equation*}.
    f''(x)=0 \quad \Leftrightarrow \quad x=0
\end{equation*}
Por lo tanto, como $(0,0)$ es un punto de inflexión, éste será el único (de haber otros para $x\neq
0$, se tendría que $f''(x)=0$; es decir, $6x =0$, lo cual es imposible).

Los resultados que sirven para analizar la función se pueden resumir en una tabla donde se los
anota a medida que se los obtiene y, finalmente, sirve para tener una idea completa del gráfico de
la función que luego, con la ayuda de un número adecuado de puntos de la forma $(a, f(a))$, podrá
ser realizado. En el ejemplo, para el tercer caso, $p>0$, tenemos:
\begin{center}
\setlength\extrarowheight{5pt}
\begin{tabular}{ c | c | c | c | c | c | c}

      % after \\: \hline or \cline{col1-col2} \cline{col3-col4} ...
      $x$   &   $\left (-\infty,-\sqrt{p}\right )$   &   $-\sqrt{p}$   &
      $\left (-\sqrt{p},-\frac{\sqrt{p}}{3}\right )$   &   $-\frac{\sqrt{p}}{3}$   &
      $\left (-\frac{\sqrt{p}}{3},0\right )$   &   $0$ \\
      \hline
     $f(x)$ &  $-$  &  0  &  +  &   + &  +  &    0    \\
      \hline
    $ f'(x) $&  +  &  +  &  +  &  0  & $-$   &  $-$    \\
      \hline
    Resultados &    &    &    & Máx. local   &    &   Punto infl.     \\ \hline

       $x$ &$0$   &   $\left (0,\frac{\sqrt{p}}{3}\right )$   &   $\frac{\sqrt{p}}{3}$   &   %
      $\left(\frac{\sqrt{p}}{3},\sqrt{p}\right )$   &   $\sqrt{p}$   &   $\left (\sqrt{p},+\infty\right )$ \\
      \hline
     $f(x)$ & 0   & $-$   & $-$   & $-$   &   0   &  + \\
      \hline
    $ f'(x) $&  $-$  & $-$   &  0  & +   &   +   &  +  \\
      \hline
    Resultados & Punto infl.   &    &  Mín local  &    &     &  \\ \hline
\end{tabular}
\end{center}

\begin{center}
\psset{unit=0.55}
\begin{pspicture}[showgrid=false](-3,-3.8)(3,3.5)
\psaxes[labels=none,ticks=none]{->}(0,0)(-3,-3.3)(3,3.5)[$x$,-90][$y$,180]
\psplot[yunit=0.2]{-2.3}{2.3}{x 3 exp x add}
\rput[bl](0.1,-0.4){\footnotesize{$O$}}
\rput[b](0,-3.8){$p<0$}
\end{pspicture}
\hspace{0.5cm}
\begin{pspicture}[showgrid=false](-3,-3.8)(3,3.5)
\psaxes[labels=none,ticks=none]{->}(0,0)(-3,-3.3)(3,3.5)[$x$,-90][$y$,180]
\psplot[yunit=0.2]{-2.3}{2.3}{x 3 exp}
\rput[bl](0.1,-0.4){\footnotesize{$O$}}
\rput[b](0,-3.8){$p=0$}
\end{pspicture}
\hspace{0.5cm}
\def\fun{x 3 exp x 5 mul sub}
\begin{pspicture}[showgrid=false](-3,-3.8)(3,3.5)
\psaxes[labels=none,ticks=none]{->}(0,0)(-3,-3.3)(3,3.5)[$x$,-90][$y$,180]
\psplot[yunit=0.2]{-2.5}{2.5}{\fun}
\pstGeonode%
[PointSymbol=none,PointName={none,none,\overline{x_1},\overline{x_2}},PosAngle={0,0,90,-90},PointNameSep=0.5em,yunit=0.2]%
          (! /x 1.29099 def x \fun){A}(! /x 1.29099 neg def x \fun){B}
          (1.29099,0){C}(-1.29099,0){D}
\pstLineAB[linestyle=dashed]{A}{C}
\pstLineAB[linestyle=dashed]{B}{D}
\rput[bl](2.23606,-0.4){\footnotesize{$x_1$}}
\rput[br](-2.23606,0){\footnotesize{$x_2$}}
\rput[br](0.1,-0.4){\footnotesize{$O$}}
\rput[b](0,-3.8){$p>0$}
\end{pspicture}
\end{center}

¿Cómo se corta la gráfica de $f$ con rectas horizontales de ecuación $y=c$? Para responder a esta
pregunta en los tres casos considerados, podemos analizar las soluciones de la ecuación $f(x)=0$:
\begin{equation}
\label{eq:da003}
    x^3-px=c.
\end{equation}

\begin{enumerate}
\item Si $p<0$, la ecuación~(\ref{eq:da003}) tiene una raíz real simple para todo $c$.

\item Si $p=0$, la ecuación tiene una raíz real simple para todo $c\neq 0$. Para $c=0$, $x=0$
    es una raíz real triple.

\item Si $p>0$, hay, a su vez, cuatro casos:

    \begin{enumerate}
    \item Si $c<f(\bar{x_2})$ o $c>f(\bar{x_1})$, la ecuación~(\ref{eq:da003}) tiene una
        sola raíz real simple.

    \item Si $f(\bar{x_2})<c<f(\bar{x_1})$, la ecuación tiene tres  raíces reales
        distintas.

    \item Si $c=f(\bar{x_2})$, $\bar{x_2}$ es raíz doble de la ecuación~(\ref{eq:da003}), y
        se tiene otra raíz simple y negativa.

    \item Si $c=f(\bar{x_1})$, $\bar{x_1}$ es raíz doble de la ecuación, y se tiene otra
        raíz simple y negativa.
    \end{enumerate}

\end{enumerate}
\end{exemplo}

\begin{multicols}{2}[\section{Ejercicios}]
\begingroup
\small
\begin{enumerate}[leftmargin=*]
\item Utilice la segunda derivada de $f$ para determinar los intervalos en los cuales $f$ es
    cóncava y en los cuales es convexa, así como los puntos de inflexión.
    \begin{enumerate}[leftmargin=*]
    \item $\displaystyle f(x) = 14x - x^2$.
    \item $\displaystyle f(x) = 2x^3 - 15x^2 + 24x$.
    \item $\displaystyle f(x) = (x + 2)^3 + 1$.
    \item $\displaystyle f(x) = (x + 1)(x - 1)^3$.
    \item $\displaystyle f(x) = 6x^4 + 2x^3 - 12x^2 + 4$.
    \item $\displaystyle f(x) = \sqrt[3]{x} + x$.
    \item $\displaystyle f(x) = x + \frac{16}{x}$.
    \item $\displaystyle f(x) = \sqrt{x^2 + 26}$.
    \item $\displaystyle f(x) = \frac{1}{x^2 + 1}$.
    \item $\displaystyle f(x) = \frac{x - 3}{x + 2}$.
    \item $\displaystyle f(x) = x^{\frac{7}{3}} + x$.
    \item $\displaystyle f(x) = \sen x$.
    \item $\displaystyle f(x) = x - \cos x + 1$.
    \item $\displaystyle f(x) = \tan x$.
    \end{enumerate}

\item Para la función $f$ dada, utilice la segunda derivada para hallar los extremos locales y
    los puntos de inflexión. Dibuje, aproximadamente, el gráfico de $f$.
    \begin{enumerate}[leftmargin=*]
    \item $\displaystyle f(x) = (2 - 3x)^2$.
    \item $\displaystyle f(x) = \frac{1}{3}x^3 - 2x^2 + 3x$.
    \item $\displaystyle f(x) = 5x^4 + x^2$.
    \item $\displaystyle f(x) = \frac{x^4 + 4}{x^2}$.
    \item $\displaystyle f(x) = x^{\frac{1}{3}}(x + 2)$.
    \item $\displaystyle f(x) = x\sqrt{x + 1}$.
    \item $\displaystyle f(x) = \sqrt{x} - \frac{x}{9}$.
    \item $\displaystyle f(x) = 3\sen x + 4\cos x$.
    \item $\displaystyle f(x) = x^4 - 6x^3 + 12x^2$.
    \item $\displaystyle f(x) = \sen(3x)$.
    \end{enumerate}

\item Dibuje, aproximadamente, el gráfico de la función $f$ que tenga las propiedades dadas.
    \begin{enumerate}[leftmargin=*]
    \item $f'(-1) = f'(2) = 0$; $f''(x) < 0$ para todo $x < 1$ y $f'(x) > 0$ para todo $x
        > 1$; y $f(0) = 1$.

    \item $f'(-2) = f'(-1) = f'(1) = f'(3) = 0$; $f''(x) > 0$ para todo $x < -\frac{3}{2}$
        o $0 < x < 2$ y $f''(x) < 0$ para todo $-\frac{3}{2} < x < 0$ o $x > 2$; y $f(0) =
        0$.
    \end{enumerate}

\item Halle los puntos de inflexión de $f$ si
    \[
      f(x) =
      \begin{cases}
      x^4 & \text{si } x \geq 0, \\
      -x^4 & \text{si } x < 0.
      \end{cases}
    \]

\item Dibuje, aproximadamente, el gráfico de $f$.
    \begin{enumerate}[leftmargin=*]
    \item $\displaystyle f(x) =
        \begin{cases}
          |x - 1| & \text{si } x \leq 1, \\
          2x^2 - 5x + 3 & \text{si } x > 1.
        \end{cases}$
    \item $\displaystyle f(x) =
        \begin{cases}
          -x^2 + 1 & \text{si } x^2 - 3x + 2 \geq 0, \\
          -3(x - 1) & \text{si no}.
        \end{cases}$
    \end{enumerate}

%\item Pruebe con un cambio de variable
%\begin{equation*}
%    x=\alpha x_1, \quad y=\beta y_1
%\end{equation*}
% (cambio de escala), que los tres casos de parábola cúbica se pueden reducir a
%\begin{equation*}
%    y_1=x_1^3-x_1, \quad y_1=x_1^3, \quad y_1=x_1^3+x_1.
%\end{equation*}
%\item Analice la función $f$ dada por
%\begin{equation}
%\label{eq:da004}
%    f(x)= a_0+a_1x+a_2x^2+x^3,
%\end{equation}
%con $a_0,a_1,a_2$ en $\mathbb{R}$. Halle los intervalos donde $f$ es monótona. Luego halle
%$\alpha, \beta$ en $\mathbb{R}$ tales que con un traslado paralelo dado por el cambio de
%variables
%\begin{equation*}
%    x=x_1+\alpha, \quad y=y_1+\beta
%\end{equation*}
%la ecuación (\ref{eq:da004}) sea equivalente a
%\begin{equation*}
%    y_1=x_1^3-x_1.
%\end{equation*}
%
%Lo puede hacer mediante cálculos directos o teniendo en cuenta que $(\alpha, \beta)$ es el
%punto de inflexión de la gráfica de $f$. Exprese entonces $p$ en función de $a_0,a_1\text{ y
%}a_2$
%\begin{equation*}
%   p = p(a_0,a_1,a_2).
%\end{equation*}
%Del análisis hecho anteriormente para la parábola cúbica, deduzca en qué casos (para qué
%relaciones entre $a_0,a_1\text{ y }a_2$) dados por $p<0$,  $p=0$ y  $p>0$, se tendrán para $f$
%una raíz real simple, una raíz real triple o tres raíces reales distintas.
%\item Analice la función  $f$ si
%\begin{equation*}
%    f(x)=b_0+b_1x+b_2x^2+b_3x^3+x^4,
%\end{equation*}
%usando para $f'$ los resultados obtenidos en el ejercicio anterior. Saque conclusiones respecto
%al número posible de máximos y mínimos en función de los coeficientes $b_0$, $b_1$, $b_2$ y
%$b_3$, y halle los puntos de inflexión en función de $b_0$, $b_1$, $b_2$ y $b_3$.
\end{enumerate}
\endgroup
\end{multicols}

\section{Graficación de funciones}
Los resultados estudiados hasta ahora, a más de proveernos de la herramienta necesaria para
resolver los problemas de optimización (encontrar el máximo o el mínimo de una función), se
utilizan para trazar el gráfico de funciones de manera aproximada, como ya se ha propuesto en los
ejercicios de la sección anterior.

En esta sección, vamos a establecer un procedimiento bastante general para dibujar el gráfico de
una función. Empecemos con un ejemplo.

\begin{exemplo}[Solución]{%
Dibujar el gráfico de la función $f$ definida por $y=f(x)=
\frac{1}{10}x^{3}-\frac{3}{10}x^{2}-\frac{8}{10}x$.
}%
La primera y la segunda derivadas de $f$ son:
\begin{eqnarray*}
  f'(x) &=& \frac{3}{10}x^{2}-\frac{3}{5}x-\frac{9}{10}, \\
  f''(x) &=& \frac{3}{5}x-\frac{3}{5}.
\end{eqnarray*}

Busquemos los puntos en los cuales se anulan $f$, $f'$ y $f''$; es decir, sus raíces. Para
\begin{equation*}
f(x_{k})=0,
\end{equation*}
tenemos que las raíces son
\[
x_{1}=-\frac{3}{2}(\sqrt{5}-1)\approx -1.85,\  x_{2}=0 \text{ y }
x_{3}=\frac{3}{2}(\sqrt{5}+1)\approx 4.85.
\]
Para
\begin{equation*}
	f'(x_{k})=0,
\end{equation*}
las raíces son:
\[
x_{4}=-1 \yjc  x_{5}=3.
\]
Finalmente, para
\begin{equation*}
	f''(x_{k})=0,
\end{equation*}
obtenemos una sola raíz:
\[
x_{6}=1.
\]

En $x_{1},x_{2}$ y $x_{3}$ el gráfico de $f$ interseca con el eje horizontal. En $x_{4}$ y $x_{5}$
es posible que $f$ tenga extremos locales,  en $x_{5}$, un punto de inflexión. La siguiente tabla
muestra los signos de  $f$, $f'$ y $f''$ en los intervalos comprendidos entre las raíces arriba
encontradas:

\[
\setlength\extrarowheight{4pt}
\begin{array}{c|c|c|c|c|c|c|c|c|c|c|c|c|c}
x & \multicolumn{1}{r|}{\hspace*{2em} -2} & -1.85 & & -1 & & 0 & & 1 &
\multicolumn{1}{r|}{2} & 3 & \multicolumn{1}{r|}{4} & \multicolumn{1}{r|}{4.85} &
\multicolumn{1}{l}{5} \\ \hline
f(x) & - & 0 & + & & + & 0 & - & & - & & - & 0 & + \\ \hline
f'(x) & + & & & 0 & - & - & & & - & 0 & + & & + \\ \hline
f''(x) & - & & - & & - & & - & 0 & + & & + & & +
\end{array}
\]

De la información contenida en esta tabla, colegimos que:
\begin{enumerate}
\item La función $f$ alcanza un máximo local en $-1$, ya que antes de este punto, la derivada
    es positiva y luego, negativa.
\item La función $f$ alcanza un mínimo local en $3$, ya que antes de este punto, la derivada es
    negativa y luego, positiva.
\item En el punto $1$, hay un punto de inflexión, ya que antes de este punto, la segunda
    derivada es negativa y luego, positiva.
\end{enumerate}

Dado que
\[
\setlength\extrarowheight{4pt}
\begin{array}{c|cccccc}
x & -1.85 & -1 & 0 & 1 & 3 & 4.85 \\ \hline
f(x) & 0 & 0.4 & 0 & -1 & -2.4 & 0
\end{array}
\]
podemos concluir que la gráfica de $f$:
\begin{enumerate}
\item corta el eje horizontal en $-1.85$, $0$ y $4.85$;
\item alcanza un máximo local igual a $0.4$ en $-1$;
\item alcanza un mínimo local igual a $-2.4$ en $3$;
\item es creciente en $]-\infty,-1]$ y $[3,+\infty[$;
\item es decreciente en $[-1,3]$;
\item es cóncava en $]-\infty, 0]$; y
\item es convexa en $[0,+\infty[$.
\end{enumerate}

Con esta información, podemos realizar el dibujo del gráfico de $f$ y obtener algo similar al
siguiente dibujo:
\begin{center}
\psset{algebraic}%
\def\psvlabel#1{\footnotesize $#1$}
\def\pshlabel#1{\footnotesize $#1$}
\begin{pspicture}(-4.5,-3.25)(6,2.5)
  \psplot[plotpoints=500]{-3}{5}{0.1*x^3-(3/10)*x^2-(8/10)*x}
  \psaxes[Dx=1,Dy=2,arrows=->](0,0)(-4.5,-3.25)(6,2.5)
  \psset{dotscale=0.8}
  \psdot[dotstyle=Bar,dotangle=-90](0,0.4)\uput[0](0,0.4){\footnotesize$0.4$}%
  \psdot(-1,0.4)\uput[90](-1,0.4){\footnotesize$(-1,0.4)$}%
  \psdot(-1.85,0)\uput[-45](-1.85,0){\footnotesize$x_1$}%
  \psdot[dotstyle=Bar,dotangle=-90](0,-2.4)\uput[0](0,-2.4){\footnotesize$-2.4$}%
  \psdot(3,-2.4)\uput[-90](3,-2.4){\footnotesize$(3,-2.4)$}%
  \psdot(4.85,0)\uput[-90](4.85,0){\footnotesize$x_2$}%
\end{pspicture}
\end{center}
\end{exemplo}


\begin{exemplo}[Solución]{Sea $f$ una función real definida por:
\begin{equation*}
	f(x)=
\begin{cases}
-2x-3 & \text{si $x\leq -1$}\\
x^3 & \text{si $|x|<1$}\\
\frac{1}{x^2}& \text{si $x\geq 1$}.
\end{cases}
\end{equation*}
Analice y dibuje el gráfico de $f$.
}%
La función $f$ está definida en $\mathbb{R}$. Hay cambio de definición en los puntos $x=-1$ y
$x=1$.

La función $f$ es continua en $\mathbb{R} - \{-1,1\}$ pues las partes en las cuales está definida
son dos polinomios y una función racional, que son continuos en sus dominios. Averigüemos si es
continua en $x=-1$ y $x=1$.

En primer lugar, $f$ está definida en dichos puntos: $f(-1)=-1$ y $f(1)=1$. Veamos ahora si existen
los límites en esos puntos. Calculamos primero los correspondientes límites laterales:
\begin{align*}
\lim_{x\to -1^-}f(x)&=\lim_{x\to -1^-}(-2x-3)= -1 & \lim_{x\to -1^+}f(x)&=\lim_{x\to -1^-}(-x^3)= -1\\
\lim_{x\to 1^-}f(x)&=\lim_{x\to -1^-}x^3= 1 & \lim_{x\to 1^+}f(x)&=\lim_{x\to -1^-}\frac{1}{x^2}= 1.
\end{align*}
Como los correspondientes límites laterales son iguales, concluimos la verdad de las siguientes
igualdades:
\begin{gather*}
\lim_{x\to -1}f(x)=-1,\\
 \lim_{x\to1}f(x)=1.
\end{gather*}
Finalmente, podemos decir que la función es continua en $x=-1$ y $x=1$ porque
\begin{gather*}
\lim_{x\to -1}f(x)=-1= f(-1),\\
 \lim_{x\to1}f(x)=1=f(1).
\end{gather*}

En resumen, la función $f$ es continua en su dominio $\mathbb{R}$.

Como no existen puntos de discontinuidad no hay posibilidad de que existan asíntotas verticales.
Veamos si existen asíntotas horizontales. Para ello, calculemos los límites al infinito:
\begin{gather*}
\lim_{x\to -\infty} f(x) =\lim_{x\to -\infty} (-2x-3) = -\infty,\\
\lim_{x\to +\infty} f(x) =\lim_{x\to +\infty} \frac{1}{x^2} = 0.
\end{gather*}
Entonces $y=0$ es una asíntota horizontal de la gráfica de $f$ cuando $x$ tiende a $+\infty$.

Para continuar con el análisis de la función $f$, necesitaremos su primera y su segunda derivada:
\begin{equation*}
	f'(x)=
\begin{cases}
-2 & \text{si $x< -1$}\\
3x^2 & \text{si $|x|<1$}\\
-\frac{2}{x^3}& \text{si $x> 1$}.
\end{cases}
\end{equation*}

\begin{equation*}
	f''(x)=
\begin{cases}
0 & \text{si $x< -1$}\\
6x & \text{si $|x|<1$}\\
\frac{6}{x^4}& \text{si $x>1$}.
\end{cases}
\end{equation*}
Veamos si existe la primera derivada de $f$ en $x=-1$ y $x=1$. Como en esos puntos la función
cambia de definición, hallemos, primeramente, las derivadas laterales correspondientes:
\begin{align*}
f_-'(-1) & =\lim_{x\to -1^-}\frac{f(x)-f(-1)}{x-(-1)}= \lim_{x\to -1^-}\frac{(-2x -3) - (-1)} {x-(-1)}\\
&=\lim_{x\to -1^-}\frac{-2(x+1)}{x+1} =\lim_{x\to -1^-}(-2)=-2,
\end{align*}

\begin{align*}
f_+'(-1) & =\lim_{x\to -1^+}\frac{f(x)-f(-1)}{x-(-1)} =\lim_{x\to -1^+}\frac{x^3 -(- 1)}{x-(-1)} \\
& =\lim_{x\to -1^+}(x^2-x+1)) = 3,
\end{align*}

\begin{equation*}
f_-'(-1) = -2\neq 3=f_+'(-1).
\end{equation*}
Entonces no existe $f'(-1)$.

\begin{align*}
f_-'(1) & =\lim_{x\to 1^-}\frac{f(x)-f(1)}{x-1}= \lim_{x\to 1^-}\frac{x^3-1}{x-1}\\
& = \lim_{x\to 1^-}(x^2+x+1)= 3,
\end{align*}

\begin{align*}
f_+'(1) & =\lim_{x\to 1^+}\frac{f(x)-f(1)}{x-1} = \lim_{x\to 1^+}\frac{\frac{1}{x^2}-1}{x-1}\\
& = \lim_{x\to 1^+}\left(-\frac{x+1}{x^2}\right) = -2,
\end{align*}
\begin{equation*}
f_-'(1) = 3\neq -2=f_+'(1).
\end{equation*}
Entonces no existe $f'(1)$.

Como en los puntos $x=-1$ y  $x=1$ no existe la derivada, la gráfica de $f$ no tiene tangente en
esos puntos. Además, existe la posibilidad de que en ellos existan extremos locales. Claramente,
$f'(0)=0$; entonces hay la posibilidad de que en $x=0$ exista un extremo local.

Como el $D(f')= \mathbb{R} - \{-1,1\}$, no existe la segunda derivada  en $x=-1$ y  $x=1$ y, como
en dichos puntos no existe recta tangente, no puede haber en ellos puntos de inflexión de la
gráfica de $f$.

Ya solo falta averiguar en qué puntos la segunda derivada es igual a cero; ello sucede en $x=0$. En
$x=0$ podría haber un punto de inflexión.

En resumen, en $x=-1$, $x=0$ y $x=1$ buscaremos extremos locales. En $x=0$ además buscaremos un
punto de inflexión. Esta búsqueda la realizamos con ayuda de la siguiente tabla:

\begin{center}
\begin{tabular}{|c|c|c|c|c|c|c|c|}
  \hline
  % after \\: \hline or \cline{col1-col2} \cline{col3-col4} ...
  $x$    & $]-\infty,-1[$ & $-1$ & $]-1,0[$ & $0$ & $]0,+1[$ & $+1$ & $]+1,+ \infty[$ \\
  \hline
  $f(x)$ & $$                 & $-1$ & $$           & $0$ & $$            & $1$ & $$ \\
  \hline
  $f'(x)$ & $-$ & n. e. & $+$ & $0$ & $+$ & n. e. & $-$ \\
  \hline
  Monotonía & $\searrow$ & $$ & $\nearrow$ & $$ & $\nearrow$ & $$ & $\searrow$ \\
  \hline
  $f''(x)$ & $0$ & n. e. & $-$ & $0$ & $+$ & n. e & $+$ \\
  \hline
  Concavidad & $$ & $$ & $\frown$ & $$ & $\smile$ & $$ & $\smile$ \\
  \hline
  Resultados & $$ & m. l. & $$ & p. i. & $$ & M. l. & $$ \\
   \hline
\end{tabular}
\end{center}

\begin{center}
\begin{tabular}{|c|}
\hline
n. e. = no existe; m. l. = mínimo local; M. l. = máximo local; p. i. = punto de inflexión \\
\hline
\end{tabular}
\end{center}

A partir de la tabla podemos dibujar la forma de la gráfica de la función $f$ como se muestra en la
figura.

\begin{center}
   \psset{xAxisLabel={},yAxisLabel={}}
   \begin{psgraph}[ticks=none](0,0)(-2.1,-1.25)(3,1.5){0.75\textwidth}{5.5cm}
      \psplot[algebraic,plotpoints=1000]%
         {-2}{3}{IfTE(x<-1,-2*x-3,IfTE(x<1,x^3,1/x^2))}%
      \psaxes{->}%
         (0,0)(-2.1,-1.25)(3.25,1.5)%
      \uput[-90](3.25,0){$x$}%
      \uput[0](0,1.5){$y$}%
   \end{psgraph}
\end{center}

\end{exemplo}

De los dos ejemplos, podemos proponer el siguiente procedimiento, bastante general, para analizar
el gráfico de una función real.
\begin{enumerate}[leftmargin=*]
\item Si la función es par o impar, basta considerar el intervalo $[0,+\infty[$.
\item Se identifican los ``valores especiales'' para $f$, que son sus raíces. También sus
    puntos de discontinuidad. Digamos que estos son $x_1 < x_2 < \cdots < x_N$.

\item En cada uno de los intervalos $]-\infty, x_1[$, $]x_k, x_{k+1}[$ y $]x_N, +\infty[$ la
    función $f$ no cambia de signo. Para determinar el signo en cada intervalo, basta calcular
    $f(x)$ para un $x$ en el intervalo correspondiente.

\item El paso anterior se aplica también a $f'$ y a $f''$.

\item Se elabora una tabla con los puntos especiales de $f$, $f'$ y $f''$.
\end{enumerate}
Con la tabla obtenida, podemos determinar los extremos locales, puntos de inflexión, intervalos de
monotonía, concavidad y convexidad.

\begin{multicols}{2}[\section{Ejercicios}]
\begingroup
\small
\begin{enumerate}[leftmargin=*]
\item Dibuje, aproximadamente, el gráfico de la función $f$.
    \begin{enumerate}
    \item $\displaystyle f(x) = x^3 + 2x$.
    \item $\displaystyle f(x) = (x + 2)^3(x - 1)^2$.
    \item $\displaystyle f(x) = x^3 - 6x^2 + 9x + 5$.
    \item $\displaystyle f(x) = x + \frac{1}{x^2}$.
    \item $\displaystyle f(x) = \frac{x - 1}{(x + 2)(x - 3)}$.
    \item $\displaystyle f(x) = \frac{6x + 1}{3x - 1}$.
    \item $\displaystyle f(x) = -x + \cos x$.
    \item $\displaystyle f(x) = \sen^3 x + \cos^3 x$.
    \item $\displaystyle f(x) = x^2\ln x$.
    \item $\displaystyle f(x) = x^2e^x$.
    \item $\displaystyle f(x) = 
    \begin{cases} 
    |x^2-5x+4| & \text{si $x\geq 0$,}\\
    4-x^2 & \text{si $x< 0$.}
    \end{cases}$
    \end{enumerate}
\end{enumerate}
\endgroup
\end{multicols}

\section{Problemas de extremos}

Una de las principales aplicaciones de las derivadas consiste en proveer de un método para resolver
problemas de optimización; es decir, problemas en los que hay que escoger de entre varias opciones
para la solución la mejor o la más conveniente, o la menos inconveniente. Por ejemplo, en una
empresa se busca que las pérdidas sean mínimas y que las utilidades sean máximas; un transportista
busca la ruta más corta o la más rápida; un bodeguero trata de guardar más objetos en cada bodega,
etcétera. El problema de Romeo y Julieta con el que iniciamos este capítulo, es un ejemplo de un
problema de optimización.

El modelo matemático de una gran variedad de problemas de optimización se ajustan al siguiente
problema matemático, denominado, \textbf{problema de optimización}:

\begin{probcal}[Básico de optimización]
Dados $I$ un intervalo y $\funcjc{f}{I}{\Rbb}$ continua en $I$, se debe determinar si existen el
máximo y el mínimo de $f$ en $I$. De existir, se debe hallar $x_m\in I$ y $x_M\in I$ tales que
\[
\underline{y} = f(x_m) = \min_{x\in I}f(x) \yjc \overline{y} = f(x_M) = \max_{x\in I}f(x).
\]
\end{probcal}

A continuación vamos a enunciar los resultados desarrollados en las secciones anteriores y que nos
permitirán establecer un ``algoritmo'' para el problema de optimización.

\begin{teocal}
Sean $I$ un intervalo y $\overline{x} \in I^\circ$ tal que $f$ alcanza un extremo local en
$\overline{x}$. Entonces $\overline{x}$ es un punto crítico de $f$; es decir, $f'(\overline{x}) =
0$ o no existe la derivada de $f$ en $\overline{x}$.
\end{teocal}

Al combinar estos dos resultados, tenemos el siguiente algoritmo para el problema de optmización
cuando $I = [a,b]$ y $f$ tiene un número finito de puntos críticos en el interior de $I$.

\begin{algocal}
Sea $\funcjc{f}{[a,b]}{\Rbb}$ una función continua en $[a,b]$.
\begin{enumerate}
\item Hallar $K_0 = \{x \in \ ]a,b[ \ : f'(x) = 0\}$ y $K_1 = \{x \in \ ]a,b[ \ : \not\exists
    f'(x)\}$. Si $K_0$ y $K_1$ son finitos, vaya al siguiente paso.

\item Sea $K = \{a,b\} \cup K_0 \cup K_1$. Este es el conjunto de los ``candidatos'' a ser los
    puntos en los que se alcancen los extremos globales. Obligatoriamente $x_m$ y $x_M$, cuya
    existencia está garantizada por el primer teorema, son elementos de $K$.

    Escribamos $K = \{a_1 = a < a_2 < a_3 < \cdots < a_N = b\}$.

\item Calcular la tabla de valores $b_k = f(a_k)$ con $1 \leq k \leq N$, ordenando los valores
    de manera que $b_{k_1} \leq b_{k_2} \leq \cdots \leq b_{k_N}$.

\item Entonces $x_m = a_{k_1}$, $x_M = a_{k_N}$, $\underline{y} = b_{k_1}$ y $\overline{y} = b
    _{k_N}$.
\end{enumerate}
\end{algocal}

Los siguientes teoremas, en cambio, nos permiten establecer un algoritmo cuando el intervalo $I$ no
es cerrado y acotado.

\begin{teocal}
Sean $I$ un intervalo y $\funcjc{f}{I}{\mathbb{R}}$ una función continua. Si existe un único extremo local en $I$, entonces este extremo es también es global.
\end{teocal}

\begin{teocal}
Sean $I$ un intervalo y $\funcjc{f}{I}{\Rbb}$ una función continua en $I$. Entonces la imagen de
$f$, el conjunto $\Img(f)$ es un intervalo. En particular, si $I=[a,b]$, entonces $\Img(f) =
[f(x_m),f(x_M)]$.
\end{teocal}

\begin{teocal}
Si $\displaystyle I = \bigcup_{k=1}^n A_n$ y $\funcjc{f}{I}{\Rbb}$, entonces:
\[
\Img(f) = \bigcup_{k=1}^n f(A_k).
\]
\end{teocal}

Y, finalmente:

\begin{teocal}
Sean $I = ]a,b[$ y $\funcjc{f}{I}{\Rbb}$ una función continua en $I$. Entonces:
\begin{enumerate}
\item Si $f$ es creciente en $I$, entonces $\displaystyle \Img(f) =
    \left]\limjc{f(x)}{x}{a^+},\limjc{f(x)}{x}{b^-}\right[$.

\item Si $f$ es decreciente en $I$, entonces $\displaystyle \Img(f) =
    \left]\limjc{f(x)}{x}{b^-},\limjc{f(x)}{x}{a^+}\right[$.

\item Resultados análogos a los anteriores son verdaderos si $I$ es uno de los siguientes
    intervalos: $]-\infty,b[$, $]a,+\infty[$ y $]-\infty,\infty[$.
\end{enumerate}
\end{teocal}

Al combinar los resultados precedentes, se tiene el siguiente algoritmo para resolver el problema
de optimización cuando $I$ no es un intervalo cerrado y acotado, pero sí es la unión de $3$ subintervalos de $I$, digamos $I_1$, $I_2$ e $I_3$, de modo que en $I_1$ y en $I_3$ $f$ es monótona y que $I_2=[a,b]$, con $a$, $b\in I$. El
algoritmo permite, además, hallar la imagen de $f$.

\begin{algocal}
Sean $I$ un intervalo que no es cerrado ni acotado y $\funcjc{f}{I}{\Rbb}$ una función continua en $I$.

\begin{enumerate}
\item Determinar $a$, $b\in I$ y los subintervalos $I_1$ e $I_3$ de $I$ en los cuales $f$ es
    monótona y que $I=I_1\cup [a,b]\cup I_3$.

\item Calcular $J_k = f(I_k)$ para todo $k \in \{1,2,3\}$ (Para $J_2$ se usa el Algoritmo $1$, mientras que para $J_1$ y $J_3$ se aplica el Teorema precedente).

\item Calcular $\displaystyle J = \Img(f) = \bigcup_{k=1}^3 J_k$.

\item El conjunto $J$ es un intervalo. Si $J$ es uno de los intervalos:
    \begin{enumerate}
    \item $\Rbb$, $]-\infty,\beta[$ o $]-\infty,\beta]$, entonces $\displaystyle \inf_{x\in
        I}f(x) = -\infty$ y $f$ no alcanza el mínimo en $I$.

    \item $]\alpha,\beta[$ o $]\alpha,\beta]$, entonces $\displaystyle \inf_{x\in I}f(x) =
        \alpha$ y $f$ no alcanza el mínimo en $I$.

    \item $[\alpha,\beta[$, $[\alpha,\beta]$ o $[\alpha,+\infty [$, entonces $\displaystyle
        \min_{x\in I}f(x) = \alpha$.

    \item $\Rbb$, $[\alpha,\infty[$ o $]\alpha,\infty[$, entonces $\displaystyle \sup_{x\in
        I}f(x) = +\infty$ y $f$ no alcanza el máximo en $I$.

    \item $]-\infty,\beta[$ o $]\alpha,\beta[$ o $[\alpha,\beta[$, entonces $\displaystyle
        \sup_{x\in I}f(x) = \beta$ y $f$ no alcanza el máximo en $I$.

    \item $]\alpha,\beta]$, $[\alpha,\beta]$ o $]-\infty,\beta]$, entonces $\displaystyle
        \max_{x\in I}f(x) = \beta$.
    \end{enumerate}
\end{enumerate}
\end{algocal}

Este algoritmo sirve también si el dominio de la función se puede expresar como la unión finita de
intervalos, pues se lo puede aplicar a cada uno de esos intervalos y luego se calcula la imagen de
$f$ con lo que, si existen, se determinan los extremos de la función.

Ahora resolvamos el problema de Romeo y Julieta.

Empecemos calculando la derivada de $f$ para todo $x\in I$ ya que $f$ es derivable en su dominio:
\begin{align*}
f'(x) &= \frac{2x}{2\sqrt{x^2 + 4}} + \frac{-2(3 - x)}{2\sqrt{}(3 - x)^2 + 1} \\
      &= \frac{x}{\sqrt{x^2 + 4}} - \frac{3 - x}{\sqrt{}(3 - x)^2 + 1}.
\end{align*}
Por lo tanto:
\begin{align*}
f'(x) = 0 &\Leftrightarrow \frac{x}{\sqrt{x^2 + 4}} - \frac{3 - x}{\sqrt{}(3 - x)^2 + 1} \\
    &\Leftrightarrow \frac{x}{\sqrt{x^2 + 4}} = \frac{3 - x}{\sqrt{}(3 - x)^2 + 1}.
\end{align*}

Pero:
\begin{align*}
\frac{x}{\sqrt{x^2 + 4}} = \frac{3 - x}{\sqrt{}(3 - x)^2 + 1} &\Rightarrow
x^2\big[(3 - x)^2 + 1\big] = (3 - x)^2(x^2 + 4) \\
&\Rightarrow 3x^2 - 24x + 36 = 0 \\
&\Rightarrow x \in\{2,6\}.
\end{align*}
Es decir, si $f'(x) = 0$, necesariamente $x = 2$ ó $x = 6$.

Dado que $6\not\in I$ y
\begin{align*}
f'(2) &= \frac{2}{\sqrt{2^2 + 4}} - \frac{3 - 2}{\sqrt{(3 - 2)^2 + 1}} \\
    &= \frac{2}{2\sqrt{2}} - \frac{1}{2\sqrt{2}} = 0,
\end{align*}
tenemos que $2$ es un punto crítico de $f$ en $I$. Además, es el único punto crítico, por que si
hubiera otro, diferente de $2$, debería ser el número $6$, lo que no es posible.

Ahora calculemos la segunda derivada de $f$ en $2$ para saber si, en este número, la función $f$
alcanza un máximo o un mínimo local.

Para ello, luego de un cálculo sencillo, vemos que, para cualquier $x\in I$, tenemos que:
\[
f''(x) = \frac{4}{(x^2 + 4)^{\frac{3}{2}}} +
\frac{1}{\big[(3 - x)^2 + 1\big]^{\frac{3}{2}}}.
\]

Entonces, $f''(x) > 0$ para todo $x\in I$. En particular, $f''(2) > 0$.  Esto significa que $f(2)$
es un mínimo local de $f$ en $I$. Por lo tanto, el mínimo de $f$ es:
\[
f(2) = \min_{x\in I}f(x) = 3\sqrt{2} \approx 4.24.
\]

\subsubsection{Solución del problema original (interpretación de la solución del problema matemático)}
Romeo irá donde Julieta, llevándole rosas recogidas a $2$ kilómetros del extremo $A$ del lago.
Cuando el joven enamorado llegue, habrá remado aproximadamente un recorrido de $4$ kilómetros con
$240$ metros aproximadamente. A su edad, y con el deseo enorme de encontrarse con su amada, ese
recorrido no representará para él ningún problema.

\subsubsection{Epílogo}
Se cuenta que Romeo llegó con un bello ramo de rosas donde Julieta cuando ella estaba a punto de
retirarse de su escondite creyendo que su amado no vendría. Las flores la llenaron de alegría y
justificaron plenamente su impaciente espera.

Para terminar, examinemos algunos ejemplos adicionales de optimización.

\begin{exemplo}[Solución]{%
Sea $f\colon ]0, \infty[\ \to \mathbb{R} $, definida por
\[
f(x) = x^2-\frac{8}{x}.
\]
Halle el punto en el cual la gráfica de $f$ tiene la recta tangente de mínima pendiente.
}%
El dominio de $f$ es el conjunto $]0,+\infty[$.

La pendiente de la recta tangente $m_x$ en un punto de coordenada $x$ viene dada por la derivada de
$f(x)$. Entonces:
\begin{equation*}
    m_x = g(x) = f'(x) = 2x + \frac{8}{x^2}.
\end{equation*}
El dominio de $g$ es el conjunto $]0,+\infty[$.

El problema a resolver es hallar los extremos de la función $g$. Para ello necesitamos su derivada:
\begin{equation*}
    m_x' = g'(x) =2 - \frac{16}{x^3}= 2\frac{x^3-8}{x^3}.
\end{equation*}
La función $g$ no es derivable en $x=0$ pero $0\notin D(g)$.

Por otro lado, $g'(2)=0$. Posiblemente, existe extremo local en $x=2$. Para averiguarlo,
necesitamos la segunda derivada de $g$ (es decir, la tercera de $f$):
\begin{equation*}
    m_x'' = g''(x) = \frac{48}{x^4}.
\end{equation*}
Tenemos que $g''(2)= 3>0$.

Por el criterio de la segunda derivada, podemos concluir que, en $x=2$, la función $g$ tiene un
mínimo local. Debido a que es el único extremo en $D(g)$, es, a la vez, un mínimo absoluto y su
valor es:
\begin{equation*}
    m_x\Big |_{x=2}= g(2) = 2(2) + \frac{8}{2^2}= 6.
\end{equation*}

Así, pues, la pendiente será mínima e igual a 6 para la recta tangente a la gráfica de $f$ en el
punto de coordenadas $(2,0)$.

\begin{center}
   %\psset{xAxisLabel={},yAxisLabel={}}
   \begin{psgraph}[ticks=none](0,0)(-0.5,-30)(4,30){0.75\textwidth}{6cm}
      \psplot[algebraic,plotpoints=1000]%
         {0.27}{4}{x^2-8/x}%
      \psaxes[ticks=none,labels=x]{->}%
         (0,0)(-0.5,-30)(4.25,30.5)%
      \uput[-90](4.25,0){$x$}%
      \uput[0](0,30){$y$}%

      \psplotTangent[algebraic,linewidth=0.75\pslinewidth,linecolor=red,arrows=<->]%
         {2}{1.5}{x^2-8/x}%
      \psplotTangent[algebraic,linewidth=0.75\pslinewidth,linecolor=red,arrows=<->]%
         {1}{0.5}{x^2-8/x}%
      \psplotTangent[algebraic,linewidth=0.75\pslinewidth,linecolor=red,arrows=<->]%
         {0.5}{0.5}{x^2-8/x}%
   \end{psgraph}
\end{center}
\end{exemplo}

\begin{exemplo}[Solución]{%
Halle dos números positivos cuya suma sea 100 y cuyo producto sea máximo.
}%
Sea $x>0$ uno de dichos números. El otro será $100-x>0$ para que su suma sea 100. Tenemos entonces
que $0<x<100$.

Sea $P$ el producto de los dos números. Naturalmente dependerá de $x$:
\begin{equation*}
    P=f(x)=x(100-x)= 100 x-x^2.
\end{equation*}

Debemos, entonces, hallar $x_1\in ]0,100[$ tal que:
\begin{equation*}
  f(x_1)=\max_{x\in (0,100)}f(x).
\end{equation*}

Como $f$ es derivable en $]0,100[$, los puntos críticos de $f$ en ese intervalo serán solución de
\begin{equation*}
  f'(x)=100-2x=0
\end{equation*}
que equivale a $x=50$.

Como $f''(x)=-2<0$, en $x_1=50$, $f$ alcanza un máximo local que por ser el único extremo en el
intervalo de interés $]0,100[$ es, a la vez, un máximo absoluto.

El otro número es $100-x_1= 100-50=50$. Los dos números buscados, tales que su suma es igual a 100
y su producto  2500 es máximo, son iguales a 50.
\end{exemplo}

\begin{exemplo}[Solución]{%
Halle el punto de la parábola que es la gráfica de $y = 1-x^2$ más cercano al punto
$A(2.5,2.75)$. Puede usar el hecho de que
\begin{equation*}
    (4x^3+9x-5)\Big |_{x=0.5}=0.
\end{equation*}
}%
Si $P(x,y)$ es un punto arbitrario de la parábola de ecuación $y = 1-x^2$, tendremos que
$P(x,1-x^2)$.

La distancia entre los puntos $A$ y $P$ está dada por:
\begin{equation*}
    d = f(x) = \sqrt{(x-2.5)^2+[  (1-x^2)-2.75 ]^2}= \sqrt{(x-2.5)^2+ (x^2+1.75)^2}, \ x\in \mathbb{R}.
\end{equation*}

Como la función $h$ definida por $h(x) = x^2$ para todo $x > 0$ es continua y estrictamente
creciente, los extremos de $h\circ f$ coinciden con los extremos de $f$. Entonces, si la distancia
$d$ va a ser mínima, la distancia al cuadrado $d^2$ también será mínima. Sea
\begin{equation*}
    d^2 = g(x) = (x-2.5)^2+ (x^2+1.75)^2, \ x\in \mathbb{R}.
\end{equation*}

Tenemos que resolver el problema:
$\displaystyle
    \text{Hallar} \min_{x\in \mathbb{R}}g(x).
$
Necesitamos la derivada
\begin{equation*}
    g'(x) = 2(x-2.5)+ 4x(x^2+1.75)= 4x^3+9x-5
\end{equation*}
para hallar los puntos críticos de $g$. La ecuación $g'(x)=4x^3+9x-5=0$ puede escribirse como
\begin{equation*}
    4x^3+9x-5=(2x-1)(2x^2+x+5)= 0.
\end{equation*}
Puesto que el segundo factor es un trinomio de discriminante negativo, la única solución real de la
ecuación cúbica es $x = \frac{1}{2}$ la cual es, a la vez, número crítico de $g$.

Ahora, usemos el criterio de la segunda derivada para saber si existe o no extremo en $x =
\frac{1}{2}$. El signo del valor en $x = \frac{1}{2}$ de la segunda derivada  $g''(x)= 12x^2+9$,
que es
\begin{equation*}
  g''\left( \frac{1}{2}\right)= 12\left(\frac{1}{2}\right)^2+9 = 12>0,
\end{equation*}
nos dice que $g$ tiene un mínimo local en $x = \frac{1}{2}$. Como $x = \frac{1}{2}$ es el único
número crítico de $g$ en su dominio dicho extremo es un mínimo absoluto, de modo que
\begin{equation*}
    d^2\Big|_{x=\frac{1}{2}} = \min_{x\in \mathbb{R}}g(x),
\end{equation*}
y
\begin{equation*}
    d\Big|_{x=\frac{1}{2}} = \min_{x\in \mathbb{R}}f(x).
\end{equation*}

El punto buscado es
\begin{equation*}
   P(x,1-x^2)\Big|_{x=\frac{1}{2}}=  P\left(\frac{1}{2},\frac{3}{4}\right),
\end{equation*}
y la distancia de $A$ a $P$ es
\begin{equation*}
 d\Big|_{x=\frac{1}{2}} = f\left(\frac{1}{2}\right) =
 \sqrt{\left(\frac{1}{2}-2.5\right)^2 + \left( \frac{1}{2}+1.75\right)^2}=
 \sqrt{2^2+2^2} = 4\sqrt{2}.
\end{equation*}
\end{exemplo}


%\subsection{Ejemplo}
%\ejemplo{Las residencias de Romeo y Julieta, $R$ y $J$ en el gráfico, están separadas por una
%laguna rectangular de 2 km de ancho. Romeo desea ir en su barca a ver a su amada pero llevándole
%rosas del rosal que se encuentra al extremo del lago. ¿En qué lugar $P$ del rosal tomará las rosas
%para llegar lo más pronto posible donde su amada?
%\begin{center}
%%\includegraphics[scale=0.25]{Derivadas/RomeoJulieta.eps}
%\begin{pspicture}(6,4)
%   \psframe(0.5,0.5)(6,3)%
%
%   \pstGeonode[PosAngle={90,180,-90},PointSymbol=none,PointNameSep={1em,0.5em,1em}]%
%      (2.5,3){R}(0.5,2){P}(1.5,0.5){J}%
%
%   \psline{|<->|}(0.5,3.1)(2.5,3.1)%
%   \uput[90](1.5,3.25){$2$ km}%
%
%   \psline{|<->|}(0.5,0.4)(1.5,0.4)%
%   \uput[-90](1,0.4){$1$ km}%
%
%   \psline{|<->|}(6.1,0.6)(6.1,3)%
%   \uput[0](6.1,1.5){$3$ km}%
%
%   \psline(R)(P)(J)%
%
%   \rput(4,2.5){Lago}%
%   \rput[t](0,1){\pstVerb{
%/vshowdict 4 dict def
%/vshow
%{ vshowdict begin
%/thestring exch def
%/lineskip exch def
%thestring
%{
%/charcode exch def
%/thechar ( ) dup 0 charcode put def
%0 lineskip neg rmoveto
%gsave
%thechar stringwidth pop 2 div neg 0 rmoveto
%thechar show
%grestore
%} forall
%end
%} def
%   64 (lasoR) vshow }}
%
%\end{pspicture}
%\end{center}}{%
%Sea $x=|AP|$ la distancia de $A$ a $P$ en km, $0\leq x\leq3$. Entonces $3-x=|PB|$ la distancia de
%$P$ a $B$ en km. Como $|AR|=2$ y $|BJ|=1$ y utilizando el teorema de Pitágoras obtenemos
%\begin{gather*}
%|RP|= \sqrt{x^2+4},\\
%|PJ|=  \sqrt{(3-x^2)+1} .
%\end{gather*}
%
%Sean $t(RP), t(PJ), \text{ y } t$ el tiempo en horas que demora Romeo en ir en su barca de $R$ a
%$P$,  de $P$ a $J$, y de  $R$ a $J$ pasando por $P$, respectivamente. Suponemos que la barca de
%Romeo avanza con velocidad constante $v$ en km/h. Estos tiempos en términos de las distancias y la
%velocidad son:
%\begin{gather*}
%t(RP)  = \frac{|RP|}{v} =  \frac{\sqrt{x^2+4}}{v}, \\
%t(PJ)  = \frac{|PJ|}{v} =  \frac{\sqrt{(3-x)^2+1}}{v}, \\
%t  = t(RP)+t(PJ)= \frac{\sqrt{x^2+4}}{v} + \frac{\sqrt{(3-x)^2+1}}{v}.
%\end{gather*}
%
%El tiempo total que dura el recorrido de Romeo en su barca es una función de $x$:
%\begin{equation*}
%   t = f(x) = \frac{1}{v}\left( \sqrt{x^2+4} +   \sqrt{(3-x)^2+1} \right), \ x\in [0,3]
%\end{equation*}
%
%Tenemos entonces que resolver el problema
%\begin{equation*}
%    \text{Hallar }\ \min_{x\in [0,3]}f(x).
%\end{equation*}
%
%Procedemos a hallar los puntos críticos de $f$ para lo cual necesitamos la derivada
%\begin{equation*}
%    T'=f'(x)= \frac{1}{v}\left(\frac{x}{\sqrt{x^2+4}} -\frac{3-x}{\sqrt{(x-3)^2+1}}\right).
%\end{equation*}
%La función $f$ es derivable en el intervalo de interés $[0,3]$. Buscamos los puntos críticos
%igualando a cero la derivada:
%\begin{align*}
%f'(x)= 0 & \quad \Leftrightarrow \quad \frac{x}{\sqrt{x^2+4}} = \frac{x-3}{\sqrt{(3-x)^2+1}}\\
% & \quad \Leftrightarrow \quad   x^2[(3-x)^2+1] = (3-x)^2(x^2+4)        \\
%  & \quad \Leftrightarrow \quad  3x^2-24x+36=0,
%\end{align*}
% y como la última ecuación tiene como soluciones $x=x_1=2\in[0,3]$ y $x=x_2=6\notin[0,3]$, el único punto crítico en $[0,3]$ es 2.
%
% Como la función $f$ es continua en el intervalo cerrado $[0,3]$ podemos aplicar el teorema del valor extremo que nos asegura que en dicho intervalo existe 1 mínimo absoluto y 1 máximo absoluto. Dichos extremos pueden estar también en los extremos del intervalo $[0,3]$; para encontrarlos hacemos la siguiente tabla:
% \begin{center}
% \begin{tabular}{c|c}
%    % after \\: \hline or \cline{col1-col2} \cline{col3-col4} ...
%   $x$ & $t=f(x)$ \\
%     \hline
%   0 & $\dfrac{2+\sqrt{10}}{v}\approx \dfrac{5.1}{v}$ \\1
%   2 & $ \dfrac{3\sqrt{2}}{v}\approx \dfrac{4.2}{v}$\\
%  3& $\dfrac{\sqrt{13}+1}{v}\approx \dfrac{4.6}{v}$ \\
% \end{tabular}
%\end{center}
%
%Vemos que $t\Big|_{x=2}= \dfrac{3 \sqrt{2}}{v}$ es el valor mínimo.
%
%Romeo irá entonces donde Julieta llevándole rosas recogidas en el lugar $P$ situado a 2 km del
%extremo $A$ de la laguna y de esa forma estará junto a su amada en el menor tiempo posible,
%$\dfrac{3 \sqrt{2}}{v}\approx \dfrac{4.2}{v}$ h.
%}%Fin de \ejemplo

\begin{multicols}{2}[\section{Ejercicios}]
\begingroup
\small
\begin{enumerate}[leftmargin=*]
\item Resuelva el problema básico de optimización para las siguientes funciones $f$ definidas
    en el intervalo $I$. Adicionalmente, determine el conjunto $\Img(f)$.
    \begin{enumerate}[leftmargin=*]
    \item $\displaystyle f(x) = 2x^3 - 15x^2 + 24x$, \ $I = [0,5]$.
    \item $\displaystyle f(x) = 2x^3 - 15x^2 + 24x$, \ $I = [0,5[$.
    \item $\displaystyle f(x) = 2x^3 - 15x^2 + 24x$, \ $I = ]0,5[$.
    \item $\displaystyle f(x) = 2x^3 - 15x^2 + 24x$, \ $I = ]0,5]$.
    \item $\displaystyle f(x) = 2x^3 - 15x^2 + 24x$, \ $I = [0,3]$.
    \item $\displaystyle f(x) = \frac{x - 3}{x - 2}$, \ $I = [0,+\infty[$.
    \item $\displaystyle f(x) = \frac{1}{\sqrt{x^2 + 25}}$, \ $I = \Rbb$.
    \item $\displaystyle f(x) = 5x^4 + 2x^2 - 7$, \ $I = [-1,1]$.
    \item $\displaystyle f(x) = x^4 + 6x^2 - 7$, \ $I = ]-3,1]$.
    \item $\displaystyle f(x) = x^4 + 6x^2 - 7$, \ $I = [-4,1]$.
    \item $\displaystyle f(x) = 4\sen x + 3\cos x$, \ $I = \Rbb$.
    \item $\displaystyle f(x) = 4\sen x + 3\cos x$, \\ $\displaystyle I =
        \left]-\frac{\pi}{4},\frac{\pi}{2}\right]$.
    \item $\displaystyle f(x) = 4\sen x + 3\cos x$, \\ $\displaystyle I =
        \left[\arctan\frac{3}{4},\pi\right[$.
        \item $\displaystyle f(x) =
        \begin{cases}
          1 - x^2 & \text{si $x < 1$ o $x >2$}, \\
          3 - 3x & \text{si } 1 \leq x \leq 2,
        \end{cases}$ \ $I = [0,3]$.
        \item $\displaystyle f(x) =
        \begin{cases}
          4 - x^2 & \text{si } x \leq 2, \\
          \frac{1}{2}(x - 2) & \text{si } 2 < x,
        \end{cases}$ \ $I = [-1,3]$.
    \item $\displaystyle f(x) = \sqrt{x^2 + x + 1}$, \ $I = [-2,1[$.
    \item $\displaystyle f(x) = (x^2 - 2x + 10)^{\frac{5}{2}}$, \ $I = ]-7,5[$.
    \item $\displaystyle f(x) =
        \begin{cases}
          x - x^2 & \text{si } 0 \leq x \leq 1, \\
          \sen(\pi x) & \text{si no},
        \end{cases}$ \ $I = \Rbb$.
    \end{enumerate}

\item Halle dos números no negativos tales que:
    \begin{enumerate}[leftmargin=*]
    \item Su suma sea igual a $1$ y su producto sea máximo.
    \item Su producto sea igual a $1$ y su suma sea mínima.
    \item Su suma sea igual a $100$ y el producto del cuadrado del primero y el cubo del
        segundo sea máximo.
    \end{enumerate}

\item Halle el punto de $P$ de coordenadas $(a,b)$ de la parábola de ecuación $y = 1 - x^2$ más
    cercano al punto $Q$ de coordenadas $(\frac{5}{2},\frac{11}{4})$.

\item Dadas la recta $l$ de ecuación $y = 4 - x$ y la parábola $p$ de ecuación $y = 1 - x^2$,
    halle los puntos $P\in l$ y $Q \in p$ más cercanos entre sí.

\item Halle el punto del gráfico de $f$ donde la pendiente de la recta tangente sea mínima si
    $f(x) = x^3 - 6x^2$.

\item Se desea cerrar un terreno que es rectangular junto a un río con una pared paralela a la
    orilla del río que cuesta $25$ dólares el metro lineal y dos alambradas perpendiculares al
    río que cuestan $20$ dólares el metro lineal. Si el terreno debe tener diez mil metros
    cuadrados de área, ¿cuáles deben ser sus dimensiones para que los costos sean mínimos?

\item Se quiere construir una caja abierta de un volumen dado $V$ metros cúbicos, cuya base
    rectangular tenga el doble de largo que de ancho. ¿Cuáles deben ser sus dimensiones para
    que el costo de los materiales sea mínimo? ¿Y si la caja es cerrada?

\item Halle las dimensiones del cilindro circular recto de máximo volumen que puede inscribirse
    en un cono circular recto de diámetro $D$ metros y de altura $H$ metros.

\item Halle las dimensiones del cono circular recto de volumen máximo que puede inscribirse en
    una esfera de radio $R$ metros.

\item Se desea elaborar tarros de un litro para conservas de duraznos. ¿Qué dimensiones tendrá
    cada tarro para que el costo por el metal utilizado sea mínimo?

\item De un trozo cuadrado de cartulina se desea elaborar una pirámide de base cuadrada
    desechando la parte rayada y doblando en las líneas punteadas como se muestra en el dibujo:
    \begin{center}
    \begin{pspicture}(0,0)(4,4)
      \psset{PointSymbol=none,PointName=none}
      \pstGeonode[]%
        (2,0){A}(4,2){B}(2,4){C}(0,2){D}%
      \pstGeonode[]%
        (1.5,1.5){E}(2.5,1.5){F}(2.5,2.5){G}(1.5,2.5){H}%

      \psset{fillstyle=vlines,hatchcolor=gray}
      \pspolygon[]%
        (A)(B)(F)%
      \pspolygon[hatchangle=-45]%
        (B)(C)(G)%
      \pspolygon[]%
        (C)(D)(H)%
      \pspolygon[hatchangle=-45]%
        (D)(A)(E)%
      \pspolygon[fillstyle=none,linestyle=dashed]%
        (E)(F)(G)(H)

    \end{pspicture}
    \end{center}
    ¿Qué dimensiones tendrá la pirámide de volumen máximo si la cartulina tiene un metro de
    lado?

\item Dos calles de diez metros de ancho se intersecan. Se desea transportar horizontalmente
    una varilla larga y delgada, pero, en la intersección, se requiere virar a la derecha.
    ¿Cuál es el largo máximo de la varilla que se puede transportar si el grosor de la varilla
    es despreciable?

    \begin{center}
    \begin{pspicture}(0,0)(5,4.5)
      \psset{PointName=none,PointSymbol=none}%
      \pstGeonode[]%
        (1.5,0){A}(3.5,0){B}(5,1.5){C}(5,3.5){D}(3.5,4.5){E}(1.5,4.5){F}(0,3.5){G}(0,1.5){H}%
      \pstGeonode[]%
        (1.5,1.5){I}(3.5,1.5){J}(3.5,3.5){K}(1.5,3.5){L}(1.75,0.25){M}(3.75,3.25){N}%
      \psline[]%
        (A)(I)(H)%
      \psline[]%
        (B)(J)(C)%
      \psline[]%
        (D)(K)(E)%
      \psline[]%
        (F)(L)(G)%
      \psline[linewidth=4\pslinewidth]%
        (M)(N)%
    \end{pspicture}
    \end{center}

\item Halle las dimensiones del cono circular recto de volumen mínimo que circunscribe una
    esfera de radio $R$.

\item Un joven puede remar a razón de dos kilómetros por hora y correr a seis kilómetros por
    hora. Si está en el punto $A$ de la isla que está a dos kilómetros del punto C de la playa y desea llegar lo antes posible al punto $B$, situado en la playa de enfrente, que está a dos kilómetros del punto $C$ en la playa, ¿hasta
    qué punto $P$ de la playa debe remar para luego correr desde $P$ hasta $B$? ¿En qué tiempo
    llegará?
    \begin{center}
    \begin{pspicture}(-0.5,0)(6.5,4)
      \pstGeonode[PosAngle={90,135,45,45}]%
        (1,1){B}(4,1){P}(5,1){C}(5,3){A}%
      \pstGeonode[PointName=none,PointSymbol=none]%
        (0,1){X}(6,1){Y}(5,4){O}%

      \pstLineAB[]%
        {X}{Y}

      \psarc[linewidth=2\pslinewidth,linestyle=solid]%
          (O){1.25}{215}{325}

      \psset{linestyle=dashed}
        \pstLineAB[]%
          {A}{P}%
        \pstLineAB[]%
          {A}{C}%

      \psline[arrows=<->]%
        (! \psGetNodeCenter{B} B.x B.y 0.25 sub)(! \psGetNodeCenter{C} C.x C.y 0.25 sub)

      \pstMiddleAB[PointName=none,PointSymbol=none]%
        {B}{C}{N}%
      \uput[-90](! \psGetNodeCenter{N} N.x N.y 0.2 sub){$4\kilometros$}%

      \pstMiddleAB[PointName=none,PointSymbol=none]%
        {A}{C}{M}%
      \uput[0](! \psGetNodeCenter{M} M.x 0.1 add M.y){$2\kilometros$}%

      \rput[l](0,0.5){Playa}%
      \rput(! \psGetNodeCenter{A} A.x A.y 0.65 add){Isla}
    \end{pspicture}
    \end{center}
\end{enumerate}
\endgroup
\end{multicols}

\section{Regla de L'Hôpital}

\subsection{Formas indeterminadas}

Al tratar de utilizar las propiedades de los límites de la suma, producto, etcétera, se puede
llegar a situaciones en que los resultados no se pueden aplicar por no cumplirse las condiciones
que las garantizan. Y resulta que bajo condiciones similares, los resultados obtenidos no siempre
son iguales. Por ejemplo, el límite del cociente de dos funciones tales que cada una tiende a $0$,
puede ser, en algunos casos, igual a $0$; en otros, $+\infty$; en otros, no existir; etcétera. Se
dice, entonces, que estamos frente a una \emph{forma indeterminada}. Veamos algunos casos.

Sean $f$ y $g$ dos funciones definidas en un intervalo abierto $I$. Sea $x_{0}\in I$:

\begin{enumerate}[leftmargin=*]
\item Cuando se calculan límites de cocientes, como $\displaystyle{\lim_{x\rightarrow a}\frac{f(x)}{g(x)}}$,
    sabemos que si existen los límites
\begin{equation*}
	\lim_{x\rightarrow a}f(x)=L\quad \text{y}\quad \lim_{x\rightarrow a}g(x)=M
\end{equation*}
y si $M\neq 0$ se tiene que
\begin{equation*}
	\lim_{x\rightarrow a}\frac{f(x)}{g(x)}=\frac{\displaystyle{\lim_{x\rightarrow a}f(x)}}{\displaystyle{\lim_{x\rightarrow a}g(x)}}=\frac{L}{M}.
\end{equation*}
Puede suceder, sin embargo, que $L=M=0$, de modo que, al tratar de aplicar este resultado se
obtiene:
\begin{equation*}
	 \displaystyle{\frac{\displaystyle{\lim_{x\rightarrow a}f(x)}}{\displaystyle{\lim_{x\rightarrow a}g(x)}}=\frac{0}{0}.}
\end{equation*}
La expresión $\frac{0}{0}$  es un ejemplo de ``indeterminación''. En este caso no se puede
afirmar nada sobre la existencia o no del límite buscado. A veces existe, a veces no, como se
ve en los siguientes ejemplos.

Por ejemplo, si $f(x)=\sen x$ y $g(x)= x$, entonces
\begin{equation*}
	 \frac{\displaystyle{\lim_{x\rightarrow a}f(x)}}{\displaystyle{\lim_{x\rightarrow a}g(x)}}=1.
\end{equation*}
En cambio, si $f(x)=x\sen \frac{1}{x}$ y $g(x)= 1-\cos x$, el límite
\begin{equation*}
	 \frac{\displaystyle{\lim_{x\rightarrow a}f(x)}}{\displaystyle{\lim_{x\rightarrow a}g(x)}}
\end{equation*}
no existe.

\item Si $\displaystyle{\lim_{x\rightarrow x_{0}}f(x)}=+\infty$ (respectivamente $-\infty$) y si
    $\displaystyle{\lim_{x\rightarrow x_{0}}g(x)}=-\infty$ (respectivamente $+\infty$), las expresiones
    $(f+g)(x)$ y $\left(\frac{f}{g}\right)(x)$ pueden tener diferentes comportamientos posibles
    cuando $x\rightarrow x_{0}$. La primera es una indeterminación del tipo $+\infty -\infty$
    ($-\infty +\infty$, respectivamente), y la segunda, del tipo $\frac{\infty}{\infty}$.

\item Si $\displaystyle\lim_{x\rightarrow x_{0}}f(x)=0$ y si $\displaystyle\lim_{x\rightarrow x_{0}}g(x)=+\infty$ (o
    $-\infty$), la expresión $(fg)(x)$ puede comportarse de diferentes maneras cuando
    $x\rightarrow x_{0}$. Es una indeterminación del tipo $0\cdot \infty$ o $0\cdot (-\infty)$.

\end{enumerate}

Hemos visto algunos casos en los que se pueden hallar los límites buscados como aplicación del
teorema 1.1. Se dice entonces que ``hemos levantado la indeterminación'' Un método muy útil para
ello también es la regla de L'H\^{o}pital que veremos a continuación.


\subsection{Regla de L'H\^{o}pital}

\begin{teocal}[Regla de L'H\^{o}pital]
Sea $a\in \mathbb{R}$. Si $\displaystyle\lim_{x\rightarrow a}\frac{f(x)}{g(x)}$ es una indeterminación de tipo $\frac{0}{0}$ o $\frac{\infty}{\infty}$ y si
\begin{equation*}
	\lim_{x\rightarrow a}\frac{f'(x)}{g'(x)}=L\in \mathbb{R}\cup \{-\infty, \infty\},
\end{equation*}
entonces
\begin{equation*}
	\lim_{x\rightarrow a}\frac{f(x)}{g(x)}=\lim_{x\rightarrow a}\frac{f'(x)}{g'(x)}.
\end{equation*}
\end{teocal}

Se tiene un resultado similar si, en vez de que $x$ tienda al número $a$, lo haga a $a^+$, $a^-$,
$+\infty$ o $-\infty$.

\begin{exemplo}[Solución]{%
Calcular $\displaystyle\lim_{x\rightarrow 0}\frac{\sen x}{x}$.
}%
Como el límite es una indeterminación de tipo $\dfrac{\infty}{\infty}$, y como
\begin{equation*}
	\lim_{x\rightarrow 0}\frac{(\sen x)'}{(x)'}=\lim_{x\rightarrow 0}\frac{\cos x}{1} = 1
\end{equation*}
existe, entonces, por la regla de L'H\^{o}pital:
\begin{equation*}
	\lim_{x\rightarrow 0}\frac{\sen x}{x}=1.
\end{equation*}
\end{exemplo}

\begin{exemplo}[Solución]{
Calcular $\displaystyle\lim_{x\rightarrow \infty}\frac{\ln x}{x}$.
}%
Como el
\begin{equation*}
	\lim_{x\rightarrow \infty}\frac{(\ln x)'}{(x)'}=\lim_{x\rightarrow \infty}\frac{\frac{1}{x}}{1} = 0
\end{equation*}
existe, entonces, por la regla de L'H\^{o}pital:
\begin{equation*}
	\lim_{x\rightarrow \infty}\frac{\ln x}{x}=0.
\end{equation*}
\end{exemplo}

Con ciertas manipulaciones previas, el teorema es también útil para levantar otras
indeterminaciones como son
\begin{equation*}
	\infty -\infty,\ 0\cdot \infty,\  0^{0},\  \infty^{0}\ \text{ y }\ 1^{\infty}.
\end{equation*}
Veamos un ejemplo.

\begin{exemplo}[Solución]{%
Calcular $\displaystyle\lim_{x\rightarrow 0^{+}}\left( \frac{1}{\sqrt{x}} -\frac{1}{x} \right)$.
}%
Este límite conduce a la indeterminación $\infty -\infty$.

Pero como
\begin{equation*}
	\frac{1}{\sqrt{x}} -\frac{1}{x}=\frac{x-\sqrt{x}}{x\sqrt{x}},
\end{equation*}
el límite del segundo miembro de la expresión anterior conduce a una indeterminación del tipo
$\frac{0}{0}$, que puede ser tratada con la regla de L'H\^{o}pital.

El límite
\begin{equation*}
	\lim_{x\rightarrow 0^{+}}\frac{(x-\sqrt{x})'}{(x\sqrt{x})'}=\lim_{x\rightarrow 0^{+}}
  \frac{1-\frac{1}{2\sqrt{x}}}{\frac{3}{2}\sqrt{x}}=\lim_{x\rightarrow 0^{+}}
  \frac{2\sqrt{x}-1}{3x}=-\infty.
\end{equation*}
De donde, por la la regla de L'H\^{o}pital se tiene el límite pedido:
\begin{equation*}
\lim_{x\rightarrow 0^{+}}\left( \frac{1}{\sqrt{x}} -\frac{1}{x}\right)= -\infty.
\end{equation*}
\end{exemplo}

\begin{exemplo}[Solución]{%
Pruebe, utilizando la regla de L'Hôpital, que
\begin{equation*}
	\lim_{x \to 0}\frac{\sen x-x\cos x}{x^2+\tan^2x}=0.
\end{equation*}
}%
Sean $f(x)= \sen x-x\cos x$ y $g(x)=x^2+\tan^2x$. Se pide demostrar que
\begin{equation*}
	\lim_{x \to 0}\frac{f(x)}{g(x)}=0.
\end{equation*}


Aplicando propiedades de los límites se encuentra fácilmente que
\begin{gather*}
\lim_{x \to 0}f(x) = \lim_{x \to 0}(\sen x-x\cos x)=0,\\
\lim_{x \to 0}g(x) = \lim_{x \to 0}(x^2+\tan^2x)=0.
\end{gather*}
Como el límite de la función del denominador $\displaystyle\lim_{x \to 0}g(x)$ es igual a cero no podemos
aplicar el teorema del límite de un cociente de funciones. Pero, por otro lado, tenemos que la
función $\frac{f}{g}$ tiene la forma indeterminada $\frac{0}{0}$ en $x=0$, lo cual nos permitiría
aplicar la regla de L'Hôpital. Para ello debemos encontrar el límite
\begin{equation*}
	\lim_{x \to 0}\frac{f'(x)}{g'(x)}.
\end{equation*}
Si este límite existe la regla de L'Hôpital nos dice que el límite buscado es igual al anterior.
Veamos, pues, si existe dicho límite:
\begin{align*}
	\lim_{x \to 0}\frac{f'(x)}{g'(x)}&= \lim_{x \to 0}\frac{(\sen x-x\cos x)'}{(x^2+\tan^2x)'}\\
&=\lim_{x \to 0}\frac{\cos x-\cos x+x\sen x}{2x+2\tan x\sec^2x}\\
 &= \lim_{x \to 0}\frac{x\sen x}{2x+2\tan x\sec^2x}\\
 &{}=\lim_{x\to 0}\frac{\sen x}{2+2\frac{\sen x}{x}\sec^3 x}=\frac{0}{2+2(1)(1)}=0.
\end{align*}
Una inspección rápida del límite anterior nos dice que la función $\frac{f'}{g'}$ tiene también la
forma indeterminada $\frac{0}{0}$ en $x=0$. Intentamos nuevamente aplicar la regla de L'Hôpital:
\begin{align*}
	\lim_{x \to 0}\frac{(f'(x))'}{(g'(x))'}&= \lim_{x \to 0}\frac{f''(x)}{g''(x)}\\
&=  \lim_{x \to 0}\frac{(x\sen x)'}{(2x+2\tan x\sec^2x)'}\\
 &= \lim_{x \to 0}\frac{\sen x+x\cos x}{2+2\sec^4 x+2\tan^2 x \sec^2 x}\\
 &=0.
\end{align*}
Como el último límite existe podemos aplicar la regla de L'Hôpital:
\begin{equation*}
	\lim_{x \to 0}\frac{f'(x)}{g'(x)}=\lim_{x \to 0}\frac{(f'(x))'}{(g'(x))'}= 0.
\end{equation*}
Esto nos permite aplicar nuevamente la regla de L'Hôpital:
\begin{equation*}
	\lim_{x \to 0}\frac{f(x)}{g(x)}=\lim_{x \to 0}\frac{f'(x)}{g'(x)}= 0.
\end{equation*}
Con lo cual hemos probado que
\begin{equation*}
	\lim_{x \to 0}\frac{f(x)}{g(x)}=\lim_{x \to 0}\frac{\sen x-x\cos x}{x^2+\tan^2x}=0.
\end{equation*}
\end{exemplo}

\begin{multicols}{2}[\section{Ejercicios}]
\begingroup
\small
\begin{enumerate}[leftmargin=*]
\item Calcule los siguientes límites.
    \begin{enumerate}[leftmargin=*]
    \item $\displaystyle \limjc{\frac{1 - \cos x^2}{x^4}}{x}{0}$.
    \item $\displaystyle \limjc{\frac{1 - \cos x^2}{x^3\sin x}}{x}{0}$.
    \item $\displaystyle \limjc{\frac{a^x - x^a}{x - a}}{x}{a}$.
    \item $\displaystyle \limjc{x^\epsilon\ln x}{x}{0^+}$ donde $\epsilon > 0$.
    \item $\displaystyle \limjc{\frac{x}{\sqrt{1 + x^2}}}{x}{+\infty}$.
    \item $\displaystyle \limjc{\frac{\sen x}{\arctan x}}{x}{0}$.
    \item $\displaystyle \limjc{\frac{(2 - x)e^x - x - 2}{x^3}}{x}{0}$.
    \item $\displaystyle \limjc{\frac{\sen(1/x)}{\arctan(1/x)}}{x}{+\infty}$.
    \item $\displaystyle \limjc{\frac{1}{\sqrt{x}}\left(\frac{1}{\sen x} -
        \frac{1}{x}\right)}{x}{0^+}$.
    \item $\displaystyle \limjc{\frac{\tan x - 5}{\sec x + 4}}{x}{\frac{\pi}{2}}$.
    \end{enumerate}
\item Demuestre la regla de L'Hopital si $\displaystyle\lim_{x\to a}\frac{f(x)}{g(x)}$ es una indeterminación de tipo $\frac 00$, para el caso $L\in\Rbb$. \emph{Indicación: }Use el Teorema 4.10 (Teorema General del Valor Intermedio) para $\displaystyle\lim_{x\to a^-}\frac{f(x)}{g(x)}$ y $\displaystyle\lim_{x\to a^+}\frac{f(x)}{g(x)}$.
\end{enumerate}
\endgroup

\end{multicols}

%Como $x_{t}=(1-t)x_{0}+tx_{1}$, se puede obtener que
%\begin{equation}
%\label{eq:algeom027}
%t=\frac{x_{t}-x_{0}}{x_{1}-x_{0}}, \qquad 1-t=\frac{x_{1}-x_{t}}{x_{1}-x_{0}}
%\end{equation}
%Como f es convexa
%\begin{equation}
%\label{eq:algeom028}
%f(x_{t})\leq (1-t)f(x_{0}) + tf(x_{1})
%\end{equation}
%que equivale a
%\begin{equation}
%\label{eq:algeom029}
%f(x_{t})\leq \frac{x_{1}-x_{t}}{x_{1}-x_{0}}f(x_{0}) + \frac{x_{t}-x_{0}}{x_{1}-x_{0}}f(x_{1}).
%\end{equation}
%Se tienen las identidades
%\begin{equation}
%\label{eq:algeom030}
%\frac{x_{1}-x_{t}}{(x_{1}-x_{0})(x_{t}-x_{0})}=\frac{1}{x_{t}-x_{0}}-\frac{1}{x_{1}-x_{0}}
%\end{equation}
%y
%\begin{equation}
%\label{eq:algeom031}
%\frac{x_{t}-x_{0}}{(x_{1}-x_{0})(x_{1}-x_{t})}=\frac{1}{x_{1}-x_{t}}-\frac{1}{x_{1}-x_{0}}
%\end{equation}
%por lo que de \ref{eq:algeom029}, multiplicando previamente por $\frac{1}{x_{t}-x_{0}}$ y
%$\frac{1}{x_{1}-x_{t}}$, respectivamente, se obtienen las desigualdades
%\begin{equation}
%\label{eq:algeom032}
%\frac{f(x_{t})}{x_{t}-x_{0}}\leq \frac{f(x_{0})}{x_{t}-x_{0}} - \frac{f(x_{0})}{x_{1}-x_{0}} + \frac{f(x_{1})}{x_{1}-x_{0}},
%\end{equation}
%y
%\begin{equation}
%\label{eq:algeom033}
%\frac{f(x_{t})}{x_{1}-x_{t}}\leq \frac{f(x_{0})}{x_{1}-x_{0}} + \frac{f(x_{1})}{x_{1}-x_{t}} - \frac{f(x_{1})}{x_{1}-x_{0}},
%\end{equation}
%de donde
%\begin{equation}
%\label{eq:algeom034}
%\frac{f(x_{t})-f(x_{0})}{x_{t}-x_{0}}\leq \frac{f(x_{1})-f(x_{0})}{x_{1}-x_{0}}\leq \frac{f(x_{1})-f(x_{t})}{x_{1}-x_{t}}.\footnote{JC: Poner aquí un "halmos".}
%\end{equation}


%\nocite{*}
\bibliography{Bibliografia/Biblio}
%\include{Integrales/IntegralIndefinida}
%\include{Integrales/IntegralDefinida}
%\include{Integrales/IntegralAplicaciones}
%\include{Integrales/IntegralLogaritmoExponencial}
%\include{Sucesiones/SucesionesSeries}
%
%\appendix
%\include{Apendices/TablasIntegracion}
\endgroup



%\mainmatter
%\def\indexname{\textsf{Índice Alfabético}}
%\setcounter{page}{191}
%\printindex
\end{document}
